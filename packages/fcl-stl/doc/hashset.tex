\chapter{THashSet}

Implements container for storing unordered set of unique elements.
Takes 2 arguments for specialization, first one is type of elements, second one is a hash functor
(class which has class function hash, which takes element and number $n$ and returns hash of the
element in range $0, 1, \dots, n-1$). 
Usage example:

\lstinputlisting[language=Pascal]{hashsetexample.pp}

Memory complexity:
Arounds two times of size of stored elements 
Members list:

\begin{longtable}{|m{10cm}|m{5cm}|}
\hline
Method & Complexity guarantees \\ \hline
\multicolumn{2}{|m{15cm}|}{Description} \\ \hline\hline

\verb!Create! & O(1) \\ \hline
\multicolumn{2}{|m{15cm}|}{Constructor. Creates empty set.} \\ \hline\hline

\verb!function Size(): SizeUInt! & O(1) \\ \hline
\multicolumn{2}{|m{15cm}|}{Returns number of elements in set.} \\\hline\hline

\verb!procedure Insert(value: T)! &
O(1) on average \\ \hline
\multicolumn{2}{|m{15cm}|}{Inserts element into set, if given element is already there nothing
happens.} \\\hline\hline

\verb!procedure Delete(value: T)! &
O(1) on average \\ \hline
\multicolumn{2}{|m{15cm}|}{Deletes value from set. If element is not in set, nothing happens.} \\\hline\hline

\verb!function Contains(value: T):boolean! & O(1) on average \\\hline
\multicolumn{2}{|m{15cm}|}{Checks whether element is in set.} \\\hline\hline

\verb!function Iterator:TIterator! & O(1) \\\hline
\multicolumn{2}{|m{15cm}|}{Returns iterator allowing traversal through set. If set is empty returns nil.} \\\hline\hline

\verb!function IsEmpty(): boolean! & O(1) \\ \hline
\multicolumn{2}{|m{15cm}|}{Returns true when set is empty.} \\\hline

\end{longtable}

Some methods return type TIterator, which has following methods:
\begin{longtable}{|m{10cm}|m{5cm}|}                                                             
\hline
Method & Complexity guarantees \\ \hline                                                  
\multicolumn{2}{|m{15cm}|}{Description} \\ \hline\hline                                               
\verb!function Next:boolean! & O(N) worst case, but traversal of whole set takes O(N) time \\\hline
\multicolumn{2}{|m{15cm}|}{Moves iterator to next larger element in set. Returns true on
success. If the iterator is already pointing on last element returns false.} \\\hline\hline

\verb!property Data:T! & $O(1)$ \\\hline
\multicolumn{2}{|m{15cm}|}{Property, which allows reading of the element.} \\\hline

\end{longtable}
