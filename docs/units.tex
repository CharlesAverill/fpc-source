%
%   $Id$
%   This file is part of the FPC documentation.
%   Copyright (C) 1997, by Michael Van Canneyt
%
%   The FPC documentation is free text; you can redistribute it and/or
%   modify it under the terms of the GNU Library General Public License as
%   published by the Free Software Foundation; either version 2 of the
%   License, or (at your option) any later version.
%
%   The FPC Documentation is distributed in the hope that it will be useful,
%   but WITHOUT ANY WARRANTY; without even the implied warranty of
%   MERCHANTABILITY or FITNESS FOR A PARTICULAR PURPOSE.  See the GNU
%   Library General Public License for more details.
%
%   You should have received a copy of the GNU Library General Public
%   License along with the FPC documentation; see the file COPYING.LIB.  If not,
%   write to the Free Software Foundation, Inc., 59 Temple Place - Suite 330,
%   Boston, MA 02111-1307, USA.
%
%%%%%%%%%%%%%%%%%%%%%%%%%%%%%%%%%%%%%%%%%%%%%%%%%%%%%%%%%%%%%%%%%%%%%%%
% Preamble.
\input{preamble.inc}
\latex{%
  \ifpdf
    \pdfinfo{/Author(Michael Van Canneyt)
             /Title(Standard units Reference Guide)
             /Subject(Free Pascal Standard units reference guide)
             /Keywords(Free Pascal, Units, RTL)
             }
  \fi
}
%
% Settings
%
\makeindex
%
% Syntax style
%
% This is just a main file. All units are described in separate files.
%
\usepackage{syntax}
%
% Here we determine the style of the syntax diagrams.
%
% Define a 'boxing' environment
\newenvironment{diagram}[2]%
{\begin{quote}\rule{0.5pt}{1ex}%
\rule[1ex]{\linewidth}{0.5pt}%
\rule{0.5pt}{1ex}\\[-0.5ex]%
\textbf{#1}\\[-0.5ex]}%
{\rule{0.5pt}{1ex}%
\rule{\linewidth}{0.5pt}%
\rule{0.5pt}{1ex}\end{quote}}
%\newenvironment{diagram}[2]{}{}
% Define mysyntdiag for my style of diagrams
\makeatletter
% Under Tex4HT, the diagrams are rendered as pictures.
\@ifpackageloaded{tex4ht}{%
\newenvironment{mysyntdiag}%
{\Picture*{}\begin{syntdiag}\setlength{\sdmidskip}{.5em}\sffamily\sloppy}%
{\end{syntdiag}\EndPicture}%
}{%
\newenvironment{mysyntdiag}%
{\begin{syntdiag}\setlength{\sdmidskip}{.5em}\sffamily\sloppy}%
{\end{syntdiag}}%
}% 
\makeatother
% Finally, define a combination of the above two.
\newenvironment{psyntax}[2]{\begin{diagram}{#1}{#2}\begin{mysyntdiag}}%
{\end{mysyntdiag}\end{diagram}}
% Redefine the styles used in the diagram.
\latex{\renewcommand{\litleft}{\bfseries\ }
\renewcommand{\ulitleft}{\bfseries\ }
\renewcommand{\syntleft}{\ }
\renewcommand{\litright}{\ \rule[.5ex]{.5em}{2\sdrulewidth}}
\renewcommand{\ulitright}{\ \rule[.5ex]{.5em}{2\sdrulewidth}}
\renewcommand{\syntright}{\ \rule[.5ex]{.5em}{2\sdrulewidth}}
}
% Finally, a referencing command.
\newcommand{\seesy}[1]{see diagram}
%
%
% Start of document.
%
\begin{document}
\title{Free Pascal supplied units : \\ Reference guide.}
\docdescription{Reference guide for standard Free Pascal units.}
\docversion{1.8}
\input{date.inc}
\author{Micha\"el Van Canneyt\\ Florian Kl\"ampfl}
\maketitle
\tableofcontents
\newpage

\section*{About this guide}
This document describes all constants, types, variables, functions and
procedures as they are declared in the units that come standard with \fpc.

Throughout this document, we will refer to functions, types and variables
with \var{typewriter} font. Functions and procedures gave their own
subsections, and for each function or procedure we have the following
topics:
\begin{description}
\item [Declaration] The exact declaration of the function.
\item [Description] What does the procedure exactly do ?
\item [Errors] What errors can occur.
\item [See Also] Cross references to other related functions/commands.
\end{description}
The cross-references come in two flavors:
\begin{itemize}
\item References to other functions in this manual. In the printed copy, a
number will appear after this reference. It refers to the page where this
function is explained. In the on-line help pages, this is a hyperlink, on
which you can click to jump to the declaration.
\item References to Unix manual pages. (For Linux related things only) they
are printed in \var{typewriter} font, and the number after it is the Unix
manual section.
\end{itemize}
The chapters are ordered alphabetically. The functions and procedures in
most cases also, but don't count on it. Use the table of contents for quick
lookup.

%
% Each unit is in its own file. Each file is a chapter.
%

% The crt unit.
%
%   $Id$
%   This file is part of the FPC documentation.
%   Copyright (C) 1997, by Michael Van Canneyt
%
%   The FPC documentation is free text; you can redistribute it and/or
%   modify it under the terms of the GNU Library General Public License as
%   published by the Free Software Foundation; either version 2 of the
%   License, or (at your option) any later version.
%
%   The FPC Documentation is distributed in the hope that it will be useful,
%   but WITHOUT ANY WARRANTY; without even the implied warranty of
%   MERCHANTABILITY or FITNESS FOR A PARTICULAR PURPOSE.  See the GNU
%   Library General Public License for more details.
%
%   You should have received a copy of the GNU Library General Public
%   License along with the FPC documentation; see the file COPYING.LIB.  If not,
%   write to the Free Software Foundation, Inc., 59 Temple Place - Suite 330,
%   Boston, MA 02111-1307, USA. 
%
\chapter{The CRT unit.}
\label{ch:crtunit}
This chapter describes the \var{CRT} unit for Free Pascal, both under \dos
and \linux. The unit was first written for \dos by Florian kl\"ampfl. 

The unit was ported to \linux by Mark May\footnote{Current
e-mail address \textsf{mmay@dnaco.net}}, and enhanced by Micha\"el Van Canneyt
It works on the \linux console, and in xterm and rxvt windows under
X-Windows. The functionality for both is the same, except that under \linux
the use of an early implementation (versions 0.9.1 an earlier of the
compiler) the crt unit automatically cleared the screen at program startup.

This chapter is divided in two sections. 
\begin{itemize}
\item The first section lists the pre-defined constants, types and variables. 
\item The second section describes the functions which appear in the
interface part of the CRT unit.
\end{itemize}

\section{Types, Variables, Constants}
Color definitions :
\begin{verbatim}
  Black = 0;
  Blue = 1;
  Green = 2;
  Cyan = 3;
  Red = 4;
  Magenta = 5;
  Brown = 6;
  LightGray = 7;
  DarkGray = 8;
  LightBlue = 9;
  LightGreen = 10;
  LightCyan = 11;
  LightRed = 12;
  LightMagenta = 13;
  Yellow = 14;
  White = 15;
  Blink = 128;
\end{verbatim}
Miscellaneous constants
\begin{verbatim}
  TextAttr: Byte = $07;
  TextChar: Char = ' ';
  CheckBreak: Boolean = True;
  CheckEOF: Boolean = False;
  CheckSnow: Boolean = False;
  DirectVideo: Boolean = False;
  LastMode: Word = 3;
  WindMin: Word = $0;
  WindMax: Word = $184f;
  ScreenWidth = 80;
  ScreenHeight = 25;
\end{verbatim}
Some variables for compatibility with Turbo Pascal. However, they're not
used by \fpc.
\begin{verbatim}
var
  checkbreak : boolean;
  checkeof : boolean;
  checksnow : boolean;
\end{verbatim}
The following constants define screen modes on a \dos system:
\begin{verbatim}
Const
  bw40 = 0;
  co40 = 1;
  bw80 = 2;
  co80 = 3;
  mono = 7;
\end{verbatim}
The \var{TextAttr} variable controls the attributes with which characters
are written to screen.
\begin{verbatim}
var TextAttr : byte;
\end{verbatim}
The \var{DirectVideo} variable controls the writing to the screen. If it is
\var{True}, the the cursor is set via direct port access. If \var{False},
then the BIOS is used. This is defined under \dos only.
\begin{verbatim}
var DirectVideo : Boolean;
\end{verbatim}
The \var{Lastmode} variable tells you which mode was last selected for the
screen. It is defined on \dos only.
\begin{verbatim}
var lastmode : Word;
\end{verbatim}

\section{Procedures and Functions}

\procedure{AssignCrt}{(Var F: Text)}
{
Assigns a file F to the console. Everything written to the file F goes to
the console instead. If the console contains a window, everything is written
to the window instead.
}
{None.}{\seep{Window}}

\latex{\inputlisting{crtex/ex1.pp}}
 \html{\input{crtex/ex1.tex}}

\Procedure{BigCursor}{Makes the cursor a big rectangle. 

Not implemented on \linux.}
{None.}{\seep{CursorOn}, \seep{CursorOff}}

\Procedure {ClrEol}
{ ClrEol clears the current line, starting from the cursor position, to the
end of the window. The cursor doesn't move}
{None.}{\seep{DelLine}, \seep{InsLine}, \seep{ClrScr}}

\latex{\inputlisting{crtex/ex9.pp}}
 \html{\input{crtex/ex9.tex}}

\procedure {ClrScr}{}
{ ClrScr clears the current window (using the current colors), 
and sets the cursor in the top left
corner of the current window.}
{None.}{ \seep{Window} }

\latex{\inputlisting{crtex/ex8.pp}}
 \html{\input{crtex/ex8.tex}}

\Procedure{CursorOff}{Switches the cursor off (i.e. the cursor is no
longer visible). 

Not implemented on \linux.}
{None.}{\seep{CursorOn}, \seep{BigCursor}}

\Procedure{CursorOn}{Switches the cursor on. 

Not implemented on \linux.}
{None.}{\seep{BigCursor}, \seep{CursorOff}}

\procedure{Delay}{(DTime: Word)}
{Delay waits a specified number of milliseconds. The number of specified
seconds is an approximation, and may be off a lot, if system load is high.}
{None}{\seep{Sound}, \seep{NoSound}}

\latex{\inputlisting{crtex/ex15.pp}}
 \html{\input{crtex/ex15.tex}}

\Procedure {DelLine}
{ DelLine removes the current line. Lines following the current line are 
scrolled 1 line up, and an empty line is inserted at the bottom of the
current window. The cursor doesn't move.}
{None.}{\seep{ClrEol}, \seep{InsLine}, \seep{ClrScr}}

\latex{\inputlisting{crtex/ex11.pp}}
 \html{\input{crtex/ex11.tex}}

\procedure {GotoXY}{(X: Byte; Y: Byte)}
{ Positions the cursor at \var{(X,Y)}, \var{X} in horizontal, \var{Y} in
vertical direction relative to the origin of the current window. The origin
is located at \var{(1,1)}, the upper-left corner of the window.
}
{None.}{ \seef{WhereX}, \seef{WhereY}, \seep{Window} }

\latex{\inputlisting{crtex/ex6.pp}}
 \html{\input{crtex/ex6.tex}}

\procedure {HighVideo}{}
{ HighVideo switches the output to highlighted text. (It sets the high
intensity bit of the video attribute)
}
{None.}{ \seep{TextColor}, \seep{TextBackground}, \seep{LowVideo},
\seep{NormVideo}}

\latex{\inputlisting{crtex/ex14.pp}}
 \html{\input{crtex/ex14.tex}}

\procedure {InsLine}{}
{ InsLine inserts an empty line at the current cursor position. 
Lines following the current line are scrolled 1 line down, 
causing the last line to disappear from the window. 
The cursor doesn't move.}
{None.}{\seep{ClrEol}, \seep{DelLine}, \seep{ClrScr}}

\latex{\inputlisting{crtex/ex10.pp}}
 \html{\input{crtex/ex10.tex}}

\Function {KeyPressed}{Boolean}
{ The Keypressed function scans the keyboard buffer and sees if a key has
been pressed. If this is the case, \var{True} is returned. If not,
\var{False} is returned. The \var{Shift, Alt, Ctrl} keys are not reported.
The key is not removed from the buffer, and can hence still be read after
the KeyPressed function has been called.
}
{None.}{\seef{ReadKey}}

\latex{\inputlisting{crtex/ex2.pp}}
 \html{\input{crtex/ex2.tex}}

\Procedure {LowVideo}
{ LowVideo switches the output to non-highlighted text. (It clears the high
intensity bit of the video attribute)
}
{None.}{ \seep{TextColor}, \seep{TextBackground}, \seep{HighVideo},
\seep{NormVideo}}

For an example, see \seep{HighVideo}

\Procedure {NormVideo}
{ NormVideo switches the output to the defaults, read at startup. (The
defaults are read from the cursor position at startup)
}
{None.}{ \seep{TextColor}, \seep{TextBackground}, \seep{LowVideo},
\seep{HighVideo}}

For an example, see \seep{HighVideo}

\Procedure{NoSound}{
Stops the speaker sound.

This is not supported in \linux}{None.}{\seep{Sound}}

\latex{\inputlisting{crtex/ex16.pp}}
 \html{\input{crtex/ex16.tex}}

\Function  {ReadKey}{Char}
{
The ReadKey function reads 1 key from the keyboard buffer, and returns this.
If an extended or function key has been pressed, then the zero ASCII code is 
returned. You can then read the scan code of the key with a second ReadKey
call.

\textbf{Remark.} Key mappings under Linux can cause the wrong key to be
reported by ReadKey, so caution is needed when using ReadKey.  
}
{None.}{\seef{KeyPressed}}

\latex{\inputlisting{crtex/ex3.pp}}
 \html{\input{crtex/ex3.tex}}


\procedure{Sound}{(hz : word)}
{ Sounds the speaker at a frequency of \var{hz}.

This is not supported in \linux}{None.}{\seep{NoSound}}

\procedure {TextBackground}{(CL: Byte)}
{
TextBackground sets the background color to \var{CL}. \var{CL} can be one of the
predefined color constants.
}
{None.}{ \seep{TextColor}, \seep{HighVideo}, \seep{LowVideo},
\seep{NormVideo}}

\latex{\inputlisting{crtex/ex13.pp}}
 \html{\input{crtex/ex13.tex}}

\procedure {TextColor}{(CL: Byte)}
{
TextColor sets the foreground color to \var{CL}. \var{CL} can be one of the
predefined color constants.
}
{None.}{ \seep{TextBackground}, \seep{HighVideo}, \seep{LowVideo},
\seep{NormVideo}}

\latex{\inputlisting{crtex/ex12.pp}}
 \html{\input{crtex/ex12.tex}}

\Function  {WhereX}{Byte}
{
WhereX returns the current X-coordinate of the cursor, relative to the
current window. The origin is \var{(1,1)}, in the upper-left corner of the
window.
}
{None.}{ \seep{GotoXY}, \seef{WhereY}, \seep{Window} }


\latex{\inputlisting{crtex/ex7.pp}}
 \html{\input{crtex/ex7.tex}}

\Function  {WhereY}{Byte}
{
WhereY returns the current Y-coordinate of the cursor, relative to the
current window. The origin is \var{(1,1)}, in the upper-left corner of the
window.
}
{None.}{ \seep{GotoXY}, \seef{WhereX}, \seep{Window} }

\latex{\inputlisting{crtex/ex7.pp}}
 \html{\input{crtex/ex7.tex}}

\procedure {Window}{(X1, Y1, X2, Y2: Byte)}
{ Window creates a window on the screen, to which output will be sent.
\var{(X1,Y1)} are the coordinates of the upper left corner of the window,
\var{(X2,Y2)} are the coordinates of the bottom right corner of the window.
These coordinates are relative to the entire screen, with the top left
corner equal to \var{(1,1)}

Further coordinate operations, except for the next Window call,
are relative to the window's top left corner.
}
{None.}{\seep{GotoXY}, \seef{WhereX}, \seef{WhereY}, \seep{ClrScr}}

\latex{\inputlisting{crtex/ex5.pp}}
\html{\input{crtex/ex5.tex}}


%\procedure {ScrollWindow}{(X1,Y1,X2,Y2 : Byte; Count : Longint)}
%{ ScrollWindow scrolls the contents of the window defined by the upper-left
%\var{(X1,Y1)} and lower-right \var{(X2,Y2)} corners \var{count} lines up if
%\var{count} is positive, it scrolls down if \var{count} is negative.
%The new lines are made blank using the current textcolors.
%}
%{None.}{\seep{Window}, \seep{ClrScr}}

%\function {SaveScreenRegion}{(X1,Y1,X2,Y2, var P : pointer)}{Boolean}
%{SaveScreenRegion writes the the contents of the window defined by the upper-left
%\var{(X1,Y1)} and lower-right \var{(X2,Y2)} corners to the location pointed
%to by \var{P}. If \var{P} is \var{nil} then enough memory is allocated to
%contain the contents of the window.
%
%The contents are written as follows : line per line, column per column,
%first the character on screen is written (1 byte), followed by the text 
%attribute (1 byte). The size required is therefore 
%\var{(Y2-Y1+1)*(X2-X1+1)*2} bytes.
%
%The function returns \var{False} if it couldn't allocate the required
%memory, \var{True} if the memory was allocated.}{None.}
%{\seep{RestoreScreenRegion}, \seep{Window} }

%\procedure {RestoreScreenRegion}{(X1,Y1,X2,Y2, var P : pointer)}
%{SaveScreenRegion writes the the contents of the memory location pointed to
%by \var{P}, to the window defined by the upper-left \var{(X1,Y1)} and 
%lower-right \var{(X2,Y2)} corners. 
%
%The contents of \var{P} should be arranged as if they are when written by 
%a call to the SaveScreenRegion () function.
%
%The memory pointed to by \var{P} is NOT freed.}{None}
%{\seef{SaveScreenRegion}, \seep{Window} }


% The Dos unit
%
%   $Id$
%   This file is part of the FPC documentation.
%   Copyright (C) 1997, by Michael Van Canneyt
%
%   The FPC documentation is free text; you can redistribute it and/or
%   modify it under the terms of the GNU Library General Public License as
%   published by the Free Software Foundation; either version 2 of the
%   License, or (at your option) any later version.
%
%   The FPC Documentation is distributed in the hope that it will be useful,
%   but WITHOUT ANY WARRANTY; without even the implied warranty of
%   MERCHANTABILITY or FITNESS FOR A PARTICULAR PURPOSE.  See the GNU
%   Library General Public License for more details.
%
%   You should have received a copy of the GNU Library General Public
%   License along with the FPC documentation; see the file COPYING.LIB.  If not,
%   write to the Free Software Foundation, Inc., 59 Temple Place - Suite 330,
%   Boston, MA 02111-1307, USA. 
%
\chapter{The DOS unit.}
\FPCexampledir{dosex}

This chapter describes the \var{DOS} unit for Free pascal. The \var{DOS}
unit gives access to some operating system calls related to files, the
file system, date and time. Except for the \palmos target, this unit is
available to all supported platforms.

The unit was first written for \dos by Florian Kl\"ampfl. It was ported to 
\linux by Mark May\footnote{Current e-mail address \textsf{mmay@dnaco.net}}, 
and enhanced by Micha\"el Van Canneyt. The \amiga version was ported by
Nils Sjoholm.

Under non-\dos systems, some of the functionality is lost, as it is either impossible 
or meaningless to implement it. Other than that, the functionality for all 
operating systems is the same.

This chapter is divided in three sections:
\begin{itemize}
\item The first section lists the pre-defined constants, types and variables. 
\item The second section gives an overview of all functions available,
grouped by category.
\item The third section describes the functions which appear in the
interface part of the DOS unit.
\end{itemize}

\section{Types, Variables, Constants}


\subsection {Constants}
The DOS unit implements the following constants:

\subsubsection{File attributes}

The File Attribute constants are used in \seep{FindFirst}, \seep{FindNext} to
determine what type of special file to search for in addition to normal files. 
These flags are also used in the \seep{SetFAttr} and \seep{GetFAttr} routines to 
set and retrieve attributes of files. For their definitions consult 
\seet{fileattributes}.

\begin{FPCltable}{lll}{Possible file attributes}{fileattributes}
\hline
Constant & Description & Value\\ \hline
\var{readonly} & Read only file & \$01\\
\var{hidden} & Hidden file & \$02 \\
\var{sysfile} & System file & \$04\\
\var{volumeid} & Volume label & \$08\\
\var{directory} & Directory & \$10\\
\var{archive} & Archive & \$20\\ 
\var{anyfile} & Any of the above special files & \$3F\\
\hline
\end{FPCltable}

\subsubsection{fmXXXX}

These constants are used in the \var{Mode} field of the \var{TextRec}
record. Gives information on the filemode of the text I/O. For their
definitions consult \seet{fmxxxconstants}.

\begin{FPCltable}{lll}{Possible mode constants}{fmxxxconstants}
\hline
Constant & Description & Value\\ \hline
\var{fmclosed} & File is closed & \$D7B0\\
\var{fminput} & File is read only & \$D7B1 \\
\var{fmoutput} & File is write only & \$D7B2\\
\var{fminout} & File is read and write & \$D7B3\\
\hline
\end{FPCltable}

\subsubsection{Other}

The following constants are not portable, and should not be used. They
are present for compatibility only.

\begin{verbatim}
  {Bitmasks for CPU Flags}
  fcarry =     $0001;
  fparity =    $0004;
  fauxiliary = $0010;
  fzero =      $0040;
  fsign =      $0080;
  foverflow  = $0800;
\end{verbatim}  
  
\subsection{Types}
The following string types are defined for easy handling of
filenames :
\begin{verbatim}
  ComStr  = String[255];   { For command-lines } 
  PathStr = String[255];   { For full path for file names }
  DirStr  = String[255];   { For Directory and (DOS) drive string }
  NameStr = String[255];   { For Name of file }
  ExtStr  = String[255];   { For Extension of file }
\end{verbatim}
\begin{verbatim}
  SearchRec = Packed Record
    Fill : array[1..21] of byte;  
    { Fill replaced with declarations below, for Linux}
    Attr : Byte; {attribute of found file}
    Time : LongInt; {last modify date of found file}
    Size : LongInt; {file size of found file}
    Reserved : Word; {future use}
    Name : String[255]; {name of found file}
    SearchSpec: String[255]; {search pattern}
    NamePos: Word; {end of path, start of name position}
    End;
\end{verbatim}
Under \linux, the \var{Fill} array is replaced with the following:
\begin{verbatim}
    SearchNum: LongInt; {to track which search this is}
    SearchPos: LongInt; {directory position}
    DirPtr: LongInt; {directory pointer for reading directory}
    SearchType: Byte;  {0=normal, 1=open will close}
    SearchAttr: Byte; {attribute we are searching for}
    Fill: Array[1..07] of Byte; {future use}
\end{verbatim}
This is because the searching meachanism on Unix systems is substantially
different from \dos's, and the calls have to be mimicked.
\begin{verbatim}
const
  filerecnamelength = 255;
type
  FileRec = Packed Record
    Handle,
    Mode,  
    RecSize   : longint;
    _private  : array[1..32] of byte;
    UserData  : array[1..16] of byte;
    name      : array[0..filerecnamelength] of char;
  End;
\end{verbatim}
\var{FileRec} is used for internal representation of typed and untyped files.
Text files are handled by the following types :
\begin{verbatim}
const
  TextRecNameLength = 256;
  TextRecBufSize    = 256;
type
  TextBuf = array[0..TextRecBufSize-1] of char;
  TextRec = Packed Record
    Handle,
    Mode,  
    bufsize,
    _private,
    bufpos,  
    bufend    : longint;
    bufptr    : ^textbuf;
    openfunc,
    inoutfunc,
    flushfunc,
    closefunc : pointer;
    UserData  : array[1..16] of byte;
    name      : array[0..textrecnamelength-1] of char;
    buffer    : textbuf;
  End;
\end{verbatim}
Remark that this is not binary compatible with the Turbo Pascal definition
of \var{TextRec}, since the sizes of the different fields are different.
\begin{verbatim}
    Registers = record
      case i : integer of
        0 : (ax,f1,bx,f2,cx,f3,dx,f4,bp,f5,si,
             f51,di,f6,ds,f7,es,f8,flags,fs,gs : word);
        1 : (al,ah,f9,f10,bl,bh,f11,f12,
             cl,ch,f13,f14,dl,dh : byte);
        2 : (eax,  ebx,  ecx,  edx,  ebp,  esi,  edi : longint);
        End;
\end{verbatim}
The  \var{registers} type is used in the \var{MSDos} call.
\begin{verbatim}
  DateTime = record
    Year: Word;
    Month: Word;
    Day: Word;
    Hour: Word;
    Min: Word;
    Sec: word;
    End;
\end{verbatim}
The \var{DateTime} type is used in \seep{PackTime} and \seep{UnPackTime} for
setting/reading file times with \seep{GetFTime} and \seep{SetFTime}.
\subsection{Variables}
\begin{verbatim}
    DosError : integer;
\end{verbatim}
The \var{DosError} variable is used by the procedures in the \dos unit to 
report errors. It can have the following values :
\begin{center}
\begin{tabular}{cl}
2 & File not found. \\
3 & path not found. \\
5 & Access denied. \\
6 & Invalid handle. \\
8 & Not enough memory. \\
10 & Invalid environment. \\
11 & Invalid format. \\
18 & No more files.
\end{tabular}
\end{center}
Other values are possible, but are not documented.
%\begin{verbatim}
%    drivestr : array [0..26] of pchar;
%\end{verbatim}
%This variable is defined in the \linux version of the \dos unit. It is used
%in the \seef{DiskFree} and \seef{DiskSize} calls.

%%%%%%%%%%%%%%%%%%%%%%%%%%%%%%%%%%%%%%%%%%%%%%%%%%%%%%%%%%%%%%%%%%%%%%%
% Functions and procedures by category
\section{Function list by category}
What follows is a listing of the available functions, grouped by category.
For each function there is a reference to the page where you can find the
function.

\subsection{File handling}
Routines to handle files on disk.
\begin{funclist}
\funcrefl{FExpand}{Dos:FExpand}{Expand filename to full path}
\procref{FindClose}{Close finfirst/findnext session}
\procref{FindFirst}{Start find of file}
\procref{FindNext}{Find next file}
\funcrefl{FSearch}{Dos:FSearch}{Search for file in a path}
\procref{FSplit}{Split filename in parts}
\procref{GetFAttr}{Return file attributes}
\procref{GetFTime}{Return file time}
\funcref{GetLongName}{Convert short filename to long filename (DOS only)}
\funcref{GetShortName}{Convert long filename to short filename (DOS only)}
\procref{SetFAttr}{Set file attributes}
\procref{SetFTime}{Set file time}
\end{funclist}

\subsection{Directory and disk handling}
Routines to handle disk information.
\begin{funclist}
\procref{AddDisk}{Add disk to list of disks (UNIX only)}
\funcref{DiskFree}{Return size of free disk space}
\funcref{DiskSize}{Return total disk size}
\end{funclist}

\subsection{Process handling}
Functions to handle process information and starting new processes.
\begin{funclist}
\funcref{DosExitCode}{Exit code of last executed program}
\funcref{EnvCount}{Return number of environment variables}
\funcref{EnvStr}{Return environment string pair}
\procref{Exec}{Execute program}
\funcrefl{GetEnv}{Dos:GetEnv}{Return specified environment string}
\end{funclist}

\subsection{System information}
Functions for retrieving and setting general system information such as date
and time.
\begin{funclist}
\funcref{DosVersion}{Get OS version}
\procref{GetCBreak}{Get setting of control-break handling flag}
\procrefl{GetDate}{Dos:GetDate}{Get system date}
\procref{GetIntVec}{Get interrupt vector status}
\procrefl{GetTime}{Dos:GetTime}{Get system time}
\procref{GetVerify}{Get verify flag}
\procref{Intr}{Execute an interrupt}
\procref{Keep}{Keep process in memory and exit}
\procref{MSDos}{Execute MS-dos function call}
\procref{PackTime}{Pack time for file time}
\procref{SetCBreak}{Set control-break handling flag}
\procref{SetDate}{Set system date}
\procref{SetIntVec}{Set interrupt vectors}
\procref{SetTime}{Set system time}
\procref{SetVerify}{Set verify flag}
\procref{SwapVectors}{Swap interrupt vectors}
\procref{UnPackTime}{Unpack file time}
\end{funclist}

%%%%%%%%%%%%%%%%%%%%%%%%%%%%%%%%%%%%%%%%%%%%%%%%%%%%%%%%%%%%%%%%%%%%%%%
% Functions and procedures
\section{Functions and Procedures}
\begin{procedure}{AddDisk}
\Declaration
Procedure AddDisk (Const S : String);
\Description
\var{AddDisk} adds a filename \var{S} to the internal list of disks. It is
implemented for systems which do not use DOS type drive letters.
 This list is used to determine which disks to use in the \seef{DiskFree}
and \seef{DiskSize} calls. 
The \seef{DiskFree} and \seef{DiskSize} functions need a file on the 
specified drive, since this is required for the \var{statfs} system call.
The names are added sequentially. The dos
initialization code presets the first three disks to:
\begin{itemize}
\item \var{'.'} for the current drive, 
\item \var{'/fd0/.'} for the first floppy-drive (linux only).
\item \var{'/fd1/.'} for the second floppy-drive (linux only).
\item \var{'/'} for the first hard disk.
\end{itemize}
The first call to \var{AddDisk} will therefore add a name for the second
harddisk, The second call for the third drive, and so on until 23 drives
have been added (corresponding to drives \var{'D:'} to \var{'Z:'})
\Errors
None
\SeeAlso
\seef{DiskFree}, \seef{DiskSize} 
\end{procedure}


\begin{function}{DiskFree}
\Declaration
Function DiskFree (Drive: byte) : int64;
\Description

\var{DiskFree} returns the number of free bytes on a disk. The parameter
\var{Drive} indicates which disk should be checked. This parameter is 1 for
floppy \var{a:}, 2 for floppy \var{b:}, etc. A value of 0 returns the free
space on the current drive. 

\textbf{For \unix only:}\\
The \var{diskfree} and \var{disksize} functions need a file on the 
specified drive, since this is required for the \var{statfs} system call.
These filenames are set in the initialization of the dos unit, and have 
been preset to :
\begin{itemize}
\item \var{'.'} for the current drive, 
\item \var{'/fd0/.'} for the first floppy-drive (linux only).
\item \var{'/fd1/.'} for the second floppy-drive (linux only).
\item \var{'/'} for the first hard disk.
\end{itemize}
There is room for 1-26 drives. You can add a drive with the
\seep{AddDisk} procedure.
These settings can be coded in \var{dos.pp}, in the initialization part.

\Errors
-1 when a failure occurs, or an invalid drive number is given.
\SeeAlso
\seef{DiskSize}, \seep{AddDisk}
\end{function}

\FPCexample{ex6}

\begin{function}{DiskSize}
\Declaration
Function DiskSize (Drive: byte) : int64;
\Description

\var{DiskSize} returns the total size (in bytes) of a disk. The parameter
\var{Drive} indicates which disk should be checked. This parameter is 1 for
floppy \var{a:}, 2 for floppy \var{b:}, etc. A value of 0 returns the size
of the current drive. 

\textbf{For \unix only:}\\
The \var{diskfree} and \var{disksize} functions need a file on the specified drive, since this
is required for the \var{statfs} system call.
  These filenames are set in the initialization of the dos unit, and have 
been preset to :
\begin{itemize}
\item \var{'.'} for the current drive, 
\item \var{'/fd0/.'} for the first floppy-drive (linux only).
\item \var{'/fd1/.'} for the second floppy-drive (linux only).
\item \var{'/'} for the first hard disk.
\end{itemize}
There is room for 1-26 drives. You can add a drive with the
\seep{AddDisk} procedure.
These settings can be coded in \var{dos.pp}, in the initialization part.

\Errors
-1 when a failure occurs, or an invalid drive number is given.
\SeeAlso
\seef{DiskFree}, \seep{AddDisk}
\end{function}
For an example, see \seef{DiskFree}.
\begin{function}{DosExitCode}
\Declaration
Function DosExitCode  : Word;
\Description

\var{DosExitCode} contains (in the low byte) the exit-code of a program 
executed with the \var{Exec} call.
\Errors
None.
\SeeAlso
\seep{Exec}
\end{function}

\FPCexample{ex5}

\begin{function}{DosVersion}
\Declaration
Function DosVersion  : Word;
\Description
\var{DosVersion} returns the operating system or kernel version. The
low byte contains the major version number, while the high byte 
contains the minor version number.
\Portability
On systems where versions consists of more then two numbers, 
only the first two numbers will be returned. For example Linux version 2.1.76 
will give you DosVersion 2.1. Some operating systems, such as \freebsd, do not
have system calls to return the kernel version, in that case a value of 0 will
be returned.
\Errors
None.
\SeeAlso

\end{function}


\FPCexample{ex1}


\begin{function}{EnvCount}
\Declaration
Function EnvCount  : longint;\Description
\var{EnvCount} returns the number of environment variables.
\Errors
None.
\SeeAlso
\seef{EnvStr}, \seef{Dos:GetEnv}
\end{function}

\begin{function}{EnvStr}
\Declaration
Function EnvStr (Index: integer) : string;\Description

\var{EnvStr} returns the \var{Index}-th \var{Name=Value} pair from the list
of environment variables. 
The index of the first pair is zero.
\Errors
The length is limited to 255 characters. 
\SeeAlso
\seef{EnvCount}, \seef{Dos:GetEnv}
\end{function}

\FPCexample{ex13}

\begin{procedure}{Exec}
\Declaration
Procedure Exec (const Path: pathstr; const ComLine: comstr);
\Description

\var{Exec} executes the program in \var{Path}, with the options given by
\var{ComLine}.
After the program has terminated, the procedure returns. The Exit value of
the program can be consulted with the \var{DosExitCode} function.

\Errors
Errors are reported in \var{DosError}.
\SeeAlso
\seef{DosExitCode}
\end{procedure}
For an example, see \seef{DosExitCode}
\begin{functionl}{FExpand}{Dos:FExpand}
\Declaration
Function FExpand (const path: pathstr) : pathstr;
\Description

\var{FExpand} takes its argument and expands it to a complete filename, i.e.
a filename starting from the root directory of the current drive, prepended
with the drive-letter or volume name (when supported).
\Portability
On case sensitive file systems (such as \unix and \linux), the resulting
name is left as it is, otherwise it is converted to uppercase.
\Errors
\seep{FSplit}
\SeeAlso
\end{functionl}

\FPCexample{ex5}

\begin{procedure}{FindClose}
\Declaration
Procedure FindClose (Var F: SearchRec);
\Description
\var{FindClose} frees any resources associated with the search record
\var{F}.

This call is needed to free any internal resources allocated by the 
\seef{FindFirst} or \seef{FindNext} calls.


The \linux implementation of the \dos unit therefore keeps a table of open
directories, and when the table is full, closes one of the directories, and
reopens another. This system is adequate but slow if you use a lot of
\var{searchrecs}.
So, to speed up the findfirst/findnext system, the \var{FindClose} call was
implemented. When you don't need a \var{searchrec} any more, you can tell
this to the \dos unit by issuing a \var{FindClose} call. The directory
which is kept open for this \var{searchrec} is then closed, and the table slot
freed.

\Portability
It is recommended to use the \linux call \var{Glob} when looking for files 
on \linux.

\Errors
Errors are reported in DosError.
\SeeAlso
\seef{Glob}.
\end{procedure}

\begin{procedure}{FindFirst}
\Declaration
Procedure FindFirst (const Path: pathstr; Attr: word; var F: SearchRec);
\Description

\var{FindFirst} searches the file specified in \var{Path}. Normal files,
as well as all special files which have the attributes specified in
\var{Attr} will be returned.

It returns a \var{SearchRec} record for further searching in \var{F}.
\var{Path} can contain the wildcard characters \var{?} (matches any single
character) and \var{*} (matches 0 ore more arbitrary characters). In this
case \var{FindFirst} will return the first file which matches the specified
criteria.
If \var{DosError} is different from zero, no file(s) matching the criteria 
was(were) found.

\Portability
On \ostwo, you cannot issue two different \var{FindFirst} calls. That is,
you must close any previous search operation with \seep{FindClose} before
starting a new one. Failure to do so will end in a Run-Time Error 6 
(Invalid file handle)

\Errors
Errors are reported in DosError.
\SeeAlso
\seep{FindNext},
\seep{FindClose}
\end{procedure}

\FPCexample{ex7}

\begin{procedure}{FindNext}
\Declaration
Procedure FindNext (var f: searchRec);
\Description

\var{FindNext} takes as an argument a \var{SearchRec} from a previous
\var{FindNext} call, or a \var{FindFirst} call, and tries to find another
file which matches the criteria, specified in the \var{FindFirst} call.
If \var{DosError} is different from zero, no more files matching the
criteria were found.
\Errors
\var{DosError} is used to report errors.
\SeeAlso
\seep{FindFirst}, \seep{FindClose}
\end{procedure}
For an example, see \seep{FindFirst}.
\begin{functionl}{FSearch}{Dos:FSearch}
\Declaration
Function FSearch (Path: pathstr; DirList: string) : pathstr;
\Description
\var{FSearch} searches the file \var{Path} in all directories listed in
\var{DirList}. The full name of the found file is returned.
\var{DirList} must be a list of directories, separated by semi-colons.
When no file is found, an empty string is returned.
\Portability
On \unix systems, \var{DirList} can also be separated by colons, as is
customary on those environments.
\Errors
None.
\SeeAlso
\seefl{FExpand}{Dos:FExpand}
\end{functionl}

\FPCexample{ex10}

 
\begin{procedure}{FSplit}
\Declaration
Procedure FSplit (path: pathstr; \\ var dir: dirstr; var name: namestr;
  var ext: extstr);
\Description

\var{FSplit} splits a full file name into 3 parts : A \var{Path}, a
\var{Name} and an extension  (in \var{ext}.) 
The extension is taken to be all letters after the {\em last} dot (.). For 
\dos, however, an exception is made when \var{LFNSupport=False}, then
the extension is defined as all characters after the {\em first} dot.

\Errors
None.
\SeeAlso
\seefl{FSearch}{Dos:FSearch}
\end{procedure}

\FPCexample{ex12}

\begin{procedure}{GetCBreak}
\Declaration
Procedure GetCBreak (var breakvalue: boolean);
\Description

\var{GetCBreak} gets the status of CTRL-Break checking under \dos and \amiga.
When \var{BreakValue} is \var{false}, then \dos only checks for the 
CTRL-Break key-press when I/O is performed. When it is set to \var{True},
then a check is done at every system call.
\Portability
Under non-\dos and non-\amiga operating systems, \var{BreakValue} always returns 
\var{True}.
\Errors 
None
\SeeAlso
\seep{SetCBreak}
\end{procedure}

\begin{procedurel}{GetDate}{Dos:GetDate}
\Declaration
Procedure GetDate (var year, month, mday, wday: word);\Description

\var{GetDate} returns the system's date. \var{Year} is a number in the range
1980..2099.\var{mday} is the day of the month,
\var{wday} is the day of the week, starting with Sunday as day 0.
\Errors
None.
\SeeAlso
\seepl{GetTime}{Dos:GetTime},\seep{SetDate}
\end{procedurel}

\FPCexample{ex2}

\begin{functionl}{GetEnv}{Dos:GetEnv}
\Declaration
Function GetEnv (EnvVar: String) : String;
\Description

\var{Getenv} returns the value of the environment variable \var{EnvVar}.
When there is no environment variable \var{EnvVar} defined, an empty
string is returned.
\Portability
Under some operating systems (such as \unix), case is important when looking
for \var{EnvVar}.
\Errors
None.
\SeeAlso
\seef{EnvCount}, \seef{EnvStr}
\end{functionl}

\FPCexample{ex14}

\begin{procedure}{GetFAttr}
\Declaration
Procedure GetFAttr (var F; var Attr: word);
\Description

\var{GetFAttr} returns the file attributes of the file-variable \var{f}.
 \var{F} can be a untyped or typed file, or of type \var{Text}. \var{f} must
have been assigned, but not opened. The attributes can be examined with the
following constants :
\begin{itemize}
\item \var{ReadOnly}
\item \var{Hidden}
\item \var{SysFile}
\item \var{VolumeId}
\item \var{Directory}
\item \var{Archive}
\end{itemize}
Under \linux, supported attributes are:
\begin{itemize}
\item \var{Directory}
\item \var{ReadOnly} if the current process doesn't have access to the file.
\item \var{Hidden} for files whose name starts with a dot \var{('.')}.
\end{itemize}

\Errors
Errors are reported in \var{DosError}
\SeeAlso
\seep{SetFAttr}
\end{procedure}

\FPCexample{ex8}

\begin{procedure}{GetFTime}
\Declaration
Procedure GetFTime (var F; var Time: longint);
\Description

\var{GetFTime} returns the modification time of a file.
This time is encoded and must be decoded with \var{UnPackTime}. 
\var{F} must be a file type, which has been assigned, and
opened.
\Errors
Errors are reported in \var{DosError}
\SeeAlso
\seep{SetFTime}, \seep{PackTime},\seep{UnPackTime}
\end{procedure}

\FPCexample{ex9}

\begin{procedure}{GetIntVec}
\Declaration
Procedure GetIntVec (IntNo: byte; var Vector: pointer);
\Description

\var{GetIntVec} returns the address of interrupt vector
\var{IntNo}.
\Portability
This call does nothing, it is present for compatibility only.
\Errors
None.
\SeeAlso
\seep{SetIntVec}
\end{procedure}

\begin{function}{GetLongName}
\Declaration
function GetLongName(var p : String) : boolean;\Description
This function is only implemented in the GO32V2 version of \fpc.

\var{GetLongName} changes the filename \var{p} to a long filename
if the \dos call to do this is successful. The resulting string
is the long  file name corresponding to the short filename \var{p}.

The function returns \var{True} if the \dos call was successful, 
\var{False} otherwise.

This function should only be necessary when using the DOS extender
under Windows 95 and higher.
\Errors
If the \dos call was not succesfull, \var{False} is returned.
\SeeAlso
\seef{GetShortName}
\end{function}

\begin{function}{GetShortName}
\Declaration
function GetShortName(var p : String) : boolean;\Description
This function is only implemented in the GO32V2 version of \fpc.

\var{GetShortName} changes the filename \var{p} to a short filename
if the \dos call to do this is successful. The resulting string
is the short file name corresponding to the long filename \var{p}.

The function returns \var{True} if the \dos call was successful, 
\var{False} otherwise.

This function should only be necessary when using the DOS extender
under Windows 95 and higher.
\Errors
If the \dos call was not successful, \var{False} is returned.
\SeeAlso
\seef{GetLongName}
\end{function}

\begin{procedurel}{GetTime}{Dos:GetTime}
\Declaration
Procedure GetTime (var hour, minute, second, sec100: word);
\Description

\var{GetTime} returns the system's time. \var{Hour} is a on a 24-hour time
scale. \var{sec100} is in hundredth of a
second.
\Portability
Certain operating systems (such as \amiga), always set the \var{sec100} field
to zero.
\Errors
None.
\SeeAlso
\seepl{GetDate}{Dos:GetDate},
\seep{SetTime}
\end{procedurel}


\FPCexample{ex3}


\begin{procedure}{GetVerify}
\Declaration
Procedure GetVerify (var verify: boolean);
\Description

\var{GetVerify} returns the status of the verify flag under \dos. When
\var{Verify} is \var{True}, then \dos checks data which are written to disk,
by reading them after writing. If \var{Verify} is \var{False}, then data
written to disk are not verified.
\Portability
Under non-\dos systems (excluding \ostwo applications running under vanilla DOS),  
Verify is always \var{True}.
\Errors
None.
\SeeAlso
\seep{SetVerify}
\end{procedure}
\begin{procedure}{Intr}
\Declaration
Procedure Intr (IntNo: byte; var Regs: registers);
\Description

\var{Intr} executes a software interrupt number \var{IntNo} (must be between
0 and 255), with processor registers set to \var{Regs}. After the interrupt call
returned, the processor registers are saved in \var{Regs}.
\Portability
Under non-\dos operating systems, this call does nothing.
\Errors
None.
\SeeAlso
\seep{MSDos}, see the \linux unit.
\end{procedure}

\begin{procedure}{Keep}
\Declaration
Procedure Keep (ExitCode: word);
\Description
\var{Keep} terminates the program, but stays in memory. This is used for TSR
(Terminate Stay Resident) programs which catch some interrupt.
\var{ExitCode} is the same parameter as the \var{Halt} function takes.
\Portability
This call does nothing, it is present for compatibility only.
\Errors
None.
\SeeAlso
\seem{Halt}{}
\end{procedure}
\begin{procedure}{MSDos}
\Declaration
Procedure MSDos (var regs: registers);
\Description

\var{MSDos} executes an operating system. This is the same as doing a
\var{Intr} call with the interrupt number for an os call.
\Portability
Under non-\dos operating systems, this call does nothing. On \dos systems,
this calls interrupt \$21.
\Errors
None.
\SeeAlso
\seep{Intr}
\end{procedure}
\begin{procedure}{PackTime}
\Declaration
Procedure PackTime (var T: datetime; var P: longint);
\Description

\var{UnPackTime} converts the date and time specified in \var{T}
to a packed-time format which can be fed to \var{SetFTime}.
\Errors
None.
\SeeAlso
\seep{SetFTime}, \seep{FindFirst}, \seep{FindNext}, \seep{UnPackTime}
\end{procedure}

\FPCexample{ex4}

\begin{procedure}{SetCBreak}
\Declaration
Procedure SetCBreak (breakvalue: boolean);
\Description

\var{SetCBreak} sets the status of CTRL-Break checking. When 
\var{BreakValue} is \var{false}, then \dos only checks for the CTRL-Break 
key-press when I/O is performed. When it is set to \var{True}, then a 
check is done at every system call.
\Portability
Under non-\dos and non-\amiga operating systems, this call does nothing.
\Errors
None.
\SeeAlso
\seep{GetCBreak}
\end{procedure}
\begin{procedure}{SetDate}
\Declaration
Procedure SetDate (year,month,day: word);
\Description

\var{SetDate} sets the system's internal date. \var{Year} is a number
between 1980 and 2099.
\Portability
On a \linux machine, there must be root privileges, otherwise this
routine will do nothing. On other \unix systems, this call currently
does nothing.
\Errors
None.
\SeeAlso
\seep{Dos:GetDate},
\seep{SetTime}
\end{procedure}

\begin{procedure}{SetFAttr}
\Declaration
Procedure SetFAttr (var F; Attr: word);
\Description

\var{SetFAttr} sets the file attributes of the file-variable \var{F}.
 \var{F} can be a untyped or typed file, or of type \var{Text}. \var{F} must
have been assigned, but not opened. The attributes can be a sum of the
following constants:
\begin{itemize}
\item \var{ReadOnly}
\item \var{Hidden}
\item \var{SysFile}
\item \var{VolumeId}
\item \var{Directory}
\item \var{Archive}
\end{itemize}

\Portability
Under \unix like systems (such as \linux and \beos) the call exists, but is not implemented, 
i.e. it does nothing.
\Errors
Errors are reported in \var{DosError}.
\SeeAlso
\seep{GetFAttr}
\end{procedure}
\begin{procedure}{SetFTime}
\Declaration
Procedure SetFTime (var F; Time: longint);
\Description

\var{SetFTime} sets the modification time of a file,
this time is encoded and must be encoded with \var{PackTime}. 
\var{F} must be a file type, which has been assigned, and
opened.
\Portability
Under \unix like systems (such as \linux and \beos) the call exists, but is not implemented, 
i.e. it does nothing.
\Errors
Errors are reported in \var{DosError}
\SeeAlso
\seep{GetFTime}, \seep{PackTime},\seep{UnPackTime}
\end{procedure}

\begin{procedure}{SetIntVec}
\Declaration
Procedure SetIntVec (IntNo: byte; Vector: pointer);
\Description
\var{SetIntVec} sets interrupt vector \var{IntNo} to \var{Vector}.
\var{Vector} should point to an interrupt procedure.
\Portability
This call does nothing, it is present for compatibility only.
\Errors
None.
\SeeAlso
\seep{GetIntVec}
\end{procedure}

\begin{procedure}{SetTime}
\Declaration
Procedure SetTime (hour,minute,second,sec100: word);
\Description
\var{SetTime} sets the system's internal clock. The \var{Hour} parameter is
on a 24-hour time scale.
\Portability
On a \linux machine, there must be root privileges, otherwise this
routine will do nothing. On other \unix systems, this call currently
does nothing.
\Errors
None.
\SeeAlso
\seep{Dos:GetTime}, \seep{SetDate}
\end{procedure}
\begin{procedure}{SetVerify}
\Declaration
Procedure SetVerify (verify: boolean);
\Description

\var{SetVerify} sets the status of the verify flag under \dos. When
\var{Verify} is \var{True}, then \dos checks data which are written to disk,
by reading them after writing. If \var{Verify} is \var{False}, then data
written to disk are not verified.
\Portability
Under non-\dos operating systems (excluding \ostwo applications running
under vanilla dos), Verify is always \var{True}.
\Errors
None.
\SeeAlso
\seep{SetVerify}
\end{procedure}
\begin{procedure}{SwapVectors}
\Declaration
Procedure SwapVectors ;
\Description

\var{SwapVectors} swaps the contents of the internal table of interrupt 
vectors with the current contents of the interrupt vectors.
This is called typically in before and after an \var{Exec} call.

\Portability
Under certain operating systems, this routine may be implemented
as an empty stub.
\Errors
None.
\SeeAlso
\seep{Exec}, \seep{SetIntVec}
\end{procedure}
\begin{procedure}{UnPackTime}
\Declaration
Procedure UnPackTime (p: longint; var T: datetime);
\Description

\var{UnPackTime} converts the file-modification time in \var{p}
to a \var{DateTime} record. The file-modification time can be 
returned by \var{GetFTime}, \var{FindFirst} or \var{FindNext} calls.
\Errors
None.
\SeeAlso
\seep{GetFTime}, \seep{FindFirst}, \seep{FindNext}, \seep{PackTime}
\end{procedure}
For an example, see \seep{PackTime}.



% The DXELoad unit
%
%   $Id$
%   This file is part of the FPC documentation.
%   Copyright (C) 1997, by Michael Van Canneyt
%
%   The FPC documentation is free text; you can redistribute it and/or
%   modify it under the terms of the GNU Library General Public License as
%   published by the Free Software Foundation; either version 2 of the
%   License, or (at your option) any later version.
%
%   The FPC Documentation is distributed in the hope that it will be useful,
%   but WITHOUT ANY WARRANTY; without even the implied warranty of
%   MERCHANTABILITY or FITNESS FOR A PARTICULAR PURPOSE.  See the GNU
%   Library General Public License for more details.
%
%   You should have received a copy of the GNU Library General Public
%   License along with the FPC documentation; see the file COPYING.LIB.  If not,
%   write to the Free Software Foundation, Inc., 59 Temple Place - Suite 330,
%   Boston, MA 02111-1307, USA.
%
\chapter{The DXELOAD unit}
\section{Introduction}
The \file{dxeload} unit was implemented by Pierre M\"uller for \dos,
it allows to load a DXE file (an object file with 1 entry point)
into memory and return a pointer to the entry point.

It exists only for \dos.

\section{Constants, types and variables}
\subsection{Constants}
The following constant is the magic number, found in the header of a DXE
file.
\begin{verbatim}
DXE_MAGIC  = $31455844;
\end{verbatim}
\subsection{Types}
The following record describes the header of a DXE file. It is used to
determine the magic number of the DXE file and number of relocations that 
must be done when the object file i sloaded in memory.
\begin{verbatim}
dxe_header = record
   magic,
   symbol_offset,
   element_size,
   nrelocs       : longint;
end;
\end{verbatim}

\section{Functions and Procedures}
\begin{functionl}{dxe\_load}{dxeload}
\Declaration
function dxe\_load(filename : string) : pointer;
\Description
\var{dxe\_load} loads the contents of the file \var{filename} into memory.
It performs the necessary relocations in the object code, and returns then
a pointer to the entry point of the code.
\Errors
If an error occurs during the load or relocations, \var{Nil} is returned.
\end{functionl}
For an example, see the \file{emu387} unit in the RTL.

% The emu387 unit
%
%   $Id$
%   This file is part of the FPC documentation.
%   Copyright (C) 1997, by Michael Van Canneyt
%
%   The FPC documentation is free text; you can redistribute it and/or
%   modify it under the terms of the GNU Library General Public License as
%   published by the Free Software Foundation; either version 2 of the
%   License, or (at your option) any later version.
%
%   The FPC Documentation is distributed in the hope that it will be useful,
%   but WITHOUT ANY WARRANTY; without even the implied warranty of
%   MERCHANTABILITY or FITNESS FOR A PARTICULAR PURPOSE.  See the GNU
%   Library General Public License for more details.
%
%   You should have received a copy of the GNU Library General Public
%   License along with the FPC documentation; see the file COPYING.LIB.  If not,
%   write to the Free Software Foundation, Inc., 59 Temple Place - Suite 330,
%   Boston, MA 02111-1307, USA.
%
\chapter{The EMU387 unit}
The \file{emu387} unit was written by Pierre M\"uller for \dos. It
sets up the coprocessor emulation for FPC under \dos. It is not necessary to
use this unit on other OS platforms because they either simply do not run on 
a machine without coprocessor, or they provide the coprocessor emulation 
themselves.

It shouldn't be necessary to use the function in this unit, it should be
enough to place this unit in the \var{uses} clause of your program to
enable the coprocessor emulation under \dos. The unit initialization
code will try and load the coprocessor emulation code and initialize it.

\section{Functions and procedures}
\begin{function}{npxsetup}
\Declaration
procedure npxsetup(prog\_name : string);
\Description
\var{npxsetup} checks whether a coprocessor is found. If not, it loads the 
file \file{wmemu387.dxe} into memory and initializes the code in it.

If the environment variable \var{387} is set to \var{N}, then the emulation
will be loaded, even if there is a coprocessor present. If the variable
doesn't exist, or is set to any other value, the unit will try to detect 
the presence of a coprocessor unit.

The function searches the file \file{wmemu387.dxe} in the following way:
\begin{enumerate}
\item If the environment variable \var{EMU387} is set, then it is assumed
to point at the \file{wmemu387.dxe} file.
\item if the environment variable \var{EMU387} does not exist, then the 
function will take the path part of  \var{prog\_name} and look in that
directory for the file \file{wmemu387.dxe}.
\end{enumerate}

It should never be necessary to call this function, because the
initialization code of the unit contains a call to the function with
as an argument \var{paramstr(0)}. This means that you should deliver the
file \var{wmemu387.dxe} together with your program.
\Errors
If there is an error, an error message is printed to standard error, and
the program is halted, since any floating-point code is bound to fail anyhow.
\end{function}

% The getopts unit
%
%   $Id$
%   This file is part of the FPC documentation.
%   Copyright (C) 1997, by Michael Van Canneyt
%
%   The FPC documentation is free text; you can redistribute it and/or
%   modify it under the terms of the GNU Library General Public License as
%   published by the Free Software Foundation; either version 2 of the
%   License, or (at your option) any later version.
%
%   The FPC Documentation is distributed in the hope that it will be useful,
%   but WITHOUT ANY WARRANTY; without even the implied warranty of
%   MERCHANTABILITY or FITNESS FOR A PARTICULAR PURPOSE.  See the GNU
%   Library General Public License for more details.
%
%   You should have received a copy of the GNU Library General Public
%   License along with the FPC documentation; see the file COPYING.LIB.  If not,
%   write to the Free Software Foundation, Inc., 59 Temple Place - Suite 330,
%   Boston, MA 02111-1307, USA. 
%
\chapter{The GETOPTS unit.}
This document describes the GETOPTS unit for Free Pascal. It was written for
\linux\ by Micha\"el Van Canneyt. It also works under DOS and Tp7.

The chapter is divided in 2 sections:
\begin{itemize}
\item The first section lists types, constants and variables from the
interface part of the unit.
\item The second section describes the functions defined in the unit.
\end{itemize}

\section {Types, Constants and variables : }
\subsection{Constants}
\var{No\_Argument=0} : Specifies that a long option does not take an
argument. \\
\var{Required\_Argument=1} : Specifies that a long option needs an
argument. \\
\var{Optional\_Argument=2} : Specifies that a long option optionally takes an
argument. \\
\var{EndOfOptions=\#255}  : Returned by getopt, getlongopts to indicate that
there are no more options.
\subsection{Types}
\begin{verbatim}
TOption = record
  Name    : String;
  Has_arg : Integer;
  Flag    : PChar;
  Value   : Char;
  end;
POption = ^TOption;
\end{verbatim}
The \var{option} type is used to communicate the long options to \var{GetLongOpts}.
The \var{Name} field is the name of the option. \var{Has\_arg} specifies if the option
wants an argument, \var{Flag} is a pointer to a \var{char}, which is set to
\var{Value}, if it is non-\var{nil}. 

\var{POption} is a pointer to a
\var{Option} record. It is used as an argument to the \var{GetLongOpts}
function.

\subsection{Variables}
\var{OptArg:String} \ Is set to the argument of an option, if the option needs
one.\\
\var{Optind:Longint} \ Is the index of the current \var{paramstr()}. When
all options have been processed, \var{optind} is the index of the first
non-option parameter. This is a read-only variable. Note that it can become
equal to \var{paramcount+1}\\
\var{OptErr:Boolean} \ Indicates whether \var{getopt()} prints error
messages.\\
\var{OptOpt:Char} \  In case of an error, contains the character causing the 
error.
\section {Procedures and functions}

\function {GetLongOpts}{(Shortopts : String, LongOpts : POption; var Longint
: Longint )}{Char}
{
Returns the next option found on the command-line, taking into account long
options as well. If no more options are
found, returns \var{EndOfOptions}. If the option requires an argument, it is
returned in the \var{OptArg} variable.

\var{ShortOptions} is a string containing all possible one-letter options.
(see \seef{Getopt} for its description and use)

\var{LongOpts} is a pointer to the first element of an array of \var{Option} 
records, the last of which needs a name of zero length.  
The function tries to match the names even partially (i.e. \var{--app} 
will match e.g. the \var{append} option), but will report an error in case of
ambiguity.

If the option needs an argument, set \var{Has\_arg} to
\var{Required\_argument}, if the option optionally has an argument, set
\var{Has\_arg} to \var{Optional\_argument}. If the option needs no argument,
set \var{Has\_arg} to zero.


Required arguments can be specified in two ways : 
\begin{enumerate}
\item \ Pasted to the option : \var{--option=value}
\item \ As a separate argument : \var {--option value}
\end{enumerate}
Optional arguments can only be specified through the first method.
}
{ see \seef{Getopt}, \seem{getopt}{3}}

For an example, see \seef{Getopt}

\function {Getopt}{(Shortopts : String)}{Char}
{
Returns the next option found on the command-line. If no more options are
found, returns \var{EndOfOptions}. If the option requires an argument, it is
returned in the \var{OptArg} variable.

\var{ShortOptions} is a string containing all possible one-letter options.
If a letter is followed by a colon (:), then that option needs an argument.
If a letter is followed by 2 colons, the option has an optional argument.

If the first character of \var{shortoptions} is a \var{'+'} then options following a non-option are
regarded as non-options (standard Unix behavior). If it is a \var{'-'},
then all non-options are treated as arguments of a option with character
\var{\#0}. This is useful for applications that require their options in
the exact order as they appear on the command-line.

If the first character of \var{shortoptions} is none of the above, options
and non-options are permuted, so all non-options are behind all options.
This allows options and non-options to be in random order on the command
line.
}
{ 
Errors are reported through giving back a \var{'?'} character. \var{OptOpt}
then gives the character which caused the error. If \var{OptErr} is
\var{True} then getopt prints an error-message to \var{stdout}.
}
{\seef{GetLongOpts}, \seem{getopt}{3}}

\input{optex/optex.tex}


% The GPM unit
\chapter{The GPM unit}

\section{Introduction}
The \file{GPM} unit implements an interface to file{libgpm}, the console
program for mouse handling. This unit was created by Peter Vreman, and 
is only available on \linux.

When this unit is used, your program is linked to the C libraries, so
you must take care of the C library version. Also, it will only work with
version 1.17 or higher of the \file{libgpm} library.

%%%%%%%%%%%%%%%%%%%%%%%%%%%%%%%%%%%%%%%%%%%%%%%%%%%%%%%%%%%%%%%%%%%%%%%
% Constants, types and variables
\section{Constants, types and variables}
\subsection{constants}
The following constants are used to denote filenames used by the library:
\begin{verbatim}
_PATH_VARRUN = '/var/run/';
_PATH_DEV    = '/dev/';
GPM_NODE_DIR = _PATH_VARRUN;
GPM_NODE_DIR_MODE = 0775;
GPM_NODE_PID  = '/var/run/gpm.pid';
GPM_NODE_DEV  = '/dev/gpmctl';
GPM_NODE_CTL  = GPM_NODE_DEV;
GPM_NODE_FIFO = '/dev/gpmdata';
\end{verbatim}
The following constants denote the buttons on the mouse:
\begin{verbatim}
GPM_B_LEFT   = 4;
GPM_B_MIDDLE = 2;
GPM_B_RIGHT  = 1;
\end{verbatim}
The following constants define events:
\begin{verbatim}
GPM_MOVE = 1;      
GPM_DRAG = 2;
GPM_DOWN = 4;
GPM_UP = 8;
GPM_SINGLE = 16;
GPM_DOUBLE = 32;
GPM_TRIPLE = 64;
GPM_MFLAG = 128;
GPM_HARD = 256;
GPM_ENTER = 512;
GPM_LEAVE = 1024;
\end{verbatim}
The following constants are used in defining margins:
\begin{verbatim}
GPM_TOP = 1;
GPM_BOT = 2;
GPM_LFT = 4;
GPM_RGT = 8;
\end{verbatim}

% Types
\subsection{Types}
The following general types are defined:
\begin{verbatim}
  TGpmEtype = longint;
  TGpmMargin = longint;
\end{verbatim}
The following type describes an event; it is passed in many of the gpm
functions.
\begin{verbatim}
PGpmEvent = ^TGpmEvent;
TGpmEvent = record
  buttons : byte;
  modifiers : byte;
  vc : word;
  dx : integer;
  dy : integer;
  x : integer;
  y : integer;
  wdx : integer;
  wdy : integer;
  EventType : TGpmEType;
  clicks : longint;
  margin : TGpmMargin;
end;
TGpmHandler=function(var event:TGpmEvent;clientdata:pointer):longint;cdecl;
\end{verbatim}
The following types are used in connecting to the \file{gpm} server:
\begin{verbatim}
PGpmConnect = ^TGpmConnect;
TGpmConnect = record
  eventMask : word;
  defaultMask : word;
  minMod : word;
  maxMod : word;
  pid : longint;
  vc : longint;
end;
\end{verbatim}
The following type is used to define {\em regions of interest}
\begin{verbatim}
PGpmRoi = ^TGpmRoi;
TGpmRoi = record
  xMin : integer;
  xMax : integer;
  yMin : integer;
  yMax : integer;
  minMod : word;
  maxMod : word;
  eventMask : word;
  owned : word;
  handler : TGpmHandler;
  clientdata : pointer;
  prev : PGpmRoi;
  next : PGpmRoi;
end;
\end{verbatim}

% Variables
\subsection{Variables}
The following variables are imported from the \var{gpm} library
\begin{verbatim}
gpm_flag           : longint;cvar;external;
gpm_fd             : longint;cvar;external;
gpm_hflag          : longint;cvar;external;
gpm_morekeys       : Longbool;cvar;external;
gpm_zerobased      : Longbool;cvar;external;
gpm_visiblepointer : Longbool;cvar;external;
gpm_mx             : longint;cvar;external;
gpm_my             : longint;cvar;external;
gpm_timeout        : TTimeVal;cvar;external;
_gpm_buf           : array[0..0] of char;cvar;external;
_gpm_arg           : ^word;cvar;external;
gpm_handler        : TGpmHandler;cvar;external;
gpm_data           : pointer;cvar;external;
gpm_roi_handler    : TGpmHandler;cvar;external;
gpm_roi_data       : pointer;cvar;external;
gpm_roi            : PGpmRoi;cvar;external;
gpm_current_roi    : PGpmRoi;cvar;external;
gpm_consolefd      : longint;cvar;external;
Gpm_HandleRoi      : TGpmHandler;cvar;external;
\end{verbatim}

%%%%%%%%%%%%%%%%%%%%%%%%%%%%%%%%%%%%%%%%%%%%%%%%%%%%%%%%%%%%%%%%%%%%%%%
% Functions and procedures
\section{Functions and procedures}

\begin{functionl}{Gpm\_AnyDouble}{GpmAnyDouble}
\Declaration
function Gpm\_AnyDouble(EventType : longint) : boolean;
\Description
\var{Gpm\_AnyDouble} returns \var{True} if \var{EventType} contains
the \var{GPM\_DOUBLE} flag, \var{False} otherwise.
\Errors
None.
\SeeAlso
\seefl{Gpm\_StrictSingle}{GpmStrictSingle},
\seefl{Gpm\_AnySingle}{GpmAnySingle},
\seefl{Gpm\_StrictDouble}{GpmStrictDouble},
\seefl{Gpm\_StrictTriple{GpmStrictTriple},
\seefl{Gpm\_AnyTriple}{GpmAnyTriple}
\end{functionl}

\begin{functionl}{Gpm\_AnySingle}{GpmAnySingle}
\Declaration
function Gpm\_AnySingle(EventType : longint) : boolean;
\Description
\var{Gpm\_AnySingle} returns \var{True} if \var{EventType} contains
the \var{GPM\_SINGLE} flag, \var{False} otherwise. 
\Errors
\SeeAlso
\seefl{Gpm\_StrictSingle}{GpmStrictSingle},
\seefl{Gpm\_AnyDoubmle}{GpmAnyDouble},
\seefl{Gpm\_StrictDouble}{GpmStrictDouble},
\seefl{Gpm\_StrictTriple{GpmStrictTriple},
\seefl{Gpm\_AnyTriple}{GpmAnyTriple}
\end{functionl}

\begin{functionl}{Gpm_AnyTriple}{GpmAnyTriple}
\Declaration
function Gpm\_AnyTriple(EventType : longint) : boolean;
\Description
\Errors
\SeeAlso
\seefl{Gpm\_StrictSingle}{GpmStrictSingle},
\seefl{Gpm\_AnyDoubmle}{GpmAnyDouble},
\seefl{Gpm\_StrictDouble}{GpmStrictDouble},
\seefl{Gpm\_StrictTriple{GpmStrictTriple},
\seefl{Gpm\_AnySingle}{GpmAnySingle}
\end{functionl}

\begin{functionl}{Gpm_Close}{GpmClose}
\Declaration
function Gpm\_Close:longint;cdecl;external;
\Description
\var{Gpm\_Close} closes the current connection, and pops the connection
stack; this means that the previous connection becomes active again.

The function returns -1 if the current connection is not the last one,
and it returns 0 if the current connection is the last one.
\Errors
None.
\SeeAlso
\seefl{Gpm\_Open}{GpmOpen}
\end{functionl}

\begin{functionl}{Gpm\_FitValues}{GpmFitValues}
\Declaration
function Gpm\_FitValues(var x,y:longint):longint;cdecl;external;
\Description
\var{Gpm_fitValues} changes \var{x} and \var{y} so they fit in the visible
screen. The actual mouse pointer is not affected by this function.
\Errors
None.
\SeeAlso
\seefl{Gpm\_FitValuesM}{Gpm\_FitValuesM},
\end{functionl}

\begin{functionl}{Gpm\_FitValuesM}{GpmFitValuesM}
\Declaration
function Gpm\_FitValuesM(var x,y:longint; margin:longint):longint;cdecl;external;
\Description
\var{Gpm\_FitValuesM} chnages \var{x} and \var{y} so they fit in the margin
indicated by \var{margin}. If \var{margin} is -1, then the values are fitted
to the screen. The actual mouse pointer is not affected by this function.
\Errors
None.
\SeeAlso
\seefl{Gpm\_FitValues}{GpmFitValues},
\end{functionl}

\begin{functionl}{Gpm_GetEvent}{GpmGetEvent}
\Declaration
function Gpm\_GetEvent(var Event:TGpmEvent):longint;cdecl;external;
\Description
\var{Gpm\_GetEvent} Reads an event from the file descriptor \var{gpm_fd}.
This file is only for internal use and should never be called by a client
application. 

It returns 1 on succes, and -1 on failue.
\Errors
On error, -1 is returned. 
\SeeAlso
seefl{Gpm\_GetSnapshot}{GpmGetSnapshot}
\end{functionl}

\begin{functionl}{Gpm\_GetLibVersion}{GpmGetLibVersion}
\Declaration
function Gpm\_GetLibVersion(var where:longint):pchar;cdecl;external;
\Description
\var{Gpm\_GetLibVersion} returns a pointer to a version string, and returns
in \var{where} an integer representing the version. The version string
represents the version of the gpm library.

The return value is a pchar, which should not be dealloacted, i.e. it is not
on the heap.
\Errors
None.
\SeeAlso
\seefl{Gpm\_GetServerVersion}{GpmGetServerVersion}
\end{functionl}

\begin{functionl}{Gpm\_GetServerVersion}{GpmGetServerVersion}
\Declaration
function Gpm\_GetServerVersion(var where:longint):pchar;cdecl;external;
\Description
\var{Gpm\_GetServerVersion} returns a pointer to a version string, and 
returns in \var{where} an integer representing the version. The version string
represents the version of the gpm server program.

The return value is a pchar, which should not be dealloacted, i.e. it is not
on the heap.
\Errors
If the gpm program is not present, then the function returns \var{Nil}
\SeeAlso
\seefl{Gpm\_GetLibVersion}{GpmGetLibVersion}
\end{functionl}

\begin{functionl}{Gpm\_GetSnapshot}{GpmGetSnapshot}
\Declaration
function Gpm\_GetSnapshot(var Event:TGpmEvent):longint;cdecl;external;
\Description
\var{Gpm\_GetSnapshot} returns the picture that the server has of the 
current situation in \var{Event}. 
This call will not read the current situation from the mouse file
descriptor, but returns a buffered version.
The meaning of the fields is as follows:
\begin{description}
\item[x,y] current position of the cursor.
\item[dx,dy] size of the window.
\item[vc] number of te virtual console.
\item[modifiers] keyboard shift state.
\item[buttons] buttons which are currently pressed.
\item[clicks] number of clicks (0,1 or 2).
\end{description}
The function returns the number of mouse buttons, or -1 if this information
is not available.
\Errors
None.
\SeeAlso
\seefl{Gpm\_GetEvent}{GpmGetEvent}
\end{functionl}

\begin{functionl}{Gpm\_LowerRoi}{GpmLowerRoi}
\Declaration
function Gpm\_LowerRoi(which:PGpmRoi; after:PGpmRoi):PGpmRoi;cdecl;external;
\Description
\var{Gpm\_LowerRoi} lowers the region of interest \var{which} after
\var{after}. If \var{after} is \var{Nil}, the region of interest is moved to
the bottom of the stack.

The return value is the new top of the region-of-interest stack.
\Errors
None.
\SeeAlso
\seefl{Gpm\_RaiseRoi}{GpmRaiseRoi},
\seefl{Gpm\_PopRoi}{GpmPopRoi},
\seefl{Gpm\_PushRoi}{GpmPopRoi} 
\end{functionl}

\begin{functionl}{Gpm\_Open}{GpmOpen}
\Declaration
function Gpm\_Open(var Conn:TGpmConnect; Flag:longint):longint;cdecl;external;
\Description
\var{Gpm\_Open} opens a new connection to the mouse server. The connection
is described by the fields of the \var{conn} record:
\begin{description}
\item[EventMask] A bitmask of the events the program wants to receive.
\item[DefaultMask] A bitmask to tell the library which events get their
default treatment (text selection).
\item[minMod] the minimum amount of modifiers needed by the program.
\item[maxMod] the maximum amount of modifiers needed by the program.
\end{description}
if \var{Flag} is 0, then the application only receives events that come from
its own terminal device. If it is negative it will receive all events. If
the value is positive then it is considered a console number to which to
connect.

The return value is -1 on error, or the file descriptor used to communicate
with the client. Under an X-Term the return value is -2.
\Errors
On Error, the return value is -1.
\SeeAlso
\seefl{Gpm\_Open}{GpmOpen}
\end{functionl}

\begin{functionl}{Gpm\_PopRoi}{GpmPopRoi}
\Declaration
function Gpm\_PopRoi(which:PGpmRoi):PGpmRoi;cdecl;external;
\Description
\var{Gpm\_PopRoi} pops the topmost region of interest from the stack.
It returns the next element on the stack, or \var{Nil} if the current 
element was the last one.
\Errors
None.
\SeeAlso
\seefl{Gpm\_RaiseRoi}{GpmRaiseRoi},
\seefl{Gpm\_LowerRoi}{GpmLowerRoi}, 
\seefl{Gpm\_PushRoi}{GpmPopRoi} 
\end{functionl}

\begin{functionl}{Gpm\_PushRoi}{GpmPushRoi}
\Declaration
function Gpm\_PushRoi(x1:longint; y1:longint; X2:longint; Y2:longint; mask:longint; fun:TGpmHandler; xtradata:pointer):PGpmRoi;cdecl;external;
\Description
\var{Gpm\_PushRoi} puts a new {\em region of interest} on the stack.
The region of interest is defined by a rectangle described by the corners
\var{(X1,Y1)} and \var{(X2,Y2)}. 

The \var{mask} describes which events the handler {fun} will handle;
\var{ExtraData} will be put in the \var{xtradata} field of the {TGPM\_Roi} 
record passed to the \var{fun} handler.
\Errors
None.
\SeeAlso
\seefl{Gpm\_RaiseRoi}{GpmRaiseRoi},
\seefl{Gpm\_PopRoi}{GpmPopRoi}, 
\seefl{Gpm\_LowerRoi}{GpmLowerRoi} 
\end{functionl}

\begin{functionl}{Gpm\_RaiseRoi}{GpmRaiseRoi}
\Declaration
function Gpm\_RaiseRoi(which:PGpmRoi; before:PGpmRoi):PGpmRoi;cdecl;external;
\Description
\var{Gpm\_RaiseRoi} raises the {\em region of interest} \var{which}
\Errors
\SeeAlso
\end{functionl}

\begin{functionl}
\Declaration
function Gpm_Repeat(millisec:longint):longint;cdecl;external;
\Description
\Errors
\SeeAlso
\end{functionl}

\begin{functionl}
\Declaration
function Gpm_StrictDouble(EventType : longint) : boolean;
\Description
\Errors
\SeeAlso
\end{functionl}

\begin{functionl}
\Declaration
function Gpm_StrictSingle(EventType : longint) : boolean;
\Description
\Errors
\SeeAlso
\end{functionl}

\begin{functionl}
\Declaration
function Gpm_StrictTriple(EventType : longint) : boolean;
\Description
\Errors
\SeeAlso
\end{functionl}

% the go32 unit
\chapter{The GO32 unit}
This chapter of the documentation describe the GO32 unit for the Free Pascal
compiler under \dos. It was donated by Thomas Schatzl
(tom\_at\_work@geocities.com), for which my thanks.
This unit was first written for \dos by Florian Klaempfl.
This chapter is divided in three sections. The first section is an
introduction to the GO32 unit. The second section lists the pre-defined
constants, types and variables. The third section describes the functions
which appear in the interface part of the GO32 unit.
\section{Introduction}
These docs contain information about the GO32 unit. Only the GO32V2 DPMI
mode is discussed by me here due to the fact that new applications shouldn't
be created with the older GO32V1 model. The former is much more advanced and
better. Additionally a lot of functions only work in DPMI mode anyway.
I hope the following explanations and introductions aren't too confusing at
all. If you notice an error or bug send it to the FPC mailing list or
directly to me.
So let's get started and happy and error free coding I wish you....
\hfill Thomas Schatzl, 25. August 1998
\section{Protected mode memory organization}
\subsection{What is DPMI}
The \dos Protected Mode Interface helps you with various aspects of protected
mode programming. These are roughly divided into descriptor handling, access
to \dos memory, management of interrupts and exceptions, calls to real mode
functions and other stuff. Additionally it automatically provides swapping
to disk for memory intensive applications.
A DPMI host (either a Windows \dos box or CWSDPMI.EXE) provides these
functions for your programs.
\subsection{Selectors and descriptors}
Descriptors are a bit like real mode segments; they describe (as the name
implies) a memory area in protected mode. A descriptor contains information
about segment length, its base address and the attributes of it (i.e. type,
access rights, ...).
These descriptors are stored internally in a so-called descriptor table,
which is basically an array of such descriptors.
Selectors are roughly an index into this table.
Because these 'segments' can be up to 4 GB in size, 32 bits aren't
sufficient anymore to describe a single memory location like in real mode.
48 bits are now needed to do this, a 32 bit address and a 16 bit sized
selector. The GO32 unit provides the tseginfo record to store such a
pointer.
But due to the fact that most of the time data is stored and accessed in the
\%ds selector, FPC assumes that all pointers point to a memory location of
this selector. So a single pointer is still only 32 bits in size. This value
represents the offset from the data segment base address to this memory
location.
\subsection{FPC specialities}
The \%ds and \%es selector MUST always contain the same value or some system
routines may crash when called. The \%fs selector is preloaded with the
DOSMEMSELECTOR variable at startup, and it MUST be restored after use,
because again FPC relys on this for some functions. Luckily we asm
programmers can still use the \%gs selector for our own purposes, but for how
long ?
See also:
% tseginfo, dosmemselector, \dos memory access,
 \seefl{get\_cs}{getcs}, 
 \seefl{get\_ds}{getds},
 \seefl{gett\_ss}{getss}, 
 \seefl{allocate\_ldt\_descriptors}{allocateldtdescriptors},
 \seefl{free\_ldt\_descriptor}{freeldtdescriptor},
 \seefl{segment\_to\_descriptor}{segmenttodescriptor},
 \seefl{get\_next\_selector\_increment\_value}{getnextselectorincrementvalue},
 \seefl{get\_segment\_base\_address}{getsegmentbaseaddress},
 \seefl{set\_segment\_base\_address}{setsegmentbaseaddress},
 \seefl{set\_segment\_limit}{setsegmentlimit},
 \seefl{create\_code\_segment\_alias\_descriptor}{createcodesegmentaliasdescriptor} 
\subsection{\dos memory access}
\dos memory is accessed by the predefined \var{dosmemselector} selector; 
the GO32 unit additionally provides some functions to help you with standard tasks,
like copying memory from heap to \dos memory and the likes. Because of this
it is strongly recommened to use them, but you are still free to use the
provided standard memory accessing functions which use 48 bit pointers. The
third, but only thought for compatibility purposes, is using the
\var{mem[]}-arrays. These arrays map the whole 1 Mb \dos space. They shouldn't be
used within new programs.
To convert a segment:offset real mode address to a protected mode linear
address you have to multiply the segment by 16 and add its offset. This
linear address can be used in combination with the DOSMEMSELECTOR variable.
See also: 
\seep{dosmemget},
\seepl{dosmemput}{dosmemput},
\seepl{dosmemmove}{dosmemmove},
\seepl{dosmemfillchar}{dosmemfillchar},
\seepl{dosmemfillword}{dosmemfillword},
mem[]-arrays, 
\seepl{seg\_move}{segmove},
\seepl{seg\_fillchar}{segfillchar},
\seepl{seg\_fillword}{segfillword}. 
\subsection{I/O port access}
The I/O port access is done via the various \seef{inportb}, \seep{outportb}
functions
which are available. Additionally Free Pascal supports the Turbo Pascal
PORT[]-arrays but it is by no means recommened to use them, because they're
only for compatibility purposes.
See also: \seep{outportb}, \seef{inportb}, PORT[]-arrays
\subsection{Processor access}
These are some functions to access various segment registers (\%cs, \%ds, \%ss)
which makes your work a bit easier.
See also: \seefl{get\_cs}{getcs}, \seefl{get\_ds}{getds}, 
\seefl{get\_ss}{getss} 
\subsection{Interrupt redirection}
Interrupts are program interruption requests, which in one or another way
get to the processor; there's a distinction between software and hardware
interrupts. The former are explicitely called by an 'int' instruction and
are a bit comparable to normal functions. Hardware interrupts come from
external devices like the keyboard or mouse. These functions are called
handlers.
\subsection{Handling interrupts with DPMI}
The interrupt functions are real-mode procedures; they normally can't be
called in protected mode without the risk of an protection fault. So the
DPMI host creates an interrupt descriptor table for the application.
Initially all software interrupts (except for int 31h, 2Fh and 21h function
4Ch) or external hardware interrupts are simply directed to a handler that
reflects the interrupt in real-mode, i.e. the DPMI host's default handlers
switch the CPU to real-mode, issue the interrupt and switch back to
protected mode. The contents of general registers and flags are passed to
the real mode handler and the modified registers and flags are returned to
the protected mode handler. Segment registers and stack pointer are not
passed between modes.
\subsection{Protected mode interrupts vs. Real mode interrupts}
As mentioned before, there's a distinction between real mode interrupts and
protected mode interrupts; the latter are protected mode programs, while the
former must be real mode programs. To call a protected mode interrupt
handler, an assembly 'int' call must be issued, while the other is called
via the realintr() or intr() function. Consequently, a real mode interrupt
then must either reside in \dos memory (<1MB) or the application must
allocate a real mode callback address via the get\_rm\_callback() function.
\subsection{Creating own interrupt handlers}
Interrupt redirection with FPC pascal is done via the set\_pm\_interrupt() for
protected mode interrupts or via the set\_rm\_interrupt() for real mode
interrupts.
\subsection{Disabling interrupts}
The GO32 unit provides the two procedures disable() and enable() to disable
and enable all interrupts.
\subsection{Hardware interrupts}
Hardware interrupts are generated by hardware devices when something unusual
happens; this could be a keypress or a mouse move or any other action. This
is done to minimize CPU time, else the CPU would have to check all installed
hardware for data in a big loop (this method is called 'polling') and this
would take much time.
A standard IBM-PC has two interrupt controllers, that are responsible for
these hardware interrupts: both allow up to 8 different interrupt sources
(IRQs, interrupt requests). The second controller is connected to the first
through IRQ 2 for compatibility reasons, e.g. if controller 1 gets an IRQ 2,
he hands the IRQ over to controller 2. Because of this up to 15 different
hardware interrupt sources can be handled.
IRQ 0 through IRQ 7 are mapped to interrupts 8h to Fh and the second
controller (IRQ 8 to 15) is mapped to interrupt 70h to 77h.
All of the code and data touched by these handlers MUST be locked (via the
various locking functions) to avoid page faults at interrupt time. Because
hardware interrupts are called (as in real mode) with interrupts disabled,
the handler has to enable them before it returns to normal program
execution. Additionally a hardware interrupt must send an EOI (end of
interrupt) command to the responsible controller; this is acomplished by
sending the value 20h to port 20h (for the first controller) or A0h (for the
second controller).
The following example shows how to redirect the keyboard interrupt.
\latex{\inputlisting{go32ex/keyclick.pp}}
\html{\begin{FPCList}
\item[Example]
\begin{verbatim}
Program Keyclick;

uses crt, 
     go32;

const kbdint = $9; 

var oldint9_handler : tseginfo;
    newint9_handler : tseginfo;

    clickproc : pointer; 

{$ASMMODE DIRECT}
procedure int9_handler; assembler;
asm
   cli
   pushal
   movw %cs:INT9_DS, %ax
   movw %ax, %ds
   movw %ax, %es
   movw U_GO32_DOSMEMSELECTOR, %ax
   movw %ax, %fs
   call *_CLICKPROC
   popal

   ljmp %cs:OLDHANDLER 

INT9_DS: .word 0
OLDHANDLER:
         .long 0
         .word 0
end;

procedure int9_dummy; begin end;

procedure clicker;
begin
     sound(500); delay(10); nosound;
end;

procedure clicker_dummy; begin end;

procedure install_click;
begin
     clickproc := @clicker;
     lock_data(clickproc, sizeof(clickproc));
     lock_data(dosmemselector, sizeof(dosmemselector));

     lock_code(@clicker, 
               longint(@clicker_dummy)-longint(@clicker));
     lock_code(@int9_handler, 
               longint(@int9_dummy)
                - longint(@int9_handler));
     newint9_handler.offset := @int9_handler;
     newint9_handler.segment := get_cs;
     get_pm_interrupt(kbdint, oldint9_handler);
     asm
        movw %ds, %ax
        movw %ax, INT9_DS
        movl _OLDINT9_HANDLER, %eax
        movl %eax, OLDHANDLER
        movw 4+_OLDINT9_HANDLER, %ax
        movw %ax, 4+OLDHANDLER
     end;
     set_pm_interrupt(kbdint, newint9_handler);
end;

procedure remove_click;
begin
     set_pm_interrupt(kbdint, oldint9_handler);
     unlock_data(dosmemselector, sizeof(dosmemselector));
     unlock_data(clickproc, sizeof(clickproc));
     unlock_code(@clicker, 
                 longint(@clicker_dummy)
                  - longint(@clicker));
     unlock_code(@int9_handler, 
                 longint(@int9_dummy)
                  - longint(@int9_handler));
end;

var ch : char;

begin
     install_click;
     Writeln('Enter any message.',
             ' Press return when finished');
     while (ch <> #13) do begin
           ch := readkey; write(ch);
     end;
     remove_click;
end.
\end{verbatim}
\end{FPCList}}
\subsection{Software interrupts}
Ordinarily, a handler installed with
\seefl{set\_pm\_interrupt}{setpminterrupt} only services software
interrupts that are executed in protected mode; real mode software
interrupts can be redirected by \seefl{set\_rm\_interrupt}{setrminterrupt}.
See also \seefl{set\_rm\_interrupt}{setrminterrupt}, 
\seefl{get\_rm\_interrupt}{getrminterrupt},
\seefl{set\_pm\_interrupt}{setpminterrupt},
\seefl{get\_pm\_interrupt}{getpminterrupt}, 
\seefl{lock\_data}{lockdata}, 
\seefl{lock\_code}{lockcode}, 
\seep{enable}, 
\seep{disable},
\seepl{outportb}{outportb} 
Executing software interrupts
Simply execute a realintr() call with the desired interrupt number and the
supplied register data structure.
But some of these interrupts require you to supply them a pointer to a
buffer where they can store data to or obtain data from in memory. These
interrupts are real mode functions and so they only can access the first Mb
of linear address space, not FPC's data segment.
For this reason FPC supplies a pre-initialized \dos memory location within
the GO32 unit. This buffer is internally used for \dos functions too and so
it's contents may change when calling other procedures. It's size can be
obtained with \seefl{tb\_size}{tbsize} and it's linear address via 
\seefl{transfer\_buffer}{transferbuffer}.
Another way is to allocate a completely new \dos memory area via the
\seefl{global\_dos\_alloc}{globaldosalloc} function for your use and 
supply its real mode address.
See also:
\seefl{tb\_size}{tbsize},
\seefl{transfer\_buffer}{transferbuffer}.
\seefl{global\_dos\_alloc}{globaldosalloc},
\seefl{global\_dos\_free}{globaldosfree},
\seef{realintr}
The following examples illustrate the use of software interrupts.
\latex{\inputlisting{go32ex/softint.pp}}
\html{\begin{FPCList}
\item[Example]
\begin{verbatim}
Program softint;

uses go32;

var r : trealregs;

begin
     r.al := $01;
     realintr($21, r);
     Writeln('DOS v', r.al,'.',r.ah, ' detected');
end.\end{verbatim}
\end{FPCList}}
\latex{\inputlisting{go32ex/rmpm_int.pp}}
\html{\begin{FPCList}
\item[Example]
\begin{verbatim}
Program rmpm_int;

uses crt, go32;

{$ASMMODE DIRECT}

var r : trealregs;
    axreg : Word; 

    oldint21h : tseginfo;
    newint21h : tseginfo;

procedure int21h_handler; assembler;
asm
   cmpw $0x3001, %ax
   jne CallOld
   movw $0x3112, %ax
   iret

CallOld:
   ljmp %cs:OLDHANDLER

OLDHANDLER: .long 0
            .word 0
end;

procedure resume;
begin
     Writeln;
     Write('-- press any key to resume --'); readkey;
     gotoxy(1, wherey); clreol;
end;

begin
     clrscr;
     Writeln('Executing real mode interrupt');
     resume;
     r.ah := $30; r.al := $01;  realintr($21, r);
     Writeln('DOS v', r.al,'.',r.ah, ' detected');
     resume;
     Writeln('Executing protected mode interrupt',
             ' without our own handler');
     Writeln;
     asm
        movb $0x30, %ah
        movb $0x01, %al
        int $0x21
        movw %ax, _AXREG
     end;
     Writeln('DOS v', r.al,'.',r.ah, ' detected');
     resume;
     Writeln('As you can see the DPMI hosts',
             ' default protected mode handler');
     Writeln('simply redirects it to the real mode handler');
     resume;
     Writeln('Now exchanging the protected mode',
             'interrupt with our own handler');
     resume;

     newint21h.offset := @int21h_handler;
     newint21h.segment := get_cs;
     get_pm_interrupt($21, oldint21h);
     asm
        movl _OLDINT21H, %eax
        movl %eax, OLDHANDLER
        movw 4+_OLDINT21H, %ax
        movw %ax, 4+OLDHANDLER
     end;
     set_pm_interrupt($21, newint21h);

     Writeln('Executing real mode interrupt again');
     resume;
     r.ah := $30; r.al := $01; realintr($21, r);
     Writeln('DOS v', r.al,'.',r.ah, ' detected');
     Writeln;
     Writeln('See, it didn''t change in any way.');
     resume;
     Writeln('Now calling protected mode interrupt');
     resume;
     asm
        movb $0x30, %ah
        movb $0x01, %al
        int $0x21
        movw %ax, _AXREG
     end;
     Writeln('DOS v', lo(axreg),'.',hi(axreg), ' detected');
     Writeln;
     Writeln('Now you can see that there''s',
             ' a distinction between the two ways of ');
     Writeln('calling interrupts...');
     set_pm_interrupt($21, oldint21h);
end.\end{verbatim}
\end{FPCList}}
\subsection{Real mode callbacks}
The callback mechanism can be thought of as the converse of calling a real
mode procedure (i.e. interrupt), which allows your program to pass
information to a real mode program, or obtain services from it in a manner
that's transparent to the real mode program.
In order to make a real mode callback available, you must first get the real
mode callback address of your procedure and the selector and offset of a
register data structure. This real mode callback address (this is a
segment:offset address) can be passed to a real mode program via a software
interrupt, a \dos memory block or any other convenient mechanism.
When the real mode program calls the callback (via a far call), the DPMI
host saves the registers contents in the supplied register data structure,
switches into protected mode, and enters the callback routine with the
following conditions:
\begin{itemize}
\item interrupts disabled
\item \var{\%CS:\%EIP} = 48 bit pointer specified in the original call to 
\seefl{get\_rm\_callback}{getrmcallback}
\item \var{\%DS:\%ESI} = 48 bit pointer to to real mode \var{SS:SP}
\item \var{\%ES:\%EDI} = 48 bit pointer of real mode register data
structure. 
\item \var{\%SS:\%ESP} = locked protected mode stack
\item  All other registers undefined
\end{itemize}
The callback procedure can then extract its parameters from the real mode
register data structure and/or copy parameters from the real mode stack to
the protected mode stack. Recall that the segment register fields of the
real mode register data structure contain segment or paragraph addresses
that are not valid in protected mode. Far pointers passed in the real mode
register data structure must be translated to virtual addresses before they
can be used with a protected mode program.
The callback procedure exits by executing an IRET with the address of the
real mode register data structure in \var{\%ES:\%EDI}, passing information back to
the real mode caller by modifying the contents of the real mode register
data structure and/or manipulating the contents of the real mode stack. The
callback procedure is responsible for setting the proper address for
resumption of real mode execution into the real mode register data
structure; typically, this is accomplished by extracting the return address
from the real mode stack and placing it into the \var{\%CS:\%EIP} fields of the real
mode register data structure. After the IRET, the DPMI host switches the CPU
back into real mode, loads ALL registers with the contents of the real mode
register data structure, and finally returns control to the real mode
program.
All variables and code touched by the callback procedure MUST be locked to
prevent page faults.
See also: \seefl{get\_rm\_callback}{getrmcallback},
\seefl{free\_rm\_callback}{freermcallback}, 
\seefl{lock\_code}{lockcode}, 
\seefl{lock\_data}{lockdata} 
\section{Types, Variables and Constants}
\subsection{Constants}
\subsubsection{Constants returned by get\_run\_mode}
Tells you under what memory environment (e.g. memory manager) the program
currently runs.
\begin{verbatim}
rm_unknown = 0; { unknown }
rm_raw     = 1; { raw (without HIMEM) } 
rm_xms     = 2; { XMS (for example with HIMEM, without EMM386) } 
rm_vcpi    = 3; { VCPI (for example HIMEM and EMM386) } 
rm_dpmi    = 4; { DPMI (for example \dos box or 386Max) }
\end{verbatim}
Note: GO32V2 {\em always} creates DPMI programs, so you need a suitable DPMI
host like \file{CWSDPMI.EXE} or a Windows \dos box. So you don't need to check it,
these constants are only useful in GO32V1 mode.
\subsubsection{Processor flags constants}
They are provided for a simple check with the flags identifier in the
trealregs type. To check a single flag, simply do an AND operation with the
flag you want to check. It's set if the result is the same as the flag
value.
\begin{verbatim}
const carryflag = $001; 
parityflag      = $004; 
auxcarryflag    = $010; 
zeroflag        = $040; 
signflag        = $080; 
trapflag        = $100; 
interruptflag   = $200;
directionflag   = $400; 
overflowflag    = $800;
\end{verbatim}
\subsection{Predefined types}
\begin{verbatim}
type tmeminfo = record
            available_memory : Longint; 
            available_pages : Longint;
            available_lockable_pages : Longint; 
            linear_space : Longint;
            unlocked_pages : Longint; 
            available_physical_pages : Longint;
            total_physical_pages : Longint; 
            free_linear_space : Longint;
            max_pages_in_paging_file : Longint; 
            reserved : array[0..2] of Longint;
   end;
\end{verbatim}
Holds information about the memory allocation, etc.
\begin{tabular}{ll}
Record entry & Description \\ \hline
\var{available\_memory} & Largest available free block in bytes. \\
\var{available\_pages} & Maximum unlocked page allocation in pages \\
\var{available\_lockable\_pages} &  Maximum locked page allocation in pages. \\
\var{linear\_space} &  Linear address space size in pages. \\
\var{unlocked\_pages} & Total number of unlocked pages. \\
\var{available\_physical\_pages} &  Total number of free pages.\\
\var{total\_physical\_pages} &  Total number of physical pages. \\
\var{free\_linear\_space} & Free linear address space in pages.\\
\var{max\_pages\_in\_paging\_file} &  Size of paging file/partition in
pages. \\
\end{tabular}
NOTE: The value of a field is -1 (0ffffffffh) if the value is unknown, it's
only guaranteed, that \var{available\_memory} contains a valid value.
The size of the pages can be determined by the get\_page\_size() function.
\begin{verbatim}
type 
trealregs = record
  case Integer of 
    1: { 32-bit } 
      (EDI, ESI, EBP, Res, EBX, EDX, ECX, EAX: Longint; 
       Flags, ES, DS, FS, GS, IP, CS, SP, SS: Word); 
    2: { 16-bit } 
      (DI, DI2, SI, SI2, BP, BP2, R1, R2: Word;
       BX, BX2, DX, DX2, CX, CX2, AX, AX2: Word);
    3: { 8-bit } 
      (stuff: array[1..4] of Longint;
       BL, BH, BL2, BH2, DL, DH, DL2, DH2, CL,
       CH, CL2, CH2, AL, AH, AL2, AH2: Byte);
    4: { Compat } 
      (RealEDI, RealESI, RealEBP, RealRES, RealEBX, 
       RealEDX, RealECX, RealEAX: Longint; 
       RealFlags, RealES, RealDS, RealFS, RealGS, 
       RealIP, RealCS, RealSP, RealSS: Word);
    end;
    registers = trealregs;
\end{verbatim}
These two types contain the data structure to pass register values to a
interrupt handler or real mode callback.
\begin{verbatim}
type tseginfo = record
             offset : Pointer; segment : Word; end;
\end{verbatim}
This record is used to store a full 48-bit pointer. This may be either a
protected mode selector:offset address or in real mode a segment:offset
address, depending on application.
See also: Selectors and descriptors, \dos memory access, Interrupt
redirection
\subsection{Variables.}
\begin{verbatim}
var dosmemselector : Word;
\end{verbatim}
Selector to the \dos memory. The whole \dos memory is automatically mapped to
this single descriptor at startup. This selector is the recommened way to
access \dos memory.
\begin{verbatim}
  var int31error : Word;
\end{verbatim}
This variable holds the result of a DPMI interrupt call. Any nonzero value
must be treated as a critical failure.
\section{Functions and Procedures}
\begin{functionl}{allocate\_ldt\_descriptors}{allocateldtdescriptors}
\Declaration
Function allocate\_ldt\_descriptors (count : Word) : Word;

\Description
Allocates a number of new descriptors.
Parameters: 
\begin{description}
\item[count:\ ] specifies the number of requested unique descriptors.
\end{description}
Return value: The base selector.
Notes: The descriptors allocated must be initialized by the application with
other function calls. This function returns descriptors with a limit and
size value set to zero. If more than one descriptor was requested, the
function returns a base selector referencing the first of a contiguous array
of descriptors. The selector values for subsequent descriptors in the array
can be calculated by adding the value returned by the
\seefl{get\_next\_selector\_increment\_value}{getnextselectorincrementvalue} 
function.

\Errors
 Check the \var{int31error} variable. 
\SeeAlso

\seefl{free\_ldt\_descriptor}{freeldtdescriptor},
\seefl{get\_next\_selector\_increment\_value}{getnextselectorincrementvalue},
\seefl{segment\_to\_descriptor}{segmenttodescriptor},
\seefl{create\_code\_segment\_alias\_descriptor}{createcodesegmentaliasdescriptor},
\seefl{set\_segment\_limit}{setsegmentlimit},
\seefl{set\_segment\_base\_address}{setsegmentbaseaddress} 

\end{functionl}
\latex{\inputlisting{go32ex/sel_des.pp}}
\html{\begin{FPCList}
\item[Example]
\begin{verbatim}
Program sel_des;

uses crt,
     go32;

const maxx = 80;
      maxy = 25;
      bytespercell = 2;
      screensize = maxx * maxy * bytespercell;

      linB8000 = $B800 * 16;

type string80 = string[80];

var
    text_save : array[0..screensize-1] of byte;
    text_oldx, text_oldy : Word;

    text_sel : Word;

procedure status(s : string80);
begin
  gotoxy(1, 1); 
  clreol; 
  write(s); 
  readkey;
end;

procedure selinfo(sel : Word);
begin
gotoxy(1, 24);
clreol; 
writeln('Descriptor base address : $', 
        hexstr(get_segment_base_address(sel), 8));
clreol; 
write('Descriptor limit : ', 
       get_segment_limit(sel));
end;

function makechar(ch : char; color : byte) : Word;
begin
     result := byte(ch) or (color shl 8);
end;

begin
seg_move(dosmemselector, linB8000, 
         get_ds, longint(@text_save), screensize);
text_oldx := wherex; text_oldy := wherey;
seg_fillword(dosmemselector, linB8000, 
             screensize div 2, 
             makechar(' ', Black or (Black shl 4)));
status('Creating selector ' + 
        '''text_sel'' to a part of text screen memory');
text_sel := allocate_ldt_descriptors(1);
set_segment_base_address(text_sel, linB8000 
                          + bytespercell * maxx * 1);
set_segment_limit(text_sel, 
                  screensize-1-bytespercell*maxx*3);
selinfo(text_sel);

status('and clearing entire memory ' + 
       'selected by ''text_sel'' descriptor');
seg_fillword(text_sel, 0, 
             (get_segment_limit(text_sel)+1) div 2, 
             makechar(' ', LightBlue shl 4));

status('Notice that only the memory described'+
       ' by the descriptor changed, nothing else');

status('Now reducing it''s limit and base and '+
       'setting it''s described memory');
set_segment_base_address(text_sel, 
     get_segment_base_address(text_sel) 
     + bytespercell * maxx);
set_segment_limit(text_sel, 
     get_segment_limit(text_sel) 
     - bytespercell * maxx * 2);
selinfo(text_sel);
status('Notice that the base addr increased by '+
       'one line but the limit decreased by 2 lines');
status('This should give you the hint that the '+
       'limit is relative to the base');
seg_fillword(text_sel, 0, 
             (get_segment_limit(text_sel)+1) div 2, 
             makechar(#176, LightMagenta or Brown shl 4));

status('Now let''s get crazy and copy 10 lines'+
       ' of data from the previously saved screen');
seg_move(get_ds, longint(@text_save), 
         text_sel, maxx * bytespercell * 2, 
         maxx * bytespercell * 10);

status('At last freeing the descriptor and '+
       'restoring the old screen contents..');
status('I hope this little program may give '+
       'you some hints on working with descriptors');
free_ldt_descriptor(text_sel);
seg_move(get_ds, longint(@text_save), 
         dosmemselector, linB8000, screensize);
gotoxy(text_oldx, text_oldy);
end.\end{verbatim}
\end{FPCList}}
\begin{functionl}{allocate\_memory\_block}{allocatememoryblock}
\Declaration
Function allocate\_memory\_block (size:Longint) : Longint;

\Description
Allocates a block of linear memory.
Parameters: 
\begin{description}
\item[size:\ ] Size of requested linear memory block in bytes.
\end{description}
Returned values: blockhandle - the memory handle to this memory block. Linear
address of the requested memory.
Notes: WARNING: According to my DPMI docs this function is not implemented
correctly. Normally you should also get a blockhandle to this block after
successful operation. This handle is used to free the memory block
afterwards or use this handle for other purposes. So this block can't be
deallocated and is henceforth unusuable !
This function doesn't allocate any descriptors for this block, it's the
applications resposibility to allocate and initialize for accessing this
memory.

\Errors
 Check the \var{int31error} variable.
\SeeAlso
 \seefl{free\_memory\_block}{freememoryblock} 
\end{functionl}
\begin{procedure}{copyfromdos}
\Declaration
Procedure copyfromdos (var addr; len : Longint);

\Description

Copies data from the pre-allocated \dos memory transfer buffer to the heap.
Parameters:
\begin{description}
\item[addr:\ ] data to copy to.
\item[len:\ ] number of bytes to copy to heap.
\end{description}
Notes:
Can only be used in conjunction with the \dos memory transfer buffer.

\Errors
Check the \var{int31error} variable.
\SeeAlso
\seefl{tb\_size}{tbsize}, \seefl{transfer\_buffer}{transferbuffer}, 
\seep{copytodos}
\end{procedure}
\begin{procedure}{copytodos}
\Declaration
Procedure copytodos (var addr; len : Longint);

\Description
Copies data from heap to the pre-allocated \dos memory buffer.
Parameters:
\begin{description}
\item[addr:\ ] data to copy from.
\item[len:\ ] number of bytes to copy to \dos memory buffer.
\end{description}
Notes: This function fails if you try to copy more bytes than the transfer
buffer is in size. It can only be used in conjunction with the transfer
buffer.

\Errors
 Check the \var{int31error} variable.
\SeeAlso
\seefl{tb\_size}{tbsize}, \seefl{transfer\_buffer}{transferbuffer}, 
\seep{copyfromdos}
\end{procedure}
\begin{functionl}{create\_code\_segment\_alias\_descriptor}{createcodesegmentaliasdescriptor}
\Declaration
Function create\_code\_segment\_alias\_descriptor (seg : Word) : Word;

\Description

Creates a new descriptor that has the same base and limit as the specified
descriptor.
Parameters: 
\begin{description}
\item[seg:\ ] selector.
\end{description}
Return values: The data selector (alias).
Notes: In effect, the function returns a copy of the descriptor. The
descriptor alias returned by this function will not track changes to the
original descriptor. In other words, if an alias is created with this
function, and the base or limit of the original segment is then changed, the
two descriptors will no longer map the same memory.

\Errors
Check the \var{int31error} variable.
\SeeAlso
 
\seefl{allocate\_ldt\_descriptors}{allocateldtdescriptors},
\seefl{set\_segment\_limit}{setsegmentlimit}, 
\seefl{set\_segment\_base\_address}{setsegmentbaseaddress} 
\end{functionl}
\begin{procedure}{disable}
\Declaration
Procedure disable ;

\Description
Disables all hardware interrupts by execution a CLI instruction.
Parameters: None.

\Errors
None.
\SeeAlso
\seep{enable}
\end{procedure}
\begin{procedure}{dosmemfillchar}
\Declaration
Procedure dosmemfillchar (seg, ofs : Word; count : Longint; c : char);

\Description
Sets a region of \dos memory to a specific byte value.
Parameters:
\begin{description}
\item[seg:\ ] real mode segment.
\item[ofs:\ ] real mode offset. 
\item[count:\ ] number of bytes to set.
\item[c:\ ] value to set memory to.
\end{description}
Notes: No range check is performed.

\Errors
 None.
\SeeAlso
 
\seep{dosmemput},
\seep{dosmemget}, 
\seep{dosmemmove}{dosmemmove}, 
\seepl{dosmemfillword}{dosmemfillword}, 
\seepl{seg\_move}{segmove},
\seepl{seg\_fillchar}{segfillchar},
\seepl{seg\_fillword}{segfillword} 
\end{procedure}
\latex{\inputlisting{go32ex/textmess.pp}}
\html{\begin{FPCList}
\item[Example]
\begin{verbatim}
Program textmess;

uses crt, go32;

const columns = 80;
      rows = 25; 
      screensize = rows*columns*2;

      text = '! Hello world !'; 

var textofs : Longint;
    save_screen : array[0..screensize-1] of byte;
    curx, cury : Integer;

begin
     randomize;
     dosmemget($B800, 0, save_screen, screensize);
     curx := wherex; cury := wherey;
     gotoxy(1, 1); Write(text);
     textofs := screensize + length(text)*2;
     dosmemmove($B800, 0, $B800, textofs, length(text)*2);
     dosmemfillchar($B800, 0, screensize, #0);
     while (not keypressed) do 
       begin
       dosmemfillchar($B800, 
                      textofs + random(length(text))*2 + 1,
                      1, char(random(255)));
       dosmemmove($B800, textofs, $B800,
                  random(columns)*2+random(rows)*columns*2,
                  length(text)*2);
           delay(1);
     end;
     readkey;
     readkey;
     dosmemput($B800, 0, save_screen, screensize);
     gotoxy(curx, cury);
end.\end{verbatim}
\end{FPCList}}
\begin{procedure}{dosmemfillword}
\Declaration
Procedure dosmemfillword (seg,ofs : Word; count : Longint; w : Word);

\Description
Sets a region of \dos memory to a specific word value.
Parameters: 
\begin{description}
\item[seg:\ ] real mode segment.
\item[ofs:\ ] real mode offset. 
\item[count:\ ] number of words to set.
\item[w:\ ] value to set memory to.
\end{description}
Notes: No range check is performed.

\Errors
 None.
\SeeAlso
 
\seep{dosmemput},
\seepl{dosmemget}{dosmemget}, 
\seepl{dosmemmove}{dosmemmove}, 
\seepl{dosmemfillchar}{dosmemfillchar}, 
\seepl{seg\_move}{segmove}, 
\seepl{seg\_fillchar}{segfillchar},
\seepl{seg\_fillword}{segfillword} 
\end{procedure}
\begin{procedure}{dosmemget}
\Declaration
Procedure dosmemget (seg : Word; ofs : Word; var data; count : Longint);

\Description
Copies data from the \dos memory onto the heap.
Parameters:
\begin{description}
\item[seg:\ ] source real mode segment.
\item[ofs:\ ] source real mode offset.
\item[data:\ ] destination. 
\item[count:\ ] number of bytes to copy.
\end{description}
Notes: No range checking is performed.

\Errors
 None. 
\SeeAlso
\seep{dosmemput},
\seep{dosmemmove},
\seep{dosmemfillchar},
\seep{dosmemfillword},
\seepl{seg\_move}{segmove},
\seepl{seg\_fillchar}{segfillchar}, 
\seepl{seg\_fillword}{segfillword}  
\end{procedure}
For an example, see \seefl{global\_dos\_alloc}{globaldosalloc}.
\begin{procedure}{dosmemmove}
\Declaration
Procedure dosmemmove (sseg, sofs, dseg, dofs : Word; count : Longint);

\Description
Copies count bytes of data between two \dos real mode memory locations.
Parameters:
\begin{description}
\item[sseg:\ ] source real mode segment. 
\item[sofs:\ ] source real mode offset.
\item[dseg:\ ] destination real mode segment. 
\item[dofs:\ ] destination real mode offset.
\item[count:\ ] number of bytes to copy.
\end{description}
Notes: No range check is performed in any way.

\Errors
 None.
\SeeAlso
\seep{dosmemput}, 
\seep{dosmemget},
\seep{dosmemfillchar}, 
\seep{dosmemfillword}
\seepl{seg\_move}{segmove}, 
\seepl{seg\_fillchar}{segfillchar},
\seepl{seg\_fillword}{segfillword} 
\end{procedure}
For an example, see \seepl{seg\_fillchar}{segfillchar}.
\begin{procedure}{dosmemput}
\Declaration
Procedure dosmemput (seg : Word; ofs : Word; var data; count : Longint);

\Description
Copies heap data to \dos real mode memory.
Parameters:
\begin{description}
\item[seg:\ ] destination real mode segment.
\item[ofs:\ ] destination real mode offset. 
\item[data:\ ] source. 
\item[count:\ ] number of bytes to copy.
\end{description}
Notes: No range checking is performed.

\Errors
 None. 
\SeeAlso
\seep{dosmemget}, 
\seep{dosmemmove},
\seep{dosmemfillchar},
\seep{dosmemfillword},
\seepl{seg\_move}{segmove},
\seepl{seg\_fillchar}{segfillchar},
\seepl{seg\_fillword}{segfillword} 
\end{procedure}
For an example, see \seefl{global\_dos\_alloc}{globaldosalloc}.
\begin{procedure}{enable}
\Declaration
Procedure enable ;

\Description

Enables all hardware interrupts by executing a STI instruction.
Parameters: None.

\Errors
None.
\SeeAlso
 \seep{disable} 
\end{procedure}
\begin{functionl}{free\_ldt\_descriptor}{freeldtdescriptor}
\Declaration
Function free\_ldt\_descriptor (des : Word) : boolean;

\Description
Frees a previously allocated descriptor.
Parameters:
\begin{description}
\item[des:\ ] The descriptor to be freed.
\end{description}
Return value: \var{True} if successful, \var{False} otherwise.
Notes: After this call this selector is invalid and must not be used for any
memory operations anymore. Each descriptor allocated with
\seefl{allocate\_ldt\_descriptors}{allocateldtdescriptors} must be freed 
individually with this function,
even if it was previously allocated as a part of a contiguous array of
descriptors.

\Errors
Check the \var{int31error} variable.
\SeeAlso

\seefl{allocate\_ldt\_descriptors}{allocateldtdescriptors},
\seefl{get\_next\_selector\_increment\_value}{getnextselectorincrementvalue} 

\end{functionl}
For an example, see 
\seefl{allocate\_ldt\_descriptors}{allocateldtdescriptors}.
\begin{functionl}{free\_memory\_block}{freememoryblock}
\Declaration
Function free\_memory\_block (blockhandle :
Longint) : boolean;

\Description
Frees a previously allocated memory block.
Parameters: 
\begin{description}
\item{blockhandle:} the handle to the memory area to free.
\end{description}
Return value: \var{True} if successful, \var{false} otherwise.
Notes: Frees memory that was previously allocated with
\seefl{allocate\_memory\_block}{allocatememoryblock} . 
This function doesn't free any descriptors mapped to this block, 
it's the application's responsibility.

\Errors
 Check \var{int31error} variable.
\SeeAlso
\seefl{allocate\_memory\_block}{allocatememoryblock} 
\end{functionl}
\begin{functionl}{free\_rm\_callback}{freermcallback}
\Declaration
Function free\_rm\_callback (var intaddr : tseginfo) : boolean;

\Description

Releases a real mode callback address that was previously allocated with the
\seefl{get\_rm\_callback}{getrmcallback}  function.
Parameters: 
\begin{description}
\item[intaddr:\ ] real mode address buffer returned by 
\seefl{get\_rm\_callback}{getrmcallback} .
\end{description}
Return values: \var{True} if successful, \var{False} if not

\Errors
 Check the \var{int31error} variable.
\SeeAlso

\seefl{set\_rm\_interrupt}{setrminterrupt},
\seefl{get\_rm\_callback}{getrmcallback}

\end{functionl}
For an example, see \seefl{get\_rm\_callback}{getrmcallback}.
\begin{functionl}{get\_cs}{getcs}
\Declaration
Function get\_cs  : Word;

\Description

Returns the cs selector.
Parameters: None.
Return values: The content of the cs segment register.

\Errors
None.
\SeeAlso
 \seefl{get\_ds}{getds}, \seefl{get\_ss}{getss}
\end{functionl}
For an example, see \seefl{set\_pm\_interrupt}{setpminterrupt}.
\begin{functionl}{get\_descriptor\_access\_rights}{getdescriptoraccessrights}
\Declaration
Function get\_descriptor\_access\_rights (d : Word) : Longint;

\Description
Gets the access rights of a descriptor.
Parameters: 
\begin{description}
\item{d} selector to descriptor.
\end{description}
Return value: Access rights bit field.

\Errors
Check the \var{int31error} variable.
\SeeAlso
 
\seefl{set\_descriptor\_access\_rights}{setdescriptoraccessrights}
\end{functionl}
\begin{functionl}{get\_ds}{getds}
\Declaration
Function get\_ds  : Word;

\Description
Returns the ds selector.
Parameters: None.
Return values: The content of the ds segment register.

\Errors
 None.
\SeeAlso
 \seefl{get\_cs}{getcs}, \seefl{get\_ss}{getss}
\end{functionl}
\begin{functionl}{get\_linear\_addr}{getlinearaddr}
\Declaration
Function get\_linear\_addr (phys\_addr : Longint; size : Longint) : Longint;

\Description
Converts a physical address into a linear address.
Parameters: 
\begin{description}
\item [phys\_addr:\ ] physical address of device.
\item [size:\ ] Size of region to map in bytes.
\end{description}
Return value: Linear address that can be used to access the physical memory.
Notes: It's the applications resposibility to allocate and set up a
descriptor for access to the memory. This function shouldn't be used to map
real mode addresses.

\Errors
 Check the \var{int31error} variable.
\SeeAlso
 
\seefl{allocate\_ldt\_descriptors}{allocateldtdescriptors}, \seefl{set\_segment\_limit}{setsegmentlimit},
\seefl{set\_segment\_base\_address}{setsegmentbaseaddress} 
\end{functionl}
\begin{functionl}{get\_meminfo}{getmeminfo}
\Declaration
Function get\_meminfo (var meminfo : tmeminfo) : boolean;

\Description
 Returns information about the amount of available physical memory, linear
address space, and disk space for page swapping.
Parameters:
\begin{description}
\item[meminfo:\ ] buffer to fill memory information into.
\end{description}
Return values: Due to an implementation bug this function always returns
\var{False}, but it always succeeds.
Notes: Only the first field of the returned structure is guaranteed to
contain a valid value. Any fields that are not supported by the DPMI host
will be set by the host to \var{-1 (0FFFFFFFFH)} to indicate that the information
is not available. The size of the pages used by the DPMI host can be
obtained with the \seefl{get\_page\_size}{getpagesize}  function.

\Errors
Check the \var{int31error} variable.
\SeeAlso
\seefl{get\_page\_size}{getpagesize} 
\end{functionl}
\latex{\inputlisting{go32ex/meminfo.pp}}
\html{\begin{FPCList}
\item[Example]
\begin{verbatim}
Program meminf;

uses go32;

var meminfo : tmeminfo;

begin
get_meminfo(meminfo);
if (int31error <> 0)  then 
 begin
 Writeln('Error getting DPMI memory information... Halting');
 Writeln('DPMI error number : ', int31error);
 end 
else 
 with meminfo do 
   begin
   Writeln('Largest available free block : ', 
           available_memory div 1024, ' kbytes');
   if (available_pages <> -1) then
     Writeln('Maximum available unlocked pages : ', 
              available_pages);
   if (available_lockable_pages <> -1) then
     Writeln('Maximum lockable available pages : ', 
              available_lockable_pages);
   if (linear_space <> -1) then
     Writeln('Linear address space size : ', 
             linear_space*get_page_size div 1024, 
             ' kbytes');
   if (unlocked_pages <> -1) then
     Writeln('Total number of unlocked pages : ', 
             unlocked_pages);
   if (available_physical_pages <> -1) then
     Writeln('Total number of free pages : ', 
             available_physical_pages);
   if (total_physical_pages <> -1) then
     Writeln('Total number of physical pages : ', 
             total_physical_pages);
   if (free_linear_space <> -1) then
     Writeln('Free linear address space : ', 
             free_linear_space*get_page_size div 1024,
             ' kbytes');
   if (max_pages_in_paging_file <> -1) then
     Writeln('Maximum size of paging file : ', 
              max_pages_in_paging_file*get_page_size div 1024, 
              ' kbytes');
  end;
end.\end{verbatim}
\end{FPCList}}
\begin{functionl}{get\_next\_selector\_increment\_value}{getnextselectorincrementvalue}
\Declaration
Function get\_next\_selector\_increment\_value  : Word;

\Description
Returns the selector increment value when allocating multiple subsequent
descriptors via \seefl{allocate\_ldt\_descriptors}{allocateldtdescriptors}.
Parameters: None.
Return value: Selector increment value.
Notes: Because \seefl{allocate\_ldt\_descriptors}{allocateldtdescriptors} only returns the selector for the
first descriptor and so the value returned by this function can be used to
calculate the selectors for subsequent descriptors in the array.

\Errors
 Check the \var{int31error} variable.
\SeeAlso
 \seefl{allocate\_ldt\_descriptors}{allocateldtdescriptors}, 
\seefl{free\_ldt\_descriptor}{freeldtdescriptor} 
\end{functionl}
\begin{functionl}{get\_page\_size}{getpagesize}
\Declaration
Function get\_page\_size  :  Longint;

\Description
 Returns the size of a single memory page.
Return value: Size of a single page in bytes.
Notes: The returned size is typically 4096 bytes.

\Errors
 Check the \var{int31error} variable.
\SeeAlso
 \seefl{get\_meminfo}{getmeminfo} 
\end{functionl}
For an example, see \seefl{get\_meminfo}{getmeminfo}.
\begin{functionl}{get\_pm\_interrupt}{getpminterrupt}
\Declaration
Function get\_pm\_interrupt (vector : byte; var intaddr : tseginfo) : boolean;

\Description
Returns the address of a current protected mode interrupt handler.
Parameters:
\begin{description}
\item[vector:\ ] interrupt handler number you want the address to.
\item[intaddr:\ ] buffer to store address.
\end{description}
Return values: \var{True} if successful, \var{False} if not.
Notes: The returned address is a protected mode selector:offset address.

\Errors
 Check the \var{int31error} variable.
\SeeAlso
 \seefl{set\_pm\_interrupt}{setpminterrupt},
\seefl{set\_rm\_interrupt}{setrminterrupt}, \seefl{get\_rm\_interrupt}{getrminterrupt} 
\end{functionl}
For an example, see \seefl{set\_pm\_interrupt}{setpminterrupt}.
\begin{functionl}{get\_rm\_callback}{getrmcallback}
\Declaration
Function get\_rm\_callback (pm\_func : pointer; const reg : trealregs; var rmcb: tseginfo) : boolean;

\Description

Returns a unique real mode \var{segment:offset} address, known as a "real mode
callback," that will transfer control from real mode to a protected mode
procedure.
Parameters:
\begin{description}
\item[pm\_func:\ ]  pointer to the protected mode callback function.
\item[reg:\ ] supplied registers structure.
\item[rmcb:\ ] buffer to real mode address of callback function.
\end{description}
Return values: \var{True} if successful, otherwise \var{False}.
Notes: Callback addresses obtained with this function can be passed by a
protected mode program for example to an interrupt handler, device driver,
or TSR, so that the real mode program can call procedures within the
protected mode program or notify the protected mode program of an event. The
contents of the supplied regs structure is not valid after function call,
but only at the time of the actual callback.

\Errors
Check the \var{int31error} variable.
\SeeAlso
\seefl{free\_rm\_callback}{freermcallback} 
\end{functionl}
\latex{\inputlisting{go32ex/callback.pp}}
\html{\begin{FPCList}
\item[Example]
\begin{verbatim}
Program callback;

uses crt,
     go32;

const mouseint = $33;          

var mouse_regs    : trealregs;
    mouse_seginfo : tseginfo;

var mouse_numbuttons : longint;

    mouse_action : word;
    mouse_x, mouse_y : Word;
    mouse_b : Word;

    userproc_installed : Longbool;
    userproc_length : Longint;
    userproc_proc : pointer;

{$ASMMODE DIRECT}
procedure callback_handler; assembler;
asm
   pushw %es
   pushw %ds
   pushl %edi
   pushl %esi
   cmpl $1, _USERPROC_INSTALLED
   je .LNoCallback
   pushal
   movw %es, %ax 
   movw %ax, %ds
   movw U_GO32_DOSMEMSELECTOR, %ax
   movw %ax, %fs  
   call *_USERPROC_PROC
   popal
.LNoCallback:

   popl %esi
   popl %edi
   popw %ds
   popw %es

   movl (%esi), %eax
   movl %eax, %es: 42(%edi) 
   addw $4, %es: 46(%edi)
   iret
end;

procedure mouse_dummy; begin end;

procedure textuserproc;
begin
     mouse_b := mouse_regs.bx;
     mouse_x := (mouse_regs.cx shr 3) + 1;
     mouse_y := (mouse_regs.dx shr 3) + 1;
end;

procedure install_mouse (userproc : pointer; 
                         userproclen : longint);
var r : trealregs;
begin
     r.eax := $0; realintr(mouseint, r);
     if (r.eax <> $FFFF) then begin
        Writeln('No Mircosoft compatible mouse found');
        Write('A Mircosoft compatible mouse driver is');
        writeln(' necessary to run this example');
        halt;
     end;
     if (r.bx = $ffff) then mouse_numbuttons := 2
     else mouse_numbuttons := r.bx;
     Writeln(mouse_numbuttons, 
             ' button Mircosoft compatible mouse found.');
     if (userproc <> nil) then begin
        userproc_proc := userproc;
        userproc_installed := true;
        userproc_length := userproclen;
        lock_code(userproc_proc, userproc_length);
     end else begin
         userproc_proc := nil;
         userproc_length := 0;
         userproc_installed := false;
     end;
     lock_data(mouse_x, sizeof(mouse_x));
     lock_data(mouse_y, sizeof(mouse_y));
     lock_data(mouse_b, sizeof(mouse_b));
     lock_data(mouse_action, sizeof(mouse_action));

     lock_data(userproc_installed, sizeof(userproc_installed));
     lock_data(@userproc_proc, sizeof(userproc_proc));

     lock_data(mouse_regs, sizeof(mouse_regs));
     lock_data(mouse_seginfo, sizeof(mouse_seginfo));
     lock_code(@callback_handler, 
               longint(@mouse_dummy) 
                - longint(@callback_handler));
     get_rm_callback(@callback_handler, mouse_regs, mouse_seginfo);
     r.eax := $0c; r.ecx := $7f; r.edx := longint(mouse_seginfo.offset);
     r.es := mouse_seginfo.segment;
     realintr(mouseint, r);
     r.eax := $01;
     realintr(mouseint, r);
end;

procedure remove_mouse;
var r : trealregs;
begin
     r.eax := $02; realintr(mouseint, r);
     r.eax := $0c; r.ecx := 0; r.edx := 0; r.es := 0;
     realintr(mouseint, r);
     free_rm_callback(mouse_seginfo);
     if (userproc_installed) then begin
        unlock_code(userproc_proc, userproc_length);
        userproc_proc := nil;
        userproc_length := 0;
        userproc_installed := false;
     end;
     unlock_data(mouse_x, sizeof(mouse_x));
     unlock_data(mouse_y, sizeof(mouse_y));
     unlock_data(mouse_b, sizeof(mouse_b));
     unlock_data(mouse_action, sizeof(mouse_action));

     unlock_data(@userproc_proc, sizeof(userproc_proc));
     unlock_data(userproc_installed, 
                 sizeof(userproc_installed));

     unlock_data(mouse_regs, sizeof(mouse_regs));
     unlock_data(mouse_seginfo, sizeof(mouse_seginfo));
     unlock_code(@callback_handler, 
                 longint(@mouse_dummy)
                  - longint(@callback_handler));
     fillchar(mouse_seginfo, sizeof(mouse_seginfo), 0);
end;


begin
     install_mouse(@textuserproc, 400);
     Writeln('Press any key to exit...');
     while (not keypressed) do begin
           { write mouse state info }
           gotoxy(1, wherey);
           write('MouseX : ', mouse_x:2, 
                 ' MouseY : ', mouse_y:2, 
                 ' Buttons : ', mouse_b:2);
     end;
     remove_mouse;
end.\end{verbatim}
\end{FPCList}}
\begin{functionl}{get\_rm\_interrupt}{getrminterrupt}
\Declaration
Function get\_rm\_interrupt (vector : byte; var intaddr :
tseginfo) : boolean;

\Description
Returns the contents of the current machine's real mode interrupt vector for
the specified interrupt.
Parameters:
\begin{description}
\item[vector:\ ] interrupt vector number. 
\item[intaddr:\ ] buffer to store real mode \var{segment:offset} address.
\end{description}
Return values: \var{True} if successful, \var{False} otherwise.
Notes: The returned address is a real mode segment address, which isn't
valid in protected mode.

\Errors
 Check the \var{int31error} variable.
\SeeAlso
 \seefl{set\_rm\_interrupt}{setrminterrupt}, 
\seefl{set\_pm\_interrupt}{setpminterrupt}, 
\seefl{get\_pm\_interrupt}{getpminterrupt} 
\end{functionl}
\begin{functionl}{get\_run\_mode}{getrunmode}
\Declaration
Function get\_run\_mode  : Word;

\Description
Returns the current mode your application runs with.
Return values: One of the constants used by this function.

\Errors
None. 
\SeeAlso
 constants returned by \seefl{get\_run\_mode}{getrunmode}  
\end{functionl}
\latex{\inputlisting{go32ex/getrunmd.pp}}
\html{\begin{FPCList}
\item[Example]
\begin{verbatim}
Program getrunmd;

uses go32;

begin
{
  depending on the detected environment,
  we simply write another message
}
case (get_run_mode) of
  rm_unknown : 
    Writeln('Unknown environment found');
  rm_raw     : 
    Writeln('You are currently running in raw mode',
            ' (without HIMEM)');
  rm_xms     : 
    Writeln('You are currently using HIMEM.SYS only');
  rm_vcpi    : 
    Writeln('VCPI server detected.',
            ' You''re using HIMEM and EMM386');
  rm_dpmi    : 
    Writeln('DPMI detected.',
            ' You''re using a DPMI host like ',
            'a windows DOS box or CWSDPMI');
end;
end.\end{verbatim}
\end{FPCList}}
\begin{functionl}{get\_segment\_base\_address}{getsegmentbaseaddress}
\Declaration
Function get\_segment\_base\_address  
(d : Word) : Longint;

\Description
 Returns the 32-bit linear base address from the descriptor table for the
specified segment.
Parameters: 
\begin{description}
\item[d:\ ] selector of the descriptor you want the base address.
\end{description}
Return values: Linear base address of specified descriptor.

\Errors
 Check the \var{int31error} variable.
\SeeAlso

\seefl{allocate\_ldt\_descriptors}{allocateldtdescriptors},
\seefl{set\_segment\_base\_address}{setsegmentbaseaddress}, 
\seefl{allocate\_ldt\_descriptors}{allocateldtdescriptors},
\seefl{set\_segment\_limit}{setsegmentlimit},
\seefl{get\_segment\_limit}{getsegmentlimit} 

\end{functionl}
For an example, see 
\seefl{allocate\_ldt\_descriptors}{allocateldtdescriptors}.
\begin{functionl}{get\_segment\_limit}{getsegmentlimit}
\Declaration
Function get\_segment\_limit (d : Word) : Longint;

\Description
Returns a descriptors segment limit.
Parameters:
\begin{description}
\item [d:\ ] selector.
\end{description}
Return value: Limit of the descriptor in bytes.

\Errors
 Returns zero if descriptor is invalid. 
\SeeAlso
\seefl{allocate\_ldt\_descriptors}{allocateldtdescriptors},
\seefl{set\_segment\_limit}{setsegmentlimit}, 
\seefl{set\_segment\_base\_address}{setsegmentbaseaddress},
\seefl{get\_segment\_base\_address}{getsegmentbaseaddress}, 

\end{functionl}
\begin{functionl}{get\_ss}{getss}
\Declaration
Function get\_ss  : Word;

\Description

Returns the ss selector.
Parameters: None.
Return values: The content of the ss segment register.

\Errors
 None.
\SeeAlso
 \seefl{get\_ds}{getds}, \seefl{get\_cs}{getcs}
\end{functionl}
\begin{functionl}{global\_dos\_alloc}{globaldosalloc}
\Declaration
Function global\_dos\_alloc (bytes : Longint) : Longint;

\Description
Allocates a block of \dos real mode memory.
Parameters: 
\begin{description}
\item [bytes:\ ] size of requested real mode memory.
\end{description}
Return values: The low word of the returned value contains the selector to
the allocated \dos memory block, the high word the corresponding real mode
segment value. The offset value is always zero.
This function allocates memory from \dos memory pool, i.e. memory below the 1
MB boundary that is controlled by \dos. Such memory blocks are typically used
to exchange data with real mode programs, TSRs, or device drivers. The
function returns both the real mode segment base address of the block and
one descriptor that can be used by protected mode applications to access the
block. This function should only used for temporary buffers to get real mode
information (e.g. interrupts that need a data structure in ES:(E)DI),
because every single block needs an unique selector. The returned selector
should only be freed by a \seefl{global\_dos\_free}{globaldosfree}  call.

\Errors
 Check the \var{int31error} variable.
\SeeAlso
 \seefl{global\_dos\_free}{globaldosfree} 
\end{functionl}
\latex{\inputlisting{go32ex/buffer.pp}}
\html{\begin{FPCList}
\item[Example]
\begin{verbatim}
Program buffer;

uses go32;

procedure dosalloc(var selector : word; var segment : word; size : longint);
var res : longint;
begin
     res := global_dos_alloc(size);
     selector := word(res);
     segment := word(res shr 16);
end;

procedure dosfree(selector : word);
begin
     global_dos_free(selector);
end;

type VBEInfoBuf = record
                Signature : array[0..3] of char; 
                Version : Word;
                reserved : array[0..505] of byte; 
     end;

var selector,       
    segment : Word; 

    r : trealregs;  
    infobuf : VBEInfoBuf;

begin
     fillchar(r, sizeof(r), 0);
     fillchar(infobuf, sizeof(VBEInfoBuf), 0);
     dosalloc(selector, segment, sizeof(VBEInfoBuf));
     if (int31error<>0) then begin
        Writeln('Error while allocating real mode memory, halting');
        halt;
     end;
     infobuf.Signature := 'VBE2';
     dosmemput(segment, 0, infobuf, sizeof(infobuf));
     r.ax := $4f00; r.es := segment;
     realintr($10, r);
     dosmemget(segment, 0, infobuf, sizeof(infobuf));
     dosfree(selector);
     if (r.ax <> $4f) then begin
        Writeln('VBE BIOS extension not available, function call failed');
        halt;
     end;
     if (infobuf.signature[0] = 'V') and (infobuf.signature[1] = 'E') and
        (infobuf.signature[2] = 'S') and (infobuf.signature[3] = 'A') then begin
        Writeln('VBE version ', hi(infobuf.version), '.', lo(infobuf.version), ' detected');
     end;
end.\end{verbatim}
\end{FPCList}}
\begin{functionl}{global\_dos\_free}{globaldosfree}
\Declaration
Function global\_dos\_free (selector :Word) : boolean;

\Description
Frees a previously allocated \dos memory block.
Parameters:
\begin{description}
\item[selector:\ ] selector to the \dos memory block.
\end{description}
Return value: \var{True} if successful, \var{False} otherwise.
Notes: The descriptor allocated for the memory block is automatically freed
and hence invalid for further use. This function should only be used for
memory allocated by \seefl{global\_dos\_alloc}{globaldosalloc}.

\Errors
 Check the \var{int31error} variable.
\SeeAlso
\seefl{global\_dos\_alloc}{globaldosalloc} 
\end{functionl}
For an example, see \seefl{global\_dos\_alloc}{globaldosalloc}.
\begin{function}{inportb}
\Declaration
Function inportb (port : Word) : byte;

\Description
Reads 1 byte from the selected I/O port.
Parameters: 
\begin{description}
\item[port:\ ] the I/O port number which is read.
\end{description}
Return values: Current I/O port value.

\Errors
 None. 
\SeeAlso
\seep{outportb}, \seef{inportw}, \seef{inportl}
\end{function}
\begin{function}{inportl}
\Declaration
Function inportl (port : Word) : Longint;

\Description

Reads 1 longint from the selected I/O port.
Parameters: 
\begin{description}
\item[port:\ ] the I/O port number which is read.
\end{description}
Return values: Current I/O port value.

\Errors
None. 
\SeeAlso
\seep{outportb}, \seef{inportb}, \seef{inportw} 
\end{function}
\begin{function}{inportw}
\Declaration
Function inportw (port : Word) : Word;

\Description

Reads 1 word from the selected I/O port.
Parameters: 
\begin{description}
\item[port:\ ] the I/O port number which is read.
\end{description}
Return values: Current I/O port value.

\Errors
 None. 
\SeeAlso
\seep{outportw} \seef{inportb}, \seef{inportl} 
\end{function}
\begin{functionl}{lock\_code}{lockcode}
\Declaration
Function lock\_code (functionaddr : pointer; size : Longint) : boolean;

\Description
Locks a memory range which is in the code segment selector.
Parameters: 
\begin{description}
\item[functionaddr:\ ] address of the function to be locked.
\item[size:\ ] size in bytes to be locked.
\end{description}
Return values: \var{True} if successful, \var{False} otherwise.

\Errors
Check the \var{int31error} variable.
\SeeAlso
 
\seefl{lock\_linear\_region}{locklinearregion},
\seefl{lock\_data}{lockdata},
\seefl{unlock\_linear\_region}{unlocklinearregion},
\seefl{unlock\_data}{unlockdata},
\seefl{unlock\_code}{unlockcode} 
\end{functionl}
For an example, see \seefl{get\_rm\_callback}{getrmcallback}.
\begin{functionl}{lock\_data}{lockdata}
\Declaration
Function lock\_data (var data; size : Longint) : boolean;

\Description
Locks a memory range which resides in the data segment selector.
Parameters:
\begin{description}
\item[data:\ ] address of data to be locked.
\item[size:\ ] length of data to be locked.
\end{description}
Return values: \var{True} if successful, \var{False} otherwise.

\Errors
 Check the \var{int31error} variable.
\SeeAlso

\seefl{lock\_linear\_region}{locklinearregion},
\seefl{lock\_code}{lockcode},
\seefl{unlock\_linear\_region}{unlocklinearregion},
\seefl{unlock\_data}{unlockdata},
\seefl{unlock\_code}{unlockcode} 
\end{functionl}
For an example, see \seefl{get\_rm\_callback}{getrmcallback}.
\begin{functionl}{lock\_linear\_region}{locklinearregion}
\Declaration
Function lock\_linear\_region (linearaddr, size : Longint) : boolean;

\Description
Locks a memory region to prevent swapping of it.
Parameters: 
\begin{description}
\item[linearaddr:\ ] the linear address of the memory are to be locked.
\item[size:\ ] size in bytes to be locked.
\end{description}
Return value: \var{True} if successful, False otherwise.

\Errors
 Check the \var{int31error} variable.
\SeeAlso

\seefl{lock\_data}{lockdata},
\seefl{lock\_code}{lockcode},
\seefl{unlock\_linear\_region}{unlocklinearregion},
\seefl{unlock\_data}{unlockdata},
\seefl{unlock\_code}{unlockcode}
\end{functionl}
\begin{procedure}{outportb}
\Declaration
Procedure outportb (port : Word; data : byte);

\Description
Sends 1 byte of data to the specified I/O port.
Parameters: 
\begin{description}
\item[port:\ ] the I/O port number to send data to.
\item[data:\ ] value sent to I/O port.
\end{description}
Return values: None.

\Errors
 None. 
\SeeAlso
\seef{inportb}, \seep{outportl}, \seep{outportw} 
\end{procedure}
\latex{\inputlisting{go32ex/outport.pp}}
\html{\begin{FPCList}
\item[Example]
\begin{verbatim}
program outport;

uses crt, go32;

begin
 { turn on speaker }
 outportb($61, $ff);
 { wait a little bit }
 delay(50);
 { turn it off again }
 outportb($61, $0);
end.\end{verbatim}
\end{FPCList}}
\begin{procedure}{outportl}
\Declaration
Procedure outportl (port : Word; data : Longint);

\Description
Sends 1 longint of data to the specified I/O port.
Parameters: 
\begin{description}
\item[port:\ ] the I/O port number to send data to.
\item[data:\ ] value sent to I/O port.
\end{description}
Return values: None.

\Errors
None. 
\SeeAlso
\seef{inportl}, \seep{outportw}, \seep{outportb}
\end{procedure}
For an example, see \seep{outportb}.
\begin{procedure}{outportw}
\Declaration
Procedure outportw (port : Word; data : Word);

\Description
Sends 1 word of data to the specified I/O port.
Parameters: 
\begin{description}
\item[port:\ ] the I/O port number to send data to.
\item[data:\ ] value sent to I/O port.
\end{description}
Return values: None.

\Errors
 None. 
\SeeAlso
\seef{inportw}, \seep{outportl}, \seep{outportb}
\end{procedure}
For an example, see \seep{outportb}.
\begin{function}{realintr}
\Declaration
Function realintr (intnr: Word; var regs : trealregs) :  boolean;

\Description
Simulates an interrupt in real mode.
Parameters:
\begin{description}
\item[intnr:\ ] interrupt number to issue in real mode.
\item[regs:\ ] registers data structure.
\end{description}
Return values: The supplied registers data structure contains the values
that were returned by the real mode interrupt. \var{True} if successful, \var{False} if
not.
Notes: The function transfers control to the address specified by the real
mode interrupt vector of intnr. The real mode handler must return by
executing an IRET.

\Errors
 Check the \var{int31error} variable.
\SeeAlso

\end{function}
\latex{\inputlisting{go32ex/flags.pp}}
\html{\begin{FPCList}
\item[Example]
\begin{verbatim}
Program flags;

uses go32;

var r : trealregs;

begin
     r.ax := $5300;
     r.bx := 0;
     realintr($15, r);
     { check if carry clear and write a suited message }
     if ((r.flags and carryflag)=0) then begin
        Writeln('APM v',(r.ah and $f), 
                '.', (r.al shr 4), (r.al and $f), 
                ' detected');
     end else
         Writeln('APM not present');
end.\end{verbatim}
\end{FPCList}}
\begin{procedurel}{seg\_fillchar}{segfillchar}
\Declaration
Procedure seg\_fillchar (seg : Word; ofs : Longint; count : Longint; c : char);

\Description

Sets a memory area to a specific value.
Parameters:
\begin{description}
\item[seg:\ ] selector to memory area.
\item[ofs:\ ] offset to memory.
\item[count:\ ] number of bytes to set.
\item[c:\ ] byte data which is set.
\end{description}
Return values: None.
Notes: No range check is done in any way.

\Errors
 None. 
\SeeAlso
\seepl{seg\_move}{segmove},
\seepl{seg\_fillword}{segfillword},
\seepl{dosmemfillchar}{dosmemfillchar},
\seepl{dosmemfillword}{dosmemfillword},
\seepl{dosmemget}{dosmemget},
\seepl{dosmemput}{dosmemput},
\seepl{dosmemmove}{dosmemmove} 
\end{procedurel}
\latex{\inputlisting{go32ex/vgasel.pp}}
\html{\begin{FPCList}
\item[Example]
\begin{verbatim}
Program svgasel;

uses go32;

var vgasel : Word;
    r : trealregs;

begin
  r.eax := $13; realintr($10, r);
  vgasel := segment_to_descriptor($A000);
  { simply fill the screen memory with color 15 }
  seg_fillchar(vgasel, 0, 64000, #15);
  readln;
 { back to text mode }
  r.eax := $3; 
  realintr($10, r);
end.\end{verbatim}
\end{FPCList}}
\begin{procedurel}{seg\_fillword}{segfillword}
\Declaration
Procedure seg\_fillword (seg : Word; ofs : Longint; count : Longint; w :Word);

\Description

Sets a memory area to a specific value.
Parameters:
\begin{description}
\item[seg:\ ] selector to memory area.
\item[ofs:\ ] offset to memory.
\item[count:\ ] number of words to set.
\item[w:\ ] word data which is set.
\end{description}
Return values: None.
Notes: No range check is done in any way.

\Errors
None. 
\SeeAlso
 
\seepl{seg\_move}{segmove},
\seepl{seg\_fillchar}{segfillchar}, 
\seepl{dosmemfillchar}{dosmemfillchar}, 
\seepl{dosmemfillword}{dosmemfillword},
\seepl{dosmemget}{dosmemget},
\seepl{dosmemput}{dosmemput},
\seepl{dosmemmove}{dosmemmove} 
\end{procedurel}
For an example, see 
\seefl{allocate\_ldt\_descriptors}{allocateldtdescriptors}.
\begin{functionl}{segment\_to\_descriptor}{segmenttodescriptor}
\Declaration
Function segment\_to\_descriptor (seg : Word) : Word;

\Description

Maps a real mode segment (paragraph) address onto an descriptor that can be
used by a protected mode program to access the same memory.
Parameters: 
\begin{description}
\item [seg:\ ] the real mode segment you want the descriptor to.
\end{description}
Return values: Descriptor to real mode segment address.
Notes: The returned descriptors limit will be set to 64 kB. Multiple calls
to this function with the same segment address will return the same
selector. Descriptors created by this function can never be modified or
freed. Programs which need to examine various real mode addresses using the
same selector should use the function 
\seefl{allocate\_ldt\_descriptors}{allocateldtdescriptors} and change
the base address as necessary.

\Errors
 Check the \var{int31error} variable. 
\SeeAlso
\seefl{allocate\_ldt\_descriptors}{allocateldtdescriptors},
\seefl{free\_ldt\_descriptor}{freeldtdescriptor},
\seefl{set\_segment\_base\_address}{setsegmentbaseaddress} 

\end{functionl}
For an example, see \seepl{seg\_fillchar}{segfillchar}.
\begin{procedurel}{seg\_move}{segmove}
\Declaration
Procedure seg\_move (sseg : Word; source : Longint; dseg : Word; dest :
Longint; count : Longint);

\Description
Copies data between two memory locations.
Parameters: 
\begin{description}
\item[sseg:\ ] source selector. 
\item[source:\ ] source offset. 
\item[dseg:\ ] destination selector.
\item[dest:\ ] destination offset.
\item[count:\ ] size in bytes to copy.
\end{description}
Return values: None.
Notes: Overlapping is only checked if the source selector is equal to the
destination selector. No range check is done.

\Errors
 None.
\SeeAlso
 
\seepl{seg\_fillchar}{segfillchar},
\seepl{seg\_fillword}{segfillword},
\seepl{dosmemfillchar}{dosmemfillchar},
\seepl{dosmemfillword}{dosmemfillword},
\seepl{dosmemget}{dosmemget},
\seepl{dosmemput}{dosmemput},
\seepl{dosmemmove}{dosmemmove} 
\end{procedurel}
For an example, see 
\seefl{allocate\_ldt\_descriptors}{allocateldtdescriptors}.
\begin{functionl}{set\_descriptor\_access\_rights}{setdescriptoraccessrights}
\Declaration
Function set\_descriptor\_access\_rights (d : Word; w : Word) : Longint;

\Description

Sets the access rights of a descriptor.
Parameters: 
\begin{description}
\item[d:\ ] selector.
\item[w:\ ] new descriptor access rights.
\end{description}
Return values: This function doesn't return anything useful.

\Errors
 Check the \var{int31error} variable.
\SeeAlso

\seefl{get\_descriptor\_access\_rights}{getdescriptoraccessrights} 
\end{functionl}
\begin{functionl}{set\_pm\_interrupt}{setpminterrupt}
\Declaration
Function set\_pm\_interrupt (vector : byte; const intaddr : tseginfo) : boolean;

\Description
Sets the address of the protected mode handler for an interrupt.
Parameters: 
\begin{description}
\item[vector:\ ] number of protected mode interrupt to set.
\item[intaddr:\ ] selector:offset address to the interrupt vector.
\end{description}
Return values: \var{True} if successful, \var{False} otherwise.
Notes: The address supplied must be a valid \var{selector:offset} 
protected mode address.

\Errors
 Check the \var{int31error} variable.
\SeeAlso
\seefl{get\_pm\_interrupt}{getpminterrupt}, 
\seefl{set\_rm\_interrupt}{setrminterrupt},
\seefl{get\_rm\_interrupt}{getrminterrupt} 
\end{functionl}
\latex{\inputlisting{go32ex/int_pm.pp}}
\html{\begin{FPCList}
\item[Example]
\begin{verbatim}
Program int_pm;

uses crt, go32;

const int1c = $1c; 

var oldint1c : tseginfo;
    newint1c : tseginfo;
    int1c_counter : Longint;

{$ASMMODE DIRECT}
procedure int1c_handler; assembler;
asm
   cli
   pushw %ds
   pushw %ax
   movw %cs:INT1C_DS, %ax
   movw %ax, %ds
   incl _INT1C_COUNTER
   popw %ax
   popw %ds
   sti
   iret
INT1C_DS: .word 0
end;

var i : Longint;

begin
     newint1c.offset := @int1c_handler;
     newint1c.segment := get_cs;
     get_pm_interrupt(int1c, oldint1c);
     asm
        movw %ds, %ax
        movw %ax, INT1C_DS
     end;
     Writeln('-- Press any key to exit --');
     set_pm_interrupt(int1c, newint1c);
     while (not keypressed) do begin
           gotoxy(1, wherey); 
           write('Number of interrupts occured : ', 
                 int1c_counter);
     end;
     set_pm_interrupt(int1c, oldint1c);
end.\end{verbatim}
\end{FPCList}}
\begin{functionl}{set\_rm\_interrupt}{setrminterrupt}
\Declaration
Function set\_rm\_interrupt (vector : byte; const intaddr :
tseginfo) : boolean;

\Description
Sets a real mode interrupt handler.
Parameters:
\begin{description}
\item[vector:\ ] the interrupt vector number to set.
\item[intaddr:\ ] address of new interrupt vector.
\end{description}
Return values: \var{True} if successful, otherwise \var{False}.
Notes: The address supplied MUST be a real mode segment address, not a
\var{selector:offset} address. So the interrupt handler must either reside in \dos
memory (below 1 Mb boundary) or the application must allocate a real mode
callback address with \seefl{get\_rm\_callback}{getrmcallback}.

\Errors
 Check the \var{int31error} variable.
\SeeAlso
 
\seefl{get\_rm\_interrupt}{getrminterrupt}, 
\seefl{set\_pm\_interrupt}{setpminterrupt}, \seefl{get\_pm\_interrupt}{getpminterrupt}, 
\seefl{get\_rm\_callback}{getrmcallback} 
\end{functionl}
\begin{functionl}{set\_segment\_base\_address}{setsegmentbaseaddress}
\Declaration
Function set\_segment\_base\_address (d : Word; s : Longint) : boolean;

\Description
Sets the 32-bit linear base address of a descriptor.
Parameters: 
\begin{description}
\item[d:\ ] selector.
\item[s:\ ] new base address of the descriptor.
\end{description}

\Errors
 Check the \var{int31error} variable.
\SeeAlso

\seefl{allocate\_ldt\_descriptors}{allocateldtdescriptors},
\seefl{get\_segment\_base\_address}{getsegmentbaseaddress}, 
\seefl{allocate\_ldt\_descriptors}{allocateldtdescriptors}, 
\seefl{set\_segment\_limit}{setsegmentlimit},
\seefl{get\_segment\_base\_address}{getsegmentbaseaddress},
\seefl{get\_segment\_limit}{getsegmentlimit} 

\end{functionl}
\begin{functionl}{set\_segment\_limit}{setsegmentlimit}
\Declaration
Function set\_segment\_limit (d : Word; s : Longint) : boolean;

\Description
Sets the limit of a descriptor.
Parameters: 
\begin{description}
\item[d:\ ] selector.
\item[s:\ ] new limit of the descriptor.
\end{description}
Return values: Returns \var{True} if successful, else \var{False}.
Notes: The new limit specified must be the byte length of the segment - 1.
Segment limits bigger than or equal to 1MB must be page aligned, they must
have the lower 12 bits set.

\Errors
 Check the \var{int31error} variable.
\SeeAlso
\seefl{allocate\_ldt\_descriptors}{allocateldtdescriptors},
\seefl{set\_segment\_base\_address}{setsegmentbaseaddress},
\seefl{get\_segment\_limit}{getsegmentlimit}, 
\seefl{set\_segment\_limit}{setsegmentlimit} 

\end{functionl}
For an example, see 
\seefl{allocate\_ldt\_descriptors}{allocateldtdescriptors}.
\begin{functionl}{tb\_size}{tbsize}
\Declaration
Function tb\_size  : Longint;

\Description
Returns the size of the pre-allocated \dos memory buffer.
Parameters: None.
Return values: The size of the pre-allocated \dos memory buffer.
Notes:
This block always seems to be 16k in size, but don't rely on this.

\Errors
None.
\SeeAlso
\seefl{transfer\_buffer}{transferbuffer}, \seep{copyfromdos}
\seep{copytodos}
\end{functionl}

\begin{functionl}{transfer\_buffer}{transferbuffer}
\Declaration
Function transfer\_buffer : Longint;
\Description
\var{transfer\_buffer} returns the offset of the transfer buffer.
\Errors
None.
\SeeAlso
\seefl{tb\_size}{tbsize}
\end{functionl}

\begin{functionl}{unlock\_code}{unlockcode}
\Declaration
Function unlock\_code (functionaddr : pointer; size : Longint) : boolean;

\Description
Unlocks a memory range which resides in the code segment selector.
Parameters:
\begin{description}
\item[functionaddr:\ ] address of function to be unlocked. 
\item[size:\ ] size bytes to be unlocked.
\end{description}
Return value: \var{True} if successful, \var{False} otherwise.

\Errors
 Check the \var{int31error} variable.
\SeeAlso
\seefl{unlock\_linear\_region}{unlocklinearregion},
 \seefl{unlock\_data}{unlockdata},
\seefl{lock\_linear\_region}{locklinearregion},
\seefl{lock\_data}{lockdata},
\seefl{lock\_code}{lockcode} 
\end{functionl}
For an example, see \seefl{get\_rm\_callback}{getrmcallback}.
\begin{functionl}{unlock\_data}{unlockdata}
\Declaration
Function unlock\_data (var data; size : Longint) : boolean;

\Description
Unlocks a memory range which resides in the data segment selector.
Paramters:
\begin{description}
\item[data:\ ] address of memory to be unlocked. 
\item[size:\ ] size bytes to be unlocked.
\end{description}
Return values: \var{True} if successful, \var{False} otherwise.

\Errors
 Check the \var{int31error} variable.
\SeeAlso
\seefl{unlock\_linear\_region}{unlocklinearregion},
\seefl{unlock\_code}{unlockcode},
\seefl{lock\_linear\_region}{locklinearregion},
\seefl{lock\_data}{lockdata},
\seefl{lock\_code}{lockcode} 
\end{functionl}
For an example, see \seefl{get\_rm\_callback}{getrmcallback}.
\begin{functionl}{unlock\_linear\_region}{unlocklinearregion}
\Declaration
Function unlock\_linear\_region (linearaddr, size : Longint) : boolean;

\Description
Unlocks a previously locked linear region range to allow it to be swapped
out again if needed.
Parameters:
\begin{description}
\item[linearaddr:\ ] linear address of the memory to be unlocked. 
\item[size:\ ] size bytes to be unlocked.
\end{description}
Return values: \var{True} if successful, \var{False} otherwise.

\Errors
 Check the \var{int31error} variable.
\SeeAlso

\seefl{unlock\_data}{unlockdata},
\seefl{unlock\_code}{unlockcode},
\seefl{lock\_linear\_region}{locklinearregion},
\seefl{lock\_data}{lockdata},
\seefl{lock\_code}{lockcode}
\end{functionl}


% The graph unit
%
%   $Id$
%   This file is part of the FPC documentation.
%   Copyright (C) 1997, by Michael Van Canneyt
%
%   The FPC documentation is free text; you can redistribute it and/or
%   modify it under the terms of the GNU Library General Public License as
%   published by the Free Software Foundation; either version 2 of the
%   License, or (at your option) any later version.
%
%   The FPC Documentation is distributed in the hope that it will be useful,
%   but WITHOUT ANY WARRANTY; without even the implied warranty of
%   MERCHANTABILITY or FITNESS FOR A PARTICULAR PURPOSE.  See the GNU
%   Library General Public License for more details.
%
%   You should have received a copy of the GNU Library General Public
%   License along with the FPC documentation; see the file COPYING.LIB.  If not,
%   write to the Free Software Foundation, Inc., 59 Temple Place - Suite 330,
%   Boston, MA 02111-1307, USA. 
%
% Documentation for the 'Graph' unit of Free Pascal.
% Michael Van Canneyt, July 1997
\chapter{The GRAPH unit.}
\FPCexampledir{graphex}
This document describes the \var{GRAPH} unit for Free Pascal, for all
platforms. The unit was first written for \dos by Florian kl\"ampfl, but was
later completely rewritten by Carl-Eric Codere to be completely portable.

This chapter is divided in 4 sections. 
\begin{itemize}
\item The first section gives an introduction to the graph unit.
\item The second section lists the pre-defined constants, types and variables. 
\item The second section describes the functions which appear in the
interface part of the \file{GRAPH} unit.
\item The last part describes some system-specific issues.

\end{itemize}
\section{Introduction}
\label{se:Introduction}
\subsection{Requirements}
The unit Graph exports functions and procedures for graphical output.
It requires at least a VGA-compatible Card or a VGA-Card with software-driver
(min. \textbf{512Kb} video memory).
\subsection{A word about mode selection}
The graph unit was implemented for compatibility with the old \tp graph
unit. For this reason, the mode constants as they were defined in the
\tp graph unit are retained. 

However, since
\begin{enumerate}
\item Video cards have evolved very much
\item Free Pascal runs on multiple platforms
\end{enumerate}
it was decided to implement new mode and graphic driver constants, 
which are more independent of the specific platform the program runs on.

In this section we give a short explanation of the new mode system. the
following drivers were defined:
\begin{verbatim}
D1bit = 11;
D2bit = 12;
D4bit = 13;
D6bit = 14;  { 64 colors Half-brite mode - Amiga }
D8bit = 15;
D12bit = 16; { 4096 color modes HAM mode - Amiga }
D15bit = 17;
D16bit = 18;
D24bit = 19; { not yet supported }
D32bit = 20; { not yet supported }
D64bit = 21; { not yet supported }

lowNewDriver = 11;
highNewDriver = 21;
\end{verbatim}
Each of these drivers specifies a desired color-depth. 

The following modes have been defined:
\begin{verbatim}
detectMode = 30000;
m320x200 = 30001;  
m320x256 = 30002; { amiga resolution (PAL) }
m320x400 = 30003; { amiga/atari resolution }
m512x384 = 30004; { mac resolution }
m640x200 = 30005; { vga resolution }
m640x256 = 30006; { amiga resolution (PAL) }
m640x350 = 30007; { vga resolution }
m640x400 = 30008;
m640x480 = 30009;
m800x600 = 30010;
m832x624 = 30011; { mac resolution }
m1024x768 = 30012;
m1280x1024 = 30013;
m1600x1200 = 30014;
m2048x1536 = 30015;

lowNewMode = 30001;
highNewMode = 30015;
\end{verbatim}
These modes start at 30000 because Borland specified that the mode number
should be ascending with increasing X resolution, and the new constants 
shouldn't interfere with the old ones.

The above constants can be used to set a certain color depth and resultion,
as demonstrated in the following example:

\FPCexample{inigraph1}

If other modes than the ones above are supported by the graphics card,
you will not be able to select them with this mechanism.

For this reason, there is also a 'dynamic' mode number, which is assigned at
run-time. This number increases with increasing X resolution. It can be
queried with the \var{getmoderange} call. This call will return the range
of modes which are valid for a certain graphics driver. The numbers are
guaranteed to be consecutive, and can be used to search for a certain 
resolution, as in the following example:

\FPCexample{inigraph2}


Thus, the \var{getmoderange} function can be used to detect all available 
modes and drivers, as in the following example:

\FPCexample{modrange}

\section{Constants, Types and Variables}
\subsection{Types}
\begin{verbatim}
ArcCoordsType = record
 X,Y,Xstart,Ystart,Xend,Yend : Integer;
end;
FillPatternType = Array [1..8] of Byte;
FillSettingsType = Record
 Pattern,Color : Word
end;
LineSettingsType = Record
  LineStyle,Pattern, Width : Word;
end;
PaletteType = Record
 Size : Byte;
 Colors : array[0..MAxColor] of shortint;
end;
PointType = Record
  X,Y : Integer;
end;
TextSettingsType = Record
 Font,Direction, CharSize, Horiz, Vert : Word
end;
ViewPortType = Record
  X1,Y1,X2,Y2 : Integer;
  Clip : Boolean
end;
\end{verbatim}

%%%%%%%%%%%%%%%%%%%%%%%%%%%%%%%%%%%%%%%%%%%%%%%%%%%%%%%%%%%%%%%%%%%%%%%
% Functions and procedures by category
\section{Function list by category}
What follows is a listing of the available functions, grouped by category.
For each function there is a reference to the page where you can find the
function.
\subsection{Initialization}
Initialization of the graphics screen.
\begin{funclist}
\procref{ClearDevice}{Empty the graphics screen}
\procref{CloseGraph}{Finish drawing session, return to text mode}
\procref{DetectGraph}{Detect graphical modes}
\procref{GetAspectRatio}{Get aspect ratio of screen}
\procref{GetModeRange}{Get range of valid modes for current driver}
\procref{GraphDefaults}{Set defaults}
\funcref{GetDriverName}{Return name of graphical driver}
\funcref{GetGraphMode}{Return current or last used graphics mode}
\funcref{GetMaxMode}{Get maximum mode for current driver}
\funcref{GetModeName}{Get name of current mode}
\funcref{GraphErrorMsg}{String representation of graphical error}
\funcref{GraphResult}{Result of last drawing operation}
\procref{InitGraph}{Initialize graphics drivers}
\funcref{InstallUserDriver}{Install a new driver}
\funcref{RegisterBGIDriver}{Register a new driver}
\procref{RestoreCRTMode}{Go back to text mode}
\procref{SetGraphBufSize}{Set buffer size for graphical operations}
\procref{SetGraphMode}{Set graphical mode}
\end{funclist}

\subsection{screen management}
General drawing screen management functions.
\begin{funclist}
\procref{ClearViewPort}{Clear the current viewport}
\procref{GetImage}{Copy image from screen to memory}
\funcref{GetMaxX}{Get maximum X coordinate}
\funcref{GetMaxY}{Get maximum Y coordinate}
\funcref{GetX}{Get current X position}
\funcref{GetY}{Get current Y position}
\funcref{ImageSize}{Get size of selected image}
\procref{GetViewSettings}{Get current viewport settings}
\procref{PutImage}{Copy image from memory to screen}
\procref{SetActivePage}{Set active video page}
\procref{SetAspectRatio}{Set aspect ratio for drawing routines}
\procref{SetViewPort}{Set current viewport}
\procref{SetVisualPage}{Set visual page}
\procref{SetWriteMode}{Set write mode for screen operations}
\end{funclist}

\subsection{Color management}
All functions related to color management.
\begin{funclist}
\funcref{GetBkColor}{Get current background color}
\funcref{GetColor}{Get current foreground color}
\procref{GetDefaultPalette}{Get default palette entries}
\funcref{GetMaxColor}{Get maximum valid color}
\funcref{GetPaletteSize}{Get size of palette for current mode}
\funcref{GetPixel}{Get color of selected pixel}
\procref{GetPalette}{Get palette entry}
\procref{SetAllPallette}{Set all colors in palette}
\procref{SetBkColor}{Set background color}
\procref{SetColor}{Set foreground color}
\procref{SetPalette}{Set palette entry}
\procref{SetRGBPalette}{Set palette entry with RGB values}
\end{funclist}

\subsection{Drawing primitives}
Functions for simple drawing.
\begin{funclist}
\procref{Arc}{Draw an arc}
\procref{Circle}{Draw a complete circle}
\procref{DrawPoly}{Draw a polygone with N points}
\procref{Ellipse}{Draw an ellipse}
\procref{GetArcCoords}{Get arc coordinates}
\procref{GetLineSettings}{Get current line drawing settings}
\procref{Line}{Draw line between 2 points}
\procref{LineRel}{Draw line relative to current position}
\procref{LineTo}{Draw line from current position to absolute position}
\procref{MoveRel}{Move cursor relative to current position}
\procref{MoveTo}{Move cursor to absolute position}
\procref{PieSlice}{Draw a pie slice}
\procref{PutPixel}{Draw 1 pixel}
\procref{Rectangle}{Draw a non-filled rectangle}
\procref{Sector}{Draw a sector}
\procref{SetLineStyle}{Set current line drawing style}
\end{funclist}

\subsection{Filled drawings}
Functions for drawing filled regions.
\begin{funclist}
\procref{Bar3D}{Draw a filled 3D-style bar}
\procref{Bar}{Draw a filled rectangle}
\procref{FloodFill}{Fill starting from coordinate}
\procref{FillEllipse}{Draw a filled ellipse}
\procref{FillPoly}{Draw a filled polygone}
\procref{GetFillPattern}{Get current fill pattern}
\procref{GetFillSettings}{Get current fill settings}
\procref{SetFillPattern}{Set current fill pattern}
\procref{SetFillStyle}{Set current fill settings}
\end{funclist}

\subsection{Text and font handling}
Functions to set texts on the screen.
\begin{funclist}
\procref{GetTextSettings}{Get current text settings}
\funcref{InstallUserFont}{Install a new font}
\procref{OutText}{Write text at current cursor position}
\procref{OutTextXY}{Write text at coordinates X,Y}
\funcref{RegisterBGIFont}{Register a new font}
\procref{SetTextJustify}{Set text justification}
\procref{SetTextStyle}{Set text style}
\procref{SetUserCharSize}{Set text size}
\funcref{TextHeight}{Calculate height of text}
\funcref{TextWidth}{Calculate width of text}
\end{funclist}


%%%%%%%%%%%%%%%%%%%%%%%%%%%%%%%%%%%%%%%%%%%%%%%%%%%%%%%%%%%%%%%%%%%%%%%
% Functions and procedures
\section{Functions and procedures}

\begin{procedure}{Arc}
\Declaration
Procedure Arc (X,Y : Integer; start,stop, radius : Word);

\Description
 \var{Arc} draws part of a circle with center at \var{(X,Y)}, radius
\var{radius}, starting from angle \var{start}, stopping at angle \var{stop}.
These  angles are measured
counterclockwise.
\Errors
None.
\SeeAlso
\seep{Circle},\seep{Ellipse} 
\seep{GetArcCoords},\seep{PieSlice}, \seep{Sector}
\end{procedure}

\begin{procedure}{Bar}
\Declaration
Procedure Bar (X1,Y1,X2,Y2 : Integer);

\Description
Draws a rectangle with corners at \var{(X1,Y1)} and \var{(X2,Y2)} 
and fills it with the current color and fill-style.
\Errors
None.
\SeeAlso
\seep{Bar3D}, 
\seep{Rectangle}
\end{procedure}

\begin{procedure}{Bar3D}
\Declaration
Procedure Bar3D (X1,Y1,X2,Y2 : Integer; depth : Word; Top : Boolean);

\Description
Draws a 3-dimensional Bar  with corners at \var{(X1,Y1)} and \var{(X2,Y2)} 
and fills it with the current color and fill-style.
\var{Depth} specifies the number of pixels used to show the depth of the
bar.
If \var{Top} is true; then a 3-dimensional top is drawn.
\Errors
None.
\SeeAlso
\seep{Bar}, \seep{Rectangle}
\end{procedure}

\begin{procedure}{Circle}
\Declaration
Procedure Circle (X,Y : Integer; Radius : Word);

\Description
 \var{Circle} draws part of a circle with center at \var{(X,Y)}, radius
\var{radius}.
\Errors
None.
\SeeAlso
\seep{Ellipse},\seep{Arc}
\seep{GetArcCoords},\seep{PieSlice}, \seep{Sector}
\end{procedure}

\begin{procedure}{ClearDevice}
\Declaration
Procedure ClearDevice ;

\Description
Clears the graphical screen (with the current
background color), and sets the pointer at \var{(0,0)}
\Errors
None.
\SeeAlso
\seep{ClearViewPort}, \seep{SetBkColor}
\end{procedure}

\begin{procedure}{ClearViewPort}
\Declaration
Procedure ClearViewPort ;

\Description
Clears the current viewport. The current background color is used as filling
color. The pointer is set at
\var{(0,0)}
\Errors
None.
\SeeAlso
\seep{ClearDevice},\seep{SetViewPort}, \seep{SetBkColor}
\end{procedure}

\begin{procedure}{CloseGraph}
\Declaration
Procedure CloseGraph ;

\Description
Closes the graphical system, and restores the
screen modus which was active before the graphical modus was
activated.
\Errors
None.
\SeeAlso
\seep{InitGraph}
\end{procedure}

\begin{procedure}{DetectGraph}
\Declaration
Procedure DetectGraph (Var Driver, Modus : Integer);

\Description
 Checks the hardware in the PC and determines the driver and screen-modus to
be used. These are returned in \var{Driver} and \var{Modus}, and can be fed
to \var{InitGraph}. 
See the \var{InitGraph} for a list of drivers and modi.
\Errors
None.
\SeeAlso
\seep{InitGraph}
\end{procedure}

\begin{procedure}{DrawPoly}
\Declaration
Procedure DrawPoly (NumberOfPoints : Word; Var PolyPoints;

\Description

Draws a polygone with \var{NumberOfPoints} corner points, using the
current color and line-style. PolyPoints is an array of type \var{PointType}.

\Errors
None.
\SeeAlso
\seep{Bar}, seep{Bar3D}, \seep{Rectangle}
\end{procedure}

\begin{procedure}{Ellipse}
\Declaration
Procedure Ellipse (X,Y : Integer; Start,Stop,XRadius,YRadius : Word);

\Description
 \var{Ellipse} draws part of an ellipse with center at \var{(X,Y)}.
\var{XRadius} and \var{Yradius} are the horizontal and vertical radii of the
ellipse. \var{Start} and \var{Stop} are the starting and stopping angles of
the part of the ellipse. They are measured counterclockwise from the X-axis 
(3 o'clock is equal to 0 degrees). Only positive angles can be specified.
\Errors
None.
\SeeAlso
\seep{Arc} \seep{Circle}, \seep{FillEllipse}
\end{procedure}

\begin{procedure}{FillEllipse}
\Declaration
Procedure FillEllipse (X,Y : Integer; Xradius,YRadius: Word);

\Description
 \var{Ellipse} draws an ellipse with center at \var{(X,Y)}.
\var{XRadius} and \var{Yradius} are the horizontal and vertical radii of the
ellipse. The ellipse is filled with the current color and fill-style.
\Errors
None.
\SeeAlso
\seep{Arc} \seep{Circle},
\seep{GetArcCoords},\seep{PieSlice}, \seep{Sector}
\end{procedure}

\begin{procedure}{FillPoly}
\Declaration
Procedure FillPoly (NumberOfPoints : Word; Var PolyPoints);

\Description

Draws a polygone with \var{NumberOfPoints} corner points and fills it
using the current color and line-style. 
PolyPoints is an array of type \var{PointType}.

\Errors
None.
\SeeAlso
\seep{Bar}, seep{Bar3D}, \seep{Rectangle}
\end{procedure}

\begin{procedure}{FloodFill}
\Declaration
Procedure FloodFill (X,Y : Integer; BorderColor : Word);

\Description

Fills the area containing the point \var{(X,Y)}, bounded by the color
\var{BorderColor}.
\Errors
None
\SeeAlso
\seep{SetColor}, \seep{SetBkColor}
\end{procedure}

\begin{procedure}{GetArcCoords}
\Declaration
Procedure GetArcCoords (Var ArcCoords : ArcCoordsType);

\Description
\var{GetArcCoords} returns the coordinates of the latest \var{Arc} or
\var{Ellipse} call.
\Errors
None.
\SeeAlso
\seep{Arc}, \seep{Ellipse}
\end{procedure}

\begin{procedure}{GetAspectRatio}
\Declaration
Procedure GetAspectRatio (Var Xasp,Yasp : Word);

\Description
\var{GetAspectRatio} determines the effective resolution of the screen. The aspect ration can
the be calculated as \var{Xasp/Yasp}.
\Errors
None.
\SeeAlso
\seep{InitGraph},\seep{SetAspectRatio}
\end{procedure}

\begin{function}{GetBkColor}
\Declaration
Function GetBkColor  : Word;

\Description
\var{GetBkColor} returns the current background color (the palette
entry).
\Errors
None.
\SeeAlso
\seef{GetColor},\seep{SetBkColor}
\end{function}

\begin{function}{GetColor}
\Declaration
Function GetColor  : Word;

\Description
\var{GetColor} returns the current drawing color (the palette
entry).
\Errors
None.
\SeeAlso
\seef{GetColor},\seep{SetBkColor}
\end{function}

\begin{procedure}{GetDefaultPalette}
\Declaration
Procedure GetDefaultPalette (Var Palette : PaletteType);

\Description
Returns the
current palette in \var{Palette}.
\Errors
None.
\SeeAlso
\seef{GetColor}, \seef{GetBkColor}
\end{procedure}

\begin{function}{GetDriverName}
\Declaration
Function GetDriverName  : String;

\Description
\var{GetDriverName} returns a string containing the name of the
current driver.
\Errors
None.
\SeeAlso
\seef{GetModeName}, \seep{InitGraph}
\end{function}

\begin{procedure}{GetFillPattern}
\Declaration
Procedure GetFillPattern (Var FillPattern : FillPatternType);

\Description
\var{GetFillPattern} returns an array with the current fill-pattern  in \var{FillPattern}
\Errors
None
\SeeAlso
\seep{SetFillPattern}
\end{procedure}

\begin{procedure}{GetFillSettings}
\Declaration
Procedure GetFillSettings (Var FillInfo : FillSettingsType);

\Description
\var{GetFillSettings} returns the current fill-settings in
\var{FillInfo}
\Errors
None.
\SeeAlso
\seep{SetFillPattern}
\end{procedure}

\begin{function}{GetGraphMode}
\Declaration
Function GetGraphMode  : Integer;

\Description
\var{GetGraphMode} returns the current graphical modus
\Errors
None.
\SeeAlso
\seep{InitGraph}
\end{function}

\begin{procedure}{GetImage}
\Declaration
Procedure GetImage (X1,Y1,X2,Y2 : Integer, Var Bitmap;

\Description
\var{GetImage}
Places a copy of the screen area \var{(X1,Y1)} to \var{X2,Y2} in \var{BitMap}
\Errors
Bitmap must have enough room to contain the image.
\SeeAlso
\seef{ImageSize},
\seep{PutImage}
\end{procedure}

\begin{procedure}{GetLineSettings}
\Declaration
Procedure GetLineSettings (Var LineInfo : LineSettingsType);

\Description
\var{GetLineSettings} returns the current Line settings in
\var{LineInfo}
\Errors
None.
\SeeAlso
\seep{SetLineStyle}
\end{procedure}

\begin{function}{GetMaxColor}
\Declaration
Function GetMaxColor  : Word;

\Description
\var{GetMaxColor} returns the maximum color-number which can be 
set with \var{SetColor}. Contrary to \tp, this color isn't always 
guaranteed to be white (for instance in 256+ color modes).
\Errors
None.
\SeeAlso
\seep{SetColor},
\seef{GetPaletteSize}
\end{function}

\begin{function}{GetMaxMode}
\Declaration
Function GetMaxMode  : Word;

\Description
\var{GetMaxMode} returns the highest modus for
the current driver.
\Errors
None.
\SeeAlso
\seep{InitGraph}
\end{function}

\begin{function}{GetMaxX}
\Declaration
Function GetMaxX  : Word;

\Description
\var{GetMaxX} returns the maximum horizontal screen
length
\Errors
None.
\SeeAlso
\seef{GetMaxY}
\end{function}

\begin{function}{GetMaxY}
\Declaration
Function GetMaxY  : Word;

\Description
\var{GetMaxY} returns the maximum number of screen
lines
\Errors
None.
\SeeAlso
\seef{GetMaxY}
\end{function}

\begin{function}{GetModeName}
\Declaration
Function GetModeName (Var modus : Integer) : String;

\Description

Returns a string with the name of modus
\var{Modus}
\Errors
None.
\SeeAlso
\seef{GetDriverName}, \seep{InitGraph}
\end{function}

\begin{procedure}{GetModeRange}
\Declaration
Procedure GetModeRange (Driver : Integer; \\ LoModus, HiModus: Integer);
\Description
\var{GetModeRange} returns the Lowest and Highest modus of the currently
installed driver. If no modes are supported for this driver, HiModus
will be -1.
\Errors
None.
\SeeAlso
\seep{InitGraph}
\end{procedure}

\begin{procedure}{GetPalette}
\Declaration
Procedure GetPalette (Var Palette : PaletteType);

\Description
\var{GetPalette} returns in \var{Palette} the current palette.
\Errors
None.
\SeeAlso
\seef{GetPaletteSize}, \seep{SetPalette}
\end{procedure}

\begin{function}{GetPaletteSize}
\Declaration
Function GetPaletteSize  : Word;

\Description
\var{GetPaletteSize} returns the maximum
number of entries in the current palette.
\Errors
None.
\SeeAlso
\seep{GetPalette},
\seep{SetPalette}
\end{function}

\begin{function}{GetPixel}
\Declaration
Function GetPixel (X,Y : Integer) : Word;

\Description
\var{GetPixel} returns the color
of the point at \var{(X,Y)} 
\Errors
None.
\SeeAlso

\end{function}

\begin{procedure}{GetTextSettings}
\Declaration
Procedure GetTextSettings (Var TextInfo : TextSettingsType);

\Description
\var{GetTextSettings} returns the current text style settings : The font,
direction, size and placement as set with \var{SetTextStyle} and
\var{SetTextJustify}
\Errors
None.
\SeeAlso
\seep{SetTextStyle}, \seep{SetTextJustify}
\end{procedure}

\begin{procedure}{GetViewSettings}
\Declaration
Procedure GetViewSettings (Var ViewPort : ViewPortType);

\Description
\var{GetViewSettings} returns the current viewport and clipping settings in
\var{ViewPort}.
\Errors
None.
\SeeAlso
\seep{SetViewPort}
\end{procedure}

\begin{function}{GetX}
\Declaration
Function GetX  : Integer;

\Description
\var{GetX} returns the X-coordinate of the current position of
the graphical pointer
\Errors
None.
\SeeAlso
\seef{GetY}
\end{function}

\begin{function}{GetY}
\Declaration
Function GetY  : Integer;

\Description
\var{GetY} returns the Y-coordinate of the current position of
the graphical pointer
\Errors
None.
\SeeAlso
\seef{GetX}
\end{function}

\begin{procedure}{GraphDefaults}
\Declaration
Procedure GraphDefaults ;

\Description
\var{GraphDefaults} resets all settings for viewport, palette,
foreground and background pattern, line-style and pattern, filling style,
filling color and pattern, font, text-placement and
text size.
\Errors
None.
\SeeAlso
\seep{SetViewPort}, \seep{SetFillStyle}, \seep{SetColor},
\seep{SetBkColor}, \seep{SetLineStyle}
\end{procedure}

\begin{function}{GraphErrorMsg}
\Declaration
Function GraphErrorMsg (ErrorCode : Integer) : String;

\Description
\var{GraphErrorMsg}
returns a string describing the error \var{Errorcode}. This string can be
used to let the user know what went wrong.
\Errors
None.
\SeeAlso
\seef{GraphResult}
\end{function}

\begin{function}{GraphResult}
\Declaration
Function GraphResult  : Integer;

\Description
\var{GraphResult} returns an error-code for
the last graphical operation. If the returned value is zero, all went well.
A value different from zero means an error has occurred.
besides all operations which draw something on the screen, 
the following procedures also can produce a \var{GraphResult} different from
zero:

\begin{itemize}
\item \seef{InstallUserFont}
\item \seep{SetLineStyle}
\item \seep{SetWriteMode}
\item \seep{SetFillStyle}
\item \seep{SetTextJustify}
\item \seep{SetGraphMode}
\item \seep{SetTextStyle}
\end{itemize}

\Errors
None.
\SeeAlso
\seef{GraphErrorMsg}
\end{function}

\begin{function}{ImageSize}
\Declaration
Function ImageSize (X1,Y1,X2,Y2 : Integer) : Word;

\Description
\var{ImageSize} returns
the number of bytes needed to store the image in the rectangle defined by
\var{(X1,Y1)} and \var{(X2,Y2)}.
\Errors
None.
\SeeAlso
\seep{GetImage}
\end{function}

\begin{procedure}{InitGraph}
\Declaration
Procedure InitGraph (var GraphDriver,GraphModus : integer;\\
const PathToDriver : string);

\Description

\var{InitGraph} initializes the \var{graph} package.
\var{GraphDriver} has two valid values: \var{GraphDriver=0} which
performs an auto detect and initializes the highest possible mode with the most
colors. 1024x768x64K is the highest possible resolution supported by the
driver, if you need a higher resolution, you must edit \file{MODES.PPI}. 
If you need another mode, then set \var{GraphDriver} to a value different
from zero
and \var{graphmode} to the mode you wish (VESA modes where 640x480x256
is \var {101h} etc.).
\var{PathToDriver} is only needed, if you use the BGI fonts from
Borland.
\Errors
None.
\SeeAlso
Introduction, (page \pageref{se:Introduction}),
\seep{DetectGraph}, \seep{CloseGraph}, \seef{GraphResult}
\end{procedure}
Example:

\begin{verbatim}
var 
   gd,gm : integer; 
   PathToDriver : string; 
begin 
   gd:=detect; { highest possible resolution } 
   gm:=0; { not needed, auto detection } 
   PathToDriver:='C:\PP\BGI'; { path to BGI fonts, 
                                drivers aren't needed } 
   InitGraph(gd,gm,PathToDriver); 
   if GraphResult<>grok then 
     halt; ..... { whatever you need } 
   CloseGraph; { restores the old graphics mode } 
end.
\end{verbatim}

\begin{function}{InstallUserDriver}
\Declaration
Function InstallUserDriver (DriverPath : String; \\AutoDetectPtr: Pointer) : Integer;

\Description
\var{InstallUserDriver} 
adds the device-driver \var{DriverPath} to the list of .BGI
drivers. \var{AutoDetectPtr} is a pointer to a possible auto-detect function.
\Errors
None.
\SeeAlso
\seep{InitGraph}, \seef{InstallUserFont}
\end{function}

\begin{function}{InstallUserFont}
\Declaration
Function InstallUserFont (FontPath : String) : Integer;

\Description
\var{InstallUserFont} adds the font in \var{FontPath} to the list of fonts
of the .BGI system.
\Errors
None.
\SeeAlso
\seep{InitGraph}, \seef{InstallUserDriver}
\end{function}

\begin{procedure}{Line}
\Declaration
Procedure Line (X1,Y1,X2,Y2 : Integer);

\Description
\var{Line} draws a line starting from
\var{(X1,Y1} to \var{(X2,Y2)}, in the current line style and color. The
current position is put to \var{(X2,Y2)}
\Errors
None.
\SeeAlso
\seep{LineRel},\seep{LineTo}
\end{procedure}

\begin{procedure}{LineRel}
\Declaration
Procedure LineRel (DX,DY : Integer);

\Description
\var{LineRel} draws a line starting from
the current pointer position to the point\var{(DX,DY}, \textbf{relative} to the
current position, in the current line style and color. The Current Position
is set to the endpoint of the line.
\Errors
None.
\SeeAlso
\seep{Line}, \seep{LineTo}
\end{procedure}

\begin{procedure}{LineTo}
\Declaration
Procedure LineTo (DX,DY : Integer);

\Description
\var{LineTo} draws a line starting from
the current pointer position to the point\var{(DX,DY}, \textbf{relative} to the
current position, in the current line style and color. The Current position
is set to the end of the line.
\Errors
None.
\SeeAlso
\seep{LineRel},\seep{Line}
\end{procedure}

\begin{procedure}{MoveRel}
\Declaration
Procedure MoveRel (DX,DY : Integer;

\Description
\var{MoveRel} moves the pointer to the
point \var{(DX,DY)}, relative to the current pointer
position
\Errors
None.
\SeeAlso
\seep{MoveTo}
\end{procedure}

\begin{procedure}{MoveTo}
\Declaration
Procedure MoveTo (X,Y : Integer;

\Description
\var{MoveTo} moves the pointer to the
point \var{(X,Y)}.
\Errors
None.
\SeeAlso
\seep{MoveRel}
\end{procedure}

\begin{procedure}{OutText}
\Declaration
Procedure OutText (Const TextString : String);

\Description
\var{OutText} puts \var{TextString} on the screen, at the current pointer
position, using the current font and text settings. The current position is
moved to the end of the text.
\Errors
None.
\SeeAlso
\seep{OutTextXY}
\end{procedure}

\begin{procedure}{OutTextXY}
\Declaration
Procedure OutTextXY (X,Y : Integer; Const TextString : String);

\Description
\var{OutText} puts \var{TextString} on the screen, at position \var{(X,Y)},
using the current font and text settings. The current position is
moved to the end of the text.
\Errors
None.
\SeeAlso
\seep{OutText}
\end{procedure}

\begin{procedure}{PieSlice}
\Declaration
Procedure PieSlice (X,Y : Integer; \\ Start,Stop,Radius : Word);

\Description
\var{PieSlice}
draws and fills a sector of a circle with center \var{(X,Y)} and radius 
\var{Radius}, starting at angle \var{Start} and ending at angle \var{Stop}.
\Errors
None.
\SeeAlso
\seep{Arc}, \seep{Circle}, \seep{Sector}
\end{procedure}

\begin{procedure}{PutImage}
\Declaration
Procedure PutImage (X1,Y1 : Integer; Var Bitmap; How : word) ;

\Description
\var{PutImage}
Places the bitmap in \var{Bitmap} on the screen at \var{(X1,Y1)}. \var{How}
determines how the bitmap will be placed on the screen. Possible values are :

\begin{itemize}
\item CopyPut
\item XORPut
\item ORPut
\item AndPut
\item NotPut
\end{itemize}
\Errors
None
\SeeAlso
\seef{ImageSize},\seep{GetImage}
\end{procedure}

\begin{procedure}{PutPixel}
\Declaration
Procedure PutPixel (X,Y : Integer; Color : Word);

\Description
Puts a point at
\var{(X,Y)} using color \var{Color}
\Errors
None.
\SeeAlso
\seef{GetPixel}
\end{procedure}

\begin{procedure}{Rectangle}
\Declaration
Procedure Rectangle (X1,Y1,X2,Y2 : Integer);

\Description
Draws a rectangle with
corners at \var{(X1,Y1)} and \var{(X2,Y2)}, using the current color and
style.
\Errors
None.
\SeeAlso
\seep{Bar}, \seep{Bar3D}
\end{procedure}

\begin{function}{RegisterBGIDriver}
\Declaration
Function RegisterBGIDriver (Driver : Pointer) : Integer;

\Description
Registers a user-defined BGI driver
\Errors
None.
\SeeAlso
\seef{InstallUserDriver},
\seef{RegisterBGIFont}
\end{function}

\begin{function}{RegisterBGIFont}
\Declaration
Function RegisterBGIFont (Font : Pointer) : Integer;

\Description
Registers a user-defined BGI driver
\Errors
None.
\SeeAlso
\seef{InstallUserFont},
\seef{RegisterBGIDriver}
\end{function}

\begin{procedure}{RestoreCRTMode}
\Declaration
Procedure RestoreCRTMode ;

\Description
Restores the screen modus which was active before
the graphical modus was started.

To get back to the graph mode you were last in, you can use
\var{SetGraphMode(GetGraphMode)}
\Errors
None.
\SeeAlso
\seep{InitGraph}
\end{procedure}

\begin{procedure}{Sector}
\Declaration
Procedure Sector (X,Y : Integer; \\ Start,Stop,XRadius,YRadius : Word);

\Description
\var{Sector}
draws and fills a sector of an ellipse  with center \var{(X,Y)} and radii 
\var{XRadius} and \var{YRadius}, starting at angle \var{Start} and ending at angle \var{Stop}.
\Errors
None.
\SeeAlso
\seep{Arc}, \seep{Circle}, \seep{PieSlice}
\end{procedure}

\begin{procedure}{SetActivePage}
\Declaration
Procedure SetActivePage (Page : Word);

\Description
Sets \var{Page} as the active page 
for all graphical output.
\Errors
None.
\SeeAlso

\end{procedure}

\begin{procedure}{SetAllPallette}
\Declaration
Procedure SetAllPallette (Var Palette);

\Description
Sets the current palette to
\var{Palette}. \var{Palette} is an untyped variable, usually pointing to a
record of type \var{PaletteType}
\Errors
None.
\SeeAlso
\seep{GetPalette}
\end{procedure}

\begin{procedure}{SetAspectRatio}
\Declaration
Procedure SetAspectRatio (Xasp,Yasp : Word);

\Description
Sets the aspect ratio of the
current screen to \var{Xasp/Yasp}.
\Errors
None
\SeeAlso
\seep{InitGraph}, \seep{GetAspectRatio}
\end{procedure}

\begin{procedure}{SetBkColor}
\Declaration
Procedure SetBkColor (Color : Word);

\Description
Sets the background color to
\var{Color}.
\Errors
None.
\SeeAlso
\seef{GetBkColor}, \seep{SetColor}
\end{procedure}

\begin{procedure}{SetColor}
\Declaration
Procedure SetColor (Color : Word);

\Description
Sets the foreground color to
\var{Color}.
\Errors
None.
\SeeAlso
\seef{GetColor}, \seep{SetBkColor}
\end{procedure}

\begin{procedure}{SetFillPattern}
\Declaration
Procedure SetFillPattern (FillPattern : FillPatternType,\\ Color : Word);

\Description
\var{SetFillPattern} sets the current fill-pattern to \var{FillPattern}, and
the filling color to \var{Color}
The pattern is an 8x8 raster, corresponding to the 64 bits in
\var{FillPattern}.
\Errors
None
\SeeAlso
\seep{GetFillPattern}, \seep{SetFillStyle}
\end{procedure}

\begin{procedure}{SetFillStyle}
\Declaration
Procedure SetFillStyle (Pattern,Color : word);

\Description
\var{SetFillStyle} sets the filling pattern and color to one of the
predefined filling patterns. \var{Pattern} can be one of the following predefined
constants :

\begin{itemize}
\item \var{EmptyFill  } Uses backgroundcolor.
\item \var{SolidFill  } Uses filling color
\item \var{LineFill   } Fills with horizontal lines.
\item \var{ltSlashFill} Fills with lines from left-under to top-right.
\item \var{SlashFill  } Idem as previous, thick lines.
\item \var{BkSlashFill} Fills with thick lines from left-Top to bottom-right.
\item \var{LtBkSlashFill} Idem as previous, normal lines.
\item \var{HatchFill}  Fills with a hatch-like pattern.
\item \var{XHatchFill} Fills with a hatch pattern, rotated 45 degrees.
\item \var{InterLeaveFill} 
\item \var{WideDotFill} Fills with dots, wide spacing.
\item \var{CloseDotFill} Fills with dots, narrow spacing.
\item \var{UserFill} Fills with a user-defined pattern.
\end{itemize}

\Errors
None.
\SeeAlso
\seep{SetFillPattern}
\end{procedure}

\begin{procedure}{SetGraphBufSize}
\Declaration
Procedure SetGraphBufSize (BufSize : Word);

\Description
\var{SetGraphBufSize} is a dummy function which does not do 
anything; it is no longer needed.
\Errors
None.
\SeeAlso

\end{procedure}

\begin{procedure}{SetGraphMode}
\Declaration
Procedure SetGraphMode (Mode : Integer);

\Description
\var{SetGraphMode} sets the graphical mode and clears the screen.
\Errors
None.
\SeeAlso
\seep{InitGraph}
\end{procedure}

\begin{procedure}{SetLineStyle}
\Declaration
Procedure SetLineStyle (LineStyle,Pattern,Width :
Word);

\Description
\var{SetLineStyle}
sets the drawing style for lines. You can specify a \var{LineStyle} which is
one of the following pre-defined constants:

\begin{itemize}
\item \var{Solidln=0;} draws a solid line.
\item \var{Dottedln=1;} Draws a dotted line.
\item \var{Centerln=2;} draws a non-broken centered line.
\item \var{Dashedln=3;} draws a dashed line.
\item \var{UserBitln=4;} Draws a User-defined bit pattern.
\end{itemize}
If \var{UserBitln} is specified then \var{Pattern} contains the bit pattern.
In all another cases, \var{Pattern} is ignored. The parameter \var{Width} 
indicates how thick the line should be. You can specify one of the following
pre-defined constants:

\begin{itemize}
\item \var{NormWidth=1}
\item \var{ThickWidth=3}
\end{itemize}

\Errors
None.
\SeeAlso
\seep{GetLineSettings}
\end{procedure}

\begin{procedure}{SetPalette}
\Declaration
Procedure SetPalette (ColorNr : Word; NewColor : ShortInt);

\Description
\var{SetPalette} changes the \var{ColorNr}-th entry in the palette to
\var{NewColor}
\Errors
None.
\SeeAlso
\seep{SetAllPallette},\seep{SetRGBPalette}
\end{procedure}

\begin{procedure}{SetRGBPalette}
\Declaration
Procedure SetRGBPalette (ColorNr,Red,Green,Blue : Integer);

\Description
\var{SetRGBPalette} sets the \var{ColorNr}-th entry in the palette to the
color with RGB-values \var{Red, Green Blue}.
\Errors
None.
\SeeAlso
\seep{SetAllPallette},
\seep{SetPalette}
\end{procedure}

\begin{procedure}{SetTextJustify}
\Declaration
Procedure SetTextJustify (Horizontal,Vertical : Word);

\Description
\var{SetTextJustify} controls the placement of new text, relative to the 
(graphical) cursor position. \var{Horizontal} controls horizontal placement, and can be
one of the following pre-defined constants:

\begin{itemize}
\item \var{LeftText=0;} Text is set left of the pointer.
\item \var{CenterText=1;} Text is set centered horizontally on the pointer.
\item \var{RightText=2;} Text is set to the right of the pointer.
\end{itemize}
\var{Vertical} controls the vertical placement of the text, relative to the
(graphical) cursor position. Its value can be one of the following
pre-defined constants :

\begin{itemize}
\item \var{BottomText=0;} Text is placed under the pointer.
\item \var{CenterText=1;} Text is placed centered vertically on the pointer.
\item \var{TopText=2;}Text is placed above the pointer.
\end{itemize}

\Errors
None.
\SeeAlso
\seep{OutText}, \seep{OutTextXY}
\end{procedure}

\begin{procedure}{SetTextStyle}
\Declaration
Procedure SetTextStyle (Font,Direction,Magnitude : Word);

\Description
\var{SetTextStyle} controls the style of text to be put on the screen.
pre-defined constants for \var{Font} are:

\begin{itemize}
\item \var{DefaultFont=0;}
\item \var{TriplexFont=2;}
\item \var{SmallFont=2;}
\item \var{SansSerifFont=3;}
\item \var{GothicFont=4;}
\end{itemize}
Pre-defined constants for \var{Direction} are :

\begin{itemize}
\item \var{HorizDir=0;}
\item \var{VertDir=1;}
\end{itemize}
\Errors
None.
\SeeAlso
\seep{GetTextSettings} 
\end{procedure}

\begin{procedure}{SetUserCharSize}
\Declaration
Procedure SetUserCharSize (Xasp1,Xasp2,Yasp1,Yasp2 : Word);

\Description
Sets the width and height of vector-fonts. The horizontal size is given
by \var{Xasp1/Xasp2}, and the vertical size by \var{Yasp1/Yasp2}.
\Errors
None.
\SeeAlso
\seep{SetTextStyle}
\end{procedure}

\begin{procedure}{SetViewPort}
\Declaration
Procedure SetViewPort (X1,Y1,X2,Y2 : Integer; Clip : Boolean);

\Description
Sets the current graphical viewport (window) to the rectangle defined by
the top-left corner \var{(X1,Y1)} and the bottom-right corner \var{(X2,Y2)}.
If \var{Clip} is true, anything drawn outside the viewport (window) will be
clipped (i.e. not drawn). Coordinates specified after this call are relative
to the top-left corner of the viewport.
\Errors
None.
\SeeAlso
\seep{GetViewSettings}
\end{procedure}

\begin{procedure}{SetVisualPage}
\Declaration
Procedure SetVisualPage (Page : Word);

\Description
\var{SetVisualPage} sets the video page to page number \var{Page}. 
\Errors
None
\SeeAlso
\seep{SetActivePage}
\end{procedure}

\begin{procedure}{SetWriteMode}
\Declaration
Procedure SetWriteMode (Mode : Integer);

\Description
\var{SetWriteMode} controls the drawing of lines on the screen. It controls
the binary operation used when drawing lines on the screen. \var{Mode} can
be one of the following pre-defined constants:

\begin{itemize}
\item CopyPut=0;
\item XORPut=1;
\end{itemize}
\Errors
None.
\SeeAlso

\end{procedure}

\begin{function}{TextHeight}
\Declaration
Function TextHeight (S : String) : Word;

\Description
\var{TextHeight} returns the height (in pixels) of the string \var{S} in
the current font and text-size.

\Errors
None.
\SeeAlso
\seef{TextWidth}
\end{function}

\begin{function}{TextWidth}
\Declaration
Function TextWidth (S : String) : Word;

\Description
\var{TextHeight} returns the width (in pixels) of the string \var{S} in
the current font and text-size.
\Errors
None.
\SeeAlso
\seef{TextHeight}
\end{function}

% Target specific issues.
%%%%%%%%%%%%%%%%%%%%%%%%%%%%%%%%%%%%%%%%%%%%%%%%%%%%%%%%%%%%%%%%%%%%%%%
\section{Target specific issues}                                                                                                                                                                                                                                                               
In what follows we describe some things that are different on the various
platforms:

\subsection{\dos}
\subsection{\windows}
\subsection{\linux}


% The HeapTrc unit
%
%   $Id$
%   This file is part of the FPC documentation.
%   Copyright (C) 1998, by Michael Van Canneyt
%
%   The FPC documentation is free text; you can redistribute it and/or
%   modify it under the terms of the GNU Library General Public License as
%   published by the Free Software Foundation; either version 2 of the
%   License, or (at your option) any later version.
%
%   The FPC Documentation is distributed in the hope that it will be useful,
%   but WITHOUT ANY WARRANTY; without even the implied warranty of
%   MERCHANTABILITY or FITNESS FOR A PARTICULAR PURPOSE.  See the GNU
%   Library General Public License for more details.
%
%   You should have received a copy of the GNU Library General Public
%   License along with the FPC documentation; see the file COPYING.LIB.  If not,
%   write to the Free Software Foundation, Inc., 59 Temple Place - Suite 330,
%   Boston, MA 02111-1307, USA. 
%
\chapter{The HEAPTRC unit.}
This chapter describes the HEAPTRC unit for \fpc. It was written by 
Pierre Muller.

\section{Purpose}

The HEAPTRC unit can be used to debug your memory allocation/deallocation.
It keeps track of the calls to getmem/freemem, and, implicitly, of
New/Dispose statements.

When the program exits, or when you request it explicitly.
It displays the total memory used, and then dumps a list of blocks that
were allocated but not freed. It also displays where the memory was
allocated.

If there are any inconsistencies, such as memory blocks being allocated
or freed twice, or a memory block that is released but with wrong size,
this will be displayed also.

The information that is stored/displayed can be customized using
some constants.

\section{Usage}

All that you need to do is to include \file{heaptrc} in the uses clause
of your program. Make sure that it is the first unit in the clause,
otherwise memory allocated in initialization code of units that precede the
heaptrc unit will not be accounted for, causing an incorrect memory usage
report.

The following example shows how to use the heaptrc unit.

\latex{\inputlisting{heapex/heapex.pp}}
\html{\input{heapex/heapex.tex}}

This is the memory dump shown when running this program:
\begin{verbatim}
Marked memory at 08052C48 invalid
Wrong size : 128 allocated 64 freed
  0x0804C29C
  0x080509E2
  0x080480A4
  0x00000000
Heap dump by heaptrc unit
13 memory blocks allocated : 1416/1424
6 memory blocks freed     : 708/712
7 unfreed memory blocks : 708
True heap size : 2097152
True free heap : 2096040
Should be : 2096104
Call trace for block 0x08052C48 size 128
  0x080509D6
  0x080480A4
Call trace for block 0x08052B98 size 128
  0x08050992
  0x080480A4
Call trace for block 0x08052AE8 size 128
  0x08050992
  0x080480A4
Call trace for block 0x08052A38 size 128
  0x08050992
  0x080480A4
Call trace for block 0x08052988 size 128
  0x08050992
  0x080480A4
Call trace for block 0x080528D8 size 128
  0x08050992
  0x080480A4
Call trace for block 0x080528A0 size 4
  0x08050961
  0x080480A4
\end{verbatim}

\section{Constants, Types and variables}

The \var{FillExtraInfoType} is a procedural type used in the
\seep{SetExtraInfo} call.

\begin{listing}
type
    FillExtraInfoType = procedure(p : pointer);
\end{listing}
The following typed constants allow to fine-tune the standard dump of the
memory usage by \seep{DumpHeap}:

\begin{listing}
const
  tracesize = 8;
  quicktrace : boolean = true;
  HaltOnError : boolean = true;
  keepreleased : boolean = false;
\end{listing}

\var{Tracesize} specifies how many levels of calls are displayed of the 
call stack during the memory dump. If you specify \var{keepreleased:=True}
then half the \var{TraceSize} is reserved for the \var{GetMem} call stack, 
and the other half is reserved for the \var{FreeMem} call stack.
For example, the default value of 8 will cause eight levels of call frames
to be dumped for the getmem call if \var{keepreleased} is \var{False}. If
\var{KeepReleased} is true, then 4 levels of call frames will be dumped for
the \var{GetMem} call and 4 frames wil be dumped for the \var{FreeMem} call.
If you want to change this value, you must recode the \file{heaptrc} unit.

\var{Quicktrace} determines whether the memory manager checks whether a 
block that is about to be released is allocated correctly. This is a rather
time consuming search, and slows program execution significantly, so by
default it is set to \var{False}.

If \var{HaltOnError} is set to \var{True} then an illegal call to 
\var{FreeMem} will cause the memory manager to execute a \var{halt(1)} 
instruction, causing a memory dump. By Default it is set to \var{True}.

If \var{keepreleased} is set to true, then a list of freed memory 
blocks is kept. This is useful if you suspect that the same memory block is
released twice. However, this option is very memory intensive, so use it
sparingly, and only when it's really necessary.

\section{Functions and procedures}

\begin{procedure}{DumpHeap}
\Declaration 
procedure DumpHeap;
\Description
\var{DumpHeap} dumps to standard output a summary of memory usage.
It is called automatically by the heaptrc unit when your program exits
(by instaling an exit procedure), but it can be called at any time
\Errors
None.
\SeeAlso
\seep{MarkHeap}
\end{procedure}

\begin{procedure}{MarkHeap}
\Declaration
procedure MarkHeap;
\Description
\var{MarkHeap} marks all memory blocks with a special signature.
You can use this if you think that you corruped the memory.
\Errors
None.
\SeeAlso
\seep{DumpHeap}
\end{procedure}

\begin{procedure}{SetExtraInfo}
\Declaration
procedure SetExtraInfo( size : longint;func : FillExtraInfoType);
\Description
You can use \var{SetExtraInfo} to store extra info in the blocks that
the heaptrc unit reserves when tracing getmem calls. \var{Size} indicates the
size (in bytes) that the trace mechanism should reserve for your extra
information. For each call to \var{getmem}, \var{func} will be called,
and passed a pointer to the memory reserved. 

When dumping the memory summary, the extra info is shown as Longint values.

\Errors
You can only call \var{SetExtraInfo} if no memroy has been allocated
yet. If memory was already allocated prior to the call to
\var{SetExtraInfo}, then an error will be displayed on standard error
output, and a \seep{DumpHeap} is executed.
\SeeAlso
\seep{DumpHeap}
\end{procedure}

\latex{\inputlisting{heapex/setinfo.pp}}
\html{\input{heapex/setinfo.tex}}


%
% $Log$
% Revision 1.2  1998-12-15 23:50:52  michael
% * Some updates
%
% Revision 1.1  1998/12/14 23:17:02  michael
% + INitial implementation
%
%
% The IPC unit
%
%   $Id$
%   This file is part of the FPC documentation.
%   Copyright (C) 1998, by Michael Van Canneyt
%
%   The FPC documentation is free text; you can redistribute it and/or
%   modify it under the terms of the GNU Library General Public License as
%   published by the Free Software Foundation; either version 2 of the
%   License, or (at your option) any later version.
%
%   The FPC Documentation is distributed in the hope that it will be useful,
%   but WITHOUT ANY WARRANTY; without even the implied warranty of
%   MERCHANTABILITY or FITNESS FOR A PARTICULAR PURPOSE.  See the GNU
%   Library General Public License for more details.
%
%   You should have received a copy of the GNU Library General Public
%   License along with the FPC documentation; see the file COPYING.LIB.  If not,
%   write to the Free Software Foundation, Inc., 59 Temple Place - Suite 330,
%   Boston, MA 02111-1307, USA. 
%
\chapter{The IPC unit.}
This chapter describes the IPC unit for Free Pascal. It was written for
\linux by Micha\"el Van Canneyt. It gives all the functionality of system V 
Inter-Process Communication: shared memory, semaphores and messages.
It works only on the \linux operating system.

The chapter is divided in 2 sections:
\begin{itemize}
\item The first section lists types, constants and variables from the
interface part of the unit.
\item The second section describes the functions defined in the unit.
\end{itemize}
\section {Types, Constants and variables : }
\subsection{Variables}

\begin{verbatim}
Var
  IPCerror : longint;
\end{verbatim}
The \var{IPCerror} variable is used to report errors, by all calls.
\subsection{Constants}

\begin{verbatim}
Const 
  IPC_CREAT  =  1 shl 9;  { create if key is nonexistent }
  IPC_EXCL   =  2 shl 9;  { fail if key exists }
  IPC_NOWAIT =  4 shl 9;  { return error on wait }
\end{verbatim}
These constants are used in the various \var{xxxget} calls.
\begin{verbatim}
  IPC_RMID = 0;     { remove resource }
  IPC_SET  = 1;     { set ipc_perm options }
  IPC_STAT = 2;     { get ipc_perm options }
  IPC_INFO = 3;     { see ipcs }
\end{verbatim}
These constants can be passed to the various \var{xxxctl} calls.
\begin{verbatim}
const
  MSG_NOERROR = 1 shl 12;
  MSG_EXCEPT  = 2 shl 12;
  MSGMNI = 128;
  MSGMAX = 4056;
  MSGMNB = 16384;
\end{verbatim}
These constants are used in the messaging system, they are not for use by
the programmer.
\begin{verbatim}
const
  SEM_UNDO = $1000;
  GETPID = 11;
  GETVAL = 12;
  GETALL = 13;
  GETNCNT = 14;
  GETZCNT = 15;
  SETVAL = 16;
  SETALL = 17;
\end{verbatim}
These constants call be specified in the \seef{semop} call.
\begin{verbatim}
  SEMMNI = 128;
  SEMMSL = 32;
  SEMMNS = (SEMMNI * SEMMSL);
  SEMOPM = 32;
  SEMVMX = 32767;
\end{verbatim}
These constanst are used internally by the semaphore system, they should not
be used by the programmer.
\begin{verbatim}
const
  SHM_R      = 4 shl 6;
  SHM_W      = 2 shl 6;
  SHM_RDONLY = 1 shl 12;
  SHM_RND    = 2 shl 12;
  SHM_REMAP  = 4 shl 12;
  SHM_LOCK   = 11;
  SHM_UNLOCK = 12;
\end{verbatim}
These constants are used in the \seef{shmctl} call.

\subsection{Types}

\begin{verbatim}
Type 
   TKey   = Longint;
\end{verbatim}
\var{TKey} is the type returned by the \seef{ftok} key generating function.
\begin{verbatim}
type
  PIPC_Perm = ^TIPC_Perm;
  TIPC_Perm = record
    key : TKey;
    uid, 
    gid,
    cuid,
    cgid,
    mode,
    seq : Word;   
  end;
\end{verbatim}
The \var{TIPC\_Perm} structure is used in all IPC systems to specify the
permissions.
\begin{verbatim}
Type  
  PSHMid_DS = ^TSHMid_ds; 
  TSHMid_ds = record
    shm_perm  : TIPC_Perm;
    shm_segsz : longint;
    shm_atime : longint;
    shm_dtime : longint;
    shm_ctime : longint;
    shm_cpid  : word;
    shm_lpid  : word;
    shm_nattch : integer;
    shm_npages : word;
    shm_pages  : Pointer;
    attaches   : pointer;
  end;
\end{verbatim}
The \var{TSHMid\_ds} strucure is used in the \seef{shmctl} call to set or
retrieve settings concerning shared memory.
\begin{verbatim}
type
  PSHMinfo = ^TSHMinfo;
  TSHMinfo = record
    shmmax : longint;
    shmmin : longint;
    shmmni : longint;
    shmseg : longint;
    shmall : longint;
  end;
\end{verbatim}
The \var{TSHMinfo} record is used by the shared memory system, and should
not be accessed by the programer directly.
\begin{verbatim}
type
  PMSG = ^TMSG;
  TMSG = record
    msg_next  : PMSG;
    msg_type  : Longint;
    msg_spot  : PChar;
    msg_stime : Longint;
    msg_ts    : Integer;
  end;
\end{verbatim}
The \var{TMSG} record is used in the handling of message queues. There
should be few cases where the programmer needs to access this data.
\begin{verbatim}
type
  PMSQid_ds = ^TMSQid_ds;
  TMSQid_ds = record
    msg_perm   : TIPC_perm;
    msg_first  : PMsg;
    msg_last   : PMsg;
    msg_stime  : Longint;
    msg_rtime  : Longint;
    msg_ctime  : Longint;
    wwait      : Pointer;
    rwait      : pointer;
    msg_cbytes : word;
    msg_qnum   : word;
    msg_qbytes : word;
    msg_lspid  : word;
    msg_lrpid  : word;
  end;
\end{verbatim}
The \var{TMSQid\_ds} record is returned by the \seef{msgctl} call, and
contains all data about a message queue.
\begin{verbatim}
  PMSGbuf = ^TMSGbuf;
  TMSGbuf = record
    mtype : longint;
    mtext : array[0..0] of char;
  end;
\end{verbatim}
The \var{TMSGbuf} record is a record containing the data of a record. you
should never use this record directly, instead you should make your own
record that follows the structure of the \var{TMSGbuf} record, but that has
a size that is big enough to accomodate your messages. The \var{mtype} field
should always be present, and should always be filled.
\begin{verbatim}
Type
  PMSGinfo = ^TMSGinfo;
  TMSGinfo = record
    msgpool : Longint;
    msgmap  : Longint;
    msgmax  : Longint;
    msgmnb  : Longint;
    msgmni  : Longint;
    msgssz  : Longint;
    msgtql  : Longint;
    msgseg  : Word;
  end;
\end{verbatim}
The  \var{TMSGinfo} record is used internally by the message queue system,
and should not be used by the programmer directly.
\begin{verbatim}
Type
  PSEMid_ds = ^PSEMid_ds;
  TSEMid_ds = record
    sem_perm : tipc_perm;
    sem_otime : longint;
    sem_ctime : longint;
    sem_base         : pointer;
    sem_pending      : pointer;
    sem_pending_last : pointer;
    undo             : pointer;
    sem_nsems : word;
  end;
\end{verbatim}
The \var{TSEMid\_ds} structure is returned by the \seef{semctl} call, and
contains all data concerning a semahore.
\begin{verbatim}
Type
  PSEMbuf = ^TSEMbuf;
  TSEMbuf = record
    sem_num : word;
    sem_op  : integer;
    sem_flg : integer;
  end;
\end{verbatim}
The \var{TSEMbuf} record us use in the \seef{semop} call, and is used to
specify which operations you want to do.
\begin{verbatim}
Type
  PSEMinfo = ^TSEMinfo;
  TSEMinfo = record
    semmap : longint;
    semmni : longint;
    semmns : longint;
    semmnu : longint;
    semmsl : longint;
    semopm : longint;
    semume : longint;
    semusz : longint;
    semvmx : longint;
    semaem : longint;
  end;
\end{verbatim}
The \var{TSEMinfo} record is used internally by the semaphore system, and
should not be used diirectly.
\begin{verbatim}
Type
  PSEMun = ^TSEMun;
  TSEMun = record
   case longint of
      0 : ( val : longint );
      1 : ( buf : PSEMid_ds );
      2 : ( arr : PWord );
      3 : ( padbuf : PSeminfo );
      4 : ( padpad : pointer );
   end;
\end{verbatim}
The \var{TSEMun} variant record (actually a C union) is used in the
\seef{semctl} call.
 
\section{Functions and procedures}

\begin{function}{ftok}
\Declaration
Function ftok (Path : String; ID : char) : TKey;
\Description
\var{ftok} returns a key that can be used in a \seef{semget},\seef{shmget}
or \seef{msgget} call to access a new or existing IPC resource.

\var{Path} is the name of a file in the file system, \var{ID} is a
character of your choice. The ftok call does the same as it's C couterpart,
so a pascal program and a C program will access the same resource if
they use the same \var{Path} and \var{ID}
\Errors
\var{ftok} returns -1 if the file in \var{Path} doesn't exist.
\SeeAlso
\seef{semget},\seef{shmget},\seef{msgget}
\end{function}

For an example, see \seef{msgctl}, \seef{semctl}, \seef{shmctl}.

\begin{function}{msgget}
\Declaration
Function msgget(key: TKey; msgflg:longint):longint;	
\Description
\var{msgget} returns the ID of the message queue described by \var{key}.
Depending on the flags in \var{msgflg}, a new queue is created.

\var{msgflg} can have one or more of the following values (combined by ORs):
\begin{description}
\item[IPC\_CREAT] The queue is created if it doesn't already exist.
\item[IPC\_EXCL] If used in combination with \var{IPC\_CREAT}, causes the
call to fail if the queue already exists. It cannot be used by itself.
\end{description}
Optionally, the flags can be \var{OR}ed with a permission mode, which is the
same mode that can be used in the file system.
\Errors
On error, -1 is returned, and \var{IPCError} is set.
\SeeAlso
\seef{ftok},\seef{msgsnd}, \seef{msgrcv}, \seef{msgctl}, \seem{semget}{2}
\end{function}

For an example, see \seef{msgctl}.

\begin{function}{msgsnd}
\Declaration
Function msgsnd(msqid:longint; msgp: PMSGBuf; msgsz: longint; msgflg:longint): Boolean;
\Description
\var{msgsend} sends a message to a message queue with ID \var{msqid}.
\var{msgp} is a pointer to a message buffer, that should be based on the
\var{TMsgBuf} type. \var{msgsiz} is the size of the message (NOT of the
message buffer record !)

The \var{msgflg} can have a combination of the following values (ORed
together):
\begin{description}
\item [0] No special meaning. The message will be written to the queue.
If the queue is full, then the process is blocked.
\item [IPC\_NOWAIT] If the queue is full, then no message is written,
and the call returns immediatly.
\end{description}

The function returns \var{True} if the message was sent successfully, 
\var{False} otherwise.
\Errors
In case of error, the call returns \var{False}, and \var{IPCerror} is set.
\SeeAlso
\seef{msgget}, \seef{msgrcv}, seef{msgctl}
\end{function}

For an example, see \seef{msgctl}.

\begin{function}{msgrcv}
\Declaration
Function msgrcv(msqid:longint; msgp: PMSGBuf; msgsz: longint; msgtyp:longint; msgflg:longint): Boolean;
\Description
\var{msgrcv} retrieves a message of type \var{msgtyp} from the message 
queue with ID \var{msqid}. \var{msgtyp} corresponds to the \var{mtype} 
field of the \var{TMSGbuf} record. The message is stored in the \var{MSGbuf}
structure pointed to by \var{msgp}.

The \var{msgflg} parameter can be used to control the behaviour of the
\var{msgrcv} call. It consists of an ORed combination of the following
flags:
\begin{description}
\item [0] No special meaning.
\item [IPC\_NOWAIT] if no messages are available, then the call returns
immediatly, with the \var{ENOMSG} error.
\item [MSG\_NOERROR] If the message size is wrong (too large), 
no error is generated, instead the message is truncated. 
Normally, in such cases, the call returns an error (E2BIG)
\end{description}

The function returns \var{True} if the message was received correctly,
\var{False} otherwise.
\Errors
In case of error, \var{False} is returned, and \var{IPCerror} is set.
\SeeAlso
\seef{msgget}, \seef{msgsnd}, \seef{msgctl}
\end{function}

For an example, see \seef{msgctl}.

\begin{function}{msgctl}
\Declaration
Function msgctl(msqid:longint; cmd: longint; buf: PMSQid\_ds): Boolean;
\Description
\var{msgctl} performs various operations on the message queue with id
\var{ID}. Which operation is performed, depends on the \var{cmd} 
parameter, which can have one of the following values:
\begin{description}
\item[IPC\_STAT] In this case, the \var{msgctl} call fills the
\var{TMSQid\_ds} structure with information about the message queue.
\item[IPC\_SET] in this case, the \var{msgctl} call sets the permissions
of the queue as specified in the \var{ipc\_perm} record inside \var{buf}.
\item[IPC\_RMID] If this is specified, the message queue will be removed 
from the system.
\end{description}

\var{buf} contains the data that are needed by the call. It can be 
\var{Nil} in case the message queue should be removed.

The function returns \var{True} if successfull, \var{False} otherwise.
\Errors
On error, \var{False} is returned, and \var{IPCerror} is set accordingly.
\SeeAlso
\seef{msgget}, \seef{msgsnd}, \seef{msgrcv}
\end{function}

\latex{\lstinputlisting{ipcex/msgtool.pp}}
\html{\input{ipcex/msgtool.tex}}

\begin{function}{semget}
\Declaration
Function semget(key:Tkey; nsems:longint; semflg:longint): longint;
\Description
\var{msgget} returns the ID of the semaphore set described by \var{key}.
Depending on the flags in \var{semflg}, a new queue is created.

\var{semflg} can have one or more of the following values (combined by ORs):
\begin{description}
\item[IPC\_CREAT] The queue is created if it doesn't already exist.
\item[IPC\_EXCL] If used in combination with \var{IPC\_CREAT}, causes the
call to fail if the set already exists. It cannot be used by itself.
\end{description}
Optionally, the flags can be \var{OR}ed with a permission mode, which is the
same mode that can be used in the file system.

if a new set of semaphores is created, then there will be \var{nsems}
semaphores in it.
\Errors
On error, -1 is returned, and \var{IPCError} is set.
\SeeAlso
\seef{ftok}, \seef{semop}, \seef{semctl}
\end{function}

\begin{function}{semop}
\Declaration
Function semop(semid:longint; sops: pointer; nsops: cardinal): Boolean;
\Description
\var{semop} performs a set of operations on a message queue.
\var{sops} points to an array of type \var{TSEMbuf}. The array should
contain \var{nsops} elements.

The fields of the \var{TSEMbuf} structure 
\begin{verbatim}
  TSEMbuf = record
    sem_num : word;
    sem_op  : integer;
    sem_flg : integer;
\end{verbatim}

should be filled as follows:
\begin{description}
\item[sem\_num] The number of the semaphore in the set on which the
operation must be performed.
\item[sem\_op] The operation to be performed. The operation depends on the
sign of \var{sem\_op}
\begin{enumerate}
\item A positive  number is simply added to the current value of the
semaphore.
\item If 0 (zero) is specified, then the process is suspended until the 
  specified semaphore reaches zero.
\item If a negative number is specified, it is substracted from the
current value of the semaphore. If the value would become negative
then the process is suspended until the value becomes big enough, unless
\var{IPC\_NOWAIT} is specified in the \var{sem\_flg}.
\end{enumerate}
\item[sem\_flg] Optional flags: if \var{IPC\_NOWAIT} is specified, then the
calling process will never be suspended.
\end{description}

The function returns \var{True} if the operations were successful,
\var{False} otherwise.
\Errors
In case of error, \var{False} is returned, and \var{IPCerror} is set.
\SeeAlso
\seef{semget}, \seef{semctl}
\end{function}

\begin{function}{semctl}
\Declaration
Function semctl(semid:longint; semnum:longint; cmd:longint; var arg: tsemun): longint;
\Description
\var{semctl} performs various operations on the semaphore \var{semnum} w
ith semaphore set id \var{ID}. 

The \var{arg} parameter supplies the data needed for each call. This is
a variant record that should be filled differently, according to the
command:
\begin{verbatim}
Type
  TSEMun = record
   case longint of
      0 : ( val : longint );
      1 : ( buf : PSEMid_ds );
      2 : ( arr : PWord );
      3 : ( padbuf : PSeminfo );
      4 : ( padpad : pointer );
   end;
\end{verbatim}


Which operation is performed, depends on the \var{cmd} 
parameter, which can have one of the following values:
\begin{description}
\item[IPC\_STAT] In this case, the arg record should have it's \var{buf}
field set to the address of a \var{TSEMid\_ds} record. 
The \var{semctl} call fills this \var{TSEMid\_ds} structure with information 
about the semaphore set. 
\item[IPC\_SET] In this case, the \var{arg} record should have it's \var{buf}
field set to the address of a \var{TSEMid\_ds} record.
The \var{semctl} call sets the permissions of the queue as specified in 
the \var{ipc\_perm} record.
\item[IPC\_RMID] If this is specified, the semaphore set is removed from 
from the system.
\item[GETALL] In this case, the \var{arr} field of \var{arg} should point
to a memory area where the values of the semaphores will be stored.
The size of this memory area is \var{SizeOf(Word)* Number of semaphores
in the set}.
This call will then fill the memory array with all the values of the
semaphores.
\item[GETNCNT] This will fill the \var{val} field of the \var{arg} union
with the bumber of processes waiting for resources.
\item[GETPID] \var{semctl} returns the process ID of the process that
performed the last \seef{semop} call.
\item[GETVAL] \var{semctl} returns the value of the semaphore with number
\var{semnum}.
\item[GETZCNT] \var{semctl} returns the number of processes waiting for 
semaphores that reach value zero.
\item[SETALL] In this case, the \var{arr} field of \var{arg} should point
to a memory area where the values of the semaphores will be retrieved from.
The size of this memory area is \var{SizeOf(Word)* Number of semaphores
in the set}.
This call will then set the values of the semaphores from the memory array.
\item[SETVAL] This will set the value of semaphore \var{semnum} to the value
in the \var{val} field of the \var{arg} parameter.
\end{description}

The function returns -1 on error.
\Errors
The function returns -1 on error, and \var{IPCerror} is set accordingly.
\SeeAlso
\seef{semget}, \seef{semop}
\end{function}

\latex{\lstinputlisting{ipcex/semtool.pp}}
\html{\input{ipcex/semtool.tex}}


\begin{function}{shmget}
\Declaration
Function shmget(key: Tkey; Size:longint; flag:longint):longint;
\Description
\var{shmget} returns the ID of a shared memory block, described by \var{key}.
Depending on the flags in \var{flag}, a new memory block is created.

\var{flag} can have one or more of the following values (combined by ORs):
\begin{description}
\item[IPC\_CREAT] The queue is created if it doesn't already exist.
\item[IPC\_EXCL] If used in combination with \var{IPC\_CREAT}, causes the
call to fail if the queue already exists. It cannot be used by itself.
\end{description}
Optionally, the flags can be \var{OR}ed with a permission mode, which is the
same mode that can be used in the file system.

if a new memory block is created, then it will have size \var{Size}
semaphores in it.
\Errors
On error, -1 is returned, and \var{IPCError} is set.
\SeeAlso
\end{function}

\begin{function}{shmat}
\Declaration
Function shmat (shmid:longint; shmaddr:pchar; shmflg:longint):pchar;
\Description
\var{shmat} attaches a shared memory block with identified \var{shmid} 
to the current process. The function returns a pointer to the shared memory
block.

If \var{shmaddr} is \var{Nil}, then the system chooses a free unmapped
memory region, as high up in memory space as possible.

If \var{shmaddr} is non-nil, and \var{SHM\_RND} is in \var{shmflg}, then 
the returned address is \var{shmaddr}, rounded down to \var{SHMLBA}.
If \var{SHM\_RND} is not specified, then \var{shmaddr} must be a
page-aligned address.

The parameter \var{shmflg} can be used to control the behaviour of the
\var{shmat} call. It consists of a ORed combination of the following
costants:
\begin{description}
\item[SHM\_RND] The suggested address in \var{shmaddr} is rounded down to
\var{SHMLBA}.
\item[SHM\_RDONLY] the shared memory is attached for read access only.
Otherwise the memory is attached for read-write. The process then needs
read-write permissions to access the shared memory.
\end{description}
\Errors
If an error occurs, -1 is returned, and \var{IPCerror} is set.
\SeeAlso
\seef{shmget}, \seef{shmdt}, \seef{shmctl}
\end{function}

For an example, see \seef{shmctl}.

\begin{function}{shmdt}
\Declaration
Function shmdt (shmaddr:pchar):boolean;
\Description
\var{shmdt} detaches the shared memory at address \var{shmaddr}. This shared
memory block is unavailable to the current process, until it is attached
again by a call to \seef{shmat}.

The function returns \var{True} if the memory block was detached
successfully, \var{False} otherwise.
\Errors
On error, False is returned, and IPCerror is set.
\SeeAlso
\seef{shmget}, \seef{shmat}, \seef{shmctl}
\end{function}

\begin{function}{shmctl}
\Declaration
Function shmctl(shmid:longint; cmd:longint; buf: pshmid\_ds): Boolean;
\Description
\var{shmctl} performs various operations on the shared memory block
identified by identifier \var{shmid}.

The \var{buf} parameter points to a \var{TSHMid\_ds} record. The 
\var{cmd} parameter is used to pass which operation is to be performed.
It can have one of the following values :
\begin{description}
\item[IPC\_STAT] \var{shmctl} fills the \var{TSHMid\_ds} record that 
\var{buf} points to with the available information about the shared memory
block.
\item[IPC\_SET] applies the values in the \var{ipc\_perm} record that
\var{buf} points to, to the shared memory block.
\item[IPC\_RMID] the shared memory block is destroyed (after all processes
to which the block is attached, have detached from it).
\end{description}

If successful, the function returns \var{True}, \var{False} otherwise.
\Errors
If an error occurs, the function returns \var{False}, and \var{IPCerror}
is set.
\SeeAlso
\seef{shmget}, \seef{shmat}, \seef{shmdt}
\end{function}

\latex{\lstinputlisting{ipcex/shmtool.pp}}
\html{\input{ipcex/shmtool.tex}}

% The keyboard unit
%
%   $Id$
%   This file is part of the FPC documentation.
%   Copyright (C) 2001, by Michael Van Canneyt
%
%   The FPC documentation is free text; you can redistribute it and/or
%   modify it under the terms of the GNU Library General Public License as
%   published by the Free Software Foundation; either version 2 of the
%   License, or (at your option) any later version.
%
%   The FPC Documentation is distributed in the hope that it will be useful,
%   but WITHOUT ANY WARRANTY; without even the implied warranty of
%   MERCHANTABILITY or FITNESS FOR A PARTICULAR PURPOSE.  See the GNU
%   Library General Public License for more details.
%
%   You should have received a copy of the GNU Library General Public
%   License along with the FPC documentation; see the file COPYING.LIB.  If not,
%   write to the Free Software Foundation, Inc., 59 Temple Place - Suite 330,
%   Boston, MA 02111-1307, USA.
%
%%%%%%%%%%%%%%%%%%%%%%%%%%%%%%%%%%%%%%%%%%%%%%%%%%%%%%%%%%%%%%%%%%%%%%%
%
%%%%%%%%%%%%%%%%%%%%%%%%%%%%%%%%%%%%%%%%%%%%%%%%%%%%%%%%%%%%%%%%%%%%%%%
% The Keyboard unit
%%%%%%%%%%%%%%%%%%%%%%%%%%%%%%%%%%%%%%%%%%%%%%%%%%%%%%%%%%%%%%%%%%%%%%%
\chapter{The KEYBOARD unit}
\FPCexampledir{kbdex}

The \file{KeyBoard} unit implements a keyboard access layer which is system
independent. It can be used to poll the keyboard state and wait for certain
events. Waiting for a keyboard event can be done with the \seef{GetKeyEvent}
function, which will return a driver-dependent key event. This key event can
be translated to a interpretable event by the \seef{TranslateKeyEvent}
function. The result of this function can be used in the other event
examining functions.

A custom keyboard driver can be installed using the \seef{SetKeyboardDriver}
function. The current keyboard driver can be retrieved using the
\seep{GetKeyboardDriver} function. The last section of this chapter
demonstrates how to make a keyboard driver.

\section{Constants, Type and variables }

\subsection{Constants}

The following constants define some error constants, which may be returned
by the keyboard functions.
\begin{verbatim}
errKbdBase           = 1010;
errKbdInitError      = errKbdBase + 0;
errKbdNotImplemented = errKbdBase + 1;
\end{verbatim}
The following constants denote special keyboard keys. The first constants
denote the function keys:
\begin{verbatim}
const
  kbdF1        = $FF01;
  kbdF2        = $FF02;
  kbdF3        = $FF03;
  kbdF4        = $FF04;
  kbdF5        = $FF05;
  kbdF6        = $FF06;
  kbdF7        = $FF07;
  kbdF8        = $FF08;
  kbdF9        = $FF09;
  kbdF10       = $FF0A;
  kbdF11       = $FF0B;
  kbdF12       = $FF0C;
  kbdF13       = $FF0D;
  kbdF14       = $FF0E;
  kbdF15       = $FF0F;
  kbdF16       = $FF10;
  kbdF17       = $FF11;
  kbdF18       = $FF12;
  kbdF19       = $FF13;
  kbdF20       = $FF14;
\end{verbatim}
Constants  \$15 till \$1F are reserved for future function keys. The
following constants denote the cursor movement keys:
\begin{verbatim}
  kbdHome      = $FF20;
  kbdUp        = $FF21;
  kbdPgUp      = $FF22;
  kbdLeft      = $FF23;
  kbdMiddle    = $FF24;
  kbdRight     = $FF25;
  kbdEnd       = $FF26;
  kbdDown      = $FF27;
  kbdPgDn      = $FF28;

  kbdInsert    = $FF29;
  kbdDelete    = $FF2A;
\end{verbatim}
Constants \$2B till \$2F are reserved for future keypad keys.
The following flags are also defined:
\begin{verbatim}
  kbASCII       = $00;
  kbUniCode     = $01;
  kbFnKey       = $02;
  kbPhys        = $03;
  kbReleased    = $04;
\end{verbatim}
They can be used to check what kind of data a key event contains.
The following shift-state flags can be used to determine the shift state of
a key (i.e. which of the SHIFT, ALT and CTRL keys were pressed
simultaneously with a key):
\begin{verbatim}
  kbLeftShift   = 1;
  kbRightShift  = 2;
  kbShift       = kbLeftShift or kbRightShift;
  kbCtrl        = 4;
  kbAlt         = 8;
\end{verbatim}
The following constant strings are used in the key name functions 
\seef{FunctionKeyName} and \seef{KeyEventToString}:
\begin{verbatim}
SShift       : Array [1..3] of string[5] = ('SHIFT','CTRL','ALT');
LeftRight   : Array [1..2] of string[5] = ('LEFT','RIGHT');
UnicodeChar : String = 'Unicode character ';
SScanCode    : String = 'Key with scancode ';
SUnknownFunctionKey : String = 'Unknown function key : ';
SAnd         : String = 'AND';
SKeyPad      : Array [0..($FF2F-kbdHome)] of string[6] = 
               ('Home','Up','PgUp','Left',
                'Middle','Right','End','Down',
                'PgDn','Insert','Delete','',
                '','','','');
\end{verbatim}
They can be changed to localize the key names when needed.

\subsection{Types}
The \var{TKeyEvent} type is the base type for all keyboard events:
\begin{verbatim}
  TKeyEvent = Longint;
\end{verbatim}
The key stroke is encoded in the 4 bytes of the \var{TKeyEvent} type. 
The various fields of the key stroke encoding can be obtained by typecasting
the \var{TKeyEvent} type to the \var{TKeyRecord} type:
\begin{verbatim}
  TKeyRecord = packed record
    KeyCode : Word;
    ShiftState, Flags : Byte;
  end;
\end{verbatim}
The structure of a \var{TKeyRecord} structure is explained in \seet{keyevent}.
\begin{FPCltable}{lp{10cm}}{Structure of TKeyRecord}{keyevent}
Field & Meaning \\ \hline
KeyCode & Depending on \var{flags} either the physical representation of a key
         (under DOS scancode, ascii code pair), or the translated
           ASCII/unicode character.\\
ShiftState & Shift-state when this key was pressed (or shortly after) \\
Flags & Determine how to interpret \var{KeyCode} \\ \hline
\end{FPCltable}
The shift-state can be checked using the various shift-state constants, 
and the flags in the last byte can be checked using one of the
kbASCII, kbUniCode, kbFnKey, kbPhys, kbReleased constants.

If there are two keys returning the same char-code, there's no way to find
out which one was pressed (Gray+ and Simple+). If it needs to be known which
was pressed, the untranslated keycodes must be used, but these are system
dependent. System dependent constants may be defined to cover those, with
possibily having the same name (but different value).

The \var{TKeyboardDriver} record can be used to install a custom keyboard
driver with the \seef{SetKeyboardDriver} function:
\begin{verbatim}
Type 
  TKeyboardDriver = Record
    InitDriver : Procedure;
    DoneDriver : Procedure;
    GetKeyEvent : Function : TKeyEvent;
    PollKeyEvent : Function : TKeyEvent;
    GetShiftState : Function : Byte;
    TranslateKeyEvent : Function (KeyEvent: TKeyEvent): TKeyEvent;
    TranslateKeyEventUniCode: Function (KeyEvent: TKeyEvent): TKeyEvent;
  end;
\end{verbatim}
The various fields correspond to the different functions of the keyboard unit 
interface. For more information about this record see \sees{kbddriver}

\section{Functions and Procedures}

\begin{procedure}{DoneKeyboard}
\Declaration
Procedure DoneKeyboard;
\Description
\var{DoneKeyboard} de-initializes the keyboard interface if the keyboard
driver is active. If the keyboard driver is not active, the function does
nothing.

This will cause the keyboard driver to clear up any allocated memory, 
or restores the console or terminal the program was running in to its 
initial state before the call to \seep{InitKeyBoard}. This function should 
be called on program exit. Failing to do so may leave the terminal or
console window in an unusable state. Its exact action depends on the 
platform on which the program is running.
\Errors
None.
\SeeAlso
\seep{InitKeyBoard}
\end{procedure}

For an example, see most other functions.

\begin{function}{FunctionKeyName}
\Declaration
Function FunctionKeyName (KeyCode : Word) : String;
\Description
\var{FunctionKeyName} returns a string representation of the function key
with code \var{KeyCode}. This can be an actual function key, or one of the
cursor movement keys.
\Errors
In case \var{KeyCode} does not contain a function code, the
\var{SUnknownFunctionKey} string is returned, appended with the
\var{KeyCode}.
\SeeAlso
\seef{ShiftStateToString}
\seef{KeyEventToString}
\end{function}

\FPCexample{ex8}

\begin{procedure}{GetKeyboardDriver}
\Declaration
Procedure GetKeyboardDriver (Var Driver : TKeyboardDriver);
\Description
\var{GetKeyBoardDriver} returns in \var{Driver} the currently active
keyboard driver. This function can be used to enhance an existing
keyboarddriver.

For more information on getting and setting the keyboard driver
\sees{kbddriver}.
\Errors
None.
\SeeAlso
\seef{SetKeyboardDriver}
\end{procedure}

\begin{function}{GetKeyEvent}
\Declaration
function GetKeyEvent: TKeyEvent;
\Description
\var{GetKeyEvent} returns the last keyevent if one was stored in
\var{PendingKeyEvent}, or waits for one if none is available.
A non-blocking version is available in \seef{PollKeyEvent}.

The returned key is encoded as a \var{TKeyEvent} type variable, and
is normally the physical key scan code, (the scan code is driver 
dependent) which can be translated with one of the translation 
functions \seef{TranslateKeyEvent} or \seef{TranslateKeyEventUniCode}. 
See the types section for a description of how the key is described.
\Errors
If no key became available, 0 is returned.
\SeeAlso
\seep{PutKeyEvent}, \seef{PollKeyEvent}, \seef{TranslateKeyEvent},
\seef{TranslateKeyEventUniCode}
\end{function}

\FPCexample{ex1}

\begin{function}{GetKeyEventChar}
\Declaration
function GetKeyEventChar(KeyEvent: TKeyEvent): Char;
\Description
\var{GetKeyEventChar} returns the charcode part of the given 
\var{KeyEvent}, if it contains a translated character key 
keycode. The charcode is simply the ascii code of the 
character key that was pressed.

It returns the null character if the key was not a character key, but e.g. a
function key.
\Errors
None.
\SeeAlso
\seef{GetKeyEventUniCode},
\seef{GetKeyEventShiftState}, 
\seef{GetKeyEventFlags},
\seef{GetKeyEventCode},
\seef{GetKeyEvent}
\end{function}

For an example, see \seef{GetKeyEvent}

\begin{function}{GetKeyEventCode}
\Declaration
function GetKeyEventCode(KeyEvent: TKeyEvent): Word;
\Description
\var{GetKeyEventCode} returns the translated function keycode part of 
the given KeyEvent, if it contains a translated function key.

If the key pressed was not a function key, the null character is returned.
\Errors
None.
\SeeAlso
\seef{GetKeyEventUniCode},
\seef{GetKeyEventShiftState}, 
\seef{GetKeyEventFlags},
\seef{GetKeyEventChar},
\seef{GetKeyEvent}
\end{function}

\FPCexample{ex2}

\begin{function}{GetKeyEventFlags}
\Declaration
function GetKeyEventFlags(KeyEvent: TKeyEvent): Byte;
\Description
\var{GetKeyEventFlags} returns the flags part of the given 
\var{KeyEvent}. 
\Errors
None.
\SeeAlso
\seef{GetKeyEventUniCode},
\seef{GetKeyEventShiftState}, 
\seef{GetKeyEventCode},
\seef{GetKeyEventChar},
\seef{GetKeyEvent}
\end{function}

For an example, see \seef{GetKeyEvent}

\begin{function}{GetKeyEventShiftState}
\Declaration
function GetKeyEventShiftState(KeyEvent: TKeyEvent): Byte;
\Description
\var{GetKeyEventShiftState} returns the shift-state values of 
the given \var{KeyEvent}. This can be used to detect which of the modifier
keys \var{Shift}, \var{Alt} or \var{Ctrl} were pressed. If none were
pressed, zero is returned.

Note that this function does not always return expected results;
In a unix X-Term, the modifier keys do not always work.
\Errors
None.
\SeeAlso
\seef{GetKeyEventUniCode},
\seef{GetKeyEventFlags}, 
\seef{GetKeyEventCode},
\seef{GetKeyEventChar},
\seef{GetKeyEvent}
\end{function}

\FPCexample{ex3}

\begin{function}{GetKeyEventUniCode}
\Declaration
function GetKeyEventUniCode(KeyEvent: TKeyEvent): Word;
\Description
\var{GetKeyEventUniCode} returns the unicode part of the 
given \var{KeyEvent} if it contains a translated unicode 
character.
\Errors
None.
\SeeAlso
\seef{GetKeyEventShiftState},
\seef{GetKeyEventFlags}, 
\seef{GetKeyEventCode},
\seef{GetKeyEventChar},
\seef{GetKeyEvent}
\end{function}

No example available yet.

\begin{procedure}{InitKeyBoard}
\Declaration
procedure InitKeyboard;
\Description
\var{InitKeyboard} initializes the keyboard driver. 
If the driver is already active, it does nothing. When the driver is
initialized, it will do everything necessary to ensure the functioning of
the keyboard, including allocating memory, initializing the terminal etc.

This function should be called once, before using any of the
keyboard functions. When it is called, the \seep{DoneKeyboard} function
should also be called before exiting the program or changing the keyboard
driver with \seef{SetKeyboardDriver}.
\Errors
None.
\SeeAlso
\seep{DoneKeyboard}, \seef{SetKeyboardDriver}
\end{procedure}

For an example, see most other functions.

\begin{function}{IsFunctionKey}
\Declaration
function IsFunctionKey(KeyEvent: TKeyEvent): Boolean;
\Description
\var{IsFunctionKey} returns \var{True} if the given key event
in \var{KeyEvent} was a function key or not.
\Errors
None.
\SeeAlso
\seef{GetKeyEvent}
\end{function}

\FPCexample{ex7}

\begin{function}{KeyEventToString}
\Declaration
Function KeyEventToString(KeyEvent : TKeyEvent) : String;
\Description
\var{KeyEventToString} translates the key event in \var{KeyEvent} to a
human-readable description of the pressed key. It will use the constants
described in the constants section to do so.
\Errors
If an unknown key is passed, the scancode is returned, prefixed with the 
\var{SScanCode} string.
\SeeAlso
\seef{FunctionKeyName}, \seef{ShiftStateToString}
\end{function}

For an example, see most other functions.

\begin{function}{PollKeyEvent}
\Declaration
function PollKeyEvent: TKeyEvent;
\Description
\var{PollKeyEvent} checks whether a key event is available, 
and returns it if one is found. If no event is pending, 
it returns 0. 

Note that this does not remove the key from the pending keys. 
The key should still be retrieved from the pending key events 
list with the \seef{GetKeyEvent} function.
\Errors
None.
\SeeAlso
\seep{PutKeyEvent}, \seef{GetKeyEvent}
\end{function}

\FPCexample{ex4}

\begin{function}{PollShiftStateEvent}
\Declaration
function PollShiftStateEvent: TKeyEvent;
\Description
\var{PollShiftStateEvent} returns the current shiftstate in a 
keyevent. This will return 0 if there is no key event pending.
\Errors
None.
\SeeAlso
\seef{PollKeyEvent}, \seef{GetKeyEvent}
\end{function}

\FPCexample{ex6}

\begin{procedure}{PutKeyEvent}
\Declaration
procedure PutKeyEvent(KeyEvent: TKeyEvent);
\Description
\var{PutKeyEvent} adds the given \var{KeyEvent} to the input 
queue. Please note that depending on the implementation this 
can hold only one value, i.e. when calling \var{PutKeyEvent}
multiple times, only the last pushed key will be remembered.
\Errors
None
\SeeAlso
\seef{PollKeyEvent}, \seef{GetKeyEvent}
\end{procedure}

\FPCexample{ex5}

\begin{function}{SetKeyboardDriver}
\Declaration
Function SetKeyboardDriver (Const Driver : TKeyboardDriver) : Boolean;
\Description
\var{SetKeyBoardDriver} sets the keyboard driver to \var{Driver}, if the
current keyboard driver is not yet initialized. If the current
keyboard driver is initialized, then \var{SetKeyboardDriver} does 
nothing. Before setting the driver, the currently active driver should
be disabled with a call to \seep{DoneKeyboard}.

The function returns \var{True} if the driver was set, \var{False} if not.

For more information on setting the keyboard driver, see \sees{kbddriver}.
\Errors
None.
\SeeAlso
\seep{GetKeyboardDriver}, \seep{DoneKeyboard}.
\end{function}

\begin{function}{ShiftStateToString}
\Declaration
Function ShiftStateToString(KeyEvent : TKeyEvent; UseLeftRight : Boolean) : String;
\Description
\var{ShiftStateToString} returns a string description of the shift state
of the key event \var{KeyEvent}. This can be an empty string. 

The shift state is described using the strings in the \var{SShift} constant.
\Errors
None.
\SeeAlso
\seef{FunctionKeyName}, \seef{KeyEventToString}
\end{function}

For an example, see \seef{PollShiftStateEvent}.

\begin{function}{TranslateKeyEvent}
\Declaration
function TranslateKeyEvent(KeyEvent: TKeyEvent): TKeyEvent;
\Description
\var{TranslateKeyEvent} performs ASCII translation of the \var{KeyEvent}.
It translates a physical key to a function key if the key is a function key,
and translates the physical key to the ordinal of the ascii character if 
there is an equivalent character key.
\Errors
None.
\SeeAlso
\seef{TranslateKeyEventUniCode}
\end{function}

For an example, see \seef{GetKeyEvent}

\begin{function}{TranslateKeyEventUniCode}
\Declaration
function TranslateKeyEventUniCode(KeyEvent: TKeyEvent): TKeyEvent;
\Description
\var{TranslateKeyEventUniCode} performs Unicode translation of the 
\var{KeyEvent}. It is not yet implemented for all platforms.

\Errors
If the function is not yet implemented, then the \var{ErrorCode} of the 
\file{system} unit will be set to \var{errKbdNotImplemented}
\SeeAlso
\end{function}

No example available yet.

\section{Keyboard scan codes}
Special physical keys are encoded with the DOS scan codes for these keys
in the second byte of the \var{TKeyEvent} type.
A complete list of scan codes can be found in \seet{keyscans}. This is the
list of keys that is used by the default key event translation mechanism.
When writing a keyboard driver, either these constants should be returned
by the various key event functions, or the \var{TranslateKeyEvent} hook
should be implemented by the driver.
\begin{FPCltable}{llllll}{Physical keys scan codes}{keyscans}
Code & Key & Code & Key & Code & Key\\ \hline
00 & NoKey           & 3D & F3              & 70 & ALT-F9           \\
01 & ALT-Esc          & 3E & F4              & 71 & ALT-F10          \\
02 & ALT-Space        & 3F & F5              & 72 & CTRL-PrtSc       \\
04 & CTRL-Ins         & 40 & F6              & 73 & CTRL-Left        \\
05 & SHIFT-Ins        & 41 & F7              & 74 & CTRL-Right       \\
06 & CTRL-Del         & 42 & F8              & 75 & CTRL-end         \\
07 & SHIFT-Del        & 43 & F9              & 76 & CTRL-PgDn        \\
08 & ALT-Back         & 44 & F10             & 77 & CTRL-Home        \\
09 & ALT-SHIFT-Back    & 47 & Home            & 78 & ALT-1            \\
0F & SHIFT-Tab        & 48 & Up              & 79 & ALT-2            \\
10 & ALT-Q            & 49 & PgUp            & 7A & ALT-3            \\
11 & ALT-W            & 4B & Left            & 7B & ALT-4            \\
12 & ALT-E            & 4C & Center          & 7C & ALT-5            \\
13 & ALT-R            & 4D & Right           & 7D & ALT-6            \\
14 & ALT-T            & 4E & ALT-GrayPlus     & 7E & ALT-7            \\
15 & ALT-Y            & 4F & end             & 7F & ALT-8            \\
16 & ALT-U            & 50 & Down            & 80 & ALT-9            \\
17 & ALT-I            & 51 & PgDn            & 81 & ALT-0            \\
18 & ALT-O            & 52 & Ins             & 82 & ALT-Minus        \\
19 & ALT-P            & 53 & Del             & 83 & ALT-Equal        \\
1A & ALT-LftBrack     & 54 & SHIFT-F1         & 84 & CTRL-PgUp        \\
1B & ALT-RgtBrack     & 55 & SHIFT-F2         & 85 & F11             \\
1E & ALT-A            & 56 & SHIFT-F3         & 86 & F12             \\
1F & ALT-S            & 57 & SHIFT-F4         & 87 & SHIFT-F11        \\
20 & ALT-D            & 58 & SHIFT-F5         & 88 & SHIFT-F12        \\
21 & ALT-F            & 59 & SHIFT-F6         & 89 & CTRL-F11         \\
22 & ALT-G            & 5A & SHIFT-F7         & 8A & CTRL-F12         \\
23 & ALT-H            & 5B & SHIFT-F8         & 8B & ALT-F11          \\
24 & ALT-J            & 5C & SHIFT-F9         & 8C & ALT-F12          \\
25 & ALT-K            & 5D & SHIFT-F10        & 8D & CTRL-Up          \\
26 & ALT-L            & 5E & CTRL-F1          & 8E & CTRL-Minus       \\
27 & ALT-SemiCol      & 5F & CTRL-F2          & 8F & CTRL-Center      \\
28 & ALT-Quote        & 60 & CTRL-F3          & 90 & CTRL-GreyPlus    \\
29 & ALT-OpQuote      & 61 & CTRL-F4          & 91 & CTRL-Down        \\
2B & ALT-BkSlash      & 62 & CTRL-F5          & 94 & CTRL-Tab         \\
2C & ALT-Z            & 63 & CTRL-F6          & 97 & ALT-Home         \\
2D & ALT-X            & 64 & CTRL-F7          & 98 & ALT-Up           \\
2E & ALT-C            & 65 & CTRL-F8          & 99 & ALT-PgUp         \\
2F & ALT-V            & 66 & CTRL-F9          & 9B & ALT-Left         \\
30 & ALT-B            & 67 & CTRL-F10         & 9D & ALT-Right        \\
31 & ALT-N            & 68 & ALT-F1           & 9F & ALT-end          \\
32 & ALT-M            & 69 & ALT-F2           & A0 & ALT-Down         \\
33 & ALT-Comma        & 6A & ALT-F3           & A1 & ALT-PgDn         \\
34 & ALT-Period       & 6B & ALT-F4           & A2 & ALT-Ins          \\
35 & ALT-Slash        & 6C & ALT-F5           & A3 & ALT-Del          \\
37 & ALT-GreyAst      & 6D & ALT-F6           & A5 & ALT-Tab          \\
3B & F1              & 6E & ALT-F7            & &                     \\
3C & F2              & 6F & ALT-F8            & &                     \\ 

\end{FPCltable}
A list of scan codes for special keys and combinations with the SHIFT, ALT
and CTRL keys can be found in \seet{speckeys}; They are for quick reference
only.
\begin{FPCltable}{llccc}{Special keys scan codes}{speckeys}
Key         & Code  & SHIFT-Key & CTRL-Key & Alt-Key \\ \hline
NoKey       &  00   &       &     &     \\
F1          &  3B   &  54   & 5E  & 68  \\
F2          &  3C   &  55   & 5F  & 69  \\
F3          &  3D   &  56   & 60  & 6A  \\
F4          &  3E   &  57   & 61  & 6B  \\
F5          &  3F   &  58   & 62  & 6C  \\
F6          &  40   &  59   & 63  & 6D  \\
F7          &  41   &  5A   & 64  & 6E  \\
F8          &  42   &  5A   & 65  & 6F  \\
F9          &  43   &  5B   & 66  & 70  \\
F10         &  44   &  5C   & 67  & 71  \\
F11         &  85   &  87   & 89  & 8B  \\
F12         &  86   &  88   & 8A  & 8C  \\
Home        &  47   &       & 77  & 97  \\ 
Up          &  48   &       & 8D  & 98  \\
PgUp        &  49   &       & 84  & 99  \\
Left        &  4B   &       & 73  & 9B  \\
Center      &  4C   &       & 8F  &     \\
Right       &  4D   &       & 74  & 9D  \\
end         &  4F   &       & 75  & 9F  \\
Down        &  50   &       & 91  & A0  \\
PgDn        &  51   &       & 76  & A1  \\
Ins         &  52   & 05    & 04  & A2  \\
Del         &  53   & 07    & 06  & A3  \\
Tab         &  8    & 0F    & 94  & A5  \\
GreyPlus    &       &       & 90  & 4E  \\
\hline
\end{FPCltable}
\section{Writing a keyboard driver}
\label{se:kbddriver}
Writing a keyboard driver means that hooks must be created for most of the 
keyboard unit functions. The \var{TKeyBoardDriver} record contains a field
for each of the possible hooks:
\begin{verbatim}
TKeyboardDriver = Record
  InitDriver : Procedure;
  DoneDriver : Procedure;
  GetKeyEvent : Function : TKeyEvent;
  PollKeyEvent : Function : TKeyEvent;
  GetShiftState : Function : Byte;
  TranslateKeyEvent : Function (KeyEvent: TKeyEvent): TKeyEvent;
  TranslateKeyEventUniCode: Function (KeyEvent: TKeyEvent): TKeyEvent;
end;
\end{verbatim}
The meaning of these hooks is explained below:
\begin{description}
\item[InitDriver] Called to initialize and enable the driver. 
Guaranteed to be called only once. This should initialize all needed things
for the driver.
\item[DoneDriver] Called to disable and clean up the driver. Guaranteed to be
called after a call to \var{initDriver}. This should clean up all
things initialized by \var{InitDriver}.
\item[GetKeyEvent] Called by \seef{GetKeyEvent}. Must wait for and return the
next key event. It should NOT store keys.
\item[PollKeyEvent] Called by \seef{PollKeyEvent}. It must return the next key
event if there is one. Should not store keys. 
\item[GetShiftState] Called by \seef{PollShiftStateEvent}.  Must return the current
shift state.
\item[TranslateKeyEvent] Should translate a raw key event to a cOrrect
key event, i.e. should fill in the shiftstate and convert function key
scancodes to function key keycodes. If the
\var{TranslateKeyEvent} is not filled in, a default translation function
will be called which converts the known scancodes from the tables in the
previous section to a correct keyevent.
\item[TranslateKeyEventUniCode] Should translate a key event to a unicode key
representation. 
\end{description}
Strictly speaking, only the \var{GetKeyEvent} and \var{PollKeyEvent}
hooks must be implemented for the driver to function correctly. 

The following unit demonstrates how a keyboard driver can be installed.
It takes the installed driver, and hooks into the \var{GetKeyEvent}
function to register and log the key events in a file. This driver
can work on top of any other driver, as long as it is inserted in the 
\var{uses} clause {\em after} the real driver unit, and the real driver unit
should set the driver record in its initialization section.
\FPCexample{logkeys}
The following program demonstrates the use of the unit:
\FPCexample{ex9}
Note that with a simple extension of this unit could be used to make a
driver that is capable of recording and storing a set of keyboard strokes,
and replaying them at a later time, so a 'keyboard macro' capable driver.
This driver could sit on top of any other driver.

% the Linux unit
%
%   $Id$
%   This file is part of the FPC documentation.
%   Copyright (C) 1997, by Michael Van Canneyt
%
%   The FPC documentation is free text; you can redistribute it and/or
%   modify it under the terms of the GNU Library General Public License as
%   published by the Free Software Foundation; either version 2 of the
%   License, or (at your option) any later version.
%
%   The FPC Documentation is distributed in the hope that it will be useful,
%   but WITHOUT ANY WARRANTY; without even the implied warranty of
%   MERCHANTABILITY or FITNESS FOR A PARTICULAR PURPOSE.  See the GNU
%   Library General Public License for more details.
%
%   You should have received a copy of the GNU Library General Public
%   License along with the FPC documentation; see the file COPYING.LIB.  If not,
%   write to the Free Software Foundation, Inc., 59 Temple Place - Suite 330,
%   Boston, MA 02111-1307, USA. 
%
\chapter{The LINUX unit.}
This chapter describes the LINUX unit for Free Pascal. The unit was written
by Micha\"el van Canneyt. It works only on the Linux operating system.

This chapter is divided in 2 sections:
\begin{itemize}
\item The first section lists all constants, types and variables, as listed
in the interface section of the LINUX unit.
\item The second section describes all procedures and functions in the LINUX
unit.
\end{itemize}

\section{Type, Variable and Constant declarations}

\subsection{Types}
\label{sec:types}
PGlob and TGlob are 2 types used in the \seef{Glob} function:
\begin{verbatim}
PGlob = ^TGlob;
TGlob = record
  Name : PChar;
  Next : PGlob;
  end;
\end{verbatim}
The following types are used in the signal-processing procedures.
\begin{verbatim}
{$Packrecords 1}
SignalHandler   = Procedure ( Sig : Integer);
PSignalHandler  = ^SignalHandler;
SignalRestorer  = Procedure;
PSignalrestorer = ^SignalRestorer;

SigActionRec = Record
  Sa_Handler  : PSignalhandler;
  Sa_Mask     : Longint;
  Sa_flags    : Integer;
  Sa_Restorer : PSignalRestorer;
end;
PSigActionRec = ^SigActionRec;
\end{verbatim}
Stat is used to store information about a file. It is defined in the
syscalls unit.

\begin{verbatim}
  stat = record
     dev    : word;
     pad1   : word;
     ino    : longint;
     mode   : word;
     nlink  : word;
     uid    : word;
     gid    : word;
     rdev   : word;
     pad2   : word;
     size   : longint;
     blksze : Longint;
     blocks : Longint;
     atime  : Longint;
     unused1 : longint;
     mtime   : Longint;
     unused2 : longint;
     ctime   : Longint;
     unused3 : longint;
     unused4 : longint;
     unused5 : longint;
     end;
 \end{verbatim}

Statfs is used to store information about a filesystem. It is defined in
the syscalls unit.
\begin{verbatim}

   statfs = record
     fstype   : longint;
     bsize    : longint;
     blocks   : longint;
     bfree    : longint;
     bavail   : longint;
     files    : longint;
     ffree    : longint;
     fsid     : longint;
     namelen  : longint; 
     spare    : array [0..6] of longint;
     end
\end{verbatim}
\var{Dir and PDir} are used in the \seef{OpenDir} and \seef{ReadDir}
functions. 
\begin{verbatim}
  TDir =record
    fd     : integer;
    loc    : longint;
    size   : integer;
    buf    : pdirent;
    nextoff: longint;
    dd_max : integer; 
    lock   : pointer;
  end;
  PDir =^TDir;
\end{verbatim}
\var{Dirent, PDirent} are used in the \seef{ReadDir} function to return files in a directory.
\begin{verbatim}
 PDirent = ^Dirent;
 Dirent = Record  
   ino,
   off    : longint;
   reclen : word;
   name   : string[255]
 end; 
\end{verbatim}
Termio and Termios are used with iotcl() calls for terminal handling.
\begin{verbatim}
Const  NCCS = 19;
       NCC = 8;
         
Type termio = record
	c_iflag,		{ input mode flags }
	c_oflag,		{ output mode flags }
	c_cflag,		{ control mode flags }
	c_lflag : Word;		{ local mode flags }
	c_line : Word;		{ line discipline - careful, only High byte in use}
	c_cc : array [0..NCC-1] of char;	{ control characters }
end;

termios = record
  c_iflag,              { input mode flags }
  c_oflag,              { output mode flags }
  c_cflag,              { control mode flags }
  c_lflag : Cardinal;	{ local mode flags }
  c_line : char;          { line discipline }
  c_cc : array [0..NCCS-1] of char;      { control characters }
end;
\end{verbatim}
\var{Utimbuf} is used in the \seef{Utime} call to set access and modificaton time
of a file.
\begin{verbatim}
utimbuf = record
  actime,modtime : Longint;
  end;
\end{verbatim}
For the \seef{Select} call, the following 4 types are needed:
\begin{verbatim}
FDSet = Array [0..31] of longint;
PFDSet = ^FDSet;

TimeVal = Record
   sec,usec : Longint;
end;
PTimeVal = ^TimeVal;
\end{verbatim}
The \seep{Uname} function uses the \var{utsname} to return information about
the current kernel :
\begin{verbatim}
utsname =record
  sysname,nodename,release,
  version,machine,domainname : Array[0..64] of char;
end;
\end{verbatim}
Its elements are null-terminated C style strings, you cannot access them
directly !

\subsection{Variables}
\var{Linuxerror} is the variable in which the procedures in the linux unit
report errors.
\begin{verbatim}
LinuxError : Longint;
\end{verbatim}
\var{StdErr} Is a \var{Text} variable, corresponding to Standard Error or
diagnostic output. It is connected to file descriptor 2. It can be freely
used, and will be closed on exit.
\begin{verbatim}
StdErr : Text;
\end{verbatim}

\subsection{Constants}
Constants for setting/getting process priorities :
\begin{verbatim}
      Prio_Process = 0;
      Prio_PGrp    = 1;
      Prio_User    = 2;
\end{verbatim}
For testing  access rights:
\begin{verbatim}
      R_OK = 4; 
      W_OK = 2;
      X_OK = 1;
      F_OK = 0;
\end{verbatim}
For signal handling functions :
\begin{verbatim}
      SA_NOCLDSTOP = 1;
      SA_SHIRQ	   = $04000000;
      SA_STACK	   = $08000000;      
      SA_RESTART   = $10000000;
      SA_INTERRUPT = $20000000;
      SA_NOMASK	   = $40000000;
      SA_ONESHOT   = $80000000;
      
      SIG_BLOCK	  = 0;
      SIG_UNBLOCK = 1;
      SIG_SETMASK = 2;

      SIG_DFL = 0 ;
      SIG_IGN = 1 ;
      SIG_ERR = -1;
      
      SIGHUP		= 1;
      SIGINT		= 2;
      SIGQUIT		= 3;
      SIGILL		= 4;
      SIGTRAP		= 5;
      SIGABRT		= 6;
      SIGIOT		= 6;
      SIGBUS		= 7;
      SIGFPE		= 8;
      SIGKILL		= 9;
      SIGUSR1		= 10;
      SIGSEGV		= 11;
      SIGUSR2		= 12;
      SIGPIPE		= 13;
      SIGALRM		= 14;
      SIGTERM		= 15;
      SIGSTKFLT		= 16;
      SIGCHLD		= 17;
      SIGCONT		= 18;
      SIGSTOP		= 19;
      SIGTSTP		= 20;
      SIGTTIN		= 21;
      SIGTTOU		= 22;
      SIGURG		= 23;
      SIGXCPU		= 24;
      SIGXFSZ		= 25;
      SIGVTALRM		= 26;
      SIGPROF		= 27;
      SIGWINCH		= 28;
      SIGIO		= 29;
      SIGPOLL		= SIGIO;
      SIGPWR		= 30;
      SIGUNUSED		= 31;
\end{verbatim}
For file control mechanism :
\begin{verbatim}
      F_GetFd  = 1;
      F_SetFd  = 2;
      F_GetFl  = 3;
      F_SetFl  = 4;
      F_GetLk  = 5;
      F_SetLk  = 6;
      F_SetLkW = 7;
      F_GetOwn = 8;
      F_SetOwn = 9;
\end{verbatim}
For Terminal handling :
\begin{verbatim}
   TCGETS	= $5401 ;
   TCSETS	= $5402 ;
   TCSETSW	= $5403 ;
   TCSETSF	= $5404 ;
   TCGETA	= $5405 ;
   TCSETA	= $5406 ;
   TCSETAW	= $5407 ;
   TCSETAF	= $5408 ;
   TCSBRK	= $5409 ;
   TCXONC	= $540A ;
   TCFLSH	= $540B ;
   TIOCEXCL	= $540C ;
   TIOCNXCL	= $540D ;
   TIOCSCTTY	= $540E ;
   TIOCGPGRP	= $540F ;
   TIOCSPGRP	= $5410 ;
   TIOCOUTQ	= $5411 ;
   TIOCSTI	= $5412 ;
   TIOCGWINSZ	= $5413 ;
   TIOCSWINSZ	= $5414 ;
   TIOCMGET	= $5415 ;
   TIOCMBIS	= $5416 ;
   TIOCMBIC	= $5417 ;
   TIOCMSET	= $5418 ;
   TIOCGSOFTCAR	= $5419 ;
   TIOCSSOFTCAR	= $541A ;
   FIONREAD	= $541B ;
   TIOCINQ	= FIONREAD;
   TIOCLINUX	= $541C ;
   TIOCCONS	= $541D ;
   TIOCGSERIAL	= $541E ;
   TIOCSSERIAL	= $541F ;
   TIOCPKT	= $5420 ;
   FIONBIO	= $5421 ;
   TIOCNOTTY	= $5422 ;
   TIOCSETD	= $5423 ;
   TIOCGETD	= $5424 ;
   TCSBRKP		= $5425	 ;
   TIOCTTYGSTRUCT	= $5426  ;
   FIONCLEX	= $5450  ;
   FIOCLEX		= $5451 ;
   FIOASYNC	= $5452 ;
   TIOCSERCONFIG	= $5453 ;
   TIOCSERGWILD	= $5454 ;
   TIOCSERSWILD	= $5455 ;
   TIOCGLCKTRMIOS	= $5456 ;
   TIOCSLCKTRMIOS	= $5457 ;
   TIOCSERGSTRUCT	= $5458  ;
   TIOCSERGETLSR   = $5459  ;
   TIOCSERGETMULTI = $545A  ;
   TIOCSERSETMULTI = $545B  ;

   TIOCMIWAIT	= $545C	;
   TIOCGICOUNT	= $545D	;

   TIOCPKT_DATA		= 0;
   TIOCPKT_FLUSHREAD	= 1;
   TIOCPKT_FLUSHWRITE	= 2;
   TIOCPKT_STOP		= 4;
   TIOCPKT_START	= 8;
   TIOCPKT_NOSTOP	= 16;
   TIOCPKT_DOSTOP	= 32;
\end{verbatim}
Other than that, all constants for setting the speed and control flags of a
terminal line, as described in the \seem{termios}{2} man
page, are defined in the linux unit. It would take too much place to list
them here. 

To check the \var{mode} field of a \var{stat} record, you ca use the
following constants :
\begin{verbatim}
  { Constants to check stat.mode }
  STAT_IFMT   = $f000; {00170000}
  STAT_IFSOCK = $c000; {0140000}
  STAT_IFLNK  = $a000; {0120000}
  STAT_IFREG  = $8000; {0100000}
  STAT_IFBLK  = $6000; {0060000}
  STAT_IFDIR  = $4000; {0040000}
  STAT_IFCHR  = $2000; {0020000}
  STAT_IFIFO  = $1000; {0010000}
  STAT_ISUID  = $0800; {0004000}
  STAT_ISGID  = $0400; {0002000}
  STAT_ISVTX  = $0200; {0001000}
  { Constants to check permissions }
  STAT_IRWXO = $7;
  STAT_IROTH = $4;
  STAT_IWOTH = $2;
  STAT_IXOTH = $1;

  STAT_IRWXG = STAT_IRWXO shl 3;
  STAT_IRGRP = STAT_IROTH shl 3;
  STAT_IWGRP = STAT_IWOTH shl 3;
  STAT_IXGRP = STAT_IXOTH shl 3;

  STAT_IRWXU = STAT_IRWXO shl 6;
  STAT_IRUSR = STAT_IROTH shl 6;
  STAT_IWUSR = STAT_IWOTH shl 6;
  STAT_IXUSR = STAT_IXOTH shl 6;
\end{verbatim}
You can test the type of a filesystem returned by a \seef{FSStat} call with
the following constants:
\begin{verbatim}
  fs_old_ext2 = $ef51;
  fs_ext2     = $ef53;
  fs_ext      = $137d;
  fs_iso      = $9660;
  fs_minix    = $137f;
  fs_minix_30 = $138f;
  fs_minux_V2 = $2468;
  fs_msdos    = $4d44;
  fs_nfs      = $6969;
  fs_proc     = $9fa0;
  fs_xia      = $012FD16D;
\end{verbatim}
the \seep{FLock} call uses the following mode constants :
\begin{verbatim}
  LOCK_SH = 1;
  LOCK_EX = 2;
  LOCK_UN = 8;
  LOCK_NB = 4;
\end{verbatim}

\section{Functions and procedures}

%\function{Name}{arguments}{return type}{explain}{errors}{refs}
%\procedure{Name}{arguments}{explain}{errors}{refs}
%\function{}{()}{}{}{}{}{}
%\procedure{}{}{}{}{}{}
\function{GetEpochTime}{}{longint}
{
returns the number of seconds since 00:00:00 gmt, january 1, 1970.
it is adjusted to the local time zone, but not to DST.
}
{no errors}
{\seep{EpochToLocal}, \seep{GetTime}, \seem{time}{2}}

\input{linuxex/ex1.tex}

\procedure
{EpochToLocal}
{(Epoch : Longint; var Year,Month,Day,Hour,Minute,Second : Word)}
{
Converts the epoch time (=Number of seconds since 00:00:00 , January 1,
1970, corrected for your time zone ) to local date and time.
}
{None}
{\seef{GetEpochTime}, \seef{LocalToEpoch}, \seep{GetTime},\seep{GetDate} }

\input{linuxex/ex3.tex}

\function{LocalToEpoch}{(Year,Month,Day,Hour,Minute,Second : Word)}{longint}
{
Converts the Local time to epoch time (=Number of seconds since 00:00:00 , January 1,
1970 ).
}
{None}
{\seef{GetEpochTime}, \seep{EpochToLocal}, \seep{GetTime},\seep{GetDate} }

\input{linuxex/ex4.tex}

\procedure{GetTime}
{ (Var Hour,Minute, Second : Word) }
{
Returns the current time of the day.
}
{None}
{\seef{GetEpochTime}, \seep{GetDate}, \seep{EpochToLocal} }

\input{linuxex/ex5.tex}

\procedure{GetDate}
{ (Var Year, Month, Day : Word) }
{
Returns the current day.
}
{None}
{\seef{GetEpochTime}, \seep{GetTime}, \seep{EpochToLocal} }

\input{linuxex/ex6.tex}

\procedure{Execve}
{(Path : pathstr; args,ep : ppchar)}
{
Replaces the currently running program with the program, specified in
\var{path}.
It gives the program the options in \var{args}, and the environment in
\var{ep}. They are pointers to an array of pointers to null-terminated
strings. The last pointer in this array should be nil.

On success, \var{execve} does not return.
}
{Errors are reported in \var{LinuxError}:
\begin{description}
\item[eacces] File is not a regular file, or has no execute permission.
A compononent of the path has no search permission.
\item[sys\_ eperm] The file system is mounted \textit{noexec}.
\item[sys\_ e2big] Argument list too big.
\item[sys\_ enoexec] The magic number in the file is incorrect.
\item[sys\_ enoent] The file does not exist.
\item[sys\_ enomem] Not enough memory for kernel.
\item[sys\_ enotdir] A component of the path is not a directory.
\item[sys\_ eloop] The path contains a circular reference (via symlinks).
\end{description}}
{\seep{Execve}, \seep{Execv}, \seep{Execvp} \seep{Execle},
\seep{Execl}, \seep{Execlp}, \seef {Fork}, \seem{execve}{2} }

\input{linuxex/ex7.tex}

\procedure{Execv}
{(Path : pathstr; args : ppchar)}
{
Replaces the currently running program with the program, specified in
\var{path}.
It gives the program the options in \var{args}.
This is a pointer to an array of pointers to null-terminated
strings. The last pointer in this array should be nil.
The current environment is passed to the program.

On success, \var{execv} does not return.
}
{Errors are reported in \var{LinuxError}:
\begin{description}
\item[sys\_eacces] File is not a regular file, or has no execute permission.
A compononent of the path has no search permission.
\item[sys\_eperm] The file system is mounted \textit{noexec}.
\item[sys\_e2big] Argument list too big.
\item[sys\_enoexec] The magic number in the file is incorrect.
\item[sys\_enoent] The file does not exist.
\item[sys\_enomem] Not enough memory for kernel.
\item[sys\_enotdir] A component of the path is not a directory.
\item[sys\_eloop] The path contains a circular reference (via symlinks).
\end{description}}
{\seep{Execve}, \seep{Execvp}, \seep{Execle},
\seep{Execl}, \seep{Execlp}, \seef {Fork}, \seem{execv}{3} }

\input{linuxex/ex8.tex}

\procedure{Execvp}
{(Path : pathstr; args : ppchar)}
{
Replaces the currently running program with the program, specified in
\var{path}. The executable in \var{path} is searched in the path, if it isn't
an absolute filename.
It gives the program the options in \var{args}. This is a pointer to an array of pointers to null-terminated
strings. The last pointer in this array should be nil.
The current environment is passed to the program.

On success, \var{execvp} does not return.
}
{Errors are reported in \var{LinuxError}:
\begin{description}
\item[sys\_eacces] File is not a regular file, or has no execute permission.
A compononent of the path has no search permission.
\item[sys\_eperm] The file system is mounted \textit{noexec}.
\item[sys\_e2big] Argument list too big.
\item[sys\_enoexec] The magic number in the file is incorrect.
\item[sys\_enoent] The file does not exist.
\item[sys\_enomem] Not enough memory for kernel.
\item[sys\_enotdir] A component of the path is not a directory.
\item[sys\_eloop] The path contains a circular reference (via symlinks).
\end{description}}
{\seep{Execve}, \seep{Execv}, \seep{Execle},
\seep{Execl}, \seep{Execlp}, \seef {Fork}, \seem{execvp}{3} }

\input{linuxex/ex9.tex}

\procedure{Execl}
{(Path : pathstr)}
{
Replaces the currently running program with the program, specified in
\var{path}. Path is split into a command and it's options.
The executable in \var{path} is NOT searched in the path.
The current environment is passed to the program.

On success, \var{execl} does not return.
}
{Errors are reported in \var{LinuxError}:
\begin{description}
\item[sys\_eacces] File is not a regular file, or has no execute permission.
A compononent of the path has no search permission.
\item[sys\_eperm] The file system is mounted \textit{noexec}.
\item[sys\_e2big] Argument list too big.
\item[sys\_enoexec] The magic number in the file is incorrect.
\item[sys\_enoent] The file does not exist.
\item[sys\_enomem] Not enough memory for kernel, or to split command line.
\item[sys\_enotdir] A component of the path is not a directory.
\item[sys\_eloop] The path contains a circular reference (via symlinks).
\end{description}}
{\seep{Execve}, \seep{Execv}, \seep{Execvp}, \seep{Execle},
 \seep{Execlp}, \seef {Fork}, \seem{execvp}{3} }

\input{linuxex/ex10.tex}

\procedure{Execle}
{(Path : pathstr, Ep : ppchar)}
{
Replaces the currently running program with the program, specified in
\var{path}. Path is split into a command and it's options.
The executable in \var{path} is searched in the path, if it isn't
an absolute filename.
The environment in \var{ep} is passed to the program.

On success, \var{execle} does not return.
}
{Errors are reported in \var{LinuxError}:
\begin{description}
\item[sys\_eacces] File is not a regular file, or has no execute permission.
A compononent of the path has no search permission.
\item[sys\_eperm] The file system is mounted \textit{noexec}.
\item[sys\_e2big] Argument list too big.
\item[sys\_enoexec] The magic number in the file is incorrect.
\item[sys\_enoent] The file does not exist.
\item[sys\_enomem] Not enough memory for kernel, or to split command line.
\item[sys\_enotdir] A component of the path is not a directory.
\item[sys\_eloop] The path contains a circular reference (via symlinks).
\end{description}}
{\seep{Execve}, \seep{Execv}, \seep{Execvp},
\seep{Execl}, \seep{Execlp}, \seef {Fork}, \seem{execvp}{3} }

\input{linuxex/ex11.tex}

\procedure{Execlp}
{(Path : pathstr)}
{
Replaces the currently running program with the program, specified in
\var{path}. Path is split into a command and it's options.
The executable in \var{path} is searched in the path, if it isn't
an absolute filename.
The current environment is passed to the program.

On success, \var{execlp} does not return.
}
{Errors are reported in \var{LinuxError}:
\begin{description}
\item[sys\_eacces] File is not a regular file, or has no execute permission.
A compononent of the path has no search permission.
\item[sys\_eperm] The file system is mounted \textit{noexec}.
\item[sys\_e2big] Argument list too big.
\item[sys\_enoexec] The magic number in the file is incorrect.
\item[sys\_enoent] The file does not exist.
\item[sys\_enomem] Not enough memory for kernel, or to split command line.
\item[sys\_enotdir] A component of the path is not a directory.
\item[sys\_eloop] The path contains a circular reference (via symlinks).
\end{description}}
{\seep{Execve}, \seep{Execv}, \seep{Execvp}, \seep{Execle},
\seep{Execl}, \seef {Fork}, \seem{execvp}{3} }

\input{linuxex/ex12.tex}

\function{Fork}{}{Longint}
{
Fork creates a child process which is a copy of the parent process.

Fork returns the process ID in the parent process, and zero in the child's
process. (you can get the parent's PID with \seef{GetPPid}).
}
{On error, -1 is returned to the parent, and no child is created.
\begin{description}
\item [sys\_eagain] Not enough memory to create child process.
\end{description}
}
{\seep{Execve}, \seem{fork}{2}}

\input{linuxex/ex14.tex}

\function{Shell}{(Command : String)}{Longint}
{\var{Shell} invokes the bash shell (\file{/bin/sh}), and feeds it the
command \var{Command} (using the \var{-c} option). The function then waits
for the command to complete, and then returns the exit
status of the command, or 127 if it could not complete the \seef{Fork} 
or \seep{Execve} calls.
}
{Errors are reported in LinuxError.}
{\seep{POpen}, \seef{Fork}, \seep{Execve}, \seem{system}{3}}

\input{linuxex/ex56.tex}

\procedure{Nice}{( N : Integer)}
{Nice adds \var{-N} to the priority of the running process. The lower the
priority numerically, the less the process is favored.

Only the superuser can specify a negative \var{N}, i.e. increase the rate at
which the process is run.
}
{ Errors are returned in \var{LinuxError}
\begin{description}
\item [sys\_eperm] A non-superuser tried to specify a negative \var{N}, i.e.
do a priority increase.
\end{description}
}{\seef{GetPriority}, \seef{SetPriority}, \seem{Nice}{2}}

\input{linuxex/ex15.tex}

\function{GetPriority}{(Which,Who : Integer)}{Integer}
{
GetPriority returns the priority with which a process is running.
Which process(es) is determined by the \var{Which} and \var{Who} variables.
\var{Which} can be one of the pre-defined \var{Prio\_Process, Prio\_PGrp,
Prio\_User}, in which case \var{Who} is the process ID, Process group ID or
User ID, respectively.
}
{
 Error checking must be done on LinuxError, since a priority can be negative.
 \begin{description}
 \item[sys\_esrch] No process found using \var{which} and \var{who}.
 \item[sys\_einval] \var{Which} was not one of \var{Prio\_Process, Prio\_Grp
or Prio\_User}.
 \end{description}
 }
{\seef{SetPriority}, \seep{Nice}, \seem{Getpriority}{2}}

For an example, see \seep{Nice}.

\function{SetPriority}{(Which,Who,Prio : Integer)}{Integer}
{
SetPriority sets the priority with which a process is running.
Which process(es) is determined by the \var{Which} and \var{Who} variables.
\var{Which} can be one of the pre-defined \var{Prio\_Process, Prio\_PGrp,
Prio\_User}, in which case \var{Who} is the process ID, Process group ID or
User ID, respectively.

\var{Prio} is a value in the range -20 to 20.
}
{
 Error checking must be done on LinuxError, since a priority can be negative.
 \begin{description}
 \item[sys\_esrch] No process found using \var{which} and \var{who}.
 \item[sys\_einval] \var{Which} was not one of \var{Prio\_Process, Prio\_Grp
or Prio\_User}.
 \item[sys\_eperm] A process was found, but neither its effective or real
 user ID match the effective user ID of the caller.
 \item [sys\_eacces] A non-superuser tried to a priority increase.
 \end{description}
 }
{\seef{GetPriority}, \seep{Nice}, \seem{Setpriority}{2}}

For an example, see \seep{Nice}.

\function{GetPid}{}{Longint}
{ Get the Process ID of the currently running process.}
{None.}
{\seef{GetPPid}, \seem{getpid}{2}}

\input{linuxex/ex16.tex}


\function{GetPPid}{}{Longint}
{ Get the Process ID of the parent process.}
{None.}
{\seef{GetPid}, \seem{getppid}{2}}

\input{linuxex/ex16.tex}

\function{GetUid}{}{Longint}
{ Get the real user ID of the currently running process.}
{None.}
{\seef{GetEUid}, \seem{getuid}{2} }

\input{linuxex/ex17.tex}

\function{GetEUid}{}{Longint}
{ Get the effective user ID of the currently running process.}
{None.}
{\seef{GetEUid}, \seem{geteuid}{2} }

\input{linuxex/ex17.tex}

\function{GetGid}{}{Longint}
{ Get the real group ID of the currently running process.}
{None.}
{\seef{GetEGid}, \seem{getgid}{2} }

\input{linuxex/ex18.tex}

\function{GetEGid}{}{Longint}
{ Get the effective group ID of the currently running process.}
{None.}
{\seef{GetGid}, \seem{getegid}{2} }

\input{linuxex/ex18.tex}

\function{IOperm}{(From,Num : Cadinal; Value : Longint)}{boolean}
{\var{IOperm}
  sets permissions on \var{Num} ports starting with port \var{From} to 
  \var{Value}. The function returns \var{True} if the call was successfull,
  \var{False} otherwise.

{\em Remark:}
\begin{itemize}
\item This works ONLY as root.
\item Only the first \var{0x03ff} ports can be set.
\item When doing a \seef{Fork}, the permissions are reset. When doing a
\seep{Execve} they are kept.
\end{itemize}
}{Errors are returned in \var{LinuxError}}{\seem{ioperm}{2}}

\function{Link}{(OldPath,NewPath : pathstr)}{Boolean}
{\var{Link} makes \var{NewPath} point to the same file als \var{OldPath}. The two files
then have the same inode number. This is known as a 'hard' link.

The function returns \var{True} if the call was succesfull, \var{False} if the call
failed.
}
{ Errors are returned in \var{LinuxError}.
\begin{description}
\item[sys\_exdev] \var {OldPath} and \var {NewPath} are not on the same
filesystem.
\item[sys\_eperm] The filesystem containing oldpath and newpath doesn't
support linking files.
\item[sys\_eaccess] Write access for the directory containing \var{Newpath}
is disallowed, or one of the directories in \var{OldPath} or {NewPath} has no
search (=execute) permission.
\item[sys\_enoent] A directory entry in \var{OldPath} or \var{NewPath} does
not exist or is a symbolic link pointing to a non-existent directory.
\item[sys\_enotdir] A directory entry in \var{OldPath} or \var{NewPath} is
nor a directory.
\item[sys\_enomem] Insufficient kernel memory.
\item[sys\_erofs] The files are on a read-only filesystem.
\item[sys\_eexist] \var{NewPath} already exists.
\item[sys\_emlink] \var{OldPath} has reached maximal link count.
\item[sys\_eloop] \var{OldPath} or \var{NewPath} has a reference to a circular
symbolic link, i.e. a symbolic link, whose expansion points to itself.
\item[sys\_enospc] The device containing \var{NewPath} has no room for anothe
entry.
\item[sys\_eperm] \var{OldPath} points to . or .. of a directory.
\end{description}
}
{\seef{SymLink}, \seef{UnLink}, \seem{Link}{2} }

\input{linuxex/ex21.tex}

\function{SymLink}{(OldPath,NewPath : pathstr)}{Boolean}
{\var{SymLink} makes \var{Newpath} point to the file in \var{OldPath}, which doesn't
necessarily exist. The two files DO NOT have the same inode number.
This is known as a 'soft' link.
The permissions of the link are irrelevant, as they are not used when
following the link. Ownership of the file is only checked in case of removal
or renaming of the link.

The function returns \var{True} if the call was succesfull, \var{False} if the call
failed.
}
{ Errors are returned in \var{LinuxError}.
\begin{description}
\item[sys\_eperm] The filesystem containing oldpath and newpath doesn't
support linking files.
\item[sys\_eaccess] Write access for the directory containing \var{Newpath}
is disallowed, or one of the directories in \var{OldPath} or {NewPath} has no
search (=execute) permission.
\item[sys\_enoent] A directory entry in \var{OldPath} or \var{NewPath} does
not exist or is a symbolic link pointing to a non-existent directory.
\item[sys\_enotdir] A directory entry in \var{OldPath} or \var{NewPath} is
nor a directory.
\item[sys\_enomem] Insufficient kernel memory.
\item[sys\_erofs] The files are on a read-only filesystem.
\item[sys\_eexist] \var{NewPath} already exists.
\item[sys\_eloop] \var{OldPath} or \var{NewPath} has a reference to a circular
symbolic link, i.e. a symbolic link, whose expansion points to itself.
\item[sys\_enospc] The device containing \var{NewPath} has no room for anothe
entry.
\end{description}
}
{\seef{Link}, \seef{UnLink}, \seem{Symlink}{2} }

\input{linuxex/ex22.tex}

\function{UnLink}{(Var Path)}{Boolean}
{
\var{UnLink} decreases the link count on file \var{Path}. \var{Path} can be
of type \var{PathStr} or \var{PChar}. If the link count is zero, the
file is removed from the disk.

The function returns \var{True} if the call was succesfull, \var{False} if the call
failed.
}
{ Errors are returned in \var{LinuxError}.
\begin{description}
\item[sys\_eaccess] You have no write access right in the directory
containing \var{Path}, or you have no search permission in one of the
directory components of \var{Path}.
\item[sys\_eperm] The  directory containing pathname has the sticky-bit 
set and the process's effective  uid is neither the uid of the 
file to be deleted nor that of the directory containing it.
\item[sys\_enoent] A component of the path doesn't exist.
\item[sys\_enotdir] A directory component of the path is not a directory.
\item[sys\_eisdir] \var{Path} refers to a directory.
\item[sys\_enomem] Insufficient kernel memory.
\item[sys\_erofs] \var{Path} is on a read-only filesystem. 
\end{description}
}
{\seef{Link}, \seef{SymLink}, \seem{Unlink}{2} }

For an example, see \seef{Link}.

\function{Chown}{(Path : Pathstr;NewUid,NewGid : Longint)}{Boolean}
{ \var{Chown} sets the User ID and Group ID of the file in \var{Path} to \var{NewUid,
NewGid}.

The function returns \var{True} if the call was succesfull, \var{False} if the call
failed.
}
{
Errors are returned in \var{LinuxError}.
\begin{description}
\item[sys\_eperm] The effective UID doesn't match the ownership of the file,
and is not zero. Owner or group were not specified correctly.
\item[sys\_eaccess] One of the directories in \var{Path} has no
search (=execute) permission.
\item[sys\_enoent] A directory entry in \var{Path} does
not exist or is a symbolic link pointing to a non-existent directory.
\item[sys\_enotdir] A directory entry in \var{OldPath} or \var{NewPath} is
nor a directory.
\item[sys\_enomem] Insufficient kernel memory.
\item[sys\_erofs] The file is on a read-only filesystem.
\item[sys\_eloop] \var{Path} has a reference to a circular
symbolic link, i.e. a symbolic link, whose expansion points to itself.
\end{description}
}
{\seef{Chmod}, \seef{Access}, \seem{Chown}(2)}

\input{linuxex/ex24.tex}

\function{Chmod}{(Path : Pathstr;NewMode : Longint)}{Boolean}
{ \var{Chmod}
Sets the Mode bits of the file in \var{Path} to \var{NewMode}. Newmode can be
specified by 'or'-ing the following:
\begin{description}
\item[S\_ISUID] Set user ID on execution.
\item[S\_ISGID] Set Group ID on execution.
\item[S\_ISVTX] Set sticky bit.
\item[S\_IRUSR] Read by owner.
\item[S\_IWUSR] Write by owner.
\item[S\_IXUSR] Execute by owner.
\item[S\_IRGRP] Read by group.
\item[S\_IWGRP] Write by group.
\item[S\_IXGRP] Execute by group.
\item[S\_IROTH] Read by others.
\item[S\_IWOTH] Write by others.
\item[S\_IXOTH] Execute by others.
\item[S\_IRWXO] Read, write, execute by others.
\item[S\_IRWXG] Read, write, execute by groups.
\item[S\_IRWXU] Read, write, execute by user.
\end{description}
}
{
Errors are returned in \var{LinuxError}.
\begin{description}
\item[sys\_eperm] The effective UID doesn't match the ownership of the file,
and is not zero. Owner or group were not specified correctly.
\item[sys\_eaccess] One of the directories in \var{Path} has no
search (=execute) permission.
\item[sys\_enoent] A directory entry in \var{Path} does
not exist or is a symbolic link pointing to a non-existent directory.
\item[sys\_enotdir] A directory entry in \var{OldPath} or \var{NewPath} is
nor a directory.
\item[sys\_enomem] Insufficient kernel memory.
\item[sys\_erofs] The file is on a read-only filesystem.
\item[sys\_eloop] \var{Path} has a reference to a circular
symbolic link, i.e. a symbolic link, whose expansion points to itself.
\end{description}
}
{\seef{Chown}, \seef{Access}, \seem{Chmod}(2)}

\input{linuxex/ex23.tex}

\function{Utime}{(path : pathstr; utim : utimbuf)}{Boolean}
{
\var{Utime} sets the access and modification times of a file.
the \var{utimbuf} record contains 2 fields, \var{actime}, and \var{modtime},
both of type Longint. They should be filled with an epoch-like time,
specifying, respectively, the last access time, and the last modification
time. 

For some filesystem (most notably, FAT), these times are the same. 
}
{Errors are returned in \var{LinuxError}.
\begin{description}
\item[sys\_eaccess] One of the directories in \var{Path} has no
search (=execute) permission.
\item[sys\_enoent] A directory entry in \var{Path} does
not exist or is a symbolic link pointing to a non-existent directory.
\end{description}
Other errors may occur, but aren't documented.
}
{\seef{GetEpochTime}, \seef{Chown}, \seef{Access}, \seem{utime}(2)}

\input{linuxex/ex25.tex}

\function{Umask}{(Mask : Integer)}{Integer}
{
Change the file creation mask for the current user to \var{Mask}. The
current mask is returned.
}
{None}
{\seef{Chmod}, \seem{Umask}{2}}

\input{linuxex/ex27.tex}

\function{Access}{(Path : Pathstr; Mode : integer)}{Boolean}
{
Tests user's access rights on the specified file. Mode is a mask existing of
one or more of
\begin{description}
\item[R\_OK] User has read rights.
\item[W\_OK] User has write rights.
\item[X\_OK] User has execute rights.
\item[F\_OK] User has search rights in the directory where the file is.
\end{description}
The test is done with the real user ID, instead of the effective user ID.

If access is denied, or an error occurred, false is returned.
}
{ \var{LinuxError} is used to report errors:
\begin{description}
\item[sys\_eaccess] The requested access is denied, either to the file or one
of the directories in its path.
\item[sys\_einval] \var{Mode} was incorrect.
\item[sys\_enoent] A directory component in \var{Path} doesn't exist or is a
dangling symbolic link.
\item[sys\_enotdir] A directory component in \var{Path} is not a directory.
\item[sys\_enomem] Insufficient kernel memory.
\item[sys\_eloop] \var{Path} has a circular symbolic link.
\end{description}
}
{\seef{Chown}, \seef{Chmod}, \seem{Access}{2} }

\input{linuxex/ex26.tex}

\function{FStat}{(Path : Pathstr; Var Info : stat)}{Boolean}
{
\var{FStat} gets information about the file specified in \var{Path}, and stores it in 
\var{Info}, which is of type \var{stat}.

The function returns \var{True} if the call was succesfull, \var{False} if the call
failed.
}
{ \var{LinuxError} is used to report errors.
\begin{description}
\item[sys\_enoent] \var{Path} does not exist.
\end{description}
}
{\seef{FSStat}, \seef{LStat}, \seem{stat}{2}}

\input{linuxex/ex28.tex}

\function{LStat}{(Path : Pathstr; Var Info : stat)}{Boolean}
{
\var{LStat} gets information about the link specified in \var{Path}, and stores it in 
\var{Info}, which is of type \var{stat}. Contrary to \var{FStat}, it stores
information about the link, not about the file the link points to.

The function returns \var{True} if the call was succesfull, \var{False} if the call
failed.
}
{ \var{LinuxError} is used to report errors.
\begin{description}
\item[sys\_enoent] \var{Path} does not exist.
\end{description}
}
{\seef{FStat}, \seef{FSStat}, \seem{stat}{2}}

\input{linuxex/ex29.tex}

\function{FSStat}{(Path : Pathstr; Var Info : statfs)}{Boolean}
{ Return in \var{Info} information about the filesystem on which the file
\var{Path} resides. Info is of type \var{statfs}.

The function returns \var{True} if the call was succesfull, \var{False} if the call
failed.
}
{ \var{LinuxError} is used to report errors.
\begin{description}
\item[sys\_enotdir] A component of \var{Path} is not a directory.
\item[sys\_einval] Invalid character in \var{Path}.
\item[sys\_enoent] \var{Path} does not exist.
\item[sys\_eaccess] Search permission is denied for  component in
\var{Path}.
\item[sys\_eloop] A circular symbolic link was encountered in \var{Path}.
\item[sys\_eio] An error occurred while reading from the filesystem.
\end{description}
}
{\seef{FStat}, \seef{LStat}, \seem{statfs}{2}}

\input{linuxex/ex30.tex}

\procedure{IOCtl}{(Handle,Ndx: Longint; Data: Pointer)}
{
This is a general interface to the Unix/ \linux ioctl call.
It performs various operations on the filedescriptor \var{Handle}.
\var{Ndx} describes the operation to perform.
\var{Data} points to data needed for the \var{Ndx} function. 
The structure of this data is function-dependent, so we don't elaborate on
this here. 

For more information on this, see various manual pages under linux.
}
{
Errors are reported in LinuxError. They are very dependent on the used
function, that's why we don't list them here
}
{\seem{ioctl}{2}}

\input{linuxex/ex54.tex}

\function{IsATTY}{(var f)}{Boolean};
{
Check if the filehandle described by \var{f} is a terminal.
f can be of type
\begin{enumerate}
\item \var{longint} for file handles;
\item \var{Text} for \var{text} variables such as \var{input} etc.
\end{enumerate}

Returns \var{True} if \var{f} is a terminal, \var{False} otherwise.
}
{No errors are reported}
{\seep{IOCtl},\seef{TTYName}}

\function{TTYName}{(var f)}{String}
{
Returns the name of the terminal pointed to by \var{f}. \var{f}
must be a terminal. \var{f} can be of type:
\begin{enumerate}
\item \var{longint} for file handles;
\item \var{Text} for \var{text} variables such as \var{input} etc.
\end{enumerate}
}
{ Returns an empty string in case of an error. \var{Linuxerror} may be set
 to indicate what error occurred, but this is uncertain.}
{\seef{IsATTY},\seep{IOCtl}}

\function{FExpand}{(Const Path: Pathstr)}{pathstr}
{ Expands \var {Path} to a full path, starting from root,
eliminating directory references such as . and .. from the result.
}
{None}
{\seef{BaseName},\seef{DirName} }

\input{linuxex/ex45.tex}

\function{FSearch}{(Path : pathstr;DirList : string)}{Pathstr}
{ Searches in \var{DirList}, a colon separated list of directories,
for a file named \var{Path}. It then returns a path to the found file.}
{An empty string if no such file was found.}
{\seef{BaseName}, \seef{DirName}, \seef{FExpand} }

\input{linuxex/ex46.tex}

\function{BaseName}{(Const Path;Suf : Pathstr)}{Pathstr}
{Returns the filename part of \var{Path}, stripping off \var{Suf} if it
exists.

The filename part is the whole name if \var{Path} contains no slash,
or the part of \var{Path} after the last slash.

The last character of the result is not a slash, unless the directory is the
root directory.
}
{None.}
{\seef{DirName}, \seef{FExpand}, \seem{Basename}{1}}

\input{linuxex/ex48.tex}

\function{DirName}{(Const Path : Pathstr)}{Pathstr}
{Returns the directory part of \var{Path}.

The directory is the part of \var{Path} before the last slash,
or empty if there is no slash.

The last character of the result is not a slash, unless the directory is the
root directory.
}
{None.}
{\seef{BaseName}, \seef{FExpand}, \seem{Dirname}{1}}

\input{linuxex/ex47.tex}

\function{Glob}{(Const Path : Pathstr)}{PGlob}
{
Glob returns a pointer to a glob structure which contains all filenames which
exist and match the pattern in \var{Path}.

The pattern can contain wildcard characters, which have their
usual meaning.

The pglob structure is defined as :
Some more text.
}
{ Returns nil on error, and \var{LinuxError} is set.
\begin{description}
\item[sys\_enomem] No memory on heap for glob structure.
\item[others] As returned by the opendir call, and sys\_readdir.
\end{description}
}
{\seep{GlobFree}, \seem{Glob}{3} }

\input{linuxex/ex49.tex}

 \procedure{GlobFree}{(Var P : Pglob)}
 {Releases the memory, occupied by a pglob structure. \var{P} is set to nil.}{None}
 { \seef{Glob} }

For an example, see \seef{Glob}.

\procedure{AssignPipe}{(Pipe\_in, Pipe\_out : Text)}
{\var{AssignePipe} creates a pipe, i.e. two file objects, one for input, one for output.
What is written to \var{Pipe\_out}, can be read from \var{Pipe\_in}.
Reading and writing happens through the usual \var{Readln(Pipe\_in,...)} and
\var{Writeln (Pipe\_out,...)} procedures.
}
{ \var{LinuxError} is used to report errors:
\begin{description}
\item[sys\_emfile] Too many file descriptors for this process.
\item[sys\_enfile] The system file table is full.
\end{description}
}
{\seep{POpen}, \seef{MkFifo}, \seem{pipe}{2}}

\input{linuxex/ex36.tex}

\function{MkFifo}{(PathName: String; Mode : Longint)}{Boolean}
{\var{MkFifo} creates named a named pipe in the filesystem, with name
\var{PathName} and mode {Mode}. 
}
{ \var{LinuxError} is used to report errors:
\begin{description}
\item[sys\_emfile] Too many file descriptors for this process.
\item[sys\_enfile] The system file table is full.
\end{description}
}
{\seep{POpen}, \seef{MkFifo}, \seem{mkfifo}{4}}


\procedure{AssignStream}{(StreamIn,StreamOut : Text; Const prog : String)}
{\var{AssignStream} creates a 2 pipes, i.e. two file objects, one for input, one for
output, the other ends of these pipes are connected to standard input and and
output of \var{Prog}. \var{Prog} is the name of a program (including path)
with options, which will be executed.
What is written to \var{StreamOut}, will go to the standard input of
\var{Prog}. Whatever is written by \var{Prog} to it's standard output be read from
\var{StreamIn}.
Reading and writing happens through the usual \var{Readln(StreamIn,...)} and
\var{Writeln (StreamOut,...)} procedures.
}
{ \var{LinuxError} is used to report errors:
\begin{description}
\item[sys\_emfile] Too many file descriptors for this process.
\item[sys\_enfile] The system file table is full.
\end{description}
Other errors include the ones by the fork and exec programs
}
{\seep{AssignPipe}, \seep{POpen},\seem{pipe}{2}}

\input{linuxex/ex38.tex}

\procedure {POpen}{(Var F : FileType; Cmd : pathstr; rw : char)}
{ Popen runs the command specified in \var{Cmd},
 and redirects the standard in or output of the
command to the other end of the pipe \var{F}. The parameter \var{rw}
indicates the direction of the pipe. If it is set to \var{'W'}, then F can
be used to write data, which will then be read by the command from stdinput.
If it is set to \var{'R'}, then the standard output of the command can be 
read from \var{F}. \var{F} should be reset or rewritten prior to using it.

\var{F} can be of type \var{Text} or \var{File}.

A file opened with \var {POpen} can be closed with \var{Close}, but also
with \seef{PClose}. The result is the same, but \var{PClose} returns the
exit status of the command \var{Cmd}.}
{Errors are reported in \var{LinuxError} and are essentially those of the
Execve, Dup and AssignPipe commands.
}
{\seep{AssignPipe}, \seem{popen}{3}, \seef{PClose}}

\input{linuxex/ex37.tex}

\function{PClose}{(Var F : FileType)}{longint}
{ \var{PClose} closes a file opened with \var{POpen}. It waits for the
command to complete, and then returns the exit status of the command. 
}
{\var{LinuxError} is used to report errors. If it is different from zero,
the exit status is not valid.}
{\seep{POpen}}

For an example, see \seep{POpen}

\function{Fcntl}{(Fd :  text, Cmd : Integer)}{Integer}
{
Read a file's attributes. \var{Fd} is an assigned file.
\var{Cmd} speciefies what to do, and is one of the following:
\begin{description}
\item[F\_GetFd] Read the close\_on\_exec flag. If the low-order bit is 0, then
the file will remain open across execve calls.
\item[F\_GetFl] Read the descriptor's flags.
\item[F\_GetOwn] Get the Process ID of the owner of a socket.
\end{description}
}
{
\var{LinuxError} is used to report errors.
\begin{description}
\item[sys\_ebadf] \var{Fd} has a bad file descriptor.
\end{description}
}
{\seep{Fcntl}, \seem{Fcntl}{2} }

\procedure{Fcntl}{(Fd :  text, Cmd : Integer; Arg : longint)}
{
Read or Set a file's attributes. \var{Fd} is an assigned file.
\var{Cmd} speciefies what to do, and is one of the following:
\begin{description}
\item[F\_SetFd] Set the close\_on\_exec flag of \var{Fd}. (only the least
siginificant bit is used).
\item[F\_GetLk] Return the \var{flock} record that prevents this process from
obtaining the lock, or set the \var{l\_type} field of the lock of there is no
obstruction. Arg is a pointer to a flock record.
\item[F\_SetLk] Set the lock or clear it (depending on \var{l\_type} in the
\var{flock} structure). if the lock is held by another process, an error
occurs.
\item[F\_GetLkw] Same as for \textbf{F\_Setlk}, but wait until the lock is
released.
\item[F\_SetOwn] Set the Process or process group that owns a socket.
\end{description}
}
{
\var{LinuxError} is used to report errors.
\begin{description}
\item[sys\_ebadf] \var{Fd} has a bad file descriptor.
\item[sys\_eagain or sys\_eaccess] For \textbf{F\_SetLk}, if the lock is
held by another process.
\end{description}
}
{\seef{Fcntl}, \seem{Fcntl}{2} }

\procedure{Dup}{(Var OldFile, NewFile : Text)}
{
Makes \var{NewFile} an exact copy of \var{OldFile}, after having flushed the
buffer of \var{OldFile}. Due to the buffering mechanism of Pascal, this has not
the same functionality as the \seem{dup}{2} call in C. The internal Pascal
buffers are not the same after this call, but when the buffers are flushed
(e.g. after output), the output is sent to the same file.
Doing an lseek will, however, work as in C, i.e. doing a lseek will change the
fileposition in both files.
}
{ \var{Linuxerror} is used to report errors.
\begin{description}
\item[sys\_ebadf] \var{OldFile} hasn't been assigned.
\item[sys\_emfile] Maximum number of open files for the process is reached.
\end{description}
}
{\seep{Dup2}, \seem{Dup}{2} }

\input{linuxex/ex31.tex}

\procedure{Dup2}{(Var OldFile, NewFile : Text)}
{
Makes \var{NewFile} an exact copy of \var{OldFile}, after having flushed the
buffer of \var{OldFile}. \var{NewFile} can be an assigned file.
If \var{newfile} was open, it is closed first.
Due to the buffering mechanism of Pascal, this has not
the same functionality as the \seem{dup2}{2} call in C. The internal Pascal
buffers are not the same after this call, but when the buffers are flushed
(e.g. after output), the output is sent to the same file.
Doing an lseek will, however, work as in C, i.e. doing a lseek will change the
fileposition in both files.
}
{ \var{Linuxerror} is used to report errors.
\begin{description}
\item[sys\_ebadf] \var{OldFile} hasn't been assigned.
\item[sys\_emfile] Maximum number of open files for the process is reached.
\end{description}
}
{ \seep{Dup}, \seem{Dup2}{2} }

\input{linuxex/ex32.tex}

\procedure{SigAction}{(Signum : Integer; Var Act,OldAct : PSigActionRec)}
{ Changes the action to take upon receipt of a signal. \var{Act} and
\var{Oldact} are pointers to a \var{SigActionRec} record.

\var{SigNum} specifies the signal, and can be any signal except
\textbf{SIGKILL} or \textbf{SIGSTOP}.

If \var{Act} is non-nil, then the new action for signal \var{SigNum} is taken
from it. If \var{OldAct} is non-nil, the old action is stored there.

\var{Sa\_Handler} may be \var{SIG\_DFL} for the default action or
\var{SIG\_IGN} to ignore the signal.

\var{Sa\_Mask} Specifies which signals should be ignord during the execution
of the signal handler.

\var{Sa\_Flags} Speciefies a series of flags which modify the behaviour of
the signal handler. You can 'or' none or more of the following :
\begin{description}
\item[SA\_NOCLDSTOP] If signum is \textbf{SIGCHLD} do not receive
notification when child processes stop.
\item[SA\_ONESHOT or SA\_RESETHAND] Restore the signal action to the default
state once the signal handler has been called.
\item[SA\_RESTART] For compatibility with BSD signals.
\item[SA\_NOMASK or SA\_NODEFER] Do not prevent the signal from being received
from within its own signal handler.
\end{description}
}
{\var{LinuxError} is used to report errors.
\begin{description}
\item[sys\_einval] an invalid signal was specified, or it was
\textbf{SIGKILL} or \textbf{SIGSTOP}.
\item[sys\_efault] \var{Act,OldAct} point outside this process address space
\item[sys\_eintr] System call was interrupted.
\end{description}
}
{
\seep{SigProcMask}, \seef{SigPending}, \seep{SigSuspend}, \seef{Kill},
\seem{Sigaction}{2}
}

\procedure {SigProcMask}{(How : Integer; SSet,OldSSet : PSigSet)}
{
Changes the list of currently blocked signals. The behaviour of the call
depends on \var{How} :
\begin{description}
\item[SIG\_BLOCK] The set of blocked signals is the union of the current set
and the \var{SSet} argument.
\item[SIG\_UNBLOCK] The signals in \var{SSet} are removed from the set of
currently blocked signals.
\item[SIG\_SETMASK] The list of blocked signals is set so \var{SSet}.
\end{description}
If \var{OldSSet} is non-nil, then the old set is stored in it.
}
{\var{LinuxError} is used to report errors.
\begin{description}
\item[sys\_efault] \var{SSet} or \var{OldSSet} point to an adress outside
the range of the process.
\item[sys\_eintr] System call was interrupted.
\end{description}
}
{\seep{SigAction}, \seef{SigPending}, \seep{SigSuspend}, \seef{Kill},
\seem{Sigprocmask}{2} }

\function{SigPending}{}{SigSet}
{
Sigpending allows the examination of pending signals (which have been raised
while blocked.) The signal mask of pending signals is returned.
}
{None}
{\seep{SigAction}, \seep{SigProcMask}, \seep{SigSuspend}, \seef{Signal},
\seef{Kill}, \seem{Sigpending}{2} }

\procedure{SigSuspend}{(Mask : SigSet)}
{SigSuspend temporarily replaces the signal mask for the process with the one
given in \var{Mask}, and then suspends the process until a signal is received.
}
{None}
{\seep{SigAction}, \seep{SigProcMask}, \seef{SigPending}, \seef{Signal},
\seef{Kill}, \seem{SigSuspend}{2} }

\function{Signal}{(SigNum : Integer; Handler : PSignalHandler)}{PSignalHandler}
{
Signal installs a new signal handler for signal \var{SigNum}. This call has
the same functionality as the \textbf{SigAction} call.

The return value for Signal is the old signal handler, or nil on error.
}
{\var {LinuxError} is used to report errors :
\begin{description}
\item[SIG\_ERR] An error occurred.
\end{description}
}
{\seep{SigAction},\seef{Kill}, \seem{Signal}{2} }

\function{Kill}{Pid : Longint; Sig : Integer)}{Integer}
{ Send a signal \var{Sig} to a process or process group. If \var{Pid}>0 then
the signal is sent to \var{Pid}, if it equals -1, then the signal is sent to
all processes except process 1. If \var{Pid}<-1 then the signal is sent to
process group -Pid.

The return value is zero, except in case three, where the return value is the
number of processes to which the signal was sent.
}
{\var{LinuxError} is used to report errors:
\begin{description}
\item[sys\_einval] An invalid signal is sent.
\item[sys\_esrch] The \var{Pid} or process group don't exist.
\item[sys\_eperm] The effective userid of the current process doesn't math
the one of process \var{Pid}.
\end{description}
}
{\seep{SigAction}, \seef{Signal}, \seem{Kill}{2} }

\function{GetHostName}{}{String}
{
Get the hostname of the machine on which the process is running.
An empty string is returned if hostname is not set.
}
{None.}
{ \seef{GetDomainName},seem{Gethostname}{2} }

\input{linuxex/ex40.tex}

\function{GetDomainName}{}{String}
{
Get the domain name of the machine on which the process is running.
An empty string is returned if the domain is not set.
}
{None.}
{ \seef{GetHostName},seem{Getdomainname}{2} }

\input{linuxex/ex39.tex}

\function{GetEnv}{(P : String)}{PChar}
{Returns the value of the environment variable in \var{P}. If the variable is
not defined, nil is returned. The value of the environment variable may be
the empty string.

A PChar is returned to accomodate for strings longer than 255 bytes,
\var{TERMCAP} and \var{LS\_COLORS}, for instance.
}
{None.}
{\seem{sh}{1}, \seem{csh}{1} }

\input{linuxex/ex41.tex}

\function{Select}{(N : Longint; \\ var readfds,writefds,exceptfds : PFDset;
Var Timeout)}{Longint}
{\var{Select} checks one of the file descriptors in the \var{FDSets} to see if its
status changed.

\var{readfds, writefds} and \var{exceptfds} are pointers to arrays of 256
bits. If you want a file descriptor to be checked, you set the
corresponding element in the array to 1. The other elements in the array
must be set to zero. Three arrays are passed : The entries in \var{readfds}
are checked to see if characters become available for reading. The entries
in \var{writefds} are checked to see if it is OK to write to them, while
entries in \var{exceptfds} are cheked to see if an exception occorred on
them.

You can use the functions \seepl{FD\_Clear}{FDClear}, \seepl{FD\_Clr}{FDClr}, 
\seepl{FD\_Set}{FDSet}, \seefl{FD\_IsSet}{FDIsSet} to manipulate the individual elements of a set.

The pointers can be nil.

\var{N} is the largest index of a nonzero entry plus 1. (= the largest
file-descriptor + 1).

\var{TimeOut} can be used to set a time limit. 
If \var{TimeOut} can be two types :
\begin{enumerate}
\item \var{TimeOut} is of type \var{PTime} and contains a
zero time, the call returns immediately. If \var{TimeOut} is \var{Nil}, the
kernel will wait forever, or until a status changed.    
\item \var{TimeOut} is of type \var{Longint}. If it is -1, this has the same
effect as a \var{Timeout} of type  \var{PTime} which is \var{Nil}.
Otherwise, \var{TimeOut} contains a time in milliseconds.
\end{enumerate}
 
When the TimeOut is reached, or one of the file descriptors has changed,
the \var{Select} call returns. On return, it will have modified the entries
in the array which have actually changed, and it returns the number of
entries that have been changed. If the timout was reached, and no decsriptor
changed, zero is returned; The arrays of indexes are undefined after that.
On error, -1 is returned.}
{On error, the function returns -1, and Errors are reported in LinuxError :
\begin{description}
\item[SYS\_EBADF\ ] An invalid descriptot was specified in one of the sets.
\item[SYS\_EINTR\ ] A non blocked signal was caught.
\item[SYS\_EINVAL\ ]  \var{N} is negative or too big.
\item[SYS\_ENOMEM\ ] \var{Select} was unable to allocate memory for its 
 internal tables.
\end{description}}
{\seef{SelectText}, \seef{GetFS}, 
\seepl{FD\_Clear}{FDClear},
\seepl{FD\_Clr}{FDClr},
\seepl{FD\_Set}{FDSet}, 
\seefl{FD\_IsSet}{FDIsSet}}

\input{linuxex/ex33.tex}

\function{SelectText}{( var T : Text; TimeOut :PTime)}{Longint}
{\var{SelectText} executes the \seef{Select} call on a file of type
\var{Text}. You can specify a timeout in \var{TimeOut}. The SelectText call
determines itself whether it should check for read or write, depending on
how the file was opened : With \var{Reset} it is checked for reading, with
\var{Rewrite} and \var{Append} it is checked for writing.}
{See \seef{Select}. \var{SYS\_EBADF} can also mean that the file wasn't
opened.}
{\seef{Select}, \seef{GetFS}}
\function{GetFS}{(Var F : Any File Type)}{Longint}
{\var{GetFS} returns the file selector that the kernel provided for your
file. In principle you don' need this file selector. Only for some calls
it is needed, such as the \seef{Select} call or so.}
{In case the file was not opened, then -1 is returned.}
{\seef{Select}}

\input{linuxex/ex34.tex}

\procedurel{FD\_Clear}{FDClear}{(var fds:fdSet)}
{\var{FD\_Clear} clears all the filedescriptors in the file descriptor 
set \var{fds}.}
{None.}
{\seef{Select}, 
\seef{SelectText}, 
\seef{GetFS}, 
\seepl{FD\_Clr}{FDClr},
\seepl{FD\_Set}{FDSet}, 
\seefl{FD\_IsSet}{FDIsSet}

}

For an example, see \seef{Select}.

\procedurel{FD\_Clr}{FDClr}{(fd:longint;var fds:fdSet)}
{ \var{FD\_Clr} clears file descriptor \var{fd} in filedescriptor s
  et \var{fds}.}
{None.}
{\seef{Select}, 
\seef{SelectText}, 
\seef{GetFS},
\seepl{FD\_Clear}{FDClear}, 
\seepl{FD\_Set}{FDSet}, 
\seefl{FD\_IsSet}{FDIsSet}}

For an example, see \seef{Select}.

\procedurel{FD\_Set}{FDSet}{(fd:longint;var fds:fdSet)}
{\var{FD\_Set} sets file descriptor \var{fd} in filedescriptor set \var{fds}.}
{None.}
{\seef{Select}, \seef{SelectText}, \seef{GetFS},\seepl{FD\_Clear}{FDClear}, 
\seepl{FD\_Clr}{FDClr}, \seefl{FD\_IsSet}{FDIsSet}}

For an example, see \seef{Select}.

\functionl{FD\_IsSet}{FDIsSet}{(fd:longint;var fds:fdSet)}{boolean}
{\var{FD\_Set} Checks whether file descriptor \var{fd} in filedescriptor set \var{fds}
is set.}
{None.}
{\seef{Select}, \seef{SelectText}, \seef{GetFS},
\seepl{FD\_Clear}{FDClear}, 
\seepl{FD\_Clr}{FDClr},
\seepl{FD\_Set}{FDSet}}

For an example, see \seef{Select}.

\function{fdOpen}{(Var PathName;flags:longint[; Mode: longint])}{longint}
{ \var{fdOpen} opens a file in \var{pathname} with flags \var{flags} a ORed combination of
  \var{Open\_Accmode, Open\_RdOnly, Open\_WrOnly, Open\_RdWr, Open\_Creat,
  Open\_Excl, Open\_NoCtty, Open\_Trunc, Open\_Append, Open\_NonBlock,
  Open\_NDelay, Open\_Sync} \var{PathName} can be of type \var{PChar} or
\var{String}

  The optional \var{mode} argument specifies the permissions to set when opening
  the file. This is modified by the umask setting. The real permissions are
  \var{Mode and not umask}.

  The return value of the function is the filedescriptor, or a negative 
  value if there was an error.
}
{Errors are returned in LinuxError}
{\seef{fdClose}, \seef{fdRead}, \seef{fdWrite},\seef{fdTruncate},
\seef{fdFlush}, \seef{fdSeek}}

\input{linuxex/ex19.tex}

\function{fdClose}{(fd:longint)}{boolean}
{
\var{fdClose} closes a file with file descriptor \var{Fd}. The function
returns \var{True} if the file was closed successfully, \var{False}
otherwise. 
}
{Errors are returned in LinuxError}
{\seef{fdOpen}, \seef{fdRead}, \seef{fdWrite},\seef{fdTruncate},
\seef{fdFlush}, seef{FdSeek}}

For an example, see \seef{fdOpen}.

\function{fdRead}{(fd:longint;var buf;size:longint}{longint}
{ \var{fdRead} reads at most \var{size} bytes from the file descriptor
\var{fd}, and stores them in \var{buf}. 

The function returns the number of bytes actually read, or -1 if
an error occurred.

No checking on the length of \var{buf} is done.
}
{Errors are returned in LinuxError.}
{\seef{fdOpen}, \seef{fdClose}, \seef{fdWrite},\seef{fdTruncate},
\seef{fdFlush}, \seef{fdSeek}}

\input{linuxex/ex20.tex}

\function{fdWrite}{(fd:longint;var buf;size:longint}{longint}  
{\var{fdWrite} writes at most \var{size} bytes from \var{buf} to
file descriptor \var{fd}.

The function returns the number of bytes actually written, or -1 if an error
occurred.
}
{Errors are returned in LinuxError.}
{\seef{fdOpen}, \seef{fdClose}, \seef{fdRead},\seef{fdTruncate},
\seef{fdSeek}, \seef{fdFlush}}

\function{fdSeek}{(fd,Pos,SeekType:longint}{longint}  
{\var{fdSeek} sets the current fileposition of file \var{fd} to
\var{Pos}, starting from \var{SeekType}, which can be one of the following:
\begin{description}
\item [Seek\_Set] \ \var{Pos} is the absolute position in the file.
\item [Seek\_Cur] \ \var{Pos} is relative to the current position.
\item [Seek\_end] \ \var{Pos} is relative to the end of the file.
\end{description}

The function returns the new fileposition, or -1 of an error occurred.
}
{Errors are returned in LinuxError.}
{\seef{fdOpen}, \seef{fdWrite}, \seef{fdClose},
\seef{fdRead},\seef{fdTruncate},
 \seef{fdFlush}}

For an example, see \seef{fdOpen}.

\function{fdTruncate}{(fd,size:longint)}{boolean}
{\var{fdTruncate} sets the length of a file in \var{fd} on \var{size}
bytes, where \var{size} must be less than or equal to the current length of
the file in \var{fd}.

The function returns \var{True} if the call was successful, \var{false} if
an error occurred.}
{Errors are returned in LinuxError.}
{\seef{fdOpen}, \seef{fdClose}, \seef{fdRead},\seef{fdWrite},\seef{fdFlush},
\seef{fdSeek}}

\function{fdFlush}{(fd:Longint)}{boolean}
{\var{fdflush} flushes the Linux kernel file buffer, so the file is actually
written to disk. This is NOT the same as the internal buffer, maintained by
Free Pascal. 

The function returns \var{True} if the call was successful, \var{false} if
an error occurred.}
{Errors are returned in LinuxError.}
{\seef{fdOpen}, \seef{fdClose}, \seef{fdRead},\seef{fdWrite},
\seef{fdTruncate}, \seef{fdSeek}}

For an example, see \seef{fdRead}.

\functionl{S\_ISLNK}{ISLNK}{(m:integer)}{boolean}
{ \var{S\_ISLNK} checks the file mode \var{m} to see whether the file is a
symbolic link. If so it returns \var{True}
}
{\seef{FStat},
 \seefl{S\_ISREG}{ISREG},
 \seefl{S\_ISDIR}{ISDIR},
 \seefl{S\_ISCHR}{ISCHR},
 \seefl{S\_ISBLK}{ISBLK},
 \seefl{S\_ISFIFO}{ISFIFO},
 \seefl{S\_ISSOCK}{ISSOCK}
}

\input{linuxex/ex53.tex}

\functionl{S\_ISREG}{ISREG}{(m:integer)}{boolean}
{ \var{S\_ISREG} checks the file mode \var{m} to see whether the file is a
regular file. If so it returns \var{True}
}
{\seef{FStat},
 \seefl{S\_ISLNK}{ISLNK}, 
 \seefl{S\_ISDIR}{ISDIR},
 \seefl{S\_ISCHR}{ISCHR},
 \seefl{S\_ISBLK}{ISBLK},
 \seefl{S\_ISFIFO}{ISFIFO},
 \seefl{S\_ISSOCK}{ISSOCK}
}

For an example, see \seef{ISLNK}.

\functionl{S\_ISDIR}{ISDIR}{(m:integer)}{boolean}	
{ \var{S\_ISDIR} checks the file mode \var{m} to see whether the file is a
directory. If so it returns \var{True}
}
{\seef{FStat},
 \seefl{S\_ISLNK}{ISLNK}, 
 \seefl{S\_ISREG}{ISREG},
 \seefl{S\_ISCHR}{ISCHR},
 \seefl{S\_ISBLK}{ISBLK},
 \seefl{S\_ISFIFO}{ISFIFO},
 \seefl{S\_ISSOCK}{ISSOCK}
}

For an example, see \seef{ISLNK}.

\functionl{S\_ISCHR}{ISCHR}{(m:integer)}{boolean}
{ \var{S\_ISCHR} checks the file mode \var{m} to see whether the file is a
character device file. If so it returns \var{True}.
}
{\seef{FStat},
 \seefl{S\_ISLNK}{ISLNK}, 
 \seefl{S\_ISREG}{ISREG},
 \seefl{S\_ISDIR}{ISDIR},
 \seefl{S\_ISBLK}{ISBLK},
 \seefl{S\_ISFIFO}{ISFIFO},
 \seefl{S\_ISSOCK}{ISSOCK}
}

For an example, see \seef{ISLNK}.

\functionl{S\_ISBLK}{ISBLK}{(m:integer)}{boolean}
{ \var{S\_ISBLK} checks the file mode \var{m} to see whether the file is a
block device file. If so it returns \var{True}.
}
{\seef{FStat},
 \seefl{S\_ISLNK}{ISLNK}, 
 \seefl{S\_ISREG}{ISREG},
 \seefl{S\_ISDIR}{ISDIR},
 \seefl{S\_ISCHR}{ISCHR},
 \seefl{S\_ISFIFO}{ISFIFO},
 \seefl{S\_ISSOCK}{ISSOCK}
}

For an example, see \seef{ISLNK}.

\functionl{S\_ISFIFO}{ISFIFO}{(m:integer)}{boolean}
{ \var{S\_ISFIFO} checks the file mode \var{m} to see whether the file is a
fifo (a named pipe). If so it returns \var{True}.
}
{\seef{FStat},
 \seefl{S\_ISLNK}{ISLNK}, 
 \seefl{S\_ISREG}{ISREG},
 \seefl{S\_ISDIR}{ISDIR},
 \seefl{S\_ISCHR}{ISCHR},
 \seefl{S\_ISBLK}{ISBLK},
 \seefl{S\_ISSOCK}{ISSOCK}
}

For an example, see \seef{ISLNK}.

\functionl{S\_ISSOCK}{ISSOCK}{(m:integer)}{boolean}
{ \var{S\_ISSOCK} checks the file mode \var{m} to see whether the file is a
socket. If so it returns \var{True}.
}
{\seef{FStat},
 \seefl{S\_ISLNK}{ISLNK}, 
 \seefl{S\_ISREG}{ISREG},
 \seefl{S\_ISDIR}{ISDIR},
 \seefl{S\_ISCHR}{ISCHR},
 \seefl{S\_ISBLK}{ISBLK},
 \seefl{S\_ISFIFO}{ISFIFO}
}


For an example, see \seef{ISLNK}.

\function{OpenDir}{(f:pchar)}{pdir}
{ \var{OpenDir} opens the directory  \var{f}, and returns a \var{pdir}
pointer to a \var{Dir} record, which can be used to read the directory 
structure. If the directory cannot be opened, \var{nil} is returned.}
{Errors are returned in LinuxError.}
{\seef{CloseDir}, \seef{ReadDir}, \seep{SeekDir}, \seef{TellDir},
\seem{opendir}{3}}

\input{linuxex/ex35.tex}

\function{CloseDir}{(p:pdir)}{integer}
{ \var{CloseDir} closes the directory pointed to by \var{p}.
It returns zero if the directory was closed succesfully, -1 otherwise.}
{Errors are returned in LinuxError.}
{\seef{OpenDir}, \seef{ReadDir}, \seep{SeekDir}, \seef{TellDir},
\seem{closedir}{3}}

For an example, see \seef{OpenDir}.

\function{ReadDir}{(p:pdir)}{pdirent}
{\var{ReadDir} reads the next entry in the directory pointed to by \var{p}.
It returns a \var{pdirent} pointer to a structure describing the entry.
If the next entry can't be read, \var{Nil} is returned.
}
{Errors are returned in LinuxError.}
{\seef{CloseDir}, \seef{OpenDir}, \seep{SeekDir}, \seef{TellDir},
\seem{readdir}{3}}

For an example, see \seef{OpenDir}.

\procedure {SeekDir}{(p:pdir;off:longint)}
{ \var{SeekDir} sets the directory pointer to the \var{off}-th entry in the
directory structure pointed to by \var{p}.}
{Errors are returned in LinuxError.}
{\seef{CloseDir}, \seef{ReadDir}, \seef{OpenDir}, \seef{TellDir},
\seem{seekdir}{3}}

For an example, see \seef{OpenDir}.

\function{TellDir}{(p:pdir)}{longint}
{ \var{TellDir} returns the current location in the directory structure
pointed to by \var{p}. It returns -1 on failure.}
{Errors are returned in LinuxError.}
{\seef{CloseDir}, \seef{ReadDir}, \seep{SeekDir}, \seef{OpenDir},
\seem{telldir}{3}}

For an example, see \seef{OpenDir}.

\procedure{Uname}{(var unamerec:utsname)}
{\var{Uname} gets the name and configuration of the current \linux kernel,
and returns it in \var{unamerec}.
}
{\var{LinuxError} is used to report errors.}
{\seef{GetHostName}, \seef{GetDomainName}, \seem{uname}{2}}

\function{WaitPid}{(Pid : longint; Status : pointer; Options : Integer)}{Longint}
{ \var{WaitPid} waits for a child process with process ID \var{Pid} to exit. The
value of \var{Pid} can be one of the following:
\begin{description}
\item[Pid < -1] Causes \var{WaitPid} to wait for  any  child  process  whose
              process group ID equals the absolute value of \var{pid}.

\item[Pid = -1] Causes \var{WaitPid} to wait for any child process.

\item[Pid = 0] Causes \var{WaitPid} to wait for  any  child  process  whose
              process  group  ID  equals the one of the calling
              process.

\item[Pid > 0] Causes \var{WaitPid} to wait for the child whose process  ID
equals the value of \var{Pid}.
\end{description}
The \var{Options} parameter can be used to specify further how \var{WaitPid}
behaves:
\begin{description}
\item [WNOHANG] Causes \var{Waitpid} to return immediately if no child  has
exited.
\item [WUNTRACED] Causes \var{WaitPid} to return also for children which are
stopped, but whose status has not yet been reported.
\end{description} 

Upon return, it returns the exit status of the process, or -1 in case of
failure. 
}
{Errors are returned in LinuxError.}
{\seef{Fork}, \seep{Execve}, \seem{waitpid}{2}}

for an example, see \seef{Fork}.

\function{TCGetAttr}{(fd:longint;var tios:TermIOS)}{Boolean}
{ \var{TCGetAttr}
  gets the terminal parameters from the terminal referred to by the file
  descriptor \var{fd} and returns them in a \var{TermIOS} structure \var{tios}. 

The function returns \var{True} if the call was succesfull, \var{False}
otherwise.
}
{Errors are reported in LinuxError}
{\seef{TCSetAttr}, \seem{termios}{2} }

\input{linuxex/ex55.tex}

\function{TCSetAttr}{(Fd:longint;OptAct:longint;var Tios:TermIOS)}{Boolean}
{ \var{TCSetAttr}
  Sets the terminal parameters you specify in a \var{TermIOS} structure
\var{Tios} for the terminal
  referred to by the file descriptor \var{Fd}. \var{OptAct} specifies an 
  optional action when the set need to be done,
  this could be one of the following pre-defined values:
 \begin{description}
\item [TCSANOW\ ] set immediately.
\item [TCSADRAIN\ ] wait for output.
\item [TCSAFLUSH\ ] wait for output and discard all input not yet read. 
\end{description}
The function Returns \var{True} if the call was succesfull, \var{False} 
otherwise.
}
{Errors are reported in LinuxError.}
{\seef{TCGetAttr}, \seem{termios}{2}}

For an example, see \seef{TCGetAttr}.

\procedure{CFSetISpeed}{(var Tios:TermIOS;Speed:Longint)}
{ \var{CFSetISpeed}
  Sets the input baudrate in the \var{TermIOS} structure \var{Tios} to 
  \var{Speed}.
}
{None.}
{\seep{CFSetOSpeed}, \seep{CFMakeRaw}, \seem{termios}{2}}

\procedure{CFSetOSpeed}{(var Tios:TermIOS;Speed:Longint)}
{ \var{CFSetOSpeed}
  Sets the output baudrate in the \var{Termios} structure \var{Tios} to
  \var{Speed}.
}
{None.}
{\seep{CFSetISpeed}, \seep{CFMakeRaw}, \seem{termios}{2}}

\procedure{CFMakeRaw}{(var Tios:TermIOS)}
{ \var{CFMakeRaw}
  Sets the flags in the \var{Termios} structure \var{Tios} to a state so that 
  the terminal will function in Raw Mode.
}
{None.}
{ \seep{CFSetOSpeed}, \seep{CFSetISpeed}, \seem{termios}{2}}

For an example, see \seef{TCGetAttr}.

\function{TCSendBreak}{(Fd,Duration:longint)}{Boolean}
{ \var{TCSendBreak} 
  Sends zero-valued bits on an asynchrone serial connection decsribed by
  file-descriptor \var{Fd}, for duration \var{Duration}.

  The function returns \var{True} if the action was performed successfully,
\var{False} otherwise.
}
{Errors are reported in LinuxError.}
{\seem{termios}{2}}

Function TCSetPGrp(Fd,Id:longint):boolean;
{ \var{TCSetPGrp}
  Sets the Process Group Id to \var{Id}. 

  The function returns \var{True} if the call was successful, \var{False}
otherwise.
}
{Errors are returned in LinuxError.}
{\seef{TCGetPGrp}\seem{termios}{2}}

\function{TCGetPGrp}{(Fd:longint;var Id:longint)}{boolean}
{ \var{TCGetPGrp}
  returns the process group ID of a foreground process group in \var{Id} 

  The function returns \var{True} if the call was succesfull, \var{False}
  otherwise
}
{Errors are reported in LinuxError}
{\seem{termios}{2}}

\function{TCDrain}{(Fd:longint)}{Boolean}
{ \var{TCDrain}
  waits until all data to file descriptor \var{Fd} is transmitted.
  
  The function returns \var{True} if the call was succesfull, \var{False}
  otherwise.
}
{Errors are reported in LinuxError}
{\seem{termios}{2}}

\function {TCFlow}{(Fd,Act:longint)}{Boolean}
{ \var{TCFlow}
  suspends/resumes transmission or reception of data to or from the file
descriptor \var{Fd}, depending
  on the action \var {Act}. This can be one of the following pre-defined
values: 
\begin{description}
\item [TCOOFF\ ] suspend reception/transmission,
\item [TCOON\ ] resume  reception/transmission,
\item [TCIOFF\ ] transmit a stop character to stop input from the terminal, 
\item [TCION\ ] transmit start to resume input from the terminal.
\end{description}

The function returns \var{True} if the call was succesfull, \var{False}
otherwise.
}
{Errors are reported in LinuxError.}
{\seem{termios}{2}}

\function{TCFlush}{(Fd,QSel:longint)}{Boolean}
{ \var{TCFlush}
  discards all data sent or received to/from file descriptor \var{fd}. 
 \var{QSel} indicates which queue
  should be discard. It can be one of the following pre-defined values :
\begin{description}
\item [TCIFLUSH\ ] input,
\item [TCOFLUSH\ ] output,
\item [TCIOFLUSH\ ] both input and output.
\end{description}

The function returns \var{True} if the call was succesfull, \var{False}
otherwise.
}
{Errors are reported in LinuxError.}
{\seem{termios}{2}}

\procedure{FLock}{(Var F; Mode : longint)}
{\var{FLock} implements file locking. it sets or removes a lock on the file
\var{F}. F can be of type \var{Text} or \var{File}, or it can be a \linux
filedescriptor (a longint)

\var{Mode} can be one of the following constants :
\begin{description}
\item [LOCK\_SH] \ sets a shared lock.
\item [LOCK\_EX] \ sets an exclusive lock.
\item [LOCK\_UN] \ unlocks the file.
\item [LOCK\_NB] \ This can be OR-ed together with the other. If this is done
the application doesn't block when locking.
\end{description}
}
{Errors are reported in \var{LinuxError}.}
{\seef{Fcntl}, \seem{flock}{2}}
% the math unit
%
%   $Id$
%   This file is part of the FPC documentation.
%   Copyright (C) 2000 by Florian Klaempfl
%
%   The FPC documentation is free text; you can redistribute it and/or
%   modify it under the terms of the GNU Library General Public License as
%   published by the Free Software Foundation; either version 2 of the
%   License, or (at your option) any later version.
%
%   The FPC Documentation is distributed in the hope that it will be useful,
%   but WITHOUT ANY WARRANTY; without even the implied warranty of
%   MERCHANTABILITY or FITNESS FOR A PARTICULAR PURPOSE.  See the GNU
%   Library General Public License for more details.
%
%   You should have received a copy of the GNU Library General Public
%   License along with the FPC documentation; see the file COPYING.LIB.  If not,
%   write to the Free Software Foundation, Inc., 59 Temple Place - Suite 330,
%   Boston, MA 02111-1307, USA.
%
\chapter{The MATH unit}
\FPCexampledir{mathex}

This chapter describes the \file{math} unit. The \var{math} unit
was initially written by Florian Klaempfl. It provides mathematical
functions which aren't covered by the system unit.

This chapter starts out with a definition of all types and constants
that are defined, after which an overview is presented of the available 
functions, grouped by category, and the last part contains a 
complete explanation of each function.

The following things must be taken into account when using this unit:
\begin{enumerate}
\item This unit is compiled in Object Pascal mode so all
\var{integers} are 32 bit.
\item Some overloaded functions exist for data arrays of integers and
floats. When using the address operator (\var{@}) to pass an array of 
data to such a function, make sure the address is typecasted to the 
right type, or turn on the 'typed address operator' feature. failing to
do so, will cause the compiler not be able to decide which function you 
want to call.
\end{enumerate}

\section{Constants and types}

The following types are defined in the \file{math} unit:
\begin{verbatim}
Type
  Float = Extended;
  PFloat = ^FLoat
\end{verbatim}
All calculations are done with the Float type. This allows to
recompile the unit with a different float type to obtain a
desired precision. The pointer type is used in functions that accept
an array of values of arbitrary length.
\begin{verbatim}
Type
   TPaymentTime = (PTEndOfPeriod,PTStartOfPeriod);
\end{verbatim}
\var{TPaymentTime} is used in the financial calculations.
\begin{verbatim}
Type
   EInvalidArgument = Class(EMathError);
\end{verbatim}
The \var{EInvalidArgument} exception is used to report invalid arguments.

\section{Function list by category}
What follows is a listing of the available functions, grouped by category.
For each function there is a reference to the page where you can find the
function.
\subsection{Min/max determination}
Functions to determine the minimum or maximum of numbers:
\begin{funclist}
\funcref{max}{Maximum of 2 values}
\funcref{maxIntValue}{Maximum of an array of integer values}
\funcref{maxvalue}{Maximum of an array of values}
\funcref{min}{Minimum of 2 values}
\funcref{minIntValue}{Minimum of an array of integer values}
\funcref{minvalue}{Minimum of an array of values}
\end{funclist}
\subsection{Angle conversion}
\begin{funclist}
\funcref{cycletorad}{convert cycles to radians}
\funcref{degtograd}{convert degrees to grads}
\funcref{degtorad}{convert degrees to radians}
\funcref{gradtodeg}{convert grads to degrees}
\funcref{gradtorad}{convert grads to radians}
\funcref{radtocycle}{convert radians to cycles}
\funcref{radtodeg}{convert radians to degrees}
\funcref{radtograd}{convert radians to grads}
\end{funclist}
\subsection{Trigoniometric functions}
\begin{funclist}
\funcref{arccos}{calculate reverse cosine}
\funcref{arcsin}{calculate reverse sine}
\funcref{arctan2}{calculate reverse tangent}
\funcref{cotan}{calculate cotangent}
\procref{sincos}{calculate sine and cosine}
\funcref{tan}{calculate tangent}
\end{funclist}
\subsection{Hyperbolic functions}
\begin{funclist}
\funcref{arcosh}{caculate reverse hyperbolic cosine}
\funcref{arsinh}{caculate reverse hyperbolic sine}
\funcref{artanh}{caculate reverse hyperbolic tangent}
\funcref{cosh}{calculate hyperbolic cosine}
\funcref{sinh}{calculate hyperbolic sine}
\funcref{tanh}{calculate hyperbolic tangent}
\end{funclist}
\subsection{Exponential and logarithmic functions}
\begin{funclist}
\funcref{intpower}{Raise float to integer power}
\funcref{ldexp}{Calculate $2^p x$}
\funcref{lnxp1}{calculate \var{log(x+1)}}
\funcref{log10}{calculate 10-base log}
\funcref{log2}{calculate 2-base log}
\funcref{logn}{calculate N-base log}
\funcref{power}{raise float to arbitrary power}
\end{funclist}
\subsection{Number converting}
\begin{funclist}
\funcref{ceil}{Round to infinity}
\funcref{floor}{Round to minus infinity}
\procref{frexp}{Return mantissa and exponent}
\end{funclist}
\subsection{Statistical functions}
\begin{funclist}
\funcref{mean}{Mean of values}
\procref{meanandstddev}{Mean and standard deviation of values}
\procref{momentskewkurtosis}{Moments, skew and kurtosis}
\funcref{popnstddev}{Population standarddeviation }
\funcref{popnvariance}{Population variance}
\funcref{randg}{Gaussian distributed randum value}
\funcref{stddev}{Standard deviation}
\funcref{sum}{Sum of values}
\funcref{sumofsquares}{Sum of squared values}
\procref{sumsandsquares}{Sum of values and squared values}
\funcref{totalvariance}{Total variance of values}
\funcref{variance}{variance of values}
\end{funclist}
\subsection{Geometrical functions}
\begin{funclist}
\funcref{hypot}{Hypotenuse of triangle}
\funcref{norm}{Euclidian norm}
\end{funclist}

\section{Functions and Procedures}

\begin{function}{arccos}
\Declaration
Function arccos(x : float) : float;
\Description
\var{Arccos} returns the inverse cosine of its argument \var{x}. The
argument \var{x} should lie between -1 and 1 (borders included). 
\Errors
If the argument \var{x} is not in the allowed range, an
\var{EInvalidArgument} exception is raised.
\SeeAlso
\seef{arcsin}, \seef{arcosh}, \seef{arsinh}, \seef{artanh}
\end{function}

\FPCexample{ex1}

\begin{function}{arcosh}
\Declaration
Function arcosh(x : float) : float;
Function arccosh(x : float) : float;
\Description
\var{Arcosh} returns the inverse hyperbolic cosine of its argument \var{x}. 
The argument \var{x} should be larger than 1. 

The \var{arccosh} variant of this function is supplied for \delphi 
compatibility.
\Errors
If the argument \var{x} is not in the allowed range, an \var{EInvalidArgument}
exception is raised.
\SeeAlso
\seef{cosh}, \seef{sinh}, \seef{arcsin}, \seef{arsinh}, \seef{artanh},
\seef{tanh}
\end{function}

\FPCexample{ex3}

\begin{function}{arcsin}
\Declaration
Function arcsin(x : float) : float;
\Description
\var{Arcsin} returns the inverse sine of its argument \var{x}. The
argument \var{x} should lie between -1 and 1. 
\Errors
If the argument \var{x} is not in the allowed range, an \var{EInvalidArgument}
exception is raised.
\SeeAlso
\seef{arccos}, \seef{arcosh}, \seef{arsinh}, \seef{artanh}
\end{function}

\FPCexample{ex2}


\begin{function}{arctan2}
\Declaration
Function arctan2(x,y : float) : float;
\Description
\var{arctan2} calculates \var{arctan(y/x)}, and returns an angle in the
correct quadrant. The returned angle will be in the range $-\pi$ to
$\pi$ radians.
The values of \var{x} and \var{y} must be between -2\^{}64 and 2\^{}64,
moreover \var{x} should be different from zero.

On Intel systems this function is implemented with the native intel
\var{fpatan} instruction.
\Errors
If \var{x} is zero, an overflow error will occur.
\SeeAlso
\seef{arccos}, \seef{arcosh}, \seef{arsinh}, \seef{artanh}
\end{function}

\FPCexample{ex6}

\begin{function}{arsinh}
\Declaration
Function arsinh(x : float) : float;
Function arcsinh(x : float) : float;
\Description
\var{arsinh} returns the inverse hyperbolic sine of its argument \var{x}. 

The \var{arscsinh} variant of this function is supplied for \delphi 
compatibility.
\Errors
None.
\SeeAlso
\seef{arcosh}, \seef{arccos}, \seef{arcsin}, \seef{artanh}
\end{function}

\FPCexample{ex4}


\begin{function}{artanh}
\Declaration
Function artanh(x : float) : float;
Function arctanh(x : float) : float;
\Description
\var{artanh} returns the inverse hyperbolic tangent of its argument \var{x},
where \var{x} should lie in the interval [-1,1], borders included.

The \var{arctanh} variant of this function is supplied for \delphi compatibility.
\Errors
In case \var{x} is not in the interval [-1,1], an \var{EInvalidArgument}
exception is raised.
\SeeAlso
\seef{arcosh}, \seef{arccos}, \seef{arcsin}, \seef{artanh}
\Errors
\SeeAlso
\end{function}

\FPCexample{ex5}


\begin{function}{ceil}
\Declaration
Function ceil(x : float) : longint;
\Description
\var{Ceil} returns the lowest integer number greater than or equal to \var{x}.
The absolute value of \var{x} should be less than \var{maxint}.
\Errors
If the asolute value of \var{x} is larger than maxint, an overflow error will
occur.
\SeeAlso
\seef{floor}
\end{function}

\FPCexample{ex7}

\begin{function}{cosh}
\Declaration
Function cosh(x : float) : float;
\Description
\var{Cosh} returns the hyperbolic cosine of it's argument {x}.
\Errors
None.
\SeeAlso
\seef{arcosh}, \seef{sinh}, \seef{arsinh}
\end{function}

\FPCexample{ex8}


\begin{function}{cotan}
\Declaration
Function cotan(x : float) : float;
\Description
\var{Cotan} returns the cotangent of it's argument \var{x}. \var{x} should
be different from zero.
\Errors
If \var{x} is zero then a overflow error will occur.
\SeeAlso
\seef{tanh}
\end{function}

\FPCexample{ex9}


\begin{function}{cycletorad}
\Declaration
Function cycletorad(cycle : float) : float;
\Description
\var{Cycletorad} transforms it's argument \var{cycle}
(an angle expressed in cycles) to radians.
(1 cycle is $2 \pi$ radians).
\Errors
None.
\SeeAlso
\seef{degtograd}, \seef{degtorad}, \seef{radtodeg},
\seef{radtograd}, \seef{radtocycle}
\end{function}

\FPCexample{ex10}


\begin{function}{degtograd}
\Declaration
Function degtograd(deg : float) : float;
\Description
\var{Degtograd} transforms it's argument \var{deg} (an angle in degrees)
to grads.

(90 degrees is 100 grad.)
\Errors
None.
\SeeAlso
\seef{cycletorad}, \seef{degtorad}, \seef{radtodeg},
\seef{radtograd}, \seef{radtocycle}
\end{function}

\FPCexample{ex11}


\begin{function}{degtorad}
\Declaration
Function degtorad(deg : float) : float;
\Description
\var{Degtorad} converts it's argument \var{deg} (an angle in degrees) to
radians.

(pi radians is 180 degrees)
\Errors
None.
\SeeAlso
\seef{cycletorad}, \seef{degtograd}, \seef{radtodeg},
\seef{radtograd}, \seef{radtocycle}
\end{function}

\FPCexample{ex12}


\begin{function}{floor}
\Declaration
Function floor(x : float) : longint;
\Description
\var{Floor} returns the largest integer smaller than or equal to \var{x}.
The absolute value of \var{x} should be less than \var{maxint}.
\Errors
If \var{x} is larger than \var{maxint}, an overflow will occur.
\SeeAlso
\seef{ceil}
\end{function}

\FPCexample{ex13}


\begin{procedure}{frexp}
\Declaration
Procedure frexp(x : float;var mantissa,exponent : float);
\Description
\var{Frexp} returns the mantissa and exponent of it's argument
\var{x} in \var{mantissa} and \var{exponent}.
\Errors
None
\SeeAlso
\end{procedure}

\FPCexample{ex14}


\begin{function}{gradtodeg}
\Declaration
Function gradtodeg(grad : float) : float;
\Description
\var{Gradtodeg} converts its argument \var{grad} (an angle in grads)
to degrees.

(100 grad is 90 degrees)
\Errors
None.
\SeeAlso
\seef{cycletorad}, \seef{degtograd}, \seef{radtodeg},
\seef{radtograd}, \seef{radtocycle}, \seef{gradtorad}
\end{function}

\FPCexample{ex15}


\begin{function}{gradtorad}
\Declaration
Function gradtorad(grad : float) : float;
\Description
\var{Gradtorad} converts its argument \var{grad} (an angle in grads)
to radians.

(200 grad is pi degrees).
\Errors
None.
\SeeAlso
\seef{cycletorad}, \seef{degtograd}, \seef{radtodeg},
\seef{radtograd}, \seef{radtocycle}, \seef{gradtodeg}
\end{function}

\FPCexample{ex16}


\begin{function}{hypot}
\Declaration
Function hypot(x,y : float) : float;
\Description
\var{Hypot} returns the hypotenuse of the triangle where the sides
adjacent to the square angle have lengths \var{x} and \var{y}.

The function uses Pythagoras' rule for this.
\Errors
None.
\SeeAlso
\end{function}

\FPCexample{ex17}


\begin{function}{intpower}
\Declaration
Function intpower(base : float;exponent : longint) : float;
\Description
\var{Intpower} returns \var{base} to the power \var{exponent},
where exponent is an integer value.
\Errors
If \var{base} is zero and the exponent is negative, then an
overflow error will occur.
\SeeAlso
\seef{power}
\end{function}

\FPCexample{ex18}


\begin{function}{ldexp}
\Declaration
Function ldexp(x : float;p : longint) : float;
\Description
\var{Ldexp} returns $2^p x$.
\Errors
None.
\SeeAlso
\seef{lnxp1}, \seef{log10},\seef{log2},\seef{logn}
\end{function}

\FPCexample{ex19}


\begin{function}{lnxp1}
\Declaration
Function lnxp1(x : float) : float;
\Description
\var{Lnxp1} returns the natural logarithm of \var{1+X}. The result
is more precise for small values of \var{x}. \var{x} should be larger
than -1.
\Errors
If $x\leq -1$ then an \var{EInvalidArgument} exception will be raised.
\SeeAlso
\seef{ldexp}, \seef{log10},\seef{log2},\seef{logn}
\end{function}

\FPCexample{ex20}

\begin{function}{log10}
\Declaration
Function log10(x : float) : float;
\Description
\var{Log10} returns the 10-base logarithm of \var{X}.
\Errors
If \var{x} is less than or equal to 0 an 'invalid fpu operation' error
will occur.
\SeeAlso
\seef{ldexp}, \seef{lnxp1},\seef{log2},\seef{logn}
\end{function}

\FPCexample{ex21}


\begin{function}{log2}
\Declaration
Function log2(x : float) : float;
\Description
\var{Log2} returns the 2-base logarithm of \var{X}.
\Errors
If \var{x} is less than or equal to 0 an 'invalid fpu operation' error
will occur.
\SeeAlso
\seef{ldexp}, \seef{lnxp1},\seef{log10},\seef{logn}
\end{function}

\FPCexample{ex22}


\begin{function}{logn}
\Declaration
Function logn(n,x : float) : float;
\Description
\var{Logn} returns the n-base logarithm of \var{X}.
\Errors
If \var{x} is less than or equal to 0 an 'invalid fpu operation' error
will occur.
\SeeAlso
\seef{ldexp}, \seef{lnxp1},\seef{log10},\seef{log2}
\end{function}

\FPCexample{ex23}

\begin{function}{max}
\Declaration
Function max(Int1,Int2:Cardinal):Cardinal;
Function max(Int1,Int2:Integer):Integer;
\Description
\var{Max} returns the maximum of \var{Int1} and \var{Int2}.
\Errors
None.
\SeeAlso
\seef{min}, \seef{maxIntValue}, \seef{maxvalue}
\end{function}

\FPCexample{ex24}

\begin{function}{maxIntValue}
\Declaration
function MaxIntValue(const Data: array of Integer): Integer;
\Description
\var{MaxIntValue} returns the largest integer out of the \var{Data}
array.

This function is provided for \delphi compatibility, use the \seef{maxvalue}
function instead.
\Errors
None.
\SeeAlso
\seef{maxvalue}, \seef{minvalue}, \seef{minIntValue}
\end{function}

\FPCexample{ex25}


\begin{function}{maxvalue}
\Declaration
Function maxvalue(const data : array of float) : float;
Function maxvalue(const data : array of Integer) : Integer;
Function maxvalue(const data : PFloat; Const N : Integer) : float;
Function maxvalue(const data : PInteger; Const N : Integer) : Integer;
\Description
\var{Maxvalue} returns the largest value in the \var{data} 
array with integer or float values. The return value has 
the same type as the elements of the array.

The third and fourth forms accept a pointer to an array of \var{N} 
integer or float values.
\Errors
None.
\SeeAlso
\seef{maxIntValue}, \seef{minvalue}, \seef{minIntValue}
\end{function}

\FPCexample{ex26}

\begin{function}{mean}
\Declaration
Function mean(const data : array of float) : float;
Function mean(const data : PFloat; Const N : longint) : float;
\Description
\var{Mean} returns the average value of \var{data}.

The second form accepts a pointer to an array of \var{N} values.
\Errors
None.
\SeeAlso
\seep{meanandstddev}, \seep{momentskewkurtosis}, \seef{sum}
\end{function}

\FPCexample{ex27}

\begin{procedure}{meanandstddev}
\Declaration
Procedure meanandstddev(const data : array of float; 
                        var mean,stddev : float);
procedure meanandstddev(const data : PFloat;
  Const N : Longint;var mean,stddev : float);
\Description
\var{meanandstddev} calculates the mean and standard deviation of \var{data}
and returns the result in \var{mean} and \var{stddev}, respectively.
Stddev is zero if there is only one value.

The second form accepts a pointer to an array of \var{N} values.
\Errors
None.
\SeeAlso
\seef{mean},\seef{sum}, \seef{sumofsquares}, \seep{momentskewkurtosis}
\end{procedure}

\FPCexample{ex28}


\begin{function}{min}
\Declaration
Function min(Int1,Int2:Cardinal):Cardinal;
Function min(Int1,Int2:Integer):Integer;
\Description
\var{min} returns the smallest value of \var{Int1} and \var{Int2};
\Errors
None.
\SeeAlso
\seef{max}
\end{function}

\FPCexample{ex29}

\begin{function}{minIntValue}
\Declaration
Function minIntValue(const Data: array of Integer): Integer;
\Description
\var{MinIntvalue} returns the smallest value in the \var{Data} array.

This function is provided for \delphi compatibility, use \var{minvalue}
instead.
\Errors
None
\SeeAlso
\seef{minvalue}, \seef{maxIntValue}, \seef{maxvalue}
\end{function}

\FPCexample{ex30}


\begin{function}{minvalue}
\Declaration
Function minvalue(const data : array of float) : float;
Function minvalue(const data : array of Integer) : Integer;
Function minvalue(const data : PFloat; Const N : Integer) : float;
Function minvalue(const data : PInteger; Const N : Integer) : Integer;
\Description
\var{Minvalue} returns the smallest value in the \var{data} 
array with integer or float values. The return value has 
the same type as the elements of the array.

The third and fourth forms accept a pointer to an array of \var{N} 
integer or float values.
\Errors
None.
\SeeAlso
\seef{maxIntValue}, \seef{maxvalue}, \seef{minIntValue}
\end{function}

\FPCexample{ex31}


\begin{procedure}{momentskewkurtosis}
\Declaration
procedure momentskewkurtosis(const data : array of float;
  var m1,m2,m3,m4,skew,kurtosis : float);
procedure momentskewkurtosis(const data : PFloat; Const N : Integer;
  var m1,m2,m3,m4,skew,kurtosis : float);
\Description
\var{momentskewkurtosis} calculates the 4 first moments of the distribution
of valuesin \var{data} and returns them in \var{m1},\var{m2},\var{m3} and
\var{m4}, as well as the \var{skew} and \var{kurtosis}.
\Errors
None.
\SeeAlso
\seef{mean}, \seep{meanandstddev}
\end{procedure}

\FPCexample{ex32}

\begin{function}{norm}
\Declaration
Function norm(const data : array of float) : float;
Function norm(const data : PFloat; Const N : Integer) : float;
\Description
\var{Norm} calculates the Euclidian norm of the array of data.
This equals \var{sqrt(sumofsquares(data))}.

The second form accepts a pointer to an array of \var{N} values.
\Errors
None.
\SeeAlso
\seef{sumofsquares}
\end{function}

\FPCexample{ex33}


\begin{function}{popnstddev}
\Declaration
Function popnstddev(const data : array of float) : float;
Function popnstddev(const data : PFloat; Const N : Integer) : float;
\Description
\var{Popnstddev} returns the square root of the population variance of
the values in the  \var{Data} array. It returns zero if there is only one value.

The second form of this function accepts a pointer to an array of \var{N}
values.
\Errors
None.
\SeeAlso
\seef{popnvariance}, \seef{mean}, \seep{meanandstddev}, \seef{stddev},
\seep{momentskewkurtosis}
\end{function}

\FPCexample{ex35}


\begin{function}{popnvariance}
\Declaration
Function popnvariance(const data : array of float) : float;
Function popnvariance(const data : PFloat; Const N : Integer) : float;
\Description
\var{Popnvariance} returns the square root of the population variance of
the values in the  \var{Data} array. It returns zero if there is only one value.

The second form of this function accepts a pointer to an array of \var{N}
values.
\Errors
None.
\SeeAlso
\seef{popnstddev}, \seef{mean}, \seep{meanandstddev}, \seef{stddev},
\seep{momentskewkurtosis}
\end{function}

\FPCexample{ex36}


\begin{function}{power}
\Declaration
Function power(base,exponent : float) : float;
\Description
\var{power} raises \var{base} to the power \var{power}. This is equivalent
to \var{exp(power*ln(base))}. Therefore \var{base} should be non-negative.
\Errors
None.
\SeeAlso
\seef{intpower}
\end{function}

\FPCexample{ex34}


\begin{function}{radtocycle}
\Declaration
Function radtocycle(rad : float) : float;
\Description
\var{Radtocycle} converts its argument \var{rad} (an angle expressed in
radians) to an angle in cycles.

(1 cycle equals 2 pi radians)
\Errors
None.
\SeeAlso
\seef{degtograd}, \seef{degtorad}, \seef{radtodeg},
\seef{radtograd}, \seef{cycletorad}
\end{function}

\FPCexample{ex37}


\begin{function}{radtodeg}
\Declaration
Function radtodeg(rad : float) : float;
\Description
\var{Radtodeg} converts its argument \var{rad} (an angle expressed in
radians) to an angle in degrees.

(180 degrees equals pi radians)
\Errors
None.
\SeeAlso
\seef{degtograd}, \seef{degtorad}, \seef{radtocycle},
\seef{radtograd}, \seef{cycletorad}
\end{function}

\FPCexample{ex38}


\begin{function}{radtograd}
\Declaration
Function radtograd(rad : float) : float;
\Description
\var{Radtodeg} converts its argument \var{rad} (an angle expressed in
radians) to an angle in grads.

(200 grads equals pi radians)
\Errors
None.
\SeeAlso
\seef{degtograd}, \seef{degtorad}, \seef{radtocycle},
\seef{radtodeg}, \seef{cycletorad}
\end{function}

\FPCexample{ex39}


\begin{function}{randg}
\Declaration
Function randg(mean,stddev : float) : float;
\Description
\var{randg} returns a random number which - when produced in large
quantities - has a Gaussian distribution with mean \var{mean} and 
standarddeviation \var{stddev}. 
\Errors
None.
\SeeAlso
\seef{mean}, \seef{stddev}, \seep{meanandstddev}
\end{function}

\FPCexample{ex40}


\begin{procedure}{sincos}
\Declaration
Procedure sincos(theta : float;var sinus,cosinus : float);
\Description
\var{Sincos} calculates the sine and cosine of the angle \var{theta},
and returns the result in \var{sinus} and \var{cosinus}.

On Intel hardware, This calculation will be faster than making 2 calls
to clculatet he sine and cosine separately.
\Errors
None.
\SeeAlso
\seef{arcsin}, \seef{arccos}.
\end{procedure}

\FPCexample{ex41}


\begin{function}{sinh}
\Declaration
Function sinh(x : float) : float;
\Description
\var{Sinh} returns the hyperbolic sine of its argument \var{x}.
\Errors
\SeeAlso
\seef{cosh}, \seef{arsinh}, \seef{tanh}, \seef{artanh}
\end{function}

\FPCexample{ex42}


\begin{function}{stddev}
\Declaration
Function stddev(const data : array of float) : float;
Function stddev(const data : PFloat; Const N : Integer) : float;
\Description
\var{Stddev} returns the standard deviation of the values in \var{Data}.
It returns zero if there is only one value.

The second form of the function accepts a pointer to an array of \var{N}
values.
\Errors
None.
\SeeAlso
\seef{mean}, \seep{meanandstddev}, \seef{variance}, \seef{totalvariance}
\end{function}

\FPCexample{ex43}


\begin{function}{sum}
\Declaration
Function sum(const data : array of float) : float;
Function sum(const data : PFloat; Const N : Integer) : float;
\Description
\var{Sum} returns the sum of the values in the \var{data} array.

The second form of the function accepts a pointer to an array of \var{N}
values.
\Errors
None.
\SeeAlso
\seef{sumofsquares}, \seep{sumsandsquares}, \seef{totalvariance}
, \seef{variance}
\end{function}

\FPCexample{ex44}


\begin{function}{sumofsquares}
\Declaration
Function sumofsquares(const data : array of float) : float;
Function sumofsquares(const data : PFloat; Const N : Integer) : float;
\Description
\var{Sumofsquares} returns the sum of the squares of the values in the \var{data} 
array.

The second form of the function accepts a pointer to an array of \var{N}
values.
\Errors
None.
\SeeAlso
\seef{sum}, \seep{sumsandsquares}, \seef{totalvariance}
, \seef{variance}
\end{function}

\FPCexample{ex45}


\begin{procedure}{sumsandsquares}
\Declaration
Procedure sumsandsquares(const data : array of float;
  var sum,sumofsquares : float);
Procedure sumsandsquares(const data : PFloat; Const N : Integer;
  var sum,sumofsquares : float);
\Description
\var{sumsandsquares} calculates the sum of the values and the sum of 
the squares of the values in the \var{data} array and returns the
results in \var{sum} and \var{sumofsquares}.

The second form of the function accepts a pointer to an array of \var{N}
values.
\Errors
None.
\SeeAlso
\seef{sum}, \seef{sumofsquares}, \seef{totalvariance}
, \seef{variance}
\end{procedure}

\FPCexample{ex46}


\begin{function}{tan}
\Declaration
Function tan(x : float) : float;
\Description
\var{Tan} returns the tangent of \var{x}.
\Errors
If \var{x} (normalized) is pi/2 or 3pi/2 then an overflow will occur.
\SeeAlso
\seef{tanh}, \seef{arcsin}, \seep{sincos}, \seef{arccos}
\end{function}

\FPCexample{ex47}


\begin{function}{tanh}
\Declaration
Function tanh(x : float) : float;
\Description
\var{Tanh} returns the hyperbolic tangent of \var{x}.
\Errors
None.
\SeeAlso
\seef{arcsin}, \seep{sincos}, \seef{arccos}
\end{function}

\FPCexample{ex48}


\begin{function}{totalvariance}
\Declaration
Function totalvariance(const data : array of float) : float;
Function totalvariance(const data : PFloat; Const N : Integer) : float;
\Description
\var{TotalVariance} returns the total variance of the values in the 
\var{data} array. It returns zero if there is only one value.

The second form of the function accepts a pointer to an array of \var{N}
values.
\Errors
None.
\SeeAlso
\seef{variance}, \seef{stddev}, \seef{mean}
\end{function}

\FPCexample{ex49}


\begin{function}{variance}
\Declaration
Function variance(const data : array of float) : float;
Function variance(const data : PFloat; Const N : Integer) : float;
\Description
\var{Variance} returns the variance of the values in the 
\var{data} array. It returns zero if there is only one value.

The second form of the function accepts a pointer to an array of \var{N}
values.
\Errors
None.
\SeeAlso
\seef{totalvariance}, \seef{stddev}, \seef{mean}
\end{function}

\FPCexample{ex50}

% The MMX unit
%
%   $Id$
%   This file is part of the FPC documentation.
%   Copyright (C) 1997, by Michael Van Canneyt
%
%   The FPC documentation is free text; you can redistribute it and/or
%   modify it under the terms of the GNU Library General Public License as
%   published by the Free Software Foundation; either version 2 of the
%   License, or (at your option) any later version.
%
%   The FPC Documentation is distributed in the hope that it will be useful,
%   but WITHOUT ANY WARRANTY; without even the implied warranty of
%   MERCHANTABILITY or FITNESS FOR A PARTICULAR PURPOSE.  See the GNU
%   Library General Public License for more details.
%
%   You should have received a copy of the GNU Library General Public
%   License along with the FPC documentation; see the file COPYING.LIB.  If not,
%   write to the Free Software Foundation, Inc., 59 Temple Place - Suite 330,
%   Boston, MA 02111-1307, USA. 
%
\chapter{The MMX unit}
This chapter describes the \file{MMX} unit. This unit allows you to use the
\var{MMX} capabilities of the \fpc compiler. It was written by Florian
Kl\"ampfl for the \var{I386} processor.
\section{Variables, Types and constants}
The following types are defined in the \var{MMX} unit:
\begin{verbatim}
tmmxshortint = array[0..7] of shortint;
tmmxbyte = array[0..7] of byte;
tmmxword = array[0..3] of word;
tmmxinteger = array[0..3] of integer;
tmmxfixed = array[0..3] of fixed16;
tmmxlongint = array[0..1] of longint;
tmmxcardinal = array[0..1] of cardinal;
{ for the AMD 3D }
tmmxsingle = array[0..1] of single;
\end{verbatim}
And the following pointers to the above types:
\begin{verbatim}
pmmxshortint = ^tmmxshortint;
pmmxbyte = ^tmmxbyte;
pmmxword = ^tmmxword;
pmmxinteger = ^tmmxinteger;
pmmxfixed = ^tmmxfixed;
pmmxlongint = ^tmmxlongint;
pmmxcardinal = ^tmmxcardinal;
{ for the AMD 3D }
pmmxsingle = ^tmmxsingle;
\end{verbatim}
The following initialized constants allow you to determine if the computer
has \var{MMX} extensions. They are set correctly in the unit's
initialization code.
\begin{verbatim}
is_mmx_cpu : boolean = false;
is_amd_3d_cpu : boolean = false;
\end{verbatim}
\section{Functions and Procedures}
\begin{procedure}{Emms}
\Declaration
Procedure Emms ;

\Description
\var{Emms} sets all floating point registers to empty. This procedure must
be called after you have used any \var{MMX} instructions, if you want to use
floating point arithmetic. If you just want to move floating point data
around, it isn't necessary to call this function, the compiler doesn't use
the FPU registers when moving data. Only when doing calculations, you should
use this function.

\Errors
None.
\SeeAlso
 \progref 
\end{procedure}
\begin{FPCList}
\item[Example:]
\begin{verbatim}
Program MMXDemo;
uses mmx;
var
   d1 : double;
   a : array[0..10000] of double;
   i : longint;
begin
   d1:=1.0;
{$mmx+}
   { floating point data is used, but we do _no_ arithmetic }
   for i:=0 to 10000 do
     a[i]:=d2;  { this is done with 64 bit moves }
{$mmx-}
   emms;   { clear fpu }
   { now we can do floating point arithmetic again }
end. 
\end{verbatim}
\end{FPCList}

% The mouse unit
\chapter{The Mouse unit}

The mouse unit provides basic Mouse handling under Dos (Go32v1 and Go32v2)

Some general remarks about the mouse unit:

\begin{itemize}
\item The mouse driver does not know when the text screen scrolls. This results
in unerased mouse cursors on the screen when the screen scrolls while the
mouse cursor is visible. The solution is to hide the mouse cursor (using
HideMouse) when you write something to the screen and to show it again
afterwards (using ShowMouse).

\item All Functions/Procedures that return and/or accept coordinates of the mouse
cursor, always do so in pixels and zero based (so the upper left corner of
the screen is (0,0)). To get the (column, row) in standard text mode, divide
both x and y by 8 (and add 1 if you want to have it 1 based).

\item The real resolution of graphic modes and the one the mouse driver uses can
differ. For example, mode 13h (320*200 pixels) is handled by the mouse driver
as 640*200, so you will have to multiply the X coordinates you give to the
driver and divide the ones you get from it by 2 in that mode.

\item By default the mouse unit is compiled with the conditional define
MouseCheck. This causes every procedure/function of the unit to check the
MouseFound variable prior to doing anything. Of course this is not necessary,
so if you are sure you are not calling any mouse unit procedures when no
mouse is found, you can recompile the mouse unit without this conditional
define.

\item
You will notice that several procedures/functions have longint sized
parameters while only the lower 16 bits are used. This is because FPC is
a 32 bit compiler and consequently 32 bit parameters result in faster code.
\end{itemize}

\section{Constants, types and variables}

The following constants are defined (to be used in e.g. the
\seef{GetLastButtonPress} call).
\begin{verbatim}
 LButton = 1; {left button}
 RButton = 2; {right button}
 MButton = 4; {middle button}
\end{verbatim}

The following variable exist: 
\begin{verbatim}
  MouseFound: Boolean;
\end{verbatim}
it is set to \var{True} or \var{False} in the unit's initialization code.

\section{Functions and procedures}

\function{GetLastButtonPress}{(Button: Longint; Var x,y:Longint)}{Longint}{ 
\var{GetLastButtonPress}
Stores the position where \var{Button} was last pressed in \var{x} and
\var{y} and returns
the number of times this button has been pressed since the last call to this
function with \var{Button} as parameter. For \var{Button} you can use the 
\var{LButton}, \var{RButton} and \var{MButton} constants for resp. the left, 
right and middle button.

For two-button mice, checking the status of the middle button seems to give
and clear the stats of the right button.
}{None.}{\seef{GetLastButtonRelease}}

\latex{\inputlisting{mouseex/mouse5.pp}}
\html{\input{mouseex/mouse5.tex}}

\function{GetLastButtonRelease}{(Button: Longint; Var x,y:Longint)}{Longint}{
\var{GetLastButtonRelease}
stores the position where \var{Button} was last released in \var{x} and 
\var{y} and returns
the number of times this button has been released since the last call to this
function with \var{Button} as parameter. For button you can use the
\var{LButton}, \var{RButton} and \var{MButton} constants for resp. 
the left, right and middle button.

For two-button mice, checking the stats of the middle button seems to give
and clear the stats of the right button.
}{None.}{\seef{GetLastButtonPress}}

For an example, see \seef{GetLastButtonPress}.

\procedure{GetMouseState}{(Var x, y, buttons: Longint)}{
\var{GetMouseState} Returns information on the current mouse position 
and which buttons are currently pressed.

\var{x} and \var{y} return the mouse cursor coordinates in pixels.

\var{Buttons} is a bitmask. Check the example program to see how you can get the
necessary information from it.
}{None.}{\seef{LPressed}, \seef{MPressed}, \seef{RPressed},
\seep{SetMousePos}}


\latex{\inputlisting{mouseex/mouse3.pp}}
\html{\input{mouseex/mouse3.tex}}

\Procedure{HideMouse}{
\var{HideMouse} makes the mouse cursor invisible.

Multiple calls to HideMouse will require just as many calls to ShowMouse to
make the mouse cursor again visible.
}{None.}{\seep{ShowMouse}, \seep{SetMouseHideWindow}}

For an example, see \seep{ShowMouse}.

\Procedure{InitMouse}{
\var{InitMouse}
Initializes the mouse driver sets the variable \var{MouseFound} depending on
whether or not a mouse is found.

This is Automatically called at the start of your program. 
You should never have to call it, unless you want to reset everything to 
its default values.
}{None.}{\var{MouseFound} variable.}

\latex{\inputlisting{mouseex/mouse1.pp}}
\html{\input{mouseex/mouse1.tex}}

\Function{LPressed}{Boolean}{

\var{LPressed} returns \var{True} if the left mouse button is pressed.

This is simply a wrapper for the GetMouseState procedure.
}{None.}{\seep{GetMouseState}, \seef{MPressed}, \seef{RPressed}}


For an example, see \seep{GetMouseState}.

\Function{MPressed}{Boolean}{
\var{MPressed} returns \var{True} if the middle mouse button is pressed.

This is simply a wrapper for the GetMouseState procedure.
}{None.}{\seep{GetMouseState}, \seef{LPressed}, \seef{RPressed}}

For an example, see \seep{GetMouseState}.

\Function{RPressed}{Boolean}{
\var{RPressed} returns \var{True} if the right mouse button is pressed.

This is simply a wrapper for the GetMouseState procedure.
}{None.}{\seep{GetMouseState}, \seef{LPressed}, \seef{MPressed}}

For an example, see \seep{GetMouseState}.

\procedure{SetMouseAscii}{(Ascii: Byte)}{
\var{SetMouseAscii}
sets the \var{Ascii} value of the character that depicts the mouse cursor in 
text mode.

The difference between this one and \seep{SetMouseShape}, is that the foreground
and background colors stay the same and that the Ascii code you enter is the
character that you will get on screen; there's no XOR'ing.
}{None}{\seep{SetMouseShape}}

\latex{\inputlisting{mouseex/mouse8.pp}}
\html{\input{mouseex/mouse8.tex}}



\procedure{SetMouseHideWindow}{(xmin,ymin,xmax,ymax: Longint)}{
\var{SetMouseHideWindow}
defines a rectangle on screen with top-left corner at (\var{xmin,ymin}) and
botto-right corner at (\var{xmax,ymax}),which causes the mouse cursor to be 
turned off when it is moved into it.

When the mouse is moved into the specified region, it is turned off until you
call \var{ShowMouse} again. However, once you've called \seep{ShowMouse}, you'll have to
call \var{SetMouseHideWindow} again to redefine the hide window... 
This may be annoying, but it's the way it's implemented in the mouse driver.

While \var{xmin, ymin, xmax} and \var{ymax} are Longint parameters, 
only the lower 16 bits are used.
}{None.}{\seep{ShowMouse}, \seep{HideMouse}}


\latex{nputlisting{mouseex/mouse9.pp}}
\html{\input{mouseex/mouse9.tex}}

\procedure{SetMousePos}{(x,y:Longint)}{
\var{SetMosusePos} sets the position of the mouse cursor on the screen.
\var{x} is the horizontal position in pixels, \var{y} the vertical position
in pixels. The upper-left hand corner of the screen is the origin.

While \var{x} and \var{y} are longints, only the lower 16 bits are used.
}{None.}{\seep{GetMouseState}}


\latex{\inputlisting{mouseex/mouse4.pp}}
\html{\input{mouseex/mouse4.tex}}

\procedure{SetMouseShape}{(ForeColor,BackColor,Ascii: Byte)}{
\var{SetMouseShape}
defines how the mouse cursor looks in textmode

The character and its attributes that are on the mouse cursor's position on
screen are XOR'ed with resp. \var{ForeColor}, \var{BackColor} and
\var{Ascii}. Set them all to 0 for a "transparent" cursor.
}{None.}{\seep{SetMouseAscii}}


\latex{\inputlisting{mouseex/mouse7.pp}}
\html{\input{mouseex/mouse7.tex}}

\procedure{SetMouseSpeed}{(Horizontal, Vertical: Longint)}{
\var{SetMouseSpeed} sets the mouse speed in mickeys per 8 pixels.

A mickey is the smallest measurement unit handled by a mouse. With this
procedure you can set how many mickeys the mouse should move to move the
cursor 8 pixels horizontally of vertically. The default values are 8 for
horizontal and 16 for vertical movement.

While this procedure accepts longint parameters, only the low 16 bits are
actually used.
}{None.}{}

\latex{\inputlisting{mouseex/mouse10.pp}}
\html{\input{mouseex/mouse10.tex}}

\procedure{SetMouseWindow}{(xmin,ymin,xmax,ymax: Longint)}{

\var{SetMousWindow}
defines a rectangle on screen with top-left corner at (\var{xmin,ymin}) and
botto-right corner at (\var{xmax,ymax}), out of which the mouse 
cursor can't move.

This procedure is simply a wrapper for the \seep{SetMouseXRange} and 
\seep{SetMouseYRange} procedures.

While \var{xmin, ymin, xmax} and \var{ymax} are Longint parameters, 
only the lower 16 bits are used.
}{None.}{\seep{SetMouseXRange}, \seep{SetMouseYRange}}

For an example, see \seep{SetMouseXRange}.

\procedure{SetMouseXRange}{(Min, Max: Longint)}{ 
\var{SetMouseXRange}
sets the minimum (\var{Min}) and maximum (\var{Max}) horizontal coordinates in between which the
mouse cursor can move.

While \var{Min} and \var{Max} are Longint parameters, only the lower 16 bits 
are used.
}{None.}{\seep{SetMouseYRange}, \seep{SetMouseWindow}}

\latex{\inputlisting{mouseex/mouse6.pp}}
\html{\input{mouseex/mouse6.tex}}

\procedure{SetMouseYRange}{(Min, Max: Longint)}{
\var{SetMouseYRange}
sets the minimum (\var{Min}) and maximum (\var{Max}) vertical coordinates in between which the
mouse cursor can move.

While \var{Min} and \var{Max} are Longint parameters, only the lower 16 bits 
are used.
}{None.}{\seep{SetMouseXRange}, \seep{SetMouseWindow}}

For an example, see \seep{SetMouseXRange}.

\Procedure{ShowMouse}{
\var{ShowMouse} makes the mouse cursor visible.

At the start of your progam, the mouse is invisible.
}{None.}{\seep{HideMouse},\seep{SetMouseHideWindow}}

\latex{\inputlisting{mouseex/mouse2.pp}}
\html{\input{mouseex/mouse2.tex}}


% The msmouse unit
%
%   $Id$
%   This file is part of the FPC documentation.
%   Copyright (C) 1997, by Michael Van Canneyt
%
%   The FPC documentation is free text; you can redistribute it and/or
%   modify it under the terms of the GNU Library General Public License as
%   published by the Free Software Foundation; either version 2 of the
%   License, or (at your option) any later version.
%
%   The FPC Documentation is distributed in the hope that it will be useful,
%   but WITHOUT ANY WARRANTY; without even the implied warranty of
%   MERCHANTABILITY or FITNESS FOR A PARTICULAR PURPOSE.  See the GNU
%   Library General Public License for more details.
%
%   You should have received a copy of the GNU Library General Public
%   License along with the FPC documentation; see the file COPYING.LIB.  If not,
%   write to the Free Software Foundation, Inc., 59 Temple Place - Suite 330,
%   Boston, MA 02111-1307, USA.
%
%%%%%%%%%%%%%%%%%%%%%%%%%%%%%%%%%%%%%%%%%%%%%%%%%%%%%%%%%%%%%%%%%%%%%%%
%
%%%%%%%%%%%%%%%%%%%%%%%%%%%%%%%%%%%%%%%%%%%%%%%%%%%%%%%%%%%%%%%%%%%%%%%
% The MSMouse unit
%%%%%%%%%%%%%%%%%%%%%%%%%%%%%%%%%%%%%%%%%%%%%%%%%%%%%%%%%%%%%%%%%%%%%%%
\chapter{The MsMouse unit}
\FPCexampledir{mmouseex}

The msmouse unit provides basic mouse handling under \dos (Go32v1 and Go32v2)
Some general remarks about the msmouse unit:
\begin{itemize}
\item For maximum portability, it is advisable to use the \file{mouse} unit;
that unit is portable across platforms, and offers a similar interface.
Under no circumstances should the two units be used together.
\item The mouse driver does not know when the text screen scrolls. This results
in unerased mouse cursors on the screen when the screen scrolls while the
mouse cursor is visible. The solution is to hide the mouse cursor (using
HideMouse) when you write something to the screen and to show it again
afterwards (using ShowMouse).
\item All Functions/Procedures that return and/or accept coordinates of the mouse
cursor, always do so in pixels and zero based (so the upper left corner of
the screen is (0,0)). To get the (column, row) in standard text mode, divide
both x and y by 8 (and add 1 if you want to have it 1 based).
\item The real resolution of graphic modes and the one the mouse driver uses can
differ. For example, mode 13h (320*200 pixels) is handled by the mouse driver
as 640*200, so you will have to multiply the X coordinates you give to the
driver and divide the ones you get from it by 2 in that mode.
\item By default the msmouse unit is compiled with the conditional define
MouseCheck. This causes every procedure/function of the unit to check the
MouseFound variable prior to doing anything. Of course this is not necessary,
so if you are sure you are not calling any mouse unit procedures when no
mouse is found, you can recompile the mouse unit without this conditional
define.
\item
You will notice that several procedures/functions have longint sized
parameters while only the lower 16 bits are used. This is because FPC is
a 32 bit compiler and consequently 32 bit parameters result in faster code.
\end{itemize}
\section{Constants, types and variables}
The following constants are defined (to be used in e.g. the
\seef{GetLastButtonPress} call).
\begin{verbatim}
 LButton = 1; {left button}
 RButton = 2; {right button}
 MButton = 4; {middle button}
\end{verbatim}
The following variable exist: 
\begin{verbatim}
  MouseFound: Boolean;
\end{verbatim}
it is set to \var{True} or \var{False} in the unit's initialization code.
\section{Functions and procedures}
\begin{function}{GetLastButtonPress}
\Declaration
Function GetLastButtonPress (Button: Longint; Var x,y:Longint) : Longint;

\Description
 
\var{GetLastButtonPress}
Stores the position where \var{Button} was last pressed in \var{x} and
\var{y} and returns
the number of times this button has been pressed since the last call to this
function with \var{Button} as parameter. For \var{Button} you can use the 
\var{LButton}, \var{RButton} and \var{MButton} constants for resp. the left, 
right and middle button.
With certain mouse drivers, checking the middle button when using a
two-button mouse to gives and clears the stats of the right button.

\Errors
None.
\SeeAlso
\seef{GetLastButtonRelease}
\end{function}

\FPCexample{mouse5}

\begin{function}{GetLastButtonRelease}
\Declaration
Function GetLastButtonRelease (Button: Longint; Var x,y:Longint) : Longint;

\Description

\var{GetLastButtonRelease}
stores the position where \var{Button} was last released in \var{x} and 
\var{y} and returns
the number of times this button has been released since the last call to this
function with \var{Button} as parameter. For button you can use the
\var{LButton}, \var{RButton} and \var{MButton} constants for resp. 
the left, right and middle button.
With certain mouse drivers, checking the middle button when using a
two-button mouse to gives and clears the stats of the right button.

\Errors
None.
\SeeAlso
\seef{GetLastButtonPress}
\end{function}

For an example, see \seef{GetLastButtonPress}.

\begin{procedure}{GetMouseState}
\Declaration
Procedure GetMouseState (Var x, y, buttons: Longint);

\Description

\var{GetMouseState} Returns information on the current mouse position 
and which buttons are currently pressed.
\var{x} and \var{y} return the mouse cursor coordinates in pixels.
\var{Buttons} is a bitmask. Check the example program to see how you can get the
necessary information from it.

\Errors
None.
\SeeAlso
\seef{LPressed}, \seef{MPressed}, \seef{RPressed},
\seep{SetMousePos}
\end{procedure}

\FPCexample{mouse3}

\begin{procedure}{HideMouse}
\Declaration
Procedure HideMouse ;

\Description

\var{HideMouse} makes the mouse cursor invisible.
Multiple calls to HideMouse will require just as many calls to ShowMouse to
make the mouse cursor visible again.

\Errors
None.
\SeeAlso
\seep{ShowMouse}, \seep{SetMouseHideWindow}
\end{procedure}
For an example, see \seep{ShowMouse}.
\begin{procedure}{InitMouse}
\Declaration
Procedure InitMouse ;

\Description

\var{InitMouse}
Initializes the mouse driver sets the variable \var{MouseFound} depending on
whether or not a mouse is found.
This is Automatically called at the start of your program. 
You should never have to call it, unless you want to reset everything to 
its default values.

\Errors
None.
\SeeAlso
\var{MouseFound} variable.
\end{procedure}

\FPCexample{mouse1}

\begin{function}{LPressed}
\Declaration
Function LPressed  : Boolean;

\Description

\var{LPressed} returns \var{True} if the left mouse button is pressed.
This is simply a wrapper for the GetMouseState procedure.

\Errors
None.
\SeeAlso
\seep{GetMouseState}, \seef{MPressed}, \seef{RPressed}
\end{function}

For an example, see \seep{GetMouseState}.

\begin{function}{MPressed}
\Declaration
Function MPressed  : Boolean;

\Description

\var{MPressed} returns \var{True} if the middle mouse button is pressed.
This is simply a wrapper for the GetMouseState procedure.

\Errors
None.
\SeeAlso
\seep{GetMouseState}, \seef{LPressed}, \seef{RPressed}
\end{function}

For an example, see \seep{GetMouseState}.

\begin{function}{RPressed}
\Declaration
Function RPressed  : Boolean;

\Description

\var{RPressed} returns \var{True} if the right mouse button is pressed.
This is simply a wrapper for the GetMouseState procedure.

\Errors
None.
\SeeAlso
\seep{GetMouseState}, \seef{LPressed}, \seef{MPressed}
\end{function}

For an example, see \seep{GetMouseState}.

\begin{procedure}{SetMouseAscii}
\Declaration
Procedure SetMouseAscii (Ascii: Byte);

\Description

\var{SetMouseAscii}
sets the \var{Ascii} value of the character that depicts the mouse cursor in 
text mode.
The difference between this one and \seep{SetMouseShape}, is that the foreground
and background colors stay the same and that the Ascii code you enter is the
character that you will get on screen; there's no XOR'ing.

\Errors
None
\SeeAlso
\seep{SetMouseShape}
\end{procedure}

\FPCexample{mouse8}

\begin{procedure}{SetMouseHideWindow}
\Declaration
Procedure SetMouseHideWindow (xmin,ymin,xmax,ymax: Longint);

\Description

\var{SetMouseHideWindow}
defines a rectangle on screen with top-left corner at (\var{xmin,ymin}) and
botto-right corner at (\var{xmax,ymax}),which causes the mouse cursor to be 
turned off when it is moved into it.
When the mouse is moved into the specified region, it is turned off until you
call \var{ShowMouse} again. However, once you've called \seep{ShowMouse}, you'll have to
call \var{SetMouseHideWindow} again to redefine the hide window... 
This may be annoying, but it's the way it's implemented in the mouse driver.
While \var{xmin, ymin, xmax} and \var{ymax} are Longint parameters, 
only the lower 16 bits are used.

Warning: it seems Win98 SE doesn't (properly) support this function,
maybe this already the case with earlier versions too!

\Errors
None.
\SeeAlso
\seep{ShowMouse}, \seep{HideMouse}
\end{procedure}

\FPCexample{mouse9}

\begin{procedure}{SetMousePos}
\Declaration
Procedure SetMousePos (x,y:Longint);

\Description

\var{SetMosusePos} sets the position of the mouse cursor on the screen.
\var{x} is the horizontal position in pixels, \var{y} the vertical position
in pixels. The upper-left hand corner of the screen is the origin.
While \var{x} and \var{y} are longints, only the lower 16 bits are used.

\Errors
None.
\SeeAlso
\seep{GetMouseState}
\end{procedure}

\FPCexample{mouse4}

\begin{procedure}{SetMouseShape}
\Declaration
Procedure SetMouseShape (ForeColor,BackColor,Ascii: Byte);

\Description

\var{SetMouseShape}
defines how the mouse cursor looks in textmode
The character and its attributes that are on the mouse cursor's position on
screen are XOR'ed with resp. \var{ForeColor}, \var{BackColor} and
\var{Ascii}. Set them all to 0 for a "transparent" cursor.

\Errors
None.
\SeeAlso
\seep{SetMouseAscii}
\end{procedure}

\FPCexample{mouse7}

\begin{procedure}{SetMouseSpeed}
\Declaration
Procedure SetMouseSpeed (Horizontal, Vertical: Longint);

\Description

\var{SetMouseSpeed} sets the mouse speed in mickeys per 8 pixels.
A mickey is the smallest measurement unit handled by a mouse. With this
procedure you can set how many mickeys the mouse should move to move the
cursor 8 pixels horizontally of vertically. The default values are 8 for
horizontal and 16 for vertical movement.
While this procedure accepts longint parameters, only the low 16 bits are
actually used.

\Errors
None.
\SeeAlso

\end{procedure}

\FPCexample{mouse10}

\begin{procedure}{SetMouseWindow}
\Declaration
Procedure SetMouseWindow (xmin,ymin,xmax,ymax: Longint);

\Description

\var{SetMousWindow}
defines a rectangle on screen with top-left corner at (\var{xmin,ymin}) and
bottom-right corner at (\var{xmax,ymax}), out of which the mouse 
cursor can't move.
This procedure is simply a wrapper for the \seep{SetMouseXRange} and 
\seep{SetMouseYRange} procedures.
While \var{xmin, ymin, xmax} and \var{ymax} are Longint parameters, 
only the lower 16 bits are used.

\Errors
None.
\SeeAlso
\seep{SetMouseXRange}, \seep{SetMouseYRange}
\end{procedure}

For an example, see \seep{SetMouseXRange}.

\begin{procedure}{SetMouseXRange}
\Declaration
Procedure SetMouseXRange (Min, Max: Longint);

\Description
 
\var{SetMouseXRange}
sets the minimum (\var{Min}) and maximum (\var{Max}) horizontal coordinates in between which the
mouse cursor can move.
While \var{Min} and \var{Max} are Longint parameters, only the lower 16 bits 
are used.

\Errors
None.
\SeeAlso
\seep{SetMouseYRange}, \seep{SetMouseWindow}
\end{procedure}

\FPCexample{mouse6}

\begin{procedure}{SetMouseYRange}
\Declaration
Procedure SetMouseYRange (Min, Max: Longint);

\Description

\var{SetMouseYRange}
sets the minimum (\var{Min}) and maximum (\var{Max}) vertical coordinates in between which the
mouse cursor can move.
While \var{Min} and \var{Max} are Longint parameters, only the lower 16 bits 
are used.

\Errors
None.
\SeeAlso
\seep{SetMouseXRange}, \seep{SetMouseWindow}
\end{procedure}

For an example, see \seep{SetMouseXRange}.

\begin{procedure}{ShowMouse}
\Declaration
Procedure ShowMouse ;

\Description

\var{ShowMouse} makes the mouse cursor visible.
At the start of your progam, the mouse cursor is invisible.

\Errors
None.
\SeeAlso
\seep{HideMouse},\seep{SetMouseHideWindow}
\end{procedure}

\FPCexample{mouse2}


% the objects unit
%
%   $Id$
%   This file is part of the FPC documentation.
%   Copyright (C) 1998, by Michael Van Canneyt
%
%   The FPC documentation is free text; you can redistribute it and/or
%   modify it under the terms of the GNU Library General Public License as
%   published by the Free Software Foundation; either version 2 of the
%   License, or (at your option) any later version.
%
%   The FPC Documentation is distributed in the hope that it will be useful,
%   but WITHOUT ANY WARRANTY; without even the implied warranty of
%   MERCHANTABILITY or FITNESS FOR A PARTICULAR PURPOSE.  See the GNU
%   Library General Public License for more details.
%
%   You should have received a copy of the GNU Library General Public
%   License along with the FPC documentation; see the file COPYING.LIB.  If not,
%   write to the Free Software Foundation, Inc., 59 Temple Place - Suite 330,
%   Boston, MA 02111-1307, USA. 
%
\chapter{The Objects unit.}
This chapter documents the \file{objects} unit. The unit was implemented by
many people, and was mainly taken from the FreeVision sources.

The methods and fields that are in a \var{Private} part of an object
declaration have been left out of this documentation.

\section{Constants}
The following constants are error codes, returned by the various stream
objects.

\begin{verbatim}
CONST
   stOk         =  0; { No stream error }
   stError      = -1; { Access error }
   stInitError  = -2; { Initialize error }
   stReadError  = -3; { Stream read error }
   stWriteError = -4; { Stream write error }
   stGetError   = -5; { Get object error }
   stPutError   = -6; { Put object error }
   stSeekError  = -7; { Seek error in stream }
   stOpenError  = -8; { Error opening stream }
\end{verbatim}
These constants can be passed to constructors of file streams:
\begin{verbatim}
CONST
   stCreate    = $3C00; { Create new file }
   stOpenRead  = $3D00; { Read access only }
   stOpenWrite = $3D01; { Write access only }
   stOpen      = $3D02; { Read/write access }
\end{verbatim}

The following constants are error codes, returned by the collection list
objects:
\begin{verbatim}
CONST
   coIndexError = -1; { Index out of range }
   coOverflow   = -2; { Overflow }
\end{verbatim}

Maximum data sizes (used in determining how many data can be used.

\begin{verbatim}
CONST
   MaxBytes = 128*1024*1024;                          { Maximum data size }
   MaxWords = MaxBytes DIV SizeOf(Word);              { Max word data size }
   MaxPtrs = MaxBytes DIV SizeOf(Pointer);            { Max ptr data size }
   MaxCollectionSize = MaxBytes DIV SizeOf(Pointer);  { Max collection size }
\end{verbatim}

\section{Types}
The follwing auxiliary types are defined:
\begin{verbatim}
TYPE
   { Character set }
   TCharSet = SET Of Char;                            
   PCharSet = ^TCharSet;

   { Byte array }
   TByteArray = ARRAY [0..MaxBytes-1] Of Byte;        
   PByteArray = ^TByteArray;

   { Word array }
   TWordArray = ARRAY [0..MaxWords-1] Of Word;        
   PWordArray = ^TWordArray;

   { Pointer array }
   TPointerArray = Array [0..MaxPtrs-1] Of Pointer;   
   PPointerArray = ^TPointerArray; 

   { String pointer }
   PString = ^String;

   { Filename array }
   AsciiZ = Array [0..255] Of Char;

   Sw_Word    = Cardinal;
   Sw_Integer = LongInt;
\end{verbatim}
The following records are used internaly for easy type conversion:
\begin{verbatim}
TYPE
   { Word to bytes}
   WordRec = packed RECORD
     Lo, Hi: Byte;     
   END;

   { LongInt to words }
   LongRec = packed RECORD
     Lo, Hi: Word;
   END;

  { Pointer to words }
   PtrRec = packed RECORD
     Ofs, Seg: Word;
   END;
\end{verbatim}
The following record is used when streaming objects:

\begin{verbatim}
TYPE
   PStreamRec = ^TStreamRec;
   TStreamRec = Packed RECORD
      ObjType: Sw_Word;
      VmtLink: pointer;
      Load : Pointer;
      Store: Pointer;
      Next : PStreamRec;
   END;
\end{verbatim}

The \var{TPoint} basic object is used in the \var{TRect} object (see
\sees{TRect}):
\begin{verbatim}
TYPE
   PPoint = ^TPoint;
   TPoint = OBJECT
      X, Y: Sw_Integer;
   END;
\end{verbatim}

\section{TRect}
\label{se:TRect}

The \var{TRect} object is declared as follows:
\begin{verbatim}
   TRect = OBJECT
      A, B: TPoint;
      FUNCTION Empty: Boolean;
      FUNCTION Equals (R: TRect): Boolean;
      FUNCTION Contains (P: TPoint): Boolean;
      PROCEDURE Copy (R: TRect);
      PROCEDURE Union (R: TRect);
      PROCEDURE Intersect (R: TRect);
      PROCEDURE Move (ADX, ADY: Sw_Integer);
      PROCEDURE Grow (ADX, ADY: Sw_Integer);
      PROCEDURE Assign (XA, YA, XB, YB: Sw_Integer);
   END;
\end{verbatim}

\begin{function}{TRect.Empty}
\Declaration
Function TRect.Empty: Boolean;
\Description
\var{Empty} returns \var{True} if the rectangle defined by the corner points 
\var{A}, \var{B} has zero or negative surface.
\Errors
None.
\SeeAlso
\seef{TRect.Equals}, \seef{TRect.Contains}
\end{function}

\latex{\inputlisting{objectex/ex1.pp}}
\html{\input{objectex/ex1.tex}}

\begin{function}{TRect.Equals}      
\Declaration
Function TRect.Equals (R: TRect): Boolean;
\Description
\var{Equals} returns \var{True} if the rectangle has the 
same corner points \var{A,B} as the rectangle R, and \var{False}
otherwise.
\Errors
None.
\SeeAlso
\seefl{Empty}{TRect.Empty}, \seefl{Contains}{TRect.Contains}
\end{function}

For an example, see \seef{TRect.Empty}

\begin{function}{TRect.Contains}
\Declaration
Function TRect.Contains (P: TPoint): Boolean;
\Description
\var{Contains} returns \var{True} if the point \var{P} is contained
in the rectangle (including borders), \var{False} otherwise.
\Errors
None.
\SeeAlso
\seepl{Intersect}{TRect.Intersect}, \seefl{Equals}{TRect.Equals}
\end{function}

\begin{procedure}{TRect.Copy}
\Declaration     
Procedure TRect.Copy (R: TRect);
\Description
Assigns the rectangle R to the object. After the call to \var{Copy}, the
rectangle R has been copied to the object that invoked \var{Copy}.
\Errors
None.
\SeeAlso
\seepl{Assign}{TRect.Assign}
\end{procedure}

\latex{\inputlisting{objectex/ex2.pp}}
\html{\input{objectex/ex2.tex}}

\begin{procedure}{TRect.Union}
\Declaration
Procedure TRect.Union (R: TRect);
\Description
\var{Union} enlarges the current rectangle so that it becomes the union
of the current rectangle with the rectangle \var{R}.
\Errors
None.
\SeeAlso
\seepl{Intersect}{TRect.Intersect}
\end{procedure}

\latex{\inputlisting{objectex/ex3.pp}}
\html{\input{objectex/ex3.tex}}

\begin{procedure}{TRect.Intersect}
\Declaration
Procedure TRect.Intersect (R: TRect);
\Description
\var{Intersect} makes the intersection of the current rectangle with
\var{R}. If the intersection is empty, then the rectangle is set to the empty
rectangle at coordinate (0,0).
\Errors
None.
\SeeAlso
\seepl{Union}{TRect.Union}
\end{procedure}

\latex{\inputlisting{objectex/ex4.pp}}
\html{\input{objectex/ex4.tex}}

\begin{procedure}{TRect.Move}
\Declaration
Procedure TRect.Move (ADX, ADY: Sw\_Integer);
\Description
\var{Move} moves the current rectangle along a vector with components
\var{(ADX,ADY)}. It adds \var{ADX} to the X-coordinate of both corner
points, and \var{ADY} to both end points.
\Errors
None.
\SeeAlso
\seepl{Grow}{TRect.Grow}
\end{procedure}

\latex{\inputlisting{objectex/ex5.pp}}
\html{\input{objectex/ex5.tex}}

\begin{procedure}{TRect.Grow}
\Declaration
Procedure TRect.Grow (ADX, ADY: Sw\_Integer);
\Description
\var{Grow} expands the rectangle with an amount \var{ADX} in the \var{X}
direction (both on the left and right side of the rectangle, thus adding a 
length 2*ADX to the width of the rectangle), and an amount \var{ADY} in 
the \var{Y} direction (both on the top and the bottom side of the rectangle,
adding a length 2*ADY to the height of the rectangle. 

\var{ADX} and \var{ADY} can be negative. If the resulting rectangle is empty, it is set 
to the empty rectangle at \var{(0,0)}.
\Errors
None.
\SeeAlso
\seepl{Move}{TRect.Move}.
\end{procedure}


\latex{\inputlisting{objectex/ex6.pp}}
\html{\input{objectex/ex7.tex}}

\begin{procedure}{TRect.Assign}
\Declaration
Procedure Trect.Assign (XA, YA, XB, YB: Sw\_Integer);
\Description
\var{Assign} sets the corner points of the rectangle to \var{(XA,YA)} and 
\var{(Xb,Yb)}.
\Errors
None.
\SeeAlso
\seepl{Copy}{TRect.Copy}
\end{procedure}

For an example, see \seep{TRect.Copy}.

\section{TObject}
\label{se:TObject}

The full declaration of the \var{TObject} type is:
\begin{verbatim}
TYPE
   TObject = OBJECT
      CONSTRUCTOR Init;
      PROCEDURE Free;
      DESTRUCTOR Done;Virtual;
   END;
   PObject = ^TObject;
\end{verbatim}
\begin{procedure}{TObject.Init}
\Declaration
Constructor TObject.Init;
\Description
Instantiates a new object of type \var{TObject}. It fills the instance up
with Zero bytes.
\Errors
None.
\SeeAlso
\seepl{Free}{TObject.Free}, \seepl{Done}{TObject.Done}
\end{procedure}

For an example, see \seepl{Free}{TObject.Free}

\begin{procedure}{TObject.Free}
\Declaration
Procedure TObject.Free;
\Description
\var{Free} calls the destructor of the object, and releases the memory
occupied by the instance of the object.
\Errors
No checking is performed to see whether \var{self} is \var{nil} and whether
the object is indeed allocated on the heap.
\SeeAlso
\seepl{Init}{TObject.Init}, \seepl{Done}{TObject.Done}
\end{procedure}

\latex{\inputlisting{objectex/ex7.pp}}
\html{\input{objectex/ex7.tex}}

\begin{procedure}{TObject.Done}
\Declaration
Destructor TObject.Done;Virtual;
\Description
\var{Done}, the destructor of \var{TObject} does nothing. It is mainly
intended to be used in the \seep{TObject.Free} method.
\Errors
None.
\SeeAlso
\seepl{Free}{TObject.Free}, \seepl{Init}{TObject.Init}
\end{procedure}

\section{TStream}
\label{se:TStream}

The \var{TStream} object is the ancestor for all streaming objects, i.e.
objects that have the capability to store and retrieve data.

It defines a number of methods that are common to all objects that implement
streaming, many of them are virtual, and are only implemented in the
descendrnt types.

Programs should not instantiate objects of type TStream directly, but
instead instantiate a descendant type, such as \var{TDosStream},
\var{TMemoryStream}.

This is the full declaration of the \var{TStream} object:
\begin{verbatim}
TYPE
   TStream = OBJECT (TObject)
         Status    : Integer; { Stream status }
         ErrorInfo : Integer; { Stream error info }
         StreamSize: LongInt; { Stream current size }
         Position  : LongInt; { Current position }
      FUNCTION Get: PObject;
      FUNCTION StrRead: PChar;
      FUNCTION GetPos: Longint; Virtual;
      FUNCTION GetSize: Longint; Virtual;
      FUNCTION ReadStr: PString;
      PROCEDURE Open (OpenMode: Word); Virtual;
      PROCEDURE Close; Virtual;
      PROCEDURE Reset;
      PROCEDURE Flush; Virtual;
      PROCEDURE Truncate; Virtual;
      PROCEDURE Put (P: PObject);
      PROCEDURE StrWrite (P: PChar);
      PROCEDURE WriteStr (P: PString);
      PROCEDURE Seek (Pos: LongInt); Virtual;
      PROCEDURE Error (Code, Info: Integer); Virtual;
      PROCEDURE Read (Var Buf; Count: Sw_Word); Virtual;
      PROCEDURE Write (Var Buf; Count: Sw_Word); Virtual;
      PROCEDURE CopyFrom (Var S: TStream; Count: Longint);
   END;
   PStream = ^TStream;
\end{verbatim}

\begin{function}{TStream.Get}
\Declaration
Function TStream.Get : PObject;
\Description
\var{Get} reads an object definition  from a stream, and returns
a pointer to an instance of this object.
\Errors
On error, \var{TStream.Status} is set, and NIL is returned.
\SeeAlso 
\seepl{Put}{TStream.Put}
\end{function}

\begin{function}{TStream.StrRead}
\Declaration
Function TStream.StrRead: PChar;
\Description
\var{StrRead} reads a string from the stream, allocates memory
for it, and returns a pointer to a null-terminated copy of the string
on the heap.
\Errors
On error, \var{Nil} is returned.
\SeeAlso
\seepl{StrWrite}{TStream.StrWrite}, \seefl{ReadStr}{TStream.ReadStr}
\end{function}

\begin{function}{TStream.GetPos}
\Declaration 
TSTream.GetPos : Longint; Virtual;
\Description
If the stream's status is \var{stOk}, \var{GetPos} returns the current 
position in the stream. Otherwise it returns \var{-1}
\Errors
\var{-1} is returned if the status is an error condition.
\SeeAlso
\seepl{Seek}{TStream.Seek}, \seefl{GetSize}{TStream.GetSize}
\end{function}

\begin{function}{TStream.GetSize}
\Declaration
Function TStream.GetSize: Longint; Virtual;
\Description
If the stream's status is \var{stOk} then \var{GetSize} returns
the size of the stream, otherwise it returns \var{-1}.
\Errors
\var{-1} is returned if the status is an error condition.
\SeeAlso
\seepl{Seek}{TStream.Seek}, \seefl{GetPos}{TStream.GetPos}
\end{function}

\begin{function}{TStream.ReadStr}
\Declaration
Function TStream.ReadStr: PString;
\Description
\var{ReadStr} reads a string from the stream, copies it to the heap
and returns a pointer to this copy. The string is saved as a pascal
string, and hence is NOT null terminated.
\Errors
On error (e.g. not enough memory), \var{Nil} is returned.
\SeeAlso
\seefl{StrRead}{TStream.StrRead}
\end{function}

\begin{procedure}{TStream.Open}
\Declaration
Procedure TStream.Open (OpenMode: Word); Virtual;
\Description
\var{Open} is an abstract method, that should be overridden by descendent
objects. Since opening a stream depends on the stream's type this is not
surprising.
\Errors
None.
\SeeAlso
\seepl{Close}{TStream.Close}, \seepl{Reset}{TStream.Reset}
\end{procedure}

\begin{procedure}{TStream.Close}
\Declaration
Procedure TStream.Close; Virtual;
\Description
\var{Close} is an abstract method, that should be overridden by descendent
objects. Since Closing a stream depends on the stream's type this is not
surprising.
\Errors
None.
\SeeAlso
\seepl{Open}{TStream.Open}, \seepl{Reset}{TStream.Reset}
\end{procedure}

\begin{procedure}{TStream.Reset}
\Declaration
PROCEDURE TStream.Reset;
\Description
\var{Reset} sets the stream's status to \var{0}, as well as the ErrorInfo
\Errors
None.
\SeeAlso
\seepl{Open}{TStream.Open}, \seepl{Close}{TStream.Close}
\end{procedure}

\begin{procedure}{TStream.Flush}
\Declaration 
Procedure TStream.Flush; Virtual;
\Description
\var{Flush} is an abstract method that should be overridden by descendent
objects. It serves to enable the programmer to tell streams that implement 
a buffer to clear the buffer.
\Errors
None.
\SeeAlso
\seepl{Truncate}{TStream.Truncate}
\end{procedure}

\begin{procedure}{TStream.Truncate}
\Declaration
Procedure TStream.Truncate; Virtual;
\Description
\var{Truncate} is an abstract procedure that should be overridden by
descendent objects. It serves to enable the programmer to truncate the
size of the stream to the current file position.
\Errors
None.
\SeeAlso
\seepl{Seek}{TStream.Seek}
\end{procedure}

\begin{procedure}{TStream.Put}
\Declaration
Procedure TStream.Put (P: PObject);
\Description
\var{Put} writes the object pointed to by \var{P}. \var{P} should be
non-nil. The object type must have been registered with \seep{RegisterType}.

After the object has been written, it can be read again with \seefl{Get}{TStream.Get}.
\Errors
No check is done whether P is \var{Nil} or not. Passing \var{Nil} will cause
a run-time error 216 to be generated. If the object has not been registered,
the status of the stream will be set to \var{stPutError}.
\SeeAlso
\seefl{Get}{TStream.Get}
\end{procedure}

\begin{procedure}{TStream.StrWrite}
\Declaration
Procedure TStream.StrWrite (P: PChar);
\Description
\var{StrWrite} writes the null-terminated string \var{P} to the stream.
\var{P} can only be 65355 bytes long.
\Errors
None.
\SeeAlso
\seepl{WriteStr}{TStream.WriteStr}, \seefl{StrRead}{TStream.StrRead},
\seefl{ReadStr}{TStream.ReadStr}
\end{procedure}

\begin{procedure}{TStream.WriteStr}
\Declaration
Procedure TStream.WriteStr (P: PString);
\Description
\var{StrWrite} writes the pascal string pointed to by \var{P} to the stream.
\Errors
None.
\SeeAlso
\seepl{StrWrite}{TStream.StrWrite}, \seefl{StrRead}{TStream.StrRead},
\seefl{ReadStr}{TStream.ReadStr}
\end{procedure}

\begin{procedure}{TStream.Seek}
\Declaration      
PROCEDURE TStream.Seek (Pos: LongInt); Virtual;
\Description
Seek sets the position to \var{Pos}. This position is counted
from the beginning, and is zero based. (i.e. seeek(0) sets the position
pointer on the first byte of the stream)
\Errors
If \var{Pos} is larger than the stream size, \var{Status} is set to
\var{StSeekError}.
\SeeAlso
\seefl{GetPos}{TStream.GetPos}, \seefl{GetSize}{TStream.GetSize}
\end{procedure}

\begin{procedure}{TStream.Error}
\Declaration
Procedure TStream.Error (Code, Info: Integer); Virtual;
\Description
\var{Error} sets the stream's status to \var{Code} and \var{ErrorInfo}
to \var{Info}. If the \var{StreamError} procedural variable is set,
\var{Error} executes it, passing \var{Self} as an argument.

This method should not be called directly from a program. It is intended to
be used in descendent objects.
\Errors
None.
\SeeAlso
\end{procedure}

\begin{procedure}{TStream.Read}
\Declaration
Procedure TStream.Read (Var Buf; Count: Sw\_Word); Virtual;
\Description
\var{Read} is an abstract method that should be overridden by descendent
objects.

\var{Read} reads \var{Count} bytes from the stream into \var{Buf}.
It updates the position pointer, increasing it's value with \var{Count}. 
\var{Buf} must be large enough to contain \var{Count} bytes.
\Errors
No checking is done to see if \var{Buf} is large enough to contain
\var{Count} bytes. 
\SeeAlso
\seepl{Write}{TStream.Write}, \seefl{ReadStr}{TStream.ReadStr},
\seefl{StrRead}{TStream.StrRead}
\end{procedure}

\begin{procedure}{TStream.Write}
\Declaration
Procedure TStream.Write (Var Buf; Count: Sw\_Word); Virtual;
\Description
\var{Write} is an abstract method that should be overridden by descendent
objects.

\var{Write} writes \var{Count} bytes to the stream from \var{Buf}.
It updates the position pointer, increasing it's value with \var{Count}. 
\Errors
No checking is done to see if \var{Buf} actually contains \var{Count} bytes. 
\SeeAlso
\seepl{Read}{TStream.Read}, \seepl{WriteStr}{TStream.WriteStr},
\seepl{StrWrite}{TStream.StrWrite}
\end{procedure}

\begin{procedure}{TStream.CopyFrom}
\Declaration
Procedure TStream.CopyFrom (Var S: TStream; Count: Longint);
\Description
\var{CopyFrom} reads Count bytes from stream \var{S} and stores them
in the current stream. It uses the \seepl{Read}{TStream.Read} method
to read the data, and the \seepl{Write}{TStream.Write} method to
write in the current stream.
\Errors
None.
\SeeAlso
\seepl{Read}{TStream.Read}, \seepl{Write}{TStream.Write}
\end{procedure}

\section{TDosStream}
\label{se:TDosStream}

\var{TDosStream} is a steam that stores it's contents in a file.
it overrides a couple of methods of \var{TSteam} for this.

In addition to the fields inherited from \var{TStream} (see \sees{TStream}),
there are some extra fields, that describe the file. (mainly the name and
the OS file handle)

No buffering in memory is done when using \var{TDosStream}. 
All data are written directly to the file. For a stream that buffers 
in memory, see \sees{TBufStream}.

Here is the full declaration of the \var{TDosStream} object:
\begin{verbatim}
TYPE
   TDosStream = OBJECT (TStream)
         Handle: THandle; { DOS file handle }
         FName : AsciiZ;  { AsciiZ filename }
      CONSTRUCTOR Init (FileName: FNameStr; Mode: Word);
      DESTRUCTOR Done; Virtual;
      PROCEDURE Close; Virtual;
      PROCEDURE Truncate; Virtual;
      PROCEDURE Seek (Pos: LongInt); Virtual;
      PROCEDURE Open (OpenMode: Word); Virtual;
      PROCEDURE Read (Var Buf; Count: Sw_Word); Virtual;
      PROCEDURE Write (Var Buf; Count: Sw_Word); Virtual;
   END;
   PDosStream = ^TDosStream;
\end{verbatim}

\begin{procedure}{TDosStream.Init}
\Declaration
Constructor Init (FileName: FNameStr; Mode: Word);
\Description
\var{Init} instantiates an instance of \var{TDosStream}. The name of the 
file that contains (or will contain) the data of the stream is given in
\var{FileName}. The \var{Mode} parameter determines whether a new file 
should be created and what access rights you have on the file. 
It can be one of the following constants:
\begin{description}
\item[stCreate] Creates a new file.
\item[stOpenRead] Read access only.
\item[stOpenWrite] Write access only.
\item[stOpen] Read and write access.
\end{description}
\Errors
On error, \var{Status} is set to \var{stInitError}, and \var{ErrorInfo}
is set to the \dos error code.
\SeeAlso
\seepl{Done}{TDosStream.Done}
\end{procedure}

\begin{procedure}{TDosStream.Done}
\Declaration
Destructor TDosStream.Done; Virtual;
\Description
\var{Done} closes the file if it was open and cleans up the 
instance of \var{TDosStream}. 
\Errors
None.
\SeeAlso
\seepl{Init}{TDosStream.Init},
\seepl{Close}{TDosStream.Close}
\end{procedure}

\begin{procedure}{TDosStream.Close}
\Declaration
Pocedure TDosStream.Close; Virtual;
\Description
\var{Close} closes the file if it was open, and sets \var{Handle} to -1. 
Contrary to \seepl{Done}{TDosStream.Done} it does not clean up the instance
of \var{TDosStream}
\Errors
None.
\SeeAlso
\seep{TStream.Close}, \seepl{Init}{TDosStream.Init},
\seepl{Done}{TDosStream.Done}
\end{procedure}

\begin{procedure}{TDosStream.Truncate}
\Declaration
Procedure TDosStream.Truncate; Virtual;
\Description
If the status of the stream is \var{stOK}, then \var{Truncate} tries to
truncate the stream size to the current file position.
\Errors
If an error occurs, the stream's status is set to \var{stError} and
\var{ErrorInfo} is set to the OS error code.
\SeeAlso
\seep{TStream.Truncate}, \seefl{GetSize}{TStream.GetSize}
\end{procedure}

\begin{procedure}{TDosStream.Seek}
\Declaration
Procedure TDosStream.Seek (Pos: LongInt); Virtual;
\Description
If the stream's status is \var{stOK}, then \var{Seek} sets the 
file position to \var{Pos}. \var{Pos} is a zero-based offset, counted from
the beginning of the file.
\Errors
In case an error occurs, the stream's status is set to \var{stSeekError},
and the OS error code is stored in \var{ErrorInfo}.
\SeeAlso
\seep{TStream.Seek}, \seefl{GetPos}{TStream.GetPos}
\end{procedure}

\begin{procedure}{TDosStream.Open}
\Declaration
Procedure TDosStream.Open (OpenMode: Word); Virtual;
\Description
If the stream's status is \var{stOK}, and the stream is closed then
\var{Open} re-opens the file stream with mode \var{OpenMode}.
This call can be used after a \seepl{Close}{TDosStream.Close} call.
\Errors
If an error occurs when re-opening the file, then \var{Status} is set
to \var{stOpenError}, and the OS error code is stored in \var{ErrorInfo}
\SeeAlso
\seep{TStream.Open}, \seepl{Close}{TDosStream.Close}
\end{procedure}

\begin{procedure}{TDosStream.Read}
\Declaration
Procedure TDosStream.Read (Var Buf; Count: Sw\_Word); Virtual;
\Description
If the Stream is open and the stream status is \var{stOK} then 
\var{Read} will read \var{Count} bytes from the stream and place them
in  \var{Buf}.
\Errors
In case of an error, \var{Status} is set to \var{StReadError}, and
\var{ErrorInfo} gets the OS specific error, or 0 when an attempt was
made to read beyond the end of the stream.
\SeeAlso
\seep{TStream.Read}, \seepl{Write}{TDosStream.Write}
\end{procedure}

\begin{procedure}{TDosStream.Write}
\Declaration
Procedure TDosStream.Write (Var Buf; Count: Sw\_Word); Virtual;
\Description
If the Stream is open and the stream status is \var{stOK} then 
\var{Write} will write \var{Count} bytes from \var{Buf} and place them
in the stream.
\Errors
In case of an error, \var{Status} is set to \var{StWriteError}, and
\var{ErrorInfo} gets the OS specific error.
\SeeAlso
\seep{TStream.Write}, \seepl{Read}{TDosStream.Read}

\end{procedure}

\section{TBufStream}
\label{se:TBufStream}

\var{Bufstream} implements a buffered file stream. That is, all data written
to the stream is written to memory first. Only when the buffer is full, or
on explicit request, the data is written to disk.

Also, when reading from the stream, first the buffer is checked if there is
any unread data in it. If so, this is read first. If not the buffer is
filled again, and then the data is read from the buffer.

The size of the buffer is fixed and is set when constructing the file.

This is useful if you need heavy throughput for your stream, because it
speeds up operations.

\begin{verbatim}
TYPE
   TBufStream = OBJECT (TDosStream)
         LastMode: Byte;       { Last buffer mode }
         BufSize : Sw_Word;    { Buffer size }
         BufPtr  : Sw_Word;    { Buffer start }
         BufEnd  : Sw_Word;    { Buffer end }
         Buffer  : PByteArray; { Buffer allocated }
      CONSTRUCTOR Init (FileName: FNameStr; Mode, Size: Word);
      DESTRUCTOR Done; Virtual;
      PROCEDURE Close; Virtual;
      PROCEDURE Flush; Virtual;
      PROCEDURE Truncate; Virtual;
      PROCEDURE Seek (Pos: LongInt); Virtual;
      PROCEDURE Open (OpenMode: Word); Virtual;
      PROCEDURE Read (Var Buf; Count: Sw_Word); Virtual;
      PROCEDURE Write (Var Buf; Count: Sw_Word); Virtual;
   END;
   PBufStream = ^TBufStream;
\end{verbatim}

\begin{procedure}{TBufStream.Init}
\Declaration
Constructor Init (FileName: FNameStr; Mode,Size: Word);
\Description
\var{Init} instantiates an instance of \var{TBufStream}. The name of the 
file that contains (or will contain) the data of the stream is given in
\var{FileName}. The \var{Mode} parameter determines whether a new file 
should be created and what access rights you have on the file. 
It can be one of the following constants:
\begin{description}
\item[stCreate] Creates a new file.
\item[stOpenRead] Read access only.
\item[stOpenWrite] Write access only.
\item[stOpen] Read and write access.
\end{description}
The \var{Size} parameter determines the size of the buffer that will be
created. It should be different from zero.
\Errors
On error, \var{Status} is set to \var{stInitError}, and \var{ErrorInfo}
is set to the \dos error code.
\SeeAlso
\seep{TDosStream.Init}, \seepl{Done}{TBufStream.Done}
\end{procedure}

\begin{procedure}{TBufStream.Done}
\Declaration
Destructor TBufStream.Done; Virtual;
\Description
\var{Done} flushes and closes the file if it was open and cleans up the 
instance of \var{TBufStream}. 
\Errors
None.
\SeeAlso
\seep{TDosStream.Done}, \seepl{Init}{TBufStream.Init},
\seepl{Close}{TBufStream.Close}
\end{procedure}

\begin{procedure}{TBufStream.Close}
\Declaration
Pocedure TBufStream.Close; Virtual;
\Description
\var{Close} flushes and closes the file if it was open, and sets \var{Handle} to -1. 
Contrary to \seepl{Done}{TBufStream.Done} it does not clean up the instance
of \var{TBufStream}
\Errors
None.
\SeeAlso
\seep{TStream.Close}, \seepl{Init}{TBufStream.Init},
\seepl{Done}{TBufStream.Done}
\end{procedure}

\begin{procedure}{TBufStream.Flush}
\Declaration
Pocedure TBufStream.Flush; Virtual;
\Description
When the stream is in write mode, the contents of the buffer are written to
disk, and the buffer position is set to zero.

When the stream is in read mode, the buffer position is set to zero.
\Errors
Write errors may occur if the file was in write mode.
see \seepl{Write}{TBufStream.Write} for more info on the errors.
\SeeAlso
\seep{TStream.Close}, \seepl{Init}{TBufStream.Init},
\seepl{Done}{TBufStream.Done}
\end{procedure}

\begin{procedure}{TBufStream.Truncate}
\Declaration
Procedure TBufStream.Truncate; Virtual;
\Description
If the status of the stream is \var{stOK}, then \var{Truncate} tries to
flush the buffer, and then truncates the stream size to the current 
file position.
\Errors
Errors can be those of \seepl{Flush}{TBufStream.Flush} or
\seep{TDosStream.Truncate}.
\SeeAlso
\seep{TStream.Truncate}, \seep{TDosStream.Truncate},
\seefl{GetSize}{TStream.GetSize}
\end{procedure}

\begin{procedure}{TBufStream.Seek}
\Declaration
Procedure TBufStream.Seek (Pos: LongInt); Virtual;
\Description
If the stream's status is \var{stOK}, then \var{Seek} sets the 
file position to \var{Pos}. \var{Pos} is a zero-based offset, counted from
the beginning of the file.
\Errors
In case an error occurs, the stream's status is set to \var{stSeekError},
and the OS error code is stored in \var{ErrorInfo}.
\SeeAlso
\seep{TStream.Seek}, \seefl{GetPos}{TStream.GetPos}
\end{procedure}

\begin{procedure}{TBufStream.Open}
\Declaration
Procedure TBufStream.Open (OpenMode: Word); Virtual;
\Description
If the stream's status is \var{stOK}, and the stream is closed then
\var{Open} re-opens the file stream with mode \var{OpenMode}.
This call can be used after a \seepl{Close}{TBufStream.Close} call.
\Errors
If an error occurs when re-opening the file, then \var{Status} is set
to \var{stOpenError}, and the OS error code is stored in \var{ErrorInfo}
\SeeAlso
\seep{TStream.Open}, \seepl{Close}{TBufStream.Close}
\end{procedure}

\begin{procedure}{TBufStream.Read}
\Declaration
Procedure TBufStream.Read (Var Buf; Count: Sw\_Word); Virtual;
\Description
If the Stream is open and the stream status is \var{stOK} then 
\var{Read} will read \var{Count} bytes from the stream and place them
in  \var{Buf}.

\var{Read} will first try to read the data from the stream's internal
buffer. If insufficient data is available, the buffer will be filled before
contiunuing to read. This process is repeated until all needed data 
has been read.

\Errors
In case of an error, \var{Status} is set to \var{StReadError}, and
\var{ErrorInfo} gets the OS specific error, or 0 when an attempt was
made to read beyond the end of the stream.
\SeeAlso
\seep{TStream.Read}, \seepl{Write}{TBufStream.Write}
\end{procedure}

\begin{procedure}{TBufStream.Write}
\Declaration
Procedure TBufStream.Write (Var Buf; Count: Sw\_Word); Virtual;
\Description
If the Stream is open and the stream status is \var{stOK} then 
\var{Write} will write \var{Count} bytes from \var{Buf} and place them
in the stream.

\var{Write} will first try to write the data to the stream's internal
buffer. When the internal buffer is full, then the contents will be written 
to disk. This process is repeated until all data has been written.
\Errors
In case of an error, \var{Status} is set to \var{StWriteError}, and
\var{ErrorInfo} gets the OS specific error.
\SeeAlso
\seep{TStream.Write}, \seepl{Read}{TBufStream.Read}

\end{procedure}

\section{TMemoryStream}
\section{se:TMemoryStream}

\begin{verbatim}
TYPE
   TMemoryStream = OBJECT (TStream)
         BlkCount: Sw_Word;                           { Number of segments }
         BlkSize : Word;                              { Memory block size }
         MemSize : LongInt;                           { Memory alloc size }
         BlkList : PPointerArray;                     { Memory block list }
      CONSTRUCTOR Init (ALimit: Longint; ABlockSize: Word);
      DESTRUCTOR Done;                                               Virtual;
      PROCEDURE Truncate;                                            Virtual;
      PROCEDURE Read (Var Buf; Count: Sw_Word);                      Virtual;
      PROCEDURE Write (Var Buf; Count: Sw_Word);                     Virtual;
      PRIVATE
      FUNCTION ChangeListSize (ALimit: Sw_Word): Boolean;
   END;
   PMemoryStream = ^TMemoryStream;
\end{verbatim}

\section{TCollection}
\label{se:TCollection}

\begin{verbatim}
TYPE
   TItemList = Array [0..MaxCollectionSize - 1] Of Pointer;
   PItemList = ^TItemList;

   TCollection = OBJECT (TObject)
         Items: PItemList;  { Item list pointer }
         Count: Sw_Integer; { Item count }
         Limit: Sw_Integer; { Item limit count }
         Delta: Sw_Integer; { Inc delta size }
      CONSTRUCTOR Init (ALimit, ADelta: Sw_Integer);
      CONSTRUCTOR Load (Var S: TStream);
      DESTRUCTOR Done; Virtual;
      FUNCTION At (Index: Sw_Integer): Pointer;
      FUNCTION IndexOf (Item: Pointer): Sw_Integer; Virtual;
      FUNCTION GetItem (Var S: TStream): Pointer; Virtual;
      FUNCTION LastThat (Test: Pointer): Pointer;
      FUNCTION FirstThat (Test: Pointer): Pointer;
      PROCEDURE Pack;
      PROCEDURE FreeAll;
      PROCEDURE DeleteAll;
      PROCEDURE Free (Item: Pointer);
      PROCEDURE Insert (Item: Pointer); Virtual;
      PROCEDURE Delete (Item: Pointer);
      PROCEDURE AtFree (Index: Sw_Integer);
      PROCEDURE FreeItem (Item: Pointer); Virtual;
      PROCEDURE AtDelete (Index: Sw_Integer);
      PROCEDURE ForEach (Action: Pointer);
      PROCEDURE SetLimit (ALimit: Sw_Integer); Virtual;
      PROCEDURE Error (Code, Info: Integer); Virtual;
      PROCEDURE AtPut (Index: Sw_Integer; Item: Pointer);
      PROCEDURE AtInsert (Index: Sw_Integer; Item: Pointer);
      PROCEDURE Store (Var S: TStream);
      PROCEDURE PutItem (Var S: TStream; Item: Pointer); Virtual;
   END;
   PCollection = ^TCollection;
\end{verbatim}

\section{TSortedCollection}
\label{se:TSortedCollection}

\begin{verbatim}
TYPE
   TSortedCollection = OBJECT (TCollection)
         Duplicates: Boolean; { Duplicates flag }
      CONSTRUCTOR Init (ALimit, ADelta: Sw_Integer);
      CONSTRUCTOR Load (Var S: TStream);
      FUNCTION KeyOf (Item: Pointer): Pointer; Virtual;
      FUNCTION IndexOf (Item: Pointer): Sw_Integer; Virtual;
      FUNCTION Compare (Key1, Key2: Pointer): Sw_Integer; Virtual;
      FUNCTION Search (Key: Pointer; Var Index: Sw_Integer): Boolean;Virtual;
      PROCEDURE Insert (Item: Pointer); Virtual;
      PROCEDURE Store (Var S: TStream);
   END;
   PSortedCollection = ^TSortedCollection;
\end{verbatim}

\section{TStringCollection}
\label{se:TStringCollection}

\begin{verbatim}
TYPE
   TStringCollection = OBJECT (TSortedCollection)
      FUNCTION GetItem (Var S: TStream): Pointer; Virtual;
      FUNCTION Compare (Key1, Key2: Pointer): Sw_Integer; Virtual;
      PROCEDURE FreeItem (Item: Pointer); Virtual;
      PROCEDURE PutItem (Var S: TStream; Item: Pointer); Virtual;
   END;
   PStringCollection = ^TStringCollection;
\end{verbatim}

\section{TStrCollection}
\label{se:TStrCollection}

\begin{verbatim}
TYPE
   TStrCollection = OBJECT (TSortedCollection)
      FUNCTION Compare (Key1, Key2: Pointer): Sw_Integer; Virtual;
      FUNCTION GetItem (Var S: TStream): Pointer; Virtual;
      PROCEDURE FreeItem (Item: Pointer); Virtual;
      PROCEDURE PutItem (Var S: TStream; Item: Pointer); Virtual;
   END;
   PStrCollection = ^TStrCollection;
\end{verbatim}


\section{TUnSortedStrCollection}
\label{se:TUnSortedStrCollection}

\begin{verbatim}
TYPE
   TUnSortedStrCollection = OBJECT (TStringCollection)
      PROCEDURE Insert (Item: Pointer); Virtual;
   END;
   PUnSortedStrCollection = ^TUnSortedStrCollection;
\end{verbatim}

\section{TResourceCollection}
\label{se:TResourceCollection}

\begin{verbatim}
TYPE
   TResourceCollection = OBJECT (TStringCollection)
      FUNCTION KeyOf (Item: Pointer): Pointer; Virtual;
      FUNCTION GetItem (Var S: TStream): Pointer; Virtual;
      PROCEDURE FreeItem (Item: Pointer); Virtual;
      PROCEDURE PutItem (Var S: TStream; Item: Pointer); Virtual;
   END;
   PResourceCollection = ^TResourceCollection;
\end{verbatim}

\section{TResourceFile}
\label{se:TResourceFile}

\begin{verbatim}
TYPE
   TResourceFile = OBJECT (TObject)
         Stream  : PStream; { File as a stream }
         Modified: Boolean; { Modified flag }
      CONSTRUCTOR Init (AStream: PStream);
      DESTRUCTOR Done; Virtual;
      FUNCTION Count: Sw_Integer;
      FUNCTION KeyAt (I: Sw_Integer): String;
      FUNCTION Get (Key: String): PObject;
      FUNCTION SwitchTo (AStream: PStream; Pack: Boolean): PStream;
      PROCEDURE Flush;
      PROCEDURE Delete (Key: String);
      PROCEDURE Put (Item: PObject; Key: String);
   END;
   PResourceFile = ^TResourceFile;
\end{verbatim}

\section{TStringList}
\label{se:TStringList}

\begin{verbatim}
TYPE
   TStrIndexRec = Packed RECORD
      Key, Count, Offset: Word;
   END;

   TStrIndex = Array [0..9999] Of TStrIndexRec;
   PStrIndex = ^TStrIndex;

   TStringList = OBJECT (TObject)
      CONSTRUCTOR Load (Var S: TStream);
      DESTRUCTOR Done; Virtual;
      FUNCTION Get (Key: Sw_Word): String;
   END;
   PStringList = ^TStringList;
\end{verbatim}

\section{TStrListMaker}
\label{se:TStrListMaker}

\begin{verbatim}
TYPE
   TStrListMaker = OBJECT (TObject)
      CONSTRUCTOR Init (AStrSize, AIndexSize: Sw_Word);
      DESTRUCTOR Done; Virtual;
      PROCEDURE Put (Key: Sw_Word; S: String);
      PROCEDURE Store (Var S: TStream);
   END;
   PStrListMaker = ^TStrListMaker;
\end{verbatim}

\begin{verbatim}
FUNCTION NewStr (Const S: String): PString;
PROCEDURE DisposeStr (P: PString);
PROCEDURE Abstract;
PROCEDURE RegisterObjects;
\end{verbatim}
\begin{procedure}{RegisterType}
\Declaration
PROCEDURE RegisterType (Var S: TStreamRec);
\end{procedure}
\begin{verbatim}
FUNCTION LongMul (X, Y: Integer): LongInt;
FUNCTION LongDiv (X: Longint; Y: Integer): Integer;

CONST
   StreamError: Pointer = Nil;                        { Stream error ptr }
   DosStreamError: Word = $0;                      { Dos stream error }

CONST
   RCollection: TStreamRec = (
     ObjType: 50;
     VmtLink: Ofs(TypeOf(TCollection)^);
     Load: @TCollection.Load;
     Store: @TCollection.Store);

   RStringCollection: TStreamRec = (
     ObjType: 51;
     VmtLink: Ofs(TypeOf(TStringCollection)^);
     Load: @TStringCollection.Load;
     Store: @TStringCollection.Store);

   RStrCollection: TStreamRec = (
     ObjType: 69;
     VmtLink: Ofs(TypeOf(TStrCollection)^);
     Load:    @TStrCollection.Load;
     Store:   @TStrCollection.Store);

   RStringList: TStreamRec = (
     ObjType: 52;
     VmtLink: Ofs(TypeOf(TStringList)^);
     Load: @TStringList.Load;
     Store: Nil);

   RStrListMaker: TStreamRec = (
     ObjType: 52;
     VmtLink: Ofs(TypeOf(TStrListMaker)^);
     Load: Nil;
     Store: @TStrListMaker.Store);
\end{verbatim}
% the ports unit
%
%   $Id$
%   This file is part of the FPC documentation.
%   Copyright (C) 1997, by Michael Van Canneyt
%
%   The FPC documentation is free text; you can redistribute it and/or
%   modify it under the terms of the GNU Library General Public License as
%   published by the Free Software Foundation; either version 2 of the
%   License, or (at your option) any later version.
%
%   The FPC Documentation is distributed in the hope that it will be useful,
%   but WITHOUT ANY WARRANTY; without even the implied warranty of
%   MERCHANTABILITY or FITNESS FOR A PARTICULAR PURPOSE.  See the GNU
%   Library General Public License for more details.
%
%   You should have received a copy of the GNU Library General Public
%   License along with the FPC documentation; see the file COPYING.LIB.  If not,
%   write to the Free Software Foundation, Inc., 59 Temple Place - Suite 330,
%   Boston, MA 02111-1307, USA.
%
\chapter{The PORTS unit}

\section{Introduction}
The ports unit implements the \var{port} constructs found in \tp. 
It uses classes and default array properties to do this.

The unit exists on \linux, \ostwo and \dos. It is implemented only for
compatibility with \tp. It's usage is discouraged, because using ports
is not portable programming, and the operating system may not even allow
it (for instance \windows).

Under \linux, your program must be run as root, or the \var{IOPerm} call
must be set in order to set appropriate permissions on the port access.

\section{Types,constants and variables}

\subsection{Types}
The following types are defined to implement the port access.
\begin{verbatim}
tport = class
  protected
    procedure writeport(p : longint;data : byte);
    function  readport(p : longint) : byte;
  public
    property pp[w : longint] : byte read readport write writeport;default;
end;

tportw = class
  protected
    procedure writeport(p : longint;data : word);
    function  readport(p : longint) : word;
  public
    property pp[w : longint] : word read readport write writeport;default;
end;

tportl = class
  Protected
    procedure writeport(p : longint;data : longint);
    function  readport(p : longint) : longint;
  Public
   property pp[w : Longint] : longint read readport write writeport;default;
end;
\end{verbatim}
Each of these types allows access to the ports using respectively, a byte, a
word or a longint sized argument.

Since there is a default property for each of this types, a sentence as
\begin{verbatim}
  port[221]:=12;
\end{verbatim}
Will result in the byte 12 being written to port 221, if port is defined
as a variable of type \var{tport}
\subsection{variables}       
The following variables are defined:
\begin{verbatim}
port,
portb : tport;
portw : tportw;
portl : tportl;
\end{verbatim}
They allow access to the ports in a \tp compatible way.

% the printer unit
%
%   $Id$
%   This file is part of the FPC documentation.
%   Copyright (C) 1997, by Michael Van Canneyt
%
%   The FPC documentation is free text; you can redistribute it and/or
%   modify it under the terms of the GNU Library General Public License as
%   published by the Free Software Foundation; either version 2 of the
%   License, or (at your option) any later version.
%
%   The FPC Documentation is distributed in the hope that it will be useful,
%   but WITHOUT ANY WARRANTY; without even the implied warranty of
%   MERCHANTABILITY or FITNESS FOR A PARTICULAR PURPOSE.  See the GNU
%   Library General Public License for more details.
%
%   You should have received a copy of the GNU Library General Public
%   License along with the FPC documentation; see the file COPYING.LIB.  If not,
%   write to the Free Software Foundation, Inc., 59 Temple Place - Suite 330,
%   Boston, MA 02111-1307, USA. 
%
\chapter{The PRINTER unit.}
This chapter describes the PRINTER unit for Free Pascal. It was written for
\dos by Florian Kl\"ampfl, and it was written for \linux by Micha\"el Van 
Canneyt, and has been ported to \windows and \ostwo as well. 
Its basic functionality is the same for al supported systems, although there 
are minor differences on \linux.

The chapter is divided in 2 sections:
\begin{itemize}
\item The first section lists types, constants and variables from the
interface part of the unit.
\item The second section describes the functions defined in the unit.
\end{itemize}
\section {Types, Constants and variables : }
\begin{verbatim}
var 
  lst : text;
\end{verbatim}
\var{Lst} is the standard printing device. \\ On \linux, 
\var{Lst} is set up using \var{AssignLst('/tmp/PID.lst')}. 
You can change this behaviour at compile time, setting the DefFile constant.
\section {Procedures and functions}
\begin{procedure}{AssignLst}
\Declaration
Procedure AssignLst  ( Var F : text; ToFile : string[255]);

\Description
 \linux only. \\
 Assigns to F a printing device. ToFile is a string with the following form:
\begin{itemize}
\item \var{ '|filename options'}  : This sets up a pipe with the program filename,
             with the given options, such as in the popen() call.
\item \var{ 'filename'} : Prints to file filename. Filename can contain the string 'PID'
              (No Quotes), which will be replaced by the PID of your program.
              When closing lst, the file will be sent to lpr and deleted.
              (lpr should be in PATH)
                
\item \var {'filename|'} Idem as previous, only the file is NOT sent to lpr, nor is it
             deleted.
             (useful for opening /dev/printer or for later printing)
\end{itemize}

\Errors
 Errors are reported in Linuxerror.
\SeeAlso
\seem{lpr}{1}
\end{procedure}
\latex{\lstinputlisting{printex/printex.pp}}
\html{\input{printex/printex.tex}}

% the sockets unit
%
%   $Id$
%   This file is part of the FPC documentation.
%   Copyright (C) 1997, by Michael Van Canneyt
%
%   The FPC documentation is free text; you can redistribute it and/or
%   modify it under the terms of the GNU Library General Public License as
%   published by the Free Software Foundation; either version 2 of the
%   License, or (at your option) any later version.
%
%   The FPC Documentation is distributed in the hope that it will be useful,
%   but WITHOUT ANY WARRANTY; without even the implied warranty of
%   MERCHANTABILITY or FITNESS FOR A PARTICULAR PURPOSE.  See the GNU
%   Library General Public License for more details.
%
%   You should have received a copy of the GNU Library General Public
%   License along with the FPC documentation; see the file COPYING.LIB.  If not,
%   write to the Free Software Foundation, Inc., 59 Temple Place - Suite 330,
%   Boston, MA 02111-1307, USA. 
%
\chapter{The SOCKETS unit.}
This chapter describes the SOCKETS unit for Free Pascal. 
it was written for \linux by Micha\"el Van Canneyt. 

The chapter is divided in 2 sections:
\begin{itemize}
\item The first section lists types, constants and variables from the
interface part of the unit.
\item The second section describes the functions defined in the unit.
\end{itemize}

\section {Types, Constants and variables : }
The following constants identify the different socket types, as needed in
the \seef{Socket} call.
\begin{verbatim}
SOCK_STREAM     = 1; { stream (connection) socket   }
SOCK_DGRAM      = 2; { datagram (conn.less) socket  }
SOCK_RAW        = 3; { raw socket                   }
SOCK_RDM        = 4; { reliably-delivered message   }
SOCK_SEQPACKET  = 5; { sequential packet socket     }
SOCK_PACKET     =10;
\end{verbatim}
The following constants determine the socket domain, they are used in the
\seef{Socket} call.
\begin{verbatim}
AF_UNSPEC       = 0;
AF_UNIX         = 1; { Unix domain sockets          }
AF_INET         = 2; { Internet IP Protocol         }
AF_AX25         = 3; { Amateur Radio AX.25          }
AF_IPX          = 4; { Novell IPX                   }
AF_APPLETALK    = 5; { Appletalk DDP                }
AF_NETROM       = 6; { Amateur radio NetROM         }
AF_BRIDGE       = 7; { Multiprotocol bridge         }
AF_AAL5         = 8; { Reserved for Werner's ATM    }
AF_X25          = 9; { Reserved for X.25 project    }
AF_INET6        = 10; { IP version 6                 }
AF_MAX          = 12;
\end{verbatim}
The following constants determine the protocol family, they are used in the
\seef{Socket} call.
\begin{verbatim} 
PF_UNSPEC       = AF_UNSPEC;
PF_UNIX         = AF_UNIX;
PF_INET         = AF_INET;
PF_AX25         = AF_AX25;
PF_IPX          = AF_IPX;
PF_APPLETALK    = AF_APPLETALK;
PF_NETROM       = AF_NETROM;
PF_BRIDGE       = AF_BRIDGE;
PF_AAL5         = AF_AAL5;
PF_X25          = AF_X25;
PF_INET6        = AF_INET6;
PF_MAX          = AF_MAX;   
\end{verbatim}
The following types are used to store different kinds of eddresses for the
\seef{Bind}, \seef{Recv} and \seef{Send} calls.
\begin{verbatim}  
TSockAddr = packed Record
  family:word;
  data  :array [0..13] of char;
  end;

TUnixSockAddr = packed Record
  family:word;
  path:array[0..108] of char;
  end;

TInetSockAddr = packed Record
  family:Word;
  port  :Word;
  addr  :Cardinal;
  pad   :array [1..8] of byte; 
  end;
\end{verbatim}
The following type is returned by the \seef{SocketPair} call.
\begin{verbatim}
TSockArray = Array[1..2] of Longint;
\end{verbatim}

\section {Functions and Procedures}

\function{Accept}{(Sock:Longint;Var Addr;Var Addrlen:Longint)}{Longint}
{\var{Accept} accepts a connection from a socket \var{Sock}, which was
listening for a connection. The accepted socket may NOT be used to accept
more connections. The original socket remains open.

The \var{Accept} call fills the address of the connecting entity in \var{Addr},
and sets its length in \var{Addrlen}. \var{Addr} should be pointing to
enough space, and \var{Addrlen} should be set to the amount of space
available, prior to the call.
}
{Errors are reported in \var{SocketError}, and include the following:
\begin{description}
\item[SYS\_EBADF] The socket descriptor is invalid.
\item[SYS\_ENOTSOCK] The descriptor is not a socket.
\item[SYS\_EOPNOTSUPP] The socket type doesn't support the \var{Listen}
operation.
\item[SYS\_EFAULT] \var{Addr} points outside your address space.
\item[SYS\_EWOULDBLOCK] The requested operation would block the process.
\end{description}
}
{\seef{Listen}, \seef{Connect}}

\input{sockex/sock_svr.tex}

\functionl{Accept}{AltAAccept}{(Sock:longint;var addr:string;var SockIn,SockOut:text)}{Boolean}
{ This is an alternate form of the \seef{Accept} command. It is equivalent
to subsequently calling the regular \seef{Accept}
function and the \seep{Sock2Text} function.

The function returns \var{True} if successfull, \var{False} otherwise.
}
{The errors are those of \seef{Accept}.}
{\seef{Accept}}

\functionl{Accept}{AltBAccept}{(Sock:longint;var addr:string;var SockIn,SockOut:File)}{Boolean}
{ This is an alternate form of the \seef{Accept} command. 
It is equivalent
to subsequently calling the regular \seef{Accept} function and the 
\seep{Sock2File} function.
The \var{Addr} parameter contains the name of the unix socket file to be
opened. 

The function returns \var{True} if successfull, \var{False} otherwise.
}
{The errors are those of \seef{Accept}.}
{\seef{Accept}}


\functionl{Accept}{AltCAccept}{(Sock:longint;var addr:TInetSockAddr;var SockIn,SockOut:File)}{Boolean}
{ This is an alternate form of the \seef{Accept} command. 
It is equivalent
to subsequently calling the regular \seef{Accept} function and the 
\seep{Sock2File} function.
The \var{Addr} parameter contains the parameters of the internet socket that
should be opened.

The function returns \var{True} if successfull, \var{False} otherwise.
}
{The errors are those of \seef{Accept}.}
{\seef{Accept}}

\function{Bind}{(Sock:Longint;Var Addr;AddrLen:Longint)}{Boolean}
{\var{Bind} binds the socket \var{Sock} to address \var{Addr}. \var{Addr}
has length \var{Addrlen}.

The function returns \var{True} if the call was succesful, \var{False} if
not.
}
{Errors are returned in \var{SocketError} and include the following:
\begin{description}
\item[SYS\_EBADF] The socket descriptor is invalid.
\item[SYS\_EINVAL] The socket is already bound to an address,
\item[SYS\_EACCESS] Address is protected and you don't have permission to
open it.
\end{description}
More arrors can be found in the Unix man pages.
}{\seef{Socket}}

\functionl{Bind}{AltBind}{(Sock:longint;const addr:string)}{boolean}
{This is an alternate form of the \var{Bind} command.
This form of the \var{Bind} command is equivalent to subsequently 
calling \seep{Str2UnixSockAddr} and the regular \seef{Bind} function.

The function returns \var{True} if successfull, \var{False} otherwise.
}
{Errors are those of the regular \seef{Bind} command.}
{\seef{Bind}}

\function{Connect}{(Sock:Longint;Var Addr;Addrlen:Longint)}{Boolean}
{\var{Connect} opens a connection to a peer, whose address is described by
var{Addr}. \var{AddrLen} contains the length of the address.

The type of \var{Addr} depends on the kind of connection you're trying to
make, but is generally one of \var{TSockAddr} or \var{TUnixSockAddr}.
}
{Errors are reported in \var{SocketError}.}
{\seef{Listen}, \seef{Bind},\seef{Accept}}

\input{sockex/sock_cli.tex}

\functionl{Connect}{AltAConnect}{(Sock:longint;const addr:string;var SockIn,SockOut:text)}{Boolean}
{ This is an alternate form of the \seef{Connect} command. 
It is equivalent
to subsequently calling the regular \seef{Connect} function and the 
\seep{Sock2Text} function.

The function returns \var{True} if successfull, \var{False} otherwise.
}{The errors are those of \seef{Connect}.}
{\seef{Connect}}

\functionl{Connect}{AltBConnect}{(Sock:longint;const addr:string;var SockIn,SockOut:file)}{Boolean}
{ This is an alternate form of the \seef{Connect} command. The parameter
\var{addr} contains the name of the unix socket file to be opened. 
It is equivalent to subsequently calling the regular \seef{Connect} 
function and the  \seep{Sock2File} function.

The function returns \var{True} if successfull, \var{False} otherwise.
}{The errors are those of \seef{Connect}.}
{\seef{Connect}}


\functionl{Connect}{AltCConnect}{(Sock:longint;const addr: TInetSockAddr;var SockIn,SockOut:file)}{Boolean}
{ This is another alternate form of the \seef{Connect} command. 
It is equivalent
to subsequently calling the regular \seef{Connect} function and the 
\seep{Sock2File} function. The \var{Addr} parameter contains the parameters
of the internet socket to connect to.

The function returns \var{True} if successfull, \var{False} otherwise.
}{The errors are those of \seef{Connect}.}
{\seef{Connect}}

\input{sockex/pfinger.tex}

\function{GetPeerName}{(Sock:Longint;Var Addr;Var Addrlen:Longint)}{Longint}
{\var{GetPeerName} returns the name of the entity connected to the 
specified socket \var{Sock}. The Socket must be connected for this call to
work. 
\var{Addr} should point to enough space to store the name, the
amount of space pointed to should be set in \var{Addrlen}. 
When the function returns succesfully, \var{Addr} will be filled with the 
name, and \var{Addrlen} will be set to the length of \var{Addr}.
}
{Errors are reported in \var{SocketError}, and include the following:
\begin{description}
\item[SYS\_EBADF] The socket descriptor is invalid.
\item[SYS\_ENOBUFS] The system doesn't have enough buffers to perform the
operation.
\item[SYS\_ENOTSOCK] The descriptor is not a socket.
\item[SYS\_EFAULT] \var{Addr} points outside your address space.
\item[SYS\_ENOTCONN] The socket isn't connected.
\end{description}
}{\seef{Connect}, \seef{Socket}, \seem{connect}{2}}

\function{GetSocketName}{(Sock:Longint;Var Addr;Var Addrlen:Longint)}{Longint}
{\var{GetSockName} returns the current name of the specified socket
\var{Sock}. \var{Addr} should point to enough space to store the name, the
amount of space pointed to should be set in \var{Addrlen}. 
When the function returns succesfully, \var{Addr} will be filled with the 
name, and \var{Addrlen} will be set to the length of \var{Addr}.}
{Errors are reported in \var{SocketError}, and include the following:
\begin{description}
\item[SYS\_EBADF] The socket descriptor is invalid.
\item[SYS\_ENOBUFS] The system doesn't have enough buffers to perform the
operation.
\item[SYS\_ENOTSOCK] The descriptor is not a socket.
\item[SYS\_EFAULT] \var{Addr} points outside your address space.
\end{description}
}{\seef{Bind}}


\function{GetSocketOptions}{(Sock,Level,OptName:Longint;Var OptVal;optlen:longint)}{Longint}
{\var{GetSocketOptions} gets the connection options for socket \var{Sock}.
The socket may be obtained from different levels, indicated by \var{Level},
which can be one of the following:
\begin{description}
\item[SOL\_SOCKET] From the socket itself. 
\item[XXX] set \var{Level} to \var{XXX}, the protocol number of the protocol
which should interprete the option.
 \end{description}
For more information on this call, refer to the unix manual page \seem{getsockopt}{2}.
}
{Errors are reported in \var{SocketError}, and include the following:
\begin{description}
\item[SYS\_EBADF] The socket descriptor is invalid.
\item[SYS\_ENOTSOCK] The descriptor is not a socket.
\item[SYS\_EFAULT] \var{OptVal} points outside your address space.
\end{description}
}
{\seef{GetSocketOptions}}

\function{Listen}{(Sock,MaxConnect:Longint)}{Boolean}
{\var{Listen} listens for up to \var{MaxConnect} connections from socket
\var{Sock}. The socket \var{Sock} must be of type \var{SOCK\_STREAM} or
\var{Sock\_SEQPACKET}.

The function returns \var{True} if a connection was accepted, \var{False} 
if an error occurred.
}
{Errors are reported in \var{SocketError}, and include the following:
\begin{description}
\item[SYS\_EBADF] The socket descriptor is invalid.
\item[SYS\_ENOTSOCK] The descriptor is not a socket.
\item[SYS\_EOPNOTSUPP] The socket type doesn't support the \var{Listen}
operation.
\end{description}
}{\seef{Socket}, \seef{Bind}, \seef{Connect}}

\function{Recv}{(Sock:Longint;Var Addr;AddrLen,Flags:Longint)}{Longint}
{\var{Recv} reads at most \var{Addrlen} bytes from socket \var{Sock} into
address \var{Addr}. The socket must be in a connected state.

\var{Flags} can be one of the following:
\begin{description}
\item [1] : Process out-of band data.
\item [4] : Bypass routing, use a direct interface.
\item [??] : Wait for full request or report an error.
\end{description}

The functions returns the number of bytes actually read from the socket, or
-1 if a detectable error occurred.}
{Errors are reported in \var{SocketError}, and include the following:
\begin{description}
\item[SYS\_EBADF] The socket descriptor is invalid.
\item[SYS\_ENOTCONN] The socket isn't connected.
\item[SYS\_ENOTSOCK] The descriptor is not a socket.
\item[SYS\_EFAULT] The address is outside your address space.
\item[SYS\_EMSGSIZE] The message cannot be sent atomically.
\item[SYS\_EWOULDBLOCK] The requested operation would block the process.
\item[SYS\_ENOBUFS] The system doesn't have enough free buffers available.
\end{description}
}{\seef{Send}}

\function{Send}{(Sock:Longint;Var Addr;AddrLen,Flags:Longint)}{Longint}
{\var{Send} sends \var{AddrLen} bytes starting from address \var{Addr}
to socket \var{Sock}. \var{Sock} must be in a connected state.

The function returns the number of bytes sent, or -1 if a detectable 
error occurred.

\var{Flags} can be one of the following:
\begin{description}
\item [1] : Process out-of band data.
\item [4] : Bypass routing, use a direct interface.
\end{description}
}
{Errors are reported in \var{SocketError}, and include the following:
\begin{description}
\item[SYS\_EBADF] The socket descriptor is invalid.
\item[SYS\_ENOTSOCK] The descriptor is not a socket.
\item[SYS\_EFAULT] The address is outside your address space.
\item[SYS\_EMSGSIZE] The message cannot be sent atomically.
\item[SYS\_EWOULDBLOCK] The requested operation would block the process.
\item[SYS\_ENOBUFS] The system doesn't have enough free buffers available.
\end{description}
}{\seef{Recv}, \seem{send}{2}}

\function{SetSocketOptions}{(Sock,Level,OptName:Longint;Var OptVal;optlen:longint)}{Longint}
{\var{SetSocketOptions} sets the connection options for socket \var{Sock}.
The socket may be manipulated at different levels, indicated by \var{Level},
which can be one of the following:
\begin{description}
\item[SOL\_SOCKET] To manipulate the socket itself. 
\item[XXX] set \var{Level} to \var{XXX}, the protocol number of the protocol
which should interprete the option.
 \end{description}
For more information on this call, refer to the unix manual page \seem{setsockopt}{2}.
}
{Errors are reported in \var{SocketError}, and include the following:
\begin{description}
\item[SYS\_EBADF] The socket descriptor is invalid.
\item[SYS\_ENOTSOCK] The descriptor is not a socket.
\item[SYS\_EFAULT] \var{OptVal} points outside your address space.
\end{description}
}
{\seef{GetSocketOptions}}

\function{Shutdown}{(Sock:Longint;How:Longint)}{Longint}
{\var{ShutDown} closes one end of a full duplex socket connection, described
by \var{Sock}. \var{How} determines how the connection will be shut down,
and can be one of the following:
\begin{description}
\item[0] : Further receives are disallowed.
\item[1] : Further sends are disallowed.
\item[2] : Sending nor receiving are allowed.
\end{description}

On succes, the function returns 0, on error -1 is returned.
}
{\var{SocketError} is used to report errors, and includes the following:
\begin{description}
\item[SYS\_EBADF] The socket descriptor is invalid.
\item[SYS\_ENOTCONN] The socket isn't connected.
\item[SYS\_ENOTSOCK] The descriptor is not a socket.
\end{description}
}{\seef{Socket}, \seef{Connect}}

\procedure{Sock2File}{(Sock:Longint;Var SockIn,SockOut:File)}
{\var{Sock2File} transforms a socket \var{Sock} into 2 Pascal file
descriptors of type \var{File}, one for reading from the socket
(\var{SockIn}), one for writing to the socket (\var{SockOut}).}
{None.}
{\seef{Socket}, \seep{Sock2Text}}

\procedure{Sock2Text}{(Sock:Longint;Var SockIn,SockOut: Text)}
{\var{Sock2Text} transforms a socket \var{Sock} into 2 Pascal file
descriptors of type \var{Text}, one for reading from the socket
(\var{SockIn}), one for writing to the socket (\var{SockOut}).}
{None.}
{\seef{Socket}, \seep{Sock2File}}



\function{Socket}{(Domain,SocketType,Protocol:Longint)}{Longint}
{\var{Socket} creates a new socket in domain \var{Domain}, from type
\var{SocketType} using protocol \var{Protocol}.

The Domain, Socket type and Protocol can be specified using predefined
constants (see the section on constants for available constants)

If succesfull, the function returns a socket descriptor, which can be passed
to a subsequent \seef{Bind} call. If unsuccesfull, the function returns -1.
}
{Errors are returned in \var{SocketError}, and include the follwing:
\begin{description}
\item[SYS\_EPROTONOSUPPORT]
The protocol type or the specified protocol is not
supported within this domain.
\item[SYS\_EMFILE]
The per-process descriptor table is full.
\item[SYS\_ENFILE]
The system file table is full.
\item[SYS\_EACCESS]
 Permission  to  create  a  socket of the specified
 type and/or protocol is denied.
\item[SYS\_ENOBUFS]
 Insufficient  buffer  space  is  available.    The
 socket   cannot   be   created   until  sufficient
 resources are freed.
\end{description}}
{\seef{SocketPair}, \seem{socket}{2}}
for an example, see \seef{Accept}.

\function{SocketPair}{(Domain,SocketType,Protocol:Longint;var Pair:TSockArray)}{Longint}
{\var{SocketPair} creates 2 sockets in domain \var{Domain}, from type
\var{SocketType} and using protocol \var{Protocol}.

The pair is returned in \var{Pair}, and they are indistinguishable.

The function returns -1 upon error and 0 upon success.
}
{Errors are reported in \var{SocketError}, and are the same as in \seef{Socket}}

\procedure{Str2UnixSockAddr}{(const addr:string;var t:TUnixSockAddr;var len:longint)}
{\var{Str2UnixSockAddr} transforms a Unix socket address in a string to a
\var{TUnixSockAddr} sturcture which can be passed to the \seef{Bind} call.
}
{None.}
{\seef{Socket}, \seef{Bind}}

% the strings unit
%
%   $Id$
%   This file is part of the FPC documentation.
%   Copyright (C) 1997, by Michael Van Canneyt
%
%   The FPC documentation is free text; you can redistribute it and/or
%   modify it under the terms of the GNU Library General Public License as
%   published by the Free Software Foundation; either version 2 of the
%   License, or (at your option) any later version.
%
%   The FPC Documentation is distributed in the hope that it will be useful,
%   but WITHOUT ANY WARRANTY; without even the implied warranty of
%   MERCHANTABILITY or FITNESS FOR A PARTICULAR PURPOSE.  See the GNU
%   Library General Public License for more details.
%
%   You should have received a copy of the GNU Library General Public
%   License along with the FPC documentation; see the file COPYING.LIB.  If not,
%   write to the Free Software Foundation, Inc., 59 Temple Place - Suite 330,
%   Boston, MA 02111-1307, USA. 
%
\chapter{The STRINGS unit.}
This chapter describes the \var{STRINGS} unit for 
\fpk. 

Since the unit only provides some procedures and functions, there is
only one section, which gives the declarations of these functions, together
with an explanation. 

\section{Functions and procedures.}

\function{StrLen}{(p : PChar)}{Longint}
{
Returns the length of the null-terminated string \var{P}.
}
{None.}{\seem{Length}{}}

\input{stringex/ex1.tex}

\function{StrPCopy}{(Dest : PChar; Const Source : String)}{PChar}
{
Converts the Pascal string in \var{Source} to a Null-terminated 
string, and copies it to \var{Dest}. \var{Dest} needs enough room to contain
the string \var{Source}, i.e. \var{Length(Source)+1} bytes.
}
{No length checking is performed.}{ \seef{StrPas}}

\input{stringex/ex2.tex}

\function {StrPas}{(P : PChar)}{String}
{
Converts a null terminated string in \var{P} to a Pascal string, and returns
this string. The string is truncated at 255 characters.
}
{None.}{ \seef{StrPCopy}}

\input{stringex/ex3.tex}

\function {StrCopy}{(Dest,Source : PChar)}{PChar}
{ 
Copy the null terminated string in \var{Source} to \var{Dest}, and
returns a pointer to \var{Dest}. \var{Dest} needs enough room to contain
\var{Source}, i.e. \var{StrLen(Source)+1} bytes.
}
{No length checking is performed.}{ \seef{StrPCopy}, \seef{StrLCopy}, \seef{StrECopy}}

\input{stringex/ex4.tex}

\function{StrLCopy}{(Dest,Source : PChar; MaxLen : Longint)}{PChar}
{
Copies \var{MaxLen} characters from \var{Source} to \var{Dest}, and makes
\var{Dest} a null terminated string. 
}
{No length checking is performed.}
{\seef{StrCopy}, \seef{StrECopy}}
 
\input{stringex/ex5.tex}

\function{StrECopy}{(Dest,Source : PChar)}{PChar}
{
Copies the Null-terminated string in \var{Source} to \var{Dest}, and
returns a pointer to the end (i.e. the terminating Null-character) of the
copied string.
}
{No length checking is performed.}
{\seef{StrLCopy}, \seef{StrCopy}}

\input{stringex/ex6.tex}

\function{StrEnd}{(P : PChar)}{PChar}
{
Returns a pointer to the end of \var{P}. (i.e. to the terminating
null-character.
}
{None.}{\seef{StrLen}}

\input{stringex/ex7.tex}

\function{StrCat}{(Dest,Source : PChar)}{PChar}
{
Attaches \var{Source} to \var{Dest} and returns \var{Dest}.
}
{No length checking is performed.}
{\seem{Concat}{}}

\input{stringex/ex11.tex}

\function{StrComp}{(S1,S2 : PChar)}{Longint}
{
Compares the null-terminated strings \var{S1} and \var{S2}.

The result is 
\begin{itemize}
\item A negative \var{Longint} when \var{S1<S2}.
\item 0 when \var{S1=S2}.
\item A positive \var{Longint} when \var{S1>S2}.
\end{itemize}
}
{None.}{\seef{StrLComp}, \seef{StrIComp}, \seef{StrLIComp}}

For an example, see \seef{StrLComp}.

\function{StrLComp}{(S1,S2 : PChar; L : Longint)}{Longint}
{
Compares maximum \var{L} characters of the null-terminated strings 
\var{S1} and \var{S2}. 

The result is 
\begin{itemize}
\item A negative \var{Longint} when \var{S1<S2}.
\item 0 when \var{S1=S2}.
\item A positive \var{Longint} when \var{S1>S2}.
\end{itemize}
}
{None.}{\seef{StrComp}, \seef{StrIComp}, \seef{StrLIComp}}

\input{stringex/ex8.tex}

\function{StrIComp}{(S1,S2 : PChar)}{Longint}
{
Compares the null-terminated strings \var{S1} and \var{S2}, ignoring case.

The result is 
\begin{itemize}
\item A negative \var{Longint} when \var{S1<S2}.
\item 0 when \var{S1=S2}.
\item A positive \var{Longint} when \var{S1>S2}.
\end{itemize}
}
{None.}{\seef{StrLComp}, \seef{StrComp}, \seef{StrLIComp}}

\input{stringex/ex8.tex}

\function{StrLIComp}{(S1,S2 : PChar; L : Longint)}{Longint}
{
Compares maximum \var{L} characters of the null-terminated strings \var{S1} 
and \var{S2}, ignoring case.

The result is 
\begin{itemize}
\item A negative \var{Longint} when \var{S1<S2}.
\item 0 when \var{S1=S2}.
\item A positive \var{Longint} when \var{S1>S2}.
\end{itemize}
}
{None.}{\seef{StrLComp}, \seef{StrComp}, \seef{StrIComp}}

For an example, see \seef{StrIComp}

\function{StrMove}{(Dest,Source : PChar; MaxLen : Longint)}{PChar}
{
Copies \var{MaxLen} characters from \var{Source} to \var{Dest}. No
terminating null-character is copied.
Returns \var {Dest}.
}
{None.}{\seef{StrLCopy}, \seef{StrCopy}}

\input{stringex/ex10.tex}

\function{StrLCat}{(Dest,Source : PChar; MaxLen : Longint)}{PChar}
{
Adds \var{MaxLen} characters from \var{Source} to \var{Dest}, and adds a
terminating null-character. Returns \var{Dest}.
}
{None.}{\seef{StrCat}}

\input{stringex/ex12.tex}

\function{StrScan}{(P : PChar; C : Char)}{PChar}
{
Returns a pointer to the first occurrence of the character \var{C} in the
null-terminated string \var{P}. If \var{C} does not occur, returns
\var{Nil}.
}
{None.}{\seem{Pos}{}, \seef{StrRScan}, \seef{StrPos}}

\input{stringex/ex13.tex}

\function{StrRScan}{(P : PChar; C : Char)}{PChar}
{
Returns a pointer to the last occurrence of the character \var{C} in the
null-terminated string \var{P}. If \var{C} does not occur, returns
\var{Nil}.
}
{None.}{\seem{Pos}{}, \seef{StrScan}, \seef{StrPos}}

For an example, see \seef{StrScan}.

\function{StrLower}{(P : PChar)}{PChar}
{
Converts \var{P} to an all-lowercase string. Returns \var{P}.
}
{None.}{\seem{Upcase}{}, \seef{StrUpper}}

\input{stringex/ex14.tex}

\function{StrUpper}{(P : PChar)}{PChar}
{
Converts \var{P} to an all-uppercase string. Returns \var{P}.
}
{None.}{\seem{Upcase}{}, \seef{StrLower}}

For an example, see \seef{StrLower}

\function{StrPos}{(S1,S2 : PChar)}{PChar}
{
Returns a pointer to the first occurrence of \var{S2} in \var{S1}.
If \var{S2} does not occur in \var{S1}, returns \var{Nil}.
}
{None.}{\seem{Pos}{}, \seef{StrScan}, \seef{StrRScan}}

\input{stringex/ex15.tex}

\function{StrNew}{(P : PChar)}{PChar}
{
Copies \var{P} to the Heap, and returns a pointer to the copy.
}
{Returns \var{Nil} if no memory was available for the copy.}
{\seem{New}{}, \seef{StrCopy}, \seep{StrDispose}}

\input{stringex/ex16.tex}

\procedure{StrDispose}{(P : PChar)}
{
Removes the string in \var{P} from the heap and releases the memory.
}
{None.}{\seem{Dispose}{}, \seef{StrNew}}

\input{stringex/ex17.tex}

\procedure{StrAlloc}{(Len : Longint)}{PChar}
{
\var{StrAlloc} reserves memory on the heap for a string with length \var{Len},
terminating \var{\#0} included, and returns a pointer to it.
}

For an example, see \seef{StrPCopy}.

% the sysutils unit
%
%   $Id$
%   This file is part of the FPC documentation.
%   Copyright (C) 1999, by Michael Van Canneyt
%
%   The FPC documentation is free text; you can redistribute it and/or
%   modify it under the terms of the GNU Library General Public License as
%   published by the Free Software Foundation; either version 2 of the
%   License, or (at your option) any later version.
%
%   The FPC Documentation is distributed in the hope that it will be useful,
%   but WITHOUT ANY WARRANTY; without even the implied warranty of
%   MERCHANTABILITY or FITNESS FOR A PARTICULAR PURPOSE.  See the GNU
%   Library General Public License for more details.
%
%   You should have received a copy of the GNU Library General Public
%   License along with the FPC documentation; see the file COPYING.LIB.  If not,
%   write to the Free Software Foundation, Inc., 59 Temple Place - Suite 330,
%   Boston, MA 02111-1307, USA.
%
\chapter{The SYSUTILS unit.}
\FPCexampledir{sysutex}

This chapter describes the \file{sysutils} unit. The \file{sysutils} unit
was largely written by Gertjan Schouten, and completed by Micha\"el Van Canneyt.
It aims to be compatible to the Delphi \file{sysutils} unit, but in contrast
with  the latter, it is designed to work on multiple platforms. It is implemented
on all supported platforms.

This chapter starts out with a definition of all types and constants
that are defined, followed by an overview of functions grouped by
functionality, and lastly the complete explanation of each function.

\section{Constants and types}
The following general-purpose types are defined:
\begin{verbatim}
tfilename = string;

tsyscharset = set of char;
tintegerset = set of 0..sizeof(integer)*8-1;

longrec = packed record
   lo,hi : word;
end;

wordrec = packed record
   lo,hi : byte;
end;

TMethod = packed record
  Code, Data: Pointer;
end;
\end{verbatim}
The use and meaning of these types should be clear, so no extra information
will be provided here.

The following general-purpose constants are defined:
\begin{verbatim}
const
   SecsPerDay = 24 * 60 * 60; // Seconds and milliseconds per day
   MSecsPerDay = SecsPerDay * 1000;
   DateDelta = 693594;        // Days between 1/1/0001 and 12/31/1899
   Eoln = #10;
\end{verbatim}
The following types are used frequently in date and time functions.
They are the same on all platforms.
\begin{verbatim}
type
   TSystemTime = record
      Year, Month, Day: word;
      Hour, Minute, Second, MilliSecond: word;
   end ;

   TDateTime = double;

   TTimeStamp = record
      Time: integer;   { Number of milliseconds since midnight }
      Date: integer;   { One plus number of days since 1/1/0001 }
   end ;
\end{verbatim}
The following type is used in the \seef{FindFirst},\seef{FindNext}
and \seepl{FindClose}{FindCloseSys} functions. The \var{win32} version differs from
the other versions. If code is to be portable, that part  shouldn't
be used.
\begin{verbatim}
Type
  THandle = Longint;
  TSearchRec = Record
    Time,Size, Attr : Longint;
    Name : TFileName;
    ExcludeAttr : Longint;
    FindHandle : THandle;
    {$ifdef Win32}
    FindData : TWin32FindData;
    {$endif}
    end;
\end{verbatim}
The following constants are file-attributes that need to be matched in the
findfirst call.
\begin{verbatim}
Const
  faReadOnly  = $00000001;
  faHidden    = $00000002;
  faSysFile   = $00000004;
  faVolumeId  = $00000008;
  faDirectory = $00000010;
  faArchive   = $00000020;
  faAnyFile   = $0000003f;
\end{verbatim}
The following constants can be used in the \seef{FileOpen} call.
\begin{verbatim}
Const
  fmOpenRead       = $0000;
  fmOpenWrite      = $0001;
  fmOpenReadWrite  = $0002;
\end{verbatim}
The following constants can be used in the \seef{FileSeek} call.
\begin{verbatim}
Const
  fsFromBeginning = 0;
  fsFromCurrent   = 1;
  fsFromEnd       = 2;

\end{verbatim}
The following variables are used in the case translation routines.
\begin{verbatim}
type
   TCaseTranslationTable = array[0..255] of char;
var
   UpperCaseTable: TCaseTranslationTable;
   LowerCaseTable: TCaseTranslationTable;
\end{verbatim}
The initialization code of the \file{sysutils} unit fills these
tables with the appropriate values. For the win32 and go32v2
versions, this information is obtained from the operating system.

The following constants control the formatting of dates.
For the Win32 version of the \file{sysutils} unit, these
constants are set according to the internationalization
settings of Windows by the initialization code of the unit.
\begin{verbatim}
Const
   DateSeparator: char = '-';
   ShortDateFormat: string = 'd/m/y';
   LongDateFormat: string = 'dd" "mmmm" "yyyy';
   ShortMonthNames: array[1..12] of string[128] =
     ('Jan','Feb','Mar','Apr','May','Jun',
      'Jul','Aug','Sep','Oct','Nov','Dec');
   LongMonthNames: array[1..12] of string[128] =
     ('January','February','March','April',
      'May','June','July','August',
      'September','October','November','December');
   ShortDayNames: array[1..7] of string[128] =
     ('Sun','Mon','Tue','Wed','Thu','Fri','Sat');
   LongDayNames: array[1..7] of string[128] =
     ('Sunday','Monday','Tuesday','Wednesday',
       'Thursday','Friday','Saturday');
\end{verbatim}

The following constants control the formatting of times.
For the Win32 version of the \file{sysutils} unit, these
constants are set according to the internationalization
settings of Windows by the initialization code of the unit.
\begin{verbatim}
Const
   ShortTimeFormat: string = 'hh:nn';
   LongTimeFormat: string = 'hh:nn:ss';
   TimeSeparator: char = ':';
   TimeAMString: string[7] = 'AM';
   TimePMString: string[7] = 'PM';
\end{verbatim}

The following constants control the formatting of currencies
and numbers. For the Win32 version of the \file{sysutils} unit,
these  constants are set according to the internationalization
settings of Windows by the initialization code of the unit.
\begin{verbatim}
Const
  DecimalSeparator : Char = '.';
  ThousandSeparator : Char = ',';
  CurrencyDecimals : Byte = 2;
  CurrencyString : String[7] = '$';
  { Format to use when formatting currency :
    0 = $1        1 = 1$         2 = $ 1      3 = 1 $
    4 = Currency string replaces decimal indicator.
        e.g. 1$50
   }
  CurrencyFormat : Byte = 1;
  { Same as above, only for negative currencies:
    0 = ($1)
    1 = -$1
    2 = $-1
    3 = $1-
    4 = (1$)
    5 = -1$
    6 = 1-$
    7 = 1$-
    8 = -1 $
    9 = -$ 1
    10 = $ 1-
   }
  NegCurrFormat : Byte = 5;
\end{verbatim}
The following types are used in various string functions.
\begin{verbatim}
type
   PString = ^String;
   TFloatFormat = (ffGeneral, ffExponent, ffFixed, ffNumber, ffCurrency);
\end{verbatim}
The following constants are used in the file name handling routines. Do not
use a slash of backslash character directly as a path separator; instead
use the \var{OsDirSeparator} character.
\begin{verbatim}
Const
  DirSeparators : set of char = ['/','\'];
{$ifdef unix}
  OSDirSeparator = '/';
{$else}
  OsDirSeparator = '\';
{$endif}
\end{verbatim}

%%%%%%%%%%%%%%%%%%%%%%%%%%%%%%%%%%%%%%%%%%%%%%%%%%%%%%%%%%%%%%%%%%%%%%%
% Functions and procedures by category
\section{Function list by category}
What follows is a listing of the available functions, grouped by category.
For each function there is a reference to the page where you can find the
function.

\subsection{String functions}
Functions for handling strings.
\begin{funclist}
\funcref{AnsiCompareStr}{Compare two strings}
\funcref{AnsiCompareText}{Compare two strings, case insensitive}
\funcref{AnsiExtractQuotedStr}{Removes quotes from string}
\funcref{AnsiLastChar}{Get last character of string}
\funcref{AnsiLowerCase}{Convert string to all-lowercase}
\funcref{AnsiQuotedStr}{Qoutes a string}
\funcref{AnsiStrComp}{Compare strings case-sensitive}
\funcref{AnsiStrIComp}{Compare strings case-insensitive}
\funcref{AnsiStrLComp}{Compare L characters of strings case sensitive}
\funcref{AnsiStrLIComp}{Compare L characters of strings case insensitive}
\funcref{AnsiStrLastChar}{Get last character of string}
\funcref{AnsiStrLower}{Convert string to all-lowercase}
\funcref{AnsiStrUpper}{Convert string to all-uppercase}
\funcref{AnsiUpperCase}{Convert string to all-uppercase}
\procref{AppendStr}{Append 2 strings}
\procref{AssignStr}{Assign value of strings on heap}
\funcref{CompareStr}{Compare two strings case sensitive}
\funcref{CompareText}{Compare two strings case insensitive}
\procrefl{DisposeStr}{DisposeStrSys}{Remove string from heap}
\funcref{IsValidIdent}{Is string a valid pascal identifier}
\funcref{LastDelimiter}{Last occurance of character in a string}
\funcref{LeftStr}{Get first N characters of a string}
\funcref{LoadStr}{Load string from resources}
\funcref{LowerCase}{Convert string to all-lowercase}
\funcrefl{NewStr}{NewStrSys}{Allocate new string on heap}
\funcref{RightStr}{Get last N characters of a string}
\funcrefl{StrAlloc}{StrAllocSys}{Allocate memory for string}
\funcref{StrBufSize}{Reserve memory for a string}
\procrefl{StrDispose}{StrDisposeSys}{Remove string from heap}
\funcrefl{StrPas}{StrPasSys}{Convert PChar to pascal string}
\funcrefl{StrPCopy}{StrPCopySys}{Copy pascal string}
\funcrefl{StrPLCopy}{StrPLCopySys}{Copy N bytes of pascal string}
\funcref{UpperCase}{Convert string to all-uppercase}
\end{funclist}

\subsection{Formatting strings}
Functions for formatting strings.
\begin{funclist}
\funcref{AdjustLineBreaks}{Convert line breaks to line breaks for system}
\funcref{FormatBuf}{Format a buffer}
\funcref{Format}{Format arguments in string}
\procref{FmtStr}{Format buffer}
\funcref{QuotedStr}{Quote a string}
\funcref{StrFmt}{Format arguments in a string}
\funcref{StrLFmt}{Format maximum L characters in a string}
\funcref{TrimLeft}{Remove whitespace at the left of a string}
\funcref{TrimRight}{Remove whitespace at the right of a string}
\funcref{Trim}{Remove whitespace at both ends of a string}
\end{funclist}

\subsection{File input/output routines}
Functions for reading/writing to file.
\begin{funclist}
\funcref{FileCreate}{Create a file and return handle}
\funcref{FileOpen}{Open file end return handle}
\funcref{FileRead}{Read from file}
\funcref{FileSeek}{Set file position}
\funcref{FileTruncate}{Truncate file length}
\funcref{FileWrite}{Write to file}
\procref{FileClose}{Close file handle}
\end{funclist}

\subsection{File handling routines}
Functions for file manipulation.
\begin{funclist}
\funcref{AddDisk}{Add sisk to list of disk drives}
\funcref{ChangeFileExt}{Change extension of file name}
\funcref{CreateDir}{Create a directory}
\funcref{DeleteFile}{Delete a file}
\funcrefl{DiskFree}{DiskFreeSys}{Free space on disk}
\funcrefl{DiskSize}{DiskSizeSys}{Total size of disk}
\funcref{ExpandFileName}{Create full file name}
\funcref{ExpandUNCFileName}{Create full UNC file name}
\funcref{ExtractFileDir}{Extract directory part of filename}
\funcref{ExtractFileDrive}{Extract drive part of filename}
\funcref{ExtractFileExt}{Extract extension part of filename}
\funcref{ExtractFileName}{Extract name part of filename}
\funcref{ExtractFilePath}{Extrct path part of filename}
\funcref{ExtractRelativePath}{Construct relative path between two files}
\funcref{FileAge}{Return file age}
\funcref{FileDateToDateTime}{Convert file date to system date}
\funcref{FileExists}{Determine whether a file exists on disk}
\funcref{FileGetAttr}{Get attributes of file}
\funcref{FileGetDate}{Get date of last file modification}
\funcref{FileSearch}{Search for file in path}
\funcrefl{FileSetAttr}{FileSetAttr}{Get file attributes}
\funcrefl{FileSetDate}{FileSetDate}{Get file dates}
\funcref{FindFirst}{Start finding a file}
\funcref{FindNext}{Find next file}
\funcref{GetCurrentDir}{Return current working directory}
\funcref{RemoveDir}{Remove a directory from disk}
\funcref{RenameFile}{Rename a file on disk}
\funcref{SetCurrentDir}{Set current working directory}
\funcref{SetDirSeparators}{Set directory separator characters}
\procrefl{FindClose}{FindCloseSys}{Stop searching a file}
\procref{DoDirSeparators}{Replace directory separator characters}
\end{funclist}

\subsection{Date/time routines}
Functions for date and time handling.
\begin{funclist}
\funcref{DateTimeToFileDate}{Convert DateTime type to file date}
\funcref{DateTimeToStr}{Construct string representation of DateTime}
\procref{DateTimeToString}{Construct string representation of DateTime}
\procref{DateTimeToSystemTime}{Convert DateTime to system time}
\funcref{DateTimeToTimeStamp}{Convert DateTime to timestamp}
\funcref{DateToStr}{Construct string representation of date}
\funcref{Date}{Get current date}
\funcref{DayOfWeek}{Get day of week}
\procref{DecodeDate}{Decode DateTime to year month and day}
\procref{DecodeTime}{Decode DateTime to hours, minutes and seconds}
\funcref{EncodeDate}{Encode year, day and month to DateTime}
\funcref{EncodeTime}{Encode hours, minutes and seconds to DateTime}
\funcref{FormatDateTime}{Return string representation of DateTime}
\funcref{IncMonth}{Add 1 to month}
\funcref{IsLeapYear}{Determine if year is leap year}
\funcref{MSecsToTimeStamp}{Convert nr of milliseconds to timestamp}
\funcref{Now}{Get current date and time}
\funcref{StrToDateTime}{Convert string to DateTime}
\funcref{StrToDate}{Convert string to date}
\funcref{StrToTime}{Convert string to time}
\funcref{SystemTimeToDateTime}{Convert system time to datetime}
\funcref{TimeStampToDateTime}{Convert time stamp to DateTime}
\funcref{TimeStampToMSecs}{Convert Timestamp to number of millicseconds}
\funcref{TimeToStr}{return string representation of Time}
\funcref{Time}{Get current tyme}
\end{funclist}

\section{Miscellaneous conversion routines}
Functions for various conversions.
\begin{funclist}
\funcref{BCDToInt}{Convert BCD number to integer}
\funcref{CompareMem}{Compare two memory regions}
\funcref{FloatToStrF}{Convert float to formatted string}
\funcref{FloatToStr}{Convert float to string}
\funcref{FloatToText}{Convert float to string}
\funcref{FormatFloat}{Format a floating point value}
\funcref{GetDirs}{Split string in list of directories}
\funcref{IntToHex}{return hexadecimal representation of integer}
\funcref{IntToStr}{return decumal representation of integer}
\funcref{StrToIntDef}{Convert string to integer with default value}
\funcref{StrToInt}{Convert string to integer}
\funcref{StrToFloat}{Convert string to float}
\funcref{TextToFloat}{Convert null-terminated string to float}
\end{funclist}

\section{Date and time functions}

\subsection{Date and time formatting characters}
\label{se:formatchars}

Various date and time formatting routines accept a format string.
to format the date and or time. The following characters can be used
to control the date and time formatting:
\begin{description}
\item[c] : shortdateformat + ' ' + shorttimeformat
\item[d] : day of month
\item[dd] : day of month (leading zero)
\item[ddd] : day of week (abbreviation)
\item[dddd] : day of week (full)
\item[ddddd] : shortdateformat
\item[dddddd] : longdateformat
\item[m] : month
\item[mm] : month (leading zero)
\item[mmm] : month (abbreviation)
\item[mmmm] : month (full)
\item[y] : year (four digits)
\item[yy] : year (two digits)
\item[yyyy] : year (with century)
\item[h] : hour
\item[hh] : hour (leading zero)
\item[n] : minute
\item[nn] : minute (leading zero)
\item[s] : second
\item[ss] : second (leading zero)
\item[t] : shorttimeformat
\item[tt] : longtimeformat
\item[am/pm] : use 12 hour clock and display am and pm accordingly
\item[a/p] : use 12 hour clock and display a and p accordingly
\item[/] : insert date seperator
\item[:] : insert time seperator
\item["xx"] : literal text
\item['xx'] : literal text
\end{description}

\begin{type}{TDateTime}
\Declaration
  TDateTime = Double;
\Description
Many functions return or require a \var{TDateTime} type, which contains
a date and time in encoded form. The date and time are converted to a double
as follows:
\begin{itemize}
\item The date part is stored in the integer part of the double as the
number of days passed since January 1, 1900.
\item The time part is stored in the fractional part of the double, as
the number of milliseconds passed since midnight (00:00), divided by the
total number of milliseconds in a day.
\end{itemize}
\end{type}

\begin{function}{Date}
\Declaration
Function Date: TDateTime;
\Description
\var{Date} returns the current date in \var{TDateTime} format.
For more information about the \var{TDateTime} type, see \seetype{TDateTime}.
\Errors
None.
\SeeAlso
\seef{Time},\seef{Now}, \seetype{TDateTime}.
\end{function}

\FPCexample{ex1}


\begin{function}{DateTimeToFileDate}
\Declaration
Function DateTimeToFileDate(DateTime : TDateTime) : Longint;
\Description
\var{DateTimeToFileDate} function converts a date/time indication in
\var{TDateTime} format to a filedate function, such as returned for
instance by the \seef{FileAge} function.
\Errors
None.
\SeeAlso
\seef{Time}, \seef{Date}, \seef{FileDateToDateTime},
\seep{DateTimeToSystemTime}, \seef{DateTimeToTimeStamp}
\end{function}

\FPCexample{ex2}


\begin{function}{DateTimeToStr}
\Declaration
Function DateTimeToStr(DateTime: TDateTime): string;
\Description
\var{DateTimeToStr} returns a string representation of
\var{DateTime} using the formatting specified in
\var{ShortDateTimeFormat}. It corresponds to a call to
\var{FormatDateTime('c',DateTime)} (see \sees{formatchars}).
\Errors
None.
\SeeAlso
\seef{FormatDateTime}, \seetype{TDateTime}.
\end{function}

\FPCexample{ex3}


\begin{procedure}{DateTimeToString}
\Declaration
Procedure DateTimeToString(var Result: string; const FormatStr: string; const DateTime: TDateTime);
\Description
\var{DateTimeToString} returns in \var{Result} a string representation of
\var{DateTime} using the formatting specified in \var{FormatStr}.

for a list of characters that can be used in the \var{FormatStr} formatting
string, see \sees{formatchars}.
\Errors
In case a wrong formatting character is found, an \var{EConvertError} is
raised.
\SeeAlso
\seef{FormatDateTime}, \sees{formatchars}.
\end{procedure}

\FPCexample{ex4}


\begin{procedure}{DateTimeToSystemTime}
\Declaration
Procedure DateTimeToSystemTime(DateTime: TDateTime; var SystemTime: TSystemTime);
\Description
\var{DateTimeToSystemTime} converts a date/time pair in \var{DateTime}, with
\var{TDateTime} format to a system time \var{SystemTime}.
\Errors
None.
\SeeAlso
\seef{DateTimeToFileDate}, \seef{SystemTimeToDateTime},
\seef{DateTimeToTimeStamp}
\end{procedure}

\FPCexample{ex5}


\begin{function}{DateTimeToTimeStamp}
\Declaration
Function DateTimeToTimeStamp(DateTime: TDateTime): TTimeStamp;
\Description
\var{DateTimeToSystemTime} converts a date/time pair in \var{DateTime}, with
\var{TDateTime} format to a \var{TTimeStamp} format.
\Errors
None.
\SeeAlso
\seef{DateTimeToFileDate}, \seef{SystemTimeToDateTime},
\seep{DateTimeToSystemTime}
\end{function}

\FPCexample{ex6}


\begin{function}{DateToStr}
\Declaration
Function DateToStr(Date: TDateTime): string;
\Description
\var{DateToStr} converts \var{Date} to a string representation. It uses
\var{ShortDateFormat} as it's formatting string. It is hence completely
equivalent to a \var{FormatDateTime('ddddd', Date)}.
\Errors
None.
\SeeAlso
\seef{TimeToStr}, \seef{DateTimeToStr}, \seef{FormatDateTime},
\seef{StrToDate}
\end{function}


\FPCexample{ex7}


\begin{function}{DayOfWeek}
\Declaration
Function DayOfWeek(DateTime: TDateTime): integer;
\Description
\var{DayOfWeek} returns the day of the week from \var{DateTime}.
\var{Sunday} is counted as day 1, \var{Saturday} is counted as
day 7. The result of \var{DayOfWeek} can serve as an index to
the \var{LongDayNames} constant array, to retrieve the name of
the day.
\Errors
None.
\SeeAlso
\seef{Date}, \seef{DateToStr}
\end{function}


\FPCexample{ex8}


\begin{procedure}{DecodeDate}
\Declaration
Procedure DecodeDate(Date: TDateTime; var Year, Month, Day: word);
\Description
\var{DecodeDate} decodes the Year, Month and Day stored in \var{Date},
and returns them in the \var{Year}, \var{Month} and \var{Day} variables.
\Errors
None.
\SeeAlso
\seef{EncodeDate}, \seep{DecodeTime}.
\end{procedure}

\FPCexample{ex9}



\begin{procedure}{DecodeTime}
\Declaration
Procedure DecodeTime(Time: TDateTime; var Hour, Minute, Second, MilliSecond: word);
\Description
\var{DecodeDate} decodes the hours, minutes, second and milliseconds stored
in \var{Time}, and returns them in the \var{Hour}, \var{Minute} and
\var{Second} and \var{MilliSecond} variables.
\Errors
None.
\SeeAlso
\seef{EncodeTime}, \seep{DecodeDate}.
\end{procedure}

\FPCexample{ex10}


\begin{function}{EncodeDate}
\Declaration
Function EncodeDate(Year, Month, Day :word): TDateTime;
\Description
\var{EncodeDate} encodes the \var{Year}, \var{Month} and \var{Day} variables to
a date in \var{TDateTime} format. It does the opposite of the
\seep{DecodeDate} procedure.

The parameters must lie withing valid ranges (boundaries included):
\begin{description}
\item[Year] must be between 1 and 9999.
\item[Month] must be within the range 1-12.
\item[Day] msut be between 1 and 31.
\end{description}
\Errors
In case one of the parameters is out of it's valid range, 0 is returned.
\SeeAlso
\seef{EncodeTime}, \seep{DecodeDate}.
\end{function}

\FPCexample{ex11}


\begin{function}{EncodeTime}
\Declaration
Function EncodeTime(Hour, Minute, Second, MilliSecond:word): TDateTime;
\Description
\var{EncodeTime} encodes the \var{Hour}, \var{Minute}, \var{Second},
\var{MilliSecond} variables to a \var{TDateTime} format result.
It does the opposite of the \seep{DecodeTime} procedure.

The parameters must have a valid range (boundaries included):
\begin{description}
\item[Hour] must be between 0 and 23.
\item[Minute,second] must both be between 0 and 59.
\item[Millisecond] must be between 0 and 999.
\end{description}
\Errors
In case one of the parameters is outside of it's valid range, 0 is returned.
\SeeAlso
\seef{EncodeDate}, \seep{DecodeTime}.
\end{function}

\FPCexample{ex12}



\begin{function}{FileDateToDateTime}
\Declaration
Function FileDateToDateTime(Filedate : Longint) : TDateTime;
\Description
\var{FileDateToDateTime} converts the date/time encoded in \var{filedate}
to a \var{TDateTime} encoded form. It can be used to convert date/time values
returned by the \seef{FileAge} or \seef{FindFirst}/\seef{FindNext}
functions to \var{TDateTime} form.
\Errors
None.
\SeeAlso
\seef{DateTimeToFileDate}
\end{function}

\FPCexample{ex13}


\begin{function}{FormatDateTime}
\Declaration
Function FormatDateTime(FormatStr: string; DateTime: TDateTime):string;
\Description
\var{FormatDateTime} formats the date and time encoded in \var{DateTime}
according to the formatting given in \var{FormatStr}. The complete list
of formatting characters can be found in \sees{formatchars}.
\Errors
On error (such as an invalid character in the formatting string), and
\var{EConvertError} exception is raised.
\SeeAlso
\seef{DateTimeToStr}, \seef{DateToStr}, \seef{TimeToStr},
\seef{StrToDateTime}
\end{function}

\FPCexample{ex14}



\begin{function}{IncMonth}
\Declaration
Function IncMonth(const DateTime: TDateTime; NumberOfMonths: integer): TDateTime;
\Description
\var{IncMonth} increases the month number in \var{DateTime} with
\var{NumberOfMonths}. It wraps the result as to get a month between 1 and
12, and updates the year accordingly. \var{NumberOfMonths} can be negative,
and can be larger than 12 (in absolute value).
\Errors
None.
\SeeAlso
\seef{Date}, \seef{Time}, \seef{Now}
\end{function}


\FPCexample{ex15}


\begin{function}{IsLeapYear}
\Declaration
Function IsLeapYear(Year: Word): boolean;
\Description
\var{IsLeapYear} returns \var{True} if \var{Year} is a leap year,
\var{False} otherwise.
\Errors
None.
\SeeAlso
\seef{IncMonth}, \seef{Date}
\end{function}

\FPCexample{ex16}


\begin{function}{MSecsToTimeStamp}
\Declaration
Function MSecsToTimeStamp(MSecs: Comp): TTimeStamp;
\Description
\var{MSecsTiTimeStamp} converts the given number of milliseconds to
a \var{TTimeStamp} date/time notation.

Use \var{TTimeStamp} variables if you need to keep very precise track of
time.
\Errors
None.
\SeeAlso
\seef{TimeStampToMSecs}, \seef{DateTimeToTimeStamp},
\end{function}

\FPCexample{ex17}


\begin{function}{Now}
\Declaration
Function Now: TDateTime;
\Description
\var{Now} returns the current date and time. It is equivalent to
\var{Date+Time}.
\Errors
None.
\SeeAlso
\seef{Date}, \seef{Time}
\end{function}

\FPCexample{ex18}


\begin{function}{StrToDate}
\Declaration
Function StrToDate(const S: string): TDateTime;
\Description
\var{StrToDate} converts the string \var{S} to a \var{TDateTime} date
value. The Date must consist of 1 to three digits, separated by the
\var{DateSeparator} character. If two numbers are given, they
are supposed to form the day and month of the current year. If only
one number is given, it is supposed to represent the day of the
current month. (This is \em{not} supported in Delphi)

The order of the digits (y/m/d, m/d/y, d/m/y) is determined from the
\var{ShortDateFormat} variable.
\Errors
On error (e.g. an invalid date or invalid character),
an \var{EConvertError} exception is raised.
\SeeAlso
\seef{StrToTime}, \seef{DateToStr}n \seef{TimeToStr}.
\end{function}

\FPCexample{ex19}


\begin{function}{StrToDateTime}
\Declaration
Function StrToDateTime(const S: string): TDateTime;
\Description
\var{StrToDateTime} converts the string \var{S} to a \var{TDateTime} date
and time value. The Date must consist of 1 to three digits, separated by the
\var{DateSeparator} character. If two numbers are given, they
are supposed to form the day and month of the current year. If only
one number is given, it is supposed to represent the day of the
current month. (This is \em{not} supported in Delphi)

The order of the digits (y/m/d, m/d/y, d/m/y) is determined from the
\var{ShortDateFormat} variable.
\Errors
On error (e.g. an invalid date or invalid character),
an \var{EConvertError} exception is raised.
\SeeAlso
\seef{StrToDate}, \seef{StrToTime}, \seef{DateTimeToStr}
\end{function}

\FPCexample{ex20}


\begin{function}{StrToTime}
\Declaration
Function StrToTime(const S: string): TDateTime;
\Description
\var{StrToTime} converts the string \var{S} to a \var{TDateTime} time
value. The time must consist of 1 to 4 digits, separated by the
\var{TimeSeparator} character. If two numbers are given, they
are supposed to form the hour and minutes.
\Errors
On error (e.g. an invalid date or invalid character),
an \var{EConvertError} exception is raised.
\SeeAlso
\seef{StrToDate}, \seef{StrToDateTime}, \seef{TimeToStr}
\end{function}

\FPCexample{ex21}


\begin{function}{SystemTimeToDateTime}
\Declaration
Function SystemTimeToDateTime(const SystemTime: TSystemTime): TDateTime;
\Description
\var{SystemTimeToDateTime} converts a \var{TSystemTime} record to a
\var{TDateTime} style date/time indication.
\Errors
None.
\SeeAlso
\seep{DateTimeToSystemTime}
\end{function}

\FPCexample{ex22}


\begin{function}{Time}
\Declaration
Function Time: TDateTime;
\Description
\var{Time} returns the current time in \var{TDateTime} format. The date
part of the \var{TDateTimeValue} is set to zero.
\Errors
None.
\SeeAlso
\seef{Now}, \seef{Date}
\end{function}


\FPCexample{ex23}


\begin{function}{TimeStampToDateTime}
\Declaration
Function TimeStampToDateTime(const TimeStamp: TTimeStamp): TDateTime;
\Description
\var{TimeStampToDateTime} converts \var{TimeStamp} to a \var{TDateTime}
format variable. It is the inverse operation of \seef{DateTimeToTimeStamp}.
\Errors
None.
\SeeAlso
\seef{DateTimeToTimeStamp}, \seef{TimeStampToMSecs}
\end{function}

\FPCexample{ex24}


\begin{function}{TimeStampToMSecs}
\Declaration
Function TimeStampToMSecs(const TimeStamp: TTimeStamp): comp;
\Description
\var{TimeStampToMSecs} converts {TimeStamp} to the number of seconds
since \var{1/1/0001}.

Use \var{TTimeStamp} variables if you need to keep very precise track of
time.
\Errors
None.
\SeeAlso
\seef{MSecsToTimeStamp}, \seef{TimeStampToDateTime}
\end{function}

For an example, see \seef{MSecsToTimeStamp}.

\begin{function}{TimeToStr}
\Declaration
Function TimeToStr(Time: TDateTime): string;
\Description
\var{TimeToStr} converts the time in \var{Time} to a string. It uses
the \var{ShortTimeFormat} variable to see what formatting needs to be
applied. It is therefor entirely equivalent to a
\var{FormatDateTime('t',Time)} call.
\Errors
None.
\SeeAlso
\end{function}

\FPCexample{ex25}



\section{Disk functions}

\begin{functionl}{AddDisk (Linux only)}{AddDisk}
\Declaration
Function AddDisk (Const PAth : String) : Longint;
\Description
On Linux  both the \seef{DiskFree} and \seef{DiskSize} functions need a
file on the specified drive, since is required for the statfs system call.

These filenames are set in drivestr[0..26], and the first 4 have been
preset to :
\begin{description}
\item[Disk 0]  \var{'.'} default drive - hence current directory is used.
\item[Disk 1]  \var{'/fd0/.'} floppy drive 1.
\item[Disk 2]  \var{'/fd1/.'} floppy drive 2.
\item[Disk 3]  \var{'/'} \file{C:} equivalent of DOS is the root partition.
\end{description}
Drives 4..26 can be set by your own applications with the \var{AddDisk} call.

The \var{AddDisk} call adds \var{Path} to the names of drive files, and
returns the number of the disk that corresponds to this drive. If you
add more than 21 drives, the count is wrapped to 4.
\Errors
None.
\SeeAlso
\seefl{DiskFree}{DiskFreeSys}, \seefl{DiskSize}{DiskSizeSys}
\end{functionl}

\begin{function}{CreateDir}
\Declaration
Function CreateDir(Const NewDir : String) : Boolean;
\Description
\var{CreateDir} creates a new directory with name \var{NewDir}.
If the directory doesn't contain an absolute path, then the directory is
created below the current working directory.

The function returns \var{True} if the directory was successfully
created, \var{False} otherwise.
\Errors
In case of an error, the function returns \var{False}.
\SeeAlso
\seef{RemoveDir}
\end{function}

\FPCexample{ex26}


\begin{functionl}{DiskFree}{DiskFreeSys}
\Declaration
Function DiskFree(Drive : Byte) : Int64;
\Description
\var{DiskFree} returns the free space (in bytes) on disk \var{Drive}.
Drive is the number of the disk drive:
\begin{description}
\item[0] for the current drive.
\item[1] for the first floppy drive.
\item[2] for the second floppy drive.
\item[3] for the first hard-disk parttion.
\item[4-26] for all other drives and partitions.
\end{description}

{\em Remark} Under \linux, and Unix in general, the concept of disk is
different than the \dos one, since the filesystem is seen as one big
directory tree. For this reason, the \var{DiskFree} and \seef{DiskSize}
functions must be mimicked using filenames that reside on the partitions.
For more information, see \seef{AddDisk}
\Errors
On error, \var{-1} is returned.
\SeeAlso
\seefl{DiskSize}{DiskSizeSys}, \seef{AddDisk}
\end{functionl}

\FPCexample{ex27}


\begin{functionl}{DiskSize}{DiskSizeSys}
\Declaration
Function DiskSize(Drive : Byte) : Int64;
\Description
\var{DiskSize} returns the size (in bytes) of disk \var{Drive}.
Drive is the number of the disk drive:
\begin{description}
\item[0] for the current drive.
\item[1] for the first floppy drive.
\item[2] for the second floppy drive.
\item[3] for the first hard-disk parttion.
\item[4-26] for all other drives and partitions.
\end{description}

{\em Remark} Under \linux, and Unix in general, the concept of disk is
different than the \dos one, since the filesystem is seen as one big
directory tree. For this reason, the \seef{DiskFree} and \var{DiskSize}
functions must be mimicked using filenames that reside on the partitions.
For more information, see \seef{AddDisk}
\Errors
On error, \var{-1} is returned.
\SeeAlso
\seefl{DiskFree}{DiskFreeSys}, \seef{AddDisk}
\end{functionl}

For an example, see \seefl{DiskFree}{DiskFreeSys}.

\begin{function}{GetCurrentDir}
\Declaration
Function GetCurrentDir : String;
\Description
\var{GetCurrentDir} returns the current working directory.
\Errors
None.
\SeeAlso
\seef{SetCurrentDir}, \seef{DiskFree}, \seef{DiskSize}
\end{function}

\FPCexample{ex28}


\begin{function}{RemoveDir}
\Declaration
Function RemoveDir(Const Dir : String) : Boolean;
\Description
\var{RemoveDir} removes directory \var{Dir} from the disk.
If the directory is not absolue, it is appended to the current working
directory.
\Errors
In case of error (e.g. the directory isn't empty) the function returns
\var{False}. If successful, \var{True} is returned.
\SeeAlso
\end{function}

For an example, see \seef{CreateDir}.

\begin{function}{SetCurrentDir}
\Declaration
Function SetCurrentDir(Const NewDir : String) : Boolean;
\Description
\var{SetCurrentDir} sets the current working directory of your program
to \var{NewDir}. It returns \var{True} if the function was successfull,
\var{False} otherwise.
\Errors
In case of error, \var{False} is returned.
\SeeAlso
\seef{GetCurrentDir}
\end{function}

\FPCexample{ex29}


\section{File handling functions}

\begin{function}{ChangeFileExt}
\Declaration
Function ChangeFileExt(const FileName, Extension: string): string;
\Description
\var{ChangeFileExt} changes the file extension in \var{FileName} to
\var{Extension}.
The extension \var{Extension} includes the starting \var{.} (dot).
The previous extension of \var{FileName} are all characters after the
last \var{.}, the \var{.} character included.

If \var{FileName} doesn't have an extension, \var{Extension} is just
appended.
\Errors
None.
\SeeAlso
\seef{ExtractFileName}, \seef{ExtractFilePath}, \seef{ExpandFileName}
\end{function}


\begin{function}{DeleteFile}
\Declaration
Function DeleteFile(Const FileName : String) : Boolean;
\Description
\var{DeleteFile} deletes file \var{FileName} from disk. The function
returns \var{True} if the file was successfully removed, \var{False}
otherwise.
\Errors
On error, \var{False} is returned.
\SeeAlso
\seef{FileCreate}, \seef{FileExists}
\end{function}

\FPCexample{ex31}


\begin{procedure}{DoDirSeparators}
\Declaration
Procedure DoDirSeparators(Var FileName : String);
\Description
This function replaces all directory separators \var{'\' and '/'}
to the directory separator character for the current system.
\Errors
None.
\SeeAlso
\seef{ExtractFileName}, \seef{ExtractFilePath}
\end{procedure}

\FPCexample{ex32}


\begin{function}{ExpandFileName}
\Declaration
Function ExpandFileName(Const FileName : string): String;
\Description
\var{ExpandFileName} expands the filename to an absolute filename.
It changes all directory separator characters to the one appropriate for the
system first.
\Errors
None.
\SeeAlso
\seef{ExtractFileName}, \seef{ExtractFilePath}, \seef{ExtractFileDir},
\seef{ExtractFileDrive}, \seef{ExtractFileExt}, \seef{ExtractRelativePath}
\end{function}

\FPCexample{ex33}



\begin{function}{ExpandUNCFileName}
\Declaration
Function ExpandUNCFileName(Const FileName : string): String;
\Description
\var{ExpandUNCFileName} runs \seef{ExpandFileName} on \var{FileName}
and then attempts to replace the driveletter by the name of a shared disk.
\Errors
\SeeAlso
\seef{ExtractFileName}, \seef{ExtractFilePath}, \seef{ExtractFileDir},
\seef{ExtractFileDrive}, \seef{ExtractFileExt}, \seef{ExtractRelativePath}
\end{function}


\begin{function}{ExtractFileDir}
\Declaration
Function ExtractFileDir(Const FileName : string): string;
\Description
\var{ExtractFileDir} returns only the directory part of \var{FileName},
not including a driveletter. The directory name has NO ending directory
separator, in difference with \seef{ExtractFilePath}.
\Errors
None.
\SeeAlso
\seef{ExtractFileName}, \seef{ExtractFilePath}, \seef{ExtractFileDir},
\seef{ExtractFileDrive}, \seef{ExtractFileExt}, \seef{ExtractRelativePath}
\end{function}

\FPCexample{ex34}


\begin{function}{ExtractFileDrive}
\Declaration
Function ExtractFileDrive(const FileName: string): string;
\Description
\var{Extract}
\Errors
\SeeAlso
\seef{ExtractFileName}, \seef{ExtractFilePath}, \seef{ExtractFileDir},
\seef{ExtractFileDrive}, \seef{ExtractFileExt}, \seef{ExtractRelativePath}
\end{function}

For an example, see \seef{ExtractFileDir}.

\begin{function}{ExtractFileExt}
\Declaration
Function ExtractFileExt(const FileName: string): string;
\Description
\var{ExtractFileExt} returns the extension (including the
\var{.}(dot) character) of \var{FileName}.
\Errors
None.
\SeeAlso
\seef{ExtractFileName}, \seef{ExtractFilePath}, \seef{ExtractFileDir},
\seef{ExtractFileDrive}, \seef{ExtractFileExt}, \seef{ExtractRelativePath}
\end{function}

For an example, see \seef{ExtractFileDir}.

\begin{function}{ExtractFileName}
\Declaration
Function ExtractFileName(const FileName: string): string;
\Description
\var{ExtractFileName} returns the filename part from \var{FileName}.
The filename consists of all characters after the last directory separator
character ('/' or '\') or drive letter.

The full filename can always be reconstucted by concatenating the result
of \seef{ExtractFilePath} and \var{ExtractFileName}.
\Errors
None.
\SeeAlso
\seef{ExtractFileName}, \seef{ExtractFilePath}, \seef{ExtractFileDir},
\seef{ExtractFileDrive}, \seef{ExtractFileExt},\seef{ExtractRelativePath}
\end{function}

For an example, see \seef{ExtractFileDir}.

\begin{function}{ExtractFilePath}
\Declaration
Function ExtractFilePath(const FileName: string): string;
\Description
\var{ExtractFilePath} returns the path part (including driveletter) from
\var{FileName}. The path consists of all characters before the last
directory separator character ('/' or '\'), including the directory
separator itself.
In case there is only a drive letter, that will be returned.

The full filename can always be reconstucted by concatenating the result
of \var{ExtractFilePath} and \seef{ExtractFileName}.
\Errors
None.
\SeeAlso
\seef{ExtractFileName}, \seef{ExtractFilePath}, \seef{ExtractFileDir},
\seef{ExtractFileDrive}, \seef{ExtractFileExt}, \seef{ExtractRelativePath}
\end{function}

For an example, see \seef{ExtractFileDir}.

\begin{function}{ExtractRelativePath}
\Declaration
Function ExtractRelativePath(Const BaseName,DestNAme : String): String;
\Description
\var{ExtractRelativePath} constructs a relative path to go from
\var{BaseName} to \var{DestName}. If \var{DestName} is on another drive
(Not on Linux) then the whole \var{Destname} is returned.

{\em Note:} This function does not exist in the Delphi unit.
\Errors
None.
\SeeAlso
\seef{ExtractFileName}, \seef{ExtractFilePath}, \seef{ExtractFileDir},
\seef{ExtractFileDrive}, \seef{ExtractFileExt},
\end{function}

\FPCexample{ex35}


\begin{function}{FileAge}
\Declaration
Function FileAge(Const FileName : String): Longint;
\Description
\var{FileAge} returns the last modification time of file \var{FileName}.
The FileDate format can be transformed to \var{TDateTime} format with the
\seef{FileDateToDateTime} function.
\Errors
In case of errors, \var{-1} is returned.
\SeeAlso
\seef{FileDateToDateTime}, \seef{FileExists}, \seef{FileGetAttr}
\end{function}

\FPCexample{ex36}



\begin{procedure}{FileClose}
\Declaration
Procedure FileClose(Handle : Longint);
\Description
\var{FileClose} closes the file handle \var{Handle}. After this call,
attempting to read or write from the handle will result in an error.
\Errors
None.
\SeeAlso
\seef{FileCreate}, \seef{FileWrite}, \seef{FileOpen}, \seef{FileRead},
\seef{FileTruncate}, \seef{FileSeek}
\end{procedure}

For an example, see \seef{FileCreate}

\begin{function}{FileCreate}
\Declaration
Function FileCreate(Const FileName : String) : Longint;
\Description
\var{FileCreate} creates a new file with name \var{FileName} on the disk and
returns a file handle which can be used to read or write from the file with
the \seef{FileRead} and \seef{FileWrite} functions.

If a file with name \var{FileName} already existed on the disk, it is
overwritten.
\Errors
If an error occurs (e.g. disk full or non-existent path), the function
returns \var{-1}.
\SeeAlso
\seep{FileClose}, \seef{FileWrite}, \seef{FileOpen}, \seef{FileRead},
\seef{FileTruncate}, \seef{FileSeek}
\end{function}

\FPCexample{ex37}


\begin{function}{FileExists}
\Declaration
Function FileExists(Const FileName : String) : Boolean;
\Description
\var{FileExists} returns \var{True} if a file with name \var{FileName}
exists on the disk, \var{False} otherwise.
\Errors
None.
\SeeAlso
\seef{FileAge}, \seef{FileGetAttr}, \seef{FileSetAttr}
\end{function}


\FPCexample{ex38}



\begin{function}{FileGetAttr}
\Declaration
Function FileGetAttr(Const FileName : String) : Longint;
\Description
\var{FileGetAttr} returns the attribute settings of file
\var{FileName}. The attribute is a \var{OR}-ed combination
of the following constants:
\begin{description}
\item[faReadOnly] The file is read-only.
\item[faHidden] The file is hidden. (On \linux, this means that the filename
starts with a dot)
\item[faSysFile] The file is a system file (On \linux, this means that the
file is a character, block or FIFO file).
\item[faVolumeId] Volume Label. Not possible under \linux.
\item[faDirectory] File is a directory.
\item[faArchive] file is an archive. Not possible on \linux.
\end{description}
\Errors
In case of error, -1 is returned.
\SeeAlso
\seef{FileSetAttr}, \seef{FileAge}, \seef{FileGetDate}.
\end{function}

\FPCexample{ex40}


\begin{function}{FileGetDate}
\Declaration
Function FileGetDate(Handle : Longint) : Longint;
\Description
\var{FileGetdate} returns the filetime of the opened file with filehandle
\var{Handle}. It is the same as \seef{FileAge}, with this difference that
\var{FileAge} only needs the file name, while \var{FilegetDate} needs an
open file handle.
\Errors
On error, -1 is returned.
\SeeAlso
\seef{FileAge}
\end{function}

\FPCexample{ex39}


\begin{function}{FileOpen}
\Declaration
Function FileOpen(Const FileName : string; Mode : Integer) : Longint;
\Description
\var{FileOpen} opens a file with name \var{FileName} with mode \var{Mode}.
\var{Mode} can be one of the following constants:
\begin{description}
\item[fmOpenRead] The file is opened for reading.
\item[fmOpenWrite] The file is opened for writing.
\item[fmOpenReadWrite] The file is opened for reading and writing.
\end{description}
If the file has been successfully opened, it can be read  from or written to
(depending on the \var{Mode} parameter) with the \seef{FileRead} and
\var{FileWrite} functions.

Remark that you cannot open a file if it doesn't exist yet, i.e. it will not
be created for you. If you want tp create a new file, or overwrite an old
one, use the \seef{FileCreate} function.
\Errors
On Error, -1 is returned.
\SeeAlso
\seep{FileClose}, \seef{FileWrite}, \seef{FileCreate}, \seef{FileRead},
\seef{FileTruncate}, \seef{FileSeek}
\end{function}

For an example, see \seef{FileOpen}

\begin{function}{FileRead}
\Declaration
Function FileRead(Handle : Longint; Var Buffer; Count : longint) : Longint;
\Description
\var{FileRead} reads \var{Count} bytes from file-handle \var{Handle} and
stores them into \var{Buffer}. Buffer must be at least \var{Count} bytes
long. No checking on this is performed, so be careful not to overwrite any
memory.  \var{Handle} must be the result of a \seef{FileOpen} call.
\Errors
On error, -1 is returned.
\SeeAlso
\seep{FileClose}, \seef{FileWrite}, \seef{FileCreate}, \seef{FileOpen},
\seef{FileTruncate}, \seef{FileSeek}
\end{function}

For an example, see \seef{FileCreate}

\begin{function}{FileSearch}
\Declaration
Function FileSearch(Const Name, DirList : String) : String;
\Description
\var{FileSearch} looks for the file \var{Name} in \var{DirList}, where
dirlist is a list of directories, separated by semicolons or colons.
It returns the full filename of the first match found.
\Errors
On error, an empty string is returned.
\SeeAlso
\seef{ExpandFileName}, \seef{FindFirst}
\end{function}

\FPCexample{ex41}


\begin{function}{FileSeek}
\Declaration
Function FileSeek(Handle,Offset,Origin : Longint) : Longint;
\Description
\var{FileSeek} sets the file pointer on position \var{Offset}, starting from
\var{Origin}. Origin can be one of the following values:
\begin{description}
\item[fsFromBeginning]  \var{Offset} is relative to the first byte of the file. This
position is zero-based. i.e. the first byte is at offset 0.
\item[fsFromCurrent]  \var{Offset} is relative to the current position.
\item[fsFromEnd] \var{Offset} is relative to the end of the file. This means
that \var{Offset} can only be zero or negative in this case.
\end{description}
If successfull, the function returns the new file position, relative to the
beginning of the file.

{\em Remark:} The abovementioned constants do not exist in Delphi.
\Errors
On error, -1 is returned.
\SeeAlso
\seep{FileClose}, \seef{FileWrite}, \seef{FileCreate}, \seef{FileOpen}
\seef{FileRead}, \seef{FileTruncate}
\end{function}

\FPCexample{ex42}


For an example, see \seef{FileCreate}

\begin{functionl}{FileSetAttr (Not on Linux)}{FileSetAttr}
\Declaration
Function FileSetAttr(Const Filename : String; Attr: longint) : Longint;
\Description
\var{FileSetAttr} sets the attributes of \var{FileName} to \var{Attr}.
If the function was successful, 0 is returned, -1 otherwise.

\var{Attr} can be set to an OR-ed combination of the pre-defined
\var{faXXX} constants.
\Errors
On error, -1 is returned (always on linux).
\SeeAlso
\seef{FileGetAttr}, \seef{FileGetDate}, \seef{FileSetDate}.
\end{functionl}


\begin{functionl}{FileSetDate (Not on Linux)}{FileSetDate}
\Declaration
Function FileSetDate(Handle,Age : Longint) : Longint;
\Description
\var{FileSetDate} sets the file date of the file with handle \var{Handle}
to \var{Age}, where \var{Age} is a DOS date-and-time stamp value.

The function returns zero of successfull.
\Errors
On Linux, -1 is always returned, since this is impossible to implement.
On Windows and DOS, a negative error code is returned.
\SeeAlso
\end{functionl}


\begin{function}{FileTruncate}
\Declaration
Function FileTruncate(Handle,Size: Longint) : boolean;
\Description
\var{FileTruncate} truncates the file with handle \var{Handle} to
\var{Size} bytes. The file must have been opened for writing prior
to this call. The function returns \var{True} is successful, \var{False}
otherwise.
\Errors
On error, the function returns \var{False}.
\SeeAlso
\seep{FileClose}, \seef{FileWrite}, \seef{FileCreate}, \seef{FileOpen}
\seef{FileRead}, \seef{FileSeek}
\end{function}

For an example, see \seef{FileCreate}.

\begin{function}{FileWrite}
\Declaration
Function FileWrite(Handle : Longint; Var Buffer; Count : Longint) : Longint;
\Description
\var{FileWrite} writes \var{Count} bytes from \var{Buffer} to the file with
handle \var{Handle}. Prior to this call, the file must have been opened
for writing. \var{Buffer} must be at least \var{Count} bytes large, or
a memory access error may occur.

The function returns the number of bytes written, or -1 in case of an
error.
\Errors
In case of error, -1 is returned.
\SeeAlso
\seep{FileClose}, \seef{FileCreate}, \seef{FileOpen}
\seef{FileRead}, \seef{FileTruncate}, \seef{FileSeek}
\end{function}

For an example, see \seef{FileCreate}.

\begin{procedurel}{FindClose}{FindCloseSys}
\Declaration
Procedure FindClose(Var F : TSearchrec);
\Description
\var{FindClose} ends a series of \seef{FindFirst}/\seef{FindNext} calls,
and frees any memory used by these calls. It is {\em absolutely} necessary
to do this call, or huge memory losses may occur.
\Errors
None.
\SeeAlso
\seef{FindFirst}, \seef{FindNext}.
\end{procedurel}

For an example, see \seef{FindFirst}.

\begin{function}{FindFirst}
\Declaration
Function FindFirst(Const Path : String; Attr : Longint; Var Rslt : TSearchRec) : Longint;
\Description
\var{FindFirst} looks for files that match the name (possibly with
wildcards) in \var{Path} and attributes \var{Attr}. It then fills up the
\var{Rslt} record with data gathered about the file. It returns 0 if a file
matching the specified criteria is found, a nonzero value (-1 on linux)
otherwise.

The \var{Rslt} record can be fed to subsequent calls to \var{FindNext}, in
order to find other files matching the specifications.

{\em remark:} A \var{FindFirst} call must {\em always} be followed by a
\seepl{FindClose}{FindCloseSys} call with the same \var{Rslt} record. Failure to do so will
result in memory loss.
\Errors
On error the function returns -1 on linux, a nonzero error code on Windows.
\SeeAlso
\seep{FindClose}{FindCloseSys}, \seef{FindNext}.
\end{function}

\FPCexample{ex43}


\begin{function}{FindNext}
\Declaration
Function FindNext(Var Rslt : TSearchRec) : Longint;
\Description
\var{FindNext} finds a next occurrence of a search sequence initiated by
\var{FindFirst}. If another record matching the criteria in Rslt is found, 0
is returned, a nonzero constant is returned otherwise.

{\em remark:} The last \var{FindNext} call must {\em always} be followed by a
\var{FindClose} call with the same \var{Rslt} record. Failure to do so will
result in memory loss.
\Errors
On error (no more file is found), a nonzero constant is returned.
\SeeAlso
\seef{FindFirst}, \seep{FindClose}
\end{function}

For an example, see \seef{FindFirst}

\begin{function}{GetDirs}
\Declaration
Function GetDirs(Var DirName : String; Var Dirs : Array of pchar) : Longint;
\Description
\var{GetDirs} splits DirName in a null-byte separated list of directory names,
\var{Dirs} is an array of \var{PChars}, pointing to these directory names.
The function returns the number of directories found, or -1 if none were found.
DirName must contain only OSDirSeparator as Directory separator chars.
\Errors
None.
\SeeAlso
\seef{ExtractRelativePath}
\end{function}

\FPCexample{ex45}


\begin{function}{RenameFile}
\Declaration
Function RenameFile(Const OldName, NewName : String) : Boolean;
\Description
\var{RenameFile} renames a file from \var{OldName} to \var{NewName}. The
function returns \var{True} if successful, \var{False} otherwise.

{\em Remark:} you cannot rename across disks or partitions.
\Errors
On Error, \var{False} is returned.
\SeeAlso
\seef{DeleteFile}
\end{function}

\FPCexample{ex44}


\begin{function}{SetDirSeparators}
\Declaration
Function SetDirSeparators(Const FileName : String) : String;
\Description
\var{SetDirSeparators} returns \var{FileName} with all possible
DirSeparators replaced by \var{OSDirSeparator}.
\Errors
None.
\SeeAlso
\seef{ExpandFileName}, \seef{ExtractFilePath}, \seef{ExtractFileDir}
\end{function}

\FPCexample{ex47}


\section{PChar functions}

\subsection{Introduction}

Most PChar functions are the same as their counterparts in the \file{STRINGS}
unit. The following functions are the same :

\begin{enumerate}
\item \seef{StrCat} : Concatenates two \var{PChar} strings.
\item \seef{StrComp} : Compares two \var{PChar} strings.
\item \seef{StrCopy} : Copies a \var{PChar} string.
\item \seef{StrECopy} : Copies a \var{PChar} string and returns a pointer to
the terminating null byte.
\item \seef{StrEnd} : Returns a pointer to the terminating null byte.
\item \seef{StrIComp} : Case insensitive compare of 2 \var{PChar} strings.
\item \seef{StrLCat} : Appends at most L characters from one \var{PChar} to
another \var{PChar}.
\item \seef{StrLComp} : Case sensitive compare of at most L characters of 2
 \var{PChar} strings.
\item \seef{StrLCopy} : Copies at most L characters from one \var{PChar} to
another.
\item \seef{StrLen} : Returns the length (exclusive terminating null byte)
of a \var{PChar} string.
\item \seef{StrLIComp} : Case insensitive compare of at most L characters of 2
 \var{PChar} strings.
\item \seef{StrLower} : Converts a \var{PChar} to all lowercase letters.
\item \seef{StrMove} : Moves one \var{PChar} to another.
\item \seef{StrNew} : Makes a copy of a \var{PChar} on the heap, and returns
a pointer to this copy.
\item \seef{StrPos} : Returns the position of one \var{PChar} string in
another?
\item \seef{StrRScan} : returns a pointer to the last occurrence of on
 \var{PChar} string in another one.
\item \seef{StrScan} : returns a pointer to the first occurrence of on
 \var{PChar} string in another one.
\item \seef{StrUpper} : Converts a \var{PChar} to all uppercase letters.
\end{enumerate}
The subsequent functions are different from their counterparts in
\file{STRINGS}, although the same examples can be used.


\begin{functionl}{StrAlloc}{StrAllocSys}
\Declaration
Function StrAlloc(Size: cardinal): PChar;
\Description
\var{StrAlloc} reserves memory on the heap for a string with length \var{Len},
terminating \var{\#0} included, and returns a pointer to it.

Additionally, \var{StrAlloc} allocates 4 extra bytes to store the size of
the allocated memory. Therefore this function is NOT compatible with the
\seef{StrAlloc} function of the \var{Strings} unit.
\Errors
None.
\SeeAlso
\seef{StrBufSize}, \seepl{StrDispose}{StrDisposeSys}, \seef{StrAlloc}
\end{functionl}

For an example, see \seef{StrBufSize}.

\begin{function}{StrBufSize}
\Declaration
Function StrBufSize(var Str: PChar): cardinal;
\Description
\var{StrBufSize} returns the memory allocated for \var{Str}. This function
ONLY gives the correct result if \var{Str} was allocated using
\seefl{StrAlloc}{StrAllocSys}.
\Errors
If no more memory is available, a runtime error occurs.
\SeeAlso
\seefl{StrAlloc}{StrAllocSys}.\seepl{StrDispose}{StrDisposeSys}.
\end{function}

\FPCexample{ex46}



\begin{procedurel}{StrDispose}{StrDisposeSys}
\Declaration
Procedure StrDispose(var Str: PChar);
\Description
\var{StrDispose} frees any memory allocated for \var{Str}. This function
will only function correctly if \var{Str} has been allocated using
\seefl{StrAlloc}{StrAllocSys} from the \file{SYSUTILS} unit.
\Errors
If an invalid pointer is passed, or a pointer not allocated with
\var{StrAlloc}, an error may occur.
\SeeAlso
\seef{StrBufSize}, \seefl{StrAlloc}{StrAllocSys}, \seep{StrDispose}
\end{procedurel}

For an example, see \seef{StrBufSize}.

\begin{functionl}{StrPCopy}{StrPCopySys}
\Declaration
Function StrPCopy(Dest: PChar; Source: string): PChar;
\Description
\var{StrPCopy} Converts the Ansistring in \var{Source} to a Null-terminated
string, and copies it to \var{Dest}. \var{Dest} needs enough room to contain
the string \var{Source}, i.e. \var{Length(Source)+1} bytes.
\Errors
No checking is performed to see whether \var{Dest} points to enough memory
to contain \var{Source}.
\SeeAlso
\seefl{StrPLCopy}{StrPLCopySys}, \seef{StrPCopy}
\end{functionl}

For an example, see \seef{StrPCopy}.

\begin{functionl}{StrPLCopy}{StrPLCopySys}
\Declaration
Function StrPLCopy(Dest: PChar; Source: string; MaxLen: cardinal): PChar;
\Description
\var{StrPLCopy} Converts maximally \var{MaxLen} characters of the
Ansistring in \var{Source} to a Null-terminated  string, and copies
it to \var{Dest}. \var{Dest} needs enough room to contain
the  characters.
\Errors
No checking is performed to see whether \var{Dest} points to enough memory
to contain L characters of \var{Source}.
\Errors
\SeeAlso
\seefl{StrPCopy}{StrPCopySys}.
\end{functionl}


\begin{functionl}{StrPas}{StrPasSys}
\Declaration
Function StrPas(Str: PChar): string;
\Description
Converts a null terminated string in \var{Str} to an Ansitring, and returns
this string. This string is NOT truncated at 255 characters as is the
\Errors
None.
\SeeAlso
\seef{StrPas}.
\end{functionl}

For an example, see \seef{StrPas}.

\section{String handling functions}

\begin{function}{AdjustLineBreaks}
\Declaration
Function AdjustLineBreaks(const S: string): string;
\Description
\var{AdjustLineBreaks} will change all \var{\#13} characters with
\var{\#13\#10} on \windowsnt and \dos. On \linux, all \var{\#13\#10}
character pairs are converted to \var{\#10} and single \var{\#13}
characters also.
\Errors
None.
\SeeAlso
\seef{AnsiCompareStr}, \seef{AnsiCompareText}
\end{function}

\FPCexample{ex48}


\begin{function}{AnsiCompareStr}
\Declaration
Function AnsiCompareStr(const S1, S2: string): integer;
\Description
\var{AnsiCompareStr} compares two strings and returns the following
result:
\begin{description}
\item[<0]  if \var{S1<S2}.
\item[0]  if \var{S1=S2}.
\item[>0] if \var{S1>S2}.
\end{description}
the comparision takes into account Ansi characters, i.e. it takes
care of strange accented characters. Contrary to \seef{AnsiCompareText},
the comparision is case sensitive.
\Errors
None.
\SeeAlso
\seef{AdjustLineBreaks}, \seef{AnsiCompareText}
\end{function}

\FPCexample{ex49}


\begin{function}{AnsiCompareText}
\Declaration
Function AnsiCompareText(const S1, S2: string): integer;
\Description
\Description
\var{AnsiCompareText} compares two strings and returns the following
result:
\begin{description}
\item[<0]  if \var{S1<S2}.
\item[0]  if \var{S1=S2}.
\item[>0] if \var{S1>S2}.
\end{description}
the comparision takes into account Ansi characters, i.e. it takes
care of strange accented characters. Contrary to \seef{AnsiCompareStr},
the comparision is case insensitive.
\Errors
None.
\SeeAlso
\seef{AdjustLineBreaks}, \seef{AnsiCompareText}
\end{function}

\FPCexample{ex50}


\begin{function}{AnsiExtractQuotedStr}
\Declaration
Function AnsiExtractQuotedStr(var Src: PChar; Quote: Char): string;
\Description
\var{AnsiExtractQuotedStr} Returns \var{Src} as a string, with \var{Quote}
characters removed from the beginning and end of the string, and double
\var{Quote} characters replaced by a single \var{Quote} characters.
As such, it revereses the action of \seef{AnsiQuotedStr}.
\Errors
None.
\SeeAlso
\seef{AnsiQuotedStr}
\end{function}

\FPCexample{ex51}


\begin{function}{AnsiLastChar}
\Declaration
Function AnsiLastChar(const S: string): PChar;
\Description
This function returns a pointer to the last character of \var{S}.
Since multibyte characters are not yet supported, this is the same
as \var{@S[Length(S)])}.
\Errors
None.
\SeeAlso
\seef{AnsiStrLastChar}
\end{function}

\FPCexample{ex52}


\begin{function}{AnsiLowerCase}
\Declaration
Function AnsiLowerCase(const s: string): string;
\Description
\var{AnsiLowerCase} converts the string \var{S} to lowercase characters
and returns the resulting string.
It takes into account the operating system language
settings when doing this, so spcial characters are converted correctly as
well.

{\em Remark} On linux, no language setting is taken in account yet.
\Errors
None.
\SeeAlso
\seef{AnsiUpperCase}, \seef{AnsiStrLower}, \seef{AnsiStrUpper}
\end{function}

\FPCexample{ex53}


\begin{function}{AnsiQuotedStr}
\Declaration
Function AnsiQuotedStr(const S: string; Quote: char): string;
\Description
\var{AnsiQuotedString} quotes the string \var{S} and returns the result.
This means that it puts the \var{Quote} character at both the beginning and
end of the string and replaces any occurrence of \var{Quote} in \var{S}
with 2 \var{Quote} characters. The action of \var{AnsiQuotedString} can be
reversed by \seef{AnsiExtractQuotedStr}.
\Errors
None.
\SeeAlso
\seef{AnsiExtractQuotedStr}
\end{function}

For an example, see \seef{AnsiExtractQuotedStr}

\begin{function}{AnsiStrComp}
\Declaration
Function AnsiStrComp(S1, S2: PChar): integer;
\Description
\var{AnsiStrComp} compares 2 \var{PChar} strings, and returns the following
result:
\begin{description}
\item[<0]  if \var{S1<S2}.
\item[0]  if \var{S1=S2}.
\item[>0]  if \var{S1>S2}.
\end{description}
The comparision of the two strings is case-sensitive.
The function does not yet take internationalization settings into account.
\Errors
None.
\SeeAlso
\seef{AnsiCompareText}, \seef{AnsiCompareStr}
\end{function}

\FPCexample{ex54}


\begin{function}{AnsiStrIComp}
\Declaration
Function AnsiStrIComp(S1, S2: PChar): integer;
\Description
\var{AnsiStrIComp} compares 2 \var{PChar} strings, and returns the following
result:
\begin{description}
\item[<0]  if \var{S1<S2}.
\item[0]  if \var{S1=S2}.
\item[>0]  if \var{S1>S2}.
\end{description}
The comparision of the two strings is case-insensitive.
The function does not yet take internationalization settings into account.
\Errors
None.
\SeeAlso
\seef{AnsiCompareText}, \seef{AnsiCompareStr}
\end{function}

\FPCexample{ex55}


\begin{function}{AnsiStrLastChar}
\Declaration
function AnsiStrLastChar(Str: PChar): PChar;
\Declaration
\var{AnsiStrLastChar} returns a pointer to the last character of \var{Str}.
Since multibyte characters are not yet supported, this is the same
as \var{StrEnd(Str)-1}.
\Errors
None.
\SeeAlso
\seef{AnsiLastChar}
\end{function}

\FPCexample{ex58}


\begin{function}{AnsiStrLComp}
\Declaration
Function AnsiStrLComp(S1, S2: PChar; MaxLen: cardinal): integer;
\Description
\var{AnsiStrLComp} compares the first \var{Maxlen} characters of
2 \var{PChar} strings, \var{S1} and \var{S2}, and returns the following
result:
\begin{description}
\item[<0]  if \var{S1<S2}.
\item[0]  if \var{S1=S2}.
\item[>0]  if \var{S1>S2}.
\end{description}
The comparision of the two strings is case-sensitive.
The function does not yet take internationalization settings into account.
\Errors
None.
\SeeAlso
\seef{AnsiCompareText}, \seef{AnsiCompareStr}
\end{function}

\FPCexample{ex56}


\begin{function}{AnsiStrLIComp}
\Declaration
Function AnsiStrLIComp(S1, S2: PChar; MaxLen: cardinal): integer;
\Description
\var{AnsiStrLIComp} compares the first \var{Maxlen} characters of
2 \var{PChar} strings, \var{S1} and \var{S2}, and returns the following
result:
\begin{description}
\item[<0]  if \var{S1<S2}.
\item[0]  if \var{S1=S2}.
\item[>0]  if \var{S1>S2}.
\end{description}
The comparision of the two strings is case-insensitive.
The function does not yet take internationalization settings into account.
\Errors
None.
\SeeAlso
\seef{AnsiCompareText}, \seef{AnsiCompareStr}
\end{function}

\FPCexample{ex57}




\begin{function}{AnsiStrLower}
\Declaration
Function AnsiStrLower(Str: PChar): PChar;
\Description
\var{AnsiStrLower} converts the PChar \var{Str} to lowercase characters
and returns the resulting pchar. Note that \var{Str} itself is modified,
not a copy, as in the case of \seef{AnsiLowerCase}.
It takes into account the operating system language
settings when doing this, so spcial characters are converted correctly as
well.

{\em Remark} On linux, no language setting is taken in account yet.
\Errors
None.
\SeeAlso
\seef{AnsiStrUpper}, \seef{AnsiLowerCase}
\end{function}

\FPCexample{ex59}


\begin{function}{AnsiStrUpper}
\Declaration
Function AnsiStrUpper(Str: PChar): PChar;
\Description
\var{AnsiStrUpper} converts the \var{PChar} \var{Str} to uppercase characters
and returns the resulting string. Note that \var{Str} itself is modified,
not a copy, as in the case of \seef{AnsiUpperCase}.
It takes into account the operating system language
settings when doing this, so spcial characters are converted correctly as
well.

{\em Remark} On linux, no language setting is taken in account yet.
\Errors
None.
\SeeAlso
\seef{AnsiUpperCase}, \seef{AnsiStrLower}, \seef{AnsiLowerCase}
\end{function}

\FPCexample{ex60}


\begin{function}{AnsiUpperCase}
\Declaration
Function AnsiUpperCase(const s: string): string;
\Description
\var{AnsiUpperCase} converts the string \var{S} to uppercase characters
and returns the resulting string.
It takes into account the operating system language
settings when doing this, so spcial characters are converted correctly as
well.

{\em Remark} On linux, no language setting is taken in account yet.
\Errors
None.
\SeeAlso
\seef{AnsiStrUpper}, \seef{AnsiStrLower}, \seef{AnsiLowerCase}
\end{function}

\FPCexample{ex61}


\begin{procedure}{AppendStr}
\Declaration
Procedure AppendStr(var Dest: String; const S: string);
\Description
\var{AppendStr} appends \var{S} to Dest.

This function is provided for Delphi
compatibility only, since it is completely equivalent to \var{Dest:=Dest+S}.
\Errors
None.
\SeeAlso
\seep{AssignStr},\seef{NewStr}, \seep{DisposeStr}
\end{procedure}

\FPCexample{ex62}


\begin{procedure}{AssignStr}
\Declaration
Procedure AssignStr(var P: PString; const S: string);
\Description
\var{AssignStr} allocates \var{S} to P. The old value of \var{P} is
disposed of.

This function is provided for Delphi compatibility only. \var{AnsiStrings}
are managed on the heap and should be preferred to the mechanism of
dynamically allocated strings.
\Errors
None.
\SeeAlso
\seef{NewStr}, \seep{AppendStr}, \seep{DisposeStr}
\end{procedure}

\FPCexample{ex63}


\begin{function}{BCDToInt}
\Declaration
Function BCDToInt(Value: integer): integer;
\Description
\var{BCDToInt} converts a \var{BCD} coded integer to a normal integer.
\Errors
None.
\SeeAlso
\seef{StrToInt}, \seef{IntToStr}
\end{function}

\FPCexample{ex64}



\begin{function}{CompareMem}
\Declaration
Function CompareMem(P1, P2: Pointer; Length: cardinal): integer;
\Description
\var{CompareMem} compares, byte by byte,  2 memory areas pointed
to by \var{P1} and \var{P2}, for a length of \var{L} bytes.

It returns the following values:
\begin{description}
\item[<0] if at some position the byte at \var{P1} is less than the byte at the
same postion at \var{P2}.
\item[0] if all \var{L} bytes are the same.
\item[3]
\end{description}
\Errors
\SeeAlso
\end{function}


\begin{function}{CompareStr}
\Declaration
Function CompareStr(const S1, S2: string): Integer;
\Description
\var{CompareStr} compares two strings, \var{S1} and \var{S2},
and returns the following
result:
\begin{description}
\item[<0]  if \var{S1<S2}.
\item[0]  if \var{S1=S2}.
\item[>0]  if \var{S1>S2}.
\end{description}
The comparision of the two strings is case-sensitive.
The function does not take internationalization settings into account, it
simply compares ASCII values.
\Errors
None.
\SeeAlso
\seef{AnsiCompareText}, \seef{AnsiCompareStr}, \seef{CompareText}
\end{function}

\FPCexample{ex65}


\begin{function}{CompareText}
\Declaration
Function CompareText(const S1, S2: string): integer;
\Description
\var{CompareText} compares two strings, \var{S1} and \var{S2},
and returns the following
result:
\begin{description}
\item[<0]  if \var{S1<S2}.
\item[0]  if \var{S1=S2}.
\item[>0]  if \var{S1>S2}.
\end{description}
The comparision of the two strings is case-insensitive.
The function does not take internationalization settings into account, it
simply compares ASCII values.
\Errors
None.
\SeeAlso
\seef{AnsiCompareText}, \seef{AnsiCompareStr}, \seef{CompareStr}
\end{function}

\FPCexample{ex66}




\begin{procedurel}{DisposeStr}{DisposeStrSys}
\Declaration
Procedure DisposeStr(S: PString);
\Description
\var{DisposeStr} removes the dynamically allocated string \var{S} from the
heap, and releases the occupied memory.

This function is provided for Delphi compatibility only. \var{AnsiStrings}
are managed on the heap and should be preferred to the mechanism of
dynamically allocated strings.
\Errors
None.
\SeeAlso
\seef{NewStr}, \seep{AppendStr}, \seep{AssignStr}
\end{procedurel}

For an example, see \seep{DisposeStr}.

\begin{function}{FloatToStr}
\Declaration
Function FloatToStr(Value: Extended): String;
\Description
\var{FloatToStr} converts the floating point variable \var{Value} to a
string representation.  It will choose the shortest possible notation of the
two following formats:
\begin{description}
\item[Fixed format] will represent the string in fixed notation,
\item[Decimal format] will represent the string in scientific notation.
\end{description}
(more information on these formats can be found in \seef{FloatToStrF})
\var{FloatToStr} is completely equivalent to a \var{FloatToStrF(Value, ffGeneral,
15, 0);} call.
\Errors
None.
\SeeAlso
\seef{FloatToStrF}, \seef{FormatFloat}, \seef{StrToFloat}
\end{function}

\FPCexample{ex67}


\begin{function}{FloatToStrF}
\Declaration
Function FloatToStrF(Value: Extended; format: TFloatFormat; Precision, Digits: Integer): String;
\Description
\var{FloatToStrF} converts the floating point number \var{value} to a string
representation, according to the settings of the parameters \var{Format},
\var{Precision} and \var{Digits}.

The meaning of the \var{Precision} and \var{Digits} parameter depends on the
\var{Format} parameter. The format is controlled mainly by the \var{Format}
parameter. It can have one of the following values:
\begin{description}
\item[ffcurrency] Money format. \var{Value} is converted to a string using
the global variables \var{CurrencyString}, \var{CurrencyFormat} and
\var{NegCurrencyFormat}. The \var{Digits} paramater specifies the number of digits
following the decimal point and should be in the range -1 to 18. If Digits
equals \var{-1}, \var{CurrencyDecimals} is assumed. The \var{Precision} parameter is ignored.
%
\item[ffExponent] Scientific format. \var{Value} is converted to a
string using scientific notation: 1 digit before the decimal point, possibly
preceded by a minus sign if \var{Value} is negative. The number of
digits after the decimal point is controlled by \var{Precision} and must lie
in the range 0 to 15.
%
\item[ffFixed] Fixed point format. \var{Value} is converted to a string
using fixed point notation. The result is composed of all digits of the
integer part of \var{Value}, preceded by a minus sign if \var{Value} is
negative. Following the integer part is \var{DecimalSeparator} and then the
fractional part of \var{Value}, rounded off to \var{Digits} numbers.
If the number is too large then the result will be in scientific notation.
%
\item[ffGeneral] General number format. The argument is converted to a
string using \var{ffExponent} or \var{ffFixed} format, depending on wich one
gives the shortest string. There will be no trailing zeroes. If \var{Value}
is less than \var{0.00001} or if the number of decimals left of the decimal
point is larger than \var{Precision} then scientific notation is used, and
\var{Digits} is the minimum number of digits in the exponent. Otherwise
\var{Digits} is ignored.
\item[ffnumber] Is the same as \var{ffFixed}, except that thousand separators
are inserted in the resultig string.
\end{description}
\Errors
None.
\SeeAlso
\seef{FloatToStr}, \seef{FloatToText}
\end{function}

\FPCexample{ex68}


\begin{function}{FloatToText}
\Declaration
Function FloatToText(Buffer : Pchar;Value: Extended; Format: TFloatFormat; Precision, Digits: Integer): Longint;
\Description
\var{FloatToText} converts the floating point variable \var{Value} to a
string representation and stores it in \var{Buffer}.  The conversion is
giverned by \var{format}, \var{Precisison} and \var{Digits}.
more information on these parameters can be found in \seef{FloatToStrF}.
\var{Buffer} should point to enough space to hold the result. No checking on
this is performed.

The result is the number of characters that was copied in \var{Buffer}.
\Errors
None.
\SeeAlso
\seef{FloatToStr}, \seef{FloatToStrF}
\end{function}

\FPCexample{ex69}


\begin{procedure}{FmtStr}
\Declaration
Procedure (Var Res: String; Const Fmt : String; Const args: Array of const);
\Description
\var{FmtStr} calls \seef{Format} with \var{Fmt} and \var{Args} as arguments,
and stores the result in \var{Res}. For more information on how the
resulting string is composed, see \seef{Format}.
\Errors
In case of error, a \var{EConvertError} exception is raised.
\SeeAlso
\seef{Format}, \seef{FormatBuf}.
\end{procedure}

\FPCexample{ex70}


\begin{function}{Format}
\Declaration
Function Format(Const Fmt : String; const Args : Array of const) : String;
\Description
Format replaces all placeholders in\var{Fmt} with the arguments passed in
\var{Args} and returns the resulting string. A placeholder looks as follows:
\begin{verbatim}
'%' [Index':'] ['-'] [Width] ['.' Precision] ArgType
\end{verbatim}
elements between single quotes must be typed as shown without the quotes,
and elements between square brackets \var{[ ]} are optional. The meaning
of the different elements is shown below:
\begin{description}
\item['\%'] starts the placeholder. If you want to insert a literal
\var{\%} character, then you must insert two of them : \var{\%\%}.
\item[Index ':'] takes the \var{Index}-th element in the argument array
as the element to insert.
\item['-'] tells \var{Format} to left-align the inserted text. The default
behaviour is to right-align inserted text. This can only take effect if the
\var{Width} element is also specified.
\item[Width] the inserted string must have at least have \var{Width}
characters. If not, the inserted string will be padded with spaces. By
default, the string is left-padded, resulting in a right-aligned string.
This behaviour can be changed by the \var{'-'} character.
\item['.' Precision] Indicates the precision to be used when converting
the argument. The exact meaning of this parameter depends on \var{ArgType}.
\end{description}
The \var{Index}, \var{Width} and \var{Precision} parameters can be replaced
by \var{*}, in which case their value will be read from the next element in
the \var{Args} array. This value must be an integer, or an
\var{EConvertError} exception will be raised.

The argument type is determined from \var{ArgType}. It can have one of the
following values (case insensitive):
\begin{description}
\item[D] Decimal format. The next argument in the \var{Args} array should be
an integer. The argument is converted to a decimal string,. If precision is
specified, then the string will have at least \var{Precision} digits in it.
If needed, the string is (left) padded with zeroes.
\item[E] scientific format. The next argument in the \var{Args} array should
be a Floating point value. The argument is converted to a decimal string
using scientific notation, using \seef{FloatToStrF}, where the optional
precision is used to specify the total number of decimals. (defalt a valueof
15 is used). The exponent is formatted using maximally 3 digits.

In short, the \var{E} specifier formats it's arguument as follows:
\begin{verbatim}
FloatToStrF(Argument,ffexponent,Precision,3)
\end{verbatim}

\item[F] fixed point format. The next argument in the \var{Args} array
should be a floating point value. The argument is converted to a
decimal string, using fixed notation (see \seef{FloatToStrF}).
\var{Precision} indicates the number of digits following the
decimal point.

In short, the \var{F} specifier formats it's arguument as follows:
\begin{verbatim}
FloatToStrF(Argument,ffFixed,ffixed,9999,Precision)
\end{verbatim}

\item[G] General number format. The next argument in the \var{Args} array
should be a floating point value. The argument is converted to a decimal
string using fixed point notation or scientific notation, depending on which
gives the shortest result. \var{Precision} is used to determine the number
of digits after the decimal point.

In short, the \var{G} specifier formats it's arguument as follows:
\begin{verbatim}
FloatToStrF(Argument,ffGeneral,Precision,3)
\end{verbatim}

\item[M] Currency format. the next argument in the var{Args} array must
be a floating point value. The argument is converted to a decimal string
using currency notation. This means that fixed-point notation is used, but
that the currency symbol is appended. If precision is specified, then
then it overrides the \var{CurrencyDecimals} global variable used in the
\seef{FloatToStrF}

In short, the \var{M} specifier formats it's arguument as follows:
\begin{verbatim}
FloatToStrF(Argument,ffCurrency,9999,Precision)
\end{verbatim}

\item[N] Number format. This is the same as fixed point format, except that
thousand separators are inserted in the resulting string.

\item[P] Pointer format. The next argument in the \var{Args} array must be a
pointer (typed or untyped). The pointer value is converted to a string of
length 8, representing the hexadecimal value of the pointer.

\item[S] String format. The next argument in the \var{Args} array must be
a string. The argument is simply copied to the result string. If
\var{Precision} is specified, then only \var{Precision} characters are
copied to the result string.

\item[X] hexadecimal format. The next argument in the \var{Args} array must
be an integer. The argument is converted to a hexadecimal string with just
enough characters to contain the value of the integer. If \var{Precision}
is specified then the resulting hexadecimal representation will have at
least \var{Precision} characters in it (with a maximum value of 32).
\end{description}
\Errors
In case of error, an \var{EConversionError} exception is raised. Possible
errors are:
\begin{enumerate}
\item Errors in the format specifiers.
\item The next argument is not of the type needed by a specifier.
\item The number of arguments is not sufficient for all format specifiers.
\end{enumerate}
\SeeAlso
\seef{FormatBuf}
\end{function}

\FPCexample{ex71}


\begin{function}{FormatBuf}
\Declaration
Function FormatBuf(Var Buffer; BufLen : Cardinal; Const Fmt; fmtLen : Cardinal; Const Args : Array of const) : Cardinal;
\Description
\var{Format}
\Errors
\SeeAlso
\end{function}

\FPCexample{ex72}

\begin{function}{FormatFloat}
\Declaration
Function FormatFloat(Const format: String; Value: Extended): String;
\Description
FormatFloat formats the floating-point value given by \var{Value} using 
the format specifications in \var{Format}. The format specifier can give
format specifications for positive, negative or zero values (separated by 
a semicolon).


If the formatspecifier is empty or the value needs more than 18 digits to
be correctly represented, the result is formatted with a call to 
\seef{FloatToStrF} with the \var{ffGeneral} format option.

The following format specifiers are supported:
\begin{description}
\item[0] is a digit place holder. If there is a corresponding digit in 
the value being formatted, then it replaces the 0. If not, the 0 is left
as-is.
\item[\#] is also a digit place holder. If there is a corresponding digit in
the value being formatted, then it replaces the \#. If not, it is removed.
by a space.
\item[.] determines the location of the decimal point. Only the first '.'
character is taken into account. If the value contains digits after the
decimal point, then it is replaced by the value of the \var{DecimalSeparator}
character.
\item[,] determines the use of the thousand separator character in the
output string. If the format string contains one or more ',' charactes, 
then thousand separators will be used. The \var{ThousandSeparator} character
is used.
\item[E+] determines the use of scientific notation. If 'E+' or 'E-' (or
their lowercase counterparts) are present then scientific notation is used.
The number of digits in the output string is determined by the number of
\var{0} characters after the '\var{E+}'
\item[;] This character separates sections for positive, negative, and zero numbers in the
format string.	
\end{description}
\Errors
If an error occurs, an exception is raised.
\SeeAlso
\seef{FloatToStr}
\end{function}

\FPCexample{ex89}

\begin{function}{IntToHex}
\Declaration
Function IntToHex(Value: integer; Digits: integer): string;
\Description
\var{IntToHex} converts \var{Value} to a hexadecimal string
representation. The result will contain at least \var{Digits}
characters. If \var{Digits} is less than the needed number of characters,
the string will NOT be truncated. If \var{Digits} is larger than the needed
number of characters, the result is padded with zeroes.
\Errors
None.
\SeeAlso
\seef{IntToStr}, \var{StrToInt}
\end{function}

\FPCexample{ex73}


\begin{function}{IntToStr}
\Declaration
Function IntToStr(Value: integer): string;
\Description
\var{IntToStr} coverts \var{Value} to it's string representation.
The resulting string has only as much characters as needed to represent
the value. If the value is negative a minus sign is prepended to the
string.
\Errors
None.
\SeeAlso
\seef{IntToHex}, \seef{StrToInt}
\end{function}

\FPCexample{ex74}


\begin{function}{IsValidIdent}
\Declaration
Function IsValidIdent(const Ident: string): boolean;
\Description
\var{IsValidIdent} returns \var{True} if \var{Ident} can be used as a
compoent name. It returns \var{False} otherwise. \var{Ident} must consist of
a letter or underscore, followed by a combination of letters, numbers or
underscores to be a valid identifier.
\Errors
None.
\SeeAlso
\end{function}

\FPCexample{ex75}


\begin{function}{LastDelimiter}
\Declaration
Function LastDelimiter(const Delimiters, S: string): Integer;
\Description
\var{LastDelimiter} returns the {\em last} occurrence of any character in
the set \var{Delimiters} in the string \var{S}.
\Errors
\SeeAlso
\end{function}

\FPCexample{ex88}


\begin{function}{LeftStr}
\Declaration
Function LeftStr(const S: string; Count: integer): string;
\Description
\var{LeftStr} returns the \var{Count} leftmost characters of \var{S}.
It is equivalent to a call to \var{Copy(S,1,Count)}.
\Errors
None.
\SeeAlso
\seef{RightStr}, \seef{TrimLeft}, \seef{TrimRight}, \seef{Trim}
\end{function}

 \FPCexample{ex76}


\begin{function}{LoadStr}
\Declaration
Function LoadStr(Ident: integer): string;
\Description
This function is not yet implemented. resources are not yet supported.
\Errors
\SeeAlso
\end{function}

\begin{function}{LowerCase}
\Declaration
Function LowerCase(const s: string): string;
\Description
\var{LowerCase} returns the lowercase equivalent of \var{S}. Ansi characters
are not taken into account, only ASCII codes below 127 are converted. It is
completely equivalent to the lowercase function of the system unit, and is
provided for compatiibility only.
\Errors
None.
\SeeAlso
\seef{AnsiLowerCase}, \seef{UpperCase}, \seef{AnsiUpperCase}
\end{function}

\FPCexample{ex77}


\begin{functionl}{NewStr}{NewStrSys}
\Declaration
Function NewStr(const S: string): PString;
\Description
\var{NewStr} assigns a new dynamic string on the heap, copies \var{S} into
it, and returns a pointer to the newly assigned string.

This function is obsolete, and shouldn't be used any more. The
\var{AnsiString} mechanism also allocates ansistrings on the heap, and
should be preferred over this mechanism.
\Errors
If not enough memory is present, an EOutOfMemory exception will be raised.
\SeeAlso
\seep{AssignStr}, \seepl{DisposeStr}{DisposeStrSys}
\end{functionl}

For an example, see \seep{AssignStr}.

\begin{function}{QuotedStr}
\Declaration
Function QuotedStr(const S: string): string;
\Description
\var{QuotedStr} returns the string \var{S}, quoted with single quotes. This means
that \var{S} is enclosed in single quotes, and every single quote in \var{S}
is doubled. It is equivalent to a call to \var{AnsiQuotedStr(s, '''')}.
\Errors
None.
\SeeAlso
\seef{AnsiQuotedStr}, \seef{AnsiExtractQuotedStr}.
\end{function}

\FPCexample{ex78}



\begin{function}{RightStr}
\Declaration
Function RightStr(const S: string; Count: integer): string;
\Description
\var{RightStr} returns the \var{Count} rightmost characters of \var{S}.
It is equivalent to a call to \var{Copy(S,Length(S)+1-Count,Count)}.

If \var{Count} is larger than the actual length of \var{S} only the real
length will be used.
\Errors
None.
\SeeAlso
\seef{LeftStr},\seef{Trim}, \seef{TrimLeft}, \seef{TrimRight}
\end{function}

\FPCexample{ex79}


\begin{function}{StrFmt}
\Declaration
Function StrFmt(Buffer,Fmt : PChar; Const args: Array of const) : Pchar;
\Description
\var{StrFmt} will format \var{fmt} with \var{Args}, as the \seef{Format}
function does, and it will store the result in \var{Buffer}. The function
returns \var{Buffer}. \var{Buffer} should point to enough space to contain
the whole result.
\Errors
for a list of errors, see \seef{Format}.
\SeeAlso
\seef{StrLFmt}, \seep{FmtStr}, \seef{Format}, \seef{FormatBuf}
\end{function}

\FPCexample{ex80}


\begin{function}{StrLFmt}
\Declaration
Function StrLFmt(Buffer : PCHar; Maxlen : Cardinal;Fmt : PChar; Const args: Array of const) : Pchar;
\Description
\var{StrLFmt} will format \var{fmt} with \var{Args}, as the \seef{Format}
function does, and it will store maximally \var{Maxlen characters} of the
result in \var{Buffer}. The function returns \var{Buffer}. \var{Buffer}
should point to enough space to contain \var{MaxLen} characters.
\Errors
for a list of errors, see \seef{Format}.
\SeeAlso
\seef{StrFmt}, \seep{FmtStr}, \seef{Format}, \seef{FormatBuf}
\end{function}

\FPCexample{ex81}

\begin{function}{StrToFloat}
\Declaration
Function StrToFloat(Const S : String) : Extended;
\Description
\var{StrToFloat} converts the string \var{S} to a floating point value.
\var{S} should contain a valid stroing representation of a floating point 
value (either in decimal or scientific notation). If the string
contains a decimal value, then the decimal separator character can either be
a '.' or the value of the \var{DecimalSeparator} variable.
\Errors
If the string \var{S} doesn't contain a valid floating point string, then an
exception will be raised.
\SeeAlso
\seef{TextToFloat},\seef{FloatToStr},\seef{FormatFloat},\seef{StrToInt}
\end{function}

\FPCexample{ex90}


\begin{function}{StrToInt}
\Declaration
Function StrToInt(const s: string): integer;
\Description
\var{StrToInt} will convert the string \var{S}to an integer.
If the string contains invalid characters or has an invalid format,
then an \var{EConvertError} is raised.

To be successfully converted, a string can contain a combination
of \var{numerical} characters, possibly preceded by a minus sign (\var{-}).
Spaces are not allowed.
\Errors
In case of error, an \var{EConvertError} is raised.
\SeeAlso
\seef{IntToStr}, \seef{StrToIntDef}
\end{function}

\FPCexample{ex82}


\begin{function}{StrToIntDef}
\Declaration
Function StrToIntDef(const S: string; Default: integer): integer;
\Description
\var{StrToIntDef} will convert a string to an integer. If the string contains
invalid characters or has an invalid format, then \var{Default} is returned.

To be successfully converted, a string can contain a combination of
\var{numerical} characters, possibly preceded by a minus sign (\var{-}).
Spaces are not allowed.
\Errors
None.
\SeeAlso
\seef{IntToStr}, \seef{StrToInt}
\end{function}

\FPCexample{ex83}

\begin{function}{TextToFloat}
\Declaration
Function TextToFloat(Buffer: PChar; Var Value: Extended): Boolean;
\Description
\var{TextToFloat} converts the string in \var{Buffer} to a floating point 
value. \var{Buffer} should contain a valid stroing representation of a 
floating point value (either in decimal or scientific notation). 
If the buffer contains a decimal value, then the decimal separator 
character can either be a '.' or the value of the \var{DecimalSeparator} 
variable.

The function returns \var{True} if the conversion was successful.
\Errors
If there is an invalid character in the buffer, then the function returns
\var{False}
\SeeAlso
\seef{StrToFloat},\seef{FloatToStr}, \seef{FormatFloat}
\end{function}

\FPCexample{ex91}

\begin{function}{Trim}
\Declaration
Function Trim(const S: string): string;
\Description
\var{Trim} strips blank characters (spaces) at the beginning and end of \var{S}
and returns the resulting string. Only \var{\#32} characters are stripped.

If the string contains only spaces, an empty string is returned.
\Errors
None.
\SeeAlso
\seef{TrimLeft}, \seef{TrimRight}
\end{function}

\FPCexample{ex84}


\begin{function}{TrimLeft}
\Declaration
Function TrimLeft(const S: string): string;
\Description
\var{TrimLeft} strips blank characters (spaces) at the beginning of \var{S}
and returns the resulting string. Only \var{\#32} characters are stripped.

If the string contains only spaces, an empty string is returned.
\Errors
None.
\SeeAlso
\seef{Trim}, \seef{TrimRight}
\end{function}

\FPCexample{ex85}


\begin{function}{TrimRight}
\Declaration
Function TrimRight(const S: string): string;
\Description
\var{Trim} strips blank characters (spaces) at the end of \var{S}
and returns the resulting string. Only \var{\#32} characters are stripped.

If the string contains only spaces, an empty string is returned.
\Errors
None.
\SeeAlso
\seef{Trim}, \seef{TrimLeft}
\end{function}

\FPCexample{ex86}



\begin{function}{UpperCase}
\Declaration
Function UpperCase(const s: string): string;
\Description
\var{UpperCase} returns the uppercase equivalent of \var{S}. Ansi characters
are not taken into account, only ASCII codes below 127 are converted. It is
completely equivalent to the \var{UpCase} function of the system unit, and is
provided for compatiibility only.
\Errors
None.
\SeeAlso
\seef{AnsiLowerCase}, \seef{LowerCase}, \seef{AnsiUpperCase}
\Errors
\SeeAlso
\end{function}

\FPCexample{ex87}



% The typinfo unit
%
%   $Id$
%   This file is part of the FPC documentation.
%   Copyright (C) 2001, by Michael Van Canneyt
%
%   The FPC documentation is free text; you can redistribute it and/or
%   modify it under the terms of the GNU Library General Public License as
%   published by the Free Software Foundation; either version 2 of the
%   License, or (at your option) any later version.
%
%   The FPC Documentation is distributed in the hope that it will be useful,
%   but WITHOUT ANY WARRANTY; without even the implied warranty of
%   MERCHANTABILITY or FITNESS FOR A PARTICULAR PURPOSE.  See the GNU
%   Library General Public License for more details.
%
%   You should have received a copy of the GNU Library General Public
%   License along with the FPC documentation; see the file COPYING.LIB.  If not,
%   write to the Free Software Foundation, Inc., 59 Temple Place - Suite 330,
%   Boston, MA 02111-1307, USA.
%
%%%%%%%%%%%%%%%%%%%%%%%%%%%%%%%%%%%%%%%%%%%%%%%%%%%%%%%%%%%%%%%%%%%%%%%
%%%%%%%%%%%%%%%%%%%%%%%%%%%%%%%%%%%%%%%%%%%%%%%%%%%%%%%%%%%%%%%%%%%%%%%
% The TYPINFO unit
%%%%%%%%%%%%%%%%%%%%%%%%%%%%%%%%%%%%%%%%%%%%%%%%%%%%%%%%%%%%%%%%%%%%%%%
\chapter{The TYPINFO unit}
\FPCexampledir{typinfex}
The \file{TypeInfo} unit contains many routines which can be used for
the querying of the Run-Time Type Information (RTTI) which is generated
by the compiler for classes that are compiled under the \var{\{\$M+\}}
switch. This information can be used to retrieve or set property values
for published properties for totally unknown classes. In particular, it
can be used to stream classes. The \var{TPersistent} class in the 
\file{Classes} unit is compiled in the \var{\{\$M+\}} state and serves
as the base class for all classes that need to be streamed.

The unit should be compatible to the Delphi 5 unit with the same name. 
The only calls that are still missing are the Variant calls, since \fpc
does not support the variant type yet.

The examples in this chapter use a \file{rttiobj} file, which contains
an object that has a published property of all supported types. It also
contains some auxiliary routines and definitions.
%%%%%%%%%%%%%%%%%%%%%%%%%%%%%%%%%%%%%%%%%%%%%%%%%%%%%%%%%%%%%%%%%%%%%%%
% Constants, Types and variables
\section{Constants, Types and variables}
\subsection{Constants}
The following constants are used in the implementation section of the unit.

\begin{verbatim}
BooleanIdents: array[Boolean] of String = ('False', 'True');
DotSep: String = '.';
\end{verbatim}
The following constants determine the access method for the \var{Stored} 
identifier of a property as used in the \var{PropProcs} field of the 
\var{TPropInfo} record:
\begin{verbatim}
ptField = 0;
ptStatic = 1;
ptVirtual = 2;
ptConst = 3;
\end{verbatim}
The following typed constants are used for easy selection of property types.
\begin{verbatim}
tkAny = [Low(TTypeKind)..High(TTypeKind)];
tkMethods = [tkMethod];
tkProperties = tkAny-tkMethods-[tkUnknown];
\end{verbatim}

\subsection{types}
The following pointer types are defined:
\begin{verbatim}
PShortString =^ShortString;
PByte        =^Byte;
PWord        =^Word;
PLongint     =^Longint;
PBoolean     =^Boolean;
PSingle      =^Single;
PDouble      =^Double;
PExtended    =^Extended;
PComp        =^Comp;
PFixed16     =^Fixed16;
Variant      = Pointer;
\end{verbatim}

The \var{TTypeKind} determines the type of a property:
\begin{verbatim}
TTypeKind = (tkUnknown,tkInteger,tkChar,tkEnumeration,
             tkFloat,tkSet,tkMethod,tkSString,tkLString,tkAString,
             tkWString,tkVariant,tkArray,tkRecord,tkInterface,
             tkClass,tkObject,tkWChar,tkBool,tkInt64,tkQWord,
             tkDynArray,tkInterfaceRaw);
tkString = tkSString;
\end{verbatim}
\var{tkString} is an alias that is introduced for Delphi compatibility.

If the property is and ordinal type, then \var{TTOrdType} determines the 
size and sign of the ordinal type:
\begin{verbatim}
TTOrdType = (otSByte,otUByte,otSWord,otUWord,otSLong,otULong);
\end{verbatim}
The size of a float type is determined by \var{TFloatType}:
\begin{verbatim}
TFloatType = (ftSingle,ftDouble,ftExtended,ftComp,ftCurr,
              ftFixed16,ftFixed32);
\end{verbatim}
A method property (e.g. an event) can have one of several types:
\begin{verbatim}
TMethodKind = (mkProcedure,mkFunction,mkConstructor,mkDestructor,
               mkClassProcedure, mkClassFunction);
\end{verbatim}
The kind of parameter to a method is determined by \var{TParamFlags}:
\begin{verbatim}
TParamFlags = set of (pfVar,pfConst,pfArray,pfAddress,pfReference,pfOut);
\end{verbatim}
Interfaces are described further with \var{TntfFlags}:
\begin{verbatim}
TIntfFlags = set of (ifHasGuid,ifDispInterface,ifDispatch);
\end{verbatim}
The following defines a set of \var{TTypeKind}:
\begin{verbatim}
TTypeKinds = set of TTypeKind;
\end{verbatim}
The \var{TypeInfo} function returns a pointer to a \var{TTypeInfo} record:
\begin{verbatim}
TTypeInfo = record
  Kind : TTypeKind;
  Name : ShortString;
end;
PTypeInfo = ^TTypeInfo;
PPTypeInfo = ^PTypeInfo;
\end{verbatim}
Note that the Name is stored with as much bytes as needed to store the name,
it is not padded to 255 characters. 
The type data immediatly follows the \var{TTypeInfo} record as a \var{TTypeData} record:
\begin{verbatim}
PTypeData = ^TTypeData;
TTypeData = packed record
case TTypeKind of
  tkUnKnown,tkLString,tkWString,tkAString,tkVariant:
    ();
  tkInteger,tkChar,tkEnumeration,tkWChar:
    (OrdType : TTOrdType;
     case TTypeKind of
       tkInteger,tkChar,tkEnumeration,tkBool,tkWChar : (
         MinValue,MaxValue : Longint;
         case TTypeKind of
           tkEnumeration: (
             BaseType : PTypeInfo;
             NameList : ShortString
           )
         );
       tkSet: (
             CompType : PTypeInfo
         )
     );
  tkFloat: (
    FloatType : TFloatType
    );
  tkSString:
    (MaxLength : Byte);
  tkClass:
    (ClassType : TClass;
     ParentInfo : PTypeInfo;
     PropCount : SmallInt;
     UnitName : ShortString
     );
  tkMethod:
    (MethodKind : TMethodKind;
     ParamCount : Byte;
     ParamList : array[0..1023] of Char
     {in reality ParamList is a array[1..ParamCount] of:
	  record
	    Flags : TParamFlags;
	    ParamName : ShortString;
	    TypeName : ShortString;
	  end;
	followed by
	  ResultType : ShortString}
    );
  tkInt64:
   (MinInt64Value, MaxInt64Value: Int64);
  tkQWord:
   (MinQWordValue, MaxQWordValue: QWord);
  tkInterface:
	();
end;
\end{verbatim}
If the typeinfo kind is \var{tkClass}, then the property 
information follows the \var{UnitName} string, as an array of \var{TPropInfo} records.

The \var{TPropData} record is not used, but is provided for completeness and
compatibility with Delphi.
\begin{verbatim}
TPropData = packed record
  PropCount : Word;
  PropList : record end;
end;
\end{verbatim}
The \var{TPropInfo} record describes one published property of a class:
\begin{verbatim}
PPropInfo = ^TPropInfo;
TPropInfo = packed record
  PropType : PTypeInfo;
  GetProc : Pointer;
  SetProc : Pointer;
  StoredProc : Pointer;
  Index : Integer;
  Default : Longint;
  NameIndex : SmallInt;
  PropProcs : Byte;
  Name : ShortString;
end;
\end{verbatim}
The \var{Name} field is stored not with 255 characters, but with just as many characters
as required to store the name.
\begin{verbatim}
TProcInfoProc = procedure(PropInfo : PPropInfo) of object;
\end{verbatim}
The following pointer and array types are used for typecasts:
\begin{verbatim}
PPropList = ^TPropList;
TPropList = array[0..65535] of PPropInfo;
\end{verbatim}

%%%%%%%%%%%%%%%%%%%%%%%%%%%%%%%%%%%%%%%%%%%%%%%%%%%%%%%%%%%%%%%%%%%%%%%
% Functions and procedures by category
\section{Function list by category}
What follows is a listing of the available functions, grouped by category.
For each function there is a reference to the page where the function
can be found.

\subsection{Examining published property information}
Functions for retrieving or examining property information
\begin{funclist}
\funcref{FindPropInfo}{Getting property type information, With error checking.}
\funcref{GetPropInfo}{Getting property type information, No error checking.}
\funcref{GetPropInfos}{Find property information of a certain kind}
\funcref{GetObjectPropClass}{Return the declared class of an object property }
\funcref{GetPropList}{Get a list of all published properties}
\funcref{IsPublishedProp}{Is a property published}
\funcref{IsStoredProp}{Is a property stored}
\funcref{PropIsType}{Is a property of a certain kind}
\funcref{PropType}{Return the type of a property}
\end{funclist}

\subsection{Getting or setting property values}
Functions to set or set a property's value.
\begin{funclist}
\funcref{GetEnumProp}{Return the value of an enumerated type property}
\funcref{GetFloatProp}{Return the value of a float property}
\funcref{GetInt64Prop}{Return the value of an Int64 property}
\funcref{GetMethodProp}{Return the value of a procedural type property}
\funcref{GetObjectProp}{Return the value of an object property}
\funcref{GetOrdProp}{Return the value of an ordinal type property}
\funcref{GetPropValue}{Return the value of a property as a variant}
\funcref{GetSetProp}{Return the value of a set property}
\funcref{GetStrProp}{Return the value of a string property}
\funcref{GetVariantProp}{Return the value of a variant property}
\funcref{SetEnumProp}{Set the value of an enumerated type property}
\funcref{SetFloatProp}{Set the value of a float property}
\funcref{SetInt64Prop}{Set the value of an Int64 property}
\funcref{SetMethodProp}{Set the value of a procedural type property}
\funcref{SetObjectProp}{Set the value of an object property}
\funcref{SetOrdProp}{Set the value of an ordinal type property}
\funcref{SetPropValue}{Set the value of a property trhough a variant}
\funcref{SetSetProp}{Set the value of a set property}
\funcref{SetStrProp}{Set the value of a string property}
\funcref{SetVariantProp}{Set the value of a variant property}
\end{funclist}

\subsection{Auxiliary functions}
\begin{funclist}
\funcref{GetEnumName}{Get an enumerated type element name}
\funcref{GetEnumValue}{Get ordinal number of an enumerated tye, based on the
name.}
\funcref{GetTypeData}{Skip type name and return a pointer to the type data}
\funcref{SetToString}{Convert a set to its string representation}
\funcref{StringToSet}{Convert a string representation of a set to a set}
\end{funclist}

%%%%%%%%%%%%%%%%%%%%%%%%%%%%%%%%%%%%%%%%%%%%%%%%%%%%%%%%%%%%%%%%%%%%%%%
% Functions and procedures
\section{Functions and Procedures}

\begin{function}{FindPropInfo}
\Declaration
Function FindPropInfo(AClass:TClass;const PropName: string): PPropInfo;\\
Function FindPropInfo(Instance: TObject; const PropName: string): PPropInfo;   
\Description
\var{FindPropInfo} examines the published property information of a class and
returns a pointer to the property information for property \var{PropName}.
The class to be examined can be specified in one of two ways:
\begin{description}
\item[AClass] a class pointer.
\item[Instance] an instance of the class to be investigated.
\end{description}
If the property does not exist, a \var{EPropertyError} exception will be
raised. The \seef{GetPropInfo} function has the same function as the
\var{FindPropInfo} function, but returns \var{Nil} if the property does not
exist.
\Errors
Specifying an invalid property name in \var{PropName} will result in an
\var{EPropertyError} exception.
\SeeAlso
\seef{GetPropInfo}, \seef{GetPropList}, \seep{GetPropInfos}
\end{function}

\FPCexample{ex14}

\begin{function}{GetEnumName}
\Declaration
Function GetEnumName(TypeInfo : PTypeInfo;Value : Integer) : string;
\Description
\var{GetEnumName} scans the type information for the enumeration type
described by \var{TypeInfo} and returns the name of the enumeration 
constant for the element with ordinal value equal to \var{Value}.

If \var{Value} is out of range, the first element of the enumeration type
is returned. The result is lowercased, but this may change in the future.

This can be used in combination with \var{GetOrdProp} to stream a property
of an enumerated type.
\Errors
No check is done to determine whether \var{TypeInfo} really points to the 
type information for an enumerated type. 
\SeeAlso
\seef{GetOrdProp}, \seef{GetEnumValue}
\end{function}

\FPCexample{ex9}

\begin{function}{GetEnumProp}
\Declaration
Function GetEnumProp(Instance: TObject;const PropInfo: PPropInfo): string;\\
Function GetEnumProp(Instance: TObject;const PropName: string): string;       
\Description
\var{GetEnumProp} returns the value of an property of an enumerated type
and returns the name of the enumerated value for the objetc \var{Instance}. 
The property whose value must be returned can be specified by its property 
info in \var{PropInfo} or by its name in \var{PropName}
\Errors
No check is done to determine whether \var{PropInfo} really points to the 
property information for an enumerated type. 
Specifying an invalid property name in \var{PropName} will result in an
\var{EPropertyError} exception.
\SeeAlso
\seep{SetEnumProp} \seef{GetOrdProp}, \seef{GetStrProp},
\seef{GetInt64Prop},\seef{GetMethodProp}, \seef{GetSetProp},
\seef{GetObjectProp}, \seef{GetEnumProp}
\end{function}

\FPCexample{ex2}

\begin{function}{GetEnumValue}
\Declaration
Function GetEnumValue(TypeInfo : PTypeInfo;const Name : string) : Integer;
\Description
\var{GetEnumValue} scans the type information for the enumeration type
described by \var{TypeInfor} and returns the ordinal value for the element
in the enumerated type that has identifier \var{Name}. The identifier is
searched in a case-insensitive manner.

This can be used to set the value of enumerated properties from a stream. 
\Errors
If \var{Name} is not found in the list of enumerated values, then -1 is
returned. No check is done whether \var{TypeInfo} points to the type information
for an enumerated type. 
\SeeAlso
\seef{GetEnumName}, \seep{SetOrdProp}
\end{function}

For an example, see \seef{GetEnumName}.

\begin{function}{GetFloatProp}
\Declaration
Function GetFloatProp(Instance : TObject;PropInfo : PPropInfo) : Extended;\\
Procedure SetFloatProp(Instance: TObject; const PropName: string; Value: Extended);
\Description
\var{GetFloatProp} returns the value of the float property described by 
\var{PropInfo} or with name \var{Propname} for the object \var{Instance}. 
All float types are converted
to extended.
\Errors
No checking is done whether \var{Instance} is non-nil, or whether
\var{PropInfo} describes a valid float property of \var{Instance}.
Specifying an invalid property name in \var{PropName} will result in an
\var{EPropertyError} exception.
\SeeAlso
\seep{SetFloatProp}, \seef{GetOrdProp}, \seef{GetStrProp},
\seef{GetInt64Prop},\seef{GetMethodProp}, \seef{GetSetProp},
\seef{GetObjectProp}, \seef{GetEnumProp}
\end{function}

\FPCexample{ex4}

\begin{function}{GetInt64Prop}
\Declaration
Function GetInt64Prop(Instance: TObject; PropInfo: PPropInfo): Int64;\\
Function GetInt64Prop(Instance: TObject; const PropName: string): Int64;
\Description
{\em Publishing of Int64 properties is not yet supported by \fpc. This
function is provided for Delphi compatibility only at the moment.}

\var{GetInt64Prop} returns the value of the property of type
\var{Int64} that is described by \var{PropInfo} or with name \var{Propname} 
for the object \var{Instance}.

\Errors
No checking is done whether \var{Instance} is non-nil, or whether
\var{PropInfo} describes a valid \var{Int64} property of \var{Instance}.
Specifying an invalid property name in \var{PropName} will result in an
\var{EPropertyError} exception
\SeeAlso
\seep{SetInt64Prop}, \seef{GetOrdProp}, \seef{GetStrProp},
\seef{GetFloatProp}, \seef{GetMethodProp}, \seef{GetSetProp},
\seef{GetObjectProp}, \seef{GetEnumProp}
\end{function}

\FPCexample{ex15}

\begin{function}{GetMethodProp}
\Declaration
Function GetMethodProp(Instance : TObject;PropInfo : PPropInfo) : TMethod;\\
Function GetMethodProp(Instance: TObject; const PropName: string): TMethod;
\Description
\var{GetMethodProp} returns the method the property described by
\var{PropInfo} or with name \var{Propname} for object \var{Instance}.
The return type \var{TMethod} is defined in the \file{SysUtils} unit as:
\begin{verbatim}
TMethod = packed record
  Code, Data: Pointer;
end;                                                                         
\end{verbatim}
\var{Data} points to the instance of the class with the method \var{Code}.

\Errors
No checking is done whether \var{Instance} is non-nil, or whether
\var{PropInfo} describes a valid method property of \var{Instance}.
Specifying an invalid property name in \var{PropName} will result in an
\var{EPropertyError} exception
\SeeAlso
\seep{SetMethodProp}, \seef{GetOrdProp}, \seef{GetStrProp},
\seef{GetFloatProp}, \seef{GetInt64Prop}, \seef{GetSetProp},
\seef{GetObjectProp}, \seef{GetEnumProp}
\end{function}

\FPCexample{ex6}

\begin{function}{GetObjectProp}
\Declaration
Function GetObjectProp(Instance: TObject; const PropName: string): TObject;\\
Function GetObjectProp(Instance: TObject; const PropName: string; MinClass:TClass): TObject; \\
Function GetObjectProp(Instance: TObject; PropInfo: PPropInfo; MinClass: TClass):
TObject;\\
\Description
\var{GetObjectProp} returns the object which the property descroibed by
\var{PropInfo} with name \var{Propname} points to for object \var{Instance}.

If \var{MinClass} is specified, then if the object is not descendent of
class \var{MinClass}, then \var{Nil} is returned.

\Errors
No checking is done whether \var{Instance} is non-nil, or whether
\var{PropInfo} describes a valid method property of \var{Instance}. 
Specifying an invalid property name in \var{PropName} will result in an
\var{EPropertyError} exception.
\SeeAlso
\seep{SetMethodProp}, \seef{GetOrdProp}, \seef{GetStrProp},
\seef{GetFloatProp}, \seef{GetInt64Prop}, \seef{GetSetProp},
\seef{GetObjectProp}, \seef{GetEnumProp}
\end{function}

\FPCexample{ex5}

\begin{function}{GetObjectPropClass}
\Declaration
Function GetObjectPropClass(Instance: TObject; const PropName: string): TClass;                                             
\Description
\var{GetObjectPropClass} returns the declared class of the property with name 
\var{PropName}. This may not be the actual class of the property value.
\Errors
No checking is done whether \var{Instance} is non-nil.
Specifying an invalid property name in \var{PropName} will result in an
\var{EPropertyError} exception.
\SeeAlso
\seep{SetMethodProp}, \seef{GetOrdProp}, \seef{GetStrProp},
\seef{GetFloatProp}, \seef{GetInt64Prop}
\end{function}

For an example, see \seef{GetObjectProp}.

\begin{function}{GetOrdProp}
\Declaration
Function GetOrdProp(Instance : TObject;PropInfo : PPropInfo) : Longint;\\
Function GetOrdProp(Instance: TObject;const PropName: string): Longint;
\Description
\var{GetOrdProp} returns the value of the ordinal property described by
\var{PropInfo} or with name \var{PropName} for the object \var{Instance}. 
The value is returned as a longint, which should be typecasted to the 
needed type.

Ordinal properties that can be retrieved include:
\begin{description}
\item[Integers and subranges of integers] The value of the integer will be
returned.
\item[Enumerated types and subranges of enumerated types] The ordinal value
of the enumerated type will be returned.
\item[Sets] If the base type of the set has less than 31 possible values.
If a bit is set in the return value, then the corresponding element of the
base ordinal class of the set type must be included in the set.
\end{description}
\Errors
No checking is done whether \var{Instance} is non-nil, or whether
\var{PropInfo} describes a valid ordinal property of \var{Instance}
Specifying an invalid property name in \var{PropName} will result in an
\var{EPropertyError} exception.
\SeeAlso
\seep{SetOrdProp}, \seef{GetStrProp}, \seef{GetFloatProp},
\seef{GetInt64Prop},\seef{GetMethodProp}, \seef{GetSetProp},
\seef{GetObjectProp}, \seef{GetEnumProp}
\end{function}

\FPCexample{ex1}

\begin{function}{GetPropInfo}
\Declaration
Function GetPropInfo(AClass: TClass; const PropName: string; AKinds: TTypeKinds) : PPropInfo;\\
Function GetPropInfo(AClass: TClass; const PropName: string): PPropInfo;\\
Function GetPropInfo(Instance: TObject; const PropName: string): PPropInfo;\\
Function GetPropInfo(Instance: TObject; const PropName: string; AKinds: TTypeKinds) : PPropInfo;\\
Function GetPropInfo(TypeInfo: PTypeInfo;const PropName: string) : PPropInfo;\\
Function GetPropInfo(TypeInfo: PTypeInfo;const PropName: string; AKinds : TTypeKinds) : PPropInfo;
\Description
\var{GetPropInfo} returns a pointer to the \var{TPropInfo} record for a the 
\var{PropName} property of a class. The class to examine can be specified 
in one of three ways:
\begin{description}
\item[Instance] An instance of the class.
\item[AClass] A class pointer to the class.
\item[TypeInfo] A pointer to the type information of the class.
\end{description}
In each of these three ways, if \var{AKinds} is specified, if the property 
has \var{TypeKind} which is not included in \var{Akinds}, \var{Nil} will be
returned.
\Errors
If the property \var{PropName} does not exist, \var{Nil} is returned.
\SeeAlso
\seep{GetPropInfos},\seef{GetPropList}
\end{function}

For an example, see most of the other functions.

\begin{procedure}{GetPropInfos}
\Declaration
Procedure GetPropInfos(TypeInfo: PTypeInfo;PropList: PPropList);
\Description
\var{GetPropInfos} stores pointers to the property information of all published
properties of a class with class info \var{TypeInfo} in the list pointed to by
\var{Proplist}. The \var{PropList} pointer must point to a memory location that
contains enough space to hold all properties of the class and its parent classes.
\Errors
No checks are done to see whether \var{PropList} points to a memory area that 
is big enough to hold all pointers.
\SeeAlso
\seef{GetPropInfo},\seef{GetPropList}
\end{procedure}

\FPCexample{ex12}

\begin{function}{GetPropList}
\Declaration
Function GetPropList(TypeInfo : PTypeInfo;
                     TypeKinds : TTypeKinds; 
                     PropList : PPropList) : Integer;
\Description
\var{GetPropList} stores pointers to property information of the class with class
info \var{TypeInfo} for properties of kind \var{TypeKinds} in the list pointed to
by \var{Proplist}. \var{PropList} must contain enough space to hold all properties.

The function returns the number of pointers that matched the criteria and were stored
in \var{PropList}.
\Errors
No checks are done to see whether \var{PropList} points to a memory area that is big enough
to hold all pointers.
\SeeAlso
\seep{GetPropInfos}, \seef{GetPropInfo}
\end{function}

\FPCexample{ex13}

\begin{function}{GetPropValue}
\Declaration
Function GetPropValue(Instance: TObject; const PropName: string): Variant;\\
Function GetPropValue(Instance: TObject; const PropName: string; PreferStrings: Boolean): Variant;
\Description
Due to missing \var{Variant} support, \var{GetPropValue} is not yet implemented. 
The declaration is provided for compatibility with Delphi.
\Errors
\SeeAlso
\end{function}

\begin{function}{GetSetProp}
\Declaration
Function GetSetProp(Instance: TObject; const PropInfo: PPropInfo; Brackets: Boolean):
string;\\
Function GetSetProp(Instance: TObject; const PropName: string): string;\\
Function GetSetProp(Instance: TObject; const PropName: string; Brackets: Boolean): string;                 
\Description
\var{GetSetProp} returns the contents of a set property as a string.
The property to be returned can be specified by it's name in \var{PropName}
or by its property information in \var{PropInfo}.

The returned set is a string representation of the elements in the set as
returned by \seef{SetToString}. The \var{Brackets} option can be used to 
enclose the string representation in square brackets.
\Errors
No checking is done whether \var{Instance} is non-nil, or whether
\var{PropInfo} describes a valid ordinal property of \var{Instance}
Specifying an invalid property name in \var{PropName} will result in an
\var{EPropertyError} exception.                                                 
\SeeAlso
\seep{SetSetProp}, \seef{GetStrProp}, \seef{GetFloatProp},
\seef{GetInt64Prop},\seef{GetMethodProp}
\end{function}

\FPCexample{ex7}

\begin{function}{GetStrProp}
\Declaration
Function GetStrProp(Instance : TObject;
                    PropInfo : PPropInfo) : Ansistring;\\
Function GetStrProp(Instance: TObject; 
                    const PropName: string): string;
\Description
\var{GetStrProp} returns the value of the string property described by
\var{PropInfo} or with name \var{PropName} for object \var{Instance}. 
\Errors
No checking is done whether \var{Instance} is non-nil, or whether
\var{PropInfo} describes a valid string property of \var{Instance}.
Specifying an invalid property name in \var{PropName} will result in an
\var{EPropertyError} exception.
\SeeAlso
\seep{SetStrProp}, \seef{GetOrdProp}, \seef{GetFloatProp},
\seef{GetInt64Prop},\seef{GetMethodProp}
\end{function}

\FPCexample{ex3}

\begin{function}{GetTypeData}
\Declaration
Function GetTypeData(TypeInfo : PTypeInfo) : PTypeData;
\Description
\var{GetTypeData} returns a pointer to the \var{TTypeData} record that
follows after the \var{TTypeInfo} record pointed to by \var{TypeInfo}.
It essentially skips the \var{Kind} and \var{Name} fields in the 
\var{TTypeInfo} record.
\Errors
None.
\SeeAlso
\end{function}

\begin{function}{GetVariantProp}
\Declaration
Function GetVariantProp(Instance : TObject;PropInfo : PPropInfo): Variant;
\Description
Due to mising Variant support, the \var{GetVariantProp} function is not 
yet implemented. Provided for Delphi compatibility only.
\Errors
\SeeAlso
\seep{SetVariantProp}
\end{function}

\begin{function}{IsPublishedProp}
\Declaration
Function IsPublishedProp(AClass: TClass; const PropName: string): Boolean;\\
Function IsPublishedProp(Instance: TObject; const PropName: string): Boolean;                              
\Description
\var{IsPublishedProp} returns true if a class has a published property with
name \var{PropName}. The class can be specfied in one of two ways:
\begin{description}
\item[AClass] A class pointer to the class.
\item[Instance] An instance of the class.
\end{description}
\Errors
No checks are done to ensure \var{Instance} or \var{AClass} are valid
pointers. Specifying an invalid property name in \var{PropName} will result
in an \var{EPropertyError} exception.                                                 
\SeeAlso
\seef{IsStoredProp}, \seef{PropIsType}
\end{function}

\FPCexample{ex10}

\begin{function}{IsStoredProp}
\Declaration
Function IsStoredProp(Instance : TObject;PropInfo : PPropInfo) : Boolean;\\
Function IsStoredProp(Instance: TObject; const PropName: string): Boolean; 
\Description
\var{IsStoredProp} returns \var{True} if the \var{Stored} modifier evaluates
to \var{True} for the property described by \var{PropInfo} or with name
\var{PropName} for object \var{Instance}. 
It returns \var{False} otherwise. If the function returns
\var{True}, this indicates that the property should be written when
streaming the object \var{Instance}.

If there was no \var{stored} modifier in the declaration of the property, 
\var{True} will be returned. 
\Errors
No checking is done whether \var{Instance} is non-nil, or whether
\var{PropInfo} describes a valid property of \var{Instance}.
Specifying an invalid property name in \var{PropName} will result in an
\var{EPropertyError} exception.                                                 
\SeeAlso
\seef{IsPublishedProp}, \seef{PropIsType}
\end{function}

\FPCexample{ex11}

\begin{function}{PropIsType}
\Declaration
Function PropIsType(AClass: TClass; 
                    const PropName: string; TypeKind: TTypeKind): Boolean;\\
Function PropIsType(Instance: TObject; 
                    const PropName: string; TypeKind: TTypeKind): Boolean;              
\Description
\var{PropIsType} returns \var{True} if the property with name \var{PropName}
has type \var{TypeKind}. It returns \var{False} otherwise. The class to be
examined can be specified in one of two ways:
\begin{description}
\item[AClass] A class pointer. 
\item[Instance] An instance of the class.
\end{description}
\Errors
No checks are done to ensure \var{Instance} or \var{AClass} are valid
pointers.Specifying an invalid property name in \var{PropName} will result
in an \var{EPropertyError} exception.                                                 
\SeeAlso
\seef{IsPublishedProp}, \seef{IsStoredProp}, \seef{PropType}
\end{function}

\FPCexample{ex16}

\begin{function}{PropType}
\Declaration
Function PropType(AClass: TClass; const PropName: string): TTypeKind;\\
Function PropType(Instance: TObject; const PropName: string): TTypeKind;
\Description
\var{Proptype} returns the type of the property \var{PropName} for a class.
The class to be examined can be specified in one of 2 ways:
\begin{description}
\item[AClass] A class pointer. 
\item[Instance] An instance of the class.
\end{description}
\Errors
No checks are done to ensure \var{Instance} or \var{AClass} are valid
pointers. Specifying an invalid property name in \var{PropName} will result
in an \var{EPropertyError} exception.
\SeeAlso
\seef{IsPublishedProp}, \seef{IsStoredProp}, \seef{PropIsType}
\end{function}

\FPCexample{ex17}

\begin{procedure}{SetEnumProp}
\Declaration
Procedure SetEnumProp(Instance: TObject; const PropInfo: PPropInfo;
                      const Value: string);\\
Procedure SetEnumProp(Instance: TObject; const PropName: string;
                      const Value: string);                      
\Description
\var{SetEnumProp} sets the property described by \var{PropInfo} or with name
\var{PropName} to \var{Value}. \var{Value} must be a string with the name
of the enumerate value, i.e. it can be used as an argument to 
\seef{GetEnumValue}.
\Errors
No checks are done to ensure \var{Instance} or \var{PropInfo} are valid
pointers. Specifying an invalid property name in \var{PropName} will result
in an \var{EPropertyError} exception.                                           
\SeeAlso
\seef{GetEnumProp}, \seep{SetStrProp}, \seep{SetFloatProp},
\seep{SetInt64Prop},\seep{SetMethodProp}.
\end{procedure}

For an example, see \seef{GetEnumProp}.

\begin{procedure}{SetFloatProp}
\Declaration
Procedure SetFloatProp(Instance : TObject;
                       PropInfo : PPropInfo;
                       Value : Extended);\\
Procedure SetFloatProp(Instance: TObject; 
                       const PropName: string; 
                       Value: Extended);
\Description
\var{SetFloatProp} assigns \var{Value} to the property described by
\var{PropInfo} or with name \var{Propname} for the object \var{Instance}.
\Errors
No checking is done whether \var{Instance} is non-nil, or whether
\var{PropInfo} describes a valid float property of \var{Instance}.
Specifying an invalid property name in \var{PropName} will result in an
\var{EPropertyError} exception.
\SeeAlso
\seef{GetFloatProp}, \seep{SetOrdProp}, \seep{SetStrProp},
\seep{SetInt64Prop},\seep{SetMethodProp}
\end{procedure}

For an example, see \seef{GetFloatProp}.


\begin{procedure}{SetInt64Prop}
\Declaration
Procedure SetInt64Prop(Instance: TObject; PropInfo: PPropInfo; const Value: Int64);\\
Procedure SetInt64Prop(Instance: TObject; const PropName: string; const Value: Int64);
\Description
\var{SetInt64Prop} assigns \var{Value} to the property of type
\var{Int64} that is described by \var{PropInfo} or with name \var{Propname} 
for the object \var{Instance}.
\Errors
No checking is done whether \var{Instance} is non-nil, or whether
\var{PropInfo} describes a valid \var{Int64} property of \var{Instance}.
Specifying an invalid property name in \var{PropName} will result in an
\var{EPropertyError} exception.
\SeeAlso
\seef{GetInt64Prop}, \seef{GetMethodProp}, \seep{SetOrdProp}, \seep{SetStrProp},
\seep{SetFloatProp}
\end{procedure}

For an example, see \seef{GetInt64Prop}.


\begin{procedure}{SetMethodProp}
\Declaration
Procedure SetMethodProp(Instance : TObject;PropInfo : PPropInfo; const Value :
TMethod);\\
Procedure SetMethodProp(Instance: TObject; const PropName: string; const Value: TMethod);
\Description
\var{SetMethodProp} assigns \var{Value} to the method the property described 
by \var{PropInfo} or with name \var{Propname} for object \var{Instance}.

The type \var{TMethod} of the \var{Value} parameter is defined in the
\file{SysUtils} unit as:
\begin{verbatim}
TMethod = packed record
  Code, Data: Pointer;
end;                                                                         
\end{verbatim}
\var{Data} should point to the instance of the class with the method \var{Code}.

\Errors
No checking is done whether \var{Instance} is non-nil, or whether
\var{PropInfo} describes a valid method property of \var{Instance}.
Specifying an invalid property name in \var{PropName} will result in an
\var{EPropertyError} exception.
\SeeAlso
\seef{GetMethodProp}, \seep{SetOrdProp}, \seep{SetStrProp},
\seep{SetFloatProp}, \seep{SetInt64Prop}
\end{procedure}

For an example, see \seef{GetMethodProp}.


\begin{procedure}{SetObjectProp}
\Declaration
Procedure SetObjectProp(Instance: TObject; 
                        PropInfo: PPropInfo; Value: TObject);\\
Procedure SetObjectProp(Instance: TObject; 
                        const PropName: string; Value: TObject);                        
\Description
\var{SetObjectProp} assigns \var{Value} to the the object property described by
\var{PropInfo} or with name \var{Propname} for the object \var{Instance}. 
\Errors
No checking is done whether \var{Instance} is non-nil, or whether
\var{PropInfo} describes a valid method property of \var{Instance}.
Specifying an invalid property name in \var{PropName} will result in an
\var{EPropertyError} exception.
\SeeAlso
\seef{GetObjectProp}, \seep{SetOrdProp}, \seep{SetStrProp},
\seep{SetFloatProp}, \seep{SetInt64Prop}, \seep{SetMethodProp}
\end{procedure}

For an example, see \seef{GetObjectProp}.

\begin{procedure}{SetOrdProp}
\Declaration
Procedure SetOrdProp(Instance : TObject; PropInfo : PPropInfo; 
                     Value : Longint);\\
Procedure SetOrdProp(Instance: TObject; const PropName: string;
                     Value: Longint);
\Description
\var{SetOrdProp} assigns \var{Value} to the the ordinal property described by 
\var{PropInfo} or with name \var{Propname} for the object \var{Instance}. 

Ordinal properties that can be set include:
\begin{description}
\item[Integers and subranges of integers] The actual value of the integer must be 
passed.
\item[Enumerated types and subranges of enumerated types] The ordinal value
of the enumerated type must be passed.
\item[Subrange types] of integers or enumerated types. Here the ordinal
value must be passed.
\item[Sets] If the base type of the set has less than 31 possible values.
For each possible value; the corresponding bit of \var{Value} must be set.
\end{description}
\Errors
No checking is done whether \var{Instance} is non-nil, or whether
\var{PropInfo} describes a valid ordinal property of \var{Instance}. 
No range checking is performed.
Specifying an invalid property name in \var{PropName} will result in an
\var{EPropertyError} exception.
\SeeAlso
\seef{GetOrdProp}, \seep{SetStrProp}, \seep{SetFloatProp},
\seep{SetInt64Prop},\seep{SetMethodProp}
\end{procedure}


For an example, see \seef{GetOrdProp}.


\begin{procedure}{SetPropValue}
\Declaration
Procedure SetPropValue(Instance: TObject; 
                       const PropName: string; const Value: Variant);                   
\Description
Due to missing Variant support, this function is not yet implemented;
it is provided for Delphi compatibility only.
\Errors
\SeeAlso
\end{procedure}

\begin{procedure}{SetSetProp}
\Declaration
Procedure SetSetProp(Instance: TObject; 
                     const PropInfo: PPropInfo; const Value: string);\\
Procedure SetSetProp(Instance: TObject;
                     const PropName: string; const Value: string);                      
\Description
\var{SetSetProp} sets the property specified by \var{PropInfo} or
\var{PropName} for object \var{Instance} to \var{Value}. \var{Value} is a
string which contains a comma-separated list of values, each value being a
string-representation of the enumerated value that should be included in
the set. The value should be accepted by the \seef{StringToSet} function.

The value can be formed using the \seef{SetToString} function.
\Errors
No checking is done whether \var{Instance} is non-nil, or whether
\var{PropInfo} describes a valid ordinal property of \var{Instance}.
No range checking is performed.
Specifying an invalid property name in \var{PropName} will result in an
\var{EPropertyError} exception.                                                 
\SeeAlso
\seef{GetSetProp}, \seep{SetOrdProp}, \seep{SetStrProp}, \seep{SetFloatProp},
\seep{SetInt64Prop},\seep{SetMethodProp}, \seef{SetToString},
\seef{StringToSet}
\end{procedure}

For an example, see \seef{GetSetProp}.

\begin{procedure}{SetStrProp}
\Declaration
procedure SetStrProp(Instance : TObject; PropInfo : PPropInfo; 
                     const Value : Ansistring);\\
Procedure SetStrProp(Instance: TObject; const PropName: string; 
                     const Value: AnsiString);
\Description
\var{SetStrProp} assigns \var{Value} to the string property described by
\var{PropInfo} or with name \var{Propname} for object \var{Instance}. 

\Errors
No checking is done whether \var{Instance} is non-nil, or whether
\var{PropInfo} describes a valid string property of \var{Instance}.
Specifying an invalid property name in \var{PropName} will result in an
\var{EPropertyError} exception.
\SeeAlso
\seef{GetStrProp}, \seep{SetOrdProp}, \seep{SetFloatProp},
\seep{SetInt64Prop},\seep{SetMethodProp}
\end{procedure}

For an example, see \seef{GetStrProp}

\begin{function}{SetToString}
\Declaration
function SetToString(PropInfo: PPropInfo; 
                     Value: Integer) : String;\\
function SetToString(PropInfo: PPropInfo; 
                     Value: Integer; Brackets: Boolean) : String;
\Description
\var{SetToString} takes an integer representation of a set (as received e.g.
by \var{GetOrdProp}) and turns it into a string representing the elements in
the set, based on the type information found in the \var{PropInfo} property
information. By default, the string representation is not surrounded by
square brackets. Setting the \var{Brackets} parameter to \var{True} will 
surround the string representation with brackets.

The function returns the string representation of the set.
\Errors
No checking is done to see whether \var{PropInfo} points to valid property
information.
\SeeAlso
\seef{GetEnumName}, \seef{GetEnumValue}, \seef{StringToSet}
\end{function}

\FPCexample{ex18}

\begin{procedure}{SetVariantProp}
\Declaration
Procedure SetVariantProp(Instance : TObject;
                         PropInfo : PPropInfo;
                         Const Value: Variant);\\
Procedure SetVariantProp(Instance: TObject; 
                         const PropName: string; 
                         const Value: Variant);                 
\Description
Due to missing Variant support, this function is not yet implemented. 
Provided for Delphi compatibility only.
\Errors
\SeeAlso
\end{procedure}


\begin{function}{StringToSet}
\Declaration
function StringToSet(PropInfo: PPropInfo; const Value: string): Integer;
\Description
\var{StringToSet} converts the string representation of a set in \var{Value}
to a integer representation of the set, using the property information found
in \var{PropInfo}. This property information should point to the property
information of a set property. The function returns the integer
representation of the set. (i.e, the set value, typecast to an integer)

The string representation can be surrounded with square brackets, and must 
consist of the names of the elements of the base type of the set. The base
type of the set should be an enumerated type. The elements should be
separated by commas, and may be surrounded by spaces.
each of the names will be fed to the \seef{GetEnumValue} function.
\Errors
No checking is done to see whether \var{PropInfo} points to valid property
information. If a wrong name is given for an enumerated value, then an
\var{EPropertyError} will be raised.
\SeeAlso
\seef{GetEnumName}, \seef{GetEnumValue}, \seef{SetToString}
\end{function}

For an example, see \seef{SetToString}.
% The VIDEO unit
%
%   $Id$
%   This file is part of the FPC documentation.
%   Copyright (C) 1997, by Michael Van Canneyt
%
%   The FPC documentation is free text; you can redistribute it and/or
%   modify it under the terms of the GNU Library General Public License as
%   published by the Free Software Foundation; either version 2 of the
%   License, or (at your option) any later version.
%
%   The FPC Documentation is distributed in the hope that it will be useful,
%   but WITHOUT ANY WARRANTY; without even the implied warranty of
%   MERCHANTABILITY or FITNESS FOR A PARTICULAR PURPOSE.  See the GNU
%   Library General Public License for more details.
%
%   You should have received a copy of the GNU Library General Public
%   License along with the FPC documentation; see the file COPYING.LIB.  If not,
%   write to the Free Software Foundation, Inc., 59 Temple Place - Suite 330,
%   Boston, MA 02111-1307, USA.
%
%%%%%%%%%%%%%%%%%%%%%%%%%%%%%%%%%%%%%%%%%%%%%%%%%%%%%%%%%%%%%%%%%%%%%%%
%
%%%%%%%%%%%%%%%%%%%%%%%%%%%%%%%%%%%%%%%%%%%%%%%%%%%%%%%%%%%%%%%%%%%%%%%
% The Video unit
%%%%%%%%%%%%%%%%%%%%%%%%%%%%%%%%%%%%%%%%%%%%%%%%%%%%%%%%%%%%%%%%%%%%%%%
\chapter{The VIDEO unit}
\FPCexampledir{videoex}

The \file{Video} unit implements a screen access layer which is system
independent. It can be used to write on the screen in a system-independent
way, which should be optimal on all platforms for which the unit is
implemented.

The working of the \file{Video} is simple: After calling \seep{InitVideo},
the array \var{VideoBuf} contains a representation of the video screen of
size \var{ScreenWidth*ScreenHeight}, going from left to right and top to
bottom when walking the array elements: \var{VideoBuf[0]} contains the 
character and color code of the top-left character on the screen.
\var{VideoBuf[ScreenWidth]} contains the data for the character in the
first column of the second row on the screen, and so on.

To write to the 'screen', the text to be written should be written to the
\var{VideoBuf} array. Calling \seep{UpdateScreen} will then cp the text to
the screen in the most optimal way. (an example can be found further on).

The color attribute is a combination of the foreground and background color,
plus the blink bit. The bits describe the various color combinations:
\begin{description}
\item[bits 0-3] The foreground color. Can be set using all color constants. 
\item[bits 4-6] The background color. Can be set using a subset of the
color constants.
\item[bit 7] The blinking bit. If this bit is set, the character will appear
blinking.
\end{description}
Each possible color has a constant associated with it, see page
\pageref{vidcolorconst} for a list of constants.

The contents of the \var{VideoBuf} array may be modified: This is 'writing'
to the screen. As soon as everything that needs to be written in the array
is in the \var{VideoBuf} array, calling \var{UpdateScreen} will copy the
contents of the array screen to the screen, in a manner that is as efficient
as possible.

The updating of the screen can be prohibited to optimize performance; To
this end, the \seep{LockScreenUpdate} function can be used: This will
increment an internal counter. As long as the counter differs from zero,
calling \seep{UpdateScreen} will not do anything. The counter can be
lowered with \seep{UnlockScreenUpdate}. When it reaches zero, the next call
to \seep{UpdateScreen} will actually update the screen. This is useful when
having nested procedures that do a lot of screen writing.

The video unit also presents an interface for custom screen drivers, thus
it is possible to override the default screen driver with a custom screen 
driver, see the \seef{SetVideoDriver} call. The current video driver can
be retrieved using the \seep{GetVideoDriver} call.

\begin{remark}
The video unit should {\em not} be used together with the \file{crt} unit.
Doing so will result in very strange behaviour, possibly program crashes.
\end{remark}

\section{Constants, Type and variables }

\subsection{Constants}
\label{vidcolorconst}
The following constants describe colors that can be used as 
foreground and background colors.
\begin{verbatim}
Black         = 0;
Blue          = 1;
Green         = 2;
Cyan          = 3;
Red           = 4;
Magenta       = 5;
Brown         = 6;
LightGray     = 7;
\end{verbatim}
The following color constants can be used as foreground colors only:
\begin{verbatim}
DarkGray      = 8;
LightBlue     = 9;
LightGreen    = 10;
LightCyan     = 11;
LightRed      = 12;
LightMagenta  = 13;
Yellow        = 14;
White         = 15;
\end{verbatim}
The foreground and background color can be combined to a color attribute
with the following code:
\begin{verbatim}
Attr:=ForeGroundColor + (BackGroundColor shl 4);
\end{verbatim}
The color attribute can be logically or-ed with the blink attribute to
produce a blinking character:
\begin{verbatim}
Blink         = 128;
\end{verbatim}
But not all drivers may support this.

The following constants describe the capabilities of a certain video mode:
\begin{verbatim}
cpUnderLine     = $0001;
cpBlink         = $0002;
cpColor         = $0004;
cpChangeFont    = $0008;
cpChangeMode    = $0010;
cpChangeCursor  = $0020;
\end{verbatim}
The following constants describe the various supported cursor modes:
\begin{verbatim}
crHidden        = 0;
crUnderLine     = 1;
crBlock         = 2;
crHalfBlock     = 3;
\end{verbatim}
When a video function needs to report an error condition, the following
constants are used:
\begin{verbatim}
vioOK              = 0;
errVioBase         = 1000;
errVioInit         = errVioBase + 1; { Initialization error}
errVioNotSupported = errVioBase + 2; { Unsupported function }
errVioNoSuchMode   = errVioBase + 3; { No such video mode }
\end{verbatim}
The following constants can be read to get some information about the
current screen:
\begin{verbatim}
ScreenWidth     : Word = 0;
ScreenHeight    : Word = 0;
LowAscii        : Boolean = true;
NoExtendedFrame : Boolean = false;
FVMaxWidth      = 132;
\end{verbatim}
The error-handling code uses the following constants:
\begin{verbatim}
errOk              = 0;
ErrorCode: Longint = ErrOK;
ErrorInfo: Pointer = nil;
ErrorHandler: TErrorHandler = DefaultErrorHandler;
\end{verbatim}
The \var{ErrorHandler} variable can be set to a custom-error handling
function. It is set by default to the \seep{DefaultErrorHandler} function.

\subsection{Types}
The \var{TVideoMode} record describes a videomode. Its fields are
self-explaining: \var{Col,Row} describe the number of columns and 
rows on the screen for this mode. \var{Color} is \var{True} if this mode
supports colors, or \var{False} if not.
\begin{verbatim}
  PVideoMode = ^TVideoMode;
  TVideoMode = record
    Col,Row : Word;
    Color   : Boolean;
  end;
\end{verbatim}
\var{TVideoCell} describes one character on the screen. The high byte
contains the color attribute with which the character is drawn on the screen,
and the low byte contains the ASCII code of the character to be drawn.
\begin{verbatim}
TVideoCell = Word;
PVideoCell = ^TVideoCell;
\end{verbatim}
The \var{TVideoBuf} and \var{PVideoBuf} are two types used to represent the
screen.
\begin{verbatim}
TVideoBuf = array[0..32759] of TVideoCell;
PVideoBuf = ^TVideoBuf;
\end{verbatim}
The following type is used when reporting error conditions:
\begin{verbatim}
TErrorHandlerReturnValue = (errRetry, errAbort, errContinue);
\end{verbatim}
Here, \var{errRetry} means retry the operation, \var{errAbort}
abort and return error code and \var{errContinue} means abort
without returning an errorcode.

The \var{TErrorHandler} function is used to register an own error
handling function. It should be used when installing a custom error
handling function, and must return one of the above values.
\begin{verbatim}
TErrorHandler = 
  function (Code: Longint; Info: Pointer): TErrorHandlerReturnValue;
\end{verbatim}
\var{Code} should contain the error code for the error condition, 
and the \var{Info} parameter may contain any data type specific to 
the error code passed to the function.

The \var{TVideoDriver} record can be used to install a custom video
driver, with the \seef{SetVideoDriver} call:
\begin{verbatim}
TVideoDriver = Record
  InitDriver        : Procedure;
  DoneDriver        : Procedure;
  UpdateScreen      : Procedure(Force : Boolean);
  ClearScreen       : Procedure;
  SetVideoMode      : Function (Const Mode : TVideoMode) : Boolean;
  GetVideoModeCount : Function : Word;
  GetVideoModeData  : Function(Index : Word; Var Data : TVideoMode) : Boolean;
  SetCursorPos      : procedure (NewCursorX, NewCursorY: Word);
  GetCursorType     : function : Word;
  SetCursorType     : procedure (NewType: Word);
  GetCapabilities   : Function : Word;
end;
\end{verbatim}

\subsection{Variables}
The following variables contain information about the current screen
status:
\begin{verbatim}
ScreenColor      : Boolean;
CursorX, CursorY : Word;
\end{verbatim}
\var{ScreenColor} indicates whether the current screen supports colors.
\var{CursorX,CursorY} contain the current cursor position.

The following variables form the heart of the \file{Video} unit: The
\var{VideoBuf} array represents the physical screen. Writing to this
array and calling \seep{UpdateScreen} will write the actual characters
to the screen. \var{VideoBufSize} contains the actual screen size, and is
equal to the product of the number of columns times the number of lines 
on the screen (\var{ScreenWidth*ScreenHeight}).
\begin{verbatim}
VideoBuf     : PVideoBuf;
OldVideoBuf  : PVideoBuf;
VideoBufSize : Longint;
\end{verbatim}
The \var{OldVideoBuf} contains the state of the video screen after the last
screen update. The \seep{UpdateScreen} function uses this array to decide
which characters on screen should be updated, and which not. 

Note that the \var{OldVideoBuf} array may be ignored by some drivers, so
it should not be used. The Array is in the interface section of the video
unit mainly so drivers that need it can make use of it. 

\section{Functions and Procedures}

The examples in this section make use of the unit \file{vidutil}, which 
contains the \var{TextOut} function. This function writes a text to the
screen at a given location. It looks as follows:

\FPCexample{vidutil}

\begin{procedure}{ClearScreen}
\Declaration
procedure ClearScreen; 
\Description
\var{ClearScreen} clears the entire screen, and calls \seep{UpdateScreen}
after that. This is done by writing spaces to all character cells of the
video buffer in the default color (lightgray on black, color attribute \$07).
\Errors
None.
\SeeAlso
\seep{InitVideo}, \seep{UpdateScreen}
\end{procedure}

\FPCexample{ex3}

\begin{procedure}{DefaultErrorHandler}
\Declaration
function  DefaultErrorHandler(AErrorCode: Longint; AErrorInfo: Pointer): TErrorHandlerReturnValue; 
\Description
\var{DefaultErrorHandler} is the default error handler used by the video
driver. It simply sets the error code \var{AErrorCode} and \var{AErrorInfo} 
in the global variables \var{ErrorCode} and \var{ErrorInfo} and returns 
\var{errContinue}.
\Errors
None.
\SeeAlso
\end{procedure}

\begin{procedure}{DoneVideo}
\Declaration
procedure DoneVideo; 
\Description
\var{DoneVideo} disables the Video driver if the video driver is active. If
the videodriver was already disabled or not yet initialized, it does
nothing. Disabling the driver means it will clean up any allocated
resources, possibly restore the screen in the state it was before
\var{InitVideo} was called. Particularly, the \var{VideoBuf} and
\var{OldVideoBuf} arrays are no longer valid after a call to 
\var{DoneVideo}.

The \var{DoneVideo} should always be called if \var{InitVideo} was called.
Failing to do so may leave the screen in an unusable state after the program
exits.
\Errors
Normally none. If the driver reports an error, this is done through the
\var{ErrorCode} variable.
\SeeAlso
\seep{InitVideo}
\end{procedure}

For an example, see most other functions.

\begin{function}{GetCapabilities}
\Declaration
function GetCapabilities: Word; 
\Description
\var{GetCapabilities} returns the capabilities of the current driver.
It is an or-ed combination of the following constants:
\begin{description}
\item[cpUnderLine] The driver supports underlined characters.
\item[cpBlink] The driver supports blinking characters.
\item[cpColor] The driver supports colors.
\item[cpChangeFont] The driver supports the setting of a screen font.
Note, however, that a font setting API is not supported by the video unit.
\item[cpChangeMode] The driver supports the setting of screen modes. 
\item[cpChangeCursor] The driver supports changing the cursor shape.
\end{description}
Note that the video driver should not yet be initialized to use this
function. It is a property of the driver.
\Errors
None.
\SeeAlso
\seef{GetCursorType}, \seep{GetVideoDriver}
\end{function}

\FPCexample{ex4}

\begin{function}{GetCursorType}
\Declaration
function GetCursorType: Word; 
\Description
\var{GetCursorType} returns the current cursor type. It is one of the
following values: 
\begin{description}
\item[crHidden] The cursor is currently hidden.
\item[crUnderLine] The cursor is currently the underline character.
\item[crBlock] The cursor is currently the block character.
\item[crHalfBlock] The cursur is currently a block with height of half the
character cell height.
\end{description}
Note that not all drivers support all types of cursors.
\Errors
None.
\SeeAlso
\seep{SetCursorType}, \seef{GetCapabilities}
\end{function}

\FPCexample{ex5}

\begin{function}{GetLockScreenCount}
\Declaration
Function GetLockScreenCount : integer;
\Description
\var{GetLockScreenCount} returns the current lock level. When the lock
level is zero, a call to \seep{UpdateScreen} will actually update the
screen.
\Errors
None.
\SeeAlso
\seep{LockScreenUpdate}, \seep{UnlockScreenUpdate}, \seep{UpdateScreen}
\end{function}

\FPCexample{ex6}

\begin{procedure}{GetVideoDriver}
\Declaration
Procedure GetVideoDriver (Var Driver : TVideoDriver);
\Declaration
\var{GetVideoDriver} retrieves the current videodriver and returns it in
\var{Driver}. This can be used to chain video drivers.
\Errors
None.
\SeeAlso
\seef{SetVideoDriver}
\end{procedure}

For an example, see the section on writing a custom video driver.

\begin{procedure}{GetVideoMode}
\Declaration
procedure GetVideoMode(var Mode: TVideoMode); 
\Description
\var{GetVideoMode} returns the settings of the currently active video mode.
The \var{row,col} fields indicate the dimensions of the current video mode,
and \var{Color} is true if the current video supports colors.
\Errors
None.
\SeeAlso
\seef{SetVideoMode}, \seef{GetVideoModeData}
\end{procedure}

\FPCexample{ex7}

\begin{function}{GetVideoModeCount}
\Declaration
Function GetVideoModeCount : Word;
\Description
\var{GetVideoModeCount} returns the number of video modes that the current
driver supports. If the driver does not support switching of modes, then 1
is returned.

This function can be used in conjunction with the \seef{GetVideoModeData}
function to retrieve data for the supported video modes.
\Errors
None.
\SeeAlso
\seef{GetVideoModeData}, \seep{GetVideoMode}
\end{function}

\FPCexample{ex8}

\begin{function}{GetVideoModeData}
\Declaration
Function GetVideoModeData(Index : Word; Var Data: TVideoMode) : Boolean;  
\Description
\var{GetVideoModeData} returns the characteristics of the \var{Index}-th
video mode in \var{Data}. \var{Index} is zero based, and has a maximum value of
\var{GetVideoModeCount-1}. If the current driver does not support setting of
modes (\var{GetVideoModeCount=1}) and \var{Index} is zero, the current mode 
is returned.

The function returns \var{True} if the mode data was retrieved succesfully,
\var{False} otherwise.
\Errors
In case \var{Index} has a wrong value, \var{False} is returned.
\SeeAlso
\seef{GetVideoModeCount}, \seef{SetVideoMode}, \seep{GetVideoMode}
\end{function}

For an example, see \seef{GetVideoModeCount}.

\begin{procedure}{InitVideo}
\Declaration
procedure InitVideo; 
\Description
\var{InitVideo} Initializes the video subsystem. If the video system was
already initialized, it does nothing. 
After the driver has been initialized, the \var{VideoBuf} and \var{OldVideoBuf} 
pointers are initialized, based on the \var{ScreenWidth} and
\var{ScreenHeight} variables. When this is done, the screen is cleared.
\Errors
if the driver fails to initialize, the \var{ErrorCode} variable is set.
\SeeAlso
\seep{DoneVideo}
\end{procedure}

For an example, see most other functions.

\begin{procedure}{LockScreenUpdate}
\Declaration
Procedure LockScreenUpdate;
\Description
\var{LockScreenUpdate} increments the screen update lock count with one.
As long as the screen update lock count is not zero, \seep{UpdateScreen}
will not actually update the screen.

This function can be used to optimize screen updating: If a lot of writing
on the screen needs to be done (by possibly unknown functions), calling
\var{LockScreenUpdate} before the drawing, and \seep{UnlockScreenUpdate}
after the drawing, followed by a \seep{UpdateScreen} call, all writing will
be shown on screen at once.
\Errors
None.
\SeeAlso
\seep{UpdateScreen}, \seep{UnlockScreenUpdate}, \seef{GetLockScreenCount}
\end{procedure}

For an example, see \seef{GetLockScreenCount}.

\begin{procedure}{SetCursorPos}
\Declaration
procedure SetCursorPos(NewCursorX, NewCursorY: Word); 
\Description
\var{SetCursorPos} positions the cursor on the given position: Column 
\var{NewCursorX} and row \var{NewCursorY}. The origin of the screen is the
upper left corner, and has coordinates \var{(0,0)}.

The current position is stored in the \var{CursorX} and \var{CursorY}
variables.
\Errors
None.
\SeeAlso
\seep{SetCursorType}
\end{procedure}

\FPCexample{ex2}

\begin{procedure}{SetCursorType}
\Declaration
procedure SetCursorType(NewType: Word); 
\Description
\var{SetCursorType} sets the cursor to the type specified in \var{NewType}.
\begin{description}
\item[crHidden] the cursor is not visible.
\item[crUnderLine] the cursor is a small underline character (usually 
denoting insert mode).
\item[crBlock] the cursor is a block the size of a screen cell (usually
denoting overwrite mode).
\item[crHalfBlock] the cursor is a block half the size of a screen cell.
\end{description}
\Errors
None.
\SeeAlso
\seep{SetCursorPos}
\end{procedure}

\begin{function}{SetVideoDriver}
\Declaration
Function SetVideoDriver (Const Driver : TVideoDriver) : Boolean;
\Description
\var{SetVideoDriver} sets the videodriver to be used to \var{Driver}. If
the current videodriver is initialized (after a call to \var{InitVideo})
then it does nothing and returns \var{False}.

A new driver can only be installed if the previous driver was not yet
activated (i.e. before a call to \seep{InitVideo}) or after it was
deactivated (i.e after a call to \var{DoneVideo}).

For more information about installing a videodriver, see \sees{viddriver}.
\Errors
If the current driver is initialized, then \var{False} is returned.
\SeeAlso
The example video driver in \sees{viddriver}
\end{function}

For an example, see the section on writing a custom video driver.

\begin{function}{SetVideoMode}
\Declaration
Function SetVideoMode(Mode: TVideoMode) : Boolean;
\Description
\var{SetVideoMode} sets the video mode to the mode specified in \var{Mode}:
\begin{verbatim}
  TVideoMode = record
    Col,Row : Word;
    Color   : Boolean;
  end;
\end{verbatim}
If the call was succesful, then the screen will have \var{Col} columns and
\var{Row} rows, and will be displaying in color if \var{Color} is
\var{True}. 

The function returns \var{True} if the mode was set succesfully, \var{False}
otherwise.

Note that the video mode may not always be set. E.g. a console on Linux
or a telnet session cannot always set the mode. It is important to check
the error value returned by this function if it was not succesful.

The mode can be set when the video driver has not yet been initialized
(i.e. before \seep{InitVideo} was called) In that case, the video mode will
be stored, and after the driver was initialized, an attempt will be made to
set the requested mode. Changing the video driver before the call to
\var{InitVideo} will clear the stored video mode.

To know which modes are valid, use the \seef{GetVideoModeCount} and
\seef{GetVideoModeData} functions. To retrieve the current video mode, 
use the \seep{GetVideoMode} procedure.
\Errors
If the specified mode cannot be set, then \var{errVioNoSuchMode} may be set
in \var{ErrorCode}
\SeeAlso
\seef{GetVideoModeCount}
\seef{GetVideoModeData}
\seep{GetVideoMode}
\end{function}

\begin{procedure}{UnlockScreenUpdate}
\Declaration
Procedure UnlockScreenUpdate;
\Description
\var{UnlockScreenUpdate} decrements the screen update lock count with one if
it is larger than zero. When the lock count reaches zero, the 
\seep{UpdateScreen} will actually update the screen. No screen update will 
be performed as long as the screen update lock count is nonzero. This 
mechanism can be used to increase screen performance in case a lot of 
writing is done. 

It is important to make sure that each call to \seep{LockScreenUpdate} is
matched by exactly one call to \var{UnlockScreenUpdate}
\Errors
None.
\SeeAlso
\seep{LockScreenUpdate}, \seef{GetLockScreenCount}, \seep{UpdateScreen}
\end{procedure}

For an example, see \seef{GetLockScreenCount}.

\begin{procedure}{UpdateScreen}
\Declaration
procedure UpdateScreen(Force: Boolean); 
\Description
\var{UpdateScreen} synchronizes the actual screen with the contents
of the \var{VideoBuf} internal buffer. The parameter \var{Force}
specifies whether the whole screen has to be redrawn (\var{Force=True})
or only parts that have changed since the last update of the screen.

The \var{Video} unit keeps an internal copy of the screen as it last 
wrote it to the screen (in the \var{OldVideoBuf} array). The current 
contents of \var{VideoBuf} are examined to see what locations on the 
screen need to be updated. On slow terminals (e.g. a \linux telnet 
session) this mechanism can speed up the screen redraw considerably.
\Errors
None.
\SeeAlso
\seep{ClearScreen}
\end{procedure}

For an example, see most other functions.

\section{Writing a custom video driver}
\label{se:viddriver}
Writing a custom video driver is not difficult, and generally means
implementing a couple of functions, which whould be registered with
the \seef{SetVideoDriver} function. The various functions that can be
implemented are located in the \var{TVideoDriver} record:
\begin{verbatim}
TVideoDriver = Record
  InitDriver        : Procedure;
  DoneDriver        : Procedure;
  UpdateScreen      : Procedure(Force : Boolean);
  ClearScreen       : Procedure;
  SetVideoMode      : Function (Const Mode : TVideoMode) : Boolean;
  GetVideoModeCount : Function : Word;
  GetVideoModeData  : Function(Index : Word; Var Data : TVideoMode) : Boolean;
  SetCursorPos      : procedure (NewCursorX, NewCursorY: Word);
  GetCursorType     : function : Word;
  SetCursorType     : procedure (NewType: Word);
  GetCapabilities   : Function : Word;
end;
\end{verbatim}
Not all of these functions must be implemented. In fact, the only absolutely
necessary function to write a functioning driver is the \var{UpdateScreen} 
function. The general calls in the \file{Video} unit will check which
functionality is implemented by the driver.

The functionality of these calls is the same as the functionality of the
calls in the video unit, so the expected behaviour can be found in the
previous section. Some of the calls, however, need some additional remarks.
\begin{description}
\item[InitDriver] Called by \var{InitVideo}, this function should initialize 
any data structures needed for the functionality of the driver, maybe do some 
screen initializations. The function is guaranteed to be called only once; It 
can only be called again after a call to \var{DoneVideo}. The variables
\var{ScreenWidth} and \var{ScreenHeight} should be initialized correctly
after a call to this function, as the \var{InitVideo} call will initialize
the \var{VideoBuf} and \var{OldVideoBuf} arrays based on their values.
\item[DoneDriver] This should clean up any structures that have been
initialized in the \var{InitDriver} function. It should possibly also
restore the screen as it was before the driver was initialized. The VideoBuf
and \var{OldVideoBuf} arrays will be disposed of by the general \var{DoneVideo}
call.
\item[UpdateScreen] This is the only required function of the driver. It
should update the screen based on the \var{VideoBuf} array's contents. It
can optimize this process by comparing the values with values in the
\var{OldVideoBuf} array. After updating the screen, the \var{UpdateScreen}
procedure should update the \var{OldVideoBuf} by itself. If the \var{Force}
parameter is \var{True}, the whole screen should be updated, not just the
changed values.
\item[ClearScreen] If there is a faster way to clear the screen than to
write spaces in all character cells, then it can be implemented here. If the
driver does not implement this function, then the general routines will
write spaces in all video cells, and will call \var{UpdateScreen(True)}.
\item[SetVideoMode] Should set the desired video mode, if available. It
should return \var{True} if the mode was set, \var{False} if not.
\item[GetVideoModeCount] Should return the number of supported video modes.
If no modes are supported, this function should not be implemented; the
general routines will return 1. (for the current mode)
\item[GetVideoModeData] Should return the data for the \var{Index}-th mode;
\var{Index} is zero based. The function should return true if the data was
returned correctly, false if \var{Index} contains an invalid index.
If this is not implemented, then the general routine will return the current 
video mode when \var{Index} equals 0.
\item[GetCapabilities] If this function is not implemented, zero (i.e.
no capabilities) will be returned by the general function.
\end{description}

The following unit shows how to override a video driver, with a driver
that writes debug information to a file.

\FPCexample{viddbg}

The unit can be used in any of the demonstration programs, by simply
including it in the \var{uses} clause. Setting \var{DetailedVideoLogging} to
\var{True} will create a more detailed log (but will also slow down
functioning)
% end of units. Index.
\printindex
\end{document}