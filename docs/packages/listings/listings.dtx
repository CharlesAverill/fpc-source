% \iffalse
%
% NOTE: This file contains very long lines (upto approx 270 characters).
%       I haven't wrapped them since I won't do that by hand and haven't
%       written any program doing the work.
%
%       If you want to read this .dtx file, you might get in trouble with
%       such long lines!
%
%
%  S O F T W A R E   L I C E N S E
% =================================
%
% The files  listings.dtx  and  listings.ins  and all files generated
% from only these two files are referred as 'the listings package' or
% simply 'the package'. A driver file is any file generated mainly from
% lstdrvrs.dtx.
%
% Copyright.
%	The listings package is copyright 1996--1998 Carsten Heinz.
%	The driver files are copyright 1997, 1998 or 1997--1998 any
%	individual author listed in these files.
%
% Distribution.
%	The listings package as well as lstdrvrs.dtx and all driver files
%	are distributed freely. You are not allowed to take money for the
%	distribution, except for a nominal charge for copying etc..
%
% Use of the package.
%	The listings package is free for any non-commercial use. Commercial
%	use needs (i) explicit permission of the author of this package and
%	(ii) the payment of a license fee. This fee is to be determined in
%	each instance by the commercial user and the package author and is
%	to be payed as donation to the LaTeX3 project.
%
% No warranty.
%	The listings package as well as lstdrvrs.dtx and all driver files
%	are distributed without any warranty, express or implied, as to
%	merchantability or fitness for any particular purpose.
%
% Modification advice.
%	Permission is granted to change the listings package as well as
%	lstdrvrs.dtx. You are not allowed to distribute any changed version
%	of the package or any changed version of lstdrvrs.dtx, neither under
%	the same name nor under a different one. Tell the author of the
%	package about your local changes: other users will welcome removed
%	bugs, new features and additional programming languages.
%
% Contacts.
%	Send comments and ideas on the package, error reports and additional
%	programming languages to
%
%		Carsten Heinz
%		Tellweg 6
%		42275 Wuppertal
%		Germany
%
%	or preferably to 
%
%		cheinz@gmx.de
%
% Trademarks
%	appear throughout this documentation without any trademark symbol,
%	so you can't assume that a name is free. There is no intention of
%	infringement; the usage is to the benefit of the trademark owner.
%
% end of software license
%
%<*driver>
\documentclass{ltxdoc}

\usepackage[doc]{listings}[1998/11/09]
\newif\iffancyvrb
\IfFileExists{fancyvrb.sty}
    {\fancyvrbtrue \usepackage{fancyvrb}}
    {\fancyvrbfalse}

\EnableCrossrefs         
\CodelineIndex
\OldMakeindex
\OnlyDescription

\begin{document}
    \DocInput{listings.dtx}
\end{document}
%</driver>
% \fi ^^A balance the \iffancyvrb! :-)
% \fi
%
%
% \makeatletter
%^^A
%^^A A modified `environment' environment
%^^A
%\def\aspect{^^A
%    \def\SpecialMainEnvIndex##1{^^A
%        \@bsphack ^^A
%        \index{aspects:\levelchar{\protect\ttfamily##1}\encapchar main}^^A
%        \@esphack}^^A
%    \begingroup \catcode`\\12 \MakePrivateLetters ^^A
%    \m@cro@ \iffalse}
%\let\endaspect\endmacro
%
%^^A
%^^A We define a sample environment. All material between
%^^A \begin{lstsample} and \end{lstsample} is executed
%^^A 'on the left side' and typeset verbatim on the right.
%^^A
%^^A The environment is *not* designed for purposes other than
%^^A this documentation.
% \lst@Environment{lstsample}[1]\is
%    {\gdef\lst@sample{#1}^^A
%     \lst@BeginWriteFile{listings.tmp}^^A
%     \def\lst@BOLGobble##1{}}^^A to gobble each first %
%    {\lst@EndWriteFile ^^A
%^^A We execute the code resp. typset it verbatim (= empty language
%^^A with \ttfamily).
%     \begin{center}
%     \small ^^A \lstCC@Let{`\^^M}\space ^^A
%     \begin{minipage}{0.45\linewidth}^^A
%       \MakePercentComment\lst@sample
%       % \iffalse
%<*driver>
\documentclass{ltxdoc}

\usepackage{listings}[1997/09/29]
\selectlisting{cpp}
\selectlisting{fortran}
\selectlisting{pascal}

\EnableCrossrefs         
\CodelineIndex
\OldMakeindex     % used for MakeIndex pre v2.9

\begin{document}
    \DocInput{listings.dtx}
\end{document}
%</driver>
%
% listings package for LaTeX2e
% (w)(c) 1996-1997 Carsten Heinz, all rights reserved.
% 
%<+package>\NeedsTeXFormat{LaTeX2e}
%<+package>\ProvidesPackage{listings}[1997/09/29 v0.17 by Carsten Heinz]
% \fi
%
% \CheckSum{2784}
% \DoNotIndex{\@ifundefined,\@tempa,\ ,\@dottedtocline,\@empty,\@ne}
% \DoNotIndex{\if@twocolumn,\if@restonecol,\@mkboth,\@restonecoltrue}
% \DoNotIndex{\@restonecolfalse,\@starttoc,\z@}
% \DoNotIndex{\[,\{,\},\],\1,\2,\3,\4,\5,\6,\7,\8,\9,\0}
% \DoNotIndex{\`,\,,\!,\#,\$,\&,\',\(,\),\+,\.,\:,\;,\<,\=,\>,\?,\_}
% \DoNotIndex{\active,\addtocontents,\advance,\begin,\backslash}
% \DoNotIndex{\baselineskip,\begingroup,\bfseries,\bgroup,\catcode}
% \DoNotIndex{\chapter,\chardef,\closein,\csname,\def,\divide,\do}
% \DoNotIndex{\edef,\egroup,\else,\@empty,\end,\endcsname,\endgroup}
% \DoNotIndex{\expandafter,\fi,\gdef,\global,\hbox,\hskip,\hss,\if}
% \DoNotIndex{\ifcat,\ifdim,\ifeof,\iffalse,\ifnum,\iftrue,\ifx}
% \DoNotIndex{\ignorespaces,\input,\itshape,\lccode,\let,\llap,\long}
% \DoNotIndex{\loop,\lst@,\makeatletter,\makeatother,\MakeUppercase}
% \DoNotIndex{\message,\multiply,\newcommand,\newenvironment,\newcount}
% \DoNotIndex{\newdimen,\newif,\newread,\next,\noindent,\onecolumn}
% \DoNotIndex{\openin,\par,\parshape,\parskip,\protect,\read,\relax}
% \DoNotIndex{\removelastskip,\repeat,\section,\setbox,\space}
% \DoNotIndex{\smallbreak,\string,\textwidth,\the,\thepage,\ttfamily}
% \DoNotIndex{\twocolumn,\undefined,\vskip,\vspace}
%
%
% \title{{LISTINGS.DTX} Version {0.17}}
% \author{{\copyright} 1996--1997 by Carsten Heinz}
% \date{}
%
% \makeatletter\@twocolumntrue\makeatother
% \maketitle
% \tableofcontents
% \vfill
% \noindent \textbf{\uppercase{incompatible changes}}
% have been made from version 0.16 to version 0.17. The commands
% \cs{blanklisting}, \cs{clisting}, {\ldots} have been removed.
% Languages are selected with the command \cs{selectlisting}.
% The command \cs{labelstyle} has a new syntax and semantics.
% These features are described on page \pageref{ssSelecting} and
% \pageref{newlabelstyle}, respectively.
% Sorry, but now it's possible to add languages without changing
% the kernel \texttt{listings.sty}.
% \onecolumn
%
%
% \section{User's guide}
%
%
% \subsection{Introduction}
%
% You have a problem? You wanna typeset source code of a programming
% language within \LaTeXe{} like
%     \selectlisting{cpp}
% \iffalse
%<*sample>
% \fi
% \begin{listing}
	switch (direction)
	{// choose direction
		case backward     : { ... break; }
		case backwarddown : { ... break; }
		case up           : { ... break; }
	}

	for (int i=0; i<1000; i++)
	{// do something
		if (i == i) { i += 1; i--; }
	}
% \end{listing}
% \iffalse
%</sample>
% \fi
% \noindent or
%     \labelstyle{\small\ttfamily}
%     \selectlisting{pascal}
% \iffalse
%<*sample>
% \fi
% \begin{listing}
	for i:=maxint downto 0 do
	begin
		{ do nothing }
	end;

	WriteLn('Pascal keywords are');
	writeln('not case sensitive.');
% \end{listing}
% \iffalse
%</sample>
% \fi
% Obviously the verbatim environment doesn't fit. You might use the
% tabbing environment or something like that. But you have to mark all
% keywords yourself and doing the indention is also a heavy work. But
% the most striking disadvantage is that you can't run your \LaTeXe-file
% through your C++, Pascal or whatever else compiler.
%
% Fortunately there is another possibility. You might use a cross
% compiler, e.g.\ you run a special compiler on your Pascal source
% code to get a \TeX-file with marked keywords, etc. Then \TeX{}
% produces a pretty output (usage of a well-designed cross compiler
% granted). If your sources change, you would have to rerun the cross
% compiler to get your \TeX-file right and have to compile this to get
% your document. Again and again.
%
% By now these days have gone. The listings package goes around this by
% reading the source code directly. Comments, strings and keywords can
% be typeset in different styles as shown in the examples above. The
% indention is taken from the sources. Hence, the programmer is
% responsible for writing 'readable' code. The package only helps to
% present it.
%
% This package is surely not the final utility for typesetting
% listings. May be it's a matter of pureness whether to use a cross
% compiler or the listings package. Hope you join it and help to get
% this tool more powerful. Please report all errors and offer
% improvements!
% 
%
% \subsection{Installation}
%
% The main files of this package are \texttt{listings.dtx} and
% \texttt{listings.ins}. Run the latter file through \TeX{}. This
% will create \texttt{listings.sty} and a couple of driver files
% with the prefix \texttt{lst}, e.g.\ \texttt{lstfortran.sty},
% but the exact name is system dependent.
% Copy all created \texttt{.sty}-files to a directory searched by
% \TeX{}. This completes the installation. Run \texttt{listings.dtx}
% through \TeX{} to get the documentation \texttt{listings.dvi}.
%
% \textbf{Important note:} All files of the listings package are
% distributed freely. You are not allowed to take money for the
% distribution or use of these files, except for a nomial charge for
% copying etc.
% The files are distributed without any warranty; without even the
% implied warranty of merchantability or fitness for a particular
% purpose.
% You are not allowed to change one of the files, except using a
% clearly different filename.
%
% \textbf{Trademarks} appear throughout this documentation without any
% trademark symbol. So you can't assume that a name is not trademarked.
% There is no intention of infringement. The usage is to the benefit
% of the trademark owner.
%
%
% \subsection{Selecting a language}
% \label{ssSelecting}\DescribeMacro\selectlisting
% You choose a language with \cs{selectlisting}. The argument is (nearly)
% one of the language names listed below. The driver files are loaded on
% demand. The command also has an optional argument to select different
% kinds of the same language, e.g.\ 
% \begin{verbatim}
%    \selectlisting[1974]{cobol}\end{verbatim}
% selects COBOL ANS-1974, whereas the default would be ANS-1985. All
% languages currently supported are listed together with the available
% options, where always the first option is default, i.e.\ selected
% if none is given.
% I also mention what the listings package is not capable of.
% '\texttt{???}' in driver files indicate things I don't know.
% \textbf{To help me}:
% \emph{Please send me an e-mail with the languages and options you
% use (i.e.\ are tested in a way), languages and options you want
% to be corrected or want additionally.}
% See section \ref{ssItsNotAllFine}.
% Note: Language selection is local now.
%
% \begin{description}
% \item[Blank] \verb!\selectlisting{blank}!.
%
%	This is the default language: no keywords and no comments are
%	detected (unless additional specified, see section
%	\ref{sOtherLanguages}).
% \item[Ada] \verb!\selectlisting{ada}!.
%
%	This package can't handle strings where the usual quotation marks
%	(double quotes) are replaced by percent characters.
% \item[Algol] \verb!\selectlisting{algol}!,
%	options: \texttt{68}, \texttt{60}.
%
%	Algol 60 seems to be ok. But there are problems concerning the
%	comments of Algol 68. First, comments enclosed by \rlap{/}c
%	are not	supported. Second, if you use a sharp $\#$ within
%	\verb!co!...\verb!co! (and dito within \verb!comment!), there
%	\emph{must} be a matching second $\#$, or the output is not all
%	right.\footnote{The problem is the following: The sharp is a special
%	character, whereas \texttt{co} is a predefined word and consists
%	of letters. Comments enclosed by special characters are handled
%	easily, and even comments starting or ending with keywords are
%	possible (Algol 60). But a mixture of both (without matching sharp)
%	can only be handled by changing the kernel \texttt{listings.sty}.
%	Since that would slow down all other languages, I decided to do
%	it this way (matching sharp). I even don't know, if this way is
%	standard Algol 68 --- the books are not very detailed, or I haven't
%	found it.} The other way round, i.e.\ a lonely \texttt{comment}
%	within $\#$...$\#$ is harmless.
% \item[C] \verb!\selectlisting{c}!.
% \item[C++] \verb!\selectlisting{cpp}!,
%	options: \texttt{ansi}, \texttt{vc} (Visual C++).
% \item[Cobol] \verb!\selectlisting{cobol}!,
%	options: \texttt{1985}, \texttt{1974}, \texttt{ibm}.
%
%	Keywords are not marked, if their names are broken in two parts,
%	i.e.\ continued in the following line.
%	Sometimes portions of a string are not printed as a string.
%	This happens, if the double quote is not doubled to insert a quote,
%	e.g.\ \verb!""bad" cobol"! won't be printed correctly.
% \item[Comal 80] \verb!\selectlisting{comal}!.
% \item[Eiffel] \verb!\selectlisting{eiffel}!.
% \item[Elan] \verb!\selectlisting{elan}!.
% \item[Fortran] \verb!\selectlisting{fortran}!,
%	options: \texttt{90}, \texttt{77}.
%
%	The keywords are assumed to be \emph{not} case sensitive.
% \item[Java] \verb!\selectlisting{java}!.
% \item[Lisp] \verb!\selectlisting{lisp}! (Common Lisp).
% \item[Logo] \verb!\selectlisting{logo}!.
% \item[Matlab] \verb!\selectlisting{matlab}!.
%
%	Matlab uses the quote with two meanings. This package always assumes
%	the beginning or ending of a string, even if you use the quote as
%	transpose-conjugate operator.
% \item[Modula-2] \verb!\selectlisting{modula}!.
% \item[Oberon-2] \verb!\selectlisting{oberon}!.
% \item[Pascal] \verb!\selectlisting{pascal}!.
% \item[Pascal XSC] \verb!\selectlisting{pxsc}!.
%
%	Tell me, if you want more words to be keywords.
% \item[Turbo Pascal] \verb!\selectlisting{tp}!.
%
%	Keywords are from version 6.0. Possibly too many keywords present.
% \item[Perl] \verb!\selectlisting{perl}!.
% \item[PL/I] \verb!\selectlisting{pli}!.
% \item[Simula 67] \verb!\selectlisting{simula}!,
%	options: \texttt{67}, \texttt{cii}, \texttt{dec}, \texttt{ibm}.
% \item[SQL-92] \verb!\selectlisting{sql}!.
% \item[\TeX] \verb!\selectlisting{tex}!,
%	options: \texttt{plain}, \texttt{primitive}, \texttt{latex},
%		\texttt{allatex}.
% \end{description}
%
%
% \subsection{Typesetting a listing}
%
% \DescribeMacro\inputlisting
% The main command for this purpose is \cs{inputlisting}. The syntax
% is
% \begin{verbatim}
%    \inputlisting[FIRST,LAST]{WHOLE FILENAME}\end{verbatim}
% where \verb![FIRST,LAST]! is optional and determines the range of lines
% to typeset. The default range is $[1,999999]$. If you specify negative
% lines or an empty range, for example, you won't get any warning.
% There is no default extension, so you must specify the whole filename.
% The example
% \begin{verbatim}
%    \inputlisting[3,10]{testfile.pas}\end{verbatim}
% would print lines 3,4,\ldots,10 of \verb!testfile.pas!, if file and
% lines are present.
%
% \DescribeEnv{listing}
% There is also a listing environment. It typesets the source code
% enclosed within \verb!\begin{listing}! and \verb!\end{listing}!.
% If the source code starts right after \verb!\begin{listing}!, it is
% dropped upto the end of line. Dito the source preceding and following
% \verb!\end{listing}! directly, i.e.\ in the same line. You may don't
% like this, but the lines
% \begin{verbatim}
%       \selectlisting{fortran}
%       \begin{listing}DROPPED
%* This is a test for the listing environment
%C and not considered to be an example for
%   ! Fortran!
%      Program example
%      ...
%DROPPED\end{listing}DROPPED\end{verbatim}
% result in
%    \normallisting
%    \selectlisting{fortran}
% \iffalse
%<*sample>
% \fi
%    \begin{listing}DROPPED
* This is a test for the listing environment
C and not considered to be an example for
   ! Fortran!
      Program example
      ...
%DROPPED\end{listing}DROPPED
% \iffalse
%</sample>
% \fi
%
% \DescribeMacro\listingfalse
% Since typesetting listings with this package is possibly slow,
% you can suppress the output with \cs{listingfalse}. Afterwards this
% \DescribeMacro\listingtrue
% package only prints the name of the source like
%     \selectlisting{pascal}\listingfalse
%     \inputlisting{testfile.pas}\listingtrue
% \cs{listingtrue} reverses the effect.
% 
% \DescribeMacro\tablength
% If you use tabulators in your source codes, you have to tell this
% package the number of characters between two tabulator stops.
% Default value is 4. Changes are made by, e.g.,\ 
% \begin{verbatim}
%    \tablength{8}\end{verbatim}
% i.e. tabulator stops are set at the columns 1, 9, 17, 25 {\ldots}
% Negative values are prohibited. They won't confuse this package,
% since it gives an error message and proceeds with the old value of
% tablength.
%
% \DescribeMacro\listoflistings
% Two more commands should be mentioned here. Use \cs{listoflistings}
% as you do \cs{listoftables} and \cs{listoffigures} and renew
% \DescribeMacro\listlistingsname\cs{listlistingsname} to change the
% default header 'Listings'. The list shows either the filename or
% the optional argument of the listing environment, e.g.\ 
% \begin{verbatim}
%    \inputlisting{Quicksort.pas}
%    \begin{listing}[Heapsort]
%    ...
%    \end{listing}\end{verbatim}
% will show the names 'Quicksort.pas' and 'Heapsort'.
%
%
% \subsection{Figure out the appearance}\label{ssFigureOutTheAppearance}
% You have some possibilities to change how the output looks like.
%
% \DescribeMacro\keywordstyle
% By default keywords\footnote{Let's talk about keywords: I tried to get
% all keywords of the languages, but there are so many books with no
% list of keywords. So I looked at the index, but some books print
% keywords bold and predefined functions in italics, other books print
% both in bold. So I had to guess, which words are keywords. Moreover
% it's a great deal to find out whether the keywords are case sensitive
% or not. Of course, there are really fine books (e.g.\ \cite{Ada}).
% And I know that some languages don't have keywords, and so there
% can't be a list of them. But even these languages have predefined
% words --- which are not listed. Hence, for our purpose \emph{keywords}
% are predefined words, fixed words, reserved words, real keywords and
% all words you or I want to name so.} are typeset bold, comments in
% italic shape and strings using no special style.
% \DescribeMacro\commentstyle
% You can change this with the commands \cs{keywordstyle},
% \DescribeMacro\stringstyle
% \cs{commentstyle} and \cs{stringstyle}, e.g.
% \begin{verbatim}
%    \keywordstyle{\bfseries\itshape} % bold and italic
%    \commentstyle{\tt\small}         % LaTeX 2.09 typewriter, small
%    \stringstyle{\ttfamily}          % typewriter\end{verbatim}
% will typeset the second example of the introduction as follows
%     \keywordstyle{\bfseries\itshape} \commentstyle{\tt\small}
%     \stringstyle{\ttfamily}
%     \selectlisting{pascal}
% \iffalse
%<*sample>
% \fi
% \begin{listing}
	for i:=maxint downto 0 do
	begin
		{ do nothing }
	end;

	WriteLn('Pascal keywords are');
	writeln('not case sensitive.');
% \end{listing}
% \iffalse
%</sample>
% \fi
%    \normallisting
%
% \DescribeMacro\blankstringtrue
% Once again blank spaces of a string have been printed \textvisiblespace.
% Suppress this by using \cs{blankstringtrue} and switch to default
% \DescribeMacro\blankstringfalse
% with \cs{blankstringfalse}.
%
% \DescribeMacro\labelstyle\label{newlabelstyle}
% By default no line numbers are printed. You can change this with the
% \cs{labelstyle} command. The syntax is
% \begin{verbatim}
%    \labelstyle[step]{the style}\end{verbatim}
% The style determines how the line numbers look like (\cs{ttfamily},
% \cs{small}, and so on). Line numbers are printed all 'step' lines.
% More precisely all line numbers are printed, which are divisible by
% step. If step is zero, no line numbers will be typeset.
% The step is an optional argument. If no step is specified, it is
% assumed to be 1. Here some (exotic) examples:
% \begin{verbatim}
%   \labelstyle[3]{\ttfamily\small} % line 3,6,9,..., small typewriter
%   \labelstyle{\small\oldstylenums}% small oldstyle numbers
%   \makeatletter
%   \labelstyle{\@roman} % roman numbers(!) each line
%   \labelstyle{\@Roman} % upper case roman numbers
%   \makeatother\end{verbatim}
%
% The C++ source line
%     \selectlisting{cpp}
% \iffalse
%<*sample>
% \fi
% \begin{listing}
    if (ThereIsALongLine) { ItExceedsASingleLineInTheListing(); }
% \end{listing}
% \iffalse
%</sample>
% \fi
% \noindent\DescribeMacro\spreadlisting
% gives rise to an overfull \verb!\hbox!. To avoid things like that
% you can first spread the width taken by a listing, and second
% \DescribeMacro\prelisting
% change the fontsize. The commands \cs{spreadlisting}, \cs{prelisting}
% \DescribeMacro\postlisting
% and \cs{postlisting} have one argument each. For the example above
% you should use
% \begin{verbatim}
%    \spreadlisting{1in} % spread the width by 1in, 0.5in left and right
%    \prelisting{\small} % select new fontsize
%    \postlisting{}      % nothing to do (default)\end{verbatim}
% or something similar and get
%     \spreadlisting{1in}
%     \prelisting{\small}
% \iffalse
%<*sample>
% \fi
% \begin{listing}
    if (ThereIsALongLine) { ItExceedsASingleLineInTheListing(); }
% \end{listing}
% \iffalse
%</sample>
% \fi
%     \normallisting
% You can use the pre-post-mechanism to add something right before a
% listing starts or after it ends, e.g.\ \verb!\prelisting{\bigbreak}!
% will lead to an empty line between a listing and the preceding text
% (if on the same page).
%
% \DescribeMacro\normallisting
% Changes made by the commands above are reset by \cs{normallisting}
% (there are no arguments): Keywords are typeset bold, comments in
% italic shape, stringstyle, labelstyle and pre- and postlisting are
% empty, blank spaces in strings are printed \textvisiblespace, and
% spreadlisting is set to 0pt. Nothing other is affected.
%
% \DescribeMacro\lstlineskip
% The one and only argument of this command is a skip. It's the
% \emph{additional} space between two lines in the output of a listing.
% The default is 0pt.
%
% \DescribeMacro\lstbaseem
% Here is one more command, which should be used in driver files only,
% since the one and only parameter is language specific. But you can
% use it for adjustment in your document. The parameter gives the witdh
% one character takes in the output in units of 1em. The parameter is a
% floating number and not a \TeX{} dimension, i.e.\ the unit em is
% added by the listings package. If you specify \verb!\lstbaseem{1}!,
% i.e.\ 1em, characters will never overlap. And the output produced
% after \verb!\lstbaseem{0}! looks quite funny. Better try values near
% $0.6$.
%
%
% \subsection{Other languages}\label{sOtherLanguages}
%
% You have some possibilities to adapt this package to other languages.
% For our purpose here the main characteristics of a language are
% keywords, comments and strings.
%
% Let's say, we want to produce following output:
% \selectlisting{blank}
% \keywords{function,integer,begin,end,if,then,else,return,print}
% \DeclareCommentLine//\relax
% \begingroup\catcode`\"=12\stringizer{"}\endgroup
% \stringstyle{\ttfamily}\blankstringfalse
% \iffalse
%<*sample>
% \fi
% \begin{listing}
function Fib(integer n):integer;
// Function returns the n-th Fibonacci.
begin
   if (n<2) then return 1;
			else return Fib(n-1)+Fib(n-2);
end;

// Main program.
begin
   print("Fib(10) = ",Fib(10), "\n");
end.
% \end{listing}
% \iffalse
%</sample>
% \fi
% You already know about \cs{stringstyle} and \cs{blankstringfalse}.
% We assume \verb!\selectlisting{blank}!. And now comes all the rest
% to print such a language mixture.
% \DescribeMacro\keywords
% To set the keywords use the \cs{keywords} command. After
% \begin{verbatim}
%    \keywords{one,two,three,four,five,six,seven,eight,nine,ten,
%        eleven,twelve}\end{verbatim}
% exactly these twelve keywords are present. As you see, each two
% keywords are separated by a comma.
% \DescribeMacro\morekeywords
% If you want to add keywords only use \cs{morekeywords} like
% \begin{verbatim}
%    \morekeywords{maxreal,minreal} % add these keywords
%    \morekeywords{zero,MAXINT}     % add also\end{verbatim}
%
% \DescribeMacro\sensitivetrue
% Some languages are case sensitive, others are not. You can select
% this with \cs{sensitivetrue} and \cs{sensitivefalse}, respectively.
% \DescribeMacro\sensitivefalse
% This package will then use the right keyword test.
%
% \DescribeMacro\stringizer
% The command \cs{stringizer} changes the character, which begins and
% ends a string. If your parameter consists of several characters,
% each character can start a string. But the string must be terminated
% with the same character. Using a letter, a digit or the underbar as
% stringizer will not work. The Modula-2 option defines the stringizer
% by
% \begin{verbatim}
%    \catcode`\"=12
%    \stringizer{'"}\end{verbatim}
% The catcode is changed (locally) for compatibility with
% \verb!german.sty!. The double quote is defined active there, but
% when we input a listing the double quote will have catcode 12
% (=other). The stringizer command has an optional argument to select
% whether the stringizer is doubled (default, e.g.\ Pascal) or preceded
% by a backslash (e.g.\ C) to insert the stringizer itself at the current
% position. The arguments are \verb!d! and \verb!b!, respectively, e.g.\ 
% \begin{verbatim}
%    \stringizer[d]{'} % doubled stringizer inserts stringizer
%    \stringizer[b]{"} % backslashed stringizer inserts\end{verbatim}
%
% Let's come to comments. For our purpose a 'comment line' is a comment,
% which starts with a character sequence and goes upto the end of a
% line. 'Comments' start and end with given character sequences.
% Comment lines need not to begin at the first column and comments
% can start and end wherever you want.
%
% \DescribeMacro\DeclareCommentLine
% \cs{DeclareCommentLine} has one parameter, which is terminated by
% \cs{relax}, i.e.\ 
% \begin{verbatim}
%    \DeclareCommentLine //\relax\end{verbatim}
% is legal. The parameter is a (nearly) arbitrary character sequence,
% which separates a comment line. Using a letter, a digit or the
% underbar will not work. And: When we input a listing, the characters
% must have the same category codes. Refer section
% \ref{SpecialCharacters} and \ref{ssSpecialCharacters}.
% If you don't want any comment lines, let the parameter empty, but
% don't forget the \cs{relax}!
%
% \DescribeMacro\DeclareSingleComment
% Four types of comments are supported: \textbf{single} for languages
% with one kind of comment, e.g.\ C or C++; \textbf{double} for
% \DescribeMacro\DeclareDoubleComment
% languages with two kinds of comments, e.g.\ Pascal comments enclosed
% \DescribeMacro\DeclareNestedComment
% within \verb!(*! and \verb!*)! and within \verb!{! and \verb!}!;
% \textbf{nested} for languages with one kind of comment, which can be
% \DescribeMacro\DeclarePairedComment
% nested, e.g.\ Modula-2. A \textbf{paired} comment is a single comment,
% but the two comment delimiters are the same and therefore occur paired.
% The declaration of such comments is similar to comment lines.
% The character sequences are separated by one blank space, e.g.\ 
% \begin{verbatim}
%    \DeclareSingleComment /* */\relax
%    \DeclareDoubleComment (* *) { }\relax
%    \DeclareNestedComment (* *)\relax
%    \DeclarePairedComment #\relax\end{verbatim}
% But hold on: When we input a listing, the characters must have the
% same category codes. So the examples won't work without changing
% them. See section \ref{ssSpecialCharacters}. If you don't want
% comments, let the second parameter of \cs{DeclareSingleComment}
% empty, i.e.\ type something like
% \begin{verbatim*}
%    \DeclareSingleComment stuff \relax\end{verbatim*}
% If a supported language already uses your favourite keywords,
% stringizer and/or comments, you are free to select that language
% and adjust only the wrong data.
%
%
% \subsection{Troubleshooting}
% After receiving an e-mail from Andreas Bartelt\footnote{
% Andreas.Bartelt@Informatik.Uni-Oldenburg.DE} the idea comes up
% to write this section. It is intend to help you handling unusual
% situations.
% \begin{itemize}
% \item	If you have problems using \cs{inputlisting} within \LaTeX{}
%		environments or commands, put the listing in a \cs{vbox}.
% \item	A source line might exceeds the textwidth, as shown in section
%		\ref{ssFigureOutTheAppearance}. Put the listing in a \cs{hbox}
%		to decrease the logical width like
% \begin{verbatim}
%    \hbox to 3.5in{
%        \vbox{\inputlisting{testfile.pas}}
%        \hss}\end{verbatim}
%		The \cs{hss} avoids an overfull \cs{hbox}. You can use that
%		construction together with \cs{fbox} to get something like
% \iffalse
%<*sample>
% \fi
% \setbox0=\vbox{\begin{listing}
function Fib(integer n):integer;
// Function returns the n-th Fibonacci.
begin
   if (n<2) then return 1;
			else return Fib(n-1)+Fib(n-2);
end;
% \end{listing}
% }
% \iffalse
%</sample>
% \fi
%		$$\fbox{\hbox to 3.5in{\box0\hss}}$$
% \item	The example just presented uses \cs{inputlisting} within an
%		argument of a command like \cs{fbox}. It's not possible with the
%		listing environment! To go around this type
% \begin{verbatim}
%    \setbox0=\vbox{
%    \begin{listing}
%function Fib(integer n):integer;
%...
%    \end{listing}
%    }% not on the same line as \end{listing}!\end{verbatim}
%		and then \verb!$$\fbox{\hbox to 3.5in{\box0\hss}}$$! to get
%		the example.
% \end{itemize}
%
%
% \subsection{It's not all fine}\label{ssItsNotAllFine}
%
% Please be patient of my English.
%
% Not all languages have been tested fully, since I don't know all
% languages well. Hope I haven't forgotten any important feature.
% Tell me, if you want additional languages or more options for a
% language already present. What's about Delphi, Prolog or Reduce?
% \begin{itemize}
% \item	If you use italic cmr fonts (comments), the dollar \${} comes
%		out as \pounds. Change italic to slanted (\cs{slshape})
%		or don't use cmr fonts.
% \item	If you use \cs{DeclarePairedComment}, the first comment delimiter
%		appears not in commentstyle, e.g.\ \#\textit{ comment \#}.
% \end{itemize}
% You're not lucky about the listings package? You found errors?
% Damn and blast it or contact me:
% \begin{verbatim}
%    cheinz@wmpi04.math.uni-wuppertal.de\end{verbatim}
%
%
% \StopEventually{}
%
%
% \section{How the package works}
%
% What happens when a user calls \cs{inputlisting}? First we set up a
% bit (execute prelisting, open input file, do other initialization)
% and at the end we have to close the input file for example.
%
%
% \subsection{The main idea}\label{TheMainIdea}
% The interesting part is the loop between: The source file is read,
% processed and typeset line by line. Before looking closer we have to
% deal with the alignment of columns, since this influences processing
% and typesetting. The problem: I don't like listings, which are
% printed with typewriter fonts. But only in these fonts all characters
% have the same width. So we have to handle different widths.
% The Pascal source lines
% \begin{verbatim}
%    if x=y then write('alignment')
%           else print('alignment');\end{verbatim}
% shouldn't come out as
% \begin{center}\hfill
% \begin{tabular}{l}
%	if\ x=y\ then\ write('alignment')\\
%	\ \ \ \ \ \ \ else\ print('alignment');
% \end{tabular}
% \hfill or \hfill
% \begin{tabular}{l@{\space}l@{\space}l}
%    \textbf{if} x=y&\textbf{then}&\textbf{write}('alignment')\\
%                   &\textbf{else}&print('alignment');
% \end{tabular}\hfill
% \end{center}
% only because blank spaces are not wide enough or because a bold
% letter is wider than a normal one.
% There is a simple trick to avoid things like that. We make boxes of
% the same width and put one character in each box:
% \begin{center}
%     \vbox{\def\makeboxes#1{\fbox{\hbox to 1em{\hss\vphantom{fy}#1\ignorespaces%
%         \hss}}\ifx#1\relax\else\expandafter\makeboxes\fi}
%     \makeboxes if\ x=y\ then\ write\relax\space\ldots\\
%     \makeboxes \ \ \ \ \ \ \ else \ print\relax\space\ldots}
% \end{center}
% Going this way the alignment of columns can't be disturbed. But if
% the boxes are not wide enaugh, we get
% \begin{center}
%     \def\makeboxes#1{\hbox{\hbox to 0.45em{\hss\vphantom{fy}#1\ignorespaces%
%         \hss}}\ifx#1\relax\else\expandafter\makeboxes\fi}
%     \begin{tabular}{l}
%     \makeboxes if\ x=y\ then\ write\relax\space\ldots
%     \end{tabular}
% \end{center}
% And choosing the width so that the widest character fits in, leads to
% \begin{center}
%     \def\makeboxes#1{\hbox to 1em{\hss\vphantom{fy}#1\hss}\ignorespaces%
%         \ifx#1\relax\else\expandafter\makeboxes\fi}
%     \begin{tabular}{l}
%     \makeboxes if\ x=y\ then\ write\relax\space\ldots
%     \end{tabular}
% \end{center}
% Both are not acceptable. So there is more to do. Each input line
% will be cut up in units. Since we want to scan for keywords, this
% is no extra work. In the example the units are
% \begin{center}\begin{tabular}{ccccccc}
%     if, & x, & =, & y, & then, & write, & \ldots
% \end{tabular}\end{center}
% and the blank spaces between. We put each unit in a box, which width
% is multiplied by the number of characters we put in, of course. The
% result is
% \begin{center}
%     \def\makeboxes#1#2{\fbox{\hbox to #1em{\hss\vphantom{fy}#2\hss}\ignorespaces%
%         }\ifx#2\relax\else\expandafter\makeboxes\fi}
%     \begin{tabular}{l}
%     \makeboxes2{i\hss f}1{\ }1{x}1{=}1{y}1{\ }4{t\hss h\hss e\hss n}%
%         1{\ }5{w\hss r\hss i\hss t\hss e}1\relax\space\ldots
%     \end{tabular}
% \end{center}
% Now we are ready to choose the base width of a box --- that's
% \cs{lstbaseem}. Consider the 'write'. The 'w' needs more space than
% base width, but the 'i' needs less. This will compensate each other,
% because we put them in the same box. In general: Since wide characters
% use space actually reserved for thin characters, the base width need
% not to be the width of the widest character. This would be 1em
% (a \cs{quad}). At the moment most languages use the empirical value
% 0.6em. It's a compromise between overlapping characters and the
% number of boxes exceeding not the textwidth, i.e.\ how many
% characters fit a line without getting an overfull \verb!\hbox!.
%
% 
% \subsection{Line processing}
% Let's discuss the line processing. The job is to classify comments,
% strings and other source code, where we search for keywords. We do
% that in several steps.
% \begin{itemize}
% \item First we pay attention on comments, which started in preceding
%	lines, e.g.\ 
%	\begin{verbatim}
%    comment } for i:=1 to maxint do ...\end{verbatim}
%	In that case we look for the end of comment and output the comment.
%	If there is other source code left, we run the line processing on
%	that code.
% \item If no comment is in work, we look for one and split the input
%	into comment and other source code. Since the other code comes
%	first, we give it to a routine, which cuts it into the units
%	described above (tokenizing). The output takes place there.
% \item Afterwards we have to output the comment. But when we look for
%	a comment, we don't take notice of other things. Consider
%	\begin{verbatim}
%    writeln('This is a string { and not a comment }');\end{verbatim}
%	We cut this line at the left brace into other code and comment.
%	After typesetting the other code we notice, that a string started,
%	but hasn't been finished. Our first classification was wrong.
%	We use the comment as new input and simply run the line processing
%	again to continue the string.
% \item If all strings are finished, we can output the comment, since
%	it's really a comment.
% \end{itemize}
% This is a sketch of line processing. For more detailed information
% see section \ref{ssLineProcessing}, where the implementation is
% described.
%
% One question about the third item is left: Why do we cut off a
% comment, even if we don't know that it is a comment? It's a matter
% of modularity and extension. Consider the alternative: The tokenize
% routine below (where we cut the classified input into the smallest
% units) would have to look for the start and end of comment.
% To support different languages we have to write either more than one
% tokenize routine or one, which handles comments of all languages.
% Adding another language means extending tokenizing or writing a new
% routine, which looks only for the particular comment. But finding a
% comment has nearly nothing to do with looking for delimiters (blank
% space, comma, plus, bracket, etc.). So we write exactly one tokenize
% macro and separate this from finding a comment.
%
%
% \subsection{Tokenizing}
% The next step is breaking up the parts produced by line processing
% into the units mentioned above. We gather all characters, until
% reaching a nonletter. Then we output the collected characters using
% a routine, which selects the right style (comment, keyword, string).
% Afterwards we handle the nonletter:
%
% If it is a stringizer, we switch a string boolean to indicate whether
% we are in string mode or not. This boolean is also used by the line
% processing.
%
% If we found a tabulator, we go to the next tabulator stop. The output
% routines are responsible for knowing the number of blanks we need.
%
% If it's any other nonletter, we gather all coming nonletter (upto the
% next letter, of course) and output these nonletter. Then we start
% with a letter again, \ldots
%
%
% \subsection{Special characters}\label{SpecialCharacters}
% As you know some characters have a special meaning to \TeX, e.g.\ the
% subscript \verb!_! or the backslash \verb!\!. \TeX{} realizes this
% using so called category codes (catcodes). \TeX{} knows sixteen
% different ones:
% \begin{center}\begin{tabular}{rll}
%	 0 & escape delimiter (backslash)& \verb!\catcode`\\=0!\\
%	 1 & beginning of a group ($\{$)& \verb!\catcode`\{=1!\\
%	 2 & ending of a group ($\}$)	& \verb!\catcode`\}=2!\\
%	 3 & mathematics shift ($\$$)	& \verb!\catcode`\$=3!\\
%	 4 & tabulator ($\&$)		& \verb!\catcode`\&=4!\\
%	 5 & end of line (carriage return)& \verb!\catcode`\^^M=5!\\
%	 6 & macro parameter (\verb!#!) & \verb!\catcode`\#=6!\\
%	 7 & superscript (\verb!^!)	& \verb!\catcode`\^=7!\\
%	 8 & subscript (\verb!_!)	& \verb!\catcode`\_=8!\\
%	 9 & ignore			& \verb!\catcode`\^^@=9!\\
%	10 & blank space		& \verb!\catcode`\ =10!\\
%	11 & letter			& A..Z,a..z\\
%	12 & other			& \\
%	13 & active (e.g.\ \verb!~!)	& \\
%	14 & comment			& \verb!\catcode`\%=14!\\
%	15 & invalid characters		&
% \end{tabular}\end{center}
% When \TeX{} reads a character, it looks up the associated catcode and
% treats the character accordingly, e.g.\ an active character stands
% for a command without a leading backslash (like \verb!~! or the
% double quote in \verb!german.sty!).
%
% We can't change catcodes of characters, which \TeX{} has already
% read. But we are able to change them before. After
% \verb!\catcode`\A=10! an upper 'A' has the same effect as a blank
% space! Consider the \TeX source
% \begin{verbatim}
%    \def\uppera{A} % saving upper 'A'; we shouldn't use \upperA. Why?
%    \catcode`\A=10 % 'A' becomes blank space. That's the reason!
%    MayAbeAaAcuriousAexample. \upperaAnd it work's.\end{verbatim}
% The output is
% \begin{verbatim}
%    May be a curious example. And it work's.\end{verbatim}
% First we save the 'A'. Changing the catcode has no effect on the
% definition in the first line, since \TeX{} has read it before. All
% coming 'A' are blank spaces.
%
% We want to input and typeset characters 'as they are'. But the
% subscript or the macro parameter character are not printable. So it
% is necessary to change catcodes. We make these characters active.
% We also change catcodes of printable characters, e.g.\ a digit
% (normally other) will be treated as a letter, since indentifiers like
% \verb!x1!, \verb!y2! are legal. Then it is possible to gather all
% characters upto a nonletter to cut up a unit!
%
% We change catcodes for input and output. That's clear. But we also
% do it to define some commands. One example: In Pascal we have to look
% for \verb!{!. Let's say, we compare it with a left brace from a
% source file. This only works, if both have the same meaning.
% And when we declare the comment, \TeX{} must know the associated
% catcode, since it can't be changed later.
% Hence, we have to change it also at the time of defining the macros.
%
% There is at least one more thing to say: Catcodes aren't easy to
% work with.
%
%
% \section{General \TeX{}niques}
%
% \subsection{Macro parameter parsing}
% A macro has at most nine parameters. They are numbered consecutively
% and are referenced by $\#1,\ldots,\#9$. In the definition they appear
% after the macro name, e.g.\ 
% \begin{verbatim}
%    \def\macro#1#2#3{\textbf{#1}\texttt{#2}\textit{#3}}\end{verbatim}
% has three parameters. \verb!\macro the! and \verb!\macro{t}{h}{e}!
% both lead to the output '\textbf{t}\texttt{h}\textit{e}'.
% Parameters with more than one character must be enclosed in braces.
% If you call the macro with less than three parameters, you will get
% an error message (hopefully most times), since the three parameters
% are the calling syntax of the macro. That's all well known.
%
% But \TeX's macro processor is more powerful. Nearly any character
% sequence and parameter mixture can be used for the calling syntax,
% e.g.\ 
% \begin{verbatim}
%    \def\macro z=(#1,#2){ ... }\end{verbatim}
% The call \verb!\macro{10}{0}! is illegal, but \verb!\macro z=(10,0)!
% is ok. The parameters are '10' and '0', respectively.
%
% This mechanism can be used to check whether a character sequence
% is a substring of another sequence --- which will be used to
% extract comments and keywords, but this comes later. Consider
% \begin{verbatim}
%    \def\macro #1substring#2\relax{ ... }\end{verbatim}
% If we call this macro, we should always add \verb!substring\relax!
% at the end, since the sequence belongs to the syntax of the macro:
% \begin{verbatim}
%    \macro This is our first call.substring\relax\end{verbatim}
% Here the second parameter of the macro will be empty, but using
% \begin{verbatim}
%    \macro This is a substring example.substring\relax\end{verbatim}
% the second parameter is not empty, since the first 'substring'
% terminates the first parameter and lets the second begin after
% (upto the closing \cs{relax}). The character sequence 'substring'
% is a substring of the preceding character sequence, if and only if
% the second parameter is not empty.
%
% Later we let the preceding character sequence be a list of keywords
% to test whether a character sequence is a keyword or not.
% Or we let the preceding character sequence be the current soure line
% and replace 'substring' by  '//' to look for a C++ comment line.
%
%
% \subsection{Quick 'if parameter empty'}
% There are many situations where you have to look whether a macro
% parameter is empty or not. Let's say, we want to test the first
% parameter, which is refered by $\#1$. The \emph{natural} way would
% be something like
% \begin{verbatim}
%    \def\test{#1}%
%    \ifx \test\empty %
%            % #1 is empty
%    \else %
%            % #1 is not empty
%    \fi %\end{verbatim}
% where \cs{empty} is defined by \verb!\def\empty{}!, of course.
% And now the \emph{mad} way:
% \begin{verbatim}
%    \ifx \empty#1\empty %
%            % #1 is empty
%    \else %
%            % #1 is not empty
%    \fi %\end{verbatim}
% Having an empty parameter the \cs{empty} left from $\#1$ is compared
% with the \cs{empty} on the right. Since they are the same, it's all ok.
% If the parameter is not empty, the \cs{empty} on the left is compared
% with the first token of the parameter. Assuming this token is not
% equivalent to \cs{empty} the \cs{else} section is executed as desired.
% The paramater must not be the macro \cs{empty}, e.g.
%
% The mad way works, if and only if the first token of the parameter is
% not equivalent to \cs{empty}. You must check, if this meets your
% purpose. The two \cs{empty}s might be replaced by any other macro,
% which is not equivalent to the first token of the parameter.
% But the definition of that macro shouldn't be too complex, since
% this slows down the \cs{ifx}. Consider
% \begin{verbatim}
%    \def\test{#1}\ifx \test\empty \else \fi % natural\end{verbatim}
% and
% \begin{verbatim}
%    \ifx \empty#1\empty \else \fi % mad\end{verbatim}
% The mad version needs about $45\%$ of the natural's time.
%
%
% \section{Implementation}
%
% Before considering the implementation, here some conventions I used:
% \begin{itemize}
% \item The names of all public macros (the user commands) have lower
%	case letters. (That's not true: \cs{DeclareCommentLine}, \ldots,
%	but consistent with the third item.)
% \item The name of all private macros and variables have prefixes:
%	\verb!lst@! for a general macro or variable, \verb!lstdrv@! when
%	it is defined in a driver file, and \verb!lstenv@! if it is
%	defined for the listing environment.
% \item To distinguish procedure-like macros from macros holding data,
%	the name of procedure macros use upper case letters with each
%	beginning word, e.g.\ \verb!\lst@ProcessLine!. (The only exception
%	is the 'procedure-macro' \verb!\lst@ifoneof!.)
% \end{itemize}
%
%
% \subsection{Registers and variables}\label{ssRegisters}
% The current version needs 1 read register, 5 counters, 2 dimensions
% and 1 token register. The counter \verb!\@tempcnta! and the dimension
% \verb!\@tempdima! are also used, see the index.
%
% \begin{macro}{\lst@inputfile}
% \begin{macro}{\lst@ifendinput}
% \begin{macro}{\iflisting}
% \verb!\iflisting! is described in the user's guide. The input file
% is \verb!\lst@inputfile!. The output of the file terminates, if and
% only if the boolean \verb!\lst@ifendinput! is true.
%    \begin{macrocode}
%<*package>
\newif\iflisting
\newread\lst@inputfile
\def\lst@endinputtrue{\let\lst@ifendinput\iftrue}
\def\lst@endinputfalse{\let\lst@ifendinput\iffalse}
%    \end{macrocode}
% \end{macro}\end{macro}\end{macro}
%
% \begin{macro}{\lst@lastno}
% \begin{macro}{\lst@lineno}
% The user specified last line and the current input line are kept
% in two counters:
%    \begin{macrocode}
\newcount\lst@lastno \newcount\lst@lineno
%    \end{macrocode}
% \end{macro}\end{macro}
%
% \begin{macro}{\lst@ifcomment}
% \begin{macro}{\lst@ifstring}
% \begin{macro}{\lst@commentdepth}
% Comments (not whole comment lines) and strings are indicated by
% following booleans. The current comment depth uses a counter.
%    \begin{macrocode}
\newcount\lst@commentdepth
\def\lst@commenttrue{\let\lst@ifcomment\iftrue}
\def\lst@commentfalse{\let\lst@ifcomment\iffalse}
\def\lst@stringtrue{\let\lst@ifstring\iftrue}
\def\lst@stringfalse{\let\lst@ifstring\iffalse}
%    \end{macrocode}
% \end{macro}\end{macro}\end{macro}
%
% \begin{macro}{\lst@width}
% \begin{macro}{\lst@length}
% \begin{macro}{\lst@pos}
% The dimension \verb!\lst@width! holds the width of a single character
% box and is set to \verb!\lst@baseem em!, when we output a listing.
% The counter \verb!\lst@length! holds the length of the current
% character string we want to output.
% \verb!\lst@pos! holds the current column number to handle tabulators.
%    \begin{macrocode}
\newdimen\lst@width \newcount\lst@length \newcount\lst@pos
%    \end{macrocode}
% \end{macro}\end{macro}\end{macro}
%
% \begin{macro}{\lst@halfspread}
% That's the half of spreadlisting's argument.
%    \begin{macrocode}
\newdimen\lst@halfspread
%    \end{macrocode}
% \end{macro}
%
% \begin{macro}{\lst@line} \begin{macro}{\lst@comment}
% \begin{macro}{\lst@commentline}
% Some variables need no allocation here, since they are macros, e.g.\ 
% most user commands save their parameter within a macro. The most
% important 'data' macros are \verb!\lst@line!, \verb!\lst@comment!
% and \verb!\lst@commentline!. They hold the current line, comment and
% comment line for line processing.
% \end{macro}\end{macro}\end{macro}
%
% \begin{macro}{\lst@other}
% This version gather non-alphanumeric characters before their output.
% Since these characters might expand, we need a token register to
% gather them, where they surely do not expand.
%    \begin{macrocode}
\newtoks\lst@other
%    \end{macrocode}
% \end{macro}
%
%
% \subsection{Comments}\label{ssComments}
%
% First we discuss comment lines (CL). Let's say, we have a C++ source
% line with a comment, which is separated by '//'. To cut the line into
% code and comment we use a macro like this:
% \DescribeMacro\lst@CutCL
% \begin{verbatim}
%    \def\lst@CutCL#1//#2\relax{%
%        \def\lst@line{#1}\def\lst@commentline{//#2}}\end{verbatim}
% The parameter definition ensures that the first parameter holds the
% code and our second paramater the comment upto a \cs{relax}, which
% marks the end of the source. We (re)define the input and comment line,
% where we must add the separator '//' again. That's easy.
% \DescribeMacro\lst@CommentLine
% But if we call this macro using a line without any '//', \TeX{}
% detects a syntax error, since the line doesn't meet the definition.
% Hence, we can't use the cut macro to look for a comment line,
% but this:
% \begin{verbatim}
%    \def\lst@CommentLine{%
%        \expandafter\lst@TestCL\lst@line//\relax}\end{verbatim}
% Additional '//' and '\cs{relax}' hold up the syntax of the cut macro.
% That's the point. We use an \cs{expandafter}, because we need the
% \emph{content} of \verb!\lst@line! and not the macro token itself as
% parameter. So the test macro is called (or expanded) after the
% content is written behind.
%
% \DescribeMacro\lst@TestCL
% Why we don't use the cut macro here? Our additional '//' would be
% added to the comment! Therefore:
% \begin{verbatim}
%    \def\lst@TestCL#1//#2\relax{%
%        \ifx\empty#2\empty %
%            \let\lst@commentline\empty %
%        \else %
%            \expandafter\lst@CutCL\lst@line\relax %
%        \fi}\end{verbatim}
% If you compare the definition of \verb!\lst@TestCL! here and the
% call above, you will see: The second parameter is empty, if and only
% if there is no comment (no double slash). So we are able to call
% the cut macro without additional '//' if we've found a comment.
%
% \begin{macro}{\DeclareCommentLine}
% \begin{macro}{\lst@CommentLine}
% \begin{macro}{\lst@TestCL}
% \begin{macro}{\lst@CutCL}
% Now comes the real implementation. It's different from the C++
% example, but it only seems to. The one and only parameter of
% \cs{DeclareCommentLine} is the comment line separator.
% First we test, if comment lines are desired. If not we let the main
% macro be empty. Otherwise we define three macros similar to the
% example. But: To distinguish their parameters from the comment line
% separator $\#1$, we have to double the '$\#$': The $n$th parameter
% of a macro within a macro is $\#\#n$. If you know this, following
% is clear: It's the same as the simplified example.
%    \begin{macrocode}
\def\DeclareCommentLine#1\relax{%
  \ifx\@empty#1\@empty \let\lst@CommentLine\@empty %
  \else
      \def\lst@CommentLine{\expandafter\lst@TestCL\lst@line#1\relax}%
      \def\lst@TestCL##1#1##2\relax{%
          \ifx\@empty##2\@empty \let\lst@commentline\@empty %
          \else \expandafter\lst@CutCL\lst@line\relax %
          \fi}%
      \def\lst@CutCL##1#1##2\relax{%
          \def\lst@line{##1}\def\lst@commentline{#1##2}}%
  \fi}
%    \end{macrocode}
% \end{macro}\end{macro}\end{macro}\end{macro}
%
% \begin{macro}{\lst@DefineSingleComment}
% \begin{macro}{\lst@SOC}
% \begin{macro}{\lst@EOC}
% For single comments we use the same cut mechanism as above, except
% that we use it for the start of comment (SOC) and for the end of
% comment (EOC): We define six macros here. Additional the comment
% boolean is switched by the cut macros. The last two parameters of
% \verb!lst@DefineSingleComment! are the character sequences for the
% beginning and ending of a single comment, of course. The first
% parameter is used as suffix for the macro names.
%    \begin{macrocode}
\def\lst@DefineSingleComment[#1]#2 #3\relax{%
  \ifx\@empty#3\@empty %
    \expandafter\let\csname lst@SOC#1\endcsname\relax %
    \expandafter\let\csname lst@EOC#1\endcsname\relax %
  \else %
    \expandafter\def\csname lst@SOC#1\endcsname{\expandafter %
        \expandafter\csname lst@TestSOC#1\endcsname \lst@line#2\relax}%
    \expandafter\def\csname lst@TestSOC#1\endcsname##1#2##2\relax{%
        \ifx\@empty##2\@empty\else \expandafter %
            \expandafter\csname lst@CutSOC#1\endcsname \lst@line\relax %
        \fi}%
    \expandafter\def\csname lst@CutSOC#1\endcsname##1#2##2\relax{%
        \lst@commenttrue \def\lst@line{##1}\def\lst@comment{#2##2}}%
    \expandafter\def\csname lst@EOC#1\endcsname{\expandafter %
        \expandafter\csname lst@TestEOC#1\endcsname \lst@line#3\relax}%
%    \end{macrocode}
%    \begin{macrocode}
    \expandafter\def\csname lst@TestEOC#1\endcsname ##1#3##2\relax{%
          \ifx\@empty##2\@empty %
%    \end{macrocode}
% Here we are in the situation that we look for the end of comment but
% haven't found one. Since we are in comment mode, the whole line is
% the comment and the line gets empty.
%    \begin{macrocode}
              \let\lst@comment\lst@line \let\lst@line\@empty %
          \else \expandafter %
              \expandafter\csname lst@CutEOC#1\endcsname\lst@line\relax%
          \fi}%
    \expandafter\def\csname lst@CutEOC#1\endcsname##1#3##2\relax{%
        \lst@commentfalse \def\lst@line{##2}\def\lst@comment{##1#3}}%
  \fi}
%    \end{macrocode}
% \end{macro}\end{macro}\end{macro}
%
% \begin{macro}{\DeclareSingleComment}
% That's an abbreviation of the macro just defined, but where the
% suffix-parameter is empty.
%    \begin{macrocode}
\def\DeclareSingleComment{\lst@DefineSingleComment[]}
%    \end{macrocode}
% \end{macro}
%
% \begin{macro}{\DeclarePairedComment}
% First we define a single comment. Then we adjust one cut macro:
% The SOC delimiter is defined to belong to the line and not to the
% comment.
%    \begin{macrocode}
\def\DeclarePairedComment#1\relax{%
    \lst@DefineSingleComment[]#1 #1\relax %
    \def\lst@CutSOC##1#1##2\relax{%
        \lst@commenttrue \def\lst@line{##1#1}\def\lst@comment{##2}}}%
%    \end{macrocode}
% \end{macro}
%
% \begin{macro}{\DeclareDoubleComment}
% We define a single comment and another one with slightly different
% names (additional \verb!@!). Then we compose the two single comments.
%    \begin{macrocode}
\def\DeclareDoubleComment#1 #2 #3 #4\relax{%
    \lst@DefineSingleComment[]#1 #2\relax %
    \lst@DefineSingleComment[@]#3 #4\relax %
    \lst@HookComment#1 #3 \lst@TestSOC@\lst@EOC@\lst@SavedEOC}%
%    \end{macrocode}
% \end{macro}
%
% \begin{macro}{\lst@HookComment}
% The purpose of this macro is to compose a SOC/EOC pair already
% defined with another SOC/EOC pair, which need not to be defined yet.
% The job is to coincide start and end of comment: We have to look
% whether the first, the second or none comment starts, and must
% select the right macro to find the corresponding end of comment.
% The first and second parameter are the SOC delimiters of the first
% (defined) and second SOC/EOC pair, respectively. Therefor they are
% separated by blank spaces. The third and fourth parameters are the
% \verb!TestSOC! and \verb!EOC! macros of the second SOC/EOC pair.
% Parameter five is a macro name used to save a EOC macro. That's done
% in the second line:
%    \begin{macrocode}
\def\lst@HookComment#1 #2 #3#4#5{%
    \let#5\lst@EOC %
    \def\lst@TestSOC##1#1##2\relax{%
%    \end{macrocode}
% Now $\#\#1$ holds the code of the line and $\#\#2$ (possibly) the
% comment. But we have to look whether the other comment type starts
% before this here or not. So we call the test macro of the other
% comment first. If we've found one, we let \verb!\lst@EOC! be the
% macro, which looks for the end of that comment.
%    \begin{macrocode}
        #3##1#2\relax %
        \lst@ifcomment \let\lst@EOC#4%
%    \end{macrocode}
% Otherwise we do the test for this comment. If there is a comment,
% we cut up the line and let \verb!\lst@EOC! be the macro, which
% looks for the end of this comment.
%    \begin{macrocode}
        \else %
            \ifx\@empty##2\@empty \else %
                \expandafter\lst@CutSOC\lst@line\relax %
                \let\lst@EOC#5%
            \fi %
        \fi}}
%    \end{macrocode}
% \end{macro}
%
% \begin{macro}{\DeclareNestedComment}
% For nested comments we do it the same way: We declare a single
% comment and adjust some macros. Testing for an end of comment is
% nearly the same as for single comments.
%    \begin{macrocode}
\def\DeclareNestedComment#1 #2\relax{%
    \lst@DefineSingleComment[]#1 #2\relax %
    \def\lst@TestEOC##1#2##2\relax{%
%    \end{macrocode}
% The only difference is that we have to count how many (new) comments
% start in the comment. And this is done here, the rest is the same as
% above.
%    \begin{macrocode}
        \lst@CountSOC##1#1\relax %
        \ifx\@empty##2\@empty %
            \let\lst@comment\lst@line \let\lst@line\@empty %
        \else %
            \expandafter\lst@CutEOC\lst@line\relax %
        \fi}%
%    \end{macrocode}
% An end of comment lets the comment depth decrease. If we reach 0,
% we set the comment boolean false. That's not new, since this happens
% always with not nested comments.
%    \begin{macrocode}
    \def\lst@CutEOC##1#2##2\relax{%
        \advance\lst@commentdepth by -1\relax %
        \ifnum 0=\lst@commentdepth \lst@commentfalse\fi %
        \def\lst@line{##2}\def\lst@comment{##1#2}}%
%    \end{macrocode}
% Now we have to do something for increasing the comment depth. We
% count how many new comments start within a comment. To move through
% a comment we use the same separation mechanism as all the time. The
% second parameter is empty, if and only if there is no (more) start
% of comment. While it is not empty we increase the comment depth and
% call this macro again with the current rest of the comment.
%    \begin{macrocode}
    \def\lst@CountSOC##1#1##2\relax{%
        \ifx\@empty##2\@empty\else %
            \advance\lst@commentdepth by 1%
            \def\@tempa{\lst@CountSOC##2\relax}%
%    \end{macrocode}
% Note: \verb!\@tempa! is executed after the closing \verb!\fi! to call
% the macro again.
%    \begin{macrocode}
            \expandafter\@tempa %
        \fi}}
%    \end{macrocode}
% \end{macro}
%
% \begin{macro}{\DeclareCLPercent}
% This command is equivalent to \verb!\DeclareCommentLine %\relax!,
% where the percent has the catcode we need when we input a listing.
%    \begin{macrocode}
{\catcode`\%=12 \gdef\DeclareCLPercent{\DeclareCommentLine %\relax}}
%    \end{macrocode}
% \end{macro}
%
% \begin{macro}{\DeclareDoubleCommentPascal}
% Dito for Pascal comments.
%    \begin{macrocode}
\begingroup \catcode`\[=1 \catcode`\]=2
\catcode`\{=\active \catcode`\}=\active \catcode`\*=\active
\gdef\DeclareDoubleCommentPascal[\DeclareDoubleComment (* *) { }\relax]
\endgroup
%    \end{macrocode}
% \end{macro}
%
%
% \subsection{Keywords}\label{ssKeywords}
%
% \begin{macro}{\lst@CaseSensitiveKeywords}
% We have to decide whether a given character sequence is a reserved
% word or not. Doing this test is very familiar with the cut mechanism
% for comments. To test, if \verb!key! is a keyword, we define a macro
% \begin{verbatim}
%    \def\lst@test#1,key,#2\relax{...}\end{verbatim}
% Afterwards we call the macro with the parameters
% \begin{center}\begin{tabular}{l@{}l}
%    all current keywords&\verb!,key,\relax!
% \end{tabular}\end{center}
% When \TeX{} passes the arguments, the second parameter is empty,
% if and only if \verb!key! is not a current keyword. So we are able to
% decide whether to make a normal box, or a box with keywordstyle.
%    \begin{macrocode}
\def\lst@CaseSensitiveKeywords#1\relax{%
    \def\lst@test##1,#1,##2\relax{%
        \ifx \@empty##2\@empty \lst@MakeBox{}%
        \else \lst@MakeBox{\lst@keywordstyle}%
        \fi}%
%    \end{macrocode}
% We only need to call the macro defined right before. \verb!,#1,\relax!
% holds up the syntax of the macro.
%    \begin{macrocode}
    \expandafter\lst@test\lst@keywords,#1,\relax}%
%    \end{macrocode}
% Since \TeX{} always passes two arguments to the test macro, \TeX{}
% splits the whole 'input' \verb!\lst@keywords,#1,\relax! in two parts.
% So there is no need to sort the keywords by probability.\footnote{If
% you sort the keywords by probability and make a loop for the keyword
% tests, which terminates right after finding a keyword, you might think
% that's faster than the \TeX{}nique used here. Well, if your source
% code uses the three or four most common keywords only, you are right.
% Most cases it will be slow. In fact the versions 0.1 and 0.11 have
% used something like loops, even something faster, but which is much
% slower than this here.}
% \end{macro}
%
% \begin{macro}{\lst@NonCaseSensitiveKeywords}
% Now we implement the test for keywords, which are not case sensitive.
% We use two \cs{uppercase} to normalize the test string and the keywords.
% For the keywords we need some expandafters, so that the keywords are
% expanded before making the characters upper case.
%    \begin{macrocode}
\def\lst@NonCaseSensitiveKeywords#1\relax{%
    \uppercase{\def\lst@test##1,#1,##2\relax{%
        \ifx \@empty##2\@empty \lst@MakeBox{}%
        \else \lst@MakeBox{\lst@keywordstyle}%
        \fi}}%
    \expandafter\uppercase\expandafter{%
        \expandafter\lst@test\lst@keywords,#1,\relax}}%
%    \end{macrocode}
% \end{macro}
%
% \begin{macro}{\lst@IfOneOf}
% \begin{macro}{\lst@ifoneof}
% We define two macros, which are very familiar with the keyword tests.
% The first macro is a case sensitive version of the second one. The
% first parameter is terminated by \cs{relax}, the other three are not.
% If the first parameter is found in the second parameter (a keyword
% list) the third parameter is executed. Otherwise we perform the forth.
% The implementation is clear.
%    \begin{macrocode}
\def\lst@IfOneOf#1\relax#2{%
    \def\lst@test##1,#1,##2\relax{%
        \ifx \@empty##2\@empty \expandafter\@secondoftwo %
        \else \expandafter\@firstoftwo %
        \fi}%
    \lst@test,#2,#1,\relax}%
\def\lst@ifoneof#1\relax#2{%
    \uppercase{\def\lst@test##1,#1,##2\relax{%
        \ifx \@empty##2\@empty \expandafter\@secondoftwo %
        \else \expandafter\@firstoftwo %
        \fi}}%
    \uppercase{\lst@test,#2,#1,\relax}}%
%    \end{macrocode}
% \end{macro}\end{macro}
%
%
% \subsection{Commands}\label{ssCommands}
% \begin{macro}{\selectlisting}
% This command loads the specified driver file and selects the language.
% We give an error message, if the driver file doesn't support the
% necessary macro.
%    \begin{macrocode}
\newcommand\selectlisting[2][]{%
    \@ifundefined{lstdrv@#2@}{\input{lst#2.sty}}{}%
    \@ifundefined{lstdrv@#2@}{%
        \PackageError{Listings}{Driver file for `#2' corrupt}{%
        The driver file doesn't define \string\lstdrv@#2@.}}{%
    \@ifundefined{lstdrv@#2@#1}{%
        \PackageError{Listings}{Option `#1' not supported}{%
        The driver file doesn't define \string\lstdrv@#2@#1.}}{%
    \csname lstdrv@#2@#1\endcsname \def\lst@curr{#2}\def\lst@opt{#1}}}}
%    \end{macrocode}
% \end{macro}
%
% \begin{macro}{\lst@ifselect}
% \begin{macro}{\lst@ifoption}
% These two private macros are similar to \LaTeXe{}'s \verb!\@ifundefined!,
% except that the selected language and option are tested, of course.
%    \begin{macrocode}
\def\lst@ifselect#1{\def\lst@test{#1}%
    \ifx\lst@curr\lst@test \expandafter\@firstoftwo %
    \else \expandafter\@secondoftwo \fi}
\def\lst@ifoption#1{\def\lst@test{#1}%
    \ifx\lst@opt\lst@test \expandafter\@firstoftwo %
    \else \expandafter\@secondoftwo \fi}
%    \end{macrocode}
% \end{macro}\end{macro}
%
% \begin{macro}{\keywordstyle}
% \begin{macro}{\commentstyle}
% \begin{macro}{\stringstyle}
% \begin{macro}{\labelstyle}
% The following user commands save the parameter in (private) macros.
% The labelstyle command checks for a legal step count for labels.
%    \begin{macrocode}
\newcommand\keywordstyle[1]{\def\lst@keywordstyle{#1}}
\newcommand\commentstyle[1]{\def\lst@commentstyle{#1}}
\newcommand\stringstyle[1]{\def\lst@stringstyle{#1}}
\newcommand\labelstyle[2][1]{\def\lst@labelstyle{#2}%
    \ifnum #1>-1 \def\lst@labelstep{#1}\else %
    \PackageError{Listings}{Nonnegative integer expected}{%
    You can't use `#1' as step count for labels.^^J%
    I'll forget it and proceed.}\fi}
%    \end{macrocode}
% \end{macro}\end{macro}\end{macro}\end{macro}
%
% \begin{macro}{\tablength}
% \begin{macro}{\lstbaseem}
% \begin{macro}{\lstlineskip}
% \cs{tablength} and \cs{lstbaseem} also look, if the arguments are
% legal.
%    \begin{macrocode}
\newcommand\tablength[1]{\ifnum#1>0 \def\lst@tablength{#1}\else %
    \PackageError{Listings}{Strict positive integer expected}{%
    You can't use `#1' as tablength.^^J I'll forget it and proceed.}\fi}
\newcommand\lstbaseem[1]{\ifdim #1em>0pt \def\lst@baseem{#1}\else %
    \PackageError{Listings}{Strict positive number expected}{%
    You can't use `#1' as baseem.^^J I'll forget it and proceed.}\fi}
\newcommand\lstlineskip[1]{\def\lst@lineskip{#1}}
%    \end{macrocode}
% \end{macro}\end{macro}\end{macro}
%
% \begin{macro}{\prelisting}
% \begin{macro}{\postlisting}
% \begin{macro}{\spreadlisting}
% \begin{macro}{\keywords}
% \begin{macro}{\morekeywords}
% \begin{macro}{\stringizer}
% More 'parameter-saving' commands:
%    \begin{macrocode}
\newcommand\prelisting[1]{\def\lst@prelisting{#1}}
\newcommand\postlisting[1]{\def\lst@postlisting{#1}}
\newcommand\spreadlisting[1]{\lst@halfspread#1\relax %
    \lst@halfspread 0.5\lst@halfspread\relax}
\newcommand\keywords[1]{%
    \edef\lst@keywords{,\zap@space#1 \@empty}}
\newcommand\morekeywords[1]{%
    \edef\lst@keywords{\lst@keywords,\zap@space#1 \@empty}}
\newcommand\stringizer[2][d]{%
    \@ifundefined{lst@#1TestStringizer}{%
        \PackageError{Listings}{Illegal stringizer option `#1'}{%
        Available options are 'b' and 'd'.}}{%
    \expandafter\let\expandafter\lst@TestStringizer %
    \csname lst@#1TestStringizer\endcsname \def\lst@stringizer{#2}}}
%    \end{macrocode}
% \end{macro}\end{macro}\end{macro}\end{macro}
% \end{macro}\end{macro}
%
% \begin{macro}{\blankstringtrue}
% \begin{macro}{\blankstringfalse}
% \begin{macro}{\sensitivetrue}
% \begin{macro}{\sensitivefalse}
% The user switches assign the right macros.
%    \begin{macrocode}
\newcommand\blankstringtrue{%
    \let\lst@MakeStringBox\lst@MakeBox}
\newcommand\blankstringfalse{%
    \let\lst@MakeStringBox\lst@MakeSpecialStringBox}
\newcommand\sensitivetrue{%
    \let\lst@KeywordOrNot\lst@CaseSensitiveKeywords}
\newcommand\sensitivefalse{%
    \let\lst@KeywordOrNot\lst@NonCaseSensitiveKeywords}
%    \end{macrocode}
% \end{macro}\end{macro}\end{macro}\end{macro}
%
% \begin{macro}{\normallisting}
% The default style is set using the previous commands:
%    \begin{macrocode}
\newcommand\normallisting{%
    \keywordstyle{\bfseries}\commentstyle{\itshape}%
    \stringstyle{}\labelstyle[0]{}%
    \prelisting{}\postlisting{}%
    \spreadlisting{0pt}\blankstringfalse}
%    \end{macrocode}
% \end{macro}
%
% \begin{macro}{\listoflistings}
% At the end of this section we define commands for the list of
% listings. It's a derivation of \cs{listoffigures}.
%    \begin{macrocode}
\newcommand\listoflistings{%
    \ifx\chapter\undefined %
        \expandafter\section \else \expandafter\chapter %
    \fi *{\listlistingsname %
      \@mkboth{\MakeUppercase\listlistingsname}%
              {\MakeUppercase\listlistingsname}}%
    \@starttoc{lol}}
\newcommand\listlistingsname{Listings}
%    \end{macrocode}
% \end{macro}
%
% \begin{macro}{\lst@ListOfListingsEntry}
% And at the very end a command to add an entry to the list. The first
% parameter is the name of the listing, and the second is reserved for
% the line range, I think, but not used so far.
%    \begin{macrocode}
\newcommand\lst@ListOfListingsEntry[2]{%
    \ifx \@empty#1\@empty \else %
        \addtocontents{lol}{\protect\ListOfListingsLine{#1}{#2}%
            {\lst@curr}{\thepage}}%
    \fi}
\newcommand\ListOfListingsLine[4]{%
    \@dottedtocline{1}{1.5em}{2.3em}{#1}{#4}}
%    \end{macrocode}
% \end{macro}
%
%
% \subsection{Typesetting a listing}
% \begin{macro}{\inputlisting}
% Now we define the main command. The first parameter is optional and
% set to $[1,999999]$, if none is given. The second parameter is the
% filename. If the file doesn't exist, we give an error message.
% Otherwise we process the listing. But all this only happens,
% if listings are desired.
%    \begin{macrocode}
\newcommand\inputlisting[2][1,999999]{%
    \lst@ListOfListingsEntry{#2}{}%
    \iflisting %
        \batchmode \openin\lst@inputfile#2 \errorstopmode %
        \ifeof\lst@inputfile %
            \PackageError{Listings}{File `#2' not found}{%
            You must tell me the right name or I can't do the job.}%
        \else %
            \message{(#2}\lst@ProcessListing[#1]\message{)}%
        \fi \closein\lst@inputfile %
    \else %
        \begin{center}%
        --- Listing of #2 has been skipped. ---
        \end{center}%
    \fi}
%    \end{macrocode}
% \end{macro}
%
% \begin{macro}{\lst@ProcessListing}
% The two parameters of this command are the first and last output
% line, respectively. First we set up a bit and skip the lines of the
% listing upto the first printing line.
%    \begin{macrocode}
\def\lst@ProcessListing[#1,#2]{%
    \lst@Begin{#1}%
    \lst@lastno #2%
    \@whilenum #1>\lst@lineno \do %
        {\read\lst@inputfile to\lst@line %
         \advance\lst@lineno\@ne}%
%    \end{macrocode}
% The following loop terminates, if the end of file or the specified
% last line is reached. In the loop we read and process the listing
% line by line.
%    \begin{macrocode}
    \lst@endinputfalse %
    \loop %
        \read\lst@inputfile to\lst@line %
        \ifeof\lst@inputfile \lst@endinputtrue \fi %
        \ifnum\lst@lastno<\lst@lineno\lst@endinputtrue\fi %
    \lst@ifendinput\else %
        \expandafter\lst@RemovePar\lst@line\par\relax %
        \lst@AllLineProcessing %
    \repeat %
    \lst@End}
%    \end{macrocode}
% \end{macro}
%
% \begin{macro}{\lst@RemovePar}
% The next macro removes (together with the special call above) a
% \cs{par} from the input line. So we need only one \cs{long}
% definition, namely this here.
%    \begin{macrocode}
\long\def\lst@RemovePar#1\par#2\relax{\def\lst@line{#1}}
%    \end{macrocode}
% \end{macro}
%
% \begin{macro}{\listing}
% Now we define the environment for typesetting listings. To read the
% \TeX-file line by line we have to make the end of line character
% \verb!^^M! active. If no output is desired, we typeset a message and
% read the listing without doing any output.
% Ending the environment is very easy.
%    \begin{macrocode}
\newenvironment{listing}{%
    \iflisting\else
        \begin{center}%
        --- Listing has been skipped. ---
        \end{center}%
    \fi %
    \lst@Begin{1}\catcode`\^^M=\active %
    \lstenv@SkipLineAndProcess}{}
%    \end{macrocode}
% \end{macro}
%
% \begin{macro}{\lstenv@SkipLineAndProcess}
% We will use a mechanism used in the new implementation of \LaTeX's
% \verb!verbatim! environments. I don't explain it here. It is
% described in section 3.4 of \cite{verbatim}.
%    \begin{macrocode}
\begingroup
\catcode`\!=\active \catcode`\(=\active \catcode`\)=\active
\lccode`\!=`\\ \lccode`\(=`\{ \lccode`\)=`\}
\catcode`\~=\active \lccode`\~=`\^^M
\lowercase{
%    \end{macrocode}
% The effect of \cs{lowercase} is that all \verb+!+, \verb!(!, \verb!)!
% and \verb!~! of the argument are converted to \verb!\!, \verb!{!,
% \verb!}! and \verb!^^M!.
% The macro \verb!\lstenv@SkipLineAndProcess! skips the rest of the
% line, tests the optional argument and begins the loop of processing.
%    \begin{macrocode}
\gdef\lstenv@SkipLineAndProcess#1~{%
    \lstenv@TestOptional#1[]~\lstenv@ReadAndProcess}
%    \end{macrocode}
% \end{macro}
%
% \begin{macro}{\lstenv@TestOptional}
% The second parameter is empty, if and only if there is no optional
% argument.
%    \begin{macrocode}
\gdef\lstenv@TestOptional#1[#2]#3~{%
    \ifx\@empty#2\@empty\else \lst@ListOfListingsEntry{#2}{}\fi}
%    \end{macrocode}
% \end{macro}
%
% \begin{macro}{\lstenv@ReadAndProcess}
% The macro \verb!\lstenv@ReadAndProcess! gets the input upto the next
% end of line character. We append \verb!\end{listing}\relax!
% to hold up the syntax of \verb!\lstenv@Process!.
%    \begin{macrocode}
\gdef\lstenv@ReadAndProcess#1~{\lstenv@Process#1!end(listing)\relax}
%    \end{macrocode}
% \end{macro}
%
% \begin{macro}{\lstenv@Process}
% The second parameter of this macro is empty, if and only if the end
% of the environment is not reached. If the listing goes on, we define
% the current line, do the line processing and define the next macro.
% If the listing is over, we define the right macro to be done next.
%    \begin{macrocode}
\gdef\lstenv@Process#1!end(listing)#2\relax{%
    \ifx \@empty#2\@empty %
        \iflisting \def\lst@line{#1}\lst@AllLineProcessing \fi %
        \let\lst@next\lstenv@ReadAndProcess %
    \else %
        \def\lst@next{\lst@End\end{listing}}%
    \fi \lst@next}
}\endgroup
%    \end{macrocode}
% \end{macro}
%
% \begin{macro}{\lst@Begin}
% \begin{macro}{\lst@End}
% Let's look what's to do to begin and end a listing, respectively.
% We insert a small skip, the user defined \verb!\lst@prelisting!
% and \verb!\lst@postlisting! and initialize some variables.
% Of course, we use a new group level, so that outer blocks are not
% affected by the changes made here.
%    \begin{macrocode}
\newcommand\lst@Begin[1]{%
    \smallbreak\bgroup\lst@prelisting %
    \parskip\lst@lineskip %
    \lst@stringfalse \lst@commentdepth0 \lst@commentfalse %
    \let\lst@comment\@empty \let\lst@commentline\@empty %
    \lst@lineno\@ne %
%    \end{macrocode}
% The following two lines ensure, that \TeX{} doesn't read font
% information from other files when catcodes are changed.
%    \begin{macrocode}
    \setbox0\hbox{{\lst@keywordstyle}{\lst@commentstyle}%
        {\lst@stringstyle}{{\lst@labelstyle0}}}%
%    \end{macrocode}
% The textwidth for a listing is spread using a parshape command
% every paragraph. If \verb!\lst@labelstep! is zero, no labels (=line
% numbers) are printed.
%    \begin{macrocode}
    \@tempdima\textwidth \advance\@tempdima by 2\lst@halfspread %
    \expandafter\ifnum \lst@labelstep=0 %
        \everypar{\parshape 1 -\lst@halfspread \@tempdima}%
    \else %
%    \end{macrocode}
% The parameter is the number of the first printing line. We calculate
% how many lines it will take to set the first label (a line number)
% and assign it to \verb!\@tempcnta!.
%    \begin{macrocode}
        \@tempcnta#1\divide\@tempcnta\lst@labelstep %
        \multiply\@tempcnta-\lst@labelstep \advance\@tempcnta#1\relax %
        \ifnum\@tempcnta>0 \advance\@tempcnta-\lst@labelstep \fi %
%    \end{macrocode}
% Defining \cs{everypar} is easy now and some other things also.
%    \begin{macrocode}
        \everypar{\parshape 1 -\lst@halfspread \@tempdima %
            \ifnum \@tempcnta=0 %
                \advance\@tempcnta-\lst@labelstep \llap{%
                \bgroup\lst@labelstyle\the\lst@lineno\relax\egroup\ }%
            \fi %
            \advance\@tempcnta\@ne}%
    \fi %
    \expandafter\lst@width\lst@baseem em %
    \lst@ChangeCatcodes %
%    \end{macrocode}
% Finally we call the language's own prepare macro.
%    \begin{macrocode}
    \edef\next{lstdrv@\lst@curr @PrepareListing}%
    \expandafter\csname\next\endcsname}
%    \end{macrocode}
% The ending of a listing is not very exciting.
%    \begin{macrocode}
\def\lst@End{\lst@postlisting\catcode`\%=\lst@ccPercent %
    \par\removelastskip\egroup \smallbreak\ignorespaces}
%    \end{macrocode}
% \end{macro}\end{macro}
%
%
% \subsection{Line processing}\label{ssLineProcessing}
%
% \begin{macro}{\lst@AllLineProcessing}
% Here comes the single line processing. First we collect the line
% processing in one macro: Some initialization, pre-processing,
% processing and post-processing, but only if the line is not empty.
% We increase the current line number also.
%    \begin{macrocode}
\def\lst@AllLineProcessing{%
    \ifx \lst@line\@empty \par\noindent\hbox{}\else
        \let\lst@text\@empty \lst@length0 \global\lst@pos0 %
        \let\lst@lastother\@empty %
        \par\noindent \lst@PrePL \lst@ProcessLine \lst@PostPL %
    \fi \advance\lst@lineno\@ne}
%    \end{macrocode}
% \end{macro}
%
% \begin{macro}{\lst@PrePLDefault}
% \begin{macro}{\lst@PostPLDefault}
% Before considering the main macro we define the defaults for
% \verb!\lst@PrePL! and \verb!\lst@PostPL!. The latter gives a
% warning, if a string exceeds a line.
%    \begin{macrocode}
\let\lst@PrePLDefault\relax
\def\lst@PostPLDefault{%
    \lst@ifstring %
        \PackageWarning{Listings}{string constant exceeds line}%
        \lst@stringfalse %
    \fi}
%    \end{macrocode}
% \end{macro}\end{macro}
%
% \begin{macro}{\lst@ProcessLine}
% Now the main macro for line processing. There is no parameter,
% because the data is given by \verb!\lst@line!. We construct a loop
% by defining the macro \cs{next} to call this macro again. But this
% definition might changes soon to terminate line processing.
%    \begin{macrocode}
\def\lst@ProcessLine{\let\next\lst@ProcessLine %
%    \end{macrocode}
% Since a comment might exceeds a line, we have to handle this case
% first. We try to find the end of comment, which saves the comment
% in \verb!\lst@comment! and might changes the boolean
% \verb!\lst@ifcomment!.
%    \begin{macrocode}
    \lst@ifcomment \lst@EOC %
%    \end{macrocode}
% To output the comment we use a little trick (concerning the naming
% of a macro). The tokenize macro below not only cuts up the input for
% scanning keywords. In comment mode it will cut the input in the same
% way, but always uses commentstyle for the output. Therefore we
% switch to comment mode locally and call the macro.
%    \begin{macrocode}
        {\lst@commenttrue \expandafter\lst@Tokenize\lst@comment\relax}%
%    \end{macrocode}
% We empty the comment just output. If the comment goes on, i.e.\ the
% comment was the whole line, line processing is over, i.e.\ there is
% nothing to do next: We redefine the macro \cs{next}.
%    \begin{macrocode}
        \let\lst@comment\@empty \lst@ifcomment \let\next\relax \fi %
    \else %
%    \end{macrocode}
% If no comment is in progress, the macro tries to find one and cuts
% off a comment line, if possible (or a part of the line as a comment
% line).
%    \begin{macrocode}
        \lst@SOC \lst@CommentLine %
%    \end{macrocode}
% Now \verb!\lst@line! has been cut to the noncomment rest.
% We have to find the keywords. \cs{relax} is used as a brake:
% It marks the end and will terminate tokenizing.
%    \begin{macrocode}
        {\lst@commentfalse \expandafter\lst@Tokenize\lst@line\relax}%
%    \end{macrocode}
% If a string has begun in \verb!\lst@Tokenize!, but not finished,
% the potential comment above belongs to the string. What's to do? The
% input line gets the comment and the next character is typeset. Then
% we call this macro without being in danger classifing the comment as
% a comment again, because the first character is missing now. But
% hold on: The comment might be empty and so the input line. To handle
% this we use two \cs{relax} as brake and might redefine \cs{next}.
% \verb!\lst@PostPL! might gives a warning to the user.
%    \begin{macrocode}
        \lst@ifstring %
            \lst@commentfalse %
            \expandafter\expandafter\expandafter\lst@TypesetChar %
                \expandafter\lst@commentline\lst@comment\relax\relax %
                \let\lst@comment\@empty \let\lst@commentline\@empty %
            \ifx \lst@line\@empty \let\next\relax \fi %
        \else %
%    \end{macrocode}
% If no string is in work, it's all fine. We output the comment line,
% if present. In that case there is no other comment.
%    \begin{macrocode}
            \ifx\lst@commentline\@empty \else %
                \lst@commenttrue %
                \expandafter\expandafter\expandafter\lst@Tokenize %
                    \expandafter\lst@commentline\lst@comment\relax %
                \lst@commentfalse %
            \fi %
%    \end{macrocode}
% For comments this macro calls itself to find the end of comment.
% As this is the default, we only have to assign the comment to
% \verb!\lst@line! --- the implizit parameter of this macro.
% If there is no comment, we redefine \cs{next}.
%    \begin{macrocode}
            \lst@ifcomment \let\lst@line\lst@comment %
            \else \let\next\relax %
            \fi %
%    \end{macrocode}
% We have to close some ifs and do the next thing.
%    \begin{macrocode}
        \fi %
    \fi\next}
%    \end{macrocode}
% \end{macro}
%
% \begin{macro}{\lst@TypesetChar}
% The following macro has been used in the last macro. It typesets the
% next character and removes it from the input line.
%    \begin{macrocode}
\def\lst@TypesetChar#1#2\relax{\def\lst@line{#2}\lst@OutputChar#1}
%    \end{macrocode}
% \end{macro}
%
% \begin{macro}{\lst@ProcessWhitespaces}
% This macro processes all whitespaces (blank spaces and tabulators here)
% from the beginning of \verb!\lst@line! upto the first non-whitespace.
% The macro assumes that \verb!\lst@text! is already output.
%    \begin{macrocode}
\def\lst@ProcessWhitespaces{%
    \expandafter\lst@TestWhitespace\lst@line\relax}
%    \end{macrocode}
% \end{macro}
%
% \begin{macro}{\lst@EndTestWhitespace}
% To end the whitespace test we get the line upto the closing \cs{relax}
% and assign it to \verb!\lst@line!.
%    \begin{macrocode}
\def\lst@EndTestWhitespace#1\relax{\def\lst@line{#1}}
%    \end{macrocode}
% \end{macro}
%
% \begin{macro}{\lst@TestWhitespace}
% The real work is done here. We test for a whitespace, do the output
% and decide what's to be done next.
%    \begin{macrocode}
\def\lst@TestWhitespace#1{\let\next\lst@TestWhitespace %
    \ifcat #1&\lst@GotoNextTabStop %
    \else \if #1\lst@outputblank \lst@OutputChar#1%
    \else
        \ifx#1\relax \let\lst@line\@empty \let\next\relax %
        \else \def\next{\lst@EndTestWhitespace#1}%
        \fi %
    \fi \fi \next}
%    \end{macrocode}
% \end{macro}
% The line processing is complete. Now we will try to scan keywords.
%
%
% \subsection{Tokenizing}\label{ssTokenizing}
% We define two tokenize macros. The first determines potential
% keywords by collecting characters upto a nonletter. After outputting
% the potential keyword the second macro is called to gather all
% characters upto the next letter. Then these characters are output
% and the first macro is called again. That's how the two macros work
% together.
%
% \begin{macro}{\lst@Tokenize}
% The tokenize macro gets code upto \cs{relax} and scans for keywords
% and strings. In fact it only cuts up potential keywords, which are
% tested right before their output. We define the next operation first:
%    \begin{macrocode}
\def\lst@Tokenize#1{\let\lst@next\lst@Tokenize%
%    \end{macrocode}
% An incoming letter is gathered. If we found a nonletter, we output
% the preceding text. The keyword test takes place there.
%    \begin{macrocode}
    \ifcat #1a%
        \edef\lst@text{\lst@text#1}\advance\lst@length\@ne %
    \else %
        \lst@Output %
%    \end{macrocode}
% If the other character is a tabulator, we go to the next tab stop.
%    \begin{macrocode}
        \ifcat #1&\lst@GotoNextTabStop %
    \else %
%    \end{macrocode}
% Getting to the terminator \cs{relax} ends tokenizing by defining
% next operation empty, here \cs{relax}. Otherwise our next operation
% is 'tokenizing' nonletter characters.
%    \begin{macrocode}
        \ifx#1\relax \let\lst@next\relax %
        \else \let\lst@next\lst@TokenizeOther %
%    \end{macrocode}
% If we are in comment mode, we have not to look for strings and gather
% the nonletter character. Otherwise we look for stringizer. Then come
% some closing \cs{fi}s and the next operation.
%    \begin{macrocode}
            \lst@ifcomment \lst@length\z@ \lst@AppendOther#1%
            \else \lst@TestStringizer#1\fi %
        \fi %
    \fi \fi \lst@next}
%    \end{macrocode}
% \end{macro}
%
% \begin{macro}{\lst@TokenizeOther}
% This macro looks like \verb!lst@Tokenize! except that we have to
% insert some \verb!\lst@OutputOther!.
%    \begin{macrocode}
\def\lst@TokenizeOther#1{%
    \let\lst@next\lst@TokenizeOther %
    \ifcat #1a%
        \lst@OutputOther \let\lst@lastother\@empty %
        \def\lst@text{#1}\lst@length\@ne %
        \let\lst@next\lst@Tokenize %
    \else \ifcat #1&%
        \lst@OutputOther \lst@GotoNextTabStop %
    \else %
        \ifx#1\relax \lst@OutputOther \let\lst@next\relax %
%    \end{macrocode}
% Now comes something new. If we find two successive blank spaces, we
% output all preceding other characters first. This avoid alignment
% problems when too many blanks follow each other.
% The rest is unchanged.
%    \begin{macrocode}
        \else \ifx\lst@lastother\lst@inputblank\if#1\lst@inputblank %
            \lst@OutputOther \fi\fi %
        \lst@ifcomment \lst@AppendOther#1%
        \else \lst@TestStringizer#1%
        \fi \fi %
    \fi \fi \lst@next}
%    \end{macrocode}
% \end{macro}
%
% \begin{macro}{\lst@AppendOther}
% This macro simply appends a character to the token register
% \verb!\lst@other!.
%    \begin{macrocode}
\def\lst@AppendOther#1{\advance\lst@length\@ne %
    \expandafter\lst@other\expandafter{\the\lst@other#1}}
%    \end{macrocode}
% \end{macro}
%
% \begin{macro}{\lst@bTestStringizer}
% Here we do the stringizer tests. If a string already started, we
% gather the current character and look for the closing stringizer.
% If we've found it, we output \verb!\lst@other! and switch the
% string boolean, but only if the preceding other character is not
% a backslash (which inserts a stringizer at the current position
% of the string).
%    \begin{macrocode}
\def\lst@bTestStringizer#1{%
    \lst@ifstring \lst@AppendOther#1%
        \expandafter\ifx\lst@closing@stringizer#1%
            \ifx\lst@lastother\lst@inputbackslash \else %
            \lst@OutputOther \global\lst@stringfalse %
        \fi \fi %
        \def\lst@lastother{#1}%
%    \end{macrocode}
% If we are not in string mode, we compare the current character with
% all given stringizers and gather the character afterwards, since
% the string boolean might changes and we might output the preceding
% characters with a different style.
%    \begin{macrocode}
    \else %
        \def\lst@lastother{#1}%
        \expandafter\lst@DoStringizerTest\lst@stringizer\relax %
        \lst@AppendOther#1%
    \fi}
%    \end{macrocode}
% \end{macro}
%
% \begin{macro}{\lst@dTestStringizer}
% Nearly the same:
%    \begin{macrocode}
\def\lst@dTestStringizer#1{\def\lst@lastother{#1}%
    \lst@ifstring \lst@AppendOther#1%
        \ifx\lst@closing@stringizer\lst@lastother %
            \lst@OutputOther \global\lst@stringfalse %
        \fi %
    \else %
        \expandafter\lst@DoStringizerTest\lst@stringizer\relax %
        \lst@AppendOther#1%
    \fi}
%    \end{macrocode}
% \end{macro}
%
% \begin{macro}{\lst@DoStringizerTest}
% Now we consider a stringizer test. Reaching \cs{relax} results in
% doing nothing and so terminating the tests.
%    \begin{macrocode}
\def\lst@DoStringizerTest#1{%
    \ifx#1\relax \else %
%    \end{macrocode}
% Otherwise we compare the current character and the current stringizer.
% If they are equal, we output the preceding characters, switch the
% string boolean and save the stringizer as closing stringizer.
%    \begin{macrocode}
        \expandafter\ifx\lst@lastother#1\relax %
            \lst@OutputOther \global\lst@stringtrue %
            \gdef\lst@closing@stringizer{#1}%
        \fi %
%    \end{macrocode}
% At the end we call this macro again. The macro terminates reaching
% the \cs{relax}!
%    \begin{macrocode}
        \expandafter\lst@DoStringizerTest %
    \fi}%
%    \end{macrocode}
% \end{macro}
%
%
% \subsection{Output}\label{ssOutput}
% \begin{macro}{\lst@Output}
% It's time to do text output. First we look, if there is anything to
% output.
%    \begin{macrocode}
\def\lst@Output{%
    \ifx\lst@text\@empty\else %
%    \end{macrocode}
% Then we decide, whether to make a stringbox (using stringstyle) or
% a box for a comment (using commentstyle as explicit parameter).
%    \begin{macrocode}
        \lst@ifstring \lst@MakeStringBox{\lst@stringstyle}\else %
        \lst@ifcomment\lst@MakeBox{\lst@commentstyle}\else %
%    \end{macrocode}
% If we output neither a string nor a comment, we look for a keyword.
% The output takes place in the macro \verb!\lst@KeywordsOrNot!
% (see section \ref{sLanguageDriverFiles}). \cs{relax} terminates the
% text parameter.
%    \begin{macrocode}
        \expandafter\lst@KeywordOrNot\lst@text\relax %
        \fi \fi %
%    \end{macrocode}
% Finally we hold up the current column, empty the text and close the
% starting 'if text not empty'.
%    \begin{macrocode}
        \global\advance\lst@pos by -\lst@length %
        \let\lst@text\@empty \lst@length0 %
    \fi}
%    \end{macrocode}
% \end{macro}
%
% \begin{macro}{\lst@OutputOther}
% To output the characters from the token register \verb!\lst@other!,
% we assign these characters to a macro. The rest is the same as in
% \verb!\lst@Output!, except that we need no keyword tests and make
% a normal box instead.
%    \begin{macrocode}
\def\lst@OutputOther{%
    \expandafter\def\expandafter\lst@text\expandafter{\the\lst@other}%
    \ifx\lst@text\@empty\else %
        \lst@ifstring \lst@MakeStringBox{\lst@stringstyle}\else %
        \lst@ifcomment\lst@MakeBox{\lst@commentstyle}\else %
            \lst@MakeBox{}%
        \fi \fi %
        \global\advance\lst@pos by -\lst@length %
        \lst@other{}\let\lst@text\@empty \lst@length0 %
    \fi}
%    \end{macrocode}
% \end{macro}
%
% \begin{macro}{\lst@MakeBox}
% Consider the different boxes now. A box must take \verb!\lst@length!
% characters: the width is \verb!\lst@length!$\cdot$\verb!\lst@width!.
% The macro parameter possibly selects another style. We insert dynamic
% space at the beginning (and at the ending) to center the text and we
% fill the box. Again \cs{relax} is a brake.
%    \begin{macrocode}
\def\lst@MakeBox#1{%
    \hbox to \lst@length\lst@width{#1\hss %
        \expandafter\lst@FillBox\lst@text\relax \hss}}
%    \end{macrocode}
% \end{macro}
%
% \begin{macro}{\lst@FillBox}
% Filling up a box is easy. If we found the end of the text, we do
% nothing. Otherwise we output the character and insert dynamic space.
% Since the underbar is not a printable character in \TeX{}, we have
% to treat it special. By the way: We make the underbar having the
% letter catcode here; we don't want subscripts. After all this is
% done, we call the fillbox macro again.
% Note: The macro is called after the closing \cs{fi}!
%    \begin{macrocode}
\begingroup
\catcode`\_=11
\gdef\lst@FillBox#1{%
    \ifx\relax#1\else %
        \ifx#1_\textunderscore \else #1\fi \hss %
        \expandafter\lst@FillBox %
    \fi}
\endgroup
%    \end{macrocode}
% \end{macro}
%
% \begin{macro}{\lst@MakeSpecialStringBox}
% A stringbox is nearly the same: We only let the 'normal blank' be
% '\textvisiblespace'.
%    \begin{macrocode}
\def\lst@MakeSpecialStringBox#1{%
    \hbox to\lst@length\lst@width{\let\lst@outputblank\textvisiblespace%
        #1\hss \expandafter\lst@FillBox\lst@text\relax \hss}}
%    \end{macrocode}
% \end{macro}
%
% \begin{macro}{\lst@OutputChar}
% We define one more macro to save time. It outputs a single character.
% The implementation should be clear.
%    \begin{macrocode}
\def\lst@OutputChar#1{\global\advance\lst@pos by -1%
    \hbox to \lst@width{\hss %
        \lst@ifstring \def\lst@text{#1}\lst@length\@ne %
            \lst@MakeStringBox{\lst@stringstyle}%
        \else \lst@ifcomment \lst@commentstyle#1\hss %
        \else #1\hss %
        \fi \fi}}
%    \end{macrocode}
% \end{macro}
%
% \begin{macro}{\lst@GotoNextTabStop}
% At the end of this section we consider tabulators. As seen, each
% typeset character decrements the counter \verb!\lst@pos!. To go to
% the next tabulator stop, we must first determine how many blanks are
% needed. We simply add \verb!\lst@tablength! until \verb!\lst@pos!
% is strict positive.
%    \begin{macrocode}
\def\lst@GotoNextTabStop{%
    \@whilenum \lst@pos<1 \do %
        {\global\advance\lst@pos\lst@tablength}%
%    \end{macrocode}
% Now we make a box having the width of \verb!\lst@pos! characters and
% set \verb!\lst@pos! to zero.
%    \begin{macrocode}
    \hbox to \lst@pos\lst@width{\hss}\lst@pos0}%
%    \end{macrocode}
% \end{macro}
%
%
% \subsection{Special characters}\label{ssSpecialCharacters}
% Some definitions of special characters:
%    \begin{macrocode}
\def\lst@leftbrace{$\{$}
\def\lst@rightbrace{$\}$}
\def\lst@less{$<$}
\def\lst@greater{$>$}
\def\lst@verticalbar{$|$}
\def\lst@minus{$-$}
\def\lst@multiply{$*$}
%    \end{macrocode}
%
% \begin{macro}{\lst@MakeDigitsLetter}
% This macro is similar to \cs{makeatletter}.
%    \begin{macrocode}
\def\lst@MakeDigitsLetter{\catcode`\0=11\catcode`\1=11%
    \catcode`\2=11\catcode`\3=11\catcode`\4=11\catcode`\5=11%
    \catcode`\6=11\catcode`\7=11\catcode`\8=11\catcode`\9=11}
%    \end{macrocode}
% \end{macro}
%
% \begin{macro}{\lst@DefineCatcodes}
% At the moment only the catcode of $-$ is language specific.
% This macro stores such values.
%    \begin{macrocode}
\def\lst@DefineCatcodes#1{%
    \def\lst@ChangeMoreCatcodes{\catcode`\-=#1}}
%    \end{macrocode}
% \end{macro}
%
% \begin{macro}{\lst@ChangeCatcodes}
% Now begins the horrible part: It is necessary to change many
% catcodes. Since we need backslash and braces with other meanings,
% we replace them by the slash $/$, $($ and $)$. Other characters like
% \$ and \& need not to be replaced, because we don't need them for
% defining macros, but $\backslash$, $\{$ and $\}$ are essential for
% doing that.
%    \begin{macrocode}
\begingroup
\catcode `/=0       \catcode `(= 1      \catcode `)=2 % new \, { and }
\catcode`\<=\active \catcode`\>=\active \catcode`\|=\active
\catcode`\-=\active \catcode`\*=\active
\catcode`\{=\active \catcode`\}=\active \catcode`\\=\active
%    \end{macrocode}
% From now on we must use the slash and $()$ instead of backslash and
% braces. Since the blank space will also be an active character, we
% must terminate each line with a comment character. Otherwise \TeX{}
% would think, we want to typeset something before beginning the
% document.
% \SpecialEscapechar/
%    \begin{macrocode}
/gdef/lst@inputbackslash(\)%
/catcode`/ =/active%
/gdef/lst@inputblank( )%
/gdef/lst@outputblank(/ )%
%    \end{macrocode}
% The following macro changes catcodes and the definition of active
% characters at the time we input a listing.
% \SpecialEscapechar/
%    \begin{macrocode}
/gdef/lst@ChangeCatcodes(%
/catcode`/ =/active/def (/lst@outputblank)%
/catcode`/\=/active/def\($/backslash$)%
/catcode`/{=/active/def{(/lst@leftbrace)%
/catcode`/}=/active/def}(/lst@rightbrace)%
/catcode`/<=/active/def<(/lst@less)%
/catcode`/>=/active/def>(/lst@greater)%
/catcode`/|=/active/def|(/lst@verticalbar)%
/catcode`/-=/active/def-(/lst@minus)%
/catcode`/*=/active/def*(/lst@multiply)%
/chardef~="7E%
%    \end{macrocode}
% A tabulator (ASCII code 9=ord(I)$-$64) gets the catcode of a
% \TeX{}-tabulator.
% \SpecialEscapechar/
%    \begin{macrocode}
/catcode`/^^I=4%
%    \end{macrocode}
% Some characters are treated as other characters (12) or as letters
% (11).
% \SpecialEscapechar/
%    \begin{macrocode}
/catcode`/$=11/catcode`/_=11/lst@MakeDigitsLetter%
/catcode`/&=12/catcode`/^=12/catcode`/"=12/catcode`/#=12%
/chardef/lst@ccPercent=/catcode`/%/catcode`/%=12% save catcode
/lst@ChangeMoreCatcodes/makeatletter)%
/endgroup
%    \end{macrocode}
% All changed catcodes (time of definition) are restored by ending the
% group.
% \end{macro}
%
%
% \subsection{Initialization}\label{ssInitialization}
% Defaults are selected and the options are processed.
%    \begin{macrocode}
\normallisting
\listingtrue
\tablength{4}
\lstlineskip{0pt}
\selectlisting{blank}
\ProcessOptions
%</package>
%    \end{macrocode}
%
%
% \section{Language driver files}\label{sLanguageDriverFiles}
% Each driver file contains language specific data:
% \begin{itemize}
% \item keywords,
% \item whether the language is case sensitive or not,
% \item information about comment lines and other comments,
% \item the stringizer,
% \item baseem --- the width of a single character box,
% \item the catcode of the minus $-$,
% \item \verb!\lst@PrePL! and \verb!\lst@PostPL! are executed before and
%	after the real line processing takes place: \verb!\lst@PrePL! is
%	intend to manipulate the input, e.g.\ Fortran uses it to detect
%	comment lines beginning with '$*$' or 'C' and Eiffel's string
%	concatenate mechanism is done there, and e.g.\ \verb!\lst@PostPL!
%	might gives a warning, if a string exceeds a line.
% \end{itemize}
% The package selects a language by calling \verb!\lstdrv@#1@!, where
% \verb!#1! is replaced by the language. Additionally the language option
% (empty or not) is used as suffix, e.g.\ \verb!\lstdrv@cobol@1974!.
% That macro must support the described things.
%
% Some driver files use own data. To setup things like that the macro
% \verb!\lstdrv@#1@PrepareListing! is called \emph{after} all init
% of the kernel is done (changing catcodes, e.g.). Here \verb!#1!
% is replaced by the language again. If no special setup is needed,
% a driver file need not to define that macro.
%
% The implemented languages follow as examples.
% '\texttt{???}' in driver files indicate things I don't know.
%
%
% \subsection{Blank listing}
% \begin{macro}{\lstdrv@blank@}
% For a blank listing we let all empty. \cs{makeatletter} lets \verb!@!
% be a letter here. Otherwise we can't access \verb!\lstdrv@blank@!.
%    \begin{macrocode}
%<*blank>
\begingroup \makeatletter %
\gdef\lstdrv@blank@{%
    \keywords{}%
    \sensitivetrue %
    \DeclareCommentLine\relax %
    \DeclareSingleComment stuff \relax %
    \stringizer{}\lstbaseem{0.6}%
    \lst@DefineCatcodes{\active}%
    \let\lst@PrePL \lst@PrePLDefault %
    \let\lst@PostPL\lst@PostPLDefault}%
\endgroup %
%</blank>
%    \end{macrocode}
% \end{macro}
%
%
% \subsection{Ada}
% \begin{macro}{\lstdrv@ada@}
% Keywords, comments, \ldots{}
% The catcode of the double quote is changed for compatibility with
% \texttt{german.sty}.
%    \begin{macrocode}
%<*ada>
\begingroup \makeatletter %
\catcode`\"=12 \catcode`\-=\active %
\gdef\lstdrv@ada@{%
    \keywords{abort,abs,accept,access,all,and,array,at,begin,body,%
        case,constant,declare,delay,delta,digits,do,else,elsif,end,%
        entry,exception,exit,for,function,generic,goto,if,in,is,%
        limited,loop,mod,new,not,null,of,or,others,out,package,pragma,%
        private,procedure,raise,range,record,rem,renames,return,%
        reverse,select,separate,subtype,task,terminate,then,type,%
        use,when,while,with,xor}%
    \sensitivefalse %
    \DeclareCommentLine --\relax %
    \DeclareSingleComment stuff \relax %
    \stringizer[d]{"'}\lstbaseem{0.6}% ??? stringizer doubled
    \lst@DefineCatcodes{\active}%
    \let\lst@PrePL \lst@PrePLDefault %
    \let\lst@PostPL\lst@PostPLDefault}%
\endgroup %
%</ada>
%    \end{macrocode}
% \end{macro}
%
%
% \subsection{Algol}
% \begin{macro}{\lstdrv@algol@}
% The main driver macro.
%    \begin{macrocode}
%<*algol>
\begingroup \makeatletter %
\catcode`\#=12 %
\gdef\lstdrv@algol@{% ??? should 'i' be a keyword
    \keywords{abs,and,arg,begin,bin,bits,bool,by,bytes,case,channel,%
        char,co,comment,compl,conj,divab,do,down,elem,elif,else,empty,%
        end,entier,eq,esac,exit,false,fi,file,flex,for,format,from,%
        ge,goto,gt,heap,if,im,in,int,is,isnt,le,leng,level,loc,long,%
        lt,lwb,minusab,mod,modab,mode,ne,nil,not,od,odd,of,op,or,ouse,%
        out,over,overab,par,plusab,plusto,pr,pragmat,prio,proc,re,real,%
        ref,repr,round,sema,shl,short,shorten,shr,sign,skip,string,%
        structthen,timesab,to,true,union,up,upb,void,while}%
    \sensitivefalse % ???
    \DeclareCommentLine\relax %
    \DeclarePairedComment #\relax %
    \stringizer{}\lstbaseem{0.6}%
    \lst@DefineCatcodes{\active}%
    \let\lst@PrePL \lst@PrePLDefault %
    \let\lst@PostPL\lst@PostPLDefault}%
%    \end{macrocode}
% \end{macro}
%
% \begin{macro}{\lstdrv@algol@68}
% \begin{macro}{\lstdrv@algol@60}
% Macros for the optional argument:
%    \begin{macrocode}
\global\@namedef{lstdrv@algol@68}{\lstdrv@algol@}%
\global\@namedef{lstdrv@algol@60}{\lstdrv@algol@ %
    \keywords{array,begin,Boolean,code,comment,div,do,else,end,false,%
        for,goto,if,integer,label,own,power,procedure,real,step,string,%
        switch,then,true,until,value,while}%
    \DeclareSingleComment stuff \relax}%
\endgroup %
%    \end{macrocode}
% \end{macro}\end{macro}
%
% \begin{macro}{\lstdrv@algol@PrepareListing}
% To implement comments beginning with \verb!comment! or \verb!co!
% we must go deep inside this package again. We need an additional
% 'ifcomment' and define Algol's Prepare\-Listing to override the
% default output macros.
%    \begin{macrocode}
\begingroup \makeatletter %
\gdef\lstdrv@algol@commenttrue{%
    \global\let\lstdrv@algol@ifcomment\iftrue}%
\gdef\lstdrv@algol@commentfalse{%
    \global\let\lstdrv@algol@ifcomment\iffalse %
    \global\let\lstdrv@algol@closingcomment\@empty}%
\gdef\lstdrv@algol@PrepareListing{%
    \lstdrv@algol@commentfalse %
    \let\lst@OutputOther\lstdrv@algol@OutputOther %
    \lst@ifoption{60}{\let\lst@Output\lstdrv@algol@Output@ %
                      \let\lst@AppendOther\lstdrv@algol@AppendOther}%
                     {\let\lst@Output\lstdrv@algol@Output}}%
%    \end{macrocode}
% \end{macro}
%
% \begin{macro}{\lstdrv@algol@OutputOther}
% This macro is a simple derivation of \verb!\lst@OutputOther!. The only
% difference is that we test Algol's additional 'ifcomment' (4th line).
%    \begin{macrocode}
\gdef\lstdrv@algol@OutputOther{%
    \expandafter\def\expandafter\lst@text\expandafter{\the\lst@other}%
    \ifx\lst@text\@empty\else %
        \lstdrv@algol@ifcomment\lst@MakeBox{\lst@commentstyle}\else %
        \lst@ifstring \lst@MakeStringBox{\lst@stringstyle}\else %
        \lst@ifcomment\lst@MakeBox{\lst@commentstyle}\else %
            \lst@MakeBox{}%
        \fi \fi \fi %
        \global\advance\lst@pos by -\lst@length %
        \lst@other{}\let\lst@text\@empty \lst@length0 %
    \fi}%
%    \end{macrocode}
% \end{macro}
%
% \begin{macro}{\lstdrv@algol@Output}
% Here we also install an additional 'ifcomment'. If we are in comment
% mode, we look whether the comment is over or not.
%    \begin{macrocode}
\gdef\lstdrv@algol@Output{%
    \ifx\lst@text\@empty\else %
        \lstdrv@algol@ifcomment %
            \ifx\lst@text\lstdrv@algol@closingcomment %
                \lstdrv@algol@commentfalse %
                \lst@MakeBox{\lst@keywordstyle}%
            \else \lst@MakeBox{\lst@commentstyle}%
            \fi \else %
%    \end{macrocode}
% The next three lines are unchanged.
%    \begin{macrocode}
        \lst@ifstring \lst@MakeStringBox{\lst@stringstyle}\else %
        \lst@ifcomment\lst@MakeBox{\lst@commentstyle}\else %
        \expandafter\lst@KeywordOrNot\lst@text\relax %
%    \end{macrocode}
% After the keyword (or non-keyword) is output, we look if a comment
% starts: If the current text equals \verb!co! or \verb!comment!, we
% enter comment mode and set the closing comment delimiter.
%    \begin{macrocode}
        \expandafter\lst@ifoneof\lst@text\relax{comment,co}%
            {\global\let\lstdrv@algol@closingcomment\lst@text}{}%
%    \end{macrocode}
% The rest is unchanged.
%    \begin{macrocode}
        \fi \fi \fi %
        \global\advance\lst@pos by -\lst@length %
        \let\lst@text\@empty \lst@length0 %
    \fi}%
%    \end{macrocode}
% \end{macro}
%
% \begin{macro}{\lstdrv@algol@Output@}
% Now we come to Algol 60. First comes the same as above. But:
% An empty 'closingcomment' indicates that no \verb!end! and no
% \verb!else! closes a comment, i.e.\ we look first whether 'end
% of comment' tests are legal or not.
%    \begin{macrocode}
\gdef\lstdrv@algol@Output@{%
    \ifx\lst@text\@empty\else %
        \lstdrv@algol@ifcomment %
            \ifx\@empty\lstdrv@algol@closingcomment %
                \lst@MakeBox{\lst@commentstyle}%
            \else %
                \expandafter\lst@ifoneof\lst@text\relax{else,end}%
                    {\lstdrv@algol@commentfalse %
                     \lst@MakeBox{\lst@keywordstyle}}%
                    {\lst@MakeBox{\lst@commentstyle}}%
            \fi \else %
%    \end{macrocode}
% The next three lines are unchanged.
%    \begin{macrocode}
        \lst@ifstring \lst@MakeStringBox{\lst@stringstyle}\else %
        \lst@ifcomment\lst@MakeBox{\lst@commentstyle}\else %
        \expandafter\lst@KeywordOrNot\lst@text\relax %
%    \end{macrocode}
% After the keyword (or non-keyword) is output, we look if a comment
% starts: If the current text equals \verb!comment!, we enter the
% comment mode and let \verb!\lstdrv@algol@closingcomment! empty;
% if the text equals \verb!end!, we let 'closingcomment' not empty,
% i.e.\ the comment may closes with \verb!else! or \verb!end!
% (see above).
%    \begin{macrocode}
        \expandafter\lst@ifoneof\lst@text\relax{comment,end}%
            {\lstdrv@simula@commenttrue %
             \expandafter\lst@ifoneof\lst@text\relax{end}%
                 {\gdef\lstdrv@simula@closingcomment{a}}{}}%
            {}% empty else from 'ifoneof'
%    \end{macrocode}
% The rest is unchanged.
%    \begin{macrocode}
        \fi \fi \fi %
        \global\advance\lst@pos by -\lst@length %
        \let\lst@text\@empty \lst@length0 %
    \fi}%
%    \end{macrocode}
% \end{macro}
%
% \begin{macro}{\lstdrv@algol@AppendOther@}
% A semicolon also ends a comment. Hence:
%    \begin{macrocode}
\gdef\lstdrv@algol@AppendOther#1{%
    \lstdrv@algol@ifcomment \if;#1%
        \lst@OutputOther \lstdrv@algol@commentfalse %
    \fi \fi %
    \advance\lst@length\@ne %
    \expandafter\lst@other\expandafter{\the\lst@other#1}}%
\endgroup %
%</algol>
%    \end{macrocode}
% \end{macro}
%
%
% \subsection{C}
% \begin{macro}{\lstdrv@c@}
% Since $*$ will be an active character when we input a listing, we
% have to change the catcode here.
%    \begin{macrocode}
%<*c>
\begingroup \makeatletter %
\catcode`\"=12 \catcode`\*=\active %
\gdef\lstdrv@c@{%
    \keywords{auto,break,case,char,const,continue,default,do,double,%
        else,enum,extern,float,for,goto,if,int,long,register,return,%
        short,signed,sizeof,static,struct,switch,typedef,union,%
        unsigned,void,volatile,while}%
    \sensitivetrue %
    \DeclareCommentLine\relax %
    \DeclareSingleComment /* */\relax %
    \stringizer[b]{"}\lstbaseem{0.6}%
    \lst@DefineCatcodes{\active}%
    \let\lst@PrePL \lstdrv@c@PrePL %
    \let\lst@PostPL\lst@PostPLDefault}%
\endgroup %
%    \end{macrocode}
% \end{macro}
%
% \begin{macro}{\lstdrv@c@PrePL}
% \begin{macro}{\lstdrv@c@directives}
% We only test for C compiler directives.
%    \begin{macrocode}
\begingroup \makeatletter %
\gdef\lstdrv@c@PrePL{%
    \expandafter\lstdrv@c@TestSharp\lst@line\relax\relax}%
\gdef\lstdrv@c@directives{,define,elif,else,endif,error,if,ifdef,%
    ifndef,line,include,pragma,undef}%
%    \end{macrocode}
% \end{macro}\end{macro}
%
% \begin{macro}{\lstdrv@c@TestSharp}
% To implement C (and C++) directives the catcode of $\#$ must be
% changed from 'parameter symbol for macros' to 'letter'.
%    \begin{macrocode}
\catcode`\&=6 \catcode`\#=12 %
\gdef\lstdrv@c@TestSharp&1&2\relax{%
    \if#&1\def\lst@line{&2}%
%    \end{macrocode}
% If the first character of a line is a sharp, we first output that
% sharp using the keywordstyle. All compiler directives become the
% current keywords and we output the line, which gets empty afterwards.
%    \begin{macrocode}
        {\lst@keywordstyle\lst@OutputChar#}%
        {\let\lst@keywords\lstdrv@c@directives \lst@ProcessLine}%
        \let\lst@line\@empty %
    \fi}%
\endgroup %
%</c>
%    \end{macrocode}
% \end{macro}
%
%
% \subsection{C++}
% \begin{macro}{\lstdrv@cpp@}
% Since C++ is an extension of C (from the point of view here), we load
% the C driver file and use the definition there.
%    \begin{macrocode}
%<*cpp>
\selectlisting{c}%
\begingroup \makeatletter %
\catcode`\_=11 %
\gdef\lstdrv@cpp@{\lstdrv@c@ %
    \morekeywords{asm,bad_cast,bad_typeid,bool,catch,class,const_cast,%
        delete,dynamic_cast,false,friend,inline,namespace,new,operator,%
        private,protected,public,reinterpret_cast,static_cast,template,%
        this,throw,true,try,type_info,typeid,using,virtual,xalloc,%
        __multiple_inheritance,__single_inheritance,%
        __virtual_inheritance}%
    \DeclareCommentLine //\relax}%
\global\let\lstdrv@cpp@ansi \lstdrv@cpp %
\gdef\lstdrv@cpp@vc{\lstdrv@cpp %
    \morekeywords{__asm,__based,__cdecl,__declspec,dllexport,dllimport,%
        __except,__fastcall,__finally,__inline,__int8,__int16,__int32,%
        __int64,naked,__stdcall,thread,__try,__leave}}%
\endgroup %
%</cpp>
%    \end{macrocode}
% \end{macro}
%
%
% \subsection{Cobol}
% \begin{macro}{\lstdrv@cobol@keys}
% \begin{macro}{\lstdrv@cobol@keys@eightyfive}
% \begin{macro}{\lstdrv@cobol@keys@ibm}
% The keywords first:
%    \begin{macrocode}
%<*cobol>
\begingroup \makeatletter %
\catcode`\-=11 \lst@MakeDigitsLetter %
\gdef\lstdrv@cobol@keys{ACCEPT,ACCESS,ADD,ADVANCING,AFTER,ALL,%
    ALPHABETIC,ALSO,ALTER,ALTERNATE,AND,ARE,AREA,AREAS,ASCENDING,%
    ASSIGN,AT,AUTHOR,BEFORE,BINARY,BLANK,BLOCK,BOTTOM,BY,CALL,CANCEL,%
    CD,CF,CH,CHARACTER,CHARACTERS,CLOCK-UNITS,CLOSE,COBOL,CODE,%
    CODE-SET,COLLATING,COLUMN,COMMA,COMMUNICATION,COMP,COMPUTE,%
    CONFIGURATION,CONTAINS,CONTROL,CONTROLS,CONVERTING,COPY,CORR,%
    CORRESPONDING,COUNT,CURRENCY,DATA,DATE,DATE-COMPILED,DATE-WRITTEN,%
    DAY,DE,DEBUG-CONTENTS,DEGUB-ITEM,DEBUG-LINE,DEBUG-NAME,DEBUG-SUB1,%
    DEBUG-SUB2,DEBUG-SUB3,DEBUGGING,DECIMAL-POINT,DECLARATIVES,DELETE,%
    DELIMITED,DELIMITER,DEPENDING,DESCENDING,DESTINATION,DETAIL,%
    DISABLE,DISPLAY,DIVIDE,DIVISION,DOWN,DUPLICATES,DYNAMIC,EGI,ELSE,%
    EMI,ENABLE,END,END-OF-PAGE,ENTER,ENVIRONMENT,EOP,EQUAL,ERROR,ESI,%
    EVERY,EXCEPTION,EXIT,EXTEND,FD,FILE,FILE-CONTROL,FILLER,FINAL,%
    FIRST,FOOTING,FOR,FROM,GENERATE,GIVING,GO,GREATER,GROUP,HEADING,%
    HIGH-VALUE,HIGH-VALUES,I-O,I-O-CONTROL,IDENTIFICATION,IF,IN,INDEX,%
    INDEXED,INDICATE,INITIAL,INITIATE,INPUT,INPUT-OUTPUT,INSPECT,%
    INSTALLATION,INTO,INVALID,IS,JUST,JUSTIFIED,KEY,LABEL,LAST,LEADING,%
    LEFT,LENGTH,LESS,LIMIT,LIMITS,LINAGE,LINAGE-COUNTER,LINE,%
    LINE-COUNTER,LINES,LINKAGE,LOCK,LOW-VALUE,LOW-VALUES,MEMORY,MERGE,%
    MESSAGE,MODE,MODULES,MOVE,MULTIPLE,MULTIPLY,NATIVE,NEGATIVE,NEXT,%
    NO,NOT,NUMBER,NUMERIC,OBJECT-COMPUTER,OCCURS,OF,OFF,OMITTED,ON,%
    OPEN,OPTIONAL,OR,ORGANIZATION,OUTPUT,OVERFLOW,PAGE,PAGE-COUNTER,%
    PERFORM,PF,PH,PIC,PICTURE,PLUS,POINTER,POSITION,PRINTING,POSITIVE,%
    PRINTING,PROCEDURE,PROCEDURES,PROCEED,PROGRAM,PROGRAM-ID,QUEUE,%
    QUOTE,QUOTES,RANDOM,RD,READ,RECEIVE,RECORD,RECORDING,RECORDS,%
    REDEFINES,REEL,REFERENCES,RELATIVE,RELEASE,REMAINDER,REMOVAL,%
    RENAMES,REPLACING,REPORT,REPORTING,REPORTS,RERUN,RESERVE,RESET,%
    RETURN,REVERSED,REWIND,REWRITE,RF,RH,RIGHT,ROUNDED,RUN,SAME,SD,%
    SEARCH,SECTION,SECURITY,SEGMENT,SEGMENT-LIMIT,SELECT,SEND,SENTENCE,%
    SEPARATE,SEQUENCE,SEQUENTIAL,SET,SIGN,SIZE,SORT,SORT-MERGE,SOURCE,%
    SOURCE-COMPUTER,SPACE,SPACES,SPECIAL-NAMES,STANDARD,START,STATUS,%
    STOP,STRING,SUB-QUEUE-1,SUB-QUEUE-2,SUB-QUEUE-3,SUBTRACT,SUM,%
    SYMBOLIC,SYNC,SYNCHRONIZED,TABLE,TALLYING,TAPE,TERMINAL,TERMINATE,%
    TEXT,THAN,THROUGH,THRU,TIME,TIMES,TO,TOP,TRAILING,TYPE,UNIT,%
    UNSTRING,UNTIL,UP,UPON,USAGE,USE,USING,VALUE,VALUES,VARYING,WHEN,%
    WITH,WORDS,WORKING-STORAGE,WRITE,ZERO,ZEROES,ZEROS}%
\gdef\lstdrv@cobol@keys@eightyfive{ALPHABET,ALPHABETIC-LOWER,%
    ALPHABETIC-UPPER,ALPHANUMERIC,ALPHANUMERIC-EDITED,ANY,CLASS,COMMON,%
    CONTENT,CONTINUE,DAY-OF-WEEK,END-ADD,END-CALL,END-COMPUTE,%
    END-DELETE,END-DIVIDE,END-EVALUATE,END-IF,END-MULTIPLY,END-PERFORM,%
    END-READ,END-RECEIVE,END-RETURN,END-REWRITE,END-SEARCH,END-START,%
    END-STRING,END-SUBTRACT,END-UNSTRING,END-WRITE,EVALUATE,EXTERNAL,%
    FALSE,GLOBAL,INITIALIZE,NUMERIC-EDITED,ORDER,OTHER,PACKED-DECIMAL,%
    PADDING,PURGE,REFERENCE,RELOAD,REPLACE,STANDARD-1,STANDARD-2,TEST,%
    THEN,TRUE}%
\gdef\lstdrv@cobol@keys@ibm{ADDRESS,BEGINNING,COMP-3,COMP-4,%
    COMPUTATIONAL,COMPUTATIONAL-3,COMPUTATIONAL-4,DISPLAY-1,EGCS,EJECT,%
    ENDING,ENTRY,GOBACK,ID,MORE-LABELS,NULL,NULLS,PASSWORD,RECORDING,%
    RETURN-CODE,SERVICE,SKIP1,SKIP2,SKIP3,SORT-CONTROL,SORT-RETURN,%
    SUPPRESS,TITLE,WHEN-COMPILED}%
\endgroup %
%    \end{macrocode}
% \end{macro}\end{macro}\end{macro}
%
% \begin{macro}{\lstdrv@cobol@}
% Now the main driver macro:
%    \begin{macrocode}
\begingroup \makeatletter %
\catcode`\"=12%
\gdef\lstdrv@cobol@{%
    \keywords{\lstdrv@cobol@keys,\lstdrv@cobol@keys@eightyfive}%
    \sensitivefalse % ???
    \DeclareCommentLine\relax %
    \DeclareSingleComment stuff \relax %
    \stringizer[d]{"}\lstbaseem{0.65}%
    \lst@DefineCatcodes{11}%
    \let\lst@PrePL\lstdrv@cobol@PrePL %
    \let\lst@PostPL\relax}%
%    \end{macrocode}
% \end{macro}
%
% \begin{macro}{\lstdrv@cobol@1985}
% \begin{macro}{\lstdrv@cobol@1974}
% \begin{macro}{\lstdrv@cobol@ibm}
% And macros for the optional argument:
%    \begin{macrocode}
\global\@namedef{lstdrv@cobol@1985}{\lstdrv@cobol@}%
\global\@namedef{lstdrv@cobol@1974}{\lstdrv@cobol@ %
    \keywords{\lstdrv@cobol@keys}}%
\gdef\lstdrv@cobol@ibm{\lstdrv@cobol@ %
    \morekeywords{\lstdrv@cobol@keys@ibm}}%
\endgroup %
%    \end{macrocode}
% \end{macro}\end{macro}\end{macro}
%
% \begin{macro}{\lstdrv@cobol@PrePL}
% \begin{macro}{\lstdrv@cobol@TestComment}
% Comments are handled with the \verb!\lst@PrePL! mechanism. We simply
% look, if the seventh character is an asterix (and output the comment
% if necessary).
%    \begin{macrocode}
\begingroup \makeatletter %
\gdef\lstdrv@cobol@PrePL{\expandafter\lstdrv@cobol@TestComment %
    \lst@line\relax\relax\relax\relax\relax\relax\relax\relax}%
\catcode`\*=\active %
\gdef\lstdrv@cobol@TestComment#1#2#3#4#5#6#7#8\relax{%
    \ifx #7*%
        \lst@commenttrue \expandafter\lst@Tokenize\lst@line\relax %
        \let\lst@line\@empty \lst@commentfalse %
    \fi}%
\endgroup %
%</cobol>
%    \end{macrocode}
% \end{macro}\end{macro}
%
%
% \subsection{Comal 80}
% \begin{macro}{\lstdrv@comal@}
% Only the lonely driver macro.
%    \begin{macrocode}
%<*comal>
\begingroup \makeatletter %
\catcode`\"=12 %
\gdef\lstdrv@comal@{%
    \keywords{AND,AUTO,CASE,DATA,DEL,DIM,DIV,DO,ELSE,ENDCASE,ENDIF,%
        ENDPROC,ENDWHILE,EOD,EXEC,FALSE,FOR,GOTO,IF,INPUT,INT,LIST,%
        LOAD,MOD,NEW,NEXT,NOT,OF,OR,PRINT,PROC,RANDOM,RENUM,REPEAT,%
        RND,RUN,SAVE,SELECT,STOP,TAB,THEN,TRUE,UNTIL,WHILE,ZONE}%
    \sensitivefalse % ???
    \DeclareCommentLine //\relax %
    \DeclareSingleComment stuff \relax %
    \stringizer{"}\lstbaseem{0.65}%
    \lst@DefineCatcodes{\active}%
    \let\lst@PrePL \lst@PrePLDefault %
    \let\lst@PostPL\lst@PostPLDefault}%
\endgroup %
%</comal>
%    \end{macrocode}
% \end{macro}
%
%
% \subsection{Eiffel}
% \begin{macro}{\lstdrv@eiffel@}
% The same procedure \ldots
%    \begin{macrocode}
%<*eiffel>
\begingroup \makeatletter %
\catcode`\"=12 \catcode`\-=\active %
\gdef\lstdrv@eiffel@{%
    \keywords{alias,all,and,as,BIT,BOOLEAN,CHARACTER,check,class,%
        creation,Current,debug,deferred,do,DOUBLE,else,elseif,end,%
        ensure,expanded,export,external,false,feature,from,frozen,if,%
        implies,indexing,infix,inherit,inspect,INTEGER,invariant,is,%
        like,local,loop,NONE,not,obsolete,old,once,or,POINTER,prefix,%
        REAL,redefine,rename,require,rescue,Result,retry,select,%
        separate,STRING,strip,then,true,undefine,unique,until,variant,%
        when,xor}%
    \sensitivetrue %
    \DeclareCommentLine --\relax %
    \DeclareSingleComment stuff \relax %
    \stringizer{"}\lstbaseem{0.6}%
    \lst@DefineCatcodes{\active}%
    \let\lst@PrePL \lstdrv@eiffel@PrePL %
    \let\lst@PostPL\relax}%
\endgroup %
%    \end{macrocode}
% \end{macro}
%
% Not the same procedure: Since '$\%$' continues a string, we need a
% different comment character and let the percent be other (catcode)
% to look for it.
%    \begin{macrocode}
\begingroup \makeatletter %
\catcode`\&=14 \catcode`\%=12 &
%    \end{macrocode}
% \begin{macro}{\lstdrv@eiffel@PrePL}
% \begin{macro}{\lstdrv@eiffel@TestPercent}
% If a string started on a preceding line, we test for the percent.
% Refer how we cut a line into a comment and other source code.
%    \begin{macrocode}
\gdef\lstdrv@eiffel@PrePL{&
  \lst@ifstring &
      \expandafter\lstdrv@eiffel@TestPercent\lst@line%\relax &
  \fi}&
\gdef\lstdrv@eiffel@TestPercent#1%#2\relax{&
    \ifx\@empty#2\@empty \lst@PostPLDefault &
        \else \expandafter\lstdrv@eiffel@CutPercent\lst@line\relax&
    \fi}&
%    \end{macrocode}
% \end{macro}\end{macro}
%
% \begin{macro}{\lstdrv@eiffel@CutPercent}
% To output the characters in front of the percent we temporary switch
% the string boolean. Atferwards we redefine the input line, so that
% the normal line processing continues the string.
%    \begin{macrocode}
\gdef\lstdrv@eiffel@CutPercent#1%#2\relax{&
    \lst@stringfalse \lst@Tokenize#1\relax &
    \lst@stringtrue \def\lst@line{%#2}}&
\endgroup %
%</eiffel>
%    \end{macrocode}
% \end{macro}
%
%
% \subsection{Elan}
% \begin{macro}{\lstdrv@elan@keys}
% Since we need digits as letters, the keywords first:
%    \begin{macrocode}
%<*elan>
\begingroup \makeatletter %
\lst@MakeDigitsLetter %
\gdef\lstdrv@elan@keys{ABS,AND,BOOL,CAND,CASE,CAT,COLUMNS,CONCR,CONJ,%
    CONST,COR,DECR,DEFINES,DET,DIV,DOWNTO,ELIF,ELSE,END,ENDIF,ENDOP,%
    ENDPACKET,ENDPROC,ENDREP,ENDSELECT,FALSE,FI,FILE,FOR,FROM,IF,INCR,%
    INT,INV,LEAVE,LENGTH,LET,MOD,NOT,OF,OP,OR,OTHERWISE,PACKET,PROC,%
    REAL,REP,REPEAT,ROW,ROWS,SELECT,SIGN,STRUCT,SUB,TEXT,THEN,TRANSP,%
    TRUE,TYPE,UNTIL,UPTO,VAR,WHILE,WITH,XOR,%
    maxint,sign,abs,min,max,random,initializerandom,subtext,code,%
    replace,text,laenge,pos,compress,change,maxreal,smallreal,floor,pi,%
    e,ln,log2,log10,sqrt,exp,tan,tand,sin,sind,cos,cosd,arctan,arctand,%
    int,real,lastconversionok,put,putline,line,page,get,getline,input,%
    output,sequentialfile,maxlinelaenge,reset,eof,close,complexzero,%
    complexone,complexi,complex,realpart,imagpart,dphi,phi,vector,norm,%
    replace,matrix,idn,row,column,sub,replacerow,replacecolumn,%
    replaceelement,transp,errorsstop,stop}%
\endgroup %
%    \end{macrocode}
% \end{macro}
%
% \begin{macro}{\lstdrv@elan@}
% Now we define the main driver macro:
%    \begin{macrocode}
\begingroup \makeatletter %
\catcode`\"=12 %
\gdef\lstdrv@elan@{%
    \keywords{\lstdrv@elan@keys}%
    \sensitivetrue %
    \DeclareCommentLine\relax %
    \DeclareSingleComment stuff \relax %
    \stringizer[d]{"}\lstbaseem{0.65}%
    \lst@DefineCatcodes{\active}%
    \let\lst@PrePL \lst@PrePLDefault %
    \let\lst@PostPL\lst@PostPLDefault}%
\endgroup %
%</elan>
%    \end{macrocode}
% \end{macro}
%
%
% \subsection{Fortran}
% \begin{macro}{\lstdrv@fortran@keys}
% Common 'keywords' of Fortran 90 and Fortran 77:
%    \begin{macrocode}
%<*fortran>
\begingroup \makeatletter %
\gdef\lstdrv@fortan@keys{ACCESS,ASSIGN,BACKSPACE,BLANK,BLOCK,CALL,%
    CHARACTER,CLOSE,COMMON,COMPLEX,CONTINUE,DATA,DIMENSION,DIRECT,DO,%
    DOUBLE,ELSE,END,ENTRY,EOF,EQUIVALENCE,ERR,EXIST,EXTERNAL,FILE,%
    FMT,FORM,FORMAT,FORMATTED,FUNCTION,GO,TO,IF,IMPLICIT,INQUIRE,%
    INTEGER,INTRINSIC,IOSTAT,LOGICAL,NAMED,NEXTREC,NUMBER,OPEN,OPENED,%
    PARAMETER,PAUSE,PRECISION,PRINT,PROGRAM,READ,REAL,REC,RECL,%
    RETURN,REWIND,SEQUENTIAL,STATUS,STOP,SUBROUTINE,THEN,TYPE,%
    UNFORMATTED,UNIT,WRITE}%
\endgroup %
%    \end{macrocode}
% \end{macro}
%
% \begin{macro}{\lstdrv@fortran@}
% The driver macro:
%    \begin{macrocode}
\begingroup \makeatletter %
\catcode`\"=12 %
\gdef\lstdrv@fortran@{%
    \keywords{\lstdrv@fortan@keys,ACTION,ADVANCE,ALLOCATE,ALLOCATABLE,%
        ASSIGNMENT,CASE,CONTAINS,CYCLE,DEALLOCATE,DEFAULT,DELIM,EXIT,%
        IN,NONE,IN,OUT,INTENT,INTERFACE,IOLENGTH,KIND,LEN,MODULE,NAME,%
        NAMELIST,NMT,NULLIFY,ONLY,OPERATOR,OPTIONAL,OUT,PAD,POINTER,%
        POSITION,PRIVATE,PUBLIC,READWRITE,RECURSIVE,RESULT,SELECT,%
        SEQUENCE,SIZE,STAT,TARGET,USE,WHERE,WHILE,%
        BLOCKDATA,DOUBLEPRECISION,ELSEIF,ENDBLOCKDATA,ENDDO,ENDFILE,%
        ENDFUNCTION,ENDIF,ENDINTERFACE,ENDMODULE,ENDPROGRAM,ENDSELECT,%
        ENDSUBROUTINE,ENDTYPE,ENDWHERE,GOTO,INOUT,SELECTCASE}%
    \sensitivefalse %% not Fortran standard %%
    \DeclareCommentLine !\relax %
    \DeclareSingleComment stuff \relax %
    \stringizer{"}\lstbaseem{0.6}%
    \lst@DefineCatcodes{\active}%
    \let\lst@PrePL \lstdrv@fortran@PrePL %
    \let\lst@PostPL\lst@PostPLDefault}%
%    \end{macrocode}
% \end{macro}
%
% \begin{macro}{\lstdrv@fortran@90}
% \begin{macro}{\lstdrv@fortran@77}
% Macros for the optional argument:
%    \begin{macrocode}
\global\@namedef{lstdrv@fortran@90}{\lstdrv@fortran@}%
\global\@namedef{lstdrv@fortran@77}{\lstdrv@fortran@ %
    \keywords{\lstdrv@fortan@keys,SAVE}%
    \DeclareCommentLine\relax}%
\endgroup %
%    \end{macrocode}
% \end{macro}\end{macro}
%
% \begin{macro}{\lstdrv@fortran@PrePL}
% \begin{macro}{\lstdrv@fortran@TestComment}
% Again something different: We use the \verb!\lst@PrePL! mechanism to
% handle '$*$' and 'C' comments. The defined test macro tests, if the
% first character of the input line is a star or an upper or lower case
% c. We output the line as comment and empty the input,
% if necessary. That's all.
%    \begin{macrocode}
\begingroup \makeatletter %
\gdef\lstdrv@fortran@PrePL{%
    \expandafter\lstdrv@fortran@TestComment\lst@line\relax}%
\catcode`\*=\active %
\gdef\lstdrv@fortran@TestComment#1#2\relax{\lst@commentfalse %
    \ifx #1*\lst@commenttrue %
    \else\if #1c\lst@commenttrue %% not Fortran standard %%
    \else\if #1C\lst@commenttrue %
    \fi \fi \fi %
    \lst@ifcomment %
        \expandafter\lst@Tokenize\lst@line\relax %
        \let\lst@line\@empty \lst@commentfalse %
    \fi}%
\endgroup %
%</fortran>
%    \end{macrocode}
% \end{macro}\end{macro}
%
%
% \subsection{Java}
% \begin{macro}{\lstdrv@java@}
% Nothing new.
%    \begin{macrocode}
%<*java>
\begingroup \makeatletter %
\catcode`\"=12 \catcode`\*=\active %
\gdef\lstdrv@java@{%
    \keywords{abstract,boolean,break,byte,case,catch,char,class,const,%
        continue,default,do,double,else,extends,final,finally,float,%
        for,goto,if,implements,import,instanceof,int,interface,long,%
        native,new,null,package,private,protected,public,return,short,%
        static,super,switch,synchronized,this,throw,throws,transient,%
        try,void,volatile,while,true,false}%
    \sensitivetrue %
    \DeclareCommentLine //\relax %
    \DeclareSingleComment /* */\relax %
    \stringizer{"}\lstbaseem{0.6}%
    \lst@DefineCatcodes{\active}%
    \let\lst@PrePL \lst@PrePLDefault %
    \let\lst@PostPL\lst@PostPLDefault}%
\endgroup %
%</java>
%    \end{macrocode}
% \end{macro}
%
%
% \subsection{Lisp}
% \begin{macro}{\lstdrv@lisp@}
% The keywords are the 'one-word' functions and macros of Common Lisp,
% i.e.\ words not containing a minus. And I left out the \texttt{caaaar},
% \ldots{} functions.
%    \begin{macrocode}
%<*lisp>
\begingroup \makeatletter %
\catcode`\"=12 \catcode`\-=11 %
\gdef\lstdrv@lisp@{%
    \keywords{abort,abs,acons,acos,acosh,adjoin,alphanumericp,alter,%
        append,apply,apropos,aref,arrayp,ash,asin,asinh,assoc,atan,%
        atanh,atom,bit,boole,boundp,break,butlast,byte,catenate,%
        ceiling,cerror,char,character,characterp,choose,chunk,cis,%
        close,clrhash,coerce,collect,commonp,compile,complement,%
        complex,complexp,concatenate,conjugate,cons,consp,constantp,%
        continue,cos,cosh,cotruncate,count,delete,denominator,%
        describe,directory,disassemble,documentation,dpb,dribble,%
        ed,eighth,elt,enclose,endp,eq,eql,equal,equalp,error,eval,%
        evalhook,evenp,every,exp,expand,export,expt,fboundp,fceiling,%
        fdefinition,ffloor,fifth,fill,find,first,float,floatp,floor,%
        fmakunbound,format,fourth,fround,ftruncate,funcall,functionp,%
        gatherer,gcd,generator,gensym,gentemp,get,getf,gethash,%
        identity,imagpart,import,inspect,integerp,intern,intersection,%
        tively,isqrt,keywordp,last,latch,lcm,ldb,ldiff,length,list,%
        listen,listp,load,log,logand,logbitp,logcount,logeqv,logior,%
        lognand,lognor,lognot,logtest,logxor,macroexpand,makunbound,%
        map,mapc,mapcan,mapcar,mapcon,maphash,mapl,maplist,mask,max,%
        member,merge,min,mingle,minusp,mismatch,mod,namestring,%
        nbutlast,nconc,nintersection,ninth,not,notany,notevery,%
        nreconc,nreverse,nsublis,nsubst,nth,nthcdr,null,numberp,%
        numerator,nunion,oddp,open,packagep,pairlis,pathname,pathnamep,%
        phase,plusp,position,positions,pprint,previous,princ,print,%
        proclaim,provide,random,rassoc,rational,rationalize,rationalp,%
        read,readtablep,realp,realpart,reduce,rem,remhash,remove,%
        remprop,replace,require,rest,revappend,reverse,room,round,%
        rplaca,rplacd,sbit,scan,schar,search,second,series,set,seventh,%
        shadow,signal,signum,sin,sinh,sixth,sleep,some,sort,split,%
        sqrt,streamp,string,stringp,sublis,subseq,subseries,subsetp,%
        subst,substitute,subtypep,svref,sxhash,symbolp,tailp,tan,tanh,%
        tenth,terpri,third,truename,truncate,typep,unexport,unintern,%
        union,until,values,vector,vectorp,warn,write,zerop,%
        and,assert,case,ccase,cond,ctypecase,decf,declaim,defclass,%
        defconstant,defgeneric,defmacro,defmethod,defpackage,%
        defparameter,defsetf,defstruct,deftype,defun,defvar,do,dolist,%
        dotimes,ecase,encapsulated,etypecase,flet,formatter,gathering,%
        incf,iterate,labels,let,locally,loop,macrolet,mapping,or,pop,%
        producing,prog,psetf,psetq,push,pushnew,remf,return,rotatef,%
        setf,shiftf,step,time,trace,typecase,unless,untrace,when}%
    \sensitivetrue % ???
    \DeclareCommentLine;\relax %
    \DeclareSingleComment stuff \relax %
    \stringizer[b]{"}\lstbaseem{0.6}%
    \lst@DefineCatcodes{11}%
    \let\lst@PrePL \lst@PrePLDefault %
    \let\lst@PostPL\lst@PostPLDefault}%
\endgroup %
%</lisp>
%    \end{macrocode}
% \end{macro}
%
%
% \subsection{Logo}
% \begin{macro}{\lstdrv@logo@}
% I don't know where I have the keywords from and what kind of Logo
% it is. Help me!
%    \begin{macrocode}
%<*logo>
\begingroup \makeatletter %
\gdef\lstdrv@logo@{% ??? end,unix
    \keywords{and,atan,arctan,both,break,bf,bl,butfirst,butlast,cbreak,%
        close,co,continue,cos,count,clearscreen,cs,debquit,describe,%
        diff,difference,ed,edit,either,emptyp,equalp,er,erase,errpause,%
        errquit,fifp,filefprint,fifty,fileftype,fip,fileprint,fird,%
        fileread,fity,filetype,fiwd,fileword,f,first,or,fp,fprint,fput,%
        fty,ftype,full,fullscreen,go,bye,goodbye,gprop,greaterp,help,%
        if,iff,iffalse,ift,iftrue,nth,item,keyp,llast,lessp,list,local,%
        lput,make,max,maximum,memberp,memtrace,min,minimum,namep,not,%
        numberp,oflush,openr,openread,openw,openwrite,op,output,pause,%
        plist,pots,pow,pprop,pps,pr,print,product,quotient,random,rc,%
        readchar,rl,readlist,remprop,repcount,repeat,request,rnd,run,%
        se,sentence,sentencep,setc,setcolor,setipause,setqpause,po,%
        show,sin,split,splitscreen,sqrt,stop,sum,test,text,textscreen,%
        thing,to,tone,top,toplevel,type,untrace,wait,word,wordp,%
        yaccdebug,is,mod,remainder,trace,zerop,%
        back,bk,bto,btouch,fd,forward,fto,ftouch,getpen,heading,hit,%
        hitoot,ht,hideturtle,loff,lampoff,lon,lampon,lt,left,lot,%
        lotoot,lto,ltouch,penc,pencolor,pd,pendown,pe,penerase,penmode,%
        pu,penup,px,penreverse,rt,right,rto,rtouch,scrunch,seth,%
        setheading,setscrun,setscrunch,setxy,shownp,st,showturtle,%
        towardsxy,clean,wipeclean,xcor,ycor,tur,turtle,display,dpy}%
    \sensitivefalse % ???
    \DeclareCommentLine\relax %
    \DeclareSingleComment stuff \relax %
    \stringizer{}\lstbaseem{0.6}%
    \lst@DefineCatcodes{\active}%
    \let\lst@PrePL \lst@PrePLDefault %
    \let\lst@PostPL\lst@PostPLDefault}%
\endgroup %
%</logo>
%    \end{macrocode}
% \end{macro}
%
%
% \subsection{Matlab}
% \begin{macro}{\lstdrv@matlab@keys}
% \begin{macro}{\lstdrv@matlab@}
% Once more \ldots
%    \begin{macrocode}
%<*matlab>
\begingroup \makeatletter %
\lst@MakeDigitsLetter %
\gdef\lstdrv@matlab@keys{gt,lt,gt,lt,amp,%
    abs,acos,acosh,acot,acoth,acsc,acsch,all,angle,ans,any,asec,asech,%
    asin,asinh,atan,atan2,atanh,auread,auwrite,axes,axis,balance,bar,%
    bessel,besselk,bessely,beta,betainc,betaln,blanks,bone,break,%
    brighten,capture,cart2pol,cart2sph,caxis,cd,cdf2rdf,cedit,ceil,%
    chol,cla,clabel,clc,clear,clf,clock,close,colmmd,Colon,colorbar,%
    colormap,ColorSpec,colperm,comet,comet3,compan,compass,computer,%
    cond,condest,conj,contour,contour3,contourc,contrast,conv,conv2,%
    cool,copper,corrcoef,cos,cosh,cot,coth,cov,cplxpair,cputime,cross,%
    csc,csch,csvread,csvwrite,cumprod,cumsum,cylinder,date,dbclear,%
    dbcont,dbdown,dbquit,dbstack,dbstatus,dbstep,dbstop,dbtype,dbup,%
    ddeadv,ddeexec,ddeinit,ddepoke,ddereq,ddeterm,ddeunadv,deblank,%
    dec2hex,deconv,del2,delete,demo,det,diag,diary,diff,diffuse,dir,%
    disp,dlmread,dlmwrite,dmperm,dot,drawnow,echo,eig,ellipj,ellipke,%
    else,elseif,end,engClose,engEvalString,engGetFull,engGetMatrix,%
    engOpen,engOutputBuffer,engPutFull,engPutMatrix,engSetEvalCallback,%
    engSetEvalTimeout,engWinInit,eps,erf,erfc,erfcx,erfinv,error,%
    errorbar,etime,etree,eval,exist,exp,expint,expm,expo,eye,fclose,%
    feather,feof,ferror,feval,fft,fft2,fftshift,fgetl,fgets,figure,%
    fill,fill3,filter,filter2,find,findstr,finite,fix,flag,fliplr,%
    flipud,floor,flops,fmin,fmins,fopen,for,format,fplot,fprintf,fread,%
    frewind,fscanf,fseek,ftell,full,function,funm,fwrite,fzero,gallery,%
    gamma,gammainc,gammaln,gca,gcd,gcf,gco,get,getenv,getframe,ginput,%
    global,gplot,gradient,gray,graymon,grid,griddata,gtext,hadamard,%
    hankel,help,hess,hex2dec,hex2num,hidden,hilb,hist,hold,home,hostid,%
    hot,hsv,hsv2rgb,i,if,ifft,ifft2,imag,image,imagesc,Inf,info,input,%
    int2str,interp1,interp2,interpft,inv,invhilb,isempty,isglobal,%
    ishold,isieee,isinf,isletter,isnan,isreal,isspace,issparse,isstr,j,%
    jet,keyboard,kron,lasterr,lcm,legend,legendre,length,lin2mu,line,%
    linspace,load,log,log10,log2,loglog,logm,logspace,lookfor,lower,ls,%
    lscov,lu,magic,matClose,matDeleteMatrix,matGetDir,matGetFp,%
    matGetFull,matGetMatrix,matGetNextMatrix,matGetString,matlabrc,%
    matlabroot,matOpen,matPutFull,matPutMatrix,matPutString,max,mean,%
    median,menu,mesh,meshc,meshgrid,meshz,mexAtExit,mexCallMATLAB,%
    mexdebug,mexErrMsgTxt,mexEvalString,mexFunction,mexGetFull,%
    mexGetMatrix,mexGetMatrixPtr,mexPrintf,mexPutFull,mexPutMatrix,%
    mexSetTrapFlag,min,more,movie,moviein,mu2lin,mxCalloc,%
    mxCopyCharacterToPtr,mxCopyComplex16ToPtr,mxCopyInteger4ToPtr,%
    mxCopyPtrToCharacter,mxCopyPtrToComplex16,mxCopyPtrToInteger4,%
    mxCopyPtrToReal8,mxCopyReal8ToPtr,mxCreateFull,mxCreateSparse,%
    mxCreateString,mxFree,mxFreeMatrix,mxGetIr,mxGetJc,mxGetM,mxGetN,%
    mxGetName,mxGetNzmax,mxGetPi,mxGetPr,mxGetScalar,mxGetString,%
    mxIsComplex,mxIsFull,mxIsNumeric,mxIsSparse,mxIsString,%
    mxIsTypeDouble,mxSetIr,mxSetJc,mxSetM,mxSetN,mxSetName,mxSetNzmax,%
    mxSetPi,mxSetPr,NaN,nargchk,nargin,nargout,newplot,nextpow2,nnls,%
    nnz,nonzeros,norm,normest,null,num2str,nzmax,ode23,ode45,orient,%
    orth,pack,pascal,patch,path,pause,pcolor,pi,pink,pinv,plot,plot3,%
    pol2cart,polar,poly,polyder,polyeig,polyfit,polyval,polyvalm,pow2,%
    print,printopt,prism,prod,pwd,qr,qrdelete,qrinsert,quad,quad8,quit,%
    quiver,qz,rand,randn,randperm,rank,rat,rats,rbbox,rcond,real,%
    realmax,realmin,refresh,rem,reset,reshape,residue,return,rgb2hsv,%
    rgbplot,rootobject,roots,rose,rosser,rot90,rotate,round,rref,%
    rrefmovie,rsf2csf,save,saxis,schur,sec,sech,semilogx,semilogy,set,%
    setstr,shading,sign,sin,sinh,size,slice,sort,sound,spalloc,sparse,%
    spaugment,spconvert,spdiags,specular,speye,spfun,sph2cart,sphere,%
    spinmap,spline,spones,spparms,sprandn,sprandsym,sprank,sprintf,spy,%
    sqrt,sqrtm,sscanf,stairs,startup,std,stem,str2mat,str2num,strcmp,%
    strings,strrep,strtok,subplot,subscribe,subspace,sum,surf,surface,%
    surfc,surfl,surfnorm,svd,symbfact,symmmd,symrcm,tan,tanh,tempdir,%
    tempname,terminal,text,tic,title,tmp,toc,toeplitz,trace,trapz,tril,%
    triu,type,uicontrol,uigetfile,uimenu,uiputfile,unix,unwrap,upper,%
    vander,ver,version,view,viewmtx,waitforbuttonpress,waterfall,%
    wavread,wavwrite,what,whatsnew,which,while,white,whitebg,who,whos,%
    wilkinson,wk1read,wk1write,xlabel,xor,ylabel,zeros,zlabel,zoom}%
\endgroup %
%    \end{macrocode}
%    \begin{macrocode}
\begingroup \makeatletter %
\catcode`\"=12 %
\gdef\lstdrv@matlab@{%
    \keywords{\lstdrv@matlab@keys}%
    \sensitivetrue %
    \DeclareCLPercent %
    \DeclareSingleComment stuff \relax %
    \stringizer{'}\lstbaseem{0.6}%
    \lst@DefineCatcodes{\active}%
    \let\lst@PrePL \lst@PrePLDefault %
    \let\lst@PostPL\lst@PostPLDefault}%
\endgroup %
%</matlab>
%    \end{macrocode}
% \end{macro}\end{macro}
%
%
% \subsection{Modula-2}
% \begin{macro}{\lstdrv@modula@}
% And once again \ldots
%    \begin{macrocode}
%<*modula>
\begingroup \makeatletter %
\catcode`\"=12 \catcode`\*=\active %
\gdef\lstdrv@modula@{%
    \keywords{AND,ARRAY,BEGIN,BY,CASE,CONST,DIV,DO,ELSE,ELSIF,END,EXIT,%
        EXPORT,FOR,FROM,IF,IMPLEMENTATION,IMPORT,IN,MOD,MODULE,NOT,OF,%
        OR,POINTER,PROCEDURE,QUALIFIED,RECORD,REPEAT,RETURN,SET,THEN,%
        TYPE,UNTIL,VAR,WHILE,WITH,ABS,BITSET,BOOLEAN,CAP,CARDINAL,CHAR,%
        CHR,DEC,EXCL,FALSE,FLOAT,HALT,HIGH,INC,INCL,INTEGER,LONGCARD,%
        LONGINT,LONGREAL,MAX,MIN,NIL,ODD,ORD,PROC,REAL,SIZE,TRUE,TRUNC,%
        VAL}%
    \sensitivetrue %
    \DeclareCommentLine\relax %
    \DeclareNestedComment (* *)\relax %
    \stringizer[d]{'"}\lstbaseem{0.65}%
    \lst@DefineCatcodes{\active}%
    \let\lst@PrePL \lst@PrePLDefault %
    \let\lst@PostPL\lst@PostPLDefault}%
\endgroup %
%</modula>
%    \end{macrocode}
% \end{macro}
%
%
% \subsection{Oberon-2}
% \begin{macro}{\lstdrv@oberon@}
% 'Nearly' Modula-2:
%    \begin{macrocode}
%<*oberon>
\begingroup \makeatletter %
\catcode`\"=12 \catcode`\*=\active %
\gdef\lstdrv@oberon@{%
    \keywords{ARRAY,BEGIN,BOOLEAN,BY,CASE,CHAR,CONST,DIV,DO,ELSE,ELSIF,%
        END,EXIT,FALSE,FOR,IF,IMPORT,IN,INTEGER,IS,LONGINT,LONGREAL,%
        LOOP,MOD,MODULE,NIL,OF,OR,POINTER,PROCEDURE,REAL,RECORD,REPEAT,%
        RETURN,SET,SHORTINT,THEN,TO,TRUE,TYPE,UNTIL,VAR,WHILE,WITH,%
        ABS,ASH,CAP,CHR,COPY,DEC,ENTIER,EXCL,HALT,INC,INCL,LEN,LONG,%
        MAX,MIN,NEW,ODD,ORD,SHORT,SIZE}
    \sensitivetrue %
    \DeclareCommentLine\relax %
    \DeclareNestedComment (* *)\relax %
    \stringizer[d]{'"}\lstbaseem{0.65}%
    \lst@DefineCatcodes{\active}%
    \let\lst@PrePL \lst@PrePLDefault %
    \let\lst@PostPL\lst@PostPLDefault}%
\endgroup %
%</oberon>
%    \end{macrocode}
% \end{macro}
%
%
% \subsection{Pascal}
% \begin{macro}{\lstdrv@pascal@}
% Since we already defined pascal comments in section \ref{ssComments},
% no catcodes must be changed here.
%    \begin{macrocode}
%<*pascal>
\begingroup \makeatletter %
\gdef\lstdrv@pascal@{%
    \keywords{alfa,and,array,begin,boolean,byte,case,char,const,div,do,%
        downto,else,end,false,file,for,function,get,goto,if,in,integer,%
        label,maxint,mod,new,not,of,or,pack,packed,page,program,%
        procedure,put,read,readln,real,record,repeat,reset,rewrite,set,%
        text,then,to,true,type,unpack,until,var,while,with,write,%
        writeln}%
    \sensitivefalse %
    \DeclareCommentLine\relax %
    \DeclareDoubleCommentPascal %
    \stringizer[d]{'}\lstbaseem{0.6}%
    \lst@DefineCatcodes{\active}%
    \let\lst@PrePL \lst@PrePLDefault %
    \let\lst@PostPL\lst@PostPLDefault}%
\endgroup %
%</pascal>
%    \end{macrocode}
% \end{macro}
%
%
% \subsection{Pascal XSC}
% \begin{macro}{\lstdrv@pxsc@}
% Tell me, if you want more words to be keywords.
%    \begin{macrocode}
%<*pxsc>
\selectlisting{pascal}%
\begingroup \makeatletter %
\gdef\lstdrv@pxsc@{\lstdrv@pascal@ %
    \morekeywords{dynamic,external,forward,global,module,nil,operator,%
        priority,sum,type,use,dispose,mark,page,release,cimatrix,%
        cinterval,civector,cmatrix,complex,cvector,dotprecision,%
        imatrix,interval,ivector,rmatrix,rvector,string,im,inf,re,sup,%
        chr,comp,eof,eoln,expo,image,ival,lb,lbound,length,loc,mant,%
        maxlength,odd,ord,pos,pred,round,rval,sign,substring,succ,%
        trunc,ub,ubound}}%
\endgroup %
%</pxsc>
%    \end{macrocode}
% \end{macro}
%
%
% \subsection{Turbo Pascal}
% \begin{macro}{\lstdrv@tp@}
% The keywords are the reserved words and predefined functions and
% procedures of Turbo Pascal 6.0, I think. It's a long list \ldots
%    \begin{macrocode}
%<*tp>
\selectlisting{pascal}%
\begingroup \makeatletter %
\gdef\lstdrv@tp@{\lstdrv@pascal@ %
    \morekeywords{asm,constructor,destructor,implementation,inline,%
        interface,nil,object,shl,shr,string,unit,uses,xor,%
        Abs,Addr,ArcTan,Chr,Concat,Copy,Cos,CSeg,DiskFree,DiskSize,%
        DosExitCode,DosVersion,DSeg,EnvCount,EnvStr,Eof,Eoln,Exp,%
        FExpand,FilePos,FileSize,Frac,FSearch,GetBkColor,GetColor,%
        GetDefaultPalette,GetDriverName,GetEnv,GetGraphMode,GetMaxMode,%
        GetMaxX,GetMaxY,GetModeName,GetPaletteSize,GetPixel,GetX,GetY,%
        GraphErrorMsg,GraphResult,Hi,ImageSize,InstallUserDriver,%
        InstallUserFont,Int,IOResult,KeyPressed,Length,Lo,MaxAvail,%
        MemAvail,MsDos,Odd,Ofs,Ord,OvrGetBuf,OvrGetRetry,ParamCount,%
        ParamStr,Pi,Pos,Pred,Ptr,Random,ReadKey,Round,SeekEof,SeekEoln,%
        Seg,SetAspectRatio,Sin,SizeOf,Sound,SPtr,Sqr,Sqrt,SSeg,Succ,%
        Swap,TextHeight,TextWidth,Trunc,TypeOf,UpCase,WhereX,WhereY,%
        Append,Arc,Assign,AssignCrt,Bar,Bar3D,BlockRead,BlockWrite,%
        ChDir,Circle,ClearDevice,ClearViewPort,Close,CloseGraph,ClrEol,%
        ClrScr,Dec,Delay,Delete,DelLine,DetectGraph,Dispose,DrawPoly,%
        Ellipse,Erase,Exec,Exit,FillChar,FillEllipse,FillPoly,%
        FindFirst,FindNext,FloodFill,Flush,FreeMem,FSplit,GetArcCoords,%
        GetAspectRatio,GetDate,GetDefaultPalette,GetDir,GetCBreak,%
        GetFAttr,GetFillSettings,GetFTime,GetImage,GetIntVec,%
        GetLineSettings,GetMem,GetPalette,GetTextSettings,GetTime,%
        GetVerify,GetViewSettings,GoToXY,Halt,HighVideo,Inc,InitGraph,%
        Insert,InsLine,Intr,Keep,Line,LineRel,LineTo,LowVideo,Mark,%
        MkDir,Move,MoveRel,MoveTo,MsDos,New,NormVideo,NoSound,OutText,%
        OutTextXY,OvrClearBuf,OvrInit,OvrInitEMS,OvrSetBuf,PackTime,%
        PieSlice,PutImage,PutPixel,Randomize,Rectangle,Release,Rename,%
        RestoreCrtMode,RmDir,RunError,Sector,Seek,SetActivePage,%
        SetAllPalette,SetBkColor,SetCBreak,SetColor,SetDate,SetFAttr,%
        SetFillPattern,SetFillStyle,SetFTime,SetGraphBufSize,%
        SetGraphMode,SetIntVec,SetLineStyle,SetPalette,SetRGBPalette,%
        SetTextBuf,SetTextJustify,SetTextStyle,SetTime,SetUserCharSize,%
        SetVerify,SetViewPort,SetVisualPage,SetWriteMode,Sound,Str,%
        SwapVectors,TextBackground,TextColor,TextMode,Truncate,%
        UnpackTime,Val,Window}}%
\endgroup %
%</tp>
%    \end{macrocode}
% \end{macro}
%
%
% \subsection{Perl}
% \begin{macro}{\lstdrv@perl@keys}
% First the keywords, \ldots
%    \begin{macrocode}
%<*perl>
\begingroup \makeatletter %
\lst@MakeDigitsLetter %
\gdef\lstdrv@perl@keys{abs,accept,alarm,atan2,bind,binmode,bless,%
    caller,chdir,chmod,chomp,chop,chown,chr,chroot,close,closedir,%
    connect,continue,cos,crypt,dbmclose,dbmopen,defined,delete,die,do,%
    dump,each,else,elsif,endgrent,endhostent,endnetent,endprotoent,%
    endpwent,endservent,eof,eval,exec,exists,exit,exp,fcntl,fileno,%
    flock,for,foreach,fork,format,formline,getc,getgrent,getgrgid,%
    getgrnam,gethostbyaddr,gethostbyname,gethostent,getlogin,%
    getnetbyaddr,getnetbyname,getnetent,getpeername,getpgrp,getppid,%
    getpriority,getprotobyname,getprotobynumber,getprotoent,getpwent,%
    getpwnam,getpwuid,getservbyname,getservbyport,getservent,%
    getsockname,getsockopt,glob,gmtime,goto,grep,hex,if,import,index,%
    int,ioctl,join,keys,kill,last,lc,lcfirst,length,link,listen,local,%
    localtime,log,lstat,m,map,mkdir,msgctl,msgget,msgrcv,msgsnd,my,%
    next,no,oct,open,opendir,ord,pack,package,pipe,pop,pos,print,%
    printf,prototype,push,q,qq,quotemeta,qw,qx,rand,read,readdir,%
    readlink,recv,redo,ref,rename,require,reset,return,reverse,%
    rewinddir,rindex,rmdir,s,scalar,seek,seekdir,select,semctl,semget,%
    semop,send,setgrent,sethostent,setnetent,setpgrp,setpriority,%
    setprotoent,setpwent,setservent,setsockopt,shift,shmctl,shmget,%
    shmread,shmwrite,shutdown,sin,sleep,socket,socketpair,sort,splice,%
    split,sprintf,sqrt,srand,stat,study,sub,substr,symlink,syscall,%
    sysopen,sysread,system,syswrite,tell,telldir,tie,tied,time,times,%
    tr,truncate,uc,ucfirst,umask,undef,unless,unlink,unpack,unshift,%
    untie,until,use,utime,values,vec,wait,waitpid,wantarray,warn,while,%
    write,y}%
\endgroup %
%    \end{macrocode}
% \end{macro}
%
% \begin{macro}{\lstdrv@perl@}
% {\ldots} and now the main driver file macro.
%    \begin{macrocode}
\begingroup \makeatletter %
\catcode`\"=12 \catcode`\#=12 %
\gdef\lstdrv@perl@{%
    \keywords{\lstdrv@perl@keys}%
    \sensitivetrue %
    \DeclareCommentLine#\relax %
    \DeclareSingleComment stuff \relax %
    \stringizer[b]{"'}\lstbaseem{0.6}%
    \lst@DefineCatcodes{\active}%
    \let\lst@PrePL \lstdrv@perl@PrePL %
    \let\lst@PostPL\relax}%
%    \end{macrocode}
% \end{macro}
%
% \begin{macro}{\lstdrv@perl@ifPOD}
% \begin{macro}{\lstdrv@perl@PrepareListing}
% We use a switch to indicate PODs. Every (Perl) listing
% \verb!\lst@Begin! calls the macro \verb!\lstdrv@perl@PrepareListing!
% to reset the switch.
%    \begin{macrocode}
\gdef\lstdrv@perl@PODtrue{\let\lstdrv@perl@ifPOD\iftrue}%
\gdef\lstdrv@perl@PODfalse{\let\lstdrv@perl@ifPOD\iffalse}%
\gdef\lstdrv@perl@PrepareListing{\lstdrv@perl@PODfalse}%
%    \end{macrocode}
% \end{macro}\end{macro}
%
% \begin{macro}{\lstdrv@perl@PrePL}
% \begin{macro}{\lstdrv@perl@TestComment}
% \begin{macro}{\lstdrv@perl@TestCommentCut}
% \begin{macro}{\lstdrv@perl@TestSharp}
% Nothing else is really new now.
%    \begin{macrocode}
\gdef\lstdrv@perl@PrePL{%
    \lst@ifstring %
        \lst@stringfalse \lst@ProcessWhitespaces \lst@stringtrue %
    \else\lstdrv@perl@ifPOD %
        \expandafter\lstdrv@perl@TestCommentCut\lst@line=cut\relax %
    \else \expandafter\lstdrv@perl@TestComment\lst@line\relax\relax %
    \fi \fi}%
\gdef\lstdrv@perl@TestCommentCut#1=cut#2\relax{%
    \lst@commenttrue \expandafter\lst@Tokenize\lst@line\relax %
    \let\lst@line\@empty \lst@commentfalse %
    \ifx\@empty#2\@empty\else \lstdrv@perl@PODfalse \fi}%
%    \end{macrocode}
% We have to change some catcodes, to look for the sharp. In particular
% we need a new 'macro parameter symbol' ($\&$).
%    \begin{macrocode}
\catcode`\$=11 \catcode`\&=6 \catcode`\#=12 %
\gdef\lstdrv@perl@TestComment&1&2\relax{%
    \ifx &1=%
        \lstdrv@perl@PODtrue \lst@commenttrue %
        \expandafter\lst@Tokenize\lst@line\relax %
        \let\lst@line\@empty \lst@commentfalse %
    \else %
        \expandafter\lstdrv@perl@TestSharp\lst@line $#\relax %
    \fi}%
%    \end{macrocode}
% The \cs{lowercase} mechanism mentioned some time before to replace
% each \verb!$#! with \verb!$#!, where the latter \verb!#! has letter
% catcode. This avoids wrong comment detection.
%    \begin{macrocode}
\catcode`\~=11 \lccode`\~=`\#%
\lowercase{%
\gdef\lstdrv@perl@TestSharp&1$#&2\relax{%
    \ifx \@empty&2\@empty \else %
        \expandafter\lstdrv@perl@ReplaceSharp\lst@line\relax %
        \expandafter\lstdrv@perl@TestSharp\lst@line $#\relax %
    \fi}%
\gdef\lstdrv@perl@ReplaceSharp&1$#&2\relax{\def\lst@line{&1$~&2}}%
}\endgroup %
%</perl>
%    \end{macrocode}
% \end{macro}\end{macro}\end{macro}\end{macro}
%
%
% \subsection{PL/I}
% \begin{macro}{\lstdrv@pli@keys}
% \begin{macro}{\lstdrv@pli@}
% Same procedure \ldots
%    \begin{macrocode}
%<*pli>
\begingroup \makeatletter %
\lst@MakeDigitsLetter %
\gdef\lstdrv@pli@keys{ABS,ATAN,AUTOMATIC,AUTO,ATAND,BEGIN,BINARY,BIN,%
    BIT,BUILTIN,BY,CALL,CHARACTER,CHAR,CHECK,COLUMN,COL,COMPLEX,CPLX,%
    COPY,COS,COSD,COSH,DATA,DATE,DECIMAL,DEC,DECLARE,DCL,DO,EDIT,ELSE,%
    END,ENDFILE,ENDPAGE,ENTRY,EXP,EXTERNAL,EXT,FINISH,FIXED,%
    FIXEDOVERFLOW,FOFL,FLOAT,FORMAT,GET,GO,GOTO,IF,IMAG,INDEX,INITIAL,%
    INIT,INTERNAL,INT,LABEL,LENGTH,LIKE,LINE,LIST,LOG,LOG2,LOG10,MAIN,%
    MAX,MIN,MOD,NOCHECK,NOFIXEDOVERFLOW,NOFOFL,NOOVERFLOW,NOOFL,NOSIZE,%
    NOUNDERFLOW,NOUFL,NOZERODIVIDE,NOZDIV,ON,OPTIONS,OVERFLOW,OFL,PAGE,%
    PICTURE,PROCEDURE,PROC,PUT,READ,REPEAT,RETURN,RETURNS,ROUND,SIN,%
    SIND,SINH,SIZE,SKIP,SQRT,STATIC,STOP,STRING,SUBSTR,SUM,SYSIN,%
    SYSPRINT,TAN,TAND,TANH,THEN,TO,UNDERFLOW,UFL,VARYING,WHILE,WRITE,%
    ZERODIVIDE,ZDIV}%
\endgroup %
\begingroup \makeatletter %
\catcode`\*=\active %
\gdef\lstdrv@pli@{%
    \keywords{\lstdrv@pli@keys}%
    \sensitivefalse %
    \DeclareCommentLine\relax %
    \DeclareSingleComment /* */\relax %
    \stringizer{'}\lstbaseem{0.6}%
    \lst@DefineCatcodes{\active}%
    \let\lst@PrePL \lst@PrePLDefault %
    \let\lst@PostPL\lst@PostPLDefault}%
\endgroup %
%</pli>
%    \end{macrocode}
% \end{macro}\end{macro}
%
%
% \subsection{Simula}
% \begin{macro}{\lstdrv@simula@}
% Same as all the time.
%    \begin{macrocode}
%<*simula>
\begingroup \makeatletter %
\catcode`\"=12 %
\gdef\lstdrv@simula@{%
    \keywords{activate,after,array,at,before,begin,boolean,character,%
        class,comment,delay,detach,do,else,end,external,false,for,go,%
        goto,if,in,inner,inspect,integer,is,label,name,new,none,notext,%
        otherwise,prior,procedure,qua,reactivate,real,ref,resume,%
        simset,simulation,step,switch,text,then,this,to,true,until,%
        value,virtual,when,while}%
    \sensitivefalse %
    \DeclareCommentLine\relax %
    \DeclareSingleComment stuff \relax %
    \stringizer{"'}\lstbaseem{0.65}%
    \lst@DefineCatcodes{\active}%
    \let\lst@PrePL \lst@PrePLDefault %
    \let\lst@PostPL\lst@PostPLDefault}%
%    \end{macrocode}
% \end{macro}
%
% \begin{macro}{\lstdrv@simula@67}
% \begin{macro}{\lstdrv@simula@cii}
% \begin{macro}{\lstdrv@simula@dec}
% \begin{macro}{\lstdrv@simula@ibm}
% Macros for the options:
%    \begin{macrocode}
\global\@namedef{lstdrv@simula@67}{\lstdrv@simula@}%
\global\@namedef{lstdrv@simula@cii}{\lstdrv@simula@%
    \morekeywords{and,equiv,exit,impl,not,or,stop}}%
\global\@namedef{lstdrv@simula@dec}{\lstdrv@simula@%
    \morekeywords{and,eq,eqv,ge,gt,hidden,imp,le,long,lt,ne,not,%
        options,or,protected,short}}%
\global\@namedef{lstdrv@simula@ibm}{\lstdrv@simula@dec}%
\endgroup %
%    \end{macrocode}
% \end{macro}\end{macro}\end{macro}\end{macro}
%
% \begin{macro}{\lstdrv@simula@PrepareListing}
% \begin{macro}{\lstdrv@simula@OutputOther}
% \begin{macro}{\lstdrv@simula@Output@}
% \begin{macro}{\lstdrv@simula@AppendOther@}
% Refer the section about Algol how comments are implemented.
%    \begin{macrocode}
\begingroup \makeatletter %
\gdef\lstdrv@simula@commenttrue{%
    \global\let\lstdrv@simula@ifcomment\iftrue}%
\gdef\lstdrv@simula@commentfalse{%
    \global\let\lstdrv@simula@ifcomment\iffalse %
    \global\let\lstdrv@simula@closingcomment\@empty}%
\gdef\lstdrv@simula@PrepareListing{%
    \lstdrv@simula@commentfalse %
    \let\lst@OutputOther\lstdrv@simula@OutputOther %
    \let\lst@Output\lstdrv@simula@Output@ %
    \let\lst@AppendOther\lstdrv@simula@AppendOther}%
%    \end{macrocode}
%    \begin{macrocode}
\gdef\lstdrv@simula@OutputOther{%
    \expandafter\def\expandafter\lst@text\expandafter{\the\lst@other}%
    \ifx\lst@text\@empty\else %
        \lstdrv@simula@ifcomment\lst@MakeBox{\lst@commentstyle}\else %
        \lst@ifstring \lst@MakeStringBox{\lst@stringstyle}\else %
        \lst@ifcomment\lst@MakeBox{\lst@commentstyle}\else %
            \lst@MakeBox{}%
        \fi \fi \fi %
        \global\advance\lst@pos by -\lst@length %
        \lst@other{}\let\lst@text\@empty \lst@length0 %
    \fi}%
%    \end{macrocode}
%    \begin{macrocode}
\gdef\lstdrv@simula@Output@{%
    \ifx\lst@text\@empty\else %
        \lstdrv@simula@ifcomment %
            \ifx\@empty\lstdrv@simula@closingcomment %
                \lst@MakeBox{\lst@commentstyle}%
            \else %
                \expandafter\lst@ifoneof\lst@text\relax %
                {else,end,otherwise,when}%
                    {\lstdrv@simula@commentfalse %
                     \lst@MakeBox{\lst@keywordstyle}}%
                    {\lst@MakeBox{\lst@commentstyle}}%
            \fi \else %
        \lst@ifstring \lst@MakeStringBox{\lst@stringstyle}\else %
        \lst@ifcomment\lst@MakeBox{\lst@commentstyle}\else %
        \expandafter\lst@KeywordOrNot\lst@text\relax %
        \expandafter\lst@ifoneof\lst@text\relax{comment,end}%
            {\lstdrv@simula@commenttrue %
             \expandafter\lst@ifoneof\lst@text\relax{end}%
                 {\gdef\lstdrv@simula@closingcomment{a}}{}}%
            {}% empty else from 'ifoneof'
        \fi \fi \fi %
        \global\advance\lst@pos by -\lst@length %
        \let\lst@text\@empty \lst@length0 %
    \fi}%
%    \end{macrocode}
%    \begin{macrocode}
\gdef\lstdrv@simula@AppendOther#1{%
    \lstdrv@simula@ifcomment \if;#1%
        \lst@OutputOther \lstdrv@simula@commentfalse %
    \fi \fi %
    \advance\lst@length\@ne %
    \expandafter\lst@other\expandafter{\the\lst@other#1}}%
\endgroup %
%</simula>
%    \end{macrocode}
% \end{macro}\end{macro}\end{macro}\end{macro}
%
%
% \subsection{SQL}
% \begin{macro}{\lstdrv@sql@}
% Do you have corrections? Do you want more data base languages?
%    \begin{macrocode}
%<*sql>
\begingroup \makeatletter %
\catcode`\"=12 \catcode`\*=\active \catcode`\_=11 %
\gdef\lstdrv@sql@{%
    \keywords{absolute,action,add,allocate,alter,are,assertion,at,%
        between,bit,bit_length,both,cascade,cascaded,case,cast,catalog,%
        char_length,character_length,coalesce,collate,collation,column,%
        connect,connection,constraint,constraints,convert,%
        corresponding,cross,current_date,current_time,%
        current_timestamp,current_user,date,day,deallocate,deferrable,%
        defered,describe,descriptor,diagnostics,disconnect,domain,drop,%
        else,end,exec,except,exception,execute,external,extract,false,%
        first,full,get,global,hour,identity,immediate,initially,inner,%
        input,insensitive,intersect,interval,isolation,join,last,%
        leading,left,level,local,lower,match,minute,month,names,%
        national,natural,nchar,next,no,nullif,octet_length,only,outer,%
        output,overlaps,pad,partial,position,prepare,preserve,prior,%
        read,relative,restrict,revoke,right,rows,scroll,second,session,%
        session_user,size,space,sqlstate,substring,system_user,%
        temporary,then,time,timestamp,timezone_hour,timezone_minute,%
        trailing,transaction,translate,translation,trim,true,unknown,%
        upper,usage,using,value,varchar,varying,when,write,year,zone}%
    \sensitivefalse %
    \DeclareCommentLine\relax %
    \DeclareSingleComment /* */\relax %
    \stringizer{'"}\lstbaseem{0.6}%
    \lst@DefineCatcodes{\active}%
    \let\lst@PrePL \lst@PrePLDefault %
    \let\lst@PostPL\lst@PostPLDefault}%
\endgroup %
%</sql>
%    \end{macrocode}
% \end{macro}
%
%
% \subsection{\TeX}
% \begin{macro}{\lstdrv@tex@primitives}
% \begin{macro}{\lstdrv@tex@commoncs}
% \begin{macro}{\lstdrv@tex@latexcs}
% We define the different classes of control sequences. The second macro
% holds the common control sequences of plain-\TeX{} and \LaTeXe.
%    \begin{macrocode}
%<*tex>
\begingroup \makeatletter %
\gdef\lstdrv@tex@primitives{above,abovedisplayshortskip,%
    abovedisplayskip,abovewithdelims,accent,adjdemerits,advance,%
    afterassignment,aftergroup,atop,atopwithdelims,badness,%
    baselineskip,batchmode,begingroup,belowdisplayshortskip,%
    belowdisplayskip,binoppenalty,botmark,box,boxmaxdepth,%
    brokenpenalty,catcode,char,chardef,cleaders,closein,closeout,%
    clubpenalty,copy,count,countdef,cr,crcr,csname,day,deadcycles,def,%
    defaulthyphenchar,defaultskewchar,delcode,delimiter,%
    delimiterfactor,delimitershortfall,dimen,dimendef,discretionary,%
    displayindent,displaylimits,displaystyle,displaywidowpenalty,%
    displaywidth,divide,doublehyphendemerits,dp,else,emergencystretch,%
    end,endcsname,endgroup,endinput,endlinechar,eqno,errhelp,%
    errmessage,errorcontextlines,errorstopmode,escapechar,everycr,%
    everydisplay,everyhbox,everyjob,everymath,everypar,everyvbox,%
    exhyphenpenalty,expandafter,fam,fi,finalhypendemerits,firstmark,%
    floatingpenalty,font,fontdimen,fontname,futurelet,gdef,global,%
    globaldefs,halign,hangafter,hangindent,hbadness,hbox,hfil,hfill,%
    hfilneg,hfuzz,hoffset,holdinginserts,hrule,hsize,hskip,hss,ht,%
    hyphenation,hyphenchar,hyphenpenalty,if,ifcase,ifcat,ifdim,ifeof,%
    iffalse,ifhbox,ifhmode,ifinner,ifmmode,ifnum,ifodd,iftrue,ifvbox,%
    ifvmode,ifvoid,ifx,ignorespaces,immediate,indent,input,insert,%
    insertpenalties,interlinepenalty,jobname,kern,language,lastbox,%
    lastkern,lastpenalty,lastskip,lccode,leaders,left,lefthyphenmin,%
    leftskip,leqno,let,limits,linepenalty,lineskip,lineskiplimits,long,%
    looseness,lower,lowercase,mag,mark,,mathaccent,mathbin,mathchar,%
    mathchardef,mathchoice,mathclose,mathcode,mathinner,mathop,%
    mathopen,mathord,mathpunct,mathrel,mathsurround,maxdeadcycles,%
    maxdepth,meaning,medmuskip,message,mkern,month,moveleft,moveright,%
    mskip,multiply,muskip,muskipdef,newlinechar,noalign,noboundary,%
    noexpand,noindent,nolimits,nonscript,nonstopmode,%
    nulldelimiterspace,nullfont,number,omit,openin,openout,or,outer,%
    output,outputpenalty,over,overfullrule,overline,overwithdelims,%
    pagedepth,pagefilllstretch,pagefillstretch,pagefilstretch,pagegoal,%
    pageshrink,pagestretch,pagetotal,par,parfillskip,parindent,%
    parshape,parskip,patterns,pausing,penalty,postdisplaypenalty,%
    predisplaypenalty,predisplaysize,pretolerance,prevdepth,prevgraf,%
    radical,raise,read,relax,relpenalty,right,righthyphenmin,rightskip,%
    romannumeral,scriptfont,scriptscriptfont,scriptscriptstyle,%
    scriptspace,scriptstyle,scrollmode,setbox,setlanguage,sfcode,%
    shipout,show,showbox,showboxbreadth,showboxdepth,showlists,showthe,%
    skewchar,skip,skipdef,spacefactor,spaceskip,span,special,%
    splitbotmark,splitfirstmark,splitmaxdepth,splittopskip,string,%
    tabskip,textfont,textstyle,the,thickmuskip,thinmuskip,time,toks,%
    toksdef,tolerance,topmark,topskip,tracingcommands,tracinglostchars,%
    tracingmacros,tracingonline,tracingoutput,tracingpages,%
    tracingparagraphs,tracingrestores,tracingstats,uccode,uchyph,%
    underline,unhbox,unhcopy,unkern,unpenalty,unskip,unvbox,unvcopy,%
    uppercase,vadjust,valign,vbadness,vbox,vcenter,vfil,vfill,vfilneg,%
    vfuzz,voffset,vrule,vsize,vskip,vsplit,vss,vtop,wd,widowpenalty,%
    write,xdef,xleaders,xspaceskip,year}%
\gdef\lstdrv@tex@commoncs{active,acute,ae,AE,aleph,allocationnumber,%
    allowbreak,alpha,amalg,angle,approx,arccos,arcsin,arctan,arg,%
    arrowvert,Arrowvert,ast,asymp,b,backslash,bar,beta,bgroup,big,Big,%
    bigbreak,bigcap,bigcirc,bigcup,bigg,Bigg,biggl,Biggl,biggm,Biggm,%
    biggr,Biggr,bigl,Bigl,bigm,Bigm,bigodot,bigoplus,bigotimes,bigr,%
    Bigr,bigskip,bigskipamount,bigsqcup,bigtriangledown,bigtriangleup,%
    biguplus,bigvee,bigwedge,bmod,bordermatrix,bot,bowtie,brace,%
    braceld,bracelu,bracerd,braceru,bracevert,brack,break,breve,%
    buildrel,bullet,c,cap,cases,cdot,cdotp,cdots,centering,centerline,%
    check,chi,choose,circ,clubsuit,colon,cong,coprod,copyright,cos,%
    cosh,cot,coth,csc,cup,d,dag,dagger,dashv,ddag,ddagger,ddot,ddots,%
    deg,delta,Delta,det,diamond,diamondsuit,dim,displaylines,div,do,%
    dospecials,dot,doteq,dotfill,dots,downarrow,Downarrow,%
    downbracefill,egroup,eject,ell,empty,emptyset,endgraf,endline,%
    enskip,enspace,epsilon,equiv,eta,exists,exp,filbreak,flat,fmtname,%
    fmtversion,footins,footnote,footnoterule,forall,frenchspacing,%
    frown,gamma,Gamma,gcd,ge,geq,gets,gg,goodbreak,grave,H,hat,hbar,%
    heartsuit,hglue,hideskip,hidewidth,hom,hookleftarrow,%
    hookrightarrow,hphantom,hrulefill,i,ialign,iff,Im,imath,in,inf,%
    infty,int,interdisplaylinepenalty,interfootnotelinepenalty,intop,%
    iota,item,j,jmath,joinrel,jot,kappa,ker,l,L,lambda,Lambda,land,%
    langle,lbrace,lbrack,lceil,ldotp,ldots,le,leavevmode,leftarrow,%
    Leftarrow,leftarrowfill,leftharpoondown,leftharpoonup,leftline,%
    leftrightarrow,Leftrightarrow,leq,lfloor,lg,lgroup,lhook,lim,%
    liminf,limsup,line,ll,llap,lmoustache,ln,lnot,log,longleftarrow,%
    Longleftarrow,longleftrightarrow,Longleftrightarrow,longmapsto,%
    longrightarrow,Longrightarrow,loop,lor,lq,magstep,magstep,%
    magstephalf,mapsto,mapstochar,mathhexbox,mathpalette,mathstrut,%
    matrix,max,maxdimen,medbreak,medskip,medskipamount,mid,min,models,%
    mp,mu,multispan,nabla,narrower,natural,ne,nearrow,neg,negthinspace,%
    neq,newbox,newcount,newdimen,newfam,newif,newinsert,newlanguage,%
    newmuskip,newread,newskip,newtoks,newwrite,next,ni,nobreak,%
    nointerlineskip,nonfrenchspacing,normalbaselines,%
    normalbaselineskip,normallineskip,normallineskiplimit,not,notin,nu,%
    null,nwarrow,o,O,oalign,obeylines,obeyspaces,odot,oe,OE,%
    offinterlineskip,oint,ointop,omega,Omega,ominus,ooalign,openup,%
    oplus,oslash,otimes,overbrace,overleftarrow,overrightarrow,owns,P,%
    parallel,partial,perp,phantom,phi,Phi,pi,Pi,pm,pmatrix,pmod,Pr,%
    prec,preceq,prime,prod,propto,psi,Psi,qquad,quad,raggedbottom,%
    raggedright,rangle,rbrace,rbrack,rceil,Re,relbar,Relbar,%
    removelastskip,repeat,rfloor,rgroup,rho,rhook,rightarrow,%
    Rightarrow,rightarrowfill,rightharpoondown,rightharpoonup,%
    rightleftharpoons,rightline,rlap,rmoustache,root,rq,S,sb,searrow,%
    sec,setminus,sharp,showhyphens,sigma,Sigma,sim,simeq,sin,sinh,skew,%
    slash,smallbreak,smallint,smallskip,smallskipamount,smash,smile,sp,%
    space,spadesuit,sqcap,sqcup,sqrt,sqsubseteq,sqsupseteq,ss,star,%
    strut,strutbox,subset,subseteq,succ,succeq,sum,sup,supset,supseteq,%
    surd,swarrow,t,tan,tanh,tau,TeX,theta,Theta,thinspace,tilde,times,%
    to,top,tracingall,triangle,triangleleft,triangleright,u,underbar,%
    underbrace,uparrow,Uparrow,upbracefill,updownarrow,Updownarrow,%
    uplus,upsilon,Upsilon,v,varepsilon,varphi,varpi,varrho,varsigma,%
    vartheta,vdash,vdots,vec,vee,vert,Vert,vglue,vphantom,wedge,%
    widehat,widetilde,wlog,wp,wr,xi,Xi,zeta}%
\gdef\lstdrv@tex@latexcs{a,AA,aa,addcontentsline,addpenalty,%
    addtocontents,addtocounter,addtolength,addtoversion,addvspace,alph,%
    Alph,and,arabic,array,arraycolsep,arrayrulewidth,arraystretch,%
    author,baselinestretch,begin,bezier,bfseries,bibcite,bibdata,%
    bibitem,bibliography,bibliographystyle,bibstyle,boldmath,%
    botfigrule,bottomfraction,Box,caption,center,CheckCommand,circle,%
    citation,cite,cleardoublepage,clearpage,cline,columnsep,%
    columnseprule,columnwidth,contentsline,dashbox,date,dblfigrule,%
    dblfloatpagefraction,dblfloatsep,dbltextfloatsep,dbltopfraction,%
    defaultscriptratio,defaultscriptscriptratio,depth,Diamond,%
    displaymath,document,documentclass,documentstyle,doublerulesep,em,%
    emph,endarray,endcenter,enddisplaymath,enddocument,endenumerate,%
    endeqnarray,endequation,endflushleft,endflushright,enditemize,%
    endlist,endlrbox,endmath,endminipage,endpicture,endsloppypar,%
    endtabbing,endtabular,endtrivlist,endverbatim,enlargethispage,%
    ensuremath,enumerate,eqnarray,equation,evensidemargin,extracolsep,%
    fbox,fboxrule,fboxsep,filecontents,fill,floatpagefraction,floatsep,%
    flushbottom,flushleft,flushright,fnsymbol,fontencoding,fontfamily,%
    fontseries,fontshape,fontsize,fontsubfuzz,footnotemark,footnotesep,%
    footnotetext,footskip,frac,frame,framebox,fussy,glossary,%
    headheight,headsep,height,hline,hspace,I,include,includeonly,index,%
    inputlineno,intextsep,itemindent,itemize,itemsep,iterate,itshape,%
    Join,kill,label,labelsep,labelwidth,LaTeX,LaTeXe,leadsto,lefteqn,%
    leftmargin,leftmargini,leftmarginii,leftmarginiii,leftmarginiv,%
    leftmarginv,leftmarginvi,leftmark,lhd,linebreak,linespread,%
    linethickness,linewidth,list,listfiles,listfiles,listparindent,%
    lrbox,makeatletter,makeatother,makebox,makeglossary,makeindex,%
    makelabel,MakeLowercase,MakeUppercase,marginpar,marginparpush,%
    marginparsep,marginparwidth,markboth,markright,math,mathbf,%
    mathellipsis,mathgroup,mathit,mathsf,mathsterling,mathtt,%
    mathunderscore,mathversion,mbox,mdseries,mho,minipage,multicolumn,%
    multiput,NeedsTeXFormat,newcommand,newcounter,newenvironment,%
    newfont,newhelp,newlabel,newlength,newline,newmathalphabet,newpage,%
    newsavebox,newtheorem,nobreakspace,nobreakspace,nocite,nocorr,%
    nocorrlist,nofiles,nolinebreak,nonumber,nopagebreak,normalcolor,%
    normalfont,normalmarginpar,numberline,obeycr,oddsidemargin,%
    oldstylenums,onecolumn,oval,pagebreak,pagenumbering,pageref,%
    pagestyle,paperheight,paperwidth,paragraphmark,parbox,parsep,%
    partopsep,picture,poptabs,pounds,protect,pushtabs,put,qbezier,%
    qbeziermax,r,raggedleft,raisebox,ref,refstepcounter,renewcommand,%
    renewenvironment,restorecr,reversemarginpar,rhd,rightmargin,%
    rightmark,rmfamily,roman,Roman,rootbox,rule,samepage,sbox,scshape,%
    secdef,sectionmark,selectfont,setcounter,settodepth,settoheight,%
    settowidth,sffamily,shortstack,showoutput,showoverfull,sloppy,%
    sloppypar,slshape,sqsubset,sqsupset,SS,stackrel,stepcounter,stop,%
    stretch,subparagraphmark,subsectionmark,subsubsectionmark,%
    suppressfloats,symbol,tabbing,tabbingsep,tabcolsep,tabular,%
    tabularnewline,textasciicircum,textasciitilde,textbackslash,%
    textbar,textbf,textbraceleft,textbraceright,textbullet,textcircled,%
    textcompwordmark,textdagger,textdaggerdbl,textdollar,textellipsis,%
    textemdash,textendash,textexclamdown,textfloatsep,textfraction,%
    textgreater,textheight,textit,textless,textmd,textnormal,%
    textparagraph,textperiodcentered,textquestiondown,textquotedblleft,%
    textquotedblright,textquoteleft,textquoteright,textregistered,%
    textrm,textsc,textsection,textsf,textsl,textsterling,%
    textsuperscript,texttrademark,texttt,textunderscore,textup,%
    textvisiblespace,textwidth,thanks,thefootnote,thempfn,thempfn,%
    thempfootnote,thepage,thepage,thicklines,thinlines,thispagestyle,%
    title,today,topfigrule,topfraction,topmargin,topsep,totalheight,%
    tracingfonts,trivlist,ttfamily,twocolumn,typein,typeout,unboldmath,%
    unitlength,unlhd,unrhd,upshape,usebox,usecounter,usefont,%
    usepackage,value,vector,verb,verbatim,vline,vspace,width}%
%    \end{macrocode}
% \end{macro}\end{macro}\end{macro}
%
% \begin{macro}{\texcs}
% \begin{macro}{\moretexcs}
% The definitions are similar to \cs{keywords} and \cs{morekeywords}.
%    \begin{macrocode}
\gdef\texcs#1{\edef\lst@cs{,\zap@space#1 \@empty}}%
\gdef\moretexcs#1{\edef\lst@cs{\lst@cs,\zap@space#1 \@empty}}%
%    \end{macrocode}
% \end{macro}\end{macro}
%
% \begin{macro}{\lstdrv@tex@}
% Main driver and option macros:
%    \begin{macrocode}
\gdef\lstdrv@tex@{%
    \keywords{}%
    \texcs{\lstdrv@tex@primitives,\lstdrv@tex@commoncs,%
        advancepageno,beginsection,bf,bffam,bye,cal,cleartabs,columns,%
        dosupereject,endinsert,eqalign,eqalignno,fiverm,fivebf,fivei,%
        fivesy,folio,footline,hang,headline,it,itemitem,itfam,%
        leqalignno,magnification,makefootline,makeheadline,midinsert,%
        mit,mscount,nopagenumbers,normalbottom,of,oldstyle,pagebody,%
        pagecontents,pageinsert,pageno,plainoutput,preloaded,proclaim,%
        rm,settabs,sevenbf,seveni,sevensy,sevenrm,sl,slfam,supereject,%
        tabalign,tabs,tabsdone,tabsyet,tenbf,tenex,teni,tenit,tenrm,%
        tensl,tensy,tentt,textindent,topglue,topins,topinsert,tt,ttfam,%
        ttraggedright,vfootnote}%
    \sensitivetrue %
    \DeclareCLPercent %
    \DeclareSingleComment stuff \relax %
    \stringizer{}\lstbaseem{0.6}%
    \lst@DefineCatcodes{\active}%
    \let\lst@PrePL \relax %
    \let\lst@PostPL\relax}%
\gdef\lstdrv@tex@plain{\lstdrv@tex@}%
\gdef\lstdrv@tex@primitive{\lstdrv@tex@ \texcs{\lstdrv@tex@primitives}}%
\gdef\lstdrv@tex@latex{\lstdrv@tex@ %
    \texcs{\lstdrv@tex@primitives,\lstdrv@tex@latexcs}}%
\gdef\lstdrv@tex@allatex{\lstdrv@tex@ %
    \keywords{array,center,displaymath,document,enumerate,eqnarray,%
        equation,flushleft,flushright,itemize,list,lrbox,math,minipage,%
        picture,sloppypar,tabbing,tabular,trivlist,verbatim}%
    \texcs{\lstdrv@tex@primitives,\lstdrv@tex@latexcs,%
        AtBeginDocument,AtBeginDocument,AtBeginDvi,AtEndDocument,%
        AtEndOfClass,AtEndOfPackage,ClassError,ClassInfo,ClassWarning,%
        ClassWarningNoLine,CurrentOption,DeclareErrorFont,%
        DeclareFixedFont,DeclareFontEncoding,%
        DeclareFontEncodingDefaults,DeclareFontFamily,DeclareFontShape,%
        DeclareFontSubstitution,DeclareMathAccent,DeclareMathAlphabet,%
        DeclareMathAlphabet,DeclareMathDelimiter,DeclareMathRadical,%
        DeclareMathSizes,DeclareMathSymbol,DeclareMathVersion,%
        DeclareOldFontCommand,DeclareOption,DeclarePreloadSizes,%
        DeclareRobustCommand,DeclareSizeFunction,DeclareSymbolFont,%
        DeclareSymbolFontAlphabet,DeclareTextAccent,%
        DeclareTextAccentDefault,DeclareTextCommand,%
        DeclareTextCommandDefault,DeclareTextComposite,%
        DeclareTextCompositeCommand,DeclareTextFontCommand,%
        DeclareTextSymbol,DeclareTextSymbolDefault,ExecuteOptions,%
        GenericError,GenericInfo,GenericWarning,IfFileExists,%
        InputIfFileExists,LoadClass,LoadClassWithOptions,MessageBreak,%
        OptionNotUsed,PackageError,PackageInfo,PackageWarning,%
        PackageWarningNoLine,PassOptionsToClass,PassOptionsToPackage,%
        ProcessOptionsProvidesClass,ProvidesFile,ProvidesFile,%
        ProvidesPackage,ProvideTextCommand,RequirePackage,%
        RequirePackageWithOptions,SetMathAlphabet,SetSymbolFont,%
        TextSymbolUnavailable,UseTextAccent,UseTextSymbol}}%
\endgroup %
%    \end{macrocode}
% \end{macro}
%
% \begin{macro}{\lstdrv@tex@PrepareListing}
% Here we assign a different \verb!\lst@Output! macro to detect control
% sequences. We also (re-) set some catcodes.
%    \begin{macrocode}
\begingroup \makeatletter %
\gdef\lstdrv@tex@PrepareListing{\let\lst@Output\lstdrv@tex@Output %
    \catcode`\0=12 \catcode`\1=12 \catcode`\2=12 \catcode`\3=12 %
    \catcode`\4=12 \catcode`\5=12 \catcode`\6=12 \catcode`\7=12 %
    \catcode`\8=12 \catcode`\9=12}%
%    \end{macrocode}
% \end{macro}
%
% \begin{macro}{\lstdrv@tex@Output}
% \begin{macro}{\lstdrv@tex@CSOrNot}
% These macros are similar to \verb!\lst@Output! and
% \verb!\lst@CaseSensitiveKeywords!. We only do some adjustments, e.g.\ 
% we need no 'ifstring' test, since the stringizer is empty for \TeX{}.
%    \begin{macrocode}
\gdef\lstdrv@tex@Output{%
    \ifx\lst@text\@empty\else %
        \lst@ifcomment\lst@MakeBox{\lst@commentstyle}%
        \else\ifx\lst@lastother\lst@inputbackslash %
            \expandafter\lstdrv@tex@CSOrNot\lst@text\relax %
        \else\expandafter\lst@KeywordOrNot\lst@text\relax %
        \fi \fi %
        \global\advance\lst@pos by -\lst@length %
        \let\lst@text\@empty \lst@length0 %
    \fi}%
\gdef\lstdrv@tex@CSOrNot#1\relax{%
    \def\lst@test##1,#1,##2\relax{%
        \ifx \@empty##2\@empty \lst@MakeBox{}%
        \else \lst@MakeBox{\lst@keywordstyle}%
        \fi}%
    \expandafter\lst@test\lst@cs,#1,\relax}%
\endgroup
%</tex>
%    \end{macrocode}
% \end{macro}\end{macro}
%
%
% \begingroup\small
% \section{Extensions}\label{sExtensions}
% Define a new driver file for an additional language.
% Refer the previous section how this is done.
%
% If you want comment lines or comments, which are not supported by
% the declaration commands of this package, you have a problem.
% Contact me or follow these steps:
% \begin{enumerate}
% \item \DescribeMacro\lst@CommentLine
%	Define the macro, which cuts up the input line into a comment
%	line and the noncomment rest: The macro gets the input via the
%	macro \verb!\lst@line!. After the call \verb!\lst@line! must hold
%	the noncomment rest and \verb!\lst@commentline! the comment line.
%	The C++ line
% \begin{verbatim}
%    if (comments!=appear) // comment stuff\end{verbatim}
%	must be cut into '\verb|if (comments!=appear) |' and
%	'\verb!// comment stuff!'.
% \item \DescribeMacro\lst@SOC
%	Define the macro, which finds the start of a comment:
%	The macro gets the input via the macro \verb!\lst@line!. After
%	the call \verb!\lst@line! must contain the source line upto the
%	first appearance of a comment (exclusive). The comment and might
%	the rest of the line must be in the macro \verb!\lst@comment!.
%	If a comment is separated (not if and only if), you have to call
%       \verb!\lst@commenttrue!. The Pascal line
% \begin{verbatim}
%    if { comment } appears then\end{verbatim}
%	is cut into '\verb!if !' and '\verb!{ comment } appears then!'.
% \item \DescribeMacro\lst@EOC
%	Define the macro, which finds the end of a comment:
%	The macro gets the input via the macro \verb!\lst@line!. After
%	the call \verb!\lst@comment! must hold the comment and
%	\verb!\lst@line! the noncomment rest of the line (which might be
%	empty, of course). If the end of comment is found, you have to call
%	\verb!\lst@commentfalse!. The Pascal line
% \begin{verbatim}
%    { comment } appears then\end{verbatim}
%	must be cut into '\verb!{ comment }!' and '\verb! appears then!'.
% \end{enumerate}
% Assign these macros to \verb!\lst@CommentLine!, \verb!\lst@SOC! and
% \verb!\lst@EOC! within your language command.
%
%
% \section{History}\label{sHistory}
% Only major changes after version 0.15 are listed here.
% Previous changes are still present in the \texttt{.dtx}-file.
% \renewcommand\labelitemi{--}
% \begin{itemize}
% \iffalse
% \item[0.1] from 1996/03/09
%	\item test version to look whether package is possible or not
% \item[0.11] from 1996/08/19
%	\item additional blank option
%	\item	\cs{keywords}, \cs{morekeywords}, \cs{keywordstyle}
%		and \cs{commentstyle} are new commands
%	\item implementation guide improved and user's guide updated
%	\item alignment improved by rewriting some macros
% \item[0.12] from 1997/01/16
%	\item nearly perfect alignment now
%	\item \cs{stringizer}, \cs{stringstyle}, \cs{prelisting}
%		and \cs{postlisting} are new
%	\item user selection \cs{listingtrue} and \cs{listingfalse} possible
%	\item \cs{blankstringtrue} and \cs{blankstringfalse} handle output
%		of blanks in strings
%	\item package supports tabulators now; new command \cs{tablength}
% \item[0.13] from 1997/02/11
%	\item additional languages: Eiffel, Fortran 90, Modula-2, Pascal XSC
%	\item load on demand: language specific macros moved to driver files
%	\item comments are declared now and not implemented for each language
%		again (this makes the \TeX{} sources easier to read)
%	\item 'string exceeds line' test moved to
%		\verb!\lst@PreProcessLineDefault!
%	\item sample files moved to .dtx-file
% \item[0.14] from 1997/02/18
%	\item user's guide rewritten
%	\item implementation guide uses macro environment from the doc
%		package
%	\item (non) case sensitivity implemented, e.g.\ Pascal is not
%	\item multiple stringizer implemented, i.e.\ Modula-2 handles
%		both string types: quotes and double quotes
%	\item comment declaration is user-accessible now
%	\item package compatible to \verb!german.sty! now
%	\item changed some identifiers
% \item[0.15] from 1997/04/18
%	\item listing environment is new
%	\item additional languages: Java, Turbo Pascal
%   \item \verb!\lst@width! changes from 0.65em to 0.8em for Fortran 90
%   \item corrected some mistakes in the documentation
%	\item package renamed from listing.dtx to listings.dtx, since there
%		is already a listing package
% \fi
% \item[0.16] from 1997/06/01
%	\item Thanks to Anders Edenbrandt\footnote{Department of Computer
%		Science\ \ Lund University, Sweden.
%       Anders.Edenbrandt@dna.lth.se} for reporting two bugs:
%		lstmodula.sty corrected (misspelled \verb!\`"!) and
%		call of \verb!\lst@???style!s in \verb!\lst@Begin! avoid
%		loading of font files when catcodes are changed.
%	\item Thanks to Rolf Niepraschk\footnote{Physikalisch--Technische
%		Bundesanstalt\ \ Berlin, Germany.
%       Niepraschk@ptb.de} for reporting wrong catcode
%		of \verb!$!. I've also changed catcode of \verb!@! and took
%		over the proposal of using \verb!\zap@space!. The catcode of
%       the percent \verb!%! is restored after typesetting a listing now.
%	\item Thanks to Knut M\"uller\footnote{knut@physik3.gwdg.de} and
%       Stefan Meister\footnote{FH--Wolfenb\"uttel, Germany.}
%		for reporting problem with the command \cs{blankstringtrue}.
%		The problem is gone.
% \iffalse
%	\item changed '$<$' to '$>$' in \verb!\lst@SkipUptoFirst!
%   \item bug removed: \verb!\lst@Begin! must be placed before
%       \verb!\lst@SkipUptoFirst!
% \fi
%   \item new commands \cs{spreadlisting}, \cs{listoflistings},
%       \cs{labelstyle}, \cs{thelstline}, \cs{lstbaseem},
%       \cs{listlistingsname}
%   \item listing environment rewritten
% \item[0.17] from 1997/09/29
%	\item \cs{spreadlisting} works correct now (e.g.\ page numbers
%		move not right any more), new commands \cs{selectlisting}
%		and \cs{lstlineskip}, \cs{labelstyle} changed
%	\item speed up things (quick 'if parameter empty', all \cs{long}
%		except one removed, faster \verb!\lst@GotoNextTabStop!, etc.)
%	\item alignment of wide other characters improved (e.g.\ $==$)
%	\item many new languages: Ada, Algol, Cobol, Comal 80, Elan,
%		Fortran 77, Lisp, Logo, Matlab, Oberon, Perl, PL/I, Simula,
%		SQL, \TeX{}
% \end{itemize}
%
%
% \begin{thebibliography}{99}
% \bibitem{Ada}
%		\textsc{Barnes, John Gilbert Presslie}:
%		\textbf{Programming in Ada plus language reference manual}\\
%		{\copyright} 1991 Addison-Wesley Publishing Company, Inc.;
%		ISBN 0-201-56539-0
% \bibitem{Algol60}
%		\textsc{Uwe Pape}:
%		\textbf{Programmieren in ALGOL 60}\\
%		{\copyright} 1973 Carl Hanser Verlag M\"unchen;
%		ISBN 3-446-11605-2
% \bibitem{Algol68}
%		\textsc{Frank G.\ Pagan}:
%		\textbf{A practical guide to ALGOL 68}\\
%		{\copyright} 1976 by John Wiley $\&$ Sohn Ltd.;
%		ISBN 0-471-65746-8 (Cloth); ISBN 0-471-65747-6 (Pbk)
% \bibitem{Comal}
%		\textsc{Borge R. Christensen}:
%		\textbf{Strukturierte Programmierung mit COMAL 80} [aus dem
%		D\"anischen \"ubertragen und bearbeitet von Margarete Kragh]\\
%		2., verb.\ Auflage -- M\"unchen; Wien: Oldenburg, 1985;
%		ISBN 3-486-26902-X
% \bibitem{Eiffel}
%       \textsc{Bertrand Meyer}: \textbf{Eiffel: the language}\\
%       Prentice Hall International (UK) Ldt, 1992;
%       ISBN 0-13-247925-7
% \bibitem{Elan}
%		\textsc{Leo~H.~Klingen, Jochen Liedtke}:
%		\textbf{Programmieren mit ELAN}\\
%		B.G.\ Teubner, Stuttgart 1983; ISBN 3-519-02507-8
% \bibitem{Fortran77}
%		\textsc{Karl Hans M\"uller}:
%		\textbf{Fortran 77: Programmierungsanleitung}\\
%		3., v\"ollig neu bearb.\ Aufl.\ -- Mannheim; Wien; Z\"urich:
%		Bibliographisches Institut, 1984;
%		ISBN 3-411-05804-8
% \bibitem{Fortran90}
%       \textsc{Thomas Michel}: \textbf{Fortran 90: Lehr-- und Handbuch}\\
%       Mannheim; Leipzig; Wien; Z\"urich: BI-Wiss.-Verlag, 1994;
%       ISBN 3-411-16861-7
% \bibitem{Matlab} \texttt{http://www.utexas.edu/math/Matlab/Manual}
% \bibitem{Modula}
%       \textsc{Niklaus Wirth}: \textbf{Programmieren in Modula-2},
%       \"Ubers.\ Guido Pfeiffer\\
%       2.\ Auflage -- Berlin; Heidelberg; New York; London; Paris; Tokyo;
%               Hong Kong: Springer, 1991;
%       ISBN 3-540-51689-1
% \bibitem{java} \texttt{http://java.sun.com}
% \bibitem{lisp}
%		\textsc{Guy Steele}:
%		\textbf{Common Lisp}\\
%		Copyright 1990 by Digital Equipment Corporation;
%		ISBN 1-55558-042-4
% \bibitem{perl} \texttt{http://www.perl.com}
% \bibitem{pli}
%		\textsc{Bernhard Fischer, Herman Fischer}:
%		\textbf{Structured Programming in PL/I and PL/C}\\
%		Copyright {\copyright} 1976 by Marcel Dekker, Inc.;
%		ISBN 0-8247-6394-7
% \bibitem{simula}
%		\textsc{G\"unther Lamprecht}:
%		\textbf{Introduction to SIMULA 67}\\
%		Braunschweig; Wiesbaden: Vieweg, 1981
% \bibitem{sql}
%		\textsc{Jim~Melton, Alan~R.~Simon}:
%		\textbf{Understanding the new SQL: A Complete Guide}\\
%		{\copyright} 1993 Morgan Kaufmann Publishers, Inc.;
%		ISBN 1-55860-245-3
% \bibitem{verbatim}
%       \textsc{Rainer Sch\"opf, Bernd Raichle, Chris Rowley}:
%       \textbf{A New Implementation of \LaTeX's \texttt{verbatim}
%           and \texttt{verbatim*} Environments.}
% \end{thebibliography}
% \endgroup
%
%
% \setcounter{IndexColumns}{2}
% \PrintIndex
%
%
% \Finale
%
\endinput
\MakePercentIgnore^^A
%     \end{minipage}
%     \qquad
%     \begin{minipage}{0.45\linewidth}
%       \lstset{language={},style={},basicstyle=\ttfamily,baseem=0.51}^^A
%       \hbox to \linewidth{\lstbox\lstinputlisting{listings.tmp}\hss}
%     \end{minipage}
%     \end{center}}
%
%^^A
%^^A We redefine the appearance of part, section, ...
%^^A
%\def\@part[#1]#2{\addcontentsline{toc}{part}{#1}%
%    {\parindent\z@ \raggedright \interlinepenalty\@M
%     \normalfont \huge \bfseries #2\markboth{}{}\par}%
%    \nobreak\vskip 3ex\@afterheading}
%\renewcommand*\l@section[2]{%
%  \ifnum \c@tocdepth >\z@
%    \addpenalty\@secpenalty
%    \addvspace{.25em \@plus\p@}%
%    \setlength\@tempdima{1.5em}%
%    \begingroup
%      \parindent \z@ \rightskip \@pnumwidth
%      \parfillskip -\@pnumwidth
%      \leavevmode ^^A \bfseries
%      \advance\leftskip\@tempdima
%      \hskip -\leftskip
%      #1\nobreak\hfil \nobreak\hb@xt@\@pnumwidth{\hss #2}\par
%    \endgroup
%  \fi}
%\renewcommand*\l@subsection{\@dottedtocline{2}{1.5em}{2.3em}}
%\renewcommand*\l@subsubsection{\@dottedtocline{3}{3.8em}{3.2em}}
%\renewcommand\paragraph{\@startsection{paragraph}{4}{\z@}%
%                                    {1.25ex \@plus1ex \@minus.2ex}%
%                                    {-1em}%
%                                    {\normalfont\normalsize\bfseries}}
%
%^^A
%^^A Suppress warning
%^^A
% \let\lstenv@DroppedWarning\relax
% \makeatother
%
%^^A
%^^A The following definitions come in handy soon.
%^^A There will be more in future.
%^^A
% \newcommand\lsthelper[4]{#1\ifx\empty#2\empty\typeout{^^JWarning: #1 has unknown email^^J}\else\space\texttt{<#2>}{}\fi}
% \newcommand\lstthanks[2]{#1\ifx\empty#2\empty\typeout{^^JWarning: #1 has unknown email^^J}\fi}
% \newcommand\lst{\texttt{lst}}
% \lstdefinelanguage[doc]{Pascal}{^^A
%  keywords={alfa,and,array,begin,boolean,byte,case,char,const,div,^^A
%      do,downto,else,end,false,file,for,function,get,goto,if,in,^^A
%      integer,label,maxint,mod,new,not,of,or,pack,packed,page,program,^^A
%      procedure,put,read,readln,real,record,repeat,reset,rewrite,set,^^A
%      text,then,to,true,type,unpack,until,var,while,with,write,writeln},^^A
%  sensitive=false,^^A
%  doublecomment={(*}{*)}{\{}{\}},^^A
%  stringizer=[d]{'}}^^A
% \lstset{defaultdialect=[doc]Pascal}
% \lstset{language=Pascal}
%
%
%^^A
%^^A The long awaited beginning of documentation
%^^A
% \newbox\abstractbox
% \setbox\abstractbox=\vbox{
%	\begin{abstract}
%	Listings.dtx is a source code printer for \LaTeX\ --- which always means \LaTeXe\ in this documentation.
%	You can typeset stand alone files as well as enter listings using an environment similar to \texttt{verbatim} as well as you can print fragments using a command similar to |\verb|.
%^^A	Since listings.dtx is a package and not a cross compiler, all listings are up to date without maintaining differences between (changed) source files and cross compiled files.
%	The package supports a wide spectrum of programming languages --- some come already with \texttt{lstdrvrs.dtx}.
%	Finally: Many parameters control the output.
%	\end{abstract}}
%
% \title{{Listings.dtx} Version {0.19}}
% \author{Copyright 1996--1998 Carsten Heinz}
% \date{\box\abstractbox}
%
% \makeatletter\@twocolumntrue\makeatother
% \maketitle
% {\makeatletter\@starttoc{toc}}
%\iftrue
% \vfill
% \noindent \textbf{Again incompatibilities!}
% And there are certainly teething troubles since the complete kernel has been rewritten.
% So read this manual with care.
%\fi
% \onecolumn
%
%
% \part{User's guide}
%
%
% \section{Preface}
%
% The files \texttt{listings.dtx} and \texttt{listings.ins} and all files generated from only these two files are referred as 'the \textsf{listings} package' or simply 'the package'.
% A driver file is any file generated mainly from \texttt{lstdrvrs.dtx}.
%
% \paragraph{Alternatives.}
% The \textsf{listings} package is certainly not the final utility for typesetting source code listings.
% Other programs do their jobs very well if you are not satiesfied with \textsf{listings}.
% I should mention \textsf{a2ps} and the \textsf{LGrind} package.
% Let me know if you're using none of these solutions; I'd like to extend my suggestions.
%
% \paragraph{Reading this manual.}
% In any case you should read section ''\emph{\ref{uGettingStarted}. Getting started}'' step by step.
% You either start from the scratch there or you get in touch with the new user interface and other changes.
% That section is an introduction and doesn't cover all of the \textsf{listings} package, but the most common.
% After reading it you are prepared for the main reference if you need more details, more control or more features.
%
% \paragraph{Sorry!}
% On 1998/11/02 I decided to make major changes to the user interface, which were planned for version 0.2.
% But it's better to get the interface stable as soon as possible.
% In particular I must say sorry to all people I posted a pre-version of \textsf{listings} 0.19.
%
% \paragraph{Thanks.}
% There are many people I have to thank for fruitful communication, posting their ideas, giving error reports (first bug finder is listed), adding programming languages to \texttt{lstdrvrs.dtx}, etc..
% If you want to know that in detail, search for the names in the implementation part.
%^^A
%^^A Thanks for error reports (first bug finder only), new programming languages, etc.
%^^A Special thanks for communication which lead to kernel extensions.
%^^A 
% Special thanks go to (alphabetical order)
% \begin{quote}
% \hyphenpenalty=10000\relax \rightskip=0pt plus \linewidth\relax
%	\lstthanks{Andreas~Bartelt}{Andreas.Bartelt@Informatik.Uni-Oldenburg.DE},
%	\lstthanks{Jan~Braun}{Jan.Braun@tu-bs.de},
%	\lstthanks{Denis~Girou}{Denis.Girou@idris.fr},
%	\lstthanks{Arne~John~Glenstrup}{panic@diku.dk},
%	\lstthanks{Rolf~Niepraschk}{NIEPRASCHK@PTB.DE},
%	\lstthanks{Rui~Oliveira}{rco@di.uminho.pt} and
%	\lstthanks{Boris~Veytsman}{boris@plmsc.psu.edu}.
% \end{quote}
% Moreover I wish to thank
% \begin{quote}
% \hyphenpenalty=10000\relax \rightskip=0pt plus \linewidth\relax
%	\lstthanks{Bj{\o}rn~{\AA}dlandsvik}{bjorn@imr.no},
%	\lstthanks{Gaurav~Aggarwal}{gaurav@ics.uci.edu},
%	\lstthanks{Kai~Below}{below@tu-harburg.de},
%	\lstthanks{Detlev~Dr\"oge}{droege@informatik.uni-koblenz.de},
%	\lstthanks{Anders~Edenbrandt}{Anders.Edenbrandt@dna.lth.se},
%	\lstthanks{Harald~Harders}{h.harders@tu-bs.de},
%	\lstthanks{Christian~Haul}{haul@dvs1.informatik.tu-darmstadt.de},
%	\lstthanks{J\"urgen~Heim}{heim@astro.uni-tuebingen.de},
%	\lstthanks{Dr.~Jobst~Hoffmann}{HOFFMANN@rz.rwth-aachen.de},
%	\lstthanks{Knut~M\"uller}{knut@physik3.gwdg.de},
%	\lstthanks{Zvezdan~V.~Petkovic}{zpetkovic@acm.org},
%	\lstthanks{Michael~Piotrowski}{mxp@linguistik.uni-erlangen.de},
%	\lstthanks{Ralf~Quast}{rquast@hs.uni-hamburg.de},
%	\lstthanks{Aslak~Raanes}{araanes@ifi.ntnu.no},
%	\lstthanks{Detlef~Reimers}{dreimers@aol.com},
%	\lstthanks{Magne~Rudshaug}{magne@ife.no},
%	\lstthanks{Andreas~Stephan}{astepha@wmpi04.math.uni-wuppertal.de},
%	\lstthanks{Dominique~de~Waleffe}{ddw@miscrit.be},
%	\lstthanks{Herbert~Weinhandl}{weinhand@grz08u.unileoben.ac.at},
%	\lstthanks{J\"orn~Wilms}{wilms@rocinante.colorado.edu} and
%	\lstthanks{Kai~Wollenweber}{kai@ece.WPI.EDU}.
% \end{quote}
%
%^^A \lsthelper{Andreas Bartelt}{Andreas.Bartelt@Informatik.Uni-Oldenburg.DE}{1997/09/11}{single line comes out using \inputlisting inside \fbox}
%^^A \lsthelper{Michael Piotrowski}{mxp@linguistik.uni-erlangen.de}{1997/11/04}{! Use of \lstdrv@perl@TestCommentCut doesn't match its definition^^J\lst@line ->I^^JRF::LangID -- Statistical identification of language for IRF/1^^J^^J(POD ist Zeile '=irgendwas' bis Zeile '=cut')}
%^^A \lsthelper{Bj{\o}rn {\AA}dlandsvik}{bjorn@imr.no}{1997/10/27}{listings.sty is incompatible to inputenc.sty}
%
%
% \section{Software license}
%
% \paragraph{Copyright.}
%	The \textsf{listings} package is copyright 1996--1998 Carsten Heinz.
%	The driver files are copyright 1997, 1998 or 1997--1998 any individual author listed in these files.
%
% \paragraph{Distribution.}
%	The \textsf{listings} package as well as \texttt{lstdrvrs.dtx} and all driver files are distributed freely.
%	You are not allowed to take money for the distribution, except for a nominal charge for copying etc..
%
% \paragraph{Use of the package.}
%	The \textsf{listings} package is free for any non-commercial use.
%	Commercial use needs (i) explicit permission of the author of this package and (ii) the payment of a license fee.
%	This fee is to be determined in each instance by the commercial user and the package author and is to be payed as donation to the \LaTeX3 project.
%
% \paragraph{No warranty.}
%	The \textsf{listings} package as well as \texttt{lstdrvrs.dtx} and all driver files are distributed without any warranty, express or implied, as to merchantability or fitness for any particular purpose.
%
% \paragraph{Modification advice.}
%	Permission is granted to change the \textsf{listings} package as well as \texttt{lstdrvrs.dtx}.
%	You are not allowed to distribute any changed version of the package or any changed version of \texttt{lstdrvrs.dtx}, neither under the same name nor under a different one.
%	Tell the author of the package about your local changes: other users will welcome removed bugs, new features and additional programming languages.
%
% \paragraph{Contacts.}
%	Send comments and ideas on the package, error reports and additional programming languages to \emph{Carsten Heinz, Tellweg 6, 42275 Wuppertal, Germany} or preferably to \texttt{cheinz@gmx.de}.
%
% \paragraph{Trademarks}
%	appear throughout this documentation without any trademark symbol, so you can't assume that a name is free.
%	There is no intention of infringement; the usage is to the benefit of the trademark owner.
%
%
% \section{Getting started}\label{uGettingStarted}
% 
%
% \iffalse
% \subsection{Installation}
%
% \begin{enumerate}
% \item	You need \texttt{listings.dtx} and \texttt{listings.ins} to install the \textsf{listings} package.
%	\texttt{lstdrvrs.dtx} contains some language drivers.
%	Move \texttt{.dtx} and \texttt{.ins} files into a separate folder if you want to.
% \item	Remove all \texttt{lst???.sty} files if you update from an earlier version.
% \item	Run \texttt{listings.ins} through \TeX.
%	This creates the kernel \texttt{listings.sty} and two interface files \texttt{lst017.sty} and \texttt{lstfvrb.sty}.
%	If you have \texttt{lstdrvrs.dtx}, a configuration file \texttt{listings.cfg} and a couple of driver files are also generated.
% \item	Copy all created \texttt{.sty} and \texttt{.cfg} files to a directory searched by \TeX.
% \end{enumerate}
% \fi
%
%
% \subsection{Package loading}
%
% As usual in \LaTeX\ the package is loaded by |\usepackage[|\meta{options}|]{listings}|, where |[|\meta{options}|]| is optional.
% Note that \textsf{listings} 0.19 needs more save stack size than any previous version.
% The following options are available:
% \begin{description}
% \item[\texttt{0.17}]
%	Use this option to compile documents created with version 0.17 of the \textsf{listings} package.
%	Note that you can't use old driver files and that the option does not guarantee full compatibility.
%	If you want to polish up an old document, use this option together with new commands and keys.
%	There shouldn't be many problems, but I haven't made deep tests.
% \item[\normalfont\texttt{index} and \texttt{procnames}] define the 'index' respectively 'procnames' aspect, whatever that means.
% \item[\normalfont\texttt{ndkeywords} and \texttt{rdkeywords}] define second respectively third order keywords.
% \end{description}
% Note: Each option slows down the package, so it becomes quite slow if you use all options, for example.
% Note also that you can't activate any optional feature after package loading --- except you're a hacker and read the implementation part.
%
% Here is some kind of minimal file.
% \begin{verbatim}
%    \documentclass{article}
%
%    \usepackage{listings}
%    %\usepackage[ndkeywords,index]{listings}% with some options
%
%    \begin{document}
%
%    \lstset{language=Pascal}
%    % Any example may be inserted here.
%
%    \end{document}\end{verbatim}
% \label{uMinimalFile}It loads the \textsf{listings} package without any option, but shows how the package is loaded with index option and second order keywords.
% Moreover we select Pascal as programming language.
%
% If you need more than one programming language in your document, I recommend the preamble command |\lstloadlanguages|, which loads the given programming languages only, e.g.\ |\lstloadlanguages{Pascal,Fortran,C++}|.
% You find more details in section \ref{uLanguagesAndStyles}.
%
%
% \subsection{Typesetting listings}
%
% You can print (a) stand alone files, (b) source code from \texttt{.tex} files or (c) code fragments within a paragraph.
% The difference between (b) and (c) is something like between display and text formulas (|$$| and |$|).
%
% We begin with code fragments.
% If you want '\lstinline!var i:integer;!', you can simply write '|\lstinline!var i:integer;!|'.
% The exclamation marks delimit the code fragment.
% They could be replaced by any character not in the code fragment, i.e.\ '|\lstinline$var i:integer;$|' would produce the same output.
%
% The \texttt{lstlisting} environment typesets the source code in between.
% It has one parameter, which we leave empty for the moment.
% \begin{lstsample}{}
%\begin{lstlisting}{}
%for i:=maxint to 0 do
%begin
%    { do nothing }
%end;
%
%Write('Keywords are case ');
%WritE('insensitive here.');
%\end{lstlisting}
% \end{lstsample}
% \noindent
% Like most examples this shows the \LaTeX\ source code on the right and the result on the left.
% If you insert the \LaTeX\ code in the minimal file and run it through \TeX, the listing uses the whole text width.
%
% Finally examples for stand alone files:
% \begin{lstsample}{\lstset{baseem=0.5}}
%\lstinputlisting{listings.tmp}
% \end{lstsample}
% \noindent
% Do you wonder about the left hand side?
% Well, the file \texttt{listings.tmp} contains the current example.
% This is exactly the line you see on the right.
% If you want this line to be typeset in Pascal mode, you get what you've got.
% If you want the lines $3,4,\ldots,10$ of \texttt{testfile.pas}, write
% \begin{verbatim}
%    \lstinputlisting[first=3,last=10]{testfile.pas}\end{verbatim}
% But be sure that the first line does exist, or you will get a ''runaway argument'' error.
%
% Problems could arise if you (try to) put a listing in a box, tabular environment or something similar.
% The |\lstbox| command in section \ref{uTypesettingListings} might solve this.
% If you use an extended character table in your listings, you must use the |extendedchars| key, see section \ref{uSomeSpecialKeys}.
%
%
% \subsection{The ''key=value'' interface}
%
% The \textsf{listings} package uses \textsf{keyval.sty} from the \textsf{graphics} bundle by David Carlisle.
% Each 'aspect' or 'parameter' you can control has an associated key.
% To select a programming language, for example, you need the key |language|, the sign |=| and the name of the language as value.
% The command |\lstset| gets this construction as argument.
% You have seen this in the minimal file on page \pageref{uMinimalFile}.
% You can set more than one parameter with a single |\lstset| if you separate two or more ''key=value'' constructions by commas:
% \begin{verbatim}
%    \lstset{language=Pascal,keywordstyle=\bfseries}\end{verbatim}
% If the value itself contains a comma, you must enclose the value in braces:
% \begin{verbatim}
%    \lstset{keywords={one,two,three}}\end{verbatim}
% |\lstset{keywords=one,two,three}| would set the one and only keyword |one| and gives an error message since the parameters |two| and |three| do not exist.
%
% Two final notes:
% (1) Some keys has default values, e.g.\ |flexiblecolumns=true| turns flexible columns on, but |flexiblecolumns| without any |=true| would do the same since the value true is default for that key.
% But you must use the value |false| if you want to turn them off.
% (2) |\lstset| sets the values local to the current group.
% Just forget it if you don't know what that means.
% The command |\lstinputlisting| (as seen above) and the environment (as you see below) both have optional arguments.
% If you use a key=value list as optional argument, these selections are valid for the particular listing only and the previous values are restored afterwards.
% For example, if the current language is Pascal, but you want \texttt{testfile.f95} from line 3 on, write
% \begin{verbatim}
%    \lstinputlisting[first=3,language=Fortran]{testfile.f95}\end{verbatim}
% Afterwards Pascal is still the current language.
% Note that |\lstinline| has no optional parameter.
%
%
% \subsection{Figure out the appearance}
%
% Keywords are typeset bold, comments in italic shape and spaces in strings appear as '\textvisiblespace' (without the two quotes).
% You can change that default behaviour:
% \begin{verbatim}
%\lstset{
%    basicstyle=\small,              % print whole listing small
%    keywordstyle=\bfseries\underbar,% 'underlined' bold keywords
%    nonkeywordstyle={},             % nothing happens to other identifiers
%    commentstyle=\itshape,          % default
%    stringstyle=\ttfamily,          % typewriter font for strings
%    blankstring=true}               % blank spaces are blank in strings\end{verbatim}
% We typeset a previous example with these styles again.
%\lstset{basicstyle=\small,
%    keywordstyle=\bfseries\underbar,nonkeywordstyle={},
%    commentstyle=\itshape,stringstyle=\ttfamily,blankstring=true}
% \begin{lstsample}{}
%\begin{lstlisting}{}
%for i:=maxint to 0 do
%begin
%    { do nothing }
%end;
%
%Write('Keywords are case ');
%WritE('insensitive here.');
%\end{lstlisting}
% \end{lstsample}
% \noindent
% The style definitions above use two kind of commands; on the one hand |\ttfamily| or |\bfseries| taking no arguments and on the other |\underline|, which gets exactly one argument.
% The \emph{very last} token of |keywordstyle|, |nonkeywordstyle| and |labelstyle| (see below) \emph{may} be a macro getting exactly one argument, namely the (non)keyword or label.
% All other tokens \emph{must not} take any arguments --- or you will get deep in trouble.
% \lstset{style={},blankstring=false}
%
% \textbf{Warning:}
% You shouldn't use striking styles too often, but 'too often' depends on how many keywords the source code contains, for example.
% Your eyes would concentrate on the framed, bold red printed keywords only, and your brain must compensate this.
% Reading such source code could be very exhausting.
% If it were longer, the last example would be quite good in this sense.
% Believe me.
%
%
% \subsection{Line numbers}
%
% You want tiny line numbers each second line?
% Here you are:
% \begin{lstsample}{}
%\lstset{labelstyle=\tiny,%   <===
%        labelstep=2}%        <===
%\begin{lstlisting}{}
%for i:=maxint to 0 do
%begin
%    { do nothing }
%end;
%
%Write('Keywords are case ');
%WritE('insensitive here.');
%\end{lstlisting}
% \end{lstsample}
% \noindent
% |labelstep=0| turns line numbering off.
% In the sequel we use |labelstyle=\tiny| and |labelstep=2|, even if it doesn't appear in the verbatim part.
% And now we try to interrupt (continue) a listing.
% For this purpose we use |{ }| as argument to the environment.
% \begin{lstsample}{\lstset{labelstyle=\tiny,labelstep=2}}
%\begin{lstlisting}{}
%for i:=maxint to 0 do
%begin
%    { do nothing }
%end;
%
%\end{lstlisting}
%
%And we continue the listing:
%
%\begin{lstlisting}{ }%       <===
%Write('Keywords are case ');
%WritE('insensitive here.');
%\end{lstlisting}
% \end{lstsample}
% \noindent
% Note that the empty line at the end of the first part is not printed, but it is responsible for correct line numbering.
% |{ }| continued the previous '|{}|'-listing.
% In general the argument is the name of the listing.
% An empty (= |{}|) named listing always starts with line number one, no matter whether line numbers are printed or not.
% A space (= |{ }|) named listing continues the last empty or space named one.
% That's easy.
% And this mechanism becomes easier if you use real names for your listings.
% In that case all listings with the same name use a common line counter: the second (same named) listing continues automatically the first.
% Even if there are other listings in between.
% \begin{lstsample}{\lstset{labelstyle=\tiny,labelstep=2}}
%\begin{lstlisting}{Test}%    <===
%for i:=maxint to 0 do
%begin
%    { do nothing }
%end;
%
%\end{lstlisting}
%
%And we continue the listing:
%
%\begin{lstlisting}{Test}%    <===
%Write('Keywords are case ');
%WritE('insensitive here.');
%\end{lstlisting}
% \end{lstsample}
% \noindent
% The next |Test| listing goes on with line number {\makeatletter\lstno@Test}.
% Note that listing names are case sensitive.
%
%
% \subsection{Tabulators and form feeds}
%
% You might get some unexpected output if your source code contains a tabulator or form feed.
% The package assumes tabulator stops at columns 9,17,25,33,\ldots, and a form feed prints an empty line.
% This is predefined via
% \begin{verbatim}
%    \lstset{tabsize=8,formfeed=\bigbreak}\end{verbatim}
% If you change the eight to the number $n$, you will get tabulator stops at columns $n+1,\allowbreak 2n+1,\allowbreak 3n+1,$ and so on.
% If you want a new page every form feed, use |formfeed=\newpage|.
% \lstset{tabsize=4}
% \begin{lstsample}{}
%\lstset{tabsize=4}
%\begin{lstlisting}{}
%123456789
%	{ one tabulator }
%		{ two tabs }
%123		{ 123 and two tabs }
%\end{lstlisting}
% \end{lstsample}
% \lstset{tabsize=8}\noindent
% Unfortunately both sides are typeset with |tabsize=4|.
%
%
% \subsection{Indent the listing}
%
% The examples are typeset with centered \texttt{minipage}s.
% That's the reason why you can't see that line numbers are printed in the margin (default).
% Now we separate the 'minipage margin' and the minipage by a vertical rule:
% \begin{lstsample}{\lstset{frame=l,frametextsep=0pt,labelstyle=\tiny,labelstep=2}}
%Some text before
%\begin{lstlisting}{}
%for i:=maxint to 0 do
%begin
%    { do nothing }
%end;
%\end{lstlisting}
% \end{lstsample}
% \noindent
% The listing is lined up with the normal text.
% You can change this if you want.
% The parameter |indent| 'moves' the listing to the right (or left if the skip is negative).
% \begin{lstsample}{\lstset{frame=l,frametextsep=0pt,labelstyle=\tiny,labelstep=2}}
%Some text before
%\lstset{indent=2em}%         <===
%\begin{lstlisting}{}
%for i:=maxint to 0 do
%begin
%    { do nothing }
%end;
%\end{lstlisting}
%
%\begin{lstlisting}{ }
%Write('Keywords are case ');
%WritE('insensitive here.');
%\end{lstlisting}
% \end{lstsample}
% \noindent
% Note that |\lstset{indent=2em}| also changes the indent for the second listings.
% If you want to indent a single listing only, use the argument of |\lstset| as optional argument to the listing environment:
% \begin{lstsample}{\lstset{frame=l,frametextsep=0pt,labelstyle=\tiny,labelstep=2}}
%\begin{lstlisting}[indent=2em]{}
%for i:=maxint to 0 do
%begin
%    { do nothing }
%end;
%\end{lstlisting}
%
%\begin{lstlisting}{ }
%Write('Keywords are case ');
%WritE('insensitive here.');
%\end{lstlisting}
% \end{lstsample}
% \noindent
% Such local changes apply to all supported parameters.
%
% If you use environments like \texttt{itemize} or \texttt{enumerate}, there is 'natural' indention coming from these environments.
% By default the \textsf{listings} package respects this.
% But you might use |wholeline=true| (or |false|) to make your own decision.
% \begin{lstsample}{\lstset{frame=l,frametextsep=0pt,labelstyle=\tiny,labelstep=2}}
%\begin{itemize}
%\item First item:
%
%\begin{lstlisting}{}
%for i:=maxint to 0 do
%begin
%    { do nothing }
%end;
%\end{lstlisting}
%
%\item Second item:
%\lstset{wholeline=true}%     <===
%\begin{lstlisting}{ }
%Write('Keywords are case ');
%WritE('insensitive here.');
%\end{lstlisting}
%\end{itemize}
% \end{lstsample}
% \noindent
% You can use |wholeline| together with |indent|, of course.
% Refer section \ref{uListingAlignment} for a description of |spread|.
%
%
% \subsection{Fixed and flexible columns}
%
% The first thing a reader notices is --- except different styles for keywords etc. --- the column alignment of a listing.
% The problem: I don't like listings in typewriter fonts and other fonts need not to have a fixed width.
% But we don't want
% \begin{itemize}\item[]\begin{tabular}{@{}l}
%	if\ x=y\ then\ write('alignment')\\
%	\ \ \ \ \ \ \ else\ print('alignment');
% \end{tabular}\end{itemize}
% only because spaces are not wide enough.
% There is a simple trick to avoid such things.
% We make boxes of the same width and put one character in each box:
% \def\docharfbox#1#2{\fbox{\hbox to#1em{\hss\vphantom{fy}#2\hss}}}
% \def\docharbox#1#2{\hbox to#1em{\hss#2\hss}}
% \begin{itemize}\item[]
%     \def\makeboxes#1{\docharfbox1#1\ifx#1\relax\else\expandafter\makeboxes\fi}
%     \makeboxes if\ x=y\ then\ write\relax\space\ldots\\
%     \makeboxes \ \ \ \ \ \ \ else \ print\relax\space\ldots
% \end{itemize}
% Going this way the alignment of columns can't be disturbed.
%	\def\makeboxes#1{\docharbox{0.45}#1\ifx#1\relax\else\expandafter\makeboxes\fi}
% But if the boxes are not wide enough, we get '\makeboxes if\ x=y\ then\relax\ldots', and choosing the width so that the widest character fits in leads to
%	\def\makeboxes#1{\docharbox{1}#1\ifx#1\relax\else\expandafter\makeboxes\fi}
% '\makeboxes if\ x=y\ then\ write\relax\ldots'.
% Both are not acceptable.
% Since all input will be cut up in units, we can put each unit in a box, which width is multiplied by the number of characters we put in, of course.
% The result is
%	\def\makeboxes#1#2{\docharfbox{#1}{#2}\ifx#2\relax\else\expandafter\makeboxes\fi}
%	\makeboxes{1.4}{i\hss f}{.7}{\ }{.7}{x}{.7}{=}{.7}{y}{.7}{\ }{2.8}{t\hss h\hss e\hss n}{.7}{\ }{3.5}{w\hss r\hss i\hss t\hss e}{.7}\relax\space.
% Since we put wide and thin characters in the same box, the width of a single character box need not to be the width of the widest character.
% The empirical value {\makeatletter\lst@baseemfixed}em (which is called 'base em' later) is a compromise between overlapping characters and the number of boxes not exceeding the text width, i.e.\ how many characters fit a line without getting an overfull |\hbox|.
%
% \begingroup
% But overlapping characters are a problem if you use many upper case letters, e.g.\ \docharbox{3}{W\hss O\hss M\hss E\hss N} --- blame me and not the women, in fact \docharbox{1.8}{M\hss E\hss N} doesn't look better.
% To go around this problem the \textsf{listings} package supports more 'flexible columns' in contrast to the fixed columns above.
% Arne John Glenstrup (whose idea the format is) pointed out that he had good experience with flexible columns and assembler listings.
% The differences can be summed up as follows: The fixed column format ruins the nice spacing intended by the font designer, and the flexible format ruins the column alignment (possibly) intended by the programmer.
% We illustrate that:
%	\lstset{style={},language={}}
%	\def\sample{\begin{lstlisting}{}^^J\ WOMEN\ \ are^^J \ \ \ \ \ \ \ MEN^^J WOMEN are^^J better MEN^^J \end{lstlisting}}
%	\begin{center}\begin{tabular}{c@{\qquad\quad}c@{\qquad\quad}c}
%	verbatim&fixed columns&flexible columns\\
%		&with {\makeatletter\lst@baseemfixed}em&with {\makeatletter\lst@baseemflexible}em\\ \noalign{\medskip}
%	\setkeys{lst}{basicstyle=\ttfamily,baseem=0.51}\lstbox\sample&\lstset{flexiblecolumns=false}\lstbox\sample&\lstset{flexiblecolumns=true}\lstbox\sample\lstset{flexiblecolumns=false}
%	\end{tabular}\end{center}
% Hope this helps.
% Note the varying numbers of spaces between 'WOMEN' and 'are' and look at the different outputs.
% The flexible column format typesets all characters at their natural width.
% In particular characters never overlap.
% If a word needs more space than reserved ('WOMEN'), the rest of the line moves to the right.
% Sometimes a following word needs less space than reserved, or there are spaces following each other.
% Such 'surplus' space is used to fix the column alignment:
% The blank space in the third line have been printed properly, but the two blanks in the first line have been printed as one blank space.
% We can show all this more drastic if we reduce the width of a single character box:
%	\begin{center}\lstset{baseem={0.3,0.0}}\begin{tabular}{c@{\qquad\quad}c@{\qquad\quad}c}
%		&{\makeatletter\lst@baseemfixed}em&{\makeatletter\lst@baseemflexible}em\\ \noalign{\medskip}
%	\setkeys{lst}{basicstyle=\ttfamily,baseem=0.51}\lstbox\sample&\lstset{flexiblecolumns=false}\lstbox\sample&\lstset{flexiblecolumns=true}\lstbox\sample\lstset{flexiblecolumns=false}
%	\end{tabular}\end{center}
% In flexible column mode the first 'MEN' moves to the left since the blanks before are $7\cdot 0.0$em$=0$em wide.
% Even in flexible mode you shouldn't reduce 'base em' to less than 0.33333em ($\approx$ width of a single space).
% \endgroup
%
% You want to know how to select the flexible column format and how to reset to fixed columns?
% Try |flexiblecolumns=true| and |flexiblecolumns=false|.
%
%
% \subsection{Selecting other languages}\label{uSelectingOtherLanguages}
%
% You already know that |language=|\meta{language name} selects programming languages --- at least Pascal and Fortran.
% But that's not the whole truth.
% Both languages know different dialects (= version or implementation), for example Fortran 77 and Fortran 90.
% You can choose such special versions with the optional argument of |language|.
% Write
% \begin{verbatim}
%    \lstset{language=[77]Fortran}% Fortran 77
%    \lstset{language=[XSC]Pascal}% Pascal XSC\end{verbatim}
% to select Fortran 77 and Pascal XSC, respectively.
%
% We give a list of all languages and dialects supported by \texttt{lstdrvrs.dtx}.
% Use the given names as (optional) values to |language|.
% An 'empty' language is also defined: |\lstset{language={}}| detects no keywords, no comments, no strings, \ldots
% \begin{center}
% \begin{tabular}{ll}
% Ada & Lisp \\
% Algol (|68|,|60|) & Logo \\
% C (|ANSI|,|Objective|)& Matlab \\
% Cobol (|1985|,|1974|,|ibm|) & Mercury \\
% Comal 80 & Modula-2 \\
% C++ (|ANSI|,|Visual|) & Oberon-2 \\
% csh & Pascal (|Standard|,|XSC|,|Borland6|) \\
% Delphi & Perl \\
% Eiffel & PL/I \\
% Elan & Prolog \\
% Euphoria & Simula (|67|,|CII|,|DEC|,|IBM|) \\
% Fortran (|95|,|90|,|77|) & SQL \\
% IDL & TeX (|plain|,|primitive|,|LaTeX|,|alLaTeX|) \\
% HTML & VHDL \\
% Java
% \end{tabular}
% \end{center}
% The configuration file \texttt{listings.cfg} defines each first dialect as default dialect, i.e.\ |\lstset{language=C}| selects ANSI C.
% After
% \begin{verbatim}
%    \lstset{defaultdialect=[Objective]C}\end{verbatim}
% Objective-C becomes default dialect for C, but the language is \emph{not} selected with that command.
%
% \medskip
% Remark: The driver files define the languages with |\lstdefinelanguage|.
% The languages have all 'bugs' coming from the language commands, e.g.\ in Ada and Matlab it is still possible that the package assumes a string where none exists.
% See \texttt{lstdrvrs.dtx} for more remarks on special programming languages.
%
%
% \section{Main reference}
%
% \newcommand\UTODO[1]{}
% \newenvironment{TODO}{\begin{quote}\footnotesize To do:}{\end{quote}}
% \newenvironment{macrosyntax}
%	{\list{}{\itemindent-\leftmargin\labelsep=0pt\relax %
%		\def\makelabel##1{\lstmklab##1,\relax}}}
%	{\endlist}
% \def\lstmklab#1,#2\relax{%
%	\ifx\empty#2\empty %
%		\lstmklabb#1,,\relax %
%	\else
%		\lstmklabb#1,#2\relax %
%	\fi}
% \def\lstmklabb#1,#2,\relax{%
%	\llap{\scriptsize\itshape#2}%
%	\rlap{\hskip-\itemindent\hskip\labelwidth\hskip\linewidth#1}}
%
% In this section we list all user environments, commands and keys together with their parameters and possible values.
% Sometimes there are examples or more detailed information at the end of the subsections.
% All user macros have the prefix \lst.
% This avoids naming conflicts with other packages or document classes.
%
% In the sequel the label \emph{changed} indicates any change in behaviour or syntax of a command, environment or key, so read these items carefully.
% Keys strongly related to special programming languages have the label \emph{lang.spec.}.
% Usually they are available only if you have loaded the corresponding programming language.
% All other labels should be clear.
% The numbers in the right margin give the version number of introduction (either as internal or as user macro/key).
%
% Note that commands, environments and keys can be distinguished easily:
% A command begins with a backslash and a key is presented in ''|key=|\meta{value}'' fashion.
% All the rest are environments, which is in fact a single one.
%
%
% \subsection{Typesetting listings}\label{uTypesettingListings}
%
% \begin{macrosyntax}
% \item[0.1,changed] |\lstinputlisting[|\meta{key=value list}|]{|\meta{file name}|}|
%
%		typesets the source code file using the active language and style (locally modified by the key=value list).
%		Empty lines at the end of listing are dropped.
%
%		The keys |first| and |last| determines the printed line range.
%		Default is \emph{always} 1--9999999, where 'always' means that you can't set |first| and |last| globally.
%		If you don't want a whole file, you must specify the line range.
%		The example |\lstinputlisting[first=3,last=10]{testfile.pas}| would print lines 3,4,\ldots,10 of \texttt{testfile.pas} if file and lines are present.
%
%		The boolean key |print| controls whether or not |\lstinputlisting| typesets a listing:
%		|print=false| speed up things for draft versions of your document and |print=true| typesets the stand alone files.
%
%		Note: The keys |first|, |last| and |print| only apply to |\lstinputlisting|.
%^^A
%^^A TODO: That's not true.
%^^A       The key |first| also applies to lstlisting.
%^^A       It sets the line number of the first line.
%^^A
%
% \item[0.15,changed] |lstlisting[|\meta{key=value list}|]{|\meta{name}|}|
%
%		typesets source code between |\begin{lstlisting}{|\meta{name}|}| (+ line break) and |\end{lstlisting}|, but empty lines at the end of listing are dropped.
%		Characters on the same line as the end of environment are \emph{not} dropped, i.e.\ source code right before and \LaTeX\ code after the end of environment is typeset respectively executed.
%		If \meta{name} is not empty and the listing is not continued, the name appears in the list of listings.\\
%		Same named listings have common line counter, i.e.\ the second (same named) listing continues the first, the third continues the second, and so on.
%		There are two exceptions: An empty named listing always starts with line number 1 and is continued with listings named |{ }| (single blank space).
%
%		The key |resetlineno| (given without any value) sets the line number of the first line to 1.
%		You can also specify a number, e.g.\ |resetlineno=10|.
%		|advancelineno=|\meta{number} advances the number of the first line by \meta{number}.
%		These two keys take also effect on |{}| and |{ }| named listings.
%
%		Note: The keys |resetlineno| and |advancelineno| only apply to the environment and must be used in the optional key=value list.
%
% \item[0.18,new] |\lstinline|
%
%		works like |\verb|, but uses the active language and style, of course.
%		You can write '|\lstinline!var i:integer;!|' and get '\lstinline!var i:integer;!'.
%		Note that these listings always use flexible columns and that |\lstinline| has no optional key=value list.
%\UTODO{implement optional key=value list}
%
% \item[0.18,new] |\lstbox[|\meta{b$\vert$c$\vert$t}|]|
%
%		is used right before |\lstinputlisting| or a listing environment.
%		Without |\lstbox| the listing uses  a new paragraph and the complete line width, even if the listing is one character wide.
%		|\lstbox| returns the listing as a |\hbox|, which has the necessary width only (line numbers possibly not included).
%		Use this command within explicit |\hbox|es, |$$|\ldots|$$|, etc..
%		The optional argument gives the vertical position of the listing: bottom, center (default) or top.
% \end{macrosyntax}
%
%
% \subsection{List of listings}
%
% \begin{macrosyntax}
% \item[0.16] |\listoflistings|
%
%		prints a list of listings.
%		The names are the file names or names of the listings.
%
% \item[0.16] |\listlistingsname|
%
%		contains the header name, which is 'Listings' by default.
%		You may adjust it via |\renewcommand\listlistingsname{whatever}|.
%
% \item[0.19,new] |\lstname|
%
%		contains the name of the current (last) listing in \emph{printable} form.
%		After |pre=\subsubsection{\lstname}| each listing begins a new subsubsection, which name is the listing name.
%		The key |pre| is described in section \ref{uSomeSpecialKeys}.
%		You can also use it to index all listings, but then you must expand the content first.
%		Simply write |\expandafter\index\expandafter{\lstname}| (as value to |pre| if you want).
%
% \item[0.19,new] |\lstintname|
%
%		contains the name of the current (last) listing possibly in nonprintable form.
%		Use this macro inside |\label| and |\ref| for example.
%		But be aware that a reference counter must be defined --- or the label is the current section, subsection, and so on.
%		You might use it like this:
% \begin{verbatim}
%    \lstset{pre=\caption{\lstname}\label{lst\lstintname}}
%
%    \begin{figure}
%    \begin{lstlisting}{Empty}
%    \end{lstlisting}
%    \end{figure}
%
%    Figure \ref{lstEmpty} shows an empty listing.\end{verbatim}
%		Here \emph{all} listings get a caption and a label (unless |pre| is redefined).
% \end{macrosyntax}
%
%
% \subsection{Basic style keys}\label{uBasicStyleKeys}
%
% \begin{macrosyntax}
% \item[0.18,new] |basicstyle=|\meta{basic style and size}
% \item[0.11] |keywordstyle=|\meta{style for keywords}
% \item[0.19,new,optional] |ndkeywordstyle=|\meta{style for second keywords}
% \item[0.19,new,optional] |rdkeywordstyle=|\meta{style for third keywords}
% \item[0.18,new] |nonkeywordstyle=|\meta{style}
% \item[0.11] |commentstyle=|\meta{style}
% \item[0.12] |stringstyle=|\meta{style}
% \item[0.16] |labelstyle=|\meta{style}
%
%		Each value of these keys determines the font and size (or more general style) in which special parts of a listing appear.
%		The \emph{last} token of the (non,nd,rd) keyword and label style might be an one-parameter command like |\textbf| or |\underline|.
%
% \item[0.16,bug] |labelstep=|\meta{step}
%
%		No line numbers are printed if \meta{step} is zero.
%		Otherwise all lines with ''line number $\equiv 0$ modulo \meta{step}'' get a label, which appearance is controlled by the label style.
%
%		Bug: If a comment (or string) goes over several lines, the line numbers (if on) also have comment (string) style.
%		The easiest work-around is possibly the use of |\normalfont| in |\lstlabelstyle|.
%
% \item[0.12] |blankstring=|\meta{true$\vert$false}
%
%		let blank spaces in strings appear as blank spaces respectively as \textvisiblespace.
%		The latter is predefined.
%
% \item[0.12] |tabsize=|\meta{number}
%
%		sets tabulator stops at columns $n+1$, $2n+1$, $3n+1$, etc. if $n=$\meta{number}.
%		Each tabulator in a listing moves the current column to the next tabulator stop.
%		It is initialized with |tabsize=8|.
%
% \item[0.19,new] |formfeed=|\meta{token sequence}
%
%		Whenever a listing contains a form feed \meta{token sequence} is executed.
%		It is initialized with |formfeed=\bigbreak|.
% \end{macrosyntax}
%
%
% \subsection{Fixed and flexible columns}
%
% \begin{macrosyntax}
% \item[0.18,new] |flexiblecolumns=|\meta{true$\vert$false}
%
%		selects the flexible respectively the fixed column format.
%
% \item[0.16,addon] |baseem=|\meta{width in units of em}\quad or\quad|baseem={|\meta{fixed}|,|\meta{flexible mode}|}|
%
%		sets the width of a single character box for fixed and flexible mode (both to the same value or individually).
%		It's pre-defined via |baseem={|{\makeatletter\lst@baseemfixed}|,|{\makeatletter\lst@baseemflexible}|}|.
% \end{macrosyntax}
%
%
% \subsection{Listing alignment}\label{uListingAlignment}
%
% \begin{macrosyntax}
% \item[0.19,new] |indent=|\meta{dimension}
%
%		indents each listing by \meta{dimension}, which is initialized with 0pt.
%		This command is the best way to move line numbers (and the listing) to the right.
%
% \item[0.19,new] |wholeline=|\meta{true$\vert$false}
%
%		prevents or lets the package use indention from list environments like \texttt{enumerate} or \texttt{itemize}.
%
% \item[0.16,bug,addon] |spread=|\meta{dimension}\quad or\quad|spread={|\meta{inner}|,|\meta{outer}|}|
%
%		defines \emph{additional} line width for listings, which may avoid overfull |\hbox|es if a listing has long lines.
%		The inner and outer spread is given explicitly or is equally shared.
%		It is initialized via |spread=0pt|.
%		For one sided documents 'inner' and 'outer' have the effect of 'left' and 'right'.
%		Note that |\lstbox| doesn't use this spread but |indent| (which is always 'left').
%
%		Bug (two sided documents):
%		At top of page it's possible that the package uses inner instead of outer spread or vice versa.
%		This happens when \TeX\ finally moves one or two source code lines to the next page, but hasn't decided it when the \textsf{listings} package processes them.
%		Work-around: interrupt the listing and/or use an explicit |\newpage|.
%
% \item[0.17] |lineskip=|\meta{additional space between lines}
%
%		specifies the additional space between lines in listings.
%		You may write |lineskip=-1pt plus 1pt minus 0.5pt| for example, but 0pt is the default.
%
% \item[0.19,new] |outputpos=|\meta{c$\vert$l$\vert$r}
%
%		controls horizontal orientation of smallest output units (keywords, identifiers, etc.).
%		The arguments work as follows, where vertical bars visualize the effect:
%			$\vert$\hbox to 4.2em{\hss l\hss i\hss s\hss t\hss i\hss n\hss g\hss }$\vert$,
%			$\vert$\hbox to 4.2em{l\hss i\hss s\hss t\hss i\hss n\hss g\hss }$\vert$ and
%			$\vert$\hbox to 4.2em{\hss l\hss i\hss s\hss t\hss i\hss n\hss g}$\vert$
%		in fixed column mode resp.\ 
%			$\vert$\hbox to 3.15em{\hss listing\hss}$\vert$,
%			$\vert$\hbox to 3.15em{listing\hss}$\vert$ and
%			$\vert$\hbox to 3.15em{\hss listing}$\vert$
%		with flexible columns (using pre-defined |baseem|).
%		By default the output is centered.
%		To make all things clear: You may write |outputpos=c|, |outputpos=l| or |outputpos=r|.
% \end{macrosyntax}
%
%
% \subsection{Escaping to \LaTeX}
%
% \textbf{Note:} {\itshape Any escape to \LaTeX\ may disturb the column alignment since the package can't control the spacing there.}
% \begin{macrosyntax}
% \item[0.18,new] |texcl=|\meta{true$\vert$false}
%
%		activates or deactivates \LaTeX\ comment lines, see the example below.
%
% \item[0.19,new] |mathescape=|\meta{true$\vert$false}
%
%		activates or deactivates special behaviour of the dollar sign.
%		If activated, a dollar sign acts as \TeX's text math shift (not display style!).
%
% \item[0.19,new] |escapechar=|\meta{single character} (or empty)
%
%		The given character escapes the user to \LaTeX:
%		All code between two such characters is interpreted as \LaTeX\ code.
%		Note that \TeX's special characters must be entered with a preceding backslash, e.g.\ |escapechar=\%|.
% \end{macrosyntax}
% Some examples clarify these commands.
% Since |texcl| only makes sense with comment lines, we will use C++ comment lines here (without saying how to define them).
% Note that only the first example runs through a C++ compiler if the source code is placed in a separate file.
% All \LaTeX\ escapes are local here, but you can also use |\lstset|.
% \lstset{commentline=//}
% \begin{lstsample}{}
%\begin{lstlisting}[texcl]{}
%// calculate $a_{ij}$
%  A[i][j] = A[j][j]/A[i][j];
%\end{lstlisting}
% \end{lstsample}
%
% \begin{lstsample}{}
%\begin{lstlisting}[mathescape]{}
%// calculate $a_{ij}$
%  $a_{ij} = a_{jj}/a_{ij}$;
%\end{lstlisting}
% \end{lstsample}
%
% \begin{lstsample}{}
%\lstset{escapechar=\%}
%\begin{lstlisting}{}
%// calc%ulate $a_{ij}$%
%  %$a_{ij} = a_{jj}/a_{ij}$%;
%\end{lstlisting}
% \end{lstsample}
% \lstset{commentline={}}
% \noindent
% In the first example the whole comment line is typset in '\LaTeX\ mode' --- except the comment line indicators |//|.
% Note that |texcl| uses comment style.
% In the second example the comment line upto $a_{ij}$ have been typeset in comment style and by the \textsf{listings} package.
% The '$a_{ij}$' itself is typeset in '\TeX\ math mode' without comment style.
% About the half comment line of the third example have been typeset by this package.
% The rest is in '\LaTeX\ mode' without comment style.
%
% To avoid problems with the current and future version of this package:
% \begin{enumerate}
% \item Don't use any command of the \textsf{listings} package when you have escaped to \LaTeX.
% \item Any environment must start and end inside the same escape.
% \item You might use |\def|, |\edef|, etc., but do not assume that the definitions are present later --- except they are |\global|.
% \item |\if \else \fi|, groups, math shifts |$| and |$$|, \ldots\ must be balanced each escape.
% \item \ldots
% \end{enumerate}
% Expand that list yourself and mail me about new items.
%
%
% \subsection{Some special keys}\label{uSomeSpecialKeys}
%
% \begin{macrosyntax}
% \item[0.18,new] |extendedchars=|\meta{true$\vert$false}
%
%		allows or prohibits extended characters in listings, i.e.\ characters with codes 128--255.
%		If you use extended characters, you should load the \texttt{inputenc} or \texttt{fontenc} package!
%
% \item[0.19,new,lang.spec.]|printpod=|\meta{true$\vert$false}
%
%		prints or drops PODs in Perl.
%
% \item[0.12,addon] |pre=[|\meta{continue}|]{|\meta{commands to execute}|}|
% \item[0.12,addon] |post=[|\meta{continue}|]{|\meta{commands to execute}|}|
%
%		The given control sequences are executed before and after typesetting resp.\ when continuing a listing, but in all cases inside a group.
%		The commands are not executed if you use |\lstbox| or |\lstinline| (since the user given pre and post commands are assumed to be unsave inside |\hbox|).
%		By default \meta{continue} is empty.
%		The other arguments are pre-set empty.
% \end{macrosyntax}
%
% \begin{lstsample}{}
%\lstset{flexiblecolumns}
%\lstset{pre=[%
%    We continue the listing.
%  ]A simple prelisting example.}
%\begin{lstlisting}{}
%{ Get listings.dtx. Now! }
%\end{lstlisting}
%
%\begin{lstlisting}{ }
%{ You have it already? Good. }
%\end{lstlisting}
%
%\begin{lstlisting}{ }
%{ Tell your friends about it. }
%\end{lstlisting}
% \end{lstsample}
%\lstset{style={}}
%
%
% \subsection{Languages and styles}\label{uLanguagesAndStyles}
%
% \begin{macrosyntax}
% \item[0.17,changed] |language=[|\meta{dialect}|]|\meta{language name}
%
%		activates a (dialect of a) programming language.
%		All arguments are case \emph{insensitive} here.
%
% \item[0.19,new] |\lstloadlanguages{|\meta{list of languages}|}|
%
%		can only be used in the preamble.
%		It loads all specified languages, where each language is given in the form |[|\meta{dialect}|]|\meta{language}.
%		List of languages means a comma separated list.
%
% \item[0.18,new]	|style=|\meta{style name}
%
%		activates a style.
%		The arguments is case \emph{insensitive}.
%
% \item[0.19,new] |\lstdefinestyle{|\meta{style name}|}{|\meta{key=value list}|}|
%
%		stores the key=value list: You can select the style via |style| using te style name as argument.
% \end{macrosyntax}
% After package loading a 'standard' style and an 'empty' language are active.
% The style is defined as follows.
% \begin{verbatim}
%    \lstdefinestyle{}
%    {basicstyle={},
%     keywordstyle=\bfseries,nonkeywordstyle={},
%     commentstyle=\itshape,
%     stringstyle={},
%     labelstyle={},labelstep=0
%    }\end{verbatim}
%
%
% \subsection{Language definitions}
%
% \begin{macrosyntax}
% \item[0.19,new] |\lstdefinelanguage[|\meta{dialect}|]{|\meta{language}|}{|\meta{key=value list}|}|
%
%		defines a programming language.
%		Use any 'language' key=value list and the definition is yours.
%
% \item[0.19,new] |\lstdefinedrvlanguage[|\meta{dialect}|]{|\meta{language}|}{|\meta{key=value list}|}|
%
%		defines a programming language only if the user has requested the language.
%		Thus this command can only be used in driver files.
%
% \item[0.19,new] |defaultdialect=[|\meta{dialect}|]|\meta{language}
%
%		defines a default dialect for a language, that means a dialect which is selected whenever you leave out the optional dialect.
%		If you have defined a default dialect other than empty, for example |defaultdialect=[iama]fool|, you can't select the 'empty' dialect, even not with |language=[]fool|.
%
%		Note that a configuration file possibly defines some default dialects.
%
% \item[0.18,new] |\lstalias{|\meta{alias}|}{|\meta{language}|}|
%
%		defines an alias for a programming language.
%		Any dialect of \meta{alias} selects in fact the same dialect of \meta{language}.
%		It's also possible to define an alias for one dialect: |\lstalias[|\meta{dialect alias}|]{|\meta{alias}|}[|\meta{dialect}|]{|\meta{language}|}|.
%		Here all four parameters are \emph{non}optional.
%		An alias with empty \meta{dialect} will select the default dialect.
%		Note that aliases can't be nested: The two aliases |\lstalias{foo1}{foo2}| and |\lstalias{foo2}{foo3}| redirect |foo1| not to |foo3|.
%
%		Note that a configuration file possibly defines some aliases.
%
% \item[0.18,new] |\lststorekeywords|\meta{macro}|{|\meta{keywords}|}|
%
%		stores \meta{keywords} in \meta{macro} for use with keyword keys.
%		This command can't be used in a language definition since it is a command and not a key.
% \end{macrosyntax}
%
% Now come all the language keys, which might be used in the key=value list of |\lstdefinelanguage|.
% Note: {\itshape If you want to enter {\upshape|\|, |{|, |}|, |%|, |#|} or {\upshape|&|} inside or as an argument here, you must do it with a preceding backslash!}
%
% \begin{macrosyntax}
% \item[0.11] |keywords={|\meta{keywords}|}|
% \item[0.11] |morekeywords={|\meta{additional keywords}|}|
% \item[0.18,new] |deletekeywords={|\meta{keywords to remove}|}|
%
% \item[0.19,new,optional] |ndkeywords={|\meta{second keywords}|}|
% \item[0.19,new,optional] |morendkeywords={|\meta{additional second keywords}|}|
% \item[0.19,new,optional] |deletendkeywords={|\meta{second keywords to remove}|}|
%
% \item[0.19,new,optional] |rdkeywords={|\meta{third keywords}|}|
% \item[0.19,new,optional] |morerdkeywords={|\meta{additional third keywords}|}|
% \item[0.19,new,optional] |deleterdkeywords={|\meta{third keywords to remove}|}|
%
%		Each 'keyword' argument (here and below) is a list of keywords separated by commas.
%		You might use macros defined with |\lststorekeywords| as elements (i.e.\ also separated by commas).
%		|keywords={save,Test,test}| defines three keywords (if keywords are case sensitive).
%		If you want to remove them all, simply write |keywords={}| --- in |\lstset|'s argument, in an optional argument or in a language definition.
%		The characters |\|, |{|, and |}| (entered as |\\|, |\{| and |\}|) are \emph{not} allowed within a keyword.
%		You might use each of
%^^A
%^^A We need some definitions here.
%^^A
%\makeatletter ^^A
%\def\lstspec#1{^^A
%    \ifx\relax#1\else ^^A
%        {\lstset{keywords=#1}\setbox\@tempboxa\hbox{\lstinline a#1a}}^^A
%        \lst@ifspec\lstinline a#1a\fi ^^A
%        \expandafter\lstspec ^^A
%    \fi}^^A
%\lst@AddToHook{Output}{\lstspectest}^^A
%\gdef\lstspecfalse{\global\let\lst@ifspec\iffalse}^^A
%\gdef\lstspectest{^^A
%    \global\let\lst@ifspec=\iftrue ^^A
%	 \ifx\lst@thestyle\lst@keywordstyle\else ^^A
%        \lstspecfalse ^^A
%	 \fi}^^A
%\makeatother ^^A
%	{\lstset{language={}}^^A
%    \lstspec !"\#$\%\&'()*+-./:;<=>?[]^`\relax.}
%^^A
%^^A a comma also works!
%^^A
%^^A End of test.
%^^A
%		But note that you must enter |\#|, |\%| and |\&| instead of |#|, |%| and |&|.
%
% \item[0.14] |sensitive=|\meta{true$\vert$false}
%
%		makes the keywords (first, second and third) case sensitive resp.\ insensitive.
%^^A	It would be nice to have |\lst|\meta{\texttt{nd}$\vert$\texttt{rd}}|sensitive|\meta{\texttt{true}\textbar\texttt{false}}?
%		This key affect the keywords only in the phase of typesetting.
%		In all other situations keywords are case sensitive, i.e.\ |deletekeywords={save,Test}| removes 'save' and 'Test', but neither 'SavE' nor 'test'.
%
% \item[0.19,new] |alsoletters={|\meta{character sequence}|}|
% \item[0.19,new] |alsodigits={|\meta{character sequence}|}|
% \item[0.19,new] |alsoother={|\meta{character sequence}|}|
%
%		These keys support the 'special character' auto-detection of the keyword commands.
%		For our purpose here, identifiers are out of letters (|A|--|Z|,|a|--|z|,|_|,|@|,|$|) and digits (|0|--|9|), but an identifier must begin with a letter.
%		If you write |keywords={one-two,\#include}|, the minus becomes necessarily a digit and the sharp a letter since the keywords can't be detected otherwise.
%		The three keys overwrite such default behaviour.
%		Each character of the sequence becomes a letter, digit and other, respectively.
%		Note that the auto-detection might fail if you remove keywords and always fails if you use special characters not listed above.
%^^A
%^^A TODO: This is mainly due to improper \lst@ReplaceIn inside ..MakeKeywordArg..
%^^A
%
% \item[0.19,new] |stringtest=|\meta{true$\vert$false}
%
%		enables or disables string tests:
%		If activated, line exceeding strings issue warnings and the package exits string mode.
%
% \item[0.12,addon] |stringizer=[|\meta{b$\vert$d$\vert$m$\vert$bd}|]{|\meta{character sequence}|}|
%
%		Each character might start a string or character literal.
%		'Stringizers' match each other, i.e.\ starting and ending delimiters are the same.
%		The optional argument controls how the stringzier(s) itself is/are represented in a string or character literal:
%		It is preceded by a |b|ackslash, |d|oubled (or both is allowed via |bd|) or it is |m|atlabed.
%		The latter one is a special type for Ada and Matlab and possibly more languages, where the stringizers are also used for other purposes.
%		In general the stringizer is also doubled, but a string does not start after a letter or a right parenthesis.
%
% \item[0.19,new,lang.spec.] |texcs={|\meta{list of control sequences \textup(without backslashes\textup)}|}|
%
%		defines control sequences for \TeX\ and \LaTeX.
%
% \item[0.18,new,lang.spec.] |cdirectives={|\meta{list of compiler directives}|}|
%
%		defines compiler directives in C, C++ and Objective-C.
% \end{macrosyntax}
% If you have already defined any of the following comments and you want to remove it, let all arguments to the comment key empty.
% \begin{macrosyntax}
% \item[0.13,changed] |commentline=|\meta{1or2 chars}
%
%		The characters (\emph{in the given order}) start a comment line, which in general starts with the comment separator and ends at end of line.
%		If the character sequence |//| starts a comment line (like in C++, Comal 80 or Java), |commentline=//| is the correct declaration.
%		For Matlab it would be |commentline=\%| --- note the preceding backslash.
%
% \item[0.18,new] |fixedcommentline=[|\meta{n=preceding columns}|]{|\meta{character sequence}|}|
%
%		Each given character becomes a 'fixed comment line' separator: It starts a comment line if and only if it is in column $n+1$.
%		Fortran 77 declares its comments via |fixedcommentline={*Cc}| ($n=0$ is default).
%
% \item[0.13,changed] |singlecomment={|\meta{1or2 chars}|}{|\meta{1or2 chars}|}|
% \item[0.13,changed] |doublecomment={|\meta{1or2 chars}|}{|\meta{1or2 chars}|}{|\meta{1or2 chars}|}{|\meta{1or2 chars}|}|
%
%		Here we have two or four comment separators.
%		The first starts and the second ends a comment, and similarly the third and fourth separator for double comments.
%		If you need three such comments use |singlecomment| and |doublecomment| at the same time.
%		C, Java, PL/I, Prolog and SQL all define single comments via |singlecomment={/*}{*/}|, and Algol does it with |singlecomment={\#}{\#}|, which means that the sharp delimits both beginning and end of a single comment.
%
% \item[0.13,changed] |nestedcomment={|\meta{1or2 chars}|}{|\meta{1or2 chars}|}|
%
%		is similar to |singlecomment|, but comments can be nested.
%		Identical arguments are not allowed --- think a while about it!
%		Modula-2 and Oberon-2 use |nestedcomment={(*}{*)}|.
%
% \item[0.17,lang.spec.] |keywordcomment={|\meta{keywords}|}|
% \item[0.17,lang.spec.] |doublekeywordcommentsemicolon={|\meta{keywords}|}{|\meta{keywords}|}{|\meta{keywords}|}|
%
%		A (paired) keyword comment begins with a keyword and ends with the same keyword.
%		Consider |keywordcomment={comment,co}|.
%		Then '\textbf{comment}\allowbreak\ldots\textbf{comment}' and '\textbf{co}\ldots\textbf{co}' are comments.\\
%^^A
%		Defining a double keyword comment (semicolon) needs three keyword lists, e.g.\ |{end}{else,end}{comment}|.
%		A semicolon always ends such a comment.
%		Any keyword of the first argument begins a comment and any keyword of the second argument ends it (and a semicolon also); a comment starting with any keyword of the third argument is terminated with the next semicolon only.
%		In the example all possible comments are '\textbf{end}\ldots\textbf{else}', '\textbf{end}\ldots\textbf{end}' (does not start a comment again) and '\textbf{comment}\ldots;' and '\textbf{end}\ldots;'.
%		Maybe a curious definition, but Algol and Simula use such comments.\\
%^^A
%		Note: The keywords here need not to be a subset of the defined keywords.
%		They won't appear in keyword style if they aren't.
% \end{macrosyntax}
%
%
% \section{Experimental features}
%
% This section describes the more or less unestablished parts of the \textsf{listings} package.
% It's unlikely that they are removed, but they are liable to (heavy) changes and maybe improvements.
%
%
% \subsection{Interface to \textsf{fancyvrb}}
%
% The \textsf{fancyvrb} package --- fancy verbatims --- from Timothy van Zandt provides macros for reading, writing and typesetting verbatim code.
% It has some remarkable features the \textsf{listings} package doesn't have --- some are also possible with \textsf{listings}, but you must find somebody who implements them ; -- ).
% The \textsf{fancyvrb} package is available from \texttt{CTAN: macros/latex/contrib/supported/fancyvrb}.
%
% \begin{macrosyntax}
% \item	[0.19,new]|fancyvrb=|\meta{true$\vert$false}
%
%		activates or deactivates the interface.
%		This defines an appropiate version of |\FancyVerbFormatLine| to make the two packages work together.
%		If active, the verbatim code read by the \textsf{fancyvrb} package is typeset by the \textsf{listings} package, i.e.\ with emphasized keywords, strings, comments, and so on.
%		--- You should know that |\FancyVerbFormatLine| is responsible for the typesetting a single code line.
%
%		If \textsf{fancyvrb} and \textsf{listings} provide similar functionality, use \textsf{fancyvrb}'s.
%
%		Note that this is the first interface.
%		It works only with |Verbatim|, neither with |BVerbatim| nor |LVerbatim|.
%		And you shouldn't use \textsf{defineactive}. (As I can see it doesn't matter since it does nothing at all.)
%		I hope to remove some restrictions in future.
% \end{macrosyntax}
%
% \iffancyvrb
% \begin{lstsample}{}
%\lstset{commentline=\ }% :-)
%
%\begin{Verbatim}[commandchars=\\\{\}]
%First verbatim line.
%\fbox{Second} verbatim line.
%\end{Verbatim}
%
%\lstset{fancyvrb}
%\begin{Verbatim}[commandchars=\\\{\}]
%First verbatim line.
%\fbox{Second} verbatim line.
%\end{Verbatim}
%\lstset{fancyvrb=false}
% \end{lstsample}
% \noindent
% The last two lines are wider than the first two since |baseem| equals not the width of a single typewriter character.
% \else
% \begin{center}
% \textsf{fancyvrb} seems to be unavailable on your platform, thus the example couldn't be printed here.
% \end{center}
% \fi
%
%
% \subsection{Listings inside arguments}\label{uListingsInsideArguments}
%
% There are some things to consider if you want to use |\lstinline| or the listing environment inside arguments.
% Since \TeX\ reads the argument before the '\lst-macro' is executed, this package can't do anything to preserve the input:
% Spaces shrink to one space, the tabulator and the end of line are converted to spaces, the comment character is not printable, and so on.
% Hence, you must work a little bit more.
% You have to put a backslash in front of each of the following four characters: |\{}%|.
% Moreover you must protect spaces in the same manner if: (i) there are two or more spaces following each other or (ii) the space is the first character in the line.
% That's not enough: Each line must be terminated with a 'line feed' |^^J|.
% Finally you can't escape to \LaTeX\ inside such listings.
%
% The easiest examples are with |\lstinline| since we need no line feed.
% \begin{verbatim}
%\footnote{\lstinline!var i:integer;! and
%          \lstinline!protected\ \ spaces! and
%          \fbox{\lstinline!\\\{\}\%!}}\end{verbatim}
% yields\lstset{language=Pascal}\footnote{\lstinline!var i:integer;! and \lstinline!protected\ \ spaces! and \fbox{\lstinline!\\\{\}\%!}} if the current language is Pascal.
% The environment possibly needs a preceding |\lstbox|, as the following examples show.
%
%{\let\smallbreak\relax\lstset{language={}}
% \begin{lstsample}{}
%\fbox{\lstbox
%\begin{lstlisting}{}^^J
%\ !"#$\%&'()*+,-./^^J
%0123456789:;<=>?^^J
%@ABCDEFGHIJKLMNO^^J
%PQRSTUVWXYZ[\\]^_^^J
%`abcdefghijklmno^^J
%pqrstuvwxyz\{|\}~^^J
%\end{lstlisting}}
% \end{lstsample}
%
% \lstset{language={}}
% \begin{lstsample}{}
%\fbox{\lstbox
%\begin{lstlisting}{}^^J
%We need no protection here,^^J
%\ but\ \ in\ \ this\ \ line.^^J
%\end{lstlisting}}
% \end{lstsample}
%}
%
%
% \subsection{Frames}
%
% \begin{macrosyntax}
% \item	[0.19,new] |frame=|\meta{any subset of \textup{\texttt{tlrbTLRB}}}
%
%		The characters |tlrbTLRB| are attached to lines at the |t|op of a listing, on the |l|eft, |r|ight and at the |b|ottom.
%		There are two lines if you use upper case letters.
%		If you want a single frame around a listing, write |frame=tlrb| or |frame=bltr| for example (but as optional argument or argument to |\lstset|, of course).
%		If you want double lines at the top and on the left and no other lines, write |frame=TL|.
%		Note that frames reside outside the listing's space.
%		Use |spread| if you want to shrink frames (to |\linewidth| for example) and use |indent| if you want to move line number inside frames.
%
% \item	[0.19,new] |framerulewidth=|\meta{dimension}
% \item	[0.19,new] |framerulesep=|\meta{dimension}
%
%		These keys control the width of the rules and the space between double rules.
%		The predefined values are {\makeatletter\lst@framewidth} width and {\makeatletter\lst@framesep} separation.
%
%\iffalse
% \item	[0.19,new] |frametextsep=|\meta{dimension}
%
%		controls the space between frame and listing, but currently only between the listing and vertical frame lines.
%		The predefined value is {\makeatletter\lst@frametextsep}.
%\fi
% \end{macrosyntax}
% |frame| does not work with |\lstbox| or |fancyvrb=true|!
% And there are certainly more problems with other commands.
% Take the time to report in.
%
% \begin{lstsample}{}
%\begin{lstlisting}[frame=tLBr]{}
%for i:=maxint to 0 do
%begin
%    { do nothing }
%end;
%\end{lstlisting}
% \end{lstsample}
%
%
% \subsection{Export of identifiers}
%
%^^A \lsthelper{Aslak Raanes}{araanes@ifi.ntnu.no}{1997/11/24}{export function names}
% It would be nice to export function or procedure names, for example to index them automatically or to use them in |\listoflistings| instead of a listing name.
% In general that's a dream so far.
% The problem is that programming languages use various syntaxes for function and procedure declaration or definition.
% A general interface is completely out of the scope of this package --- that's the work of a compiler and not of a pretty printing tool.
% However, it is possible for particular languages: in Pascal each function or procedure definition and variable declaration is preceded by a particular keyword.
% \begin{macrosyntax}
% \item	[0.19,new,optional] |index={|\meta{identifiers}|}|
%
%		\meta{identifiers} is a comma-separated list of identifiers.
%		Each appearance of such an identifier is indexed.
%
% \item	[0.19,new,optional] |indexmacro=|\meta{'one parameter' macro}
%
%		The specified macro gets exactly one parameter, namely the identifier, and must do the indexing.
%		It is predefined as |indexmacro=\lstindexmacro|, which definition is
% \begin{verbatim}
%    \newcommand\lstindexmacro[1]{\index{{\ttfamily#1}}}\end{verbatim}
% \item[0.19,new,optional] |prockeywords={|\meta{keywords}|}|
%
%		\meta{keywords} is a comma-separated list of keywords, which indicate a function or procedure definition.
%		Any identifier following such a keyword appears in 'procname' style.
%		For Pascal you might use
% \begin{verbatim}
%    prockeywords={program,procedure,function}\end{verbatim}
%
% \item[0.19,new,optional] |procnamestyle=|\meta{style for procedure names}
%
%		defines the style in which procedure and function names appear.
%
% \item[0.19,new,optional] |indexprocnames=|\meta{true$\vert$false}
%
%		If activated, procedure and function names are also indexed (if used with |index| option).
% \end{macrosyntax}
%
%
% \section{Troubleshooting}
%
% The known bugs have already been described.
% This section deals with problems concerning not only the \textsf{listings} package.
%
%
% \subsection{Problems with \texttt{.fd} files}
%
% You probably get the following error message with a different font definition file:
% \begin{verbatim}
%! LaTeX Error: Command textparagraph unavailable in encoding T1.
%
%See the LaTeX manual or LaTeX Companion for explanation.
%Type  H <return>  for immediate help.
% ...
%
%l.68 \P
%       rovidesFile{omscmr.fd}
%?\end{verbatim}
% So, what happened?
% (a) The \textsf{listings} package redefines the character table to print listings.
% (b) \LaTeX\ loads the font definition files on demand.
% (a) plus (b) gives the error: \LaTeX\ loads the \texttt{.fd} files with modified character table.
% And that goes wrong.
% The work-around is quite easy: Input the \texttt{.fd} file before typesetting the listing.
%
%
% \subsection{Language definitions}
%
% Language definitions and also some style definitions tend to have long definition parts.
% This is why we tend to forget commas between the key=value elements.
% If you select a language and get a |Missing = inserted for \ifnum| error, this is surely due to a missing comma after |keywords=|value.
% If you encounter unexspected characters after selecting a language (or style), you have either forgotten a comma or you have given to many arguments to a key, for example |commentline={--}{!}|.
%
%
% \section{Forthcoming}
%
% \begin{itemize}
% \item I'd like to support more languages, for example Maple, Mathematica, PostScript, Reduce and so on.
%		Fortunately my lifetime is limited, so other people may do that work.
%		Write a language definition and (e-)mail it to me (with a proposal in which file to place the definition).
%
% \item There will possibly a boolean |blanklisting=|\meta{true$\vert$false} or a $*$-version of the environment.
%
% \item 'procnames' is already interesting, but marks (and indexes) only the function definitions so far.
%		It would be quite easy to mark also the following function calls:
%		Write another 'keyword class' which is empty at the very beginning (and can be reset with a key); each function definition appends a 'keyword' which will appear in 'procnamestyle'.
%		But this would be another 'keyword test' within an inner loop.
% \item I plan to put all language definitions in a single file.
% \end{itemize}
%
%
% \StopEventually{}
%
%
% \part{Implementation}
%
% \CheckSum{5520}
% \DoNotIndex{\[,\{,\},\],\1,\2,\3,\4,\5,\6,\7,\8,\9,\0}
% \DoNotIndex{\`,\,,\!,\#,\$,\&,\',\(,\),\+,\.,\:,\;,\<,\=,\>,\?,\_}
% \DoNotIndex{\@@end,\@@par,\@currenvir,\@depth,\@dottedtocline,\@ehc}
% \DoNotIndex{\@empty,\@firstoftwo,\@gobble,\@gobbletwo,\@gobblefour,\@height}
% \DoNotIndex{\@ifnextchar,\@ifundefined,\@namedef,\@ne,\@secondoftwo}
% \DoNotIndex{\@spaces,\@starttoc,\@undefined,\@whilenum,\@width}
% \DoNotIndex{\A,\active,\addtocontents,\advance,\aftergroup,\batchmode}
% \DoNotIndex{\begin,\begingroup,\bfseries,\bgroup,\box,\bigbreak,\bullet}
% \DoNotIndex{\c@page,\catcode,\contentsname,\csname,\def,\divide,\do,\dp}
% \DoNotIndex{\edef,\egroup,\else,\end,\endcsname,\endgroup,\endinput}
% \DoNotIndex{\endlinechar,\escapechar,\everypar,\expandafter,\f@family}
% \DoNotIndex{\fi,\footnotesize,\gdef,\global,\hbox,\hss,\ht}
% \DoNotIndex{\if,\ifdim,\iffalse,\ifnum,\ifodd,\iftrue,\ifx}
% \DoNotIndex{\ignorespaces,\index,\input,\itshape,\kern}
% \DoNotIndex{\lccode,\l@ngrel@x,\leftskip,\let,\linewidth,\llap}
% \DoNotIndex{\long,\lowercase,\m@ne,\makeatletter,\mathchardef}
% \DoNotIndex{\message,\multiply,\NeedsTeXFormat,\newbox,\new@command}
% \DoNotIndex{\newcommand,\newcount,\newdimen,\newtoks,\noexpand}
% \DoNotIndex{\noindent,\normalbaselines,\normalbaselineskip}
% \DoNotIndex{\offinterlineskip,\par,\parfillskip,\parshape}
% \DoNotIndex{\parskip,\ProcessOptions,\protect,\ProvidesPackage}
% \DoNotIndex{\read,\relax,\removelastskip,\rightskip,\rlap,\setbox}
% \DoNotIndex{\smallbreak,\smash,\space,\string,\strut,\strutbox}
% \DoNotIndex{\the,\thepage,\ttdefault,\ttfamily,\tw@,\typeout,\uppercase}
% \DoNotIndex{\vbox,\vcenter,\vrule,\vtop,\wd,\xdef,\z@,\zap@space}
% \DoNotIndex{\char,\closeout,\immediate,\newwrite,\openout,\write}
% \DoNotIndex{\vskip}
%
% \DoNotIndex{\define@key,\setkeys}
%
% \DoNotIndex{\textasciicircum,\textasciitilde,\textasteriskcentered}
% \DoNotIndex{\textbackslash,\textbar,\textbraceleft,\textbraceright}
% \DoNotIndex{\textdollar,\textendash,\textgreater,\textless}
% \DoNotIndex{\textunderscore,\textvisiblespace}
%
% \DoNotIndex{\filename@area,\filename@base,\filename@ext}
% \DoNotIndex{\filename@parse,\IfFileExists,\InputIfFileExists}
% \DoNotIndex{\DeclareOption,\MessageBreak}
% \DoNotIndex{\GenericError,\GenericInfo,\GenericWarning}
%
% \DoNotIndex{\blankstringfalse,\blankstringtrue,\commentstyle}
% \DoNotIndex{\DeclareCommentLine,\DeclareDoubleComment}
% \DoNotIndex{\DeclareNestedComment,\DeclarePairedComment}
% \DoNotIndex{\DeclareSingleComment,\inputlisting,\inputlisting@,\keywords}
% \DoNotIndex{\keywordstyle,\labelstyle,\listingfalse,\listingtrue}
% \DoNotIndex{\morekeywords,\normallisting,\postlisting,\prelisting}
% \DoNotIndex{\selectlisting,\sensitivefalse,\sensitivetrue}
% \DoNotIndex{\spreadlisting,\stringizer,\stringstyle,\tablength}
%
% \DoNotIndex{\lst@AddToHook@,\lst@Aspect@,\is,\lst@AspectGobble}
% \DoNotIndex{\lst@DeleteKeysIn@,\lst@Environment@}
% \DoNotIndex{\lst@IfNextChars@,\lst@IfNextChars@@,\lst@KCOutput@}
% \DoNotIndex{\lst@MakeActive@,\lst@MSkipUptoFirst@,\lst@ReplaceIn@}
% \DoNotIndex{\lst@LocateLanguage@,\lst@SKS@,\lstalias@,\lstalias@@}
% \DoNotIndex{\lstbaseem@,\lstbox@,\lstCC@ECUse@,\lstCC@Escape@}
% \DoNotIndex{\lstCC@CBC@,\lstCC@CBC@@,\lstCC@CommentLine@}
% \DoNotIndex{\lstCC@EndKeywordComment@,\lstCC@FixedCL@}
% \DoNotIndex{\lstCC@ProcessEndPOD@,\lstCC@SpecialUseAfter@}
% \DoNotIndex{\lstCC@Stringizer@,\lstdrvlanguage@,\lstenv@AddArg@}
% \DoNotIndex{\lstenv@Process@,\lstenv@ProcessJ@,\lstinputlisting@}
% \DoNotIndex{\lstoutputpos@,\lstresetlineno@,\lstresetlineno@@}
% \DoNotIndex{\lstspread@}
% \DoNotIndex{\lstnestedcomment@,\lstpost@,\lstpre@,\lststringizer@}
%
%
% \section{Overture}
%
% The version 0.19 kernel needs 1 token register, 6 counters and 5 dimensions.
% The macros, boxes and counters |\@temp?a| and |\@temp?b| and the dimension |\@tempdima| are also used, see the index.
% And I shouldn't forget |\@gtempa|.
% Furthermore, the required \textsf{keyval} package allocates one token register.
% The \textsf{fancyvrb} interface needs one box.
%
% Before considering the implementation, here some conventions I used:
% \begin{itemize}
% \item All public macros have lower case letters and the prefix \lst.
% \item The name of all private macros and variables use the prefixes (possibly not up to date):
%	\begin{itemize}
%	\item |lst@| for a general macro or variable,
%	\item |lstenv@| if it is defined for the listing |env|ironment,
%	\item |lstCC@| for |c|haracter |c|lass macros,
%	\item |lsts@| for |s|aved character meanings,
%	\item |lstf@| for frame declaration,
%	\item |\lsthk@|\meta{name of hook} holds hook material,
%	\item a language may define (|Pre| for prepare, |SCT| for select char table)
%		\begin{itemize}
%		\item[] |\lstPre@|\meta{language}
%		\item[] |\lstPre@|\meta{language}|@|\meta{dialect}
%		\item[] |\lstSCT@|\meta{language}
%		\item[] |\lstSCT@|\meta{language}|@|\meta{dialect}
%		\end{itemize}
%		where the first two are executed (if they exist) one time each listing (before selecting character table, but after executing the hook Before\-Select\-Char\-Table).
%		The other two macros are called possibly more than once, namely whenever the package selects the character table.
%		The specialized \meta{dialect} versions are called after the more general \meta{language} macros.
%	\item |lstk@| is reserved for keyword tests,
%
%	\item |\lstlang@|\meta{language}|@|\meta{dialect} contains a language definition,
%	\item |\lststy@|\meta{the style} contains style definition,
%
%	\item |\lstloc@|\meta{language} keeps location (= file name without extension) of a language definition if it's not |lst|\meta{language},
%	\item |\lsta@|\meta{language}|@|\meta{dialect} contains alias,
%	\item |\lstaa@|\meta{language} contains alias for all dialects of a language,
%	\item |\lstdd@|\meta{language} contains default dialect of a language (if present).
%	\end{itemize}
% \item If a submacro does the main work, e.g.\ |\lstinputlisting@| does it for |\lstinputlisting|, we use the suffix |@|.
% \item To distinguish procedure-like macros from macros holding data, the name of procedure macros use upper case letters with each beginning word, e.g.\ |\lst@AddTo|.
% \end{itemize}
%
% The kernel starts with identification and option declaration and processing.
% \begingroup
%    \begin{macrocode}
%<*kernel>
\NeedsTeXFormat{LaTeX2e}
\ProvidesPackage{listings}[1998/11/09 v0.19 by Carsten Heinz]
\RequirePackage{keyval}
%    \end{macrocode}
%    \begin{macrocode}
\DeclareOption{0.17}{\@namedef{lst@0.17}{}}
\DeclareOption{index}{\@namedef{lst@index}{}}
\DeclareOption{procnames}{\@namedef{lst@prockeywords}{}}
\DeclareOption{ndkeywords}{\@namedef{lst@ndkeywords}{}}
\DeclareOption{rdkeywords}{\@namedef{lst@rdkeywords}{}}
\DeclareOption{doc}{\@namedef{lst@doc}{}}
\ProcessOptions
%</kernel>
%    \end{macrocode}
% Note that |doc| option is designed for this documentation only.
% \endgroup
%
% \begingroup
%    \begin{macrocode}
%<*info>
%    \end{macrocode}
% \endgroup
% \begin{macro}{\lst@InfoInfo}
% \begin{macro}{\lst@InfoWarning}
% \begin{macro}{\lst@InfoError}
% Each macro gets one argument, which is printed (\texttt{.log} and/or screen) as info, warning or error.
%    \begin{macrocode}
\def\lst@InfoInfo#1{%
    \GenericInfo %
    {(Listings) \@spaces\@spaces\space\space}%
    {Listings Info: #1}}
%    \end{macrocode}
%    \begin{macrocode}
\def\lst@InfoWarning#1{%
    \GenericWarning %
    {(Listings) \@spaces\@spaces\@spaces\space}%
    {Listings Warning: #1}}
%    \end{macrocode}
%    \begin{macrocode}
\def\lst@InfoError#1{%
    \GenericError %
    {(Listings) \@spaces\@spaces\@spaces\space}%
    {Listings Error: #1}%
    {See the package documentation for explanation.}%
    {You're using a debug version of listings.sty.^^J%
     Contact your system administrator to install nondebug one.}}
%    \end{macrocode}
% \end{macro}\end{macro}\end{macro}
% \begingroup
%    \begin{macrocode}
%</info>
%    \end{macrocode}
% \endgroup
%
%
% \section{General problems and \TeX{}niques}
%
%
% \subsection{Quick 'if parameter empty'}
%
% There are many situations where you have to look whether or not a macro parameter is empty.
% Let's say, we want to test |#1|.
% The \emph{natural} way would be something like
% \begin{verbatim}
%     \def\test{#1}%
%     \ifx \test\empty %
%             % #1 is empty
%     \else %
%             % #1 is not empty
%     \fi %\end{verbatim}
% where |\empty| is defined by |\def\empty{}|, of course.
% And now the \emph{mad} way:
% \begin{verbatim}
%     \ifx \empty#1\empty %
%             % #1 is empty
%     \else %
%             % #1 is not empty
%     \fi %\end{verbatim}
% If the parameter is empty, the |\empty| left from |#1| is compared with the |\empty| on the right.
% All is fine since they are the same.
% If the parameter is not empty, the |\empty| on the left is compared with the first token of the parameter.
% Assuming this token is not equivalent to |\empty| the |\else| section is executed, as desired.
%
% The mad way works if and only if the first token of the parameter is not equivalent to |\empty|.
% You must check if this meets your purpose.
% The two |\empty|s might be replaced by any other macro, which is not equivalent to the first token of the parameter.
% But the definition of that macro shouldn't be too complex since this would slow down the |\ifx|.
% In the examples above the mad version needs about $45\%$ of the natural's time.
% Note that this \TeX{}nique lost its importance from version 0.18 on.
%
%
% \subsection{Replacing characters}\label{iReplacingCharacters}
%
% \begingroup
%    \begin{macrocode}
%<*kernel>
%    \end{macrocode}
% \endgroup
% In this section we define the macro
% \begin{macrosyntax}
% \item	|\lst@ReplaceIn|\meta{macro}|{|\meta{replacement list $c_1m_1$\ldots$c_nm_n$}|}|
% \end{macrosyntax}
% Each character $c_i$ inside the given macro is replaced by the macro $m_i$.
% In fact, $c_i$ may be a character sequence (enclosed in braces and possibly containing macros), but $m_i$ must be a single macro which is not equivalent to |\empty| --- use |\relax| instead.
% |\lst@ReplaceIn\lst@arg{_\textunderscore {--}\textendash}| is allowed but not |\lst@ReplaceIn\lst@arg{\textunderscore_\textendash{--}}|.
% These restrictions can be dropped, see the TODO part below.
%
% We derive the main macro from \LaTeX's |\zap@space|:
% \begin{verbatim}
%     \def\zap@space#1 #2{%
%       #1%
%       \ifx#2\@empty\else\expandafter\zap@space\fi
%       #2}\end{verbatim}
% It is called like this:
% \begin{verbatim}
%     \zap@space Some characters with(out) sense and \@empty
%     \expandafter\zap@space\lst@arg{} \@empty
%     \edef\lst@arg{\expandafter\zap@space\lst@arg{} \@empty}\end{verbatim}
% Note that |{}|\textvisiblespace\ produces a space holding up the syntax of |\zap@space|: it's defined with a space between the two parameters.
% If we want to replace a space by a visible space, we could use
% \begin{verbatim}
%     \def\lst@temp#1 #2{%
%       #1%
%       \ifx#2\@empty\else %
%           \noexpand\textvisiblespace %
%           \expandafter\lst@temp %
%       \fi #2}\end{verbatim}
% The additional |\noexpand\textvisiblespace| inserts visible spaces where spaces have been.
% We want to replace several characters and thus have to save our intermediate results.
% So we will follow the third '|\edef|' example.
% But there is some danger: After replacing the first character, the macro $m_1$ will expand inside the next |\edef|.
% Knowing that letters don't expand, we simply say that $m_1$ is a letter.
% Such changes must be local, so \ldots
%
% \begin{macro}{\lst@ReplaceIn}
% we open a group, store the original macro contents in |\@gtempa| and start a loop.
% Afterwards we close the group and assign the modified character string to our given macro.
%    \begin{macrocode}
\def\lst@ReplaceIn#1#2{%
    \bgroup \global\let\@gtempa#1%
    \lst@ReplaceIn@#2\@empty\@empty %
    \egroup \let#1\@gtempa}
%    \end{macrocode}
% If we haven't reached the end of the list (the two |\@empty|s), we let the macro $m_i$ be a letter and define an appropiate macro to replace the character (sequence) |#2| by the macro |#3|.
%    \begin{macrocode}
\def\lst@ReplaceIn@#1#2{%
    \ifx\@empty#2\else %
        \let#2=a%
        \def\lst@temp##1#1##2{##1%
            \ifx\@empty##2\else %
                #2\expandafter\lst@temp %
            \fi ##2}%
%    \end{macrocode}
% Now we call that macro using |\xdef| (since the changes in |\@gtempa| must exist after closing a group).
% Finally we continue the loop.
%    \begin{macrocode}
        \xdef\@gtempa{\expandafter\lst@temp\@gtempa#1\@empty}%
        \expandafter\lst@ReplaceIn@ %
    \fi}
%    \end{macrocode}
% \begin{TODO}
% This |\lst@ReplaceIn| is currently sufficient.
% If we need to replace character sequences by character sequences (instead of single macros), we have to make some minor changes.
% The main difference is that we build the new contents inside |\lst@temp| instead of using |\edef|.
% Note that the improved version is much slower than the macros above.\vspace*{-\baselineskip}
% \begin{verbatim}
%\def\lst@ReplaceIn#1#2{\lst@ReplaceIn@#1#2\@empty\@empty}
%\def\lst@ReplaceIn@#1#2#3{%
%    \ifx\@empty#3\else %
%        \def\lst@temp##1#2##2{%
%            \ifx\@empty##2%
%                \lst@lAddTo#1{##1}%
%            \else %
%                \lst@lAddTo#1{##1#3}%
%                \expandafter\lst@temp %
%            \fi ##2}%
%        \let\@tempa#1\let#1\@empty %
%        \expandafter\lst@temp\@tempa#2\@empty %
%        \expandafter\lst@ReplaceIn@\expandafter#1%
%    \fi}\end{verbatim}
% \removelastskip
% Even here you will have a problem replacing a single brace, if it has the meaning of opening or closing a group:
% You can't enter it as an argument!
% \end{TODO}
% \end{macro}
%
% \begin{macro}{\lst@DeleteKeysIn}
% We define another macro, which looks quite similar to |\lst@ReplaceIn|.
% The arguments are two macros containing a comma separated keyword list.
% All keywords in the second list are removed from the first.
%    \begin{macrocode}
\def\lst@DeleteKeysIn#1#2{%
    \bgroup \global\let\@gtempa#1%
    \let\lst@dollar=A\let\lst@minus=B\let\lst@underscore=C%
    \expandafter\lst@DeleteKeysIn@#2,\relax,%
    \egroup \let#1\@gtempa}
%    \end{macrocode}
% To terminate the loop we remove the very last |\lst@DeleteKeysIn@| with |\@gobble|.
%    \begin{macrocode}
\def\lst@DeleteKeysIn@#1,{%
    \ifx\relax#1\@empty %
        \expandafter\@gobble %
    \else %
        \ifx\@empty#1\@empty\else %
            \def\lst@temp##1,#1##2{##1\ifx\@empty##2\else %
                \ifx,##2\else #1\fi \expandafter\lst@temp\fi ##2}%
            \xdef\@gtempa{%
                \expandafter\lst@temp\expandafter,\@gtempa,#1\@empty}%
        \fi %
    \fi %
    \lst@DeleteKeysIn@}
%    \end{macrocode}
% \end{macro}
% \begingroup
%    \begin{macrocode}
%</kernel>
%    \end{macrocode}
% \endgroup
%
%
% \subsection{Looking ahead for character sequences}
%
% \begingroup
%    \begin{macrocode}
%<*kernel>
%    \end{macrocode}
% \endgroup
% \begin{macro}{\lst@IfNextChars}
% The macro |\@ifnextchar|\meta{single character}|{|\meta{then part}|}{|\meta{else part}|}| from the \LaTeX{} kernel is well known:
% Whether or not the character behind the three arguments --- usually a character from the 'user input stream' --- equals the given single character the 'then' or 'else' part is executed.
% We define a macro which looks for an arbitrary character sequence stored in a macro:{}
% \begin{macrosyntax}
% \item |\lst@IfNextChars|\meta{macro}|{|\meta{then part}|}{|\meta{else part}|}|
% \end{macrosyntax}
% Note an important difference:
% \LaTeX's |\@ifnextchar| doesn't remove the character behind the arguments, but we remove the characters until it is possible to decide whether the 'then' or 'else' part must be executed.
% However, we save these characters in a macro called |\lst@eaten|, so they can be inserted if necessary.
%
% We save the arguments and call the macro which does the comparisons.
%    \begin{macrocode}
\def\lst@IfNextChars#1#2#3{%
    \let\lst@tofind#1\def\@tempa{#2}\def\@tempb{#3}%
    \let\lst@eaten\@empty \lst@IfNextChars@}
%    \end{macrocode}
% This macro reads the next character from the input and compares it with the next character from |\lst@tofind|.
% We append |#1| to the eaten characters and get the character we are looking for (via |\lst@IfNextChars@@|).
%    \begin{macrocode}
\def\lst@IfNextChars@#1{%
    \lst@lAddTo\lst@eaten{#1}%
    \expandafter\lst@IfNextChars@@\lst@tofind\relax %
    \ifx #1\lst@temp %
%    \end{macrocode}
% If the characters are the same, we either execute |\@tempa| or continue the test.
%    \begin{macrocode}
        \ifx\lst@tofind\@empty %
            \let\lst@next\@tempa %
        \else %
            \let\lst@next\lst@IfNextChars@ %
        \fi %
        \expandafter\lst@next %
    \else %
%    \end{macrocode}
% If the characters are different, we have to call |\@tempb|.
%    \begin{macrocode}
        \expandafter\@tempb %
    \fi}
%    \end{macrocode}
% Finally comes the subsubmacro |\lst@IfNextChars@@| used above, which assigns the next character to |\lst@temp| and the rest upto |\relax| to |\lst@tofind|.
% If the implementation is not clear, read the last sentence again.
%    \begin{macrocode}
\def\lst@IfNextChars@@#1#2\relax{\let\lst@temp#1\def\lst@tofind{#2}}
%    \end{macrocode}
% \end{macro}
%
% \begin{macro}{\lst@IfNextCharsArg}
% The difference between |\lst@IfNextChars| and the macro here is that the first parameter is a character sequence and not a macro.
% Moreover, the character sequence is made active here.
%    \begin{macrocode}
\def\lst@IfNextCharsArg#1#2#3{%
    \lst@MakeActive{#1}\let\lst@tofind\lst@arg %
    \def\@tempa{#2}\def\@tempb{#3}%
    \let\lst@eaten\@empty \lst@IfNextChars@}
%    \end{macrocode}
% \end{macro}
% \begingroup
%    \begin{macrocode}
%</kernel>
%    \end{macrocode}
% \endgroup
%
%
% \subsection{Catcode changes of characters already read}\label{iCatcodeChangesOfCharactersAlreadyRead}
%
% \begingroup
%    \begin{macrocode}
%<*kernel>
%    \end{macrocode}
% \endgroup
% Here we define two important macros:
% \begin{macrosyntax}
% \item	|\lst@MakeActive{|\meta{character sequence}|}|
%
%	stores the character sequence in |\lst@arg|, but all characters become active.
%	The string must \emph{not} contain a begin group, end group or escape character (|{}\|); it may contain a left brace, right brace or backslash with other meaning (=catcodes).
%	This command would be quite surplus if \meta{character sequence} is not already read by \TeX{} (since such catcodes can be changed easily).
%	It is explicitly allowed that the charcaters have been read, e.g.\ in |\def\test{\lst@MakeActive{ABC}}| |\test|!
%
%	Note that |\lst@MakeActive| changes |\lccode|s 0--9 without restoring them.
% \item	|\lst@IfNextCharActive{|\meta{then part}|}{|\meta{else part}|}|
%
%	executes 'then' part if next character behind the arguments is active, and the 'else' part otherwise.
% \end{macrosyntax}
% \TeX{} knows sixteen different catcodes, which say whether a character is a letter, a space, a math shift character, subscript character, and so on.
% A character gets its catcode right after reading it and \TeX{} has no primitive command to change a catcode of characters already read.
% Consider for example |\def\mathmode#1{$#1$}|.
% After that definition there is no chance to say: ''\emph{Print} the two dollar signs and the argument between instead of entering math mode''.
% And if we write |\mathmode{a_i}| and have just entered the macro |\mathmode|, the argument |a_i| is read and it's too late to change the meaning of the subscript character to a printable underbar, for example.
% But that's not the whole truth: We can change character-catcodes of an argument.
% In fact, we replace the characters by characters with same ASCII codes but different catcodes.
% It's not the same but suffices since the result is the same.
% Here we treat the very special case that all characters become active.
% A prototype macro would be
% \begin{verbatim}
%     \def\MakeActive#1{\lccode`\~=`#1\lowercase{\def\lst@arg{~}}}|\end{verbatim}
% But this macro handles a single character only:
% The |\lowercase| changes the ASCII code of |~| to that of |#1| since we have said that |~| is the lower case version of |#1|.
% Fortunately the |\lowercase| doesn't change the catcode, so we have an active version of |#1|.
% Note that |~| is usually active.
%
% \begin{macro}{\lst@MakeActive}
% We won't do this character by character.
% To increase speed we change nine characters at the same time (if nine characters are left).
% We get the argument, empty |\lst@arg| and begin a loop:
%    \begin{macrocode}
\def\lst@MakeActive#1{%
    \let\lst@arg\@empty \lst@MakeActive@#1%
    \relax\relax\relax\relax\relax\relax\relax\relax\relax}
%    \end{macrocode}
% There are nine |\relax|es since |\lst@MakeActive@| has nine parameters and we don't want any problems in the case that |#1| is empty.
% We need nine active characters now instead of a single |~|.
% We make these catcode changes local and define the coming macro |\global|.
%    \begin{macrocode}
\begingroup
\catcode`\^^@=\active \catcode`\^^A=\active \catcode`\^^B=\active %
\catcode`\^^C=\active \catcode`\^^D=\active \catcode`\^^E=\active %
\catcode`\^^F=\active \catcode`\^^G=\active \catcode`\^^H=\active %
%    \end{macrocode}
% First we |\let| the next operation be |\relax|.
% This aborts our loop for processing all characters (default and possibly changed later).
% Then we look if we have at least one character.
% If this is not the case, the loop terminates and all is done.
%    \begin{macrocode}
\gdef\lst@MakeActive@#1#2#3#4#5#6#7#8#9{\let\lst@next\relax %
    \ifx#1\relax %
    \else \lccode`\^^@=`#1%
%    \end{macrocode}
% Otherwise we say that |^^@|=chr(0) is the lower case version of the first character.
% Then we test the second character.
% If there is none, we append the lower case |^^@| to |\lst@arg|.
% Otherwise we say that |^^A|=chr(1) is the lower case version of the second character and we test the next argument, and so on.
%    \begin{macrocode}
    \ifx#2\relax %
        \lowercase{\lst@lAddTo\lst@arg{^^@}}%
    \else \lccode`\^^A=`#2%
    \ifx#3\relax %
        \lowercase{\lst@lAddTo\lst@arg{^^@^^A}}%
    \else \lccode`\^^B=`#3%
    \ifx#4\relax %
        \lowercase{\lst@lAddTo\lst@arg{^^@^^A^^B}}%
    \else \lccode`\^^C=`#4%
    \ifx#5\relax %
        \lowercase{\lst@lAddTo\lst@arg{^^@^^A^^B^^C}}%
    \else \lccode`\^^D=`#5%
    \ifx#6\relax %
        \lowercase{\lst@lAddTo\lst@arg{^^@^^A^^B^^C^^D}}%
    \else \lccode`\^^E=`#6%
    \ifx#7\relax %
        \lowercase{\lst@lAddTo\lst@arg{^^@^^A^^B^^C^^D^^E}}%
    \else \lccode`\^^F=`#7%
    \ifx#8\relax %
        \lowercase{\lst@lAddTo\lst@arg{^^@^^A^^B^^C^^D^^E^^F}}%
    \else \lccode`\^^G=`#8%
    \ifx#9\relax %
        \lowercase{\lst@lAddTo\lst@arg{^^@^^A^^B^^C^^D^^E^^F^^G}}%
%    \end{macrocode}
% If nine characters are present, we append (lower case versions of) nine active characters and call this macro again via redefining |\lst@next|.
%    \begin{macrocode}
    \else \lccode`\^^H=`#9%
        \lowercase{\lst@lAddTo\lst@arg{^^@^^A^^B^^C^^D^^E^^F^^G^^H}}%
        \let\lst@next\lst@MakeActive@ %
    \fi \fi \fi \fi \fi \fi \fi \fi \fi %
    \lst@next}
\endgroup
%    \end{macrocode}
% This |\endgroup| restores the catcodes of chr(0)--chr(8), but not the catcodes of the characters inside |\lst@MakeActive@| since they are already read.
% \end{macro}
%
% \begin{macro}{\lst@IfNextCharActive}
% The implementation is easy now: We compare the character |#3| with its active version |\lowercase{~}|.
% Note that the right brace between |\ifx~| and |#3| ends the |\lowercase|.
% The |\egroup| before restores the |\lccode|.
%    \begin{macrocode}
\def\lst@IfNextCharActive#1#2#3{%
    \bgroup \lccode`\~=`#3\lowercase{\egroup %
    \ifx~}#3%
        \def\lst@next{#1#3}%
    \else %
        \def\lst@next{#2#3}%
    \fi \lst@next}
%    \end{macrocode}
% \end{macro}
% \begingroup
%    \begin{macrocode}
%</kernel>
%    \end{macrocode}
% \endgroup
%
%
% \subsection{An application to \ref{iCatcodeChangesOfCharactersAlreadyRead}}\label{iAnApplicationTo}
%
% \begingroup
%    \begin{macrocode}
%<*kernel>
%    \end{macrocode}
% \endgroup
% We need a brief look on how the listing processing works.
% Right before processing a listing we redefine all characters, for example to preserve successive spaces or to give the tabulator a different meaning.
% More precisely, each character becomes active in the sense of \TeX, but that's not so important now.
% After the listing we switch back to the original meanings and all is fine.
% If an environment is used inside an argument the listing is already read when the environment is executed and we can do nothing to preserve the characters.
% However, (under certain circumstances) the environment can be used inside an argument --- that's at least what I've said in the user's guide.
% And now we have to work for it coming true.
% We define the macro
% \begin{macrosyntax}
% \item	|\lstenv@AddArg{|\meta{\TeX{} material (already read)}|}|
% \end{macrosyntax}
% which \emph{appends} a 'verbatim' version of the argument to |\lstenv@arg|, but all appended characters are active.
% Since it's not a character to character conversion, 'verbatim' needs to be explained.
% All characters can be typed in as they are, except |\|, |{|, |}| and |%|.
% If you want one of these, you must write |\\|, |\{|, |\}| or |\%| instead.
% If two spaces should follow each other, the second (third, fourth, \ldots) space must be entered as |\|\textvisiblespace.
%
% \begin{macro}{\lstenv@AddArg}
% We call a submacro (similar to |\zap@space|) to preserve single spaces.
%    \begin{macrocode}
\def\lstenv@AddArg#1{\lstenv@AddArg@#1 \@empty}
%    \end{macrocode}
% We will need an active space:
%    \begin{macrocode}
\begingroup \lccode`\~=`\ \relax \lowercase{%
%    \end{macrocode}
% We make all characters upto the first space (with catcode 10) active and append these (plus an active space) to |\lstenv@arg|.
% If we haven't found the end |\@empty| of the input, we continue the process.
%    \begin{macrocode}
\gdef\lstenv@AddArg@#1 #2{%
    \lst@MakeActive{#1}%
    \ifx\@empty#2\expandafter%
        \lst@lAddTo\expandafter\lstenv@arg\expandafter{\lst@arg}%
    \else \expandafter%
        \lst@lAddTo\expandafter\lstenv@arg\expandafter{\lst@arg~}%
        \expandafter\lstenv@AddArg@ %
    \fi #2}
%    \end{macrocode}
% Finally we end the |\lowercase| and close a group.
%    \begin{macrocode}
}\endgroup
%    \end{macrocode}
% \end{macro}
% \begingroup
%    \begin{macrocode}
%</kernel>
%    \end{macrocode}
% \endgroup
%
%
% \subsection{How to define \lst-aspects}\label{iHowToDefineLstAspects}
%
% \begingroup
%    \begin{macrocode}
%<*kernel>
%    \end{macrocode}
% \endgroup
% This section contains commands used in defining the style and language aspects.
% There are two command classes: one extends the internal capabilities, and the other actually defines the aspects.
% \begin{macrosyntax}
% \item	|\lst@Aspect{|\meta{aspect name}|}{|\meta{definition}|}|
%
%		is the one and only command in the latter class.
%		It defines the aspect using the \textsf{keyval} package.
%
% \item	|\lst@AddToHook||{|\meta{name of hook}|}{|\meta{\TeX{} material}|}|
%
% \item	|\lst@AddToHookAtTop||{|\meta{name of hook}|}{|\meta{\TeX{} material}|}|
%
%		Both add \TeX{} material at predefined points.
%		The first command appends the code, whereas the second places it in front of existing hook material.
% \end{macrosyntax}
% The following hooks are supported (possibly not up to date):
% \begin{center}\begin{tabular}{rp{0.6\linewidth}}
%	name of hook & point/aim of execution\\ \hline
%	|InitVars| & initializes variables each listing\\
%	|ExitVars| & deinits variables each listing\\
%	|OnExit| & executed at the very end of typesetting\\
%	|EveryLine| & called at the beginning of each line\\
%	|EOL| & executed after printing a source code line\\
%	|InitVarsEOL| & prepares variables for the next line\\
%	|EndGroup| & executed whenever the package closes a group (end of comment or string, e.g.)\\
%	|Output| & called before a printing unit with letters and digits is typeset\\
%	|OutputOther| & called before any other printing unit is typeset\\
%	|BeforeSelectCharTable| & executed before the package changes the character table\\
%	|SelectCharTable| & executed after the package has selected the standard character table\\
%	&\\
%	|SetStyle| & called before |\lststy@|\meta{style}\\
%	|SetLanguage| & called before |\lstlang@|\meta{language}|@|\meta{dialect}\\
% \end{tabular}\end{center}
% For example, the hooks make keywords case insensitive (if necessary) before the package processes a listing and call the keyword style before a keyword is printed.
%
% \begin{macro}{\lst@Aspect}
% The command defines the aspect only if not already present.
% |\lst@ifnew| becomes true (if and) only if a new aspect is defined.
%    \begin{macrocode}
\def\lst@Aspect#1{%
    \@ifundefined{KV@lst@#1}%
        {\let\lst@ifnew\iftrue}{\let\lst@ifnew\iffalse}%
    \lst@ifnew %
%<info>        \lst@InfoInfo{New aspect `#1'}%
    \else %
%<info>        \lst@InfoWarning{Gobble aspect `#1'}%
        \expandafter\@gobblefour %
    \fi %
    \lstdefine@key{lst}{#1}}
%    \end{macrocode}
% Now come renamed copies from two \textsf{keyval} macros, but the key definitions are made globally.
%    \begin{macrocode}
\def\lstdefine@key#1#2{%
    \@ifnextchar[{\lstKV@def{#1}{#2}}{\global\@namedef{KV@#1@#2}####1}}
\def\lstKV@def#1#2[#3]{%
    \global\@namedef{KV@#1@#2@default\expandafter}\expandafter %
        {\csname KV@#1@#2\endcsname{#3}}%
    \global\@namedef{KV@#1@#2}##1}
%    \end{macrocode}
%    \begin{macrocode}
\let\lst@ifnew\iftrue % init
%    \end{macrocode}
% \end{macro}
%
% \begin{macro}{\lst@AddToHook}
% \begin{macro}{\lst@AddToHookAtTop}
% The definitions are mainly in terms of |\lst@AddTo| and |\lst@AddToAtTop|.
% But we test whether the hook already exists.
%    \begin{macrocode}
\def\lst@AddToHook#1{%
    \lst@ifnew %
        \@ifundefined{lsthk@#1}{%
%<info>            \lst@InfoInfo{New hook `#1'}%
            \expandafter\gdef\csname lsthk@#1\endcsname{}}{}%
        \def\lst@next{\lst@AddToHook@\lst@AddTo{#1}}%
        \expandafter\lst@next %
    \else %
        \expandafter\@gobble %
    \fi}
%    \end{macrocode}
% The submacro makes it possible to use \texttt{if}s inside the second argument (which is the third here).
%    \begin{macrocode}
\long\def\lst@AddToHook@#1#2#3{%
    \expandafter#1\csname lsthk@#2\endcsname{#3}}
%    \end{macrocode}
% \begin{TODO}
% The usage of |\lst@ifnew| is still unsatisfactory:
% Defining an aspect and its hooks twice doesn't add the \TeX\ material a second time, but defining hooks and then the aspect would do so.
% \end{TODO}
%    \begin{macrocode}
\def\lst@AddToHookAtTop#1{%
    \lst@ifnew %
        \@ifundefined{lsthk@#1}{%
%<info>            \lst@InfoInfo{New hook `#1'}%
            \expandafter\gdef\csname lsthk@#1\endcsname{}}{}%
        \def\lst@next{\lst@AddToHook@\lst@AddToAtTop{#1}}%
        \expandafter\lst@next %
    \else %
        \expandafter\@gobble %
    \fi}
%    \end{macrocode}
% \end{macro}\end{macro}
%
% \begin{macro}{\lst@AddTo}
% \begin{macro}{\lst@AddToAtTop}
% These macros add the second argument to the macro |#1|.
% But the definition is global!
%    \begin{macrocode}
\long\def\lst@AddTo#1#2{%
    \expandafter\gdef\expandafter#1\expandafter{#1#2}}
%    \end{macrocode}
% We need a couple of |\expandafter|s now.
% Simply note that we have\\
%	{\small\hspace*{2em}
%	|\expandafter\gdef\expandafter#1\expandafter{\lst@temp|$\langle$\textit{contents of }|#1|$\rangle$|}|
%	}\\
% after the first phase of expansion.
%    \begin{macrocode}
\def\lst@AddToAtTop#1#2{\def\lst@temp{#2}%
    \expandafter\expandafter\expandafter\gdef %
    \expandafter\expandafter\expandafter#1%
    \expandafter\expandafter\expandafter{\expandafter\lst@temp#1}}
%    \end{macrocode}
% \end{macro}\end{macro}
%
% \begin{macro}{\lst@lAddTo}
% A local version of |\lst@AddTo|.
%    \begin{macrocode}
\def\lst@lAddTo#1#2{\expandafter\def\expandafter#1\expandafter{#1#2}}
%    \end{macrocode}
% \end{macro}
%
% \begingroup
% \noindent Since some hooks are unused and never defined by the hook macros above, we define them here:
%    \begin{macrocode}
\global\let\lsthk@OutputOther\@empty
\global\let\lsthk@SetStyle\@empty
\global\let\lsthk@SetLanguage\@empty
\global\let\lsthk@PreSet\@empty
\global\let\lsthk@PostSet\@empty
%    \end{macrocode}
% \endgroup
% \begingroup
%    \begin{macrocode}
%</kernel>
%    \end{macrocode}
% \endgroup
%
%
% \subsection{Interfacing with \textsf{keyval}}
%
% \begingroup
%    \begin{macrocode}
%<*kernel>
%    \end{macrocode}
% \endgroup
% The \textsf{keyval} package passes the value via the one and only paramater |#1| to the definition part of the aspect macro.
% The following commands may be used to analyse the value.
% \begin{macrosyntax}
% \item |\lstKV@SetIfKey|\meta{macro}|{#1}|
%
%		\meta{macro} becomes true (more precisely |\iftrue|) if the first character of |#1| equals |t| or |T|.
%		It becomes false otherwise.
%
% \item |\lstKV@OptArg|\meta{submacro}|[|\meta{default arg.}|]{#1}|
%
%	calls \meta{submacro} with |#1| and with inserted |[|\meta{default arg.}|]| if |#1| has no optional argument.
% \end{macrosyntax}
%
% \begin{macro}{\lstKV@SetIfKey}
% We simply test the first character of |#2|.
%    \begin{macrocode}
\def\lstKV@SetIfKey#1#2{\lstKV@SetIfKey@#1#2\relax}
\def\lstKV@SetIfKey@#1#2#3\relax{\lowercase{%
    \expandafter\let\expandafter#1%
        \csname \ifx #2t}iftrue\else iffalse\fi\endcsname}
%    \end{macrocode}
% \end{macro}
%
% \begin{macro}{\lstKV@OptArg}
% We read the arguments and insert default if necesary, what else?
%    \begin{macrocode}
\def\lstKV@OptArg#1[#2]#3{\lstKV@OptArg@{#1}{#2}#3\@}
\def\lstKV@OptArg@#1#2{%
    \@ifnextchar[{\lstKV@OptArg@@{#1}}%
                 {\lstKV@OptArg@@{#1}[#2]}}
\def\lstKV@OptArg@@#1[#2]#3\@{#1[#2]{#3}}
%    \end{macrocode}
% \end{macro}
%
% \begin{macro}{\lstset}
% Finally this main user interface macro:
%    \begin{macrocode}
\def\lstset#1{\lsthk@PreSet\setkeys{lst}{#1}\lsthk@PostSet}
%    \end{macrocode}
% \end{macro}
% \begingroup
%    \begin{macrocode}
%</kernel>
%    \end{macrocode}
% \endgroup
%
%
% \subsection{Internal modes}\label{iInternalModes}
%
% \begingroup
%    \begin{macrocode}
%<*kernel>
%    \end{macrocode}
% \endgroup
% \begin{macro}{\lst@mode}
% \begin{macro}{\lst@EnterMode}
% \begin{macro}{\lst@LeaveMode}
% While typesetting a listing we will enter 'modes' to distinguish comments from strings, comment lines from single comments, \TeX\ comment lines from fixed comment lines, and so on.
% The counter |\lst@mode| keeps the current mode number.
%    \begin{macrocode}
\newcount\lst@mode
%    \end{macrocode}
% Each mode opens a group level, stores the mode number and execute mode specific tokens.
% Moreover we keep all these changes in mind (locally).
%    \begin{macrocode}
\def\lst@EnterMode#1#2{%
    \bgroup \lst@mode=#1\relax #2%
    \lst@lAddTo\lst@entermodes{\lst@EnterMode{#1}{#2}}}
%    \end{macrocode}
% Leaving a mode simply closes the group and calls the hook |\lsthk@EndGroup|.
%    \begin{macrocode}
\def\lst@LeaveMode{\egroup\lsthk@EndGroup}
%    \end{macrocode}
%    \begin{macrocode}
\lst@AddToHook{InitVars}{\let\lst@entermodes\@empty}
%    \end{macrocode}
% \end{macro}\end{macro}\end{macro}
%
% \begin{macro}{\lst@NewMode}
% defines a new mode number.
% We simply use |\mathchardef| and advance the number in |\lst@newmode|, which is a macro --- we don't waste another counter.
%    \begin{macrocode}
\def\lst@NewMode#1{%
    \ifx\@undefined#1%
        \global\mathchardef#1=\lst@newmode\relax %
        \lst@mode=\lst@newmode\relax \advance\lst@mode\@ne %
        \edef\lst@newmode{\the\lst@mode}%
    \fi}
%    \end{macrocode}
% We start with number 0.
% The negative range is reserved for nested comments.
%    \begin{macrocode}
\def\lst@newmode{0}
%    \end{macrocode}
% \end{macro}
%
% \begin{macro}{\lst@nomode}
% \begin{macro}{\lst@Pmode}
% The very first mode initializes |\lst@mode| (every listing).
% The other mode indicates active listing processing.
%    \begin{macrocode}
\lst@NewMode\lst@nomode
\lst@AddToHook{InitVars}{\lst@modefalse \lst@mode\lst@nomode}
\lst@NewMode\lst@Pmode
%    \end{macrocode}
% \end{macro}\end{macro}
%
% \begin{macro}{\lst@modetrue}
% \begin{macro}{\lst@modefalse}
% If |\lst@ifmode| is true, no mode change is allowed --- except leaving the mode.
%    \begin{macrocode}
\def\lst@modetrue{\let\lst@ifmode\iftrue}
\def\lst@modefalse{\let\lst@ifmode\iffalse}
%    \end{macrocode}
% Note: |\lst@ifmode| is \emph{not} obsolete since |\lst@mode|$\neq$|\lst@nomode| does \emph{not} imply |\lst@ifmode|$=$|\iftrue|.
% And even the other implication is not true.
% It will happen that we enter a mode without setting |\lst@ifmode| true, and we'll set it true without assigning any mode!
% \end{macro}\end{macro}
%
% \begin{macro}{\lst@InterruptModes}
% \begin{macro}{\lst@ReenterModes}
% Since we kept all mode changes in mind, it is possible two interrupt and re-enter the current mode.
% This is useful for \TeX\ comment lines, for example.
%    \begin{macrocode}
\def\lst@InterruptModes{%
    \ifx\lst@entermodes\@empty\else %
        \global\let\lst@reenter\lst@entermodes %
        \lst@LeaveAllModes %
    \fi}
%    \end{macrocode}
% The last |\ifx| is not necessay since empty |\lst@entermodes| implies that we don't leave any mode.
% But we need the |\ifx| in the following macro since empty |\lst@reenter| doesn't tell us anything about the current mode.
%    \begin{macrocode}
\def\lst@ReenterModes{%
    \ifx\lst@reenter\@empty\else %
        \lst@LeaveAllModes \lst@reenter %
        \global\let\lst@reenter\@empty %
    \fi}
%    \end{macrocode}
%    \begin{macrocode}
\lst@AddToHook{InitVars}{\global\let\lst@reenter\@empty}
%    \end{macrocode}
% Note that these macros can't be nested:
% If you 'interrupt modes', enter some modes and say 'interrupt modes' again, then two 're-enter modes' will \emph{not} take you back in front of the first 'interrupt modes'.
% \end{macro}\end{macro}
%
% \begin{macro}{\lst@LeaveAllModes}
% This is some kind of emergency macro.
% Since any mode is terminated by closing a group, leaving all modes means closing groups until the mode equals |\lst@nomode|.
% We'll need that macro to end a listing correctly.
%    \begin{macrocode}
\def\lst@LeaveAllModes{%
    \ifnum\lst@mode=\lst@nomode %
        \expandafter\lsthk@EndGroup %
    \else %
        \expandafter\egroup\expandafter\lst@LeaveAllModes %
    \fi}
\lst@AddToHook{ExitVars}{\lst@LeaveAllModes}
%    \end{macrocode}
% \end{macro}
%
% \begin{macro}{\lst@egroupmode}
% We allocate a 'processing mode' and define a general purpose 'egroup' mode: close a group at end of line.
%    \begin{macrocode}
\lst@NewMode\lst@egroupmode
\lst@AddToHook{EOL}
    {\ifnum\lst@mode=\lst@egroupmode \lst@LeaveMode \fi}
%    \end{macrocode}
% \end{macro}
% \begingroup
%    \begin{macrocode}
%</kernel>
%    \end{macrocode}
% \endgroup
%
%
% \subsection{Styles and languages}
%
% \begingroup
%    \begin{macrocode}
%<*kernel>
%    \end{macrocode}
% \endgroup
% \begin{macro}{\lst@Normalize}
% normalizes the contents of the macro |#1| by making all characters lower case and stripping off spaces.
%    \begin{macrocode}
\def\lst@Normalize#1{%
    \lowercase\expandafter{\expandafter\edef\expandafter#1%
    \expandafter{\expandafter\zap@space#1 \@empty}}}
%    \end{macrocode}
% \end{macro}
%
% \begin{macro}{\lstdefinestyle}
% The internal macro name of a style is |\lststy@|\emph{the style} (normalized).
% If called, it simply sets the keys to given values.
%    \begin{macrocode}
\newcommand\lstdefinestyle[2]{%
    \def\lst@temp{lststy@#1}\lst@Normalize\lst@temp %
    \global\@namedef\lst@temp{\setkeys{lst}{#2}}}
%    \end{macrocode}
% \end{macro}
%
% \begin{aspect}{style}
% We give an error message if the necessary macro doesn't exist.
%    \begin{macrocode}
\lst@Aspect{style}
    {\def\lst@temp{lststy@#1}\lst@Normalize\lst@temp %
     \@ifundefined\lst@temp %
        {\PackageError{Listings}{Style `#1' undefined}%
         {You might have misspelt the name here or before.^^J%
          Type <RETURN> to proceed without changing the style.}}%
%    \end{macrocode}
% Otherwise we execute the hook and call the macro.
%    \begin{macrocode}
        {\lsthk@SetStyle \csname\lst@temp\endcsname %
%<info>         \lst@InfoInfo{\expandafter\string %
%<info>             \csname\lst@temp\endcsname\space called}%
        }}
%    \end{macrocode}
% \begin{TODO}
% It's easy to crash the listings package with |style| (and also with |language|).
% Define |\lstdefinestyle{crash}{style=crash}| and write |\lstset{style=crash}|.
% \TeX's capacity will exceed (parameter stack size), sorry.
% Only bad girls use such recursive calls, but only good girls use this package.
% Thus this problem is of minor interest.
% \end{TODO}
% \end{aspect}
%
% \begin{macro}{\lstdefinelanguage}
% Nearly |\lstdefinestyle|:
%    \begin{macrocode}
\newcommand\lstdefinelanguage[3][]{%
    \def\@tempa{#2@#1}\lst@Normalize\@tempa %
    \global\@namedef{lstlang@\@tempa}{\lstset{#3}}}
%    \end{macrocode}
% \end{macro}
%
% \begin{macro}{\lstdefinedrvlanguage}
% The driver file \texttt{lstpascal.sty} contains a couple of Pascal dialects.
% It is unnecessary to hold all these dialects in memory if the user selects Standard Pascal only.
% There are different scopes:
% \begin{itemize}
% \item[--]	All code outside any (style or language) definition is executed.
%		Definitions should be |\global| since a driver file is input inside a group.
%		Note also that |@| is a letter and |"| has catcode 12 (other).
% \item[--]	All languages declared with |\lstdefinelanguage| are defined.
% \item[--] A language declared with |\lstdefinedrvlanguage| is defined if and only if the user has requested that language.
% \end{itemize}
% We either do the definition or drop it:
%    \begin{macrocode}
\newcommand\lstdefinedrvlanguage[3][]{%
    \def\@tempa{lstlang@#2@#1}\lst@Normalize\@tempa %
    \ifx\@tempa\lst@requested %
        \global\@namedef{\@tempa}{\lstset{#3}}%
    \fi}
%    \end{macrocode}
% \begin{TODO}
% The command syntax is liable to changes.
% I conjecture that it will be something like the following.
% \begin{macrosyntax}
% \item[0.2] |\lstdefinedrvlanguage|[|[|\meta{dialect}|]|]|{|\meta{language}|}|\\
%		\qquad[|[|\meta{base dialect}|]|]|{|\meta{base language (e.g.\ empty)}|}|\\
%		\qquad|{|\meta{key=value list}|}|\\
%		\qquad[|[|\meta{list of extras (keywordcomments,texcs,etc.)}|]|]
% \end{macrosyntax}
% To make 'extras' user assessible (without loading a language) a command |\lstloadextras| would be nice.
% In any case such a macro is needed to load the extras for the supported languages.
% All in all there should be the following files in future: The kernel |listings.sty|, the language definitions in |lstdrvrs.sty| (or |lstdef.sty|?) and extras in a file |lstextra.sty|.
% \end{TODO}
% \end{macro}
%
% \begin{macro}{\lstloadlanguages}
% We iterate through the list and locate and load each language.
%    \begin{macrocode}
\def\lstloadlanguages#1{\lstloadlanguages@#1,\relax,}
\def\lstloadlanguages@#1,{%
    \ifx\relax#1\@empty \else %
        \lstKV@OptArg\lstloadlanguages@@[]{#1}%
        \expandafter\lstloadlanguages@ %
    \fi}
\def\lstloadlanguages@@[#1]#2{%
    \lst@LocateLanguage[#1]{#2}%
    \@ifundefined{\lst@requested}%
        {{\catcode`\^^M=9\catcode`\"=12\makeatletter %
          \input{\lst@driver.sty}}}%
        {}}
\@onlypreamble\lstloadlanguages
%    \end{macrocode}
% \end{macro}
%
% \begin{aspect}{language}
% If the language macro |\lstlang@|\emph{language}|@|\emph{dialect} doesn't exist, we must load the driver file.
% In that case the character |@| becomes a letter and we change the catcode of the double quote for compatibility with \texttt{german.sty}.
% Moreover we make the EOL character being ignored (which removes unwanted spaces if we've forgotten |%| at end of line).
%    \begin{macrocode}
\lst@Aspect{language}{\lstKV@OptArg\lstlanguage@[]{#1}}
\def\lstlanguage@[#1]#2{%
    \lst@LocateLanguage[#1]{#2}%
    \@ifundefined{\lst@requested}%
        {{\catcode`\^^M=9\catcode`\"=12\makeatletter %
          \input{\lst@driver.sty}}}{}%
%    \end{macrocode}
% We give an error message, if the language/dialect is undefined now.
%    \begin{macrocode}
    \@ifundefined{\lst@requested}%
        {\PackageError{Listings}%
         {\ifx\@empty\lst@dialect@\else \lst@dialect@\space of \fi %
          \lst@language@\space undefined}{The driver file is not
          loadable or doesn't support the language.^^J%
         Type <RETURN> to proceed without changing the language.}}%
%    \end{macrocode}
% Otherwise we execute the hook and select the language.
%    \begin{macrocode}
        {\lsthk@SetLanguage %
%<info>         \lst@InfoInfo{\expandafter\string %
%<info>             \csname\lst@requested\endcsname\space called}%
         \csname\lst@requested\endcsname %
         \def\lst@language{#1}\lst@Normalize\lst@language %
         \def\lst@dialect{#2}\lst@Normalize\lst@dialect}}
%    \end{macrocode}
% \end{aspect}
%
% \begin{aspect}{defaultdialect}
% We simply store the dialect.
%    \begin{macrocode}
\lst@Aspect{defaultdialect}{\lstKV@OptArg\lstdefaultdialect@[]{#1}}
\def\lstdefaultdialect@[#1]#2{%
    \def\lst@temp{#2}\lst@Normalize\lst@temp %
    \global\@namedef{lstdd@\lst@temp}{#1}}
%    \end{macrocode}
% \end{aspect}
%
% Languages might have a user name and a different internal name.
% Moreover we don't always use the standard driver file |lst|\meta{language}|.sty|.
% If the user writes |language=[fuu]foo|, this could mean that we select (internally) dialect |faa| of language |fee|, which is located in driver file |fii.sty|.
% We define the following macros:
% \begin{macrosyntax}
% \item |\lst@DriverLocation{|\meta{language}|}{|\meta{file name without extension}|}|
%
%	Afterwards the package inputs the given driver file to load the language.
%
% \item	|\lst@LocateLanguage[|\meta{dialect}|]{|\meta{language}|}|
%
%	returns the driver file name in |\lst@driver| and |\lst@requested| contains name of driver macro (i.e.\ with prefix |lstlang@|, without backslash).
%	The macro automatically chooses aliases.
% \end{macrosyntax}
% All language and dialect arguments are standardized, i.e.\ we make them free of spaces and lower case.
%
% \begin{macro}{\lstalias}
% The names are stored in |\lsta@|\emph{language}|@|\emph{dialect} and |\lstaa@|\emph{language}.
%    \begin{macrocode}
\newcommand\lstalias{\@ifnextchar[{\lstalias@}{\lstalias@@}}
%    \end{macrocode}
%    \begin{macrocode}
\def\lstalias@[#1]#2[#3]#4{%
    \def\lst@temp{lsta@#2@#1}\lst@Normalize\lst@temp %
    \global\@namedef{\lst@temp}{#4@#3}}
%    \end{macrocode}
%    \begin{macrocode}
\def\lstalias@@#1#2{%
    \def\lst@temp{lstaa@#1}\lst@Normalize\lst@temp %
    \global\@namedef{\lst@temp}{#2}}
%    \end{macrocode}
% \end{macro}
%
% \begin{macro}{\lst@DriverLocation}
% We simply define a macro containing the file name.
%    \begin{macrocode}
\def\lst@DriverLocation#1#2{%
    \def\lst@temp{lstloc@#1}\lst@Normalize\lst@temp %
    \expandafter\gdef\csname\lst@temp\endcsname{#2}%
    \expandafter\lst@Normalize\csname\lst@temp\endcsname}
%    \end{macrocode}
% \end{macro}
%
% \begin{macro}{\lst@LocateLanguage}
% First we test for a language alias, \ldots
%    \begin{macrocode}
\def\lst@LocateLanguage[#1]#2{%
    \def\lst@language@{#2}\lst@Normalize\lst@language@ %
    \@ifundefined{lstaa@\lst@language@}{}%
        {\edef\lst@language@{\csname lstaa@\lst@language@\endcsname}%
         \lst@Normalize\lst@language@}%
%    \end{macrocode}
% then we set the default dialect if necessary.
%    \begin{macrocode}
    \def\lst@dialect@{#1}%
    \ifx\@empty\lst@dialect@ %
        \@ifundefined{lstdd@\lst@language@}{}%
            {\expandafter\let\expandafter\lst@dialect@ %
             \csname lstdd@\lst@language@\endcsname}%
    \fi %
    \lst@Normalize\lst@dialect@ %
%    \end{macrocode}
% Now we are ready for an alias for a language dialect.
%    \begin{macrocode}
    \edef\lst@requested{\lst@language@ @\lst@dialect@}%
    \@ifundefined{lsta@\lst@requested}{}%
        {\expandafter\let\expandafter\lst@requested %
         \csname lsta@\lst@requested\endcsname %
         \lst@Normalize\lst@requested}%
%    \end{macrocode}
% Finally we get the driver file name and set the default dialect.
%    \begin{macrocode}
    \expandafter\lst@LocateLanguage@\lst@requested\relax}
\def\lst@LocateLanguage@#1@#2\relax{%
    \edef\lst@driver{\@ifundefined{lstloc@#1}{lst#1}%
        {\csname lstloc@#1\endcsname}}%
    \ifx\@empty#2\@empty %
        \@ifundefined{lstdd@#1}%
            {\def\lst@dialect@{#2}}%
            {\expandafter\let\expandafter\lst@dialect@ %
             \csname lstdd@#1\endcsname}%
    \fi %
    \edef\lst@requested{lstlang@#1@\lst@dialect@}%
    \lst@Normalize\lst@requested}
%    \end{macrocode}
% \end{macro}
% \begingroup
%    \begin{macrocode}
%</kernel>
%    \end{macrocode}
% \endgroup
%
%
% \section{Typesetting a listing}
%
% \begingroup
%    \begin{macrocode}
%<*kernel>
%    \end{macrocode}
% \endgroup
% \begin{macro}{\lst@lineno}
% This counter keeps the current line number.
% A register is global if and only if the allocation line shows |% \global|.
%    \begin{macrocode}
\newcount\lst@lineno \global\lst@lineno\@ne % \global
%    \end{macrocode}
% The counter is initialized and advances every line (which means |\everypar|).
%    \begin{macrocode}
\lst@AddToHook{InitVars}
    {\global\lst@lineno\@ne %
     \expandafter\everypar\expandafter{\the\everypar %
         \global\advance\lst@lineno\@ne}}
%    \end{macrocode}
% And we (try to) ensure correct line numbers for continued listings.
%    \begin{macrocode}
\lst@AddToHook{ExitVars}
    {\ifx\lst@NewLine\relax\else %
         \global\advance\lst@lineno-2\relax %
         \setbox\@tempboxa\vbox{\lst@NewLine}%
     \fi}
%    \end{macrocode}
% \begin{TODO}
% We only try to.
% If there is source code before |\end{lstlisting}|, it goes wrong: the line number is one too less.
% \end{TODO}
% \end{macro}
%
% \begin{macro}{\lst@PrintFileName}
% makes use of |\lst@ReplaceIn|:
%    \begin{macrocode}
\def\lst@PrintFileName#1{%
    \def\lst@arg{#1}%
    \lst@ReplaceIn\lst@arg{_\textunderscore $\textdollar -\textendash}%
    \lst@arg}
%    \end{macrocode}
% \end{macro}
%
% \begin{macro}{\lst@SetName}
% As proposed by \lsthelper{Boris Veytsman}{boris@plmsc.psu.edu}{1998/03/22}{listing name accessible for user} the name of a listing (file name or argument to the environment) is user accessible now.
% It is set using this macro:
%    \begin{macrocode}
\def\lst@SetName#1{%
    \gdef\lst@intname{#1}\global\let\lstintname\lst@intname %
    \let\lst@arg\lst@intname %
    \lst@ReplaceIn\lst@arg{_\textunderscore $\textdollar -\textendash}%
    \global\let\lstname\lst@arg}
%    \end{macrocode}
% \end{macro}
% \begingroup
%    \begin{macrocode}
%</kernel>
%    \end{macrocode}
% \endgroup
%
%
% \subsection{List of listings}
%
% \begingroup
%    \begin{macrocode}
%<*kernel>
%    \end{macrocode}
% \endgroup
% \begin{macro}{\listoflistings}
% Instead of imitating |\listoffigures| we make some local adjustments and call |\tableofcontents|.
% This has the advantage that redefinitions (e.g.\ without any |\MakeUppercase| inside) also take effect on the list of listings.
%    \begin{macrocode}
\newcommand\listoflistings{\bgroup %
    \let\contentsname\listlistingsname %
    \let\lst@temp\@starttoc \def\@starttoc##1{\lst@temp{lol}}%
    \tableofcontents \egroup}
%    \end{macrocode}
% \end{macro}
%
% \begin{macro}{\listlistingsname}
% Simply the header name:
%    \begin{macrocode}
\newcommand\listlistingsname{Listings}
%    \end{macrocode}
% \end{macro}
%
% \begin{macro}{\lst@AddToLOL}
% adds an entry to the list of listings.
% The first parameter is the name of the listing and the second is unused so far.
%    \begin{macrocode}
\newcommand\lst@AddToLOL[2]{%
    \ifx\@empty#1\@empty \else %
        \addtocontents{lol}{\protect\lstlolline{#1}{#2}%
            {\lst@language}{\thepage}}%
    \fi}
%    \end{macrocode}
% \end{macro}
%
% \begin{macro}{\lstlolline}
% prints one 'lol' line.
% Using |\lst@PrintFileFile| removes a bug first reported by \lsthelper{Magne Rudshaug}{magne@ife.no}{1998/01/09}{_ and list of listings}.
%    \begin{macrocode}
\newcommand\lstlolline[4]{%
    \@dottedtocline{1}{1.5em}{2.3em}{\lst@PrintFileName{#1}}{#4}}
%    \end{macrocode}
% \end{macro}
% \begingroup
%    \begin{macrocode}
%</kernel>
%    \end{macrocode}
% \endgroup
%
%
% \subsection{Init and EOL}
%
% \begingroup
%    \begin{macrocode}
%<*kernel>
%    \end{macrocode}
% \endgroup
% We need some macros to initialize registers and variables before typesetting a listing and for the update every line.
%
% \begin{macro}{\lst@Init}
% The argument |#1| (assigned at the end) is either |\relax| or |\lstenv@backslash| since the backslash has a special meaning for the environment.
% The line number is used as current label (proposed by \lsthelper{Boris Veytsman}{boris@plmsc.psu.edu}{1998/03/25}{make line numbers referenced via \label and \ref}).
% The end of line character chr(13)=|^^M| controls the processing, see the definition of |\lst@MProcessListing| below.
% The vertical space in the macro code is for clarity.
%    \begin{macrocode}
\def\lst@Init#1{%
    \begingroup \normalbaselines \smallbreak %
    \def\@currentlabel{\the\lst@lineno}%
    \lst@prelisting %
%    \end{macrocode}
%    \begin{macrocode}
    \lsthk@BeforeSelectCharTable %
    \normalbaselines \everypar{\lsthk@EveryLine}%
    \lsthk@InitVars \lsthk@InitVarsEOL %
%    \end{macrocode}
%    \begin{macrocode}
    \csname lstPre@\lst@language\endcsname %
    \csname lstPre@\lst@language @\lst@dialect\endcsname %
    \lstCC@Let{"000D}\lst@MProcessListing %
    \let\lstCC@Backslash#1%
    \lst@EnterMode{\lst@Pmode}{\lst@SelectCharTable}}
%    \end{macrocode}
% Note: From version 0.19 on 'listing processing' is implemented as an internal mode, namely a mode with special character table.
%    \begin{macrocode}
\lst@AddToHook{InitVars}
    {\rightskip\z@ \leftskip\z@ \parfillskip=0pt plus 1fil %
     \let\par\@@par}
%    \end{macrocode}
% \end{macro}
%
% \begin{macro}{\lst@DeInit}
% Here we output the remaining characters, update some variables and do some other things.
%    \begin{macrocode}
\def\lst@DeInit{%
    \lst@PrintToken \lst@EOLUpdate \par\removelastskip %
    \lsthk@ExitVars %
    \smallbreak\lst@postlisting %
    \endgroup}
%    \end{macrocode}
% \end{macro}
%
% \begin{macro}{\lst@EOLUpdate}
% This macro seems to be obsolete since in version 0.19 it degenerates to
%    \begin{macrocode}
\def\lst@EOLUpdate{\lsthk@EOL \lsthk@InitVarsEOL}%
%    \end{macrocode}
% But in future it might come in handy again.
% \end{macro}
%
% \begin{macro}{\lst@MProcessListing}
% This is what we have to do at EOL while processing a listing.
% We output all remaining characters and update the variables.
% We call |\endinput| if the next line number is greater than the last printing line.
% Finally we gobble characters to come to beginning of line.
%    \begin{macrocode}
\def\lst@MProcessListing{%
    \lst@PrintToken \lst@EOLUpdate %
    \ifnum\lst@lastline<\lst@lineno \expandafter\endinput \fi %
    \lst@BOLGobble}
%    \end{macrocode}
% \end{macro}
%
% \begin{macro}{\lst@BOLGobble}
% But this is initially |\relax|.
%    \begin{macrocode}
\let\lst@BOLGobble\relax
%    \end{macrocode}
% \end{macro}
% \begingroup
%    \begin{macrocode}
%</kernel>
%    \end{macrocode}
% \endgroup
%
%
% \subsection{The input command}
%
% \begingroup
%    \begin{macrocode}
%<*kernel>
%    \end{macrocode}
% \endgroup
% \begin{TODO}
% There is a conflict concerning |\lst@firstline|: It contains either the line number of the first code line or line number.
% This must be clarified before |lstlisting| can use the keys |first| and |last|.
% \end{TODO}
%
% \begin{aspect}{print}
% \begin{aspect}{first}
% \begin{aspect}{last}
% These aspects affect the input command only, not the environment.
%    \begin{macrocode}
\lst@Aspect{print}[t]{\lstKV@SetIfKey\lst@ifprint{#1}}
\lst@Aspect{first}{\def\lst@firstline{#1}}
\lst@Aspect{last}{\def\lst@lastline{#1}}
\lstset{print=true}% init
%    \end{macrocode}
% \end{aspect}\end{aspect}\end{aspect}
%
% \begin{macro}{\lstinputlisting}
% We define the main command.
% First we take care of the optional paramater and set it to $[1,9999999]$ if none is given.
%    \begin{macrocode}
\newcommand\lstinputlisting[2][]{%
    \begingroup %
    \def\lst@firstline{1}\def\lst@lastline{9999999}\lstset{#1}%
    \IfFileExists{#2}{\lst@InputListing{#2}}%
    {\filename@parse{#2}%
     \edef\reserved@a{\noexpand\lst@MissingFileError
         {\filename@area\filename@base}%
         {\ifx\filename@ext\relax tex\else\filename@ext\fi}}%
     \reserved@a}%
    \endgroup %
    \lsthk@OnExit}
%    \end{macrocode}
% \end{macro}
%
% \begin{macro}{\lst@MissingFileError}
% is a derivation of \LaTeX's |\@missingfileerror|:
%    \begin{macrocode}
\def\lst@MissingFileError#1#2{%
    \typeout{^^J! Package Listings Error: File `#1.#2' not found.^^J^^J%
        Type X to quit or <RETURN> to proceed,^^J%
        or enter new name. (Default extension: #2)^^J}%
    \message{Enter file name: }%
    {\endlinechar\m@ne \global\read\m@ne to\@gtempa}%
%    \end{macrocode}
% Typing |x| or |X| exits.
%    \begin{macrocode}
    \ifx\@gtempa\@empty \else %
        \def\reserved@a{x}\ifx\reserved@a\@gtempa\batchmode\@@end\fi
        \def\reserved@a{X}\ifx\reserved@a\@gtempa\batchmode\@@end\fi
%    \end{macrocode}
% In all other cases we try the new file name (with default extension).
%    \begin{macrocode}
        \filename@parse\@gtempa %
        \edef\filename@ext{%
            \ifx\filename@ext\relax#2\else\filename@ext\fi}%
        \edef\reserved@a{\noexpand\IfFileExists %
                {\filename@area\filename@base.\filename@ext}%
            {\noexpand\lst@InputListing %
                {\filename@area\filename@base.\filename@ext}}%
            {\noexpand\lst@MissingFileError
                {\filename@area\filename@base}{\filename@ext}}}%
        \expandafter\reserved@a %
    \fi}
%    \end{macrocode}
% \end{macro}
%
% \begin{macro}{\lst@InputListing}
% The one and only argument is the file name.
% We either add this name to the list of listings or the name plus a bullet to indicate that the listing has been skipped.
% Note that |\lst@Init| takes |\relax| as an argument.
%    \begin{macrocode}
\def\lst@InputListing#1{%
    \lst@SetName{#1}%
    \lst@ifprint %
        \lst@AddToLOL{#1}{}%
        \lst@Init\relax \lst@SkipUptoFirst \input{#1}\lst@DeInit %
    \else %
        \lst@AddToLOL{#1$^\bullet$}{}%
        \begin{center}%
        \footnotesize  ---  Listing of #1 has been skipped.  --- 
        \end{center}%
    \fi}
%    \end{macrocode}
% \end{macro}
%
% \begin{macro}{\lst@SkipUptoFirst}
% The end of line character either processes the listing or is responsible for skipping lines upto first printing line.
%    \begin{macrocode}
\def\lst@SkipUptoFirst{%
    \ifnum\lst@lineno=\lst@firstline\else %
%    \end{macrocode}
% To skip input lines we begin a new group level (which makes our changes local) and prohibit mode changes.
% We redefine the end of line character and all output macros becomes equivalent to |\relax|, i.e.\ nothing is typeset.
%    \begin{macrocode}
        \bgroup \lst@modetrue %
        \let\lst@Output\relax \let\lst@OutputOther\relax %
        \let\lst@GotoTabStop\relax %
        \lstCC@Let{"000D}\lst@MSkipUptoFirst %
    \fi}
%    \end{macrocode}
% \end{macro}
%
% \begin{macro}{\lst@MSkipUptoFirst}
% At the moment we use a fast and not 'everything is looking good' way.
% When |\lst@MSkipUptoFirst| is executed, one input line has already been skipped.
% We end the group opened in |\lst@SkipUptoFirst|.
% This restores the definition of the end of line character chr(13).
% Then we look whether to skip more lines or not.
%    \begin{macrocode}
\def\lst@MSkipUptoFirst{\egroup %
    \global\advance\lst@lineno\@ne %
    \ifnum\lst@lineno=\lst@firstline\else %
        \expandafter\lst@MSkipUptoFirst@ %
    \fi}
%    \end{macrocode}
% The argument of |\lst@MSkipUptoFirst@| ends with the next active chr(13), which means that the next input line is read.
% Again we look whether to skip more lines or not.
%    \begin{macrocode}
\begingroup \lccode`\~=`\^^M%
\lowercase{\gdef\lst@MSkipUptoFirst@#1~}{%
    \global\advance\lst@lineno\@ne %
    \ifnum\lst@lineno=\lst@firstline\else %
        \expandafter\lst@MSkipUptoFirst@ %
    \fi}
\endgroup
%    \end{macrocode}
% \begin{TODO}
% This definition gives rise to a ''runaway argument'' if first line doesn't exist (no |^^M| is found).
%
% We could define |\lst@MSkipUptoFirst| exactly as |\lst@MProcessListing|, except that we start the normal line processing if we reach the first printing line.
% In that case all comments and strings are processed, but not output.
% Everything looks good, even if the first printing line is in the middle of a comment.
% We need to do the following things:
%	\begin{enumerate}
%	\item Install an \texttt{if} to choose between speed and good looking, which must be noticed in |\lst@SkipUptoFirst|.
%		There we must (locally) switch to |\lsttexcloff|.
%	\item Call |\lst@BeginDropOutput{\lst@nomode}| in |\lst@SkipUptoFirst| instead of |\bgroup\lst@modetrue| and assigning |\relax|es.
%	\item Define |\lst@MSkipUptoFirstExact| by copying |\lst@MProcessListing| and renaming it.
%		Replace the |\ifnum| by\vspace*{-0.5\baselineskip}
% \begin{verbatim}
%\ifnum\lst@lineno=lst@firstline %
%    \lst@LeaveMode \global\lst@column\z@ \global\lst@pos\z@ %
%\fi\end{verbatim}
%	\item Make the fine tuning if necessary.
%	\end{enumerate}
% \end{TODO}
% \end{macro}
% \begingroup
%    \begin{macrocode}
%</kernel>
%    \end{macrocode}
% \endgroup
%
%
% \subsection{The environment}\label{iTheEnvironment}
%
% \begingroup
%    \begin{macrocode}
%<*kernel>
%    \end{macrocode}
% \endgroup
% \begin{macro}{\lst@Environment}
% This is the first attempt to provide a general macro, which defines the \lst-environments.
% The syntax comes from \LaTeX's |\newenvironment|:
% \begin{macrosyntax}
% \item	|\lst@Environment{|\meta{name}|}[|\meta{number of parameters}|][|\meta{opt.~default~arg.}|]\is|\\
%		|{|\meta{begin code}|}|\\
%		|{|\meta{end code}|}|
% \end{macrosyntax}
% Note the additional |\is|.
% I should mention that such environments could also be used in command fashion
% \begin{verbatim}
%    \lstlisting{my name}
%    Here comes the listing.
%    \endlstlisting\end{verbatim}
% But now the implementation.
% We define undefined environments only:
%    \begin{macrocode}
\def\lst@Environment#1#2\is#3#4{%
    \@ifundefined{#1}{\lst@Environment@{#1}{#2}{#3}{#4}}%
    {%
%<info>         \lst@InfoWarning{Multiple environment `#1'}%
    }}
%    \end{macrocode}
% A lonely 'end environment' produces an error:
%    \begin{macrocode}
\def\lst@Environment@#1#2#3#4{%
    \global\@namedef{end#1}{\lstenv@Error{#1}}%
%    \end{macrocode}
% The 'main' environment macro defines the environment name (for later use) and calls a submacro (getting all arguments).
% We open a group and redefine the (active) EOL character to be |\relax|.
% This ensures |\@ifnextchar[| not to read characters of the listing --- it reads the active EOL instead.
%    \begin{macrocode}
    \global\@namedef{#1}{%
        \def\lstenv@name{#1}%
        \begingroup \lstCC@Let{"000D}\relax %
        \csname#1@\endcsname}%
%    \end{macrocode}
% The submacro is defined via |\new@command|.
% We misuse |\l@ngrel@x| to make the definition |\global|.
% The submacro defines the first and last line, which are possibly changed by the user's \meta{begin code} |#3|.
%    \begin{macrocode}
    \let\l@ngrel@x\global %
    \expandafter\new@command\csname#1@\endcsname#2%
        {\def\lst@firstline{1}\def\lst@lastline{9999999}%
         #3%
%    \end{macrocode}
% The definition of the string which terminates the environment (|end{lstlisting}| or |endlstlisting|, for example) needs some care since the braces must not have catcodes 1 and 2 (or |\lst@MakeActive| fails).
% We enter them as |\{| and |\}| (with preceding |\noexpand|s).
%    \begin{macrocode}
         \ifx\@currenvir\lstenv@name %
             \edef\lst@temp{end\noexpand\{\lstenv@name\noexpand\}}%
         \else %
             \edef\lst@temp{end\lstenv@name}%
         \fi %
         \expandafter\lst@MakeActive\expandafter{\lst@temp}%
         \let\lstenv@endstring\lst@arg %
%    \end{macrocode}
% We redefine (locally) 'end environment' inside the 'begin environment' macro since ending is legal now.
% Note that the redefinition also works inside a \TeX\ comment line.
%    \begin{macrocode}
         \@namedef{end#1}{\lst@DeInit #4\endgroup \lsthk@OnExit}%
%    \end{macrocode}
% Finally the 'begin environment' macro starts the processing.
%    \begin{macrocode}
         \lstenv@Process}}
%    \end{macrocode}
% \end{macro}
%
% \begin{macro}{\lstenv@Error}
% have been used above.
%    \begin{macrocode}
\def\lstenv@Error#1{\PackageError{Listings}{Extra \string\end#1}%
    {I'm ignoring this, since I wasn't doing a \csname#1\endcsname.}}
%    \end{macrocode}
% \end{macro}
%
% \begin{macro}{\lstenv@backslash}
% We have the problem of finding end of environment, and we've already defined the 'end environment' string.
% Coming to a backslash we either end the listing or process a backslash and insert the eaten characters again.
% (Eaten means that these characters have been read (and removed) from the input to test for |\lstenv@endstring|.)
%    \begin{macrocode}
\def\lstenv@backslash{%
    \lst@IfNextChars\lstenv@endstring %
        {\lstenv@End}%
        {\lsts@backslash \lst@eaten}}%
%    \end{macrocode}
% \end{macro}
%
% \begin{macro}{\lstenv@End}
% This macro is called by the backslash macro and terminates a listing environment:
% We call the 'end environment' macro as a command or using |\end|.
%    \begin{macrocode}
\def\lstenv@End{%
    \ifx\@currenvir\lstenv@name %
        \edef\lst@next{\noexpand\end{\lstenv@name}}%
    \else %
        \def\lst@next{\csname end\lstenv@name\endcsname}%
    \fi %
    \lst@next}
%    \end{macrocode}
% \end{macro}
%
% \begin{macro}{\lstenv@Process}
% First some initialization, then call a submacro.
%    \begin{macrocode}
\def\lstenv@Process{%
    \lst@Init\lstenv@backslash %
    \global\lst@lineno\lst@firstline\relax %
    \let\lstenv@ifdropped\iffalse \lstenv@Process@}
%    \end{macrocode}
%    \begin{macrocode}
\def\lstenv@droppedtrue{\let\lstenv@ifdropped\iftrue}
%    \end{macrocode}
% We execute either |\lstenv@ProcessM| or |\lstenv@ProcessJ| according to whether we find an active EOL or a nonactive |^^J|.
%    \begin{macrocode}
\begingroup \lccode`\~=`\^^M%
\lowercase{\gdef\lstenv@Process@#1{%
    \ifx~#1%
        \expandafter\lstenv@ProcessM %
    \else\ifx^^J#1%
        \expandafter\expandafter\expandafter\lstenv@ProcessJ %
    \else %
        \lstenv@droppedtrue %
        \expandafter\expandafter\expandafter\lstenv@Process@ %
    \fi \fi}
}\endgroup
%    \end{macrocode}
% \end{macro}
%
% \begin{macro}{\lstenv@DroppedWarning}
% gives a warning if characters have been dropped.
%    \begin{macrocode}
\def\lstenv@DroppedWarning{%
    \lstenv@ifdropped %
        \PackageWarning{Listings}{Text dropped after begin of listing}%
    \fi}
%    \end{macrocode}
% \end{macro}
%
% \begin{macro}{\lstenv@ProcessM}
% There is nothing to do if we've found an active EOL, except giving a warning if necessary.
%    \begin{macrocode}
\def\lstenv@ProcessM{\lstenv@DroppedWarning \lst@BOLGobble}
%    \end{macrocode}
% \end{macro}
%
% \begin{macro}{\lstenv@ProcessJ}
% Now comes the horrible scenario: A listing inside an argument.
% We've already worked in section \ref{iAnApplicationTo} for this.
% Here we must get the listing, i.e.\ all characters upto 'end environment'.
% We must distinguish the cases 'command fashion' and 'environment'.
%    \begin{macrocode}
\def\lstenv@ProcessJ{%
    \lstenv@DroppedWarning %
    \lst@DontEscapeToLaTeX %
    \let\lstenv@arg\@empty %
    \ifx\@currenvir\lstenv@name %
        \expandafter\lstenv@ProcessJEnv %
    \else %
%    \end{macrocode}
% The first case is pretty simple: The code is terminated by |\end|\meta{name of environment}.
% Thus we expand that control sequence before defining a temporary macro, which gets all characters upto that control sequence.
% Inside the temporary macro we assign the argument and call a submacro doing the rest.
%    \begin{macrocode}
        \expandafter\def\expandafter\lst@temp\expandafter##1%
        \csname end\lstenv@name\endcsname{%
            \lstenv@AddArg{##1}\lstenv@ProcessJ@}%
%    \end{macrocode}
% Back to the definition of |\lstenv@ProcessJ| we call the temporary macro.
%    \begin{macrocode}
        \expandafter\lst@temp %
    \fi}
%    \end{macrocode}
% We must append an active backslash and the 'end string' to |\lstenv@arg|.
% So all other processing won't notice that the code has been inside an argument.
% But the EOL character is chr(10)=|^^J| now and not chr(13).
% Finally we execute |\lstenv@arg| to typeset the listing.
% Note that we need |\lccode`\A=`\A| to preserve the argument of |\lstCC@Let| --- but obviously we could also write |\lstCC@Let{10}|\ldots
%    \begin{macrocode}
\begingroup \lccode`\~=`\\ \lccode`\A=`\A
\lowercase{\gdef\lstenv@ProcessJ@{%
    \expandafter\lst@lAddTo\expandafter\lstenv@arg %
        \expandafter{\expandafter\ \expandafter~\lstenv@endstring}%
    \lstCC@Let{"000A}\lst@MProcessListing \def\lst@BOLGobble##1{}%
    \expandafter\lst@BOLGobble\lstenv@arg}
}\endgroup
%    \end{macrocode}
% \end{macro}
%
% \begin{macro}{\lstenv@ProcessJEnv}
% Here we get all characters upto an |\end| (and the following argument).
% If the following argument equals |\lstenv@name|, we have found the end of environment and start typesetting.
%    \begin{macrocode}
\def\lstenv@ProcessJEnv#1\end#2{\def\lst@temp{#2}%
    \ifx\lstenv@name\lst@temp %
        \lstenv@AddArg{#1}%
        \expandafter\lstenv@ProcessJ@ %
    \else %
%    \end{macrocode}
% Otherwise we append the characters including the eaten |\end| and the eaten argument to current |\lstenv@arg|.
% And we look again for the end of environment.
%    \begin{macrocode}
        \lstenv@AddArg{#1\\end\{#2\}}%
        \expandafter\lstenv@ProcessJEnv %
    \fi}
%    \end{macrocode}
% \end{macro}
%
% \begin{environment}{lstlisting}
% The awkward work is done, here we deal with continued line numbering.
% \lsthelper{Boris Veytsman}{boris@plmsc.psu.edu}{1998/03/25}{continue line numbering: a.c b.c a.c} proposed to continue line numbers according to listing names.
% Thus we must save the name and either make a LOL item or define the first line number.
% Note that the macro |\lstno@| will be undefined or equivalent to |\relax|, so we always start with line number 1 in case of an empty name.
% Moreover we first test if the user has forgotten the name argument.
%    \begin{macrocode}
\lst@Environment{lstlisting}[2][]\is
    {\lstenv@TestEOLChar{#2}%
     \expandafter\ifx\csname lstno@\lst@intname\endcsname \relax %
         \ifx\lst@intname\@empty\else \lst@AddToLOL{#2}{}\fi %
     \else %
         \edef\lst@firstline{\csname lstno@\lst@intname\endcsname}%
         \let\lst@prelisting\lst@@prelisting %
         \let\lst@postlisting\lst@@postlisting %
     \fi %
     \lstset{#1}}
%    \end{macrocode}
% At the end of environment we simply save the current line number.
% If the listing name is empty, we use a space instead of the name.
% This leaves the macro |\lstno@| undefined.
%    \begin{macrocode}
    {\ifx\lst@intname\@empty %
         \expandafter\xdef\csname lstno@ \endcsname{\the\lst@lineno}%
     \else %
         \expandafter\xdef\csname lstno@\lst@intname\endcsname %
             {\the\lst@lineno}%
     \fi}
%    \end{macrocode}
% \end{environment}
%
% \begin{macro}{\lstenv@TestEOLChar}
% \begin{macro}{\lstenv@EOLCharError}
% Here we test for the two possible EOL characters.
%    \begin{macrocode}
\begingroup \lccode`\~=`\^^M\lowercase{%
\gdef\lstenv@TestEOLChar#1{%
    \lst@SetName{}%
    \ifx~#1\lstenv@EOLCharError \else %
        \ifx^^J#1\lstenv@EOLCharError \else %
            \lst@SetName{#1}%
        \fi %
    \fi}
}\endgroup
%    \end{macrocode}
% A simple error message.
%    \begin{macrocode}
\def\lstenv@EOLCharError{%
    \PackageError{Listings}
    {Oops! It seems you've forgotten the argument to\MessageBreak %
     a listing environment. Assuming empty argument}%
    {Type <RETURN> to proceed.}}
%    \end{macrocode}
% \end{macro}\end{macro}
%
% \begin{aspect}{advancelineno}
% \begin{aspect}{resetlineno}
% The last two aspects in this section simply advance/reset the first line.
%    \begin{macrocode}
\lst@Aspect{advancelineno}
    {\global\lst@lineno\lst@firstline\relax %
     \global\advance\lst@lineno#1\relax %
     \edef\lst@firstline{\the\lst@lineno}}
\lst@Aspect{resetlineno}[1]{\def\lst@firstline{#1}}
%    \end{macrocode}
% \end{aspect}\end{aspect}
% \begingroup
%    \begin{macrocode}
%</kernel>
%    \end{macrocode}
% \endgroup
%
%
% \subsection{Inline listings}
%
% \begingroup
%    \begin{macrocode}
%<*kernel>
%    \end{macrocode}
% \endgroup
% \begin{macro}{\lstinline}
% We redefine some macros here since they are possibly not save inside |\hbox|.
% Furthermore we use flexible columns and suppress \TeX\ comment lines.
% After doing initialization we redefine |\everypar| and the new line macro.
% We don't want them.
%    \begin{macrocode}
\def\lstinline{\hbox\bgroup %
    \let\smallbreak\relax %
    \let\lst@prelisting\relax \let\lst@postlisting\relax %
    \let\lst@ifflexible\iftrue \lst@DontEscapeToLaTeX %
    \def\lst@firstline{1}\def\lst@lastline{1}%
    \lst@Init\relax \everypar{}\global\let\lst@NewLine\relax %
    \lst@IfNextCharActive\lst@InlineM\lst@InlineJ}
%    \end{macrocode}
% \end{macro}
%
% \begin{macro}{\lst@InlineM}
% \begin{macro}{\lst@InlineJ}
% treat the cases of 'normal' inlines and inline listings inside an argument.
% In the first case the given character ends the inline listing and EOL (within such a listing) immediately ends it and produces an error message.
%    \begin{macrocode}
\def\lst@InlineM#1{%
    \lstCC@Def{`#1}{\lst@DeInit\egroup}%
    \lstCC@Def{"000D}{\lst@DeInit\egroup %
        \PackageError{Listings}{lstinline ended by EOL}\@ehc}}
%    \end{macrocode}
% In the other case we get all characters upto |#1| (via temporary macro), make these characters active, execute (typeset) them and end the listing.
% That's all about it.
%    \begin{macrocode}
\def\lst@InlineJ#1{%
    \def\lst@temp##1#1{%
        \let\lstenv@arg\@empty \lstenv@AddArg{##1}%
        \lstenv@arg \lst@DeInit\egroup}%
    \lst@temp}
%    \end{macrocode}
% \end{macro}\end{macro}
% \begingroup
%    \begin{macrocode}
%</kernel>
%    \end{macrocode}
% \endgroup
%
%
% \subsection{The box command}
%
% \begingroup
%    \begin{macrocode}
%<*kernel>
%    \end{macrocode}
% \endgroup
% \begin{macro}{\lstbox}
% The macro opens the |\hbox| right at the beginning and indicate its usage.
% Then we adjust some \lst-parameters:
% We switch to |\lstlistingtrue| since the command is intend to be used with small listings only;
% |\smallbreak|s preceding and following each listing are removed by redefining |\smallbreak|.
%    \begin{macrocode}
\newcommand\lstbox{%
    \hbox\bgroup %
        \let\lst@ifbox\iftrue %
        \let\smallbreak\relax %
        \let\lst@prelisting\relax \let\lst@postlisting\relax %
        \let\lst@@prelisting\relax\let\lst@@postlisting\relax %
        \lst@outerspread\z@ \lst@innerspread\z@ %
        \@ifnextchar[{\lstbox@}{\lstbox@[c]}}
%    \end{macrocode}
% Here we have to choose the right box --- in fact |\vcenter| isn't a box.
%    \begin{macrocode}
\def\lstbox@[#1]{%
    \hbox to\z@\bgroup %
        $\if#1t\vtop \else \if#1b\vbox \else \vcenter \fi\fi %
        \bgroup}%
%    \end{macrocode}
% We need to
%    \begin{macrocode}
\let\lst@ifbox\iffalse
%    \end{macrocode}
% \end{macro}
%
% \begin{macro}{\lst@EndBox}
% And the counterpart: We have to close some groups (and use |\hss| inside the |\hbox| to $0$pt).
% The outer |\hbox| gets its correct width using a |\vrule|.
%    \begin{macrocode}
\def\lst@EndBox{%
    \egroup $\hss \egroup %
    \vrule width\lst@maxwidth height\z@ depth\z@ %
    \egroup}
%    \end{macrocode}
% But that macro is called (if and) only if we've begun a |\lstbox| before.
% Note that we are possibly inside a \lst-environment here, so we have to execute |\lst@EndBox| either now or after closing the environment group.
%    \begin{macrocode}
\lst@AddToHook{OnExit}
    {\lst@ifbox %
         \global\advance\lst@maxwidth-\lst@innerspread %
         \global\advance\lst@maxwidth-\lst@outerspread %
         \ifx\@currenvir\lstenv@name %
             \expandafter\expandafter\expandafter\aftergroup %
         \fi %
         \expandafter\lst@EndBox %
     \fi}
%    \end{macrocode}
% \end{macro}
%
% \begin{macro}{\lst@maxwidth}
% is to be allocated:
%    \begin{macrocode}
\newdimen\lst@maxwidth % \global
%    \end{macrocode}
%    \begin{macrocode}
\lst@AddToHook{InitVars}{\global\lst@maxwidth\z@}
%    \end{macrocode}
% Determine width of just printed line and update |\lst@maxwidth|.
% Here we assume that all characters of the line have been output.
%    \begin{macrocode}
\lst@AddToHook{InitVarsEOL}
    {\@tempdima \lst@column\lst@width %
     \advance\@tempdima -\lst@pos\lst@width %
     \ifdim\lst@lostspace<\z@ \advance\@tempdima -\lst@lostspace \fi %
     \ifdim\@tempdima>\lst@maxwidth \global\lst@maxwidth\@tempdima \fi}
%    \end{macrocode}
% \end{macro}
% \begingroup
%    \begin{macrocode}
%</kernel>
%    \end{macrocode}
% \endgroup
%
%
% \section{First \lst-aspects and related macros}
%
% In fact we've already defined some aspects, e.g.\ |language| and |style|.
%
%
% \subsection{Keywords}
%
% \begingroup
%    \begin{macrocode}
%<*kernel>
%    \end{macrocode}
% \endgroup
% \begin{TODO}
% The internal keyword managing should be reorganised, so that is also handles the special characters $<,>,\vert,\sim$.
% \end{TODO}
% We have to decide whether or not a given character sequence is a reserved word.
% For example, if the character sequence is |key|, we define a macro
% \begin{verbatim}
%     \def\lst@temp#1,key,#2\relax{...}\end{verbatim}
% Afterwards we call the macro with the following arguments:
% \begin{itemize}
% \item[] |,|all current keywords|,key,\relax|
% \end{itemize}
% The additional |,key,\relax| holds up the syntax of |\lst@temp|.
% When \TeX{} passes the arguments, the second is empty if and only if |key| is not a current keyword.
% So we know whether or not to select keyword style.
%
% Since \TeX{} always passes two arguments to |\lst@temp|, the whole 'input' |\lst@keywords,#1,\relax| is split in two parts.
% So there is no need to sort the keywords by probability.
% Alternatively you could do so and make a loop for the keyword tests, which terminates right after finding a keyword.
% You might think and guess that's faster than the \TeX{}nique used here.
% If your source code uses the three or four most common keywords only, you are right.
% In fact the very first versions 0.1 and 0.11 have used something like loops (even something faster), which in general is slower than this here.
% \begin{TODO}
% There exists a faster way.
% For each keyword we make |\lstk@|\emph{the keyword} equivalent to keyword style.
% |\csname lstk@|\emph{test word}|\endcsname| expands to |\relax| if the control sequence is undefined and to keyword style otherwise.
% That's the complete keyword test!
% It works very well, but needs more of \TeX{}'s memory.
% Therefore we should define also 'memory saving' and/or 'speeding up' option.
%
% Note: A keyword or character string in this state might contain |\lst@underscore| and |\lst@minus| (and possibly more macro names), which will expand inside |\csname|\ldots|\endcsname|.
% Hence the definition of |\lstk@|\emph{the keyword} needs some care.
% An easy trick: During keyword definition and keyword testing the macros should expand to |_| and |-|, respectively.
% But the reassignments of |_|, |-|, \ldots\ also need time, so the implemented way is possibly faster \ldots
% \end{TODO}
%
% \begin{macro}{\lst@SetStyleFor}
% |\relax| terminates the argument here since it is faster than enclosing it in braces.
% All the rest should be clear from the discussion above.
%    \begin{macrocode}
\def\lst@SetStyleFor#1\relax{%
    \def\lst@temp##1,#1,##2\relax{%
        \ifx\@empty##2\@empty \let\lst@thestyle\lst@nonkeywordstyle %
                        \else \let\lst@thestyle\lst@keywordstyle\fi}%
    \expandafter\lst@temp\expandafter,\lst@keywords,#1,\relax}%
%    \end{macrocode}
% \end{macro}
%
% \begin{macro}{\lst@SetStyleForNonSensitive}
% We implement a case insensitive version.
% Use of two |\uppercase|s normalize the argument.
%    \begin{macrocode}
\def\lst@SetStyleForNonSensitive#1\relax{%
    \uppercase{\def\lst@temp##1,#1},##2\relax{%
        \ifx\@empty##2\@empty \let\lst@thestyle\lst@nonkeywordstyle %
                        \else \let\lst@thestyle\lst@keywordstyle\fi}%
    \uppercase{%
        \expandafter\lst@temp\expandafter,\lst@keywords,#1},\relax}%
%    \end{macrocode}
% Note: This macro and the coming |\lst@IfOneOfNonSensitive| both assume that the keywords in |\lst@keywords| are already upper case!
% \end{macro}
%
% The code above must be activated, of course:
% We detect keywords if and only if we haven't entered a special mode (comment, string, etc.).
%    \begin{macrocode}
\lst@AddToHook{Output}
    {\lst@ifmode %
         \let\lst@thestyle\relax %
     \else %
         \expandafter\lst@SetStyleFor\the\lst@token\relax %
     \fi}
%    \end{macrocode}
%
% \begin{macro}{\lst@IfOneOf}
% \begin{macro}{\lst@IfOneOfNonSensitive}
% These macros are very familiar with the keyword tests.
% Roughly speaking the fixed |\lst@keywords| is replaced by an arbitrary macro.
% The first argument has the same meaning and is terminated by |\relax|, whereas the second must be a macro name.
% If and only if that macro (= keyword list) contains the first argument, the third parameter is executed (and the fourth otherwise).
%    \begin{macrocode}
\def\lst@IfOneOf#1\relax#2{%
    \def\lst@temp##1,#1,##2\relax{%
        \ifx \@empty##2\@empty \expandafter\@secondoftwo %
        \else \expandafter\@firstoftwo \fi}%
    \expandafter\lst@temp\expandafter,#2,#1,\relax}%
%    \end{macrocode}
%    \begin{macrocode}
\def\lst@IfOneOfNonSensitive#1\relax#2{%
    \uppercase{\def\lst@temp##1,#1},##2\relax{%
        \ifx \@empty##2\@empty \expandafter\@secondoftwo %
        \else \expandafter\@firstoftwo \fi}%
    \uppercase\expandafter{%
        \expandafter\lst@temp\expandafter,#2,#1},\relax}%
%    \end{macrocode}
% Instead of |\lst@SetStyleFor key\relax|, we could also write
% \begin{verbatim}
%     \lst@IfOneOf key\relax \lst@keywords %
%         {\let\lst@thestyle\lst@keywordstyle}
%         {\let\lst@thestyle\lst@nonkeywordstyle}\end{verbatim}
% \end{macro}\end{macro}
%
% \begin{macro}{\lststorekeywords}
% Note that this command stores the keywords globally.
%    \begin{macrocode}
\newcommand\lststorekeywords[2]{\gdef#1{#2}}
%    \end{macrocode}
% \end{macro}
%
% \begin{aspect}{sensitive}
% Announcement and init:
%    \begin{macrocode}
\lst@Aspect{sensitive}[t]{\lstKV@SetIfKey\lst@ifsensitive{#1}}
\lst@AddToHook{SetLanguage}{\let\lst@ifsentitive\iftrue}
%    \end{macrocode}
% \end{aspect}
%
% \noindent
% Now we consider the aspects getting keywords from the user.
% They work up the keywords a little bit, namely using |\lst@Make|\ldots|KeywordArg|.
%
% \begin{aspect}{keywords}
% \lsthelper{Ralf Quast}{rquast@hs.uni-hamburg.de}{1998/01/08}{name \keywords incompatible with AMS classes} reported a naming conflict with AMS classes.
% Usage of \textsf{keyval} package removes it.
% The easy definition:
%    \begin{macrocode}
\lst@Aspect{keywords}
    {\lst@MakeSpecKeywordArg{#1}\let\lst@keywords\lst@arg}
%    \end{macrocode}
% We have to delete all current keywords before selecting a new language --- otherwise keywords from the previous language would be present if the user specifies no keywords.
%    \begin{macrocode}
\lst@AddToHook{SetLanguage}{\let\lst@keywords\@empty}
%    \end{macrocode}
% If the user wants case insensitive keywords, she'll get it:
% We assign the correct test macros and make all keywords upper case.
% Note that these changes are local.
%    \begin{macrocode}
\lst@AddToHook{BeforeSelectCharTable}
    {\lst@ifsensitive\else %
         \let\lst@SetStyleFor\lst@SetStyleForNonSensitive %
         \let\lst@IfOneOf\lst@IfOneOfNonSensitive %
         \lst@MakeMacroUppercase\lst@keywords %
     \fi}
%    \end{macrocode}
% \end{aspect}
%
% \begin{macro}{\lst@MakeMacroUppercase}
% This macro makes the contents of a given macro (if present) upper case.
%    \begin{macrocode}
\def\lst@MakeMacroUppercase#1{%
    \ifx#1\@undefined\else \uppercase\expandafter{%
        \expandafter\def\expandafter#1\expandafter{#1}}%
    \fi}
%    \end{macrocode}
% \end{macro}
%
% \begin{aspect}{morekeywords}
% Add keywords or append control sequence with argument:
%    \begin{macrocode}
\lst@Aspect{morekeywords}
    {\lst@MakeMoreSpecKeywordArg{,#1}%
     \expandafter\lst@lAddTo\expandafter\lst@keywords %
         \expandafter{\lst@arg}}
%    \end{macrocode}
% \end{aspect}
%
% \begin{aspect}{deletekeywords}
% The 'submacro' |\lst@DeleteKeysIn| has been defined in section \ref{iReplacingCharacters}.
%    \begin{macrocode}
\lst@Aspect{deletekeywords}
    {\lst@MakeKeywordArg{#1}%
     \lst@DeleteKeysIn\lst@keywords\lst@arg}
%    \end{macrocode}
% \end{aspect}
%
% \begin{macro}{\lst@MakeKeywordArg}
% \begin{macro}{\lst@MakeSpecKeywordArg}
% \begin{macro}{\lst@MakeMoreSpecKeywordArg}
% The keyword commands don't save their parameters as they are.
% All spaces are removed and underscores, dollars and minuses are replaced by |\lst@underscore|, |\lst@dollar| and |\lst@minus|.
% The first thing prepares keyword tests and the second the output.
% The macro |\lst@arg| holds the parameter free of spaces, underscores and minuses.
% Use of |\zap@space| was proposed by \lsthelper{Rolf Niepraschk}{NIEPRASCHK@PTB.DE}{1997/04/24}{use \zap@space}.
%    \begin{macrocode}
\def\lst@MakeKeywordArg#1{\edef\lst@arg{\zap@space#1 \@empty}%
    \lst@ReplaceIn\lst@arg{_\lst@underscore $\lst@dollar -\lst@minus}}
%    \end{macrocode}
% The following two macros also scan for special characters like \#, $<$ and $>$.
%    \begin{macrocode}
\def\lst@MakeSpecKeywordArg{%
    \let\lst@ialsodigit\@empty \let\lst@ialsoletter\@empty %
    \lst@MakeMoreSpecKeywordArg}
%    \end{macrocode}
%    \begin{macrocode}
\def\lst@MakeMoreSpecKeywordArg#1{\edef\lst@arg{\zap@space#1 \@empty}%
    \lstCC@lettertrue %
    \expandafter\lst@SpecialKeywordScan\lst@arg\relax %
    \lst@ReplaceIn\lst@arg{_\lst@underscore $\lst@dollar -\lst@minus}}
%    \end{macrocode}
% \end{macro}\end{macro}\end{macro}
%
% \begin{macro}{\lst@SpecialKeywordScan}
% How does this scan work?
% Whenever we encounter a non-letter or non-digit in a keyword, we call |\lst@SKSAdd| to store that character.
% If we reach the end of keyword list, we terminate the loop by gobbling one token after the latest |\fi|.
%    \begin{macrocode}
\def\lst@SpecialKeywordScan#1{%
    \ifx\relax#1%
        \expandafter\@gobble %
    \else %
%    \end{macrocode}
% If the current character is a comma, the next character must be (or become) a letter since it starts a keyword.
% This switch is turned false after reading the next character.
%    \begin{macrocode}
        \ifx,#1%
            \lstCC@lettertrue %
        \else %
            \ifnum`#1<"40\relax %
                \ifnum`#1<"30\relax \lst@SKSAdd#1\else %
                \ifnum`#1>"39\relax \lst@SKSAdd#1\else %
%    \end{macrocode}
% If we've found a digit, we do a special test to decide whether the digit becomes a letter or not.
%    \begin{macrocode}
                    \lst@SKS@#1%
                \fi \fi %
            \else %
                \ifnum`#1<"5B\relax \else %
                \ifnum`#1="5F\relax \else %
                \ifnum`#1<"61\relax \lst@SKSAdd#1\else %
                \ifnum`#1<"7B\relax \else %
                \ifnum`#1<"80\relax \lst@SKSAdd#1 %
                \fi \fi \fi \fi \fi %
            \fi %
            \lstCC@letterfalse %
        \fi %
    \fi \lst@SpecialKeywordScan}
%    \end{macrocode}
% The special test for digits: Since any digit is already a digit it needs only to become a letter if necessary.
%    \begin{macrocode}
\def\lst@SKS@#1{\lstCC@ifletter \lst@SKSAdd#1\fi}
%    \end{macrocode}
% The macros |\lst@ialsoletter| and |\lst@ialsodigit| contain the characters.
% If not already contained in the appropiate macro, we append the character.
% Refer |\lst@SelectStyleFor| how we scan for a substring (of length 1).
%    \begin{macrocode}
\def\lst@SKSAdd#1{%
    \lstCC@ifletter %
        \def\lst@temp##1#1##2\relax{%
            \ifx\@empty##2\@empty %
                \lst@lAddTo\lst@ialsoletter{#1}%
            \fi}%
        \expandafter\lst@temp\lst@ialsoletter#1\relax %
    \else %
        \def\lst@temp##1#1##2\relax{%
            \ifx\@empty##2\@empty %
                \lst@lAddTo\lst@ialsodigit{#1}%
            \fi}%
        \expandafter\lst@temp\lst@ialsodigit#1\relax %
    \fi}
%    \end{macrocode}
%    \begin{macrocode}
\let\lst@ialsoletter\@empty \let\lst@ialsodigit\@empty % init
%    \end{macrocode}
% \end{macro}
%
% \begin{aspect}{alsoletter}
% \begin{aspect}{alsodigit}
% \begin{aspect}{alsoother}
% For now three easy definitions --- easy since I don't explain the last hook.
%    \begin{macrocode}
\lst@Aspect{alsoletter}{\def\lst@alsoletter{#1}}
\lst@Aspect{alsodigit}{\def\lst@alsodigit{#1}}
\lst@Aspect{alsoother}{\def\lst@alsoother{#1}}
%    \end{macrocode}
%    \begin{macrocode}
\lst@AddToHook{SetLanguage}
    {\let\lst@alsoletter\@empty %
     \let\lst@alsodigit\@empty %
     \let\lst@alsoother\@empty}
%    \end{macrocode}
%    \begin{macrocode}
\lst@AddToHook{SelectCharTable}
    {\lstCC@ChangeBasicClass\lstCC@ProcessOther\lst@alsoother %
     \lstCC@ChangeBasicClass\lstCC@ProcessDigit\lst@ialsodigit %
     \lstCC@ChangeBasicClass\lstCC@ProcessDigit\lst@alsodigit %
     \lstCC@ChangeBasicClass\lstCC@ProcessLetter\lst@ialsoletter %
     \lstCC@ChangeBasicClass\lstCC@ProcessLetter\lst@alsoletter}
%    \end{macrocode}
% \end{aspect}\end{aspect}\end{aspect}
%
% \begin{macro}{\lst@minus}
% What is the |\ifx|\ldots{} in |\lst@minus| good for?
% If you use typewriter fonts, it ensures that |----| is typeset |----| and not $-$$-$$-$$-$ as in version 0.17.
% Bug encountered by \lsthelper{Dr. Jobst Hoffmann}{HOFFMANN@rz.rwth-aachen.de}{1998/03/30}{\lst@minus\ and typewriter fonts}.
%    \begin{macrocode}
\def\lst@minus{\ifx\f@family\ttdefault-{}\else$-$\fi}
\def\lst@dollar{\ifx\f@family\ttdefault\textdollar\else\textdollar\fi}
\def\lst@asterisk{\ifx\f@family\ttdefault*\else\textasteriskcentered\fi}
\def\lst@less{\ifx\f@family\ttdefault<\else\textless\fi}
\def\lst@greater{\ifx\f@family\ttdefault>\else\textgreater\fi}
\def\lst@backslash{\ifx\f@family\ttdefault\char92\else\textbackslash\fi}
\def\lst@underscore{%
    \ifx\f@family\ttdefault\char95\else\textunderscore\fi}
\def\lst@lbrace{\ifx\f@family\ttdefault\char123\else\textbraceleft\fi}
\def\lst@bar{\ifx\f@family\ttdefault|\else\textbar\fi}
\def\lst@rbrace{\ifx\f@family\ttdefault\char125\else\textbraceright\fi}
%    \end{macrocode}
% |\ttdefault| is defined |\long|, so the |\ifx| doesn't work since |\f@family| isn't defined |\long|!
% We go around this problem by redefining |\ttdefault| locally:
%    \begin{macrocode}
\lst@AddToHook{BeforeSelectCharTable}{\edef\ttdefault{\ttdefault}}
%    \end{macrocode}
% \end{macro}
%
% \begin{aspect}{basicstyle}
% \begin{aspect}{keywordstyle}
% \begin{aspect}{nonkeywordstyle}
% We shouldn't forget these style aspects.
%    \begin{macrocode}
\lst@Aspect{basicstyle}{\def\lst@basicstyle{#1}}
\lst@Aspect{keywordstyle}{\def\lst@keywordstyle{#1}}
\lst@Aspect{nonkeywordstyle}{\def\lst@nonkeywordstyle{#1}}
%    \end{macrocode}
% \lsthelper{Anders Edenbrandt}{Anders.Edenbrandt@dna.lth.se}{1997/04/22}{preload of .fd files} found a bug with \texttt{.fd} files.
% Here's my solution: Since we will change catcodes, these files can't be read on demand --- it would yield to obscure error messages.
% The |\setbox| sequence ensures (most times) that they are read before.
% We simply typeset distinct characters from each \texttt{.fd} file.
% Note: We never output that box.
%    \begin{macrocode}
\lst@AddToHook{BeforeSelectCharTable}
    {\lst@basicstyle %
     \setbox\@tempboxa\hbox{\lst@loadfd %
         {\lst@keywordstyle \lst@loadfd}%
         {\lst@nonkeywordstyle \lst@loadfd}}}
%    \end{macrocode}
% \end{aspect}\end{aspect}\end{aspect}
%
% \begin{macro}{\lst@loadfd}
% These are hopefully all necessary characters.
%    \begin{macrocode}
\def\lst@loadfd{a0\lst@asterisk\lst@less}
%    \end{macrocode}
% \end{macro}
%
% \begin{aspect}{ndkeywords}
% \begin{aspect}{ndkeywordstyle}
% \begin{aspect}{morendkeywords}
% \begin{aspect}{deletendkeywords}
% We define a second keyword class in the same manner if the user wants it.
%    \begin{macrocode}
\@ifundefined{lst@ndkeywords}{}{%
\let\lst@ndkeywords\@empty \let\lst@ndkeywordstyle\@empty % init
\lst@Aspect{ndkeywords}
    {\lst@MakeSpecKeywordArg{#1}\let\lst@ndkeywords\lst@arg}
\lst@AddToHook{SetLanguage}{\let\lst@ndkeywords\@empty}
\lst@AddToHook{BeforeSelectCharTable}
    {\lst@ifsensitive\else %
         \lst@MakeMacroUppercase\lst@ndkeywords %
     \fi}
%    \end{macrocode}
%    \begin{macrocode}
\lst@Aspect{ndkeywordstyle}{\def\lst@ndkeywordstyle{#1}}
\lst@AddToHook{BeforeSelectCharTable}
    {\setbox\@tempboxa\hbox{{\lst@ndkeywordstyle \lst@loadfd}}}
%    \end{macrocode}
%    \begin{macrocode}
\lst@Aspect{morendkeywords}
    {\lst@MakeMoreSpecKeywordArg{,#1}%
     \expandafter\lst@lAddTo\expandafter\lst@ndkeywords %
         \expandafter{\lst@arg}}
%    \end{macrocode}
%    \begin{macrocode}
\lst@Aspect{deletendkeywords}
    {\lst@MakeKeywordArg{#1}%
     \lst@DeleteKeysIn\lst@ndkeywords\lst@arg}
%    \end{macrocode}
%    \begin{macrocode}
\lst@AddToHook{Output}
    {\lst@ifmode\else %
         \expandafter\lst@IfOneOf\the\lst@token\relax \lst@ndkeywords %
             {\let\lst@thestyle\lst@ndkeywordstyle}{}%
     \fi}
%    \end{macrocode}
% Finally we end the argument from |\@ifundefined|.
%    \begin{macrocode}
}
%    \end{macrocode}
% \end{aspect}\end{aspect}\end{aspect}\end{aspect}
%
% \begin{aspect}{rdkeywords}
% \begin{aspect}{rdkeywordstyle}
% \begin{aspect}{morerdkeywords}
% \begin{aspect}{deleterdkeywords}
% That's the power of \lst-aspects: An optional third keyword class.
%    \begin{macrocode}
\@ifundefined{lst@rdkeywords}{}{%
\let\lst@rdkeywords\@empty \let\lst@rdkeywordstyle\@empty % init
\lst@Aspect{rdkeywords}
    {\lst@MakeSpecKeywordArg{#1}\let\lst@rdkeywords\lst@arg}
\lst@AddToHook{SetLanguage}{\let\lst@rdkeywords\@empty}
\lst@AddToHook{BeforeSelectCharTable}
    {\lst@ifsensitive\else %
         \lst@MakeMacroUppercase\lst@rdkeywords %
     \fi}
%    \end{macrocode}
%    \begin{macrocode}
\lst@Aspect{rdkeywordstyle}{\def\lst@rdkeywordstyle{#1}}
\lst@AddToHook{BeforeSelectCharTable}
    {\setbox\@tempboxa\hbox{{\lst@rdkeywordstyle \lst@loadfd}}}
%    \end{macrocode}
%    \begin{macrocode}
\lst@Aspect{morerdkeywords}
    {\lst@MakeMoreSpecKeywordArg{,#1}%
     \expandafter\lst@lAddTo\expandafter\lst@rdkeywords %
         \expandafter{\lst@arg}}
%    \end{macrocode}
%    \begin{macrocode}
\lst@Aspect{deleterdkeywords}
    {\lst@MakeKeywordArg{#1}%
     \lst@DeleteKeysIn\lst@rdkeywords\lst@arg}
%    \end{macrocode}
%    \begin{macrocode}
\lst@AddToHook{Output}
    {\lst@ifmode\else %
         \expandafter\lst@IfOneOf\the\lst@token\relax \lst@rdkeywords %
             {\let\lst@thestyle\lst@rdkeywordstyle}{}%
     \fi}
}
%    \end{macrocode}
% \end{aspect}\end{aspect}\end{aspect}\end{aspect}
%
%
% \subsection{Export of indentifiers}
%
% \begin{aspect}{index}
% We implement this aspect in the same manner.
%    \begin{macrocode}
\@ifundefined{lst@index}{}{%
\lst@Aspect{index}{\lst@MakeSpecKeywordArg{#1}\let\lst@index\lst@arg}
\lst@AddToHook{BeforeSelectCharTable}
    {\lst@ifsensitive\else %
         \lst@MakeMacroUppercase\lst@index %
     \fi}
%    \end{macrocode}
%    \begin{macrocode}
\lst@AddToHook{Output}
    {\lst@ifmode\else %
         \expandafter\lst@IfOneOf\the\lst@token\relax \lst@index %
            {\expandafter\lst@indexmacro\expandafter{\the\lst@token}}{}%
     \fi}
%    \end{macrocode}
%    \begin{macrocode}
\lst@Aspect{indexmacro}{\let\lst@indexmacro#1}
\newcommand\lstindexmacro[1]{\index{{\ttfamily#1}}}
\let\lst@index\@empty \let\lst@indexmacro\lstindexmacro % init
}
%    \end{macrocode}
% \end{aspect}
%
% \begin{aspect}{procnamestyle}
% \begin{aspect}{prockeywords}
% \begin{aspect}{indexprocnames}
% The 'idea' here is the usage of a global |\lst@ifprocname| indicating a preceding 'procedure keyword'.
% All the other is known stuff.
%    \begin{macrocode}
\@ifundefined{lst@prockeywords}{}{%
\lst@Aspect{prockeywords}{%
    \lst@MakeSpecKeywordArg{#1}\let\lst@prockeywords\lst@arg}
\lst@Aspect{procnamestyle}{\def\lst@procnamestyle{#1}}
\lst@Aspect{indexprocnames}[t]{\lstKV@SetIfKey\lst@ifindexproc{#1}}
%    \end{macrocode}
%    \begin{macrocode}
\lst@AddToHook{BeforeSelectCharTable}
    {\setbox\@tempboxa\hbox{\lst@procnamestyle\lst@loadfd}%
     \lst@ifsensitive\else %
         \lst@MakeMacroUppercase\lst@prockeywords %
     \fi %
     \lst@ifindexproc \ifx\lst@indexmacro\@undefined %
         \let\lst@indexmacro\@gobble %
     \fi \fi}
%    \end{macrocode}
%    \begin{macrocode}
\lst@AddToHook{Output}
    {\lst@ifmode\else %
         \lst@ifprocname %
             \let\lst@thestyle\lst@procnamestyle %
             \expandafter\lst@indexmacro\expandafter{\the\lst@token}%
             \lst@procnamefalse %
         \else \expandafter%
             \lst@IfOneOf\the\lst@token\relax \lst@prockeywords %
                {\lst@procnametrue}{}%
         \fi %
     \fi}
%    \end{macrocode}
%    \begin{macrocode}
\def\lst@procnametrue{\global\let\lst@ifprocname\iftrue}
\def\lst@procnamefalse{\global\let\lst@ifprocname\iffalse}
%    \end{macrocode}
%    \begin{macrocode}
\let\lst@prockeywords\@empty % init
\lstset{procnamestyle={},indexprocnames=false}% init
\lst@procnamefalse % init
}
%    \end{macrocode}
% \end{aspect}\end{aspect}\end{aspect}
% \begingroup
%    \begin{macrocode}
%</kernel>
%    \end{macrocode}
% \endgroup
%
%
% \subsection{Labels}
%
% \begingroup
%    \begin{macrocode}
%<*kernel>
%    \end{macrocode}
% \endgroup
% \lsthelper{Rolf Niepraschk}{NIEPRASCHK@PTB.DE}{1997/04/24}{labels} asked for this feature.
%
% \begin{aspect}{labelstyle}
% \begin{aspect}{labelsep}
% \begin{aspect}{labelstep}
% The usual stuff:
% Definition with check for legal step count, \ldots
%    \begin{macrocode}
\lst@Aspect{labelstyle}{\def\lst@labelstyle{#1}}
\lst@Aspect{labelsep}{\def\lst@labelsep{#1}}
\lst@Aspect{labelstep}%
    {\ifnum #1>\m@ne %
         \def\lst@labelstep{#1}%
     \else %
         \PackageError{Listings}{Nonnegative integer expected}%
         {You can't use `#1' as step count for labels.^^J%
          I'll forget it and proceed.}%
     \fi}
%    \end{macrocode}
% and load of \texttt{.fd} files (if necessary).
%    \begin{macrocode}
\lst@AddToHook{BeforeSelectCharTable}
    {\setbox\@tempboxa\hbox{\lst@stringstyle \lst@loadfd}}
%    \end{macrocode}
% \end{aspect}\end{aspect}\end{aspect}
%
% \begin{macro}{\lst@skiplabels}
% But there are more things to do.
%    \begin{macrocode}
\newcount\lst@skiplabels % \global
%    \end{macrocode}
% We calculate how many lines must skip their label.
% The formula is
%	$$|\lst@skiplabels|=
%		\textrm{\emph{first printing line}}\bmod|\lst@labelstep|.$$
% Note that we use a nonpositive representative for |\lst@skiplabels|.
%    \begin{macrocode}
\lst@AddToHook{BeforeSelectCharTable}
    {\ifnum\lst@labelstep>\z@ %
         \global\lst@skiplabels\lst@firstline\relax %
         \global\divide\lst@skiplabels\lst@labelstep %
         \global\multiply\lst@skiplabels-\lst@labelstep %
         \global\advance\lst@skiplabels\lst@firstline\relax %
         \ifnum\lst@skiplabels>\z@ %
             \global\advance\lst@skiplabels -\lst@labelstep\relax %
         \fi %
%    \end{macrocode}
% If |\lst@labelstep| is nonpositive (in fact zero), no labels are printed:
%    \begin{macrocode}
     \else %
         \let\lst@SkipOrPrintLabel\relax %
     \fi}
\lst@AddToHook{EveryLine}{\lst@SkipOrPrintLabel}
%    \end{macrocode}
% \end{macro}
%
% \begin{macro}{\lst@SkipOrPrintLabel}
% But default is this.
% We use the fact that |\lst@skiplabels| is nonpositive.
% The counter advances every line and if that counter is zero, we print a line number and decrement the counter by |\lst@labelstep|.
%    \begin{macrocode}
\def\lst@SkipOrPrintLabel{%
    \ifnum\lst@skiplabels=\z@ %
        \global\advance\lst@skiplabels-\lst@labelstep\relax %
        \llap{\lst@labelstyle{\the\lst@lineno}\kern\lst@labelsep}%
    \fi %
    \global\advance\lst@skiplabels\@ne}
%    \end{macrocode}
% \end{macro}
% \begingroup
%    \begin{macrocode}
%</kernel>
%    \end{macrocode}
% \endgroup
%
%
% \subsection{Parshape and lineskip}
%
% \begingroup
%    \begin{macrocode}
%<*kernel>
%    \end{macrocode}
% \endgroup
% \begin{macro}{\lst@innerspread}
% \begin{macro}{\lst@outerspread}
% Just allocate these dimensions.
%    \begin{macrocode}
\newdimen\lst@innerspread \newdimen\lst@outerspread
%    \end{macrocode}
% \end{macro}\end{macro}
%
% \begin{aspect}{wholeline}
% \begin{aspect}{indent}
% \begin{aspect}{spread}
% Usual stuff.
%    \begin{macrocode}
\lst@Aspect{wholeline}[t]{\lstKV@SetIfKey\lst@ifwholeline{#1}}
\lst@Aspect{indent}{\def\lst@indent{#1}}
\lst@Aspect{spread}{\lstspread@#1,,\relax}
%    \end{macrocode}
% \lsthelper{Harald Haders}{h.haders@tu-bs.de}{1998/03/30}{inner- and outerspread} had the idea of two spreads (inner and outer).
% We either divide the dimension by two or assign the two dimensions to inner- and outerspread.
%    \begin{macrocode}
\def\lstspread@#1,#2,#3\relax{%
    \lst@innerspread#1\relax %
    \ifx\@empty#2\@empty %
        \divide\lst@innerspread\tw@\relax %
        \lst@outerspread\lst@innerspread %
    \else %
        \lst@outerspread#2\relax %
    \fi}
%    \end{macrocode}
%    \begin{macrocode}
\lstset{wholeline=false,indent=\z@,spread=\z@}% init
%    \end{macrocode}
% \end{aspect}\end{aspect}\end{aspect}
%
% \begin{macro}{\lst@parshape}
% The definition itself is easy.
% Note that we use this parshape every line (in fact every paragraph).
% Furthermore we must repeat the parshape if we close a group level --- or the shape is forgotten.
%    \begin{macrocode}
\def\lst@parshape{%
    \parshape\@ne %
        \ifodd\c@page -\lst@innerspread\else -\lst@outerspread\fi %
        \linewidth}
\lst@AddToHookAtTop{EveryLine}{\lst@parshape}
\lst@AddToHookAtTop{EndGroup}{\lst@parshape}
%    \end{macrocode}
% We calculate the line width and (inner/outer) indent for a listing.
%    \begin{macrocode}
\lst@AddToHook{BeforeSelectCharTable}
    {\advance\linewidth\lst@innerspread %
     \advance\linewidth\lst@outerspread %
     \advance\linewidth-\lst@indent\relax %
     \advance\lst@innerspread-\lst@indent\relax %
     \advance\lst@outerspread-\lst@indent\relax %
     \lst@ifwholeline %
         \advance\linewidth\@totalleftmargin %
     \else %
         \advance\lst@innerspread-\@totalleftmargin %
         \advance\lst@outerspread-\@totalleftmargin %
     \fi %
     \if@twoside\else \lst@outerspread\lst@innerspread \fi}
%    \end{macrocode}
% \end{macro}
%
% \begin{aspect}{lineskip}
% Finally we come to this attribute --- the introduction is due to communication with \lsthelper{Andreas Bartelt}{Andreas.Bartelt@Informatik.Uni-Oldenburg.DE}{1997/09/11}{problem with redefed \parskip; \lstlineskip introduced}.
%    \begin{macrocode}
\lst@Aspect{lineskip}{\def\lst@lineskip{#1}}
\lst@AddToHook{BeforeSelectCharTable}{\parskip\lst@lineskip\relax}
\lstset{lineskip=\z@}% init
%    \end{macrocode}
% \end{aspect}
% \begingroup
%    \begin{macrocode}
%</kernel>
%    \end{macrocode}
% \endgroup
%
%
% \subsection{Frames}
%
% \begingroup
%    \begin{macrocode}
%<*kernel>
%    \end{macrocode}
% \endgroup
%
% \begin{aspect}{framerulewidth}
% \begin{aspect}{framerulesep}
% \begin{aspect}{frametextsep}
% We only have to store the arguments.
%    \begin{macrocode}
\lst@Aspect{framerulewidth}{\def\lst@framewidth{#1}}
\lst@Aspect{framerulesep}{\def\lst@framesep{#1}}
\lst@Aspect{frametextsep}{\def\lst@frametextsep{#1}}
%    \end{macrocode}
%    \begin{macrocode}
\lstset{framerulewidth=.4pt,framerulesep=2pt,frametextsep=3pt}% init
%    \end{macrocode}
% \end{aspect}\end{aspect}\end{aspect}
%
% \begin{aspect}{frame}
% The main aspect saves the argument, resets all supported types |tlrbTLRB| to |\relax| and defines the user specified frame.
%    \begin{macrocode}
\lst@Aspect{frame}
    {\def\lst@frame{#1}%
     \lstframe@\relax tlrbTLRB\relax %
     \lstframe@\@empty#1\relax}
%    \end{macrocode}
% This submacro defines the macros to be equivalent to |#1| (which is |\relax| or |\@empty|).
% The second argument is the list of characters terminated by |\relax|.
%    \begin{macrocode}
\def\lstframe@#1#2{%
    \ifx\relax#2\else %
        \expandafter\let\csname lstf@#2\endcsname#1%
        \expandafter\lstframe@\expandafter#1%
    \fi}
%    \end{macrocode}
%    \begin{macrocode}
\lstset{frame={}}% init
%    \end{macrocode}
% \end{aspect}
%
% \begin{macro}{\lst@framev}
% \begin{macro}{\lst@frameV}
% These macros typeset one or two vertical rules:
%    \begin{macrocode}
\def\lst@framev{\hbox{\strut\vrule width\lst@framewidth}}
\def\lst@frameV{\hbox{\strut %
    \vrule width\lst@framewidth\kern\lst@framesep %
    \vrule width\lst@framewidth}}
%    \end{macrocode}
% \end{macro}\end{macro}
%
% \begin{macro}{\lst@framelr}
% We typeset left and right frame rule every line (if not made |\relax|).
% Note that |\lst@framel| is possibly redefined and thus equivalent to |\lst@frameL|.
%    \begin{macrocode}
\lst@AddToHook{EveryLine}{\lst@framelr}
\def\lst@framelr{%
    \llap{\lst@framel\kern\lst@indent\kern\lst@frametextsep}%
    \rlap{\kern\linewidth\kern\lst@frametextsep\lst@framer}}
%    \end{macrocode}
% \end{macro}
%
% \begin{macro}{\lst@frameh}
% This is the main macro for horizontal lines.
% The first parameter gives the size of the left and right corner.
% The other two parameters typeset these corners (and get the size parameter).
% Now: We move to the correct horizontal position, set the left corner, the horizontal line and the right corner.
%    \begin{macrocode}
\def\lst@frameh#1#2#3{%
    \hbox to\z@{%
        \kern\ifodd\c@page -\lst@innerspread\else -\lst@outerspread\fi %
        \kern-\lst@indent %
%    \end{macrocode}
%    \begin{macrocode}
        \kern-\lst@frametextsep %
        \ifx\lstf@L\@empty %
            \llap{#2#1}%
        \else \ifx\lstf@l\@empty %
            \llap{#20}%
        \fi \fi %
        \vrule\@width\lst@frametextsep %
%    \end{macrocode}
%    \begin{macrocode}
        \vrule\@width\lst@indent %
        \vrule\@width\linewidth\@height\lst@framewidth %
%    \end{macrocode}
%    \begin{macrocode}
        \vrule\@width\lst@frametextsep %
        \ifx\lstf@R\@empty %
            \rlap{#3#1}%
        \else \ifx\lstf@r\@empty %
            \rlap{#30}%
        \fi \fi %
        \hss}}
%    \end{macrocode}
% Note: There are no paramaters with numbers 20 and 30.
% It's |#2| respectively |#3| with size argument 0.
% \end{macro}
%
% \begin{macro}{\lst@framet}
% \begin{macro}{\lst@frameb}
% \begin{macro}{\lst@frameT}
% \begin{macro}{\lst@frameB}
% Now we can derive the 'top' and 'bottom' frame macros quite easily:
%    \begin{macrocode}
\def\lst@framet{\lst@frameh0\lst@frameTL\lst@frameTR}%
\let\lst@frameb\lst@framet
\def\lst@frameT{%
    \lst@frameh1\lst@frameTL\lst@frameTR %
    \vskip\lst@framesep %
    \lst@frameh0\lst@frameTL\lst@frameTR}
\def\lst@frameB{%
    \lst@frameh0\lst@frameBL\lst@frameBR %
    \vskip\lst@framesep %
    \lst@frameh1\lst@frameBL\lst@frameBR}
%    \end{macrocode}
% These macros are executed where needed.
%    \begin{macrocode}
\lst@AddToHook{BeforeSelectCharTable}
    {\ifx\lstf@T\@empty %
         \offinterlineskip\par\noindent\lst@frameT %
     \else \ifx\lstf@t\@empty %
         \offinterlineskip\par\noindent\lst@framet %
     \else %
         \let\lst@framet\relax %
     \fi \fi %
%    \end{macrocode}
% \begin{TODO}
% Replace |\offinterlineskip| by |\nointerlineskip|?
% \end{TODO}
% We look which frame types we have on the left and on the right.
%    \begin{macrocode}
     \let\lst@framel\relax \let\lst@framer\relax %
     \ifx\lstf@L\@empty %
         \let\lst@framel\lst@frameV %
     \else %
         \ifx\lstf@l\relax\else \let\lst@framel\lst@framev \fi %
     \fi %
     \ifx\lstf@R\@empty %
         \let\lst@framer\lst@frameV %
     \else %
         \ifx\lstf@r\relax\else \let\lst@framer\lst@framev \fi %
     \fi %
%    \end{macrocode}
% We can speed up things if there are no vertical frames.
%    \begin{macrocode}
     \ifx\lst@framel\relax \ifx\lst@framer\relax %
         \let\lst@framelr\relax %
     \fi \fi %
%    \end{macrocode}
% Finally we close the space between the horizontal rule and the first line (the 'depth' of the current line).
%    \begin{macrocode}
     \ifx\lst@framelr\relax\else \ifx\lst@framet\relax\else %
         {\setbox\strutbox\hbox{%
              \vrule\@height0pt\@depth.3\normalbaselineskip\@width\z@}%
          \kern\ifodd\c@page-\lst@innerspread\else-\lst@outerspread\fi %
          \lst@framelr}%
     \fi \fi}
%    \end{macrocode}
% The frame at the bottom:
%    \begin{macrocode}
\lst@AddToHook{ExitVars}
    {\ifx\lstf@B\@empty %
         \offinterlineskip\everypar{}\par\noindent\lst@frameB %
     \else \ifx\lstf@b\@empty %
         \offinterlineskip\everypar{}\par\noindent\lst@framet %
     \fi \fi}
%    \end{macrocode}
% \end{macro}\end{macro}\end{macro}\end{macro}
%
% \begin{macro}{\lst@frameTL}
% \begin{macro}{\lst@frameTR}
% \begin{macro}{\lst@frameBL}
% \begin{macro}{\lst@frameBR}
% The 'corner' macros are left.
% All save the vertical rule in a temporary box to zero the depth or height of that box.
% Then we typeset the vertical and horizontal rule (in reversed order).
%    \begin{macrocode}
\def\lst@frameTL#1{%
    \@tempdima\lst@framesep \advance\@tempdima\lst@framewidth %
    \multiply\@tempdima#1\relax %
    \setbox\@tempboxa\hbox to\z@{%
        \vrule\@width\lst@framewidth\@height\z@\@depth\@tempdima\hss}%
    \dp\@tempboxa\z@ %
    \advance\@tempdima\lst@framewidth %
    \box\@tempboxa \vrule\@width\@tempdima\@height\lst@framewidth}
%    \end{macrocode}
%    \begin{macrocode}
\def\lst@frameTR#1{%
    \@tempdima\lst@framesep \advance\@tempdima\lst@framewidth %
    \multiply\@tempdima#1\relax %
    \setbox\@tempboxa\hbox to\z@{\hss %
        \vrule\@width\lst@framewidth\@height\z@\@depth\@tempdima}%
    \dp\@tempboxa\z@ %
    \advance\@tempdima\lst@framewidth %
    \vrule\@width\@tempdima\@height\lst@framewidth \box\@tempboxa}
%    \end{macrocode}
%    \begin{macrocode}
\def\lst@frameBL#1{%
    \@tempdima\lst@framesep \advance\@tempdima\lst@framewidth %
    \multiply\@tempdima#1\relax %
    \advance\@tempdima\lst@framewidth %
    \setbox\@tempboxa\hbox to\z@{%
        \vrule\@width\lst@framewidth\@height\@tempdima\hss}%
    \ht\@tempboxa\z@ %
    \box\@tempboxa \vrule\@width\@tempdima\@height\lst@framewidth}
%    \end{macrocode}
%    \begin{macrocode}
\def\lst@frameBR#1{%
    \@tempdima\lst@framesep \advance\@tempdima\lst@framewidth %
    \multiply\@tempdima#1\relax %
    \advance\@tempdima\lst@framewidth %
    \setbox\@tempboxa\hbox to\z@{\hss %
        \vrule\@width\lst@framewidth\@height\@tempdima}%
    \ht\@tempboxa\z@ %
    \vrule\@width\@tempdima\@height\lst@framewidth \box\@tempboxa}
%    \end{macrocode}
% \end{macro}\end{macro}\end{macro}\end{macro}
% \begingroup
%    \begin{macrocode}
%</kernel>
%    \end{macrocode}
% \endgroup
%
%
% \subsection{Pre and post listing}
%
% \begingroup
%    \begin{macrocode}
%<*kernel>
%    \end{macrocode}
% \endgroup
% \begin{aspect}{pre}
% \begin{aspect}{post}
% A rather trivial section.
%    \begin{macrocode}
\lst@Aspect{pre}{\lstKV@OptArg\lstpre@[]{#1}}
\lst@Aspect{post}{\lstKV@OptArg\lstpost@[]{#1}}
%    \end{macrocode}
%    \begin{macrocode}
\def\lstpre@[#1]#2{%
    \def\lst@prelisting{#2}\def\lst@@prelisting{#1}}
\def\lstpost@[#1]#2{%
    \def\lst@postlisting{#2}\def\lst@@postlisting{#1}}
%    \end{macrocode}
%    \begin{macrocode}
\lstset{pre={},post={}}% init
%    \end{macrocode}
% \end{aspect}\end{aspect}
% \begingroup
%    \begin{macrocode}
%</kernel>
%    \end{macrocode}
% \endgroup
%
%
% \subsection{\TeX\ comment lines}
%
% \begingroup
%    \begin{macrocode}
%<*kernel>
%    \end{macrocode}
% \endgroup
% Communication with \lsthelper{J\"orn Wilms}{wilms@rocinante.colorado.edu}{1997/07/07}{\TeX\ comments} is responsible for this feature.
% Some characters have a special functionality in \TeX{}, e.g.\ the underbar or the dollar sign.
% These meanings are unwanted while typesetting normal source code, i.e.\ we have to define our own character table.
% But for the comment lines we must interrupt the current processing and switch back to the original meanings.
% And at the end we must restore all previous things.
%
% \begin{aspect}{texcl}
% Announcing the aspect:
%    \begin{macrocode}
\lst@Aspect{texcl}[t]{\lstKV@SetIfKey\lst@iftexcl{#1}}
\lstset{texcl=false}% init
%    \end{macrocode}
% Things at EOL are easy, but we must allocate new modes for (\TeX) comment lines first.
%    \begin{macrocode}
\lst@NewMode\lst@TeXCLmode
\lst@NewMode\lst@CLmode
\lst@AddToHook{EOL}
    {\ifnum\lst@mode=\lst@TeXCLmode %
         \lst@LeaveAllModes \lst@ReenterModes %
     \fi %
     \ifnum\lst@mode=\lst@CLmode \lst@LeaveMode \fi}
%    \end{macrocode}
% \end{aspect}
%
% \begin{macro}{\lstCC@BeginCommentLine}
% \begin{macro}{\lstCC@@BeginCommentLine}
% The first macro starts comment lines indicated by a single character (like |%| in \TeX), whereas the second does this for comment lines indicated by two characters (like |//| in C++).
% According to the macro naming these two macros belong more or less to section \ref{iCharacterClasses}.
% But we present them here because of \TeX\ comment lines.
%
% We print preceding characters (if any), begin the comment and output the comment separator.
%    \begin{macrocode}
\def\lstCC@BeginCommentLine#1{%
    \lst@NewLine \lst@PrintToken %
    \lst@EnterMode{\lst@CLmode}{\lst@modetrue\lst@commentstyle}%
    #1\relax %
%    \end{macrocode}
% If the user don't want \TeX\ comment lines, there is nothing more to do.
% Otherwise we have to print the comment separator and interrupt the normal processing.
%    \begin{macrocode}
    \lst@iftexcl %
        \lst@PrintToken %
        \lst@LeaveMode \lst@InterruptModes %
        \lst@EnterMode{\lst@TeXCLmode}{\lst@modetrue\lst@commentstyle}%
    \fi}
%    \end{macrocode}
% Now comes the same, but we process a 'two character'-separator.
%    \begin{macrocode}
\def\lstCC@@BeginCommentLine#1#2{%
    \lst@NewLine \lst@PrintToken %
    \lst@EnterMode{\lst@CLmode}{\lst@modetrue\lst@commentstyle}%
    #1\relax#2\relax %
    \lst@iftexcl %
        \lst@PrintToken %
        \lst@LeaveMode \lst@InterruptModes %
        \lst@EnterMode{\lst@TeXCLmode}{\lst@modetrue\lst@commentstyle}%
    \fi}
%    \end{macrocode}
% \end{macro}\end{macro}
% \begingroup
%    \begin{macrocode}
%</kernel>
%    \end{macrocode}
% \endgroup
%
%
% \section{Doing output}
%
%
% \subsection{Output aspects and helpers}
%
% \begingroup
%    \begin{macrocode}
%<*kernel>
%    \end{macrocode}
% \endgroup
% \begin{aspect}{flexiblecolumns}
% Do you have any idea what to write here?
%    \begin{macrocode}
\lst@Aspect{flexiblecolumns}[t]{\lstKV@SetIfKey\lst@ifflexible{#1}}
\lstset{flexiblecolumns=false}% init
%    \end{macrocode}
% We assign the correct output macros defined below.
% As you can see there are three main macros, which handle letters, all other printing characters and tabulator stops.
%    \begin{macrocode}
\lst@AddToHook{BeforeSelectCharTable}
    {\lst@ifflexible %
         \let\lst@Output\lst@OutputFlexible %
         \let\lst@OutputOther\lst@OutputOtherFlexible %
         \let\lst@GotoTabStop\lst@GotoTabStopFlexible %
     \fi}
%    \end{macrocode}
% \end{aspect}
%
% \begin{aspect}{baseem}
% We look whether or not the user gives two numbers, i.e.\ we test for a comma.
%    \begin{macrocode}
\lst@Aspect{baseem}{\lstbaseem@#1,,\relax}
%    \end{macrocode}
% Here we check for legal arguments \ldots
%    \begin{macrocode}
\def\lstbaseem@#1,#2,#3\relax{%
    \def\lst@next{\PackageError{Listings}%
        {Nonnegative number(s) expected}%
        {Separate one or two such numbers by a comma, next time.^^J%
         Now type <RETURN> to proceed.}}%
    \ifdim #1em<\z@\else %
        \def\lst@baseemfixed{#1}%
        \let\lst@baseemflexible\lst@baseemfixed %
%    \end{macrocode}
% and do the comma test.
%    \begin{macrocode}
        \ifx\@empty#2\@empty %
            \let\lst@next\relax %
        \else \ifdim #2em<\z@\else %
            \def\lst@baseemflexible{#2}%
            \let\lst@next\relax %
        \fi \fi %
    \fi \lst@next}
%    \end{macrocode}
%    \begin{macrocode}
\lstset{baseem={0.6,0.45}}% init
%    \end{macrocode}
% \end{aspect}
%
% \begin{macro}{\lst@width}
% The dimension holds the width of a single character box while typesetting a listing.
%    \begin{macrocode}
\newdimen\lst@width
\lst@AddToHook{InitVars}
    {\lst@width=\lst@ifflexible\lst@baseemflexible %
                          \else\lst@baseemfixed\fi em\relax}
%    \end{macrocode}
% \end{macro}
%
% \begin{aspect}{tabsize}
% We check for a legal argument before saving it.
%    \begin{macrocode}
\lst@Aspect{tabsize}
    {\ifnum#1>\z@ %
        \def\lst@tabsize{#1}%
     \else %
         \PackageError{Listings}{Strict positive integer expected}
         {You can't use `#1' as tabulator length.^^J%
          Type <RETURN> to forget it and to proceed.}%
     \fi}
%    \end{macrocode}
% Default tabsize is 8 as proposed by \lsthelper{Rolf Niepraschk}{NIEPRASCHK@PTB.DE}{1997/04/24}{tabsize=8}.
%    \begin{macrocode}
\lstset{tabsize=8}% init
%    \end{macrocode}
% \end{aspect}
%
% Before looking at the output macros, we have to introduce some registers.
%
% \begin{macro}{\lst@token}
% \begin{macro}{\lst@length}
% The token register contains the current character string, for example |char| if we have just read these characters and a whitespace before.
% The counter |\lst@length| holds the length of the string and that's 4 in the example.
%    \begin{macrocode}
\newtoks\lst@token \newcount\lst@length
%    \end{macrocode}
%    \begin{macrocode}
\lst@AddToHook{InitVarsEOL}{\lst@token{}\lst@length\z@}
%    \end{macrocode}
% \end{macro}\end{macro}
%
% \begin{macro}{\lst@lastother}
% This is not a \TeX{} register.
% This macro is equivalent to the last 'other' character, other in the sense of section \ref{iCharacterClasses}.
%    \begin{macrocode}
\lst@AddToHook{InitVarsEOL}{\let\lst@lastother\@empty}
%    \end{macrocode}
% \end{macro}
%
% \begin{macro}{\lst@column}
% \begin{macro}{\lst@pos}
% With the two counters it is possible to determine the current column.
% It's the sum of |\lst@column| and |\lst@length| plus one minus |\lst@pos| --- |\lst@pos| will be nonpositive.
% It seems to be troublesome to decide whether a new line has just begun or not, i.e.\ if the current column number is one.
% And that's true.
%    \begin{macrocode}
\newcount\lst@column \newcount\lst@pos % \global
%    \end{macrocode}
%    \begin{macrocode}
\lst@AddToHook{InitVarsEOL}{\global\lst@pos\z@ \global\lst@column\z@}
%    \end{macrocode}
% \end{macro}\end{macro}
%
% \begin{macro}{\lst@lostspace}
% The output macros have to keep track of the difference between the real and desired (current) line width; the latter one given by 'current column times |\lst@width|'.
% More precisely, |\lst@lostspace| equals 'current column times |\lst@width|' minus 'width of so far printed line'.
% Whenever this dimension is positive we can insert space to fix the column alignment.
%    \begin{macrocode}
\newdimen\lst@lostspace % \global
%    \end{macrocode}
%    \begin{macrocode}
\lst@AddToHook{InitVarsEOL}{\global\lst@lostspace\z@}
%    \end{macrocode}
% \end{macro}
%
% \begin{macro}{\lst@UseLostSpace}
% This is a service macro for the fixed and flexible column output.
% We insert space and reset it (if and) only if |\lst@lostspace| is positive.
%    \begin{macrocode}
\def\lst@UseLostSpace{%
    \ifdim\lst@lostspace>\z@ %
        \kern\lst@lostspace \global\lst@lostspace\z@ %
    \fi}
%    \end{macrocode}
% \end{macro}
%
% \begin{macro}{\lst@InsertLostSpace}
% \begin{macro}{\lst@InsertHalfLostSpace}
% Ditto, but insert always (even if negative).
%    \begin{macrocode}
\def\lst@InsertLostSpace{\kern\lst@lostspace \global\lst@lostspace\z@}
\def\lst@InsertHalfLostSpace{%
    \global\lst@lostspace.5\lst@lostspace \kern\lst@lostspace}
%    \end{macrocode}
% \end{macro}\end{macro}
%
% \begin{aspect}{outputpos}
% Note that there are two |\relax|es \ldots
%    \begin{macrocode}
\lst@Aspect{outputpos}{\lstoutputpos@#1\relax\relax}
%    \end{macrocode}
% or an empty argument would be bad here.
% We simply test for |l|, |c| and |r|.
% If none of them is given, we issue a warning and assume |r| --- it's default since it looks most bad to me.
% The fixed column format makes use of |\lst@lefthss| and |\lst@righthss|, whereas the flexible needs only |\lst@leftinsert|.
%    \begin{macrocode}
\def\lstoutputpos@#1#2\relax{%
    \ifx #1l%
        \let\lst@lefthss\relax \let\lst@righthss\hss %
        \let\lst@leftinsert\relax %
    \else\ifx #1c%
        \let\lst@lefthss\hss \let\lst@righthss\hss %
        \let\lst@leftinsert\lst@InsertHalfLostSpace %
    \else %
        \let\lst@lefthss\hss \let\lst@righthss\relax %
        \let\lst@leftinsert\lst@InsertLostSpace %
        \ifx #1r\else \PackageWarning{Listings}%
            {Unknown positioning for output boxes}%
        \fi %
    \fi\fi}
%    \end{macrocode}
%    \begin{macrocode}
\lstset{outputpos=c}% init
%    \end{macrocode}
% \end{aspect}
% \begingroup
%    \begin{macrocode}
%</kernel>
%    \end{macrocode}
% \endgroup
%
%
% \subsection{Dropping empty lines}
%
% \begingroup
%    \begin{macrocode}
%<*kernel>
%    \end{macrocode}
% \endgroup
% \begin{macro}{\lst@NewLineMacro}
% This macro is assigned to |\lst@NewLine|, which is executed whenever the (next) line is not empty.
% Once called, it deactivates itself.
%    \begin{macrocode}
\def\lst@NewLineMacro{%
    \global\let\lst@NewLine\relax \par\noindent\hbox{}}
\lst@AddToHook{InitVars}{\global\let\lst@NewLine\lst@NewLineMacro}
%    \end{macrocode}
% What we have said about |\lst@NewLine| is not the whole truth.
% Most times we'll assign |\lst@NewLineMacro|, but sometimes we append |\par\noindent\hbox{}| to prepare a new line, namely in the case that the last line has been empty.
% Then |\lst@NewLine| deactivates itself and begins two, three or more (empty) lines.
% This drops empty lines at the end of a listing.
%    \begin{macrocode}
\lst@AddToHook{EOL}
    {\ifx\lst@NewLine\relax %
        \global\let\lst@NewLine\lst@NewLineMacro %
     \else %
         \lst@AddTo\lst@NewLine{\par\noindent\hbox{}}%
     \fi}
%    \end{macrocode}
% \end{macro}
% \begingroup
%    \begin{macrocode}
%</kernel>
%    \end{macrocode}
% \endgroup
%
%
% \subsection{Fixed columns}
%
% \begingroup
%    \begin{macrocode}
%<*kernel>
%    \end{macrocode}
% \endgroup
% \begin{macro}{\lst@OutputOther}
% This macro outputs a character string with nonletters.
% If there is anything to output, we possibly start a new line.
%    \begin{macrocode}
\def\lst@OutputOther{%
    \ifnum\lst@length=\z@\else %
        \lst@NewLine \lst@UseLostSpace %
%    \end{macrocode}
% The box must take |\lst@length| characters, each |\lst@width| wide.
%    \begin{macrocode}
        \hbox to \lst@length\lst@width{%
            \lst@lefthss %
            \lsthk@OutputOther %
            \expandafter\lst@FillOutputBox\the\lst@token\relax %
            \lst@righthss}%
%    \end{macrocode}
% Finally we hold up the current column, empty the token and close the starting 'if token not empty'.
%    \begin{macrocode}
        \global\advance\lst@pos -\lst@length %
        \lst@token{}\lst@length\z@ %
    \fi}
%    \end{macrocode}
% \end{macro}
%
% \begin{macro}{\lst@Output}
% We reset |\lst@lastother| and use |\lst@thestyle|.
%    \begin{macrocode}
\def\lst@Output{%
    \let\lst@lastother\relax %
    \ifnum\lst@length=\z@\else %
        \lst@NewLine \lst@UseLostSpace %
        \hbox to \lst@length\lst@width{%
            \lst@lefthss %
            \lsthk@Output \lst@thestyle{%
                \expandafter\lst@FillOutputBox\the\lst@token\relax}%
            \lst@righthss}%
        \global\advance\lst@pos -\lst@length %
        \lst@token{}\lst@length\z@ %
    \fi}
%    \end{macrocode}
% Note that |\lst@lastother| becomes equivalent to |\relax| and not equivalent to |\@empty| as in all other places (e.g.\ InitVarsEOL).
% I don't know whether this will be important in future or not.
% \end{macro}
%
% \begin{macro}{\lst@FillOutputBox}
% Filling up a box is easy.
% If we come to the end (the |\relax| from above), we do nothing.
% Otherwise we output the argument, insert dynamic space and call the macro again.
%    \begin{macrocode}
\def\lst@FillOutputBox#1{%
    \ifx\relax#1\else #1\hss\expandafter\lst@FillOutputBox \fi}
%    \end{macrocode}
% \end{macro}
%
% \begin{macro}{\lst@GotoTabStop}
% For fixed column format we only need to advance |\lst@lostspace| (which is inserted by the output macros above) and update the column.
%    \begin{macrocode}
\def\lst@GotoTabStop{%
    \global\advance\lst@lostspace \lst@length\lst@width %
    \global\advance\lst@column\lst@length \lst@length\z@}%
%    \end{macrocode}
% \end{macro}
% \begingroup
%    \begin{macrocode}
%</kernel>
%    \end{macrocode}
% \endgroup
%
%
% \subsection{Flexible columns}
%
% \begingroup
%    \begin{macrocode}
%<*kernel>
%    \end{macrocode}
% \endgroup
% \begin{macro}{\lst@OutputOtherFlexible}
% If there is something to output, we first insert the space lost by the flexible column format.
% Then we typeset the box and update the lost space.
% Note that we don't use any |\hss| here.
%    \begin{macrocode}
\def\lst@OutputOtherFlexible{%
    \ifnum\lst@length=\z@\else %
        \lst@NewLine \lst@UseLostSpace %
        \setbox\@tempboxa\hbox{\lsthk@OutputOther\the\lst@token}%
        \lst@CalcLostSpaceAndOutput %
    \fi}
%    \end{macrocode}
% \end{macro}
%
% \begin{macro}{\lst@OutputFlexible}
% Nothing is new here.
%    \begin{macrocode}
\def\lst@OutputFlexible{%
    \let\lst@lastother\relax %
    \ifnum\lst@length=\z@\else %
        \lst@NewLine \lst@UseLostSpace %
        \setbox\@tempboxa\hbox{%
            \lsthk@Output \lst@thestyle{\the\lst@token}}%
        \lst@CalcLostSpaceAndOutput %
    \fi}
%    \end{macrocode}
% \end{macro}
%
% \begin{macro}{\lst@GotoTabStopFlexible}
% Here we look whether or not the line already contains printing characters.
%    \begin{macrocode}
\def\lst@GotoTabStopFlexible{%
    \ifx\lst@NewLine\relax %
%    \end{macrocode}
% If some characters are already printed, we output a box, which has the width of a blank space.
% Possibly more space is inserted, but that's upto the current value of |\lst@lostspace|.
%    \begin{macrocode}
        \setbox\@tempboxa\hbox{\lst@outputblank}\@tempdima\wd\@tempboxa%
        \setbox\@tempboxa\hbox{}\wd\@tempboxa\@tempdima %
        \lst@CalcLostSpaceAndOutput %
        \global\lst@pos\z@ %
    \else %
%    \end{macrocode}
% Otherwise (no printed characters) we do the same as for fixed columns.
%    \begin{macrocode}
        \global\advance\lst@lostspace \lst@length\lst@width %
        \global\advance\lst@column\lst@length \lst@length\z@ %
    \fi}
%    \end{macrocode}
% \end{macro}
%
% \begin{macro}{\lst@CalcLostSpaceAndOutput}
% The update of |\lst@lostspace| is simple, refer its definition above (difference between \ldots).
%    \begin{macrocode}
\def\lst@CalcLostSpaceAndOutput{%
    \global\advance\lst@lostspace \lst@length\lst@width %
    \global\advance\lst@lostspace-\wd\@tempboxa %
%    \end{macrocode}
% Moreover we keep track of |\lst@pos| and reset some variables.
%    \begin{macrocode}
    \global\advance\lst@pos -\lst@length %
    \lst@token{}\lst@length\z@ %
%    \end{macrocode}
% Before |\@tempboxa| is output, we insert appropiate space if there is enough lost space.
%    \begin{macrocode}
    \ifdim\lst@lostspace>\z@ \lst@leftinsert \fi %
    \box\@tempboxa}
%    \end{macrocode}
% \end{macro}
% \begingroup
%    \begin{macrocode}
%</kernel>
%    \end{macrocode}
% \endgroup
%
%
% \subsection{Dropping the whole output}
%
% \begingroup
%    \begin{macrocode}
%<*kernel>
%    \end{macrocode}
% \endgroup
% \begin{macro}{\lst@BeginDropOutput}
% It's sometimes useful to process a part of a listing as usual, but to drop the output.
% This macro does the main work and gets one argument, namely the internal mode it enters.
% We save |\lst@NewLine|, restore it |\aftergroup| and redefine the output macros.
%    \begin{macrocode}
\def\lst@BeginDropOutput#1{%
    \let\lst@BDOsave\lst@NewLine %
    \lst@EnterMode{#1}%
        {\lst@modetrue %
         \let\lst@Output\lst@EmptyOutput %
         \let\lst@OutputOther\lst@EmptyOutputOther %
         \let\lst@GotoTabStop\lst@EmptyGotoTabStop %
         \aftergroup\lst@BDORestore}}
%    \end{macrocode}
% Restoring |\lst@NewLine| is quite easy:
%    \begin{macrocode}
\def\lst@BDORestore{\global\let\lst@NewLine\lst@BDOsave}
%    \end{macrocode}
% Note that there is no |\lst@EndDropOutput| since this macro would be equivalent to |\lst@LeaveMode|.
% \end{macro}
%
% \begin{macro}{\lst@EmptyOutputOther}
% \begin{macro}{\lst@EmptyOutput}
% \begin{macro}{\lst@EmptyGotoTabStop}
% Here we only keep track of registers (possibly) needed by other processing macros.
%    \begin{macrocode}
\def\lst@EmptyOutputOther{%
    \global\advance\lst@pos -\lst@length %
    \lst@token{}\lst@length\z@}
%    \end{macrocode}
%    \begin{macrocode}
\def\lst@EmptyOutput{\let\lst@lastother\relax \lst@EmptyOutputOther}
%    \end{macrocode}
%    \begin{macrocode}
\def\lst@EmptyGotoTabStop{%
    \global\advance\lst@column\lst@length \lst@length\z@}
%    \end{macrocode}
% \end{macro}\end{macro}\end{macro}
% \begingroup
%    \begin{macrocode}
%</kernel>
%    \end{macrocode}
% \endgroup
%
%
% \subsection{Writing to an external file}
%
% The macros are defined (if and) only if the |doc| option is used.
% \begingroup
%    \begin{macrocode}
%<*kernel>
\@ifundefined{lst@doc}{}{%
%    \end{macrocode}
% \endgroup
%
% \begin{macro}{\lstdoc@out}
% The file we will write to.
%    \begin{macrocode}
\newwrite\lstdoc@out
%    \end{macrocode}
% \end{macro}
%
% \begin{macro}{\lst@BeginWriteFile}
% redefines some macros to meet our purpose.
% The file with name |#1| is opened at the end of the macro.
%    \begin{macrocode}
\def\lst@BeginWriteFile#1{%
    \begingroup %
    \lsthk@SetLanguage %
    \let\lstCC@ifec\iffalse %
    \let\lst@Output\lstdoc@Output %
    \let\lst@OutputOther\lstdoc@Output %
    \let\lst@GotoTabStop\lstdoc@GotoTabStop %
    \let\lstCC@ProcessSpace\lstdoc@ProcessSpace %
    \let\lst@MProcessListing\lstdoc@MProcessListing %
    \let\smallbreak\relax %
    \let\lst@prelisting\relax \let\lst@postlisting\relax %
    \let\lst@@prelisting\relax \let\lst@@postlisting\relax %
    \let\lstCC@Use\lstdoc@Use %
    \let\lst@DeInit\lstdoc@DeInit %
    \immediate\openout\lstdoc@out=#1\relax}
%    \end{macrocode}
% \end{macro}
%
% \begin{macro}{\lst@EndWriteFile}
% closes the file and restores original macro meanings.
%    \begin{macrocode}
\def\lst@EndWriteFile{\immediate\closeout\lstdoc@out \endgroup}
%    \end{macrocode}
% \end{macro}
%
% \begin{macro}{\lstdoc@Output}
% keeps only track of horizontal position.
%    \begin{macrocode}
\def\lstdoc@Output{\global\advance\lst@pos -\lst@length \lst@length\z@}
%    \end{macrocode}
% \end{macro}
%
% \begin{macro}{\lstdoc@ProcessSpace}
% \begin{macro}{\lstdoc@GotoTabStop}
% We append the appropiate number of spaces.
% Note that |\lstCC@Append| increases |\lst@length| by 1, thus we need -2.
%    \begin{macrocode}
\def\lstdoc@ProcessSpace{\lstCC@Append{ }}
\def\lstdoc@GotoTabStop{%
    \@whilenum \lst@length>\z@ \do %
    {\lstCC@Append{ }\advance\lst@length-2\relax}}
%    \end{macrocode}
% \end{macro}\end{macro}
%
% \begin{macro}{\lstdoc@MProcessListing}
% writes one line to external file.
%    \begin{macrocode}
\def\lstdoc@MProcessListing{%
    \immediate\write\lstdoc@out{\the\lst@token}%
    \lst@token{}\lst@length\z@ %
    \lst@BOLGobble}
%    \end{macrocode}
% \end{macro}
%
% \begin{macro}{\lstdoc@DeInit}
% We write the rest to file and end processing.
%    \begin{macrocode}
\def\lstdoc@DeInit{%
    \ifnum\lst@length=\z@\else %
        \immediate\write\lstdoc@out{\the\lst@token}%
    \fi %
    \egroup \smallbreak\lst@postlisting \endgroup}
%    \end{macrocode}
% \end{macro}
%
% \begin{macro}{\lstdoc@Use}
% Any processed character appends the character itself to |\lst@token| (but with catcode 12 or 10).
% The original |\lstCC@Use| is defined in section \ref{iCharacterTables}.
%    \begin{macrocode}
\def\lstdoc@Use#1#2#3{%
    \ifnum#2=\z@ %
        \expandafter\@gobbletwo %
    \else %
        \catcode#2=\active \lccode`\~=#2\lccode`\/=#2%
        \lowercase{\def~{\lstCC@Append{/}}}%
    \fi %
    \lstdoc@Use#1}
%    \end{macrocode}
% \end{macro}
%
% \begingroup
%    \begin{macrocode}
}% of \@ifundefined{lst@doc}{}{%
%</kernel>
%    \end{macrocode}
% \endgroup
%
%
% \subsection{Keyword comments}
%
% \begingroup
%    \begin{macrocode}
%<*keywordcomments>
%    \end{macrocode}
% \endgroup
% \begin{aspect}{keywordcomment}
% \begin{aspect}{doublekeywordcommentsemicolon}
% The same stuff as for the other comment commands.
%    \begin{macrocode}
\lst@Aspect{keywordcomment}
    {\lst@MakeKeywordArg{#1}\let\lst@KCkeywords\lst@arg %
     \let\lst@DefKC\lstCC@KeywordComment}
%    \end{macrocode}
%    \begin{macrocode}
\lst@Aspect{doublekeywordcommentsemicolon}{\lstDKCS@#1}
\gdef\lstDKCS@#1#2#3%
    {\lst@MakeKeywordArg{#1}\let\lst@KCAkeywordsB\lst@arg %
     \lst@MakeKeywordArg{#2}\let\lst@KCAkeywordsE\lst@arg %
     \lst@MakeKeywordArg{#3}\let\lst@KCBkeywordsB\lst@arg %
     \let\lst@DefKC\lstCC@DoubleKeywordCommentS}
%    \end{macrocode}
%    \begin{macrocode}
\lst@AddToHook{SelectCharTable}{\lst@DefKC}
\lst@AddToHook{BeforeSelectCharTable}
    {\lst@ifsensitive\else %
         \lst@MakeMacroUppercase\lst@KCkeywords %
         \lst@MakeMacroUppercase\lst@KCAkeywordsB %
         \lst@MakeMacroUppercase\lst@KCAkeywordsE %
         \lst@MakeMacroUppercase\lst@KCBkeywordsB %
     \fi}
\lst@AddToHook{SetLanguage}{%
    \let\lst@DefKC\relax \let\lst@KCkeywords\@undefined %
    \let\lst@KCAkeywordsB\@undefined \let\lst@KCAkeywordsE\@undefined %
    \let\lst@KCBkeywordsB\@undefined}
%    \end{macrocode}
% \end{aspect}\end{aspect}
%
% \begin{macro}{\lstCC@KeywordComment}
% For this type of keyword comments we save the old output macro and install a new one.
% Note that |\lstCC@KeywordComment| is executed after selecting the character table via |\lst@DefKC|.
%    \begin{macrocode}
\gdef\lstCC@KeywordComment{%
    \let\lst@Output@\lst@Output \let\lst@Output\lst@KCOutput}
%    \end{macrocode}
% \end{macro}
%
% \begin{macro}{\lst@KCOutput}
% And now we look how the output works here.
% It starts as all the time.
% But if the current character sequence in |\lst@token| is one of the given keywords, we call a macro which starts and ends keyword comments.
% |\lst@next| is redefined there.
% After doing all this, we output the token as usual and go on.
%    \begin{macrocode}
\gdef\lst@KCOutput{%
    \ifnum\lst@length=\z@\else %
        \let\lst@next\relax %
        \expandafter\lst@IfOneOf \the\lst@token\relax \lst@KCkeywords %
            {\lst@KCOutput@\lst@BeginKC}{}%
        \lst@Output@ \lst@next %
    \fi}
%    \end{macrocode}
% \end{macro}
%
% \begin{macro}{\lst@KCOutput@}
% Now we are in the situation that the current token is attached to a keyword comment.
% By default |\lst@next| becomes equivalent to the first argument, which is either |\lst@BeginKC| or |\lst@BeginDKCA| or |\lst@BeginDKCB|.
% Moreover we save the current token.
%    \begin{macrocode}
\gdef\lst@KCOutput@#1{\let\lst@next#1%
    \expandafter\def\expandafter\lst@save\expandafter{\the\lst@token}%
%    \end{macrocode}
% If we are not in 'keyword comment mode', nothing else is done here.
% But if we are, we must end the comment.
% The problem: Closing the comment group also ruins the current character string in |\lst@token|.
% The solution: We define a global macro to restore the token and |\lst@length|.
% |\@gtempa| becomes
% \begin{itemize}\item[]
%	|\lst@token{|\meta{current character string}|}\lst@length|\meta{current length}|\relax|
% \end{itemize}
%    \begin{macrocode}
    \lst@ifmode \ifnum\lst@mode=\lst@KCmode %
        \xdef\@gtempa{%
            \noexpand\lst@token{\the\lst@token}%
            \noexpand\lst@length\the\lst@length\relax}%
%    \end{macrocode}
% This macro is executed after closing the comment group.
% We redefine |\lst@next| just to |\relax|.
%    \begin{macrocode}
        \aftergroup\@gtempa \lst@LeaveMode \let\lst@next\relax %
    \fi \fi}
%    \end{macrocode}
% \end{macro}
%
% \begin{macro}{\lst@BeginKC}
% We call |\lstCC@BeginComment| and define |\lst@KCkeywords| to be the current token (possibly made upper case) since we want a matching keyword.
% Note: It's a local definition, i.e.\ after ending the comment all comment starting keywords are restored.
%    \begin{macrocode}
\gdef\lst@BeginKC{%
    \lstCC@BeginComment\lst@KCmode \let\lst@KCkeywords\lst@save %
    \lst@ifsensitive\else \lst@MakeMacroUppercase\lst@KCkeywords \fi}
%    \end{macrocode}
% \end{macro}
%
% \begin{macro}{\lstCC@DoubleKeywordCommentS}
% Let's look at the next macro collection.
% The first macro is the same as the first from above --- except that we must install a different semicolon, which is done at the very beginning.
%    \begin{macrocode}
\gdef\lstCC@DoubleKeywordCommentS{%
    \lstCC@EndKeywordComment{"003B}%
    \let\lst@Output@\lst@Output \let\lst@Output\lst@DKCOutput}
%    \end{macrocode}
% \end{macro}
%
% \begin{macro}{\lst@DKCOutput}
% The second macro is also the same as above, but if we haven't found a keyword from the |KCA| list, we try the |KCB| list.
%    \begin{macrocode}
\gdef\lst@DKCOutput{%
    \ifnum\lst@length=\z@\else %
        \let\lst@next\relax %
        \expandafter\lst@IfOneOf \the\lst@token\relax\lst@KCAkeywordsB %
            {\lst@KCOutput@\lst@BeginDKCA}{%
        \expandafter\lst@IfOneOf \the\lst@token\relax\lst@KCBkeywordsB %
            {\lst@KCOutput@\lst@BeginDKCB}{}}%
        \lst@Output@ \lst@next %
    \fi}
%    \end{macrocode}
% \end{macro}
%
% \begin{macro}{\lst@BeginDKCA}
% \begin{macro}{\lst@BeginDKCB}
% To begin a keyword comment, we assign appropiate lists of keywords, which might end the comment.
%    \begin{macrocode}
\gdef\lst@BeginDKCA{\lstCC@BeginComment\lst@KCmode %
    \let\lst@KCAkeywordsB\lst@KCAkeywordsE \let\lst@KCBkeywordsB\@empty}
%    \end{macrocode}
%    \begin{macrocode}
\gdef\lst@BeginDKCB{\lstCC@BeginComment\lst@KCmode %
    \let\lst@KCAkeywordsB\@empty \let\lst@KCBkeywordsB\@empty}
%    \end{macrocode}
% \end{macro}\end{macro}
%
% \begin{macro}{\lstCC@EndKeywordComment}
% Above we've installed a 'end keyword comment' semicolon.
% Before reading further you should be familiar with section \ref{iCharacterTables}.
% Roughly speaking we define commands there, which annouces the source code character $0041_{\mathrm{hex}}$='A' to be the upper case letter 'A'.
% The macro here announces a character to end keyword comments.
%    \begin{macrocode}
\gdef\lstCC@EndKeywordComment#1{%
    \lccode`\~=#1\lowercase{\lstCC@EndKeywordComment@~}{#1}}
%    \end{macrocode}
% Looking at the submacro, |#1| is an active character with ASCII code |#2|.
% We must save a previous meaning of |#1| --- or we couldn't output the character since we've forgotten it.
% Afterwards we can redefine it: If we are in comment mode and furthermore in keyword comment mode, the 'end keyword comment' character ends the comment, as desired.
% Note that we output the old meaning of the character first.
%    \begin{macrocode}
\gdef\lstCC@EndKeywordComment@#1#2{%
    \expandafter\let\csname lstCC@EKC#2\endcsname#1%
    \def#1{%
        \ifnum\lst@mode=\lst@KCmode %
            \let\lstCC@next\lstCC@EndComment %
        \else %
            \let\lstCC@next\relax %
        \fi %
        \csname lstCC@EKC#2\endcsname \lstCC@next}}
%    \end{macrocode}
% \end{macro}
% \begingroup
%    \begin{macrocode}
%</keywordcomments>
%    \end{macrocode}
% \endgroup
%
%
% \section{Character classes}\label{iCharacterClasses}
%
% \begingroup
%    \begin{macrocode}
%<*kernel>
%    \end{macrocode}
% \endgroup
% \TeX{} knows sixteen category codes.
% We define our own character classes here.
% Each input character becomes active in the sense of \TeX{} and characters of different classes expand to different meanings.
% We use the following ones:
% \begin{itemize}
% \item letters  ---  characters identifiers are of;
% \item digits  ---  characters for identifiers or numerical constants;
% \item spaces  ---  characters treated as blank spaces;
% \item tabulators  ---  characters treated as tabulators;
% \item stringizers  ---  characters beginning and ending strings;
% \item comment indicators  ---  characters or character sequences beginning and ending comments;
% \item special classes support particular programming languages;
% \item others  ---  all other characters.
% \end{itemize}
% How these classes work together?
% The digit '3' appends the digit to the current character string, e.g.\ |ear| becomes |ear3|.
% The next nonletter causes the output of the gathered characters.
% Then we collect all coming nonletters until reaching a letter again.
% This causes the output of the nonletters, and so on.
%
% But there are more details.
% Stringizers and comment indicators change the processing mode until the string or the comment is over.
% For example, no keyword tests are done within a string or comment.
% A tabulator immediately outputs the gathered characters, without looking whether they are letters or not.
% Afterwards it is possible to determine the tabulator skip.
% And there is one thing concerning spaces:
% Many spaces following each other disturb the column alignment since they are not wide enough.
% Hence, when column alignment is on, we output the space(s) preceding a space.
%
% The above classes can be divided into three types:
% \begin{itemize}
% \item The 'letter', 'digit' and 'other' class all put characters into the 'output queue' (using |\lstCC@Append| or |\lstCC@AppendOther|, see below).
% \item	Stringizer, comment indicators and special classes don't affect the output queue directly.
%	For example, before a character becomes a stringizer, the 'letter', 'digit' or 'other' meaning is saved.
%	Now the stringizer accesses the output queue via this saved meaning.
% \item Spaces and tabulators don't put any character into the output queue, but may affect the queue to do their job, as mentioned above.
%	Instances of these classes are not overwritten (in contrast to letters, digits and others).
% \end{itemize}
% Some easy implementation before looking closer \ldots
%
% \begin{macro}{\lstCC@Append}
% This macro appends the argument to the current character string and increases the counter |\lst@length|.
%    \begin{macrocode}
\def\lstCC@Append#1{\advance\lst@length\@ne %
    \expandafter\lst@token\expandafter{\the\lst@token#1}}
%    \end{macrocode}
% \end{macro}
%
% \begin{macro}{\lstCC@AppendOther}
% Nearly the same, but we save the argument (a single character or a single macro) in |\lst@lastother|.
%    \begin{macrocode}
\def\lstCC@AppendOther#1{\advance\lst@length\@ne \let\lst@lastother#1%
    \expandafter\lst@token\expandafter{\the\lst@token#1}}
%    \end{macrocode}
% \end{macro}
%
% \begin{macro}{\lstCC@ifletter}
% This \texttt{if} indicates whether the last character has been a letter or not.
%    \begin{macrocode}
\def\lstCC@lettertrue{\let\lstCC@ifletter\iftrue}
\def\lstCC@letterfalse{\let\lstCC@ifletter\iffalse}
%    \end{macrocode}
%    \begin{macrocode}
\lst@AddToHook{InitVars}{\lstCC@letterfalse}
%    \end{macrocode}
% \end{macro}
%
% \begin{macro}{\lst@PrintToken}
% This macro outputs the current character string in letter or nonletter mode.
%    \begin{macrocode}
\def\lst@PrintToken{%
    \lstCC@ifletter %
        \lst@Output\lstCC@letterfalse %
    \else %
        \lst@OutputOther \let\lst@lastother\@empty %
    \fi}
%    \end{macrocode}
% \end{macro}
% \begingroup
%    \begin{macrocode}
%</kernel>
%    \end{macrocode}
% \endgroup
%
%
% \subsection{Character tables}\label{iCharacterTables}
%
% \begingroup
%    \begin{macrocode}
%<*kernel>
%    \end{macrocode}
% \endgroup
% Before looking at interesting character classes we do some rather tedious coding.
% Consider a source file of a programming language now.
% For example, the character $41_{\mathrm{hex}}=65_{\mathrm{dec}}$ usually belongs to the letter class and represents the letter 'A'.
% The listings package says |\lstCC@Use\lstCC@ProcessLetter|\ldots|{"41}{A}|\ldots\ to make this clear.
% Roughly speaking it expands to |\def A{\lstCC@ProcessLetter A}|, but the first 'A' is active and the second not.
%
% \begin{macro}{\lstCC@Def}
% \begin{macro}{\lstCC@Let}
% For speed we won't used these helpers too often.
% The letter 'A' definition from above could be achieved via |\lstCC@Def{"41}{\lstCC@ProcessLetter A}|.
%    \begin{macrocode}
\def\lstCC@Def#1{\catcode#1=\active\lccode`\~=#1\lowercase{\def~}}
\def\lstCC@Let#1{\catcode#1=\active\lccode`\~=#1\lowercase{\let~}}
%    \end{macrocode}
% \end{macro}\end{macro}
%
% \begin{macro}{\lst@SelectCharTable}
% Each input character becomes active and gets the correct meaning here.
% We change locally the catcode of the double quote for compatibility with \texttt{german.sty}.
% Note that some macros take one argument and others a series of arguments which is terminated by |{"00}{}| (or an equivalent expression).
%    \begin{macrocode}
\begingroup \catcode`\"=12
\gdef\lst@SelectCharTable{%
    \lstCC@Tabulator{"09}%
    \lstCC@Space{"20}%
    \lstCC@Use \lstCC@ProcessOther %
        {"21}!{"22}"{"23}\#{"25}\%{"26}\&{"27}'{"28}({"29})%
        {"2A}\lst@asterisk{"2B}+{"2C},{"2D}\lst@minus{"2E}.%
        {"2F}/{"3A}:{"3B};{"3C}\lst@less{"3D}={"3E}\lst@greater{"3F}?%
        {"5B}[{"5C}\lst@backslash{"5D}]{"5E}\textasciicircum{"60}{`}%
        {"7B}\lst@lbrace{"7C}\lst@bar{"7D}\lst@rbrace %
        {"7E}\textasciitilde{"7F}-%
        \z@\@empty %
    \lstCC@Use \lstCC@ProcessDigit %
        {"30}0{"31}1{"32}2{"33}3{"34}4{"35}5{"36}6{"37}7{"38}8{"39}9%
        \z@\@empty %
    \lstCC@Use \lstCC@ProcessLetter %
        {"24}\lst@dollar{"40}@%
        {"41}A{"42}B{"43}C{"44}D{"45}E{"46}F{"47}G{"48}H{"49}I{"4A}J%
        {"4B}K{"4C}L{"4D}M{"4E}N{"4F}O{"50}P{"51}Q{"52}R{"53}S{"54}T%
        {"55}U{"56}V{"57}W{"58}X{"59}Y{"5A}Z{"5F}\lst@underscore %
        {"61}a{"62}b{"63}c{"64}d{"65}e{"66}f{"67}g{"68}h{"69}i{"6A}j%
        {"6B}k{"6C}l{"6D}m{"6E}n{"6F}o{"70}p{"71}q{"72}r{"73}s{"74}t%
        {"75}u{"76}v{"77}w{"78}x{"79}y{"7A}z%
        \z@\@empty %
%    \end{macrocode}
% Define extended characters 128--255.
%    \begin{macrocode}
    \lstCC@ifec \lstCC@DefEC \fi %
%    \end{macrocode}
% Finally we call some (hook) macros and initialize the backslash if necessary.
%    \begin{macrocode}
    \lsthk@SelectCharTable %
    \csname lstSCT@\lst@language\endcsname %
    \csname lstSCT@\lst@language @\lst@dialect\endcsname %
    \ifx\lstCC@Backslash\relax\else %
        \lccode`\~="5C\lowercase{\let\lsts@backslash~}%
        \lstCC@Let{"5C}\lstCC@Backslash %
    \fi}
%    \end{macrocode}
% \end{macro}
%
% \begin{macro}{\lstCC@DefEC}
% Currently each character in the range 128--255 is treated as a letter.
%    \begin{macrocode}
\catcode`\^^@=12
\gdef\lstCC@DefEC{%
    \lstCC@ECUse \lstCC@ProcessLetter %
      ^^80^^81^^82^^83^^84^^85^^86^^87^^88^^89^^8a^^8b^^8c^^8d^^8e^^8f%
      ^^90^^91^^92^^93^^94^^95^^96^^97^^98^^99^^9a^^9b^^9c^^9d^^9e^^9f%
      ^^a0^^a1^^a2^^a3^^a4^^a5^^a6^^a7^^a8^^a9^^aa^^ab^^ac^^ad^^ae^^af%
      ^^b0^^b1^^b2^^b3^^b4^^b5^^b6^^b7^^b8^^b9^^ba^^bb^^bc^^bd^^be^^bf%
      ^^c0^^c1^^c2^^c3^^c4^^c5^^c6^^c7^^c8^^c9^^ca^^cb^^cc^^cd^^ce^^cf%
      ^^d0^^d1^^d2^^d3^^d4^^d5^^d6^^d7^^d8^^d9^^da^^db^^dc^^dd^^de^^df%
      ^^e0^^e1^^e2^^e3^^e4^^e5^^e6^^e7^^e8^^e9^^ea^^eb^^ec^^ed^^ee^^ef%
      ^^f0^^f1^^f2^^f3^^f4^^f5^^f6^^f7^^f8^^f9^^fa^^fb^^fc^^fd^^fe^^ff%
      ^^00}
\endgroup
%    \end{macrocode}
% \end{macro}
%
% \begin{aspect}{extendedchars}
% The user aspect.
%    \begin{macrocode}
\lst@Aspect{extendedchars}[t]{\lstKV@SetIfKey\lstCC@ifec{#1}}
\lstset{extendedchars=false}% init
%    \end{macrocode}
% \end{aspect}
%
% \begin{macro}{\lstCC@Use}
% Now we define the |\lstCC@|\ldots|Use| macros used above.
% We either define the character with code |#2| to be |#1#3| (where |#1| is |\lstCC@ProcessLetter| for example) or we gobble the two token after |\fi| to terminate the loop.
%    \begin{macrocode}
\def\lstCC@Use#1#2#3{%
    \ifnum#2=\z@ %
        \expandafter\@gobbletwo %
    \else %
        \catcode#2=\active \lccode`\~=#2\lowercase{\def~}{#1#3}%
    \fi %
    \lstCC@Use#1}
%    \end{macrocode}
% Limitation: Each second argument to |\lstCC@Use| (beginning with the third one) must be exactly one character or one control sequence.
% The reason is not the definition here.
% |\lst@AppendOther| says |\let\lst@lastother#1| and that works with the mentioned limitation only.
% I'll get over this since |\let| is faster than |\def|, and that's the motivation.
% \emph{Caution}: Don't change that |\let| to a definition with |\def|.
% All |\ifx\lst@lastother|\ldots{} in all character classes wouldn't work any more!
% And rewriting all this would slow down the package.
% Beware of that.
% \end{macro}
%
% \begin{macro}{\lstCC@ECUse}
% Here we have two arguments only, namely |\lstCC@ProcessLetter|,\ldots\ and the character (and no character code).
% Reaching end of list (|^^00|) we terminate the loop.
% Otherwise we do the same as in |\lstCC@Use| if the character is not active.
% But if the character is active, we save the meaning before redefinition.
%    \begin{macrocode}
\def\lstCC@ECUse#1#2{%
    \ifnum`#2=\z@ %
        \expandafter\@gobbletwo %
    \else %
        \ifnum\catcode`#2=\active %
            \lccode`\~=`#2\lccode`\/=`#2\lowercase{\lstCC@ECUse@#1~/}%
        \else %
            \catcode`#2=\active \lccode`\~=`#2\lowercase{\def~}{#1#2}%
        \fi %
    \fi %
    \lstCC@ECUse#1}
%    \end{macrocode}
% As mentioned, we save the meaning before redefinition.
%    \begin{macrocode}
\def\lstCC@ECUse@#1#2#3{%
   \expandafter\let\csname lsts@EC#3\endcsname #2%
   \edef#2{\noexpand#1\expandafter\noexpand\csname lsts@EC#3\endcsname}}
%    \end{macrocode}
% \end{macro}
%
% \begin{macro}{\lstCC@Tabulator}
% \begin{macro}{\lstCC@Space}
% The similar space and tabulator macros have one argument only (no loop and no macro or character).
%    \begin{macrocode}
\def\lstCC@Tabulator#1{\lstCC@Let{#1}\lstCC@ProcessTabulator}
\def\lstCC@Space#1{\lstCC@Let{#1}\lstCC@ProcessSpace}
%    \end{macrocode}
% \end{macro}\end{macro}
%
% \begin{macro}{\lstCC@ChangeBasicClass}
% Finally we define a macro (collection) with the ability of moving a basic \emph{instance} of class to another class, e.g. making the digit '0' be a letter.
% The word 'basic' means any instance defined with |\lstCC@Use| or |\lstCC@ECUse|.
% The first argument of |\lstCC@ChangeBasicClass| gives the new class and the second argument is a macro containing a character sequence.
% These characters are changed.
% First we make all these characters active and call a submacro.
%    \begin{macrocode}
\def\lstCC@ChangeBasicClass#1#2{%
    \ifx\@empty#2\else %
        \expandafter\lst@MakeActive\expandafter{#2}%
        \expandafter\lstCC@CBC@\expandafter#1\lst@arg\@empty %
    \fi}
%    \end{macrocode}
% The submacro terminates (gobble two token after |\fi|) if it reaches |\@empty|.
% Otherwise another submacro redfines the character.
% Note that we expand the old meaning before calling the second submacro and that the second |\lstCC@CBC@@| is used as 'meaning' delimiter.
%    \begin{macrocode}
\def\lstCC@CBC@#1#2{%
    \ifx\@empty#2%
        \expandafter\@gobbletwo %
    \else %
        \expandafter\lstCC@CBC@@#2\lstCC@CBC@@#1#2%
    \fi %
    \lstCC@CBC@#1}
%    \end{macrocode}
%    \begin{macrocode}
\def\lstCC@CBC@@#1#2\lstCC@CBC@@#3#4{\def#4{#3#2}}
%    \end{macrocode}
% \end{macro}
% \begingroup
%    \begin{macrocode}
%</kernel>
%    \end{macrocode}
% \endgroup
%
%
% \subsection{Letters, digits and others}
%
% \begingroup
%    \begin{macrocode}
%<*kernel>
%    \end{macrocode}
% \endgroup
% \begin{macro}{\lstCC@ProcessLetter}
% To process a letter we look at the last character.
% If it hasn't been a letter, we output the preceding other characters first and switch to letter mode.
% Finally we append the current letter |#1|.
%    \begin{macrocode}
\def\lstCC@ProcessLetter#1{%
    \lstCC@ifletter\else \lst@OutputOther\lstCC@lettertrue \fi %
    \lstCC@Append{#1}}
%    \end{macrocode}
% \end{macro}
%
% \begin{macro}{\lstCC@ProcessOther}
% 'Other' characters are the other way round.
% If the last character has been a letter, the preceding letters are output and we switch to nonletter mode.
% Finally we append the current other character.
%    \begin{macrocode}
\def\lstCC@ProcessOther#1{%
    \lstCC@ifletter \lst@Output\lstCC@letterfalse \fi %
    \lstCC@AppendOther{#1}}
%    \end{macrocode}
% \end{macro}
%
% \begin{macro}{\lstCC@ProcessDigit}
% A digit simply appends the character to the current character string.
% But we must use the right macro.
% This allow digits to be part of an identifier or a numerical constant.
%    \begin{macrocode}
\def\lstCC@ProcessDigit#1{%
    \lstCC@ifletter \lstCC@Append{#1}\else \lstCC@AppendOther{#1}\fi}
%    \end{macrocode}
% \end{macro}
% \begingroup
%    \begin{macrocode}
%</kernel>
%    \end{macrocode}
% \endgroup
%
%
% \subsection{Tabulators and spaces}
%
% \begingroup
%    \begin{macrocode}
%<*kernel>
%    \end{macrocode}
% \endgroup
% Here we have to take care of two things:
% First dropping empty lines at the end of a listing, and second the flexible column format.
% In both cases we use |\lst@lostspace| for the implementation.
% Whenever this dimension is positive we insert that space before |\lst@token| is output.
%
% We've defined |\lst@EOLUpdate| to drop empty lines at the end of a listing.
% |\lst@NewLine| isn't executed or reset in that case.
% Instead we append control sequences.
% But: Lines containing tabulators and spaces only should also be viewed as empty.
% In order to achieve this tabulators and spaces at the beginning of a line advance |\lst@lostspace| and don't output any characters.
% The space is inserted if there comes a letter for example.
% If there are only tabulators and spaces, the line is 'empty' since we haven't done any output.
%
% We have to do more for flexible columns.
% The whitespaces can fix the column alignment:
% If the real line is wider than it should be (|\lst@lostspace|$<$0pt), a tabulator is at least one space wide; all the other width is used to make |\lst@lostspace| more positive.
% Spaces do the same: If there are two or more spaces, at least one space is printed; the others fix the column alignment.
% If we process a string, all spaces are output, of course.
%
% \begin{macro}{\lstCC@ProcessTabulator}
% A tabulator outputs the preceding characters.
%    \begin{macrocode}
\def\lstCC@ProcessTabulator{%
    \lst@PrintToken %
%    \end{macrocode}
% Then we must calculate how many columns we need to reach the next tabulator stop.
% Each printed character decrements the counter |\lst@pos|.
% Hence we can simply add |\lst@tabsize| until |\lst@pos| is strict positive.
% That's all.
% We assign it to |\lst@length|, reset |\lst@pos| \ldots
%    \begin{macrocode}
    \global\advance\lst@column -\lst@pos %
    \@whilenum \lst@pos<\@ne \do %
        {\global\advance\lst@pos\lst@tabsize}%
    \lst@length\lst@pos \global\lst@pos\z@ %}
%    \end{macrocode}
% and go to the tabulator stop, e.g.\ |\lst@length| columns forward:
%    \begin{macrocode}
    \lst@GotoTabStop}
%    \end{macrocode}
% \end{macro}
%
% \begin{macro}{\lstCC@AppendSpecialSpace}
% Sometimes we have special spaces:
% If there are at least two spaces, i.e.\ if the last character have been a space, we output preceding characters and advance |\lst@lostspace| to avoid alignment problems.
% Otherwise we append a space to the current character string.
% We'll need that macro soon.
%    \begin{macrocode}
\def\lstCC@AppendSpecialSpace{%
    \ifx\lst@lastother\lst@outputblank %
        \lst@OutputOther %
        \global\advance\lst@lostspace\lst@width %
        \global\advance\lst@pos\m@ne %
    \else %
        \lstCC@AppendOther\lst@outputblank %
    \fi}
%    \end{macrocode}
% \end{macro}
%
% \begin{macro}{\lst@outputblank}
% It's better not to forget this.
%    \begin{macrocode}
\def\lst@outputblank{\ }
%    \end{macrocode}
% \end{macro}
%
% \begin{macro}{\lstCC@ProcessSpace}
% If the last character has been a letter, we output the current character string and append one space.
%    \begin{macrocode}
\def\lstCC@ProcessSpace{%
    \lstCC@ifletter %
        \lst@Output\lstCC@letterfalse %
        \lstCC@AppendOther\lst@outputblank %
%    \end{macrocode}
% Otherwise we look whether we are in string mode or not.
% In the first case we must append a space; in the second case we must test if the hitherto line is empty.
%    \begin{macrocode}
    \else \ifnum\lst@mode=\lst@stringmode %
        \lstCC@AppendOther\lst@outputblank %
    \else \ifx\lst@NewLine\relax %
%    \end{macrocode}
% If the line is not empty we either advance |\lst@lostspace| or append a space to the current character string.
%    \begin{macrocode}
        \lstCC@AppendSpecialSpace %
    \else \ifnum\lst@length=\z@ %
%    \end{macrocode}
% If the line is empty so far, we advance |\lst@lostspace|.
% Otherwise we append the space.
%    \begin{macrocode}
            \global\advance\lst@lostspace\lst@width %
            \global\advance\lst@pos\m@ne %
        \else %
            \lstCC@AppendSpecialSpace %
        \fi %
    \fi \fi \fi}
%    \end{macrocode}
% Note that this version works for fixed and flexible column output.
% \end{macro}
% \begingroup
%    \begin{macrocode}
%</kernel>
%    \end{macrocode}
% \endgroup
%
%
% \subsection{Stringizer}
%
% \begingroup
%    \begin{macrocode}
%<*kernel>
%    \end{macrocode}
% \endgroup
% \begin{macro}{\lst@legalstringizer}
% Currently there are three different stringizer types: 'd'oubled, 'b'ackslashed and 'm'atlabed.
% The naming of the first two is due to how the stringizer is represented in a string.
% Pascal doubles it, i.e.\ the string |'| is represented by four single quotes |''''|, where the first and last enclose the string and the two middle quotes represent the desired stringizer.
% In C++ we would write |"\""|: A backslash indicates that the next double quote belongs to the string and is not the end of string.
% The matlabed version is described below.
% I introduced it after communication with \lsthelper{Zvezdan V. Petkovic}{zpetkovic@acm.org}{1997/11/26}{'single stringizer' not a stringizer in Ada (and Matlab)}.
%    \begin{macrocode}
\def\lst@legalstringizer{d,b,m,bd,db}
%    \end{macrocode}
% Furthermore we have the two mixed types |bd| and |db|, which in fact equal |b|.
% \end{macro}
%
% \begin{aspect}{stringizer}
% Here we test whether the user type is supported or not (leading to an error message).
% In the first case we (re-) define |\lst@DefStrings|.
%    \begin{macrocode}
\lst@Aspect{stringizer}{\lstKV@OptArg\lststringizer@[d]{#1}}
\def\lststringizer@[#1]#2%
    {\lst@IfOneOf#1\relax \lst@legalstringizer %
         {\def\lst@DefStrings{\lstCC@Stringizer[#1]#2\@empty}}%
         {\PackageError{Listings}{Illegal stringizer type `#1'}%
          {Available types are \lst@legalstringizers.}}}
\lst@AddToHook{SetLanguage}{\let\lst@DefStrings\@empty}
\lst@AddToHook{SelectCharTable}{\lst@DefStrings}
%    \end{macrocode}
% The just added hook defines the strings after selecting the standard character table.
% This adjusts the character table to the user's demands.
% \end{aspect}
%
% \begin{macro}{\lstCC@Stringizer}
% This macro is similar to |\lstCC@ECUse|, but we build the 'use'd name before defining the characters \ldots
%    \begin{macrocode}
\def\lstCC@Stringizer[#1]{%
    \expandafter\lstCC@Stringizer@ %
        \csname lstCC@ProcessStringizer@#1\endcsname}
\def\lstCC@Stringizer@#1#2{%
    \ifx\@empty#2%
        \expandafter\@gobbletwo %
%    \end{macrocode}
% which is terminated by |\@empty|.
% Otherwise we save the old meaning in |\lsts@s|\meta{the character} (with catcode 12) and redefine it.
%    \begin{macrocode}
    \else %
        \catcode`#2=\active \lccode`\~=`#2\lccode`\/=`#2%
        \lowercase{%
           \expandafter\let\csname lsts@s/\endcsname~%
           \def~{#1/}}%
    \fi %
    \lstCC@Stringizer@#1}
%    \end{macrocode}
% And now we define all 'process stringizer' macros.
% \end{macro}
%
% \begin{macro}{\lstCC@ProcessStringizer@d}
% 'd' means no extra work.
% Reaching the (first) stringizer enters string mode and coming to the next leaves it, and so on.
% Then the character sequence |''''| produces the right output:
% The second quote leaves string mode, but we enter it immediately since the stringizer is doubled.
% And now the implementation.
% First we output any preceding letters.
%    \begin{macrocode}
\def\lstCC@ProcessStringizer@d#1{%
    \lstCC@ifletter \lst@Output\lstCC@letterfalse \fi %
%    \end{macrocode}
% If we already process a string, we execute the saved meaning and look whether the last other (that's the stringizer) is the matching stringizer --- a single quote must not end a string starting with a double quote.
% The macro |\lstCC@EndString| is defined at the end of this section.
%    \begin{macrocode}
    \ifnum\lst@mode=\lst@stringmode %
        \csname lsts@s#1\endcsname %
        \ifx\lst@lastother\lstCC@closestring %
            \lstCC@EndString %
        \fi %
    \else %
        \lst@OutputOther %
%    \end{macrocode}
% If we don't process a string, we test whether or not a string is allowed.
% |\lstCC@BeginString| enters string mode and defines the closing stringizer.
% This 'begin string' macro gets one argument, hence we expand the control sequence name before executing the macro (if necessary).
%    \begin{macrocode}
        \lst@ifmode\else %
            \expandafter\expandafter\expandafter\lstCC@BeginString %
        \fi %
        \csname lsts@s#1\endcsname %
    \fi}
%    \end{macrocode}
% \end{macro}
%
% \begin{macro}{\lstCC@ProcessStringizer@b}
% 'b' means an extra if: Only if the last other is not a backslash (5-th line) the stringizer can close the string.
% The rest is the same as above.
%    \begin{macrocode}
\def\lstCC@ProcessStringizer@b#1{%
    \lstCC@ifletter \lst@Output\lstCC@letterfalse \fi %
    \ifnum\lst@mode=\lst@stringmode %
        \let\lst@temp\lst@lastother \csname lsts@s#1\endcsname %
        \ifx\lst@temp\lst@backslash\else %!def of "005C
        \ifx\lst@lastother\lstCC@closestring %
            \lstCC@EndString %
        \fi \fi %
    \else %
        \lst@OutputOther %
        \lst@ifmode\else %
            \expandafter\expandafter\expandafter\lstCC@BeginString %
        \fi %
        \csname lsts@s#1\endcsname %
    \fi}
%    \end{macrocode}
% \end{macro}
%
% \begin{macro}{\lstCC@ProcessStringizer@bd}
% \begin{macro}{\lstCC@ProcessStringizer@db}
% are just the same and the same as |\lstCC@ProcessStringizer@b|:
%    \begin{macrocode}
\let\lstCC@ProcessStringizer@bd\lstCC@ProcessStringizer@b
\let\lstCC@ProcessStringizer@db\lstCC@ProcessStringizer@bd
%    \end{macrocode}
% \end{macro}\end{macro}
%
% \begin{macro}{\lstCC@ProcessStringizer@m}
% 'm'atlabed is designed for programming languages where stringizers (for character or string literals) are also used for other purposes, like Matlab or Ada.
% Here we enter string mode only if the last character has not been a letter and has not been a right parenthesis.
% Hence, we have to move the |\lstCC@ifletter| and change the main |\else| part.
% By the way: The stringizer is doubled in a string.
%    \begin{macrocode}
\def\lstCC@ProcessStringizer@m#1{%
    \ifnum\lst@mode=\lst@stringmode %
        \lstCC@ifletter \lst@Output\lstCC@letterfalse \fi %
        \csname lsts@s#1\endcsname %
        \ifx\lst@lastother\lstCC@closestring %
            \lstCC@EndString %
        \fi %
    \else %
%    \end{macrocode}
% And now the real 'm' changes:
%    \begin{macrocode}
        \lstCC@ifletter %
            \lst@Output\lstCC@letterfalse %
        \else %
            \lst@OutputOther %
            \let\lstCC@next\relax %
            \ifx\lst@lastother)\else \lst@ifmode\else %
                \let\lstCC@next\lstCC@BeginString %
            \fi \fi %
            \expandafter\expandafter\expandafter\lstCC@next %
        \fi %
        \csname lsts@s#1\endcsname %
    \fi}
%    \end{macrocode}
% \end{macro}
%
% \begin{aspect}{stringstyle}
% \begin{aspect}{blankstring}
% We insert some easy definitions.
%    \begin{macrocode}
\lst@Aspect{stringstyle}{\def\lst@stringstyle{#1}}
%    \end{macrocode}
% Thanks to \lsthelper{Knut M\"uller}{knut@physik3.gwdg.de}{1997/04/28}{\blankstringtrue} for reporting problem with |\blankstringtrue|.
% The problem has gone.
%    \begin{macrocode}
\lst@Aspect{blankstring}[t]{\lstKV@SetIfKey\lst@ifblankstring{#1}}
%    \end{macrocode}
%    \begin{macrocode}
\lstset{blankstring=false}% init
\lst@AddToHook{BeforeSelectCharTable}
    {\setbox\@tempboxa\hbox{\lst@stringstyle \lst@loadfd}}
%    \end{macrocode}
% \end{aspect}\end{aspect}
%
% \begin{macro}{\lst@stringmode}
% It's time for a new mode allocation:
%    \begin{macrocode}
\lst@NewMode\lst@stringmode
%    \end{macrocode}
% \end{macro}
%
% \begin{macro}{\lstCC@BeginString}
% \begin{macro}{\lstCC@EndString}
% To activate string mode we do te usual things, but here we also assign the correct closing stringizer and |\lst@outputblank|.
% Note that you know that |\lst@NewLine| deactivates itself.
%    \begin{macrocode}
\def\lstCC@BeginString#1{%
    \lst@NewLine %
    \lst@EnterMode{\lst@stringmode}{\lst@modetrue\lst@stringstyle}%
    #1%
    \let\lstCC@closestring\lst@lastother %
    \lst@ifblankstring\else \let\lst@outputblank\textvisiblespace \fi}
%    \end{macrocode}
% We terminate that mode selection after printing the collected other characters --- at least the closing stringizer.
% And we reset some registers.
%    \begin{macrocode}
\def\lstCC@EndString{%
    \lst@OutputOther \lst@LeaveMode \lst@token{}\lst@length\z@}
%    \end{macrocode}
% \end{macro}\end{macro}
%
% \begin{aspect}{stringtest}
% We |\let| the test macro |\relax| if necessary.
%    \begin{macrocode}
\lst@Aspect{stringtest}[t]{\lstKV@SetIfKey\lst@ifstringtest{#1}}
\lst@AddToHook{SetLanguage}{\let\lst@ifstringtest\iftrue}
\lst@AddToHook{BeforeSelectCharTable}
    {\lst@ifstringtest\else \let\lst@TestStringMode\relax \fi}
%    \end{macrocode}
% Default definition of the test macro:
%    \begin{macrocode}
\def\lst@TestStringMode{%
     \ifnum\lst@mode=\lst@stringmode %
         \PackageWarning{Listings}{String constant exceeds line}%
         \lst@LeaveMode \lst@token{}\lst@length\z@ %
     \fi}
\lst@AddToHook{EOL}{\lst@TestStringMode}
%    \end{macrocode}
% \end{aspect}
% \begingroup
%    \begin{macrocode}
%</kernel>
%    \end{macrocode}
% \endgroup
%
%
% \subsection{Comments}
%
% \begingroup
%    \begin{macrocode}
%<*kernel>
%    \end{macrocode}
% \endgroup
% \begin{aspect}{commentstyle}
% Again we start with an easy definition.
%    \begin{macrocode}
\lst@Aspect{commentstyle}{\def\lst@commentstyle{#1}}
\lst@AddToHook{BeforeSelectCharTable}
    {\setbox\@tempboxa\hbox{\lst@commentstyle \lst@loadfd}}
%    \end{macrocode}
% \end{aspect}
%
% \begin{macro}{\lstCC@BeginComment}
% \begin{macro}{\lstCC@@BeginComment}
% \begin{macro}{\lstCC@EndComment}
% These macros start and end a comment, respectively.
% The |@@| version also eat (and output) one or two characters.
% In that case the characters belong to the comment indicator.
% |\relax| ensures that the characters are really eaten, i.e.\ the characters can't end the current comment or start a new one.
% Note again that |\lst@NewLine| deactivates itself.
%    \begin{macrocode}
\def\lstCC@BeginComment#1{%
    \lst@NewLine \lst@PrintToken %
    \lst@EnterMode{#1}{\lst@modetrue\lst@commentstyle}}
%    \end{macrocode}
%    \begin{macrocode}
\def\lstCC@@BeginComment#1#2#3{%
    \lst@NewLine \lst@PrintToken %
    \lst@EnterMode{#1}{\lst@modetrue\lst@commentstyle}%
    \lst@mode\lst@nomode #2\relax#3\relax \lst@mode#1\relax}
%    \end{macrocode}
%    \begin{macrocode}
\def\lstCC@EndComment{%
    \lst@PrintToken \lst@LeaveMode \let\lst@lastother\@empty}
%    \end{macrocode}
% \end{macro}\end{macro}\end{macro}
%
% \begin{macro}{\lstCC@TestCArg}
% Comment commands allow a single character like |!| or two characters like |//|.
% The calling syntax of this testing macro is
% \begin{macrosyntax}
% \item	|\lstCC@TestCArg|\meta{characters to test}|\@empty\relax|\meta{macro}
% \end{macrosyntax}
% \meta{macro} is called after doing the test and gets three arguments:
% The given two characters as active characters (where the second equals |\@empty| if and only if a single character is given), and the third is a catcode 12 version of the first character.
%    \begin{macrocode}
\begingroup \catcode`\^^@=\active
\gdef\lstCC@TestCArg#1#2#3\relax#4{%
    \lccode`\~=`#1\lccode`\/=`#1%
    \ifx\@empty#2%
        \lowercase{\def\lst@temp{~\@empty/}}%
    \else \lccode`\^^@=`#2%
        \lowercase{\def\lst@temp{~^^@/}}%
    \fi %
%    \end{macrocode}
% If neither |#2| nor |#3| equals |\@empty|, the user has given more than two characters:
%    \begin{macrocode}
    \ifx\@empty#2\else \ifx\@empty#3\else \lstCC@TestCArgError \fi\fi %
%    \end{macrocode}
%    \begin{macrocode}
    \expandafter #4\lst@temp}
\endgroup
%    \end{macrocode}
% \end{macro}
%
% \begin{macro}{\lstCC@TestCArgError}
%    \begin{macrocode}
\def\lstCC@TestCArgError{%
    \PackageError{Listings}{At most 2 characters allowed}%
    {The package doesn't provide more than two characters here.^^J%
     I'll simply use the first two only and proceed.}}
%    \end{macrocode}
% \end{macro}
%
% \begin{macro}{\lstCC@next}
% \begin{macro}{\lstCC@bnext}
% \begin{macro}{\lstCC@enext}
% We initialize some macros used in the sequel.
%    \begin{macrocode}
\lst@AddToHook{BeforeSelectCharTable}
    {\let\lstCC@next\relax \let\lstCC@bnext\relax\let\lstCC@enext\relax}
%    \end{macrocode}
% \end{macro}\end{macro}\end{macro}
%
% \begin{macro}{\lst@DefSC}
% \begin{macro}{\lst@DefNC}
% \begin{macro}{\lst@DefDC}
% \begin{macro}{\lst@DefCL}
% \begin{macro}{\lst@DefFCL}
% These macros are redefined by comment aspects.
% The comments are reset every language selection, and every listing we define the comment characters after selecting the standard character table.
%    \begin{macrocode}
\lst@AddToHook{SetLanguage}
    {\let\lst@DefSC\relax \let\lst@DefNC\relax \let\lst@DefDC\relax %
     \let\lst@DefCL\relax \let\lst@DefFCL\relax}
\lst@AddToHook{SelectCharTable}
    {\lst@DefSC \lst@DefNC \lst@DefDC \lst@DefCL \lst@DefFCL}
%    \end{macrocode}
% \end{macro}\end{macro}\end{macro}\end{macro}
% \end{macro}
% \begingroup
%    \begin{macrocode}
%</kernel>
%    \end{macrocode}
% \endgroup
%
%
% \subsubsection{Comment lines}
%
% \begingroup
%    \begin{macrocode}
%<*kernel>
%    \end{macrocode}
% \endgroup
% \begin{aspect}{commentline}
% We define the user aspect:
%    \begin{macrocode}
\lst@Aspect{commentline}{\def\lst@DefCL{\lstCC@CommentLine{#1}}}
%    \end{macrocode}
% \end{aspect}
%
% \begin{macro}{\lstCC@CommentLine}
% Comment lines become undefined if the one and only argument is empty.
%    \begin{macrocode}
\def\lstCC@CommentLine#1{%
    \ifx\@empty#1\@empty\else %
        \lstCC@TestCArg#1\@empty\relax\lstCC@CommentLine@ %
    \fi}
%    \end{macrocode}
% The next submacro actually defines the comment characters.
% We save the old meaning --- or we couldn't use the original meaning any more.
%    \begin{macrocode}
\def\lstCC@CommentLine@#1#2#3{%
    \expandafter\let\csname lsts@CL#3\endcsname#1%
%    \end{macrocode}
% The redefinitions:
% If a single character indicates a comment, the next operation is either |\relax| (since no mode change is allowed) or |\lstCC@BeginCommentLine|.
% And we execute the saved character meaning.
%    \begin{macrocode}
    \ifx\@empty#2%
        \def#1{%
            \lst@ifmode \let\lstCC@next\relax \else %
                \let\lstCC@next\lstCC@BeginCommentLine %
            \fi %
            \expandafter\lstCC@next \csname lsts@CL#3\endcsname}%
    \else %
%    \end{macrocode}
% Comment lines indicated by a sequence of two characters are just the same.
% But: We enter comment mode (if and) only if the next character equals |#2|.
% And we use a different macro to enter comment mode.
%    \begin{macrocode}
        \def#1##1{%
            \let\lstCC@next\relax %
            \lst@ifmode\else \ifx##1#2%
                \let\lstCC@next\lstCC@@BeginCommentLine %
            \fi \fi %
            \expandafter\lstCC@next \csname lsts@CL#3\endcsname##1}%
    \fi}
%    \end{macrocode}
% \end{macro}
%
% \begin{aspect}{fixedcommentline}
% \begin{macro}{\lst@FCLmode}
% Another user aspect \ldots
%    \begin{macrocode}
\lst@Aspect{fixedcommentline}{\lstKV@OptArg\lstfcommentline@[0]{#1}}
\def\lstfcommentline@[#1]#2%
    {\def\lst@DefFCL{\lstCC@FixedCL[#1]#2\@empty}}
%    \end{macrocode}
% and a mode allocation.
%    \begin{macrocode}
\lst@NewMode\lst@FCLmode
%    \end{macrocode}
% \end{macro}\end{aspect}
%
% \begin{macro}{\lstCC@FixedCL}
% Note that we can't use |\lstCC@TestCArg| here since the argument might consist of more than two characters.
% We enter a loop which is terminated by |\@empty|.
%    \begin{macrocode}
\def\lstCC@FixedCL[#1]#2{%
    \ifx\@empty#2\else %
        \lccode`\~=`#2\lccode`\/=`#2%
        \lowercase{\lstCC@FixedCL@~/}{#1}%
        \def\lstCC@next{\lstCC@FixedCL[#1]}%
        \expandafter\lstCC@next %
    \fi}
%    \end{macrocode}
% But now comes the same as above: We save the old meaning of |#1| and redefine it.
% We enter comment mode (if and) only if the character is in column |#3|+1.
%    \begin{macrocode}
\def\lstCC@FixedCL@#1#2#3{%
    \expandafter\let\csname lsts@FCL#2\endcsname#1%
    \def#1{\let\lstCC@next\relax %
        \lst@ifmode\else %
            \@tempcnta\lst@column %
            \advance\@tempcnta\lst@length %
            \advance\@tempcnta-\lst@pos %
            \ifnum\@tempcnta=#3%
                \let\lstCC@next\lstCC@BeginCommentLine %
            \fi %
        \fi %
        \expandafter\lstCC@next \csname lsts@FCL#2\endcsname}}
%    \end{macrocode}
% \end{macro}
% \begingroup
%    \begin{macrocode}
%</kernel>
%    \end{macrocode}
% \endgroup
%
%
% \subsubsection{Single and double comments}
%
% \begingroup
%    \begin{macrocode}
%<*kernel>
%    \end{macrocode}
% \endgroup
% \begin{aspect}{singlecomment}
% \begin{aspect}{doublecomment}
% Define the user commands \ldots
%    \begin{macrocode}
\lst@Aspect{singlecomment}{\def\lst@DefSC{\lstCC@SingleComment#1{}}}
\lst@Aspect{doublecomment}{\def\lst@DefDC{\lstCC@DoubleComment#1{}{}{}}}
%    \end{macrocode}
% \end{aspect}\end{aspect}
%
% \begin{macro}{\lst@SCmode}
% \begin{macro}{\lst@DCmodeA}
% \begin{macro}{\lst@DCmodeB}
% and allocate new internal modes.
%    \begin{macrocode}
\lst@NewMode\lst@SCmode \lst@NewMode\lst@DCmodeA\lst@NewMode\lst@DCmodeB
%    \end{macrocode}
% \end{macro}\end{macro}\end{macro}
%
% \begin{macro}{\lstCC@SingleComment}
% \begin{macro}{\lstCC@DoubleComment}
% All must be done twice here, for the beginning and end of a comment.
% The use of additional arguments like |{SC}\lst@SCmode| make it possible to define single and double comments with the same macros.
%    \begin{macrocode}
\def\lstCC@SingleComment#1#2{%
    \ifx\@empty#1\@empty\else %
        \lstCC@TestCArg#1\@empty\relax\lstCC@CommentB{SC}\lst@SCmode %
        \lstCC@TestCArg#2\@empty\relax\lstCC@CommentE{SC}\lst@SCmode %
    \fi}
%    \end{macrocode}
% Note the order: We define 'begin comment' and afterwards 'end comment'.
% This becomes important if |#1| equals |#2|, for example.
% Each such comment definition saves the old character meaning, which is executed after doing all comment specific.
% If we define 'end comment' first and then 'begin comment', |#1| would start a comment and then execute the old definition, which is 'end comment'.
% Since we are in comment mode now (and since |#1| equals |#2|), 'end comment' leaves comment mode immediately.
% That would be very bad!
%    \begin{macrocode}
\def\lstCC@DoubleComment#1#2#3#4{%
    \ifx\@empty#1\@empty\else %
        \lstCC@TestCArg#1\@empty\relax\lstCC@CommentB{DCA}\lst@DCmodeA %
        \lstCC@TestCArg#3\@empty\relax\lstCC@CommentB{DCB}\lst@DCmodeB %
        \lstCC@TestCArg#2\@empty\relax\lstCC@CommentE{DCA}\lst@DCmodeA %
        \lstCC@TestCArg#4\@empty\relax\lstCC@CommentE{DCB}\lst@DCmodeB %
    \fi}
%    \end{macrocode}
% \end{macro}\end{macro}
%
% \begin{macro}{\lstCC@CommentB}
% is not different from |\lstCC@CommentLine@| at all:
% The save name |#3| there is replaced by |B#4#3| and instead of |\lst@CLmode| we use |#5|.
%    \begin{macrocode}
\def\lstCC@CommentB#1#2#3#4#5{%
    \expandafter\let\csname lsts@B#4#3\endcsname#1%
    \ifx\@empty#2%
        \def#1{%
            \lst@ifmode \let\lstCC@bnext\relax \else %
                \def\lstCC@bnext{\lstCC@BeginComment#5}%
            \fi %
            \lstCC@bnext \csname lsts@B#4#3\endcsname}%
    \else %
        \def#1##1{%
            \let\lstCC@bnext\relax %
            \lst@ifmode\else \ifx##1#2%
                \def\lstCC@bnext{\lstCC@@BeginComment#5}%
            \fi \fi %
            \expandafter\lstCC@bnext \csname lsts@B#4#3\endcsname##1}%
    \fi}
%    \end{macrocode}
% \end{macro}
%
% \begin{macro}{\lstCC@CommentE}
% Here we insert |\ifnum\lst@mode=#5| (6-th line), where |#5| is |\lst@SCmode| for example.
% This ensures that comment delimiters match each other.
%    \begin{macrocode}
\def\lstCC@CommentE#1#2#3#4#5{%
    \expandafter\let\csname lsts@E#4#3\endcsname#1%
    \ifx\@empty#2%
        \def#1{%
            \def\lstCC@enext{\csname lsts@E#4#3\endcsname}%
            \lst@ifmode \ifnum\lst@mode=#5%
                \def\lstCC@enext{\csname lsts@E#4#3\endcsname %
                    \lstCC@EndComment}%
            \fi \fi %
            \lstCC@enext}%
    \else %
        \def#1##1{%
            \def\lstCC@enext{\csname lsts@E#4#3\endcsname ##1}%
            \lst@ifmode \ifnum\lst@mode=#5\ifx##1#2%
                \def\lstCC@enext{\csname lsts@E#4#3\endcsname ##1%
                    \lstCC@EndComment}%
            \fi \fi \fi %
            \lstCC@enext}%
    \fi}
%    \end{macrocode}
% \end{macro}
% \begingroup
%    \begin{macrocode}
%</kernel>
%    \end{macrocode}
% \endgroup
%
%
% \subsubsection{Nested comments}
%
% \begingroup
%    \begin{macrocode}
%<*kernel>
%    \end{macrocode}
% \endgroup
% \begin{aspect}{nestedcomment}
% The user aspect \ldots
%    \begin{macrocode}
\lst@Aspect{nestedcomment}{\lstnestedcomment@#1}
\def\lstnestedcomment@#1#2%
    {\def\@tempa{#1}\def\@tempb{#2}%
     \ifx\@tempa\@tempb \ifx\@tempa\@empty\else %
         \PackageError{Listings}{Identical delimitors}%
         {These delimitors make no sense with nested comments.}%
     \fi \fi %
     \def\lst@DefNC{\lstCC@NestedComment{#1}{#2}}}
%    \end{macrocode}
% \end{aspect}
%
% \begin{macro}{\lstCC@NestedComment}
% and the internal macro.
%    \begin{macrocode}
\def\lstCC@NestedComment#1#2{%
    \ifx\@empty#1\@empty\else %
        \lstCC@TestCArg#2\@empty\relax\lstCC@NCommentE %
        \lstCC@TestCArg#1\@empty\relax\lstCC@NCommentB %
    \fi}
%    \end{macrocode}
% Note the order: We define 'end comment' and afterwards 'begin comment'.
% \end{macro}
%
% \begin{macro}{\lstCC@NCommentB}
% The redefinition of the character is different now.
% Since we define nested comments, we have to count the comment depth.
% If we already process a nested comment, we increase that depth by making |\lst@mode| more negative.
%    \begin{macrocode}
\def\lstCC@NCommentB#1#2#3{%
    \let\lsts@BNC #1%
    \ifx\@empty#2%
        \def#1{%
            \let\lstCC@bnext\relax %
            \lst@ifmode %
                \ifnum\lst@mode<\z@ %
                    \advance\lst@mode\m@ne %
                \fi %
            \else %
                \def\lstCC@bnext{\lstCC@BeginComment\m@ne}%
            \fi %
            \lstCC@bnext \lsts@BNC}%
    \else %
        \def#1##1{%
            \let\lstCC@bnext\relax %
            \lst@ifmode %
                \ifnum\lst@mode<\z@ \ifx##1#2%
                    \advance\lst@mode\m@ne %
                \fi \fi %
            \else %
                \ifx##1#2%
                    \def\lstCC@bnext{\lstCC@@BeginComment\m@ne}%
                \fi %
            \fi %
            \lstCC@bnext \lsts@BNC ##1}%
    \fi}
%    \end{macrocode}
% \begin{TODO}
% The third argument (|\catcode| 12 version of the first) is unused so far --- also in |\lst@NCommentE|.
% If we are in need of 'double' nested comments, we will use this argument for the save name, as we've already done it for normal double comments (and comment lines).
% \end{TODO}
% \end{macro}
%
% \begin{macro}{\lstCC@NCommentE}
% If we are in nested comment mode, we either end the comment or decrease the comment depth by making |\lst@mode| less negative.
%    \begin{macrocode}
\def\lstCC@NCommentE#1#2#3{%
    \let\lsts@ENC #1%
    \ifx\@empty#2%
        \def#1{%
            \let\lstCC@enext\relax %
            \lst@ifmode \ifnum\lst@mode<\z@ %
                \ifnum\lst@mode=\m@ne %
                    \let\lstCC@enext\lstCC@EndComment %
                \else %
                    \advance\lst@mode\@ne %
                \fi
            \fi \fi %
            \lsts@ENC \lstCC@enext}%
    \else %
        \def#1##1{%
            \def\lstCC@enext{\lsts@ENC ##1}%
            \ifx##1#2%
                \lst@ifmode \ifnum\lst@mode<\z@ %
                    \ifnum\lst@mode=\m@ne %
                        \def\lstCC@enext{%
                            \lsts@ENC ##1\lstCC@EndComment}%
                    \else %
                        \advance\lst@mode\@ne %
                    \fi
                \fi \fi %
            \fi %
            \lstCC@enext}%
    \fi}
%    \end{macrocode}
% \end{macro}
% \begingroup
%    \begin{macrocode}
%</kernel>
%    \end{macrocode}
% \endgroup
%
%
% \subsection{Escape characters}
%
% \begingroup
%    \begin{macrocode}
%<*kernel>
%    \end{macrocode}
% \endgroup
% The introduction of this character class is due to a communication with \lsthelper{Rui Oliveira}{rco@di.uminho.pt}{1998/06/05}{escape characters}.
%
% \begin{macro}{\lstCC@Escape}
% gets three arguments all in all.
% The first is the escape character, the second is executed when the escape starts and the third right before ending it.
% We use the same grouping mechanism as for \TeX\ comment lines.
%    \begin{macrocode}
\def\lstCC@Escape#1{\lccode`\~=`#1\lowercase{\lstCC@Escape@~}}
\def\lstCC@Escape@#1#2#3{%
    \def#1{%
        \lst@NewLine\lst@UseLostSpace \lst@PrintToken %
        \lst@InterruptModes %
        \lst@EnterMode{\lst@TeXmode}{\lst@modetrue}%
%    \end{macrocode}
% After doing the grouping we must define the character to end the escape.
%    \begin{macrocode}
        \catcode`#1=\active %
        \def#1{#3\lst@LeaveAllModes \lst@ReenterModes}%
        #2}}
%    \end{macrocode}
%    \begin{macrocode}
\lst@NewMode\lst@TeXmode
%    \end{macrocode}
% \end{macro}
%
% \begin{aspect}{escapechar}
% This aspect defines |\lst@DefEsc|, which is executed after selecting the standard character table.
%    \begin{macrocode}
\lst@Aspect{escapechar}
    {\ifx\@empty#1\@empty %
         \let\lst@DefEsc\relax %
     \else %
         \def\lst@DefEsc{\lstCC@Escape{#1}\@empty\@empty}%
     \fi}
%    \end{macrocode}
%    \begin{macrocode}
\lstset{escapechar={}}% init
\lst@AddToHook{SelectCharTable}{\lst@DefEsc}
%    \end{macrocode}
% \end{aspect}
%
% \begin{aspect}{mathescape}
% A switch tested after character table selection.
% We use |\lstCC@Escape| with math shifts as arguments.
%    \begin{macrocode}
\lst@Aspect{mathescape}[t]{\lstKV@SetIfKey\lst@ifmathescape{#1}}
\lstset{mathescape=false}% init
%    \end{macrocode}
%    \begin{macrocode}
\lst@AddToHook{SelectCharTable}
    {\lst@ifmathescape \lstCC@Escape{\$}{$}{$}\fi}
%    \end{macrocode}
% \end{aspect}
%
% \begin{macro}{\lst@DontEscapeToLaTeX}
% This macro came in handy.
%    \begin{macrocode}
\def\lst@DontEscapeToLaTeX{%
    \let\lst@iftexcl\iffalse %
    \let\lst@DefEsc\relax %
    \let\lst@ifmathescape\iffalse}
%    \end{macrocode}
% \end{macro}
% \begingroup
%    \begin{macrocode}
%</kernel>
%    \end{macrocode}
% \endgroup
%
%
% \section{More \texttt{lst}-aspects and classes}
%
%
% \subsection{Form feeds}
%
% \begingroup
%    \begin{macrocode}
%<*kernel>
%    \end{macrocode}
% \endgroup
% \begin{aspect}{formfeed}
% We begin with the announcement --- the introduction is due to communication with \lsthelper{Jan Braun}{Jan.Braun@tu-bs.de}{1998/04/27}{\lstformfeed}.
%    \begin{macrocode}
\lst@Aspect{formfeed}{\def\lst@formfeed{#1}}
\lstset{formfeed=\bigbreak}% init
%    \end{macrocode}
%    \begin{macrocode}
\lst@AddToHook{SelectCharTable}{\lstCC@Let{`\^^L}\lstCC@FormFeed}
%    \end{macrocode}
% \end{aspect}
%
% \begin{macro}{\lstCC@FormFeed}
% Here we either execute some macros or append them to |\lst@NewLine|.
%    \begin{macrocode}
\def\lstCC@FormFeed{%
    \lst@PrintToken %
    \ifx\lst@NewLine\relax %
        \lst@EOLUpdate \lst@formfeed %
    \else %
        \lst@lAddTo\lst@NewLine{\lst@EOLUpdate \lst@formfeed}%
    \fi}
%    \end{macrocode}
% \end{macro}
% \begingroup
%    \begin{macrocode}
%</kernel>
%    \end{macrocode}
% \endgroup
%
%
% \subsection{\TeX\ control sequences}
%
% \begingroup
%    \begin{macrocode}
%<*tex>
%    \end{macrocode}
% \endgroup
% \begin{aspect}{texcs}
% In general, it's the same as |keywords|, but control sequences are case sensitive, have a preceding backslash, \ldots
%    \begin{macrocode}
\lst@Aspect{texcs}{\def\lst@texcs{#1}}
\lst@AddToHook{SetLanguage}{\let\lst@texcs\@empty}
%    \end{macrocode}
%    \begin{macrocode}
\lst@AddToHook{Output}
    {\lst@ifmode\else \ifx\lst@lastother\lst@backslash %
         \expandafter\lst@IfOneOf\the\lst@token\relax \lst@texcs %
             {\let\lst@thestyle\lst@keywordstyle}%
             {\let\lst@thestyle\relax}%
     \fi \fi}
%    \end{macrocode}
% \end{aspect}
% \begingroup
%    \begin{macrocode}
%</tex>
%    \end{macrocode}
% \endgroup
%
%
% \subsection{Compiler directives in C}
%
% \begingroup
%    \begin{macrocode}
%<*c>
%    \end{macrocode}
% \endgroup
% \begin{aspect}{cdirectives}
% First some usual stuff:
%    \begin{macrocode}
\lst@Aspect{cdirectives}
    {\lst@MakeKeywordArg{,#1}\let\lst@cdirectives\lst@arg}
\lst@AddToHook{SetLanguage}{\let\lst@cdirectives\relax}
%    \end{macrocode}
% If the user has defined directives, we redefine the character |#| (but save the old meaning before):
%    \begin{macrocode}
\lst@AddToHook{SelectCharTable}
    {\ifx\lst@cdirectives\relax\else %
         \lccode`\~=`\#\lowercase{\let\lsts@CCD~\def~}%
         {\lst@ifmode\else %
%    \end{macrocode}
% We enter 'directive mode' only if we are in column 1.
% Note that 'directive mode' is |\lst@egroupmode| without setting |\lst@ifmode| true.
%    \begin{macrocode}
              \@tempcnta\lst@column %
              \advance\@tempcnta\lst@length %
              \advance\@tempcnta-\lst@pos %
              \ifnum\@tempcnta=\z@ %
                  \lst@EnterMode{\lst@egroupmode}%
                      {\expandafter\lst@lAddTo\expandafter\lst@keywords%
                          \expandafter{\lst@cdirectives}}%
              \fi %
          \fi %
          \ifnum\lst@mode=\lst@egroupmode %
              {\lst@keywordstyle\lsts@CCD\lst@PrintToken}%
          \else \lsts@CCD %
          \fi}%
     \fi}
%    \end{macrocode}
% \end{aspect}
% \begingroup
%    \begin{macrocode}
%</c>
%    \end{macrocode}
% \endgroup
%
%
% \subsection{PODs in Perl}
%
% \begingroup
%    \begin{macrocode}
%<*perl>
%    \end{macrocode}
% \endgroup
% \begin{aspect}{printpod}
% We begin with two user commands, which I introduced after communication with \lsthelper{Michael Piotrowski}{mxp@linguistik.uni-erlangen.de}{1997/11/11}{\lstprintpodtrue/false}.
%    \begin{macrocode}
\lst@Aspect{printpod}[t]{\lstKV@SetIfKey\lst@ifprintpod{#1}}
\lstset{printpod=false}% init
%    \end{macrocode}
% \end{aspect}
%
% \begin{macro}{\lst@PODmode}
% Define a new mode and use |\global| since we are (possibly) in a driver file and thus inside a group.
%    \begin{macrocode}
\global\lst@NewMode\lst@PODmode
%    \end{macrocode}
% \end{macro}
%
% \begin{macro}{\lstCC@PODcut}
% We'll need the active character string |cut|:
%    \begin{macrocode}
\lst@MakeActive{cut}\global\let\lstCC@PODcut\lst@arg %
%    \end{macrocode}
% \end{macro}
%
% \begin{macro}{\lstCC@ProcessPOD}
% And now the POD character class; only |=| will belong to it.
% If we are in column 1 \ldots
%    \begin{macrocode}
\gdef\lstCC@ProcessPOD{%
    \let\lstCC@next\relax %
    \@tempcnta\lst@column %
    \advance\@tempcnta\lst@length %
    \advance\@tempcnta-\lst@pos %
    \ifnum\@tempcnta=\z@ %
%    \end{macrocode}
% and if we are already in POD mode, we either end it or proceed according to whether next characters are |cut| or not.
%    \begin{macrocode}
        \ifnum\lst@mode=\lst@PODmode %
            \def\lstCC@next{\lst@IfNextChars\lstCC@PODcut %
                {\lstCC@ProcessEndPOD}%
                {\lsts@POD\lst@eaten}}%
        \else %
%    \end{macrocode}
% If we are in column 1 and not in POD mode, we either start a usual comment or drop the output.
%    \begin{macrocode}
            \lst@ifprintpod %
                \lstCC@BeginComment\lst@PODmode %
            \else %
                \lst@BeginDropOutput\lst@PODmode %
            \fi %
            \let\lstCC@next\lsts@POD %
        \fi %
    \fi \lstCC@next}%
%    \end{macrocode}
% \end{macro}
%
% \begin{macro}{\lstCC@ProcessEndPOD}
% If the character behind |=cut| is neither |^^M| nor |^^J|, we don't end the POD.
%    \begin{macrocode}
\gdef\lstCC@ProcessEndPOD#1{%
    \let\lstCC@next\lstCC@ProcessEndPOD@ %
    \ifnum`#1=`\^^M\else \ifnum`#1=`\^^J\else %
        \def\lstCC@next{\lsts@POD\lst@eaten#1}%
    \fi \fi %
    \lstCC@next}
%    \end{macrocode}
% The ending actions depend on |\lst@ifprintpod|: Print |=cut| (or not) and end the POD.
%    \begin{macrocode}
\gdef\lstCC@ProcessEndPOD@{%
    \lst@ifprintpod \lsts@POD\lst@eaten\lst@PrintToken \fi %
    \lst@LeaveMode}
%    \end{macrocode}
% \end{macro}
% \begingroup
%    \begin{macrocode}
%</perl>
%    \end{macrocode}
% \endgroup
%
%
% \subsection{Special use of \texttt{\#} in Perl}
%
% \begingroup
%    \begin{macrocode}
%<*perl>
%    \end{macrocode}
% \endgroup
% \begin{macro}{\lstCC@SpecialUseAfter}
% We need |#| not to begin a comment in Perl if it follows |$|.
% We'll define a special use of |#| after |$| via |\lstCC@SpecialUseAfter{\$}{\#}{\lstCC@ProcessOther\#}| where the third argument is executed coming to |$#|.
% Note that the submacro gets two of the three arguments.
%    \begin{macrocode}
\gdef\lstCC@SpecialUseAfter#1{%
    \lccode`\~=`#1\lccode`\/=`#1\lowercase{\lstCC@SpecialUseAfter@~/}}
%    \end{macrocode}
% We save the old meaning of the first character (|$|) and redefine it to check for the character |#3|.
%    \begin{macrocode}
\gdef\lstCC@SpecialUseAfter@#1#2#3#4{%
    \expandafter\let\csname lsts@SUA#2\endcsname#1%
    \def#1##1{%
        \ifnum`##1=`#3%
            \def\lstCC@next{\csname lsts@SUA#3\endcsname\relax #4}%
        \else %
            \def\lstCC@next{\csname lsts@SUA#3\endcsname ##1}%
        \fi %
        \lstCC@next}}
%    \end{macrocode}
% \end{macro}
% \begingroup
%    \begin{macrocode}
%</perl>
%    \end{macrocode}
% \endgroup
%
%
% \section{Epilogue}
%
% \begingroup
%    \begin{macrocode}
%<*kernel>
%    \end{macrocode}
% \endgroup
% Most initialization has already been done, see all |% init| marked lines.
% Here is more:
%    \begin{macrocode}
\lstset{labelsep=5pt,sensitive=true}
\lstdefinelanguage{}{}
\lstdefinestyle{}
    {basicstyle={},%
     keywordstyle=\bfseries,nonkeywordstyle={},%
     commentstyle=\itshape,stringstyle={},%
     labelstyle={},labelstep=0}
\lst@mode=\z@
\lstset{language={},style={}}
%    \end{macrocode}
% Finally we load compatibility mode, a patch file and the configuration file.
%    \begin{macrocode}
\@ifundefined{lst@0.17}{}{\input{lst017.sty}}
\InputIfFileExists{lstpatch.sty}{}{}
\InputIfFileExists{listings.cfg}{}{}
%    \end{macrocode}
% \begingroup
%    \begin{macrocode}
%</kernel>
%    \end{macrocode}
% \endgroup
%
%
% \section{Interfaces to other packages}
%
%
% \subsection{Compatibility mode}
%
% \begingroup
%    \begin{macrocode}
%<*0.17>
%    \end{macrocode}
% \endgroup
% We give a warning first.
%    \begin{macrocode}
\message{^^J%
*** You have requested compatibility mode `0.17'.^^J%
*** This mode is for documents created for version 0.17 only.^^J%
*** I T\@spaces I S\@spaces N O T\@spaces F U L L Y\@spaces %
C O M P A T I B L E.^^J^^J}
%    \end{macrocode}
%
%
% \paragraph{Language names.}
%
% Since some programming languages changed their names, we define aliases here.
%    \begin{macrocode}
\lstalias[]{blank}[]{}
\lstalias{cpp}{C++}
\lstalias[vc]{C++}[Visual]{C++}
\lstalias{comal}{Comal 80}
\lstalias{modula}{Modula-2}
\lstalias{oberon}{Oberon-2}
\lstalias[]{pxsc}[XSC]{Pascal}
\lstalias[]{tp}[Borland6]{Pascal}
\lstalias{pli}{PL/I}
%    \end{macrocode}
%
%
% \paragraph{User commands.}
%
% Old commands in terms of keys:
%    \begin{macrocode}
\def\inputlisting{%
    \@ifnextchar[\inputlisting@{\inputlisting@[1,999999]}}
\def\inputlisting@[#1,#2]{\lstinputlisting[first=#1,last=#2]}
%    \end{macrocode}
%    \begin{macrocode}
\newcommand\selectlisting[2][]{\lstset{language=[#1]#2}}
\def\listingtrue{\lstset{print=true}}
\def\listingfalse{\lstset{print=false}}
\def\tablength#1{\lstset{tabsize=#1}}\tablength{4}% init
\def\keywordstyle#1{\lstset{keywordstyle={#1}}}
\def\commentstyle#1{\lstset{commentstyle={#1}}}
\def\stringstyle#1{\lstset{stringstyle={#1}}}
\def\labelstyle#1{\lstset{labelstyle={#1}}}
\newcommand\stringizer[2][d]{\lstset{stringizer=[#1]#2}}
\def\blankstringtrue{\lstset{blankstring=true}}
\def\blankstringfalse{\lstset{blankstring=false}}
\def\spreadlisting#1{\lstset{spread={#1}}}
\def\prelisting#1{\def\lst@pre{#1}}
\def\postlisting#1{\def\lst@post{#1}}
\def\normallisting{%
    \lstset{style={},spread=\z@,pre={},post={}}}
\def\keywords#1{\lstset{keywords={#1}}}
\def\morekeywords#1{\lstset{morekeywords={#1}}}
\def\sensitivetrue{\lstset{sensitive=true}}
\def\sensitivefalse{\lstset{sensitive=false}}
\let\lst@stringblank\textvisiblespace
%    \end{macrocode}
% We define the (new) old environment.
%    \begin{macrocode}
\lst@Environment{listing}[1][]\is
    {\ifx\@empty#1\@empty\else \lst@AddToLOL{#1}{}\fi}
    {}
%    \end{macrocode}
%
%
% \paragraph{Comments.}
%
% The implementation of old comment commands in terms of the new ones is all the same:
% If the last argument is empty, we remove the comment; otherwise we execute the new key with correct syntax.
%    \begin{macrocode}
\def\DeclareCommentLine#1\relax{%
    \ifx\@empty#1\@empty %
        \let\lst@DefCL\relax %
    \else %
        \lstset{commentline=#1}%
    \fi}
%    \end{macrocode}
%    \begin{macrocode}
\def\DeclareSingleComment#1 #2\relax{%
    \ifx\@empty#2\@empty %
        \let\lst@DefSC\relax \let\lst@DefDC\relax %
    \else %
        \lstset{singlecomment={#1}{#2}}%
    \fi}
%    \end{macrocode}
%    \begin{macrocode}
\def\DeclarePairedComment#1\relax{
    \ifx\@empty#1\@empty %
        \let\lst@DefSC\relax \let\lst@DefDC\relax %
    \else %
        \lstset{singlecomment={#1}{#1}}%
    \fi}
%    \end{macrocode}
%    \begin{macrocode}
\def\DeclareNestedComment#1 #2\relax{%
    \ifx\@empty#2\@empty %
        \let\lst@DefSC\relax \let\lst@DefDC\relax %
    \else %
        \lstset{nestedcomment={#1}{#2}}%
    \fi}
%    \end{macrocode}
%    \begin{macrocode}
\def\DeclareDoubleComment#1 #2 #3 #4\relax{%
    \ifx\@empty#4\@empty %
        \let\lst@DefSC\relax \let\lst@DefDC\relax %
    \else %
        \lstset{doublecomment={#1}{#2}{#3}{#4}}%
    \fi}
%    \end{macrocode}
% \begingroup
%    \begin{macrocode}
%</0.17>
%    \end{macrocode}
% \endgroup
%
%
% \subsection{\textsf{fancyvrb}}
%
% \lsthelper{Denis Girou}{Denis.Girou@idris.fr}{1998/07/26}{fancyvrb} asked whether \textsf{fancyvrb} and \textsf{listings} could work together.
% Here's the first try.
% \begingroup
%    \begin{macrocode}
%<*kernel>
%    \end{macrocode}
% \endgroup
% \begin{aspect}{fancyvrb}
% We set the boolean and input the interface file (first time only).
%    \begin{macrocode}
\lst@Aspect{fancyvrb}[t]{%
    \lstKV@SetIfKey\lst@iffancyvrb{#1}%
    \lstfancyvrb@}
\def\lstfancyvrb@{\input{lstfvrb.sty}\lstfancyvrb@}
%    \end{macrocode}
% \end{aspect}
% \begingroup
%    \begin{macrocode}
%</kernel>
%    \end{macrocode}
%
%    \begin{macrocode}
%<*fancyvrb>
%    \end{macrocode}
% We will use |@|-protected names:
%    \begin{macrocode}
\begingroup \makeatletter
%    \end{macrocode}
% \endgroup
% \begin{macro}{\lstFV@Def}
% \begin{macro}{\lstFV@Let}
% \begin{macro}{\lstFV@Use}
% \begin{macro}{\lstFV@ECUse}
% \begin{macro}{\lstFV@Tabulator}
% \begin{macro}{\lstFV@Space}
% \begin{macro}{\lstFV@Stringizer@}
% These are copies of the |\lstCC@|\ldots\ macros, but all catcode changes are removed.
%    \begin{macrocode}
\gdef\lstFV@Def#1{\lccode`\~=#1\lowercase{\def~}}%
\gdef\lstFV@Let#1{\lccode`\~=#1\lowercase{\let~}}%
%    \end{macrocode}
%    \begin{macrocode}
\gdef\lstFV@Use#1#2#3{%
    \ifnum#2=\z@ %
        \expandafter\@gobbletwo %
    \else %
        \lccode`\~=#2\lowercase{\def~}{#1#3}%
    \fi %
    \lstFV@Use#1}%
%    \end{macrocode}
%    \begin{macrocode}
\gdef\lstFV@ECUse#1#2{%
    \ifnum`#2=\z@ %
        \expandafter\@gobbletwo %
    \else %
        \ifnum\catcode`#2=\active %
            \lccode`\~=`#2\lccode`\/=`#2\lowercase{\lstCC@ECUse@#1~/}%
        \else %
            \lccode`\~=`#2\lowercase{\def~}{#1#2}%
        \fi %
    \fi %
    \lstFV@ECUse#1}%
%    \end{macrocode}
%    \begin{macrocode}
\gdef\lstFV@Tabulator#1{\lstFV@Let{#1}\lstCC@ProcessTabulator}%
\gdef\lstFV@Space#1{\lstFV@Let{#1}\lstCC@ProcessSpace}%
%    \end{macrocode}
%    \begin{macrocode}
\gdef\lstFV@Stringizer@#1#2{%
    \ifx\@empty#2%
        \expandafter\@gobbletwo %
    \else %
        \lccode`\~=`#2\lccode`\/=`#2%
        \lowercase{%
            \expandafter\let\csname lsts@s/\endcsname~%
            \def~{#1/}}%
    \fi %
    \lstFV@Stringizer@#1}%
%    \end{macrocode}
% \end{macro}\end{macro}\end{macro}\end{macro}
% \end{macro}\end{macro}\end{macro}
%
% \begin{macro}{\lstFV@FancyVerbFormatLine}
% This macro will be assigned to |\FancyVerbFormatLine|: We convert the argument and typeset it.
%    \begin{macrocode}
\gdef\lstFV@FancyVerbFormatLine#1{%
    \let\lstenv@arg\@empty\lstenv@AddArg{#1}%
    \lstenv@arg \lst@PrintToken\lst@EOLUpdate}%
%    \end{macrocode}
% \begin{TODO}
% This old macro is overwritten below.
% \end{TODO}
% \end{macro}
%
% \begin{macro}{\lstfancyvrb@}
% The command assigns some new macros (if it's not already done).
%    \begin{macrocode}
\gdef\lstfancyvrb@{%
    \ifx\FV@VerbatimBegin\lstFV@VerbatimBegin\else %
        \let\lstFV@OldVB\FV@VerbatimBegin %
        \let\lstFV@OldVE\FV@VerbatimEnd %
        \let\FV@VerbatimBegin\lstFV@VerbatimBegin %
        \let\FV@VerbatimEnd\lstFV@VerbatimEnd %
    \fi %
%    \end{macrocode}
% And we assign the correct |\FancyVerbFormatLine| macro.
%    \begin{macrocode}
    \lst@iffancyvrb %
        \ifx\FancyVerbFormatLine\lstFV@FancyVerbFormatLine\else %
            \let\lstFV@FVFL\FancyVerbFormatLine %
            \let\FancyVerbFormatLine\lstFV@FancyVerbFormatLine %
        \fi %
    \else %
        \ifx\lstFV@FVFL\@undefined\else %
            \let\FancyVerbFormatLine\lstFV@FVFL %
        \fi %
    \fi}%
%    \end{macrocode}
% \end{macro}
%
% \begin{macro}{\lstFV@VerbatimBegin}
% We initialize things if necessary.
%    \begin{macrocode}
\gdef\lstFV@VerbatimBegin{%
    \lstFV@OldVB %
    \ifx\FancyVerbFormatLine\lstFV@FancyVerbFormatLine %
        \lst@DontEscapeToLaTeX %
        \let\smallbreak\relax \let\normalbaselines\relax %
        \let\lst@prelisting\relax \let\lst@postlisting\relax %
        \def\lst@firstline{1}\def\lst@lastline{999999}%
%    \end{macrocode}
%    \begin{macrocode}
        \let\lstCC@Use\lstFV@Use %
        \let\lstCC@ECUse\lstFV@ECUse %
        \let\lstCC@Tabulator\lstFV@Tabulator %
        \let\lstCC@Space\lstFV@Space %
        \let\lstCC@Stringizer@\lstFV@Stringizer@ %
%    \end{macrocode}
%    \begin{macrocode}
        \def\lst@next{%
            \lst@Init\relax %
            \everypar{}\global\let\lst@NewLine\relax}%
        \expandafter\lst@next %
    \fi}%
%    \end{macrocode}
% \end{macro}
%
% \begin{macro}{\lstFV@VerbatimEnd}
% A (particular) box and macro must exist after |\lst@DeInit|.
% We store them globally.
%    \begin{macrocode}
\gdef\lstFV@VerbatimEnd{%
    \ifx\FancyVerbFormatLine\lstFV@FancyVerbFormatLine %
        \global\setbox\lstFV@gtempboxa\box\@tempboxa %
        \global\let\@gtempa\FV@ProcessLine %
        \lst@DeInit %
        \let\FV@ProcessLine\@gtempa %
        \setbox\@tempboxa\box\lstFV@gtempboxa %
    \fi %
    \lstFV@OldVE}%
%    \end{macrocode}
%    \begin{macrocode}
\newbox\lstFV@gtempboxa %
%    \end{macrocode}
% \end{macro}
%
% \begin{macro}{\lstFV@FancyVerbFormatLine}
% A slightly different definition now: |\lstenv@AddArg{#1}| has been replaced by |\lstFV@Convert#1@|.
% The '@' terminates the input |#1|.
%    \begin{macrocode}
\gdef\lstFV@FancyVerbFormatLine#1{%
    \let\lst@arg\@empty\lstFV@Convert#1@%
    \lst@arg \lst@PrintToken\lst@EOLUpdate}%
%    \end{macrocode}
% And this macro is to be defined right now.
%    \begin{macrocode}
\gdef\lstFV@Convert{%
    \@ifnextchar\bgroup{\lstFV@Convert@Arg}{\lstFV@Convert@}}%
%    \end{macrocode}
% Coming to a begin group character ('\{' with catcode 1) we convert the and add the conversion together with group delimiters to |\lst@arg|.
% We also add |\lst@PrintToken|, which prints all collected characters before we forget them.
% Finally we continue the conversion.
%    \begin{macrocode}
\gdef\lstFV@Convert@Arg#1{%
    {\let\lst@arg\@empty \lstFV@Convert#1@\global\let\@gtempa\lst@arg}%
    \expandafter\lst@lAddTo\expandafter\lst@arg\expandafter{%
    \expandafter{\@gtempa\lst@PrintToken}}%
    \lstFV@Convert}%
%    \end{macrocode}
% If we haven't found a |\bgroup|, we look whether we've found the end of the input.
% If not, we convert one token ((non)active character or control sequence) and continue conversion.
%    \begin{macrocode}
\gdef\lstFV@Convert@#1{%
    \ifx @#1\else %
        \expandafter\lstFV@Convert@@\string#1\@empty\@empty\relax{#1}%
        \expandafter\lstFV@Convert %
    \fi}%
%    \end{macrocode}
% |\string#1| from above expands to 'backslash + some characters' if |#1| is a control sequence.
% In this case the argument |#2| here equals not |\@empty|.
% The other implication: If |#2| equals |\@empty|, |\string#1| necessarily expanded to a single character, which can't be a control sequence (|\escapechar>=0| granted).
% Thus, we add an active character if and only if the second argument equals |\@empty|, and we append the control sequence otherwise.
% In the latter case we must print all preceding characters (and all the lost space).
%    \begin{macrocode}
\gdef\lstFV@Convert@@#1#2#3\relax#4{%
    \ifx\@empty#2%
        \lccode`\~=`#4\lowercase{\lst@lAddTo\lst@arg~}%
    \else %
        \lst@lAddTo\lst@arg{\lst@UseLostSpace\lst@PrintToken#4}%
    \fi}%
%    \end{macrocode}
% \end{macro}
% \begingroup
%    \begin{macrocode}
\endgroup %
%    \end{macrocode}
%    \begin{macrocode}
%</fancyvrb>
%    \end{macrocode}
% \endgroup
%
%
% \begingroup\small
% \section{History}
% Only major changes after version 0.15 are listed here.
% Previous changes are still present in the \texttt{.dtx}-file.
% Introductory version numbers of the user commands are listed in the user's guide.
% \renewcommand\labelitemi{--}
% \begin{itemize}
% \iffalse
% \item[0.1] from 1996/03/09
%	\item test version to look whether package is possible or not
% \item[0.11] from 1996/08/19
%	\item additional blank option (= language)
%	\item alignment improved by rewriting some macros
% \item[0.12] from 1997/01/16
%	\item nearly 'perfect' alignment now
% \item[0.13] from 1997/02/11
%	\item additional languages: Eiffel, Fortran 90, Modula-2, Pascal XSC
%	\item load on demand: language specific macros moved to driver files
%	\item comments are declared now and not implemented for each language again (this makes the \TeX{} sources easier to read)
% \item[0.14] from 1997/02/18
%	\item user's guide rewritten
%	\item implementation guide uses macro environment from the doc package
%	\item (non) case sensitivity implemented, e.g.\ Pascal is not
%	\item multiple stringizer implemented, i.e.\ Modula-2 handles both string types: quotes and double quotes
%	\item comment declaration is user-accessible now
%	\item package compatible to \verb!german.sty! now
% \item[0.15] from 1997/04/18
%	\item additional languages: Java, Turbo Pascal
%	\item package renamed from listing.dtx to listings.dtx, since there is already a package named listing
% \fi
% \item[0.16] from 1997/06/01
% \iffalse
%	\item changed '$<$' to '$>$' in \verb!\lst@SkipUptoFirst!
%   \item bug removed: \verb!\lst@Init! must be placed before \verb!\lst@SkipUptoFirst!
% \fi
%   \item listing environment rewritten
% \item[0.17] from 1997/09/29
%	\item |\spreadlisting| works correct now (e.g.\ page numbers move not right any more)
%	\item speed up things (quick 'if parameter empty', all |\long| except one removed, faster \verb!\lst@GotoNextTabStop!, etc.)
%	\item alignment of wide other characters improved (e.g.\ $==$)
% \iffalse
%	\item many new languages: Ada, Algol, Cobol, Comal 80, Elan, Fortran 77, Lisp, Logo, Matlab, Oberon, Perl, PL/I, Simula, SQL, \TeX{}
% \fi
% \item[pre-0.18] from 1998/03/24 (unpublished)
%	\item bug concerning |\labelstyle| removed (now oldstylenum example works)
%	\item experimental implementation of character classes
% \item[0.19] from 1998/11/09
%	\item character classes and \lst-aspects (new) seem to be the ultimate, all became implemented in these terms
%	\item \lst-aspects became an application (new) to \textsf{keyval} and hooks
%	\item \textsf{fancyvrb} support
% \end{itemize}
%
%
% \setcounter{IndexColumns}{2}
% \PrintIndex
%
%
% \Finale
%
\endinput
