% \begin{meta-comment}
%
% $Id$
%
% Allow @-commands
%
% (c) 1995 Mark Wooding
%
%----- Revision history -----------------------------------------------------
%
% $Log$
% Revision 1.1  2000-07-13 09:10:20  michael
% + Initial import
%
% Revision 1.1  1998/09/21 10:18:06  michael
% Initial implementation
%
% Revision 1.3  1996/11/19 20:46:55  mdw
% Entered into RCS
%
%
% \end{meta-comment}
%
% \begin{meta-comment} <general public licence>
%%
%% at package -- support for `@' commands'
%% Copyright (c) 1996 Mark Wooding
%%
%% This program is free software; you can redistribute it and/or modify
%% it under the terms of the GNU General Public License as published by
%% the Free Software Foundation; either version 2 of the License, or
%% (at your option) any later version.
%%
%% This program is distributed in the hope that it will be useful,
%% but WITHOUT ANY WARRANTY; without even the implied warranty of
%% MERCHANTABILITY or FITNESS FOR A PARTICULAR PURPOSE.  See the
%% GNU General Public License for more details.
%%
%% You should have received a copy of the GNU General Public License
%% along with this program; if not, write to the Free Software
%% Foundation, Inc., 675 Mass Ave, Cambridge, MA 02139, USA.
%%
% \end{meta-comment}
%
% \begin{meta-comment} <Package preamble>
%<+package>\NeedsTeXFormat{LaTeX2e}
%<+package>\ProvidesPackage{at}
%<+package>                [1996/05/02 1.3 @-command support (MDW)]
% \end{meta-comment}
%
% \CheckSum{355}
%% \CharacterTable
%%  {Upper-case    \A\B\C\D\E\F\G\H\I\J\K\L\M\N\O\P\Q\R\S\T\U\V\W\X\Y\Z
%%   Lower-case    \a\b\c\d\e\f\g\h\i\j\k\l\m\n\o\p\q\r\s\t\u\v\w\x\y\z
%%   Digits        \0\1\2\3\4\5\6\7\8\9
%%   Exclamation   \!     Double quote  \"     Hash (number) \#
%%   Dollar        \$     Percent       \%     Ampersand     \&
%%   Acute accent  \'     Left paren    \(     Right paren   \)
%%   Asterisk      \*     Plus          \+     Comma         \,
%%   Minus         \-     Point         \.     Solidus       \/
%%   Colon         \:     Semicolon     \;     Less than     \<
%%   Equals        \=     Greater than  \>     Question mark \?
%%   Commercial at \@     Left bracket  \[     Backslash     \\
%%   Right bracket \]     Circumflex    \^     Underscore    \_
%%   Grave accent  \`     Left brace    \{     Vertical bar  \|
%%   Right brace   \}     Tilde         \~}
%%
%
% \begin{meta-comment} <driver>
%
%<*driver>
% \begin{meta-comment}
%
% $Id$
%
% Common declarations for mdwtools.dtx files
%
% (c) 1996 Mark Wooding
%
%----- Revision history -----------------------------------------------------
%
% $Log$
% Revision 1.1  1998-09-21 10:19:01  michael
% Initial implementation
%
% Revision 1.4  1996/11/19 20:55:55  mdw
% Entered into RCS
%
%
% \end{meta-comment}
%
% \begin{meta-comment} <general public licence>
%%
%% mdwtools common declarations
%% Copyright (c) 1996 Mark Wooding
%%
%% This program is free software; you can redistribute it and/or modify
%% it under the terms of the GNU General Public License as published by
%% the Free Software Foundation; either version 2 of the License, or
%% (at your option) any later version.
%%
%% This program is distributed in the hope that it will be useful,
%% but WITHOUT ANY WARRANTY; without even the implied warranty of
%% MERCHANTABILITY or FITNESS FOR A PARTICULAR PURPOSE.  See the
%% GNU General Public License for more details.
%%
%% You should have received a copy of the GNU General Public License
%% along with this program; if not, write to the Free Software
%% Foundation, Inc., 675 Mass Ave, Cambridge, MA 02139, USA.
%%
% \end{meta-comment}
%
% \begin{meta-comment} <file preamble>
%<*mdwtools>
\ProvidesFile{mdwtools.tex}
             [1996/05/10 1.4 Shared definitions for mdwtools .dtx files]
%</mdwtools>
% \end{meta-comment}
%
% \CheckSum{668}
%% \CharacterTable
%%  {Upper-case    \A\B\C\D\E\F\G\H\I\J\K\L\M\N\O\P\Q\R\S\T\U\V\W\X\Y\Z
%%   Lower-case    \a\b\c\d\e\f\g\h\i\j\k\l\m\n\o\p\q\r\s\t\u\v\w\x\y\z
%%   Digits        \0\1\2\3\4\5\6\7\8\9
%%   Exclamation   \!     Double quote  \"     Hash (number) \#
%%   Dollar        \$     Percent       \%     Ampersand     \&
%%   Acute accent  \'     Left paren    \(     Right paren   \)
%%   Asterisk      \*     Plus          \+     Comma         \,
%%   Minus         \-     Point         \.     Solidus       \/
%%   Colon         \:     Semicolon     \;     Less than     \<
%%   Equals        \=     Greater than  \>     Question mark \?
%%   Commercial at \@     Left bracket  \[     Backslash     \\
%%   Right bracket \]     Circumflex    \^     Underscore    \_
%%   Grave accent  \`     Left brace    \{     Vertical bar  \|
%%   Right brace   \}     Tilde         \~}
%%
%
% \section{Introduction and user guide}
%
% This file is really rather strange; it gets |\input| by other package
% documentation files to set up most of the environmental gubbins for them.
% It handles almost everything, like loading a document class, finding any
% packages, and building and formatting the title.
%
% It also offers an opportunity for users to customise my nice documentation,
% by using a |mdwtools.cfg| file (not included).
%
%
% \subsection{Declarations}
%
% A typical documentation file contains something like
% \begin{listinglist} \listingsize \obeylines
% |% \begin{meta-comment}
%
% $Id$
%
% Common declarations for mdwtools.dtx files
%
% (c) 1996 Mark Wooding
%
%----- Revision history -----------------------------------------------------
%
% $Log$
% Revision 1.1  1998-09-21 10:19:01  michael
% Initial implementation
%
% Revision 1.4  1996/11/19 20:55:55  mdw
% Entered into RCS
%
%
% \end{meta-comment}
%
% \begin{meta-comment} <general public licence>
%%
%% mdwtools common declarations
%% Copyright (c) 1996 Mark Wooding
%%
%% This program is free software; you can redistribute it and/or modify
%% it under the terms of the GNU General Public License as published by
%% the Free Software Foundation; either version 2 of the License, or
%% (at your option) any later version.
%%
%% This program is distributed in the hope that it will be useful,
%% but WITHOUT ANY WARRANTY; without even the implied warranty of
%% MERCHANTABILITY or FITNESS FOR A PARTICULAR PURPOSE.  See the
%% GNU General Public License for more details.
%%
%% You should have received a copy of the GNU General Public License
%% along with this program; if not, write to the Free Software
%% Foundation, Inc., 675 Mass Ave, Cambridge, MA 02139, USA.
%%
% \end{meta-comment}
%
% \begin{meta-comment} <file preamble>
%<*mdwtools>
\ProvidesFile{mdwtools.tex}
             [1996/05/10 1.4 Shared definitions for mdwtools .dtx files]
%</mdwtools>
% \end{meta-comment}
%
% \CheckSum{668}
%% \CharacterTable
%%  {Upper-case    \A\B\C\D\E\F\G\H\I\J\K\L\M\N\O\P\Q\R\S\T\U\V\W\X\Y\Z
%%   Lower-case    \a\b\c\d\e\f\g\h\i\j\k\l\m\n\o\p\q\r\s\t\u\v\w\x\y\z
%%   Digits        \0\1\2\3\4\5\6\7\8\9
%%   Exclamation   \!     Double quote  \"     Hash (number) \#
%%   Dollar        \$     Percent       \%     Ampersand     \&
%%   Acute accent  \'     Left paren    \(     Right paren   \)
%%   Asterisk      \*     Plus          \+     Comma         \,
%%   Minus         \-     Point         \.     Solidus       \/
%%   Colon         \:     Semicolon     \;     Less than     \<
%%   Equals        \=     Greater than  \>     Question mark \?
%%   Commercial at \@     Left bracket  \[     Backslash     \\
%%   Right bracket \]     Circumflex    \^     Underscore    \_
%%   Grave accent  \`     Left brace    \{     Vertical bar  \|
%%   Right brace   \}     Tilde         \~}
%%
%
% \section{Introduction and user guide}
%
% This file is really rather strange; it gets |\input| by other package
% documentation files to set up most of the environmental gubbins for them.
% It handles almost everything, like loading a document class, finding any
% packages, and building and formatting the title.
%
% It also offers an opportunity for users to customise my nice documentation,
% by using a |mdwtools.cfg| file (not included).
%
%
% \subsection{Declarations}
%
% A typical documentation file contains something like
% \begin{listinglist} \listingsize \obeylines
% |% \begin{meta-comment}
%
% $Id$
%
% Common declarations for mdwtools.dtx files
%
% (c) 1996 Mark Wooding
%
%----- Revision history -----------------------------------------------------
%
% $Log$
% Revision 1.1  1998-09-21 10:19:01  michael
% Initial implementation
%
% Revision 1.4  1996/11/19 20:55:55  mdw
% Entered into RCS
%
%
% \end{meta-comment}
%
% \begin{meta-comment} <general public licence>
%%
%% mdwtools common declarations
%% Copyright (c) 1996 Mark Wooding
%%
%% This program is free software; you can redistribute it and/or modify
%% it under the terms of the GNU General Public License as published by
%% the Free Software Foundation; either version 2 of the License, or
%% (at your option) any later version.
%%
%% This program is distributed in the hope that it will be useful,
%% but WITHOUT ANY WARRANTY; without even the implied warranty of
%% MERCHANTABILITY or FITNESS FOR A PARTICULAR PURPOSE.  See the
%% GNU General Public License for more details.
%%
%% You should have received a copy of the GNU General Public License
%% along with this program; if not, write to the Free Software
%% Foundation, Inc., 675 Mass Ave, Cambridge, MA 02139, USA.
%%
% \end{meta-comment}
%
% \begin{meta-comment} <file preamble>
%<*mdwtools>
\ProvidesFile{mdwtools.tex}
             [1996/05/10 1.4 Shared definitions for mdwtools .dtx files]
%</mdwtools>
% \end{meta-comment}
%
% \CheckSum{668}
%% \CharacterTable
%%  {Upper-case    \A\B\C\D\E\F\G\H\I\J\K\L\M\N\O\P\Q\R\S\T\U\V\W\X\Y\Z
%%   Lower-case    \a\b\c\d\e\f\g\h\i\j\k\l\m\n\o\p\q\r\s\t\u\v\w\x\y\z
%%   Digits        \0\1\2\3\4\5\6\7\8\9
%%   Exclamation   \!     Double quote  \"     Hash (number) \#
%%   Dollar        \$     Percent       \%     Ampersand     \&
%%   Acute accent  \'     Left paren    \(     Right paren   \)
%%   Asterisk      \*     Plus          \+     Comma         \,
%%   Minus         \-     Point         \.     Solidus       \/
%%   Colon         \:     Semicolon     \;     Less than     \<
%%   Equals        \=     Greater than  \>     Question mark \?
%%   Commercial at \@     Left bracket  \[     Backslash     \\
%%   Right bracket \]     Circumflex    \^     Underscore    \_
%%   Grave accent  \`     Left brace    \{     Vertical bar  \|
%%   Right brace   \}     Tilde         \~}
%%
%
% \section{Introduction and user guide}
%
% This file is really rather strange; it gets |\input| by other package
% documentation files to set up most of the environmental gubbins for them.
% It handles almost everything, like loading a document class, finding any
% packages, and building and formatting the title.
%
% It also offers an opportunity for users to customise my nice documentation,
% by using a |mdwtools.cfg| file (not included).
%
%
% \subsection{Declarations}
%
% A typical documentation file contains something like
% \begin{listinglist} \listingsize \obeylines
% |\input{mdwtools}|
% \<declarations>
% |\mdwdoc|
% \end{listinglist}
% The initial |\input| reads in this file and sets up the various commands
% which may be needed.  The final |\mdwdoc| actually starts the document,
% inserting a title (which is automatically generated), a table of
% contents etc., and reads the documentation file in (using the |\DocInput|
% command from the \package{doc} package.
%
% \subsubsection{Describing packages}
%
% \DescribeMacro{\describespackage}
% \DescribeMacro{\describesclass}
% \DescribeMacro{\describesfile}
% \DescribeMacro{\describesfile*}
% The most important declarations are those which declare what the
% documentation describes.  Saying \syntax{"\\describespackage{<package>}"}
% loads the \<package> (if necessary) and adds it to the auto-generated
% title, along with a footnote containing version information.  Similarly,
% |\describesclass| adds a document class name to the title (without loading
% it -- the document itself must do this, with the |\documentclass| command).
% For files which aren't packages or classes, use the |\describesfile| or
% |\describesfile*| command (the $*$-version won't |\input| the file, which
% is handy for files like |mdwtools.tex|, which are already input).
%
% \DescribeMacro{\author}
% \DescribeMacro{\date}
% \DescribeMacro{\title}
% The |\author|, |\date| and |\title| declarations work slightly differently
% to normal -- they ensure that only the \emph{first} declaration has an
% effect.  (Don't you play with |\author|, please, unless you're using this
% program to document your own packages.)  Using |\title| suppresses the
% automatic title generation.
%
% \DescribeMacro{\docdate}
% The default date is worked out from the version string of the package or
% document class whose name is the same as that of the documentation file.
% You can choose a different `main' file by saying
% \syntax{"\\docdate{"<file>"}"}.
%
% \subsubsection{Contents handling}
%
% \DescribeMacro{\addcontents}
% A documentation file always has a table of contents.  Other
% contents-like lists can be added by saying
% \syntax{"\\addcontents{"<extension>"}{"<command>"}"}.  The \<extension>
% is the file extension of the contents file (e.g., \lit{lot} for the
% list of tables); the \<command> is the command to actually typeset the
% contents file (e.g., |\listoftables|).
%
% \subsubsection{Other declarations}
%
% \DescribeMacro{\implementation}
% The \package{doc} package wants you to say
% \syntax{"\\StopEventually{"<stuff>"}"}' before describing the package
% implementation.  Using |mdwtools.tex|, you just say |\implementation|, and
% everything works.  It will automatically read in the licence text (from
% |gpl.tex|, and wraps some other things up.
%
% 
% \subsection{Other commands}
%
% The |mdwtools.tex| file includes the \package{syntax} and \package{sverb}
% packages so that they can be used in documentation files.  It also defines
% some trivial commands of its own.
%
% \DescribeMacro{\<}
% Saying \syntax{"\\<"<text>">" is the same as "\\synt{"<text>"}"}; this
% is a simple abbreviation.
%
% \DescribeMacro{\smallf}
% Saying \syntax{"\\smallf" <number>"/"<number>} typesets a little fraction,
% like this: \smallf 3/4.  It's useful when you want to say that the default
% value of a length is 2 \smallf 1/2\,pt, or something like that.
%
%
% \subsection{Customisation}
%
% You can customise the way that the package documentation looks by writing
% a file called |mdwtools.cfg|.  You can redefine various commands (before
% they're defined here, even; |mdwtools.tex| checks most of the commands that
% it defines to make sure they haven't been defined already.
%
% \DescribeMacro{\indexing}
% If you don't want the prompt about whether to generate index files, you
% can define the |\indexing| command to either \lit{y} or \lit{n}.  I'd
% recommend that you use |\providecommand| for this, to allow further
% customisation from the command line.
%
% \DescribeMacro{\mdwdateformat}
% If you don't like my date format (maybe you're American or something),
% you can redefine the |\mdwdateformat| command.  It takes three arguments:
% the year, month and date, as numbers; it should expand to something which
% typesets the date nicely.  The default format gives something like
% `10 May 1996'.  You can produce something rather more exotic, like
% `10\textsuperscript{th} May \textsc{\romannumeral 1996}' by saying
%\begin{listing}
%\newcommand{\mdwdateformat}[3]{%
%  \number#3\textsuperscript{\numsuffix{#3}}\ %
%  \monthname{#2}\ %
%  \textsc{\romannumeral #1}%
%}
%\end{listing}
% \DescribeMacro{\monthname}
% \DescribeMacro{\numsuffix}
% Saying \syntax{"\\monthname{"<number>"}"} expands to the name of the
% numbered month (which can be useful when doing date formats).  Saying
% \syntax{"\\numsuffix{"<number>"}"} will expand to the appropriate suffix
% (`th' or `rd' or whatever) for the \<number>.  You'll have to superscript
% it yourself, if this is what you want to do.  Putting the year number
% in roman numerals is just pretentious |;-)|.
%
% \DescribeMacro{\mdwhook}
% After all the declarations in |mdwtools.tex|, the command |\mdwhook| is
% executed, if it exists.  This can be set up by the configuration file
% to do whatever you want.
%
% There are lots of other things you can play with; you should look at the
% implementation section to see what's possible.
%
% \implementation
%
% \section{Implementation}
%
%    \begin{macrocode}
%<*mdwtools>
%    \end{macrocode}
%
% The first thing is that I'm not a \LaTeX\ package or anything official
% like that, so I must enable `|@|' as a letter by hand.
%
%    \begin{macrocode}
\makeatletter
%    \end{macrocode}
%
% Now input the user's configuration file, if it exists.  This is fairly
% simple stuff.
%
%    \begin{macrocode}
\@input{mdwtools.cfg}
%    \end{macrocode}
%
% Well, that's the easy bit done.
%
%
% \subsection{Initialisation}
%
% Obviously the first thing to do is to obtain a document class.  Obviously,
% it would be silly to do this if a document class has already been loaded,
% either by the package documentation or by the configuration file.
%
% The only way I can think of for finding out if a document class is already
% loaded is by seeing if the |\documentclass| command has been redefined
% to raise an error.  This isn't too hard, really.
%
%    \begin{macrocode}
\ifx\documentclass\@twoclasseserror\else
  \documentclass[a4paper]{ltxdoc}
  \ifx\doneclasses\mdw@undefined\else\doneclasses\fi
\fi
%    \end{macrocode}
%
% As part of my standard environment, I'll load some of my more useful
% packages.  If they're already loaded (possibly with different options),
% I'll not try to load them again.
%
%    \begin{macrocode}
\@ifpackageloaded{doc}{}{\usepackage{doc}}
\@ifpackageloaded{syntax}{}{\usepackage[rounded]{syntax}}
\@ifpackageloaded{sverb}{}{\usepackage{sverb}}
%    \end{macrocode}
%
%
% \subsection{Some macros for interaction}
%
% I like the \LaTeX\ star-boxes, although it's a pain having to cope with
% \TeX's space-handling rules.  I'll define a new typing-out macro which
% makes spaces more significant, and has a $*$-version which doesn't put
% a newline on the end, and interacts prettily with |\read|.
%
% First of all, I need to make spaces active, so I can define things about
% active spaces.
%
%    \begin{macrocode}
\begingroup\obeyspaces
%    \end{macrocode}
%
% Now to define the main macro.  This is easy stuff.  Spaces must be
% carefully rationed here, though.
%
% I'll start a group, make spaces active, and make spaces expand to ordinary
% space-like spaces.  Then I'll look for a star, and pass either |\message|
% (which doesn't start a newline, and interacts with |\read| well) or
% |\immediate\write 16| which does a normal write well.
%
%    \begin{macrocode}
\gdef\mdwtype{%
\begingroup\catcode`\ \active\let \space%
\@ifstar{\mdwtype@i{\message}}{\mdwtype@i{\immediate\write\sixt@@n}}%
}
\endgroup
%    \end{macrocode}
%
% Now for the easy bit.  I have the thing to do, and the thing to do it to,
% so do that and end the group.
%
%    \begin{macrocode}
\def\mdwtype@i#1#2{#1{#2}\endgroup}
%    \end{macrocode}
%
%
% \subsection{Decide on indexing}
%
% A configuration file can decide on indexing by defining the |\indexing|
% macro to either \lit{y} or \lit{n}.  If it's not set, then I'll prompt
% the user.
%
% First of all, I want a switch to say whether I'm indexing.
%
%    \begin{macrocode}
\newif\ifcreateindex
%    \end{macrocode}
%
% Right: now I need to decide how to make progress.  If the macro's not set,
% then I want to set it, and start a row of stars.
%
%    \begin{macrocode}
\ifx\indexing\@@undefined
  \mdwtype{*****************************}
  \def\indexing{?}
\fi
%    \end{macrocode}
%
% Now enter a loop, asking the user whether to do indexing, until I get
% a sensible answer.
%
%    \begin{macrocode}
\loop
  \@tempswafalse
  \if y\indexing\@tempswatrue\createindextrue\fi
  \if Y\indexing\@tempswatrue\createindextrue\fi
  \if n\indexing\@tempswatrue\createindexfalse\fi
  \if N\indexing\@tempswatrue\createindexfalse\fi
  \if@tempswa\else
  \mdwtype*{* Create index files? (y/n) *}
  \read\sixt@@n to\indexing%
\repeat
%    \end{macrocode}
%
% Now, based on the results of that, display a message about the indexing.
%
%    \begin{macrocode}
\mdwtype{*****************************}
\ifcreateindex
  \mdwtype{* Creating index files      *}
  \mdwtype{* This may take some time   *}
\else
  \mdwtype{* Not creating index files  *}
\fi
\mdwtype{*****************************}
%    \end{macrocode}
%
% Now I can play with the indexing commands of the \package{doc} package
% to do whatever it is that the user wants.
%
%    \begin{macrocode}
\ifcreateindex
  \CodelineIndex
  \EnableCrossrefs
\else
  \CodelineNumbered
  \DisableCrossrefs
\fi
%    \end{macrocode}
%
% And register lots of plain \TeX\ things which shouldn't be indexed.
% This contains lots of |\if|\dots\ things which don't fit nicely in
% conditionals, which is a shame.  Still, it doesn't matter that much,
% really.
%
%    \begin{macrocode}
\DoNotIndex{\def,\long,\edef,\xdef,\gdef,\let,\global}
\DoNotIndex{\if,\ifnum,\ifdim,\ifcat,\ifmmode,\ifvmode,\ifhmode,%
            \iftrue,\iffalse,\ifvoid,\ifx,\ifeof,\ifcase,\else,\or,\fi}
\DoNotIndex{\box,\copy,\setbox,\unvbox,\unhbox,\hbox,%
            \vbox,\vtop,\vcenter}
\DoNotIndex{\@empty,\immediate,\write}
\DoNotIndex{\egroup,\bgroup,\expandafter,\begingroup,\endgroup}
\DoNotIndex{\divide,\advance,\multiply,\count,\dimen}
\DoNotIndex{\relax,\space,\string}
\DoNotIndex{\csname,\endcsname,\@spaces,\openin,\openout,%
            \closein,\closeout}
\DoNotIndex{\catcode,\endinput}
\DoNotIndex{\jobname,\message,\read,\the,\m@ne,\noexpand}
\DoNotIndex{\hsize,\vsize,\hskip,\vskip,\kern,\hfil,\hfill,\hss}
\DoNotIndex{\m@ne,\z@,\z@skip,\@ne,\tw@,\p@}
\DoNotIndex{\dp,\wd,\ht,\vss,\unskip}
%    \end{macrocode}
%
% Last bit of indexing stuff, for now: I'll typeset the index in two columns
% (the default is three, which makes them too narrow for my tastes).
%
%    \begin{macrocode}
\setcounter{IndexColumns}{2}
%    \end{macrocode}
%
%
% \subsection{Selectively defining things}
%
% I don't want to tread on anyone's toes if they redefine any of these
% commands and things in a configuration file.  The following definitions
% are fairly evil, but should do the job OK.
%
% \begin{macro}{\@gobbledef}
%
% This macro eats the following |\def|inition, leaving not a trace behind.
%
%    \begin{macrocode}
\def\@gobbledef#1#{\@gobble}
%    \end{macrocode}
%
% \end{macro}
%
% \begin{macro}{\tdef}
% \begin{macro}{\tlet}
%
% The |\tdef| command is a sort of `tentative' definition -- it's like
% |\def| if the control sequence named doesn't already have a definition.
% |\tlet| does the same thing with |\let|.
%
%    \begin{macrocode}
\def\tdef#1{
  \ifx#1\@@undefined%
    \expandafter\def\expandafter#1%
  \else%
    \expandafter\@gobbledef%
  \fi%
}
\def\tlet#1#2{\ifx#1\@@undefined\let#1=#2\fi}
%    \end{macrocode}
%
% \end{macro}
% \end{macro}
%
%
% \subsection{General markup things}
%
% Now for some really simple things.  I'll define how to typeset package
% names and environment names (both in the sans serif font, for now).
%
%    \begin{macrocode}
\tlet\package\textsf
\tlet\env\textsf
%    \end{macrocode}
%
% I'll define the |\<|\dots|>| shortcut for syntax items suggested in the
% \package{syntax} package.
%
%    \begin{macrocode}
\tdef\<#1>{\synt{#1}}
%    \end{macrocode}
%
% And because it's used in a few places (mainly for typesetting lengths),
% here's a command for typesetting fractions in text.
%
%    \begin{macrocode}
\tdef\smallf#1/#2{\ensuremath{^{#1}\!/\!_{#2}}}
%    \end{macrocode}
%
%
% \subsection{A table environment}
%
% \begin{environment}{tab}
%
% Most of the packages don't use the (obviously perfect) \package{mdwtab}
% package, because it's big, and takes a while to load.  Here's an
% environment for typesetting centred tables.  The first (optional) argument
% is some declarations to perform.  The mandatory argument is the table
% preamble (obviously).
%
%    \begin{macrocode}
\@ifundefined{tab}{%
  \newenvironment{tab}[2][\relax]{%
    \par\vskip2ex%
    \centering%
    #1%
    \begin{tabular}{#2}%
  }{%
    \end{tabular}%
    \par\vskip2ex%
  }
}{}
%    \end{macrocode}
%
% \end{environment}
%
%
% \subsection{Commenting out of stuff}
%
% \begin{environment}{meta-comment}
%
% Using |\iffalse|\dots|\fi| isn't much fun.  I'll define a gobbling
% environment using the \package{sverb} stuff.
%
%    \begin{macrocode}
\ignoreenv{meta-comment}
%    \end{macrocode}
%
% \end{environment}
%
%
% \subsection{Float handling}
%
% This gubbins will try to avoid float pages as much as possible, and (with
% any luck) encourage floats to be put on the same pages as text.
%
%    \begin{macrocode}
\def\textfraction{0.1}
\def\topfraction{0.9}
\def\bottomfraction{0.9}
\def\floatpagefraction{0.7}
%    \end{macrocode}
%
% Now redefine the default float-placement parameters to allow `here' floats.
%
%    \begin{macrocode}
\def\fps@figure{htbp}
\def\fps@table{htbp}
%    \end{macrocode}
%
%
% \subsection{Other bits of parameter tweaking}
%
% Make \env{grammar} environments look pretty, by indenting the left hand
% sides by a large amount.
%
%    \begin{macrocode}
\grammarindent1in
%    \end{macrocode}
%
% I don't like being told by \TeX\ that my paragraphs are hard to linebreak:
% I know this already.  This lot should shut \TeX\ up about most problems.
%
%    \begin{macrocode}
\sloppy
\hbadness\@M
\hfuzz10\p@
%    \end{macrocode}
%
% Also make \TeX\ shut up in the index.  The \package{multicol} package
% irritatingly plays with |\hbadness|.  This is the best hook I could find
% for playing with this setting.
%
%    \begin{macrocode}
\expandafter\def\expandafter\IndexParms\expandafter{%
  \IndexParms%
  \hbadness\@M%
}
%    \end{macrocode}
%
% The other thing I really don't like is `Marginpar moved' warnings.  This
% will get rid of them, and lots of other \LaTeX\ warnings at the same time.
%
%    \begin{macrocode}
\let\@latex@warning@no@line\@gobble
%    \end{macrocode}
%
% Put some extra space between table rows, please.
%
%    \begin{macrocode}
\def\arraystretch{1.2}
%    \end{macrocode}
%
% Most of the code is at guard level one, so typeset that in upright text.
%
%    \begin{macrocode}
\setcounter{StandardModuleDepth}{1}
%    \end{macrocode}
%
%
% \subsection{Contents handling}
%
% I use at least one contents file (the main table of contents) although
% I may want more.  I'll keep a list of contents files which I need to
% handle.
%
% There are two things I need to do to contents files here:
% \begin{itemize}
% \item I must typeset the table of contents at the beginning of the
%       document; and
% \item I want to typeset tables of contents in two columns (using the
%       \package{multicol} package).
% \end{itemize}
%
% The list consists of items of the form
% \syntax{"\\do{"<extension>"}{"<command>"}"}, where \<extension> is the
% file extension of the contents file, and \<command> is the command to
% typeset it.
%
% \begin{macro}{\docontents}
%
% This is where I keep the list of contents files.  I'll initialise it to
% just do the standard contents table.
%
%    \begin{macrocode}
\def\docontents{\do{toc}{\tableofcontents}}
%    \end{macrocode}
%
% \end{macro}
%
% \begin{macro}{\addcontents}
%
% By saying \syntax{"\\addcontents{"<extension>"}{"<command>"}"}, a document
% can register a new table of contents which gets given the two-column
% treatment properly.  This is really easy to implement.
%
%    \begin{macrocode}
\def\addcontents#1#2{%
  \toks@\expandafter{\docontents\do{#1}{#2}}%
  \edef\docontents{\the\toks@}%
}
%    \end{macrocode}
%
% \end{macro}
%
%
% \subsection{Finishing it all off}
%
% \begin{macro}{\finalstuff}
%
% The |\finalstuff| macro is a hook for doing things at the end of the
% document.  Currently, it inputs the licence agreement as an appendix.
%
%    \begin{macrocode}
\tdef\finalstuff{\appendix\part*{Appendix}\input{gpl}}
%    \end{macrocode}
%
% \end{macro}
%
% \begin{macro}{\implementation}
%
% The |\implementation| macro starts typesetting the implementation of
% the package(s).  If we're not doing the implementation, it just does
% this lot and ends the input file.
%
% I define a macro with arguments inside the |\StopEventually|, which causes
% problems, since the code gets put through an extra level of |\def|fing
% depending on whether the implementation stuff gets typeset or not.  I'll
% store the code I want to do in a separate macro.
%
%    \begin{macrocode}
\def\implementation{\StopEventually{\attheend}}
%    \end{macrocode}
%
% Now for the actual activity.  First, I'll do the |\finalstuff|.  Then, if
% \package{doc}'s managed to find the \package{multicol} package, I'll add
% the end of the environment to the end of each contents file in the list.
% Finally, I'll read the index in from its formatted |.ind| file.
%
%    \begin{macrocode}
\tdef\attheend{%
  \finalstuff%
  \ifhave@multicol%
    \def\do##1##2{\addtocontents{##1}{\protect\end{multicols}}}%
    \docontents%
  \fi%
  \PrintIndex%
}
%    \end{macrocode}
%
% \end{macro}
%
%
% \subsection{File version information}
%
% \begin{macro}{\mdwpkginfo}
%
% For setting up the automatic titles, I'll need to be able to work out
% file versions and things.  This macro will, given a file name, extract
% from \LaTeX\ the version information and format it into a sensible string.
%
% First of all, I'll put the original string (direct from the
% |\Provides|\dots\ command).  Then I'll pass it to another macro which can
% parse up the string into its various bits, along with the original
% filename.
%
%    \begin{macrocode}
\def\mdwpkginfo#1{%
  \edef\@tempa{\csname ver@#1\endcsname}%
  \expandafter\mdwpkginfo@i\@tempa\@@#1\@@%
}
%    \end{macrocode}
%
% Now for the real business.  I'll store the string I build in macros called
% \syntax{"\\"<filename>"date", "\\"<filename>"version" and
% "\\"<filename>"info"}, which store the file's date, version and
% `information string' respectively.  (Note that the file extension isn't
% included in the name.)
%
% This is mainly just tedious playing with |\expandafter|.  The date format
% is defined by a separate macro, which can be modified from the
% configuration file.
%
%    \begin{macrocode}
\def\mdwpkginfo@i#1/#2/#3 #4 #5\@@#6.#7\@@{%
  \expandafter\def\csname #6date\endcsname%
    {\protect\mdwdateformat{#1}{#2}{#3}}%
  \expandafter\def\csname #6version\endcsname{#4}%
  \expandafter\def\csname #6info\endcsname{#5}%
}
%    \end{macrocode}
%
% \end{macro}
%
% \begin{macro}{\mdwdateformat}
%
% Given three arguments, a year, a month and a date (all numeric), build a
% pretty date string.  This is fairly simple really.
%
%    \begin{macrocode}
\tdef\mdwdateformat#1#2#3{\number#3\ \monthname{#2}\ \number#1}
\def\monthname#1{%
  \ifcase#1\or%
     January\or February\or March\or April\or May\or June\or%
     July\or August\or September\or October\or November\or December%
  \fi%
}
\def\numsuffix#1{%
  \ifnum#1=1 st\else%
  \ifnum#1=2 nd\else%
  \ifnum#1=3 rd\else%
  \ifnum#1=21 st\else%
  \ifnum#1=22 nd\else%
  \ifnum#1=23 rd\else%
  \ifnum#1=31 st\else%
  th%
  \fi\fi\fi\fi\fi\fi\fi%
}
%    \end{macrocode}
%
% \end{macro}
%
% \begin{macro}{\mdwfileinfo}
%
% Saying \syntax{"\\mdwfileinfo{"<file-name>"}{"<info>"}"} extracts the
% wanted item of \<info> from the version information for file \<file-name>.
%
%    \begin{macrocode}
\def\mdwfileinfo#1#2{\mdwfileinfo@i{#2}#1.\@@}
\def\mdwfileinfo@i#1#2.#3\@@{\csname#2#1\endcsname}
%    \end{macrocode}
%
% \end{macro}
%
%
% \subsection{List handling}
%
% There are several other lists I need to build.  These macros will do
% the necessary stuff.
%
% \begin{macro}{\mdw@ifitem}
%
% The macro \syntax{"\\mdw@ifitem"<item>"\\in"<list>"{"<true-text>"}"^^A
% "{"<false-text>"}"} does \<true-text> if the \<item> matches any item in
% the \<list>; otherwise it does \<false-text>.
%
%    \begin{macrocode}
\def\mdw@ifitem#1\in#2{%
  \@tempswafalse%
  \def\@tempa{#1}%
  \def\do##1{\def\@tempb{##1}\ifx\@tempa\@tempb\@tempswatrue\fi}%
  #2%
  \if@tempswa\expandafter\@firstoftwo\else\expandafter\@secondoftwo\fi%
}
%    \end{macrocode}
%
% \end{macro}
%
% \begin{macro}{\mdw@append}
%
% Saying \syntax{"\\mdw@append"<item>"\\to"<list>} adds the given \<item>
% to the end of the given \<list>.
%
%    \begin{macrocode}
\def\mdw@append#1\to#2{%
  \toks@{\do{#1}}%
  \toks\tw@\expandafter{#2}%
  \edef#2{\the\toks\tw@\the\toks@}%
}
%    \end{macrocode}
%
% \end{macro}
%
% \begin{macro}{\mdw@prepend}
%
% Saying \syntax{"\\mdw@prepend"<item>"\\to"<list>} adds the \<item> to the
% beginning of the \<list>.
%
%    \begin{macrocode}
\def\mdw@prepend#1\to#2{%
  \toks@{\do{#1}}%
  \toks\tw@\expandafter{#2}%
  \edef#2{\the\toks@\the\toks\tw@}%
}
%    \end{macrocode}
%
% \end{macro}
%
% \begin{macro}{\mdw@add}
%
% Finally, saying \syntax{"\\mdw@add"<item>"\\to"<list>} adds the \<item>
% to the list only if it isn't there already.
%
%    \begin{macrocode}
\def\mdw@add#1\to#2{\mdw@ifitem#1\in#2{}{\mdw@append#1\to#2}}
%    \end{macrocode}
%
% \end{macro}
%
%
% \subsection{Described file handling}
%
% I'l maintain lists of packages, document classes, and other files
% described by the current documentation file.
%
% First of all, I'll declare the various list macros.
%
%    \begin{macrocode}
\def\dopackages{}
\def\doclasses{}
\def\dootherfiles{}
%    \end{macrocode}
%
% \begin{macro}{\describespackage}
%
% A document file can declare that it describes a package by saying
% \syntax{"\\describespackage{"<package-name>"}"}.  I add the package to
% my list, read the package into memory (so that the documentation can
% offer demonstrations of it) and read the version information.
%
%    \begin{macrocode}
\def\describespackage#1{%
  \mdw@ifitem#1\in\dopackages{}{%
    \mdw@append#1\to\dopackages%
    \usepackage{#1}%
    \mdwpkginfo{#1.sty}%
  }%
}
%    \end{macrocode}
%
% \end{macro}
%
% \begin{macro}{\describesclass}
%
% By saying \syntax{"\\describesclass{"<class-name>"}"}, a document file
% can declare that it describes a document class.  I'll assume that the
% document class is already loaded, because it's much too late to load
% it now.
%
%    \begin{macrocode}
\def\describesclass#1{\mdw@add#1\to\doclasses\mdwpkginfo{#1.cls}}
%    \end{macrocode}
%
% \end{macro}
%
% \begin{macro}{\describesfile}
%
% Finally, other `random' files, which don't have the status of real \LaTeX\
% packages or document classes, can be described by saying \syntax{^^A
% "\\describesfile{"<file-name>"}" or "\\describesfile*{"<file-name>"}"}.
% The difference is that the starred version will not |\input| the file.
%
%    \begin{macrocode}
\def\describesfile{%
  \@ifstar{\describesfile@i\@gobble}{\describesfile@i\input}%
}
\def\describesfile@i#1#2{%
  \mdw@ifitem#2\in\dootherfiles{}{%
    \mdw@add#2\to\dootherfiles%
    #1{#2}%
    \mdwpkginfo{#2}%
  }%
}
%    \end{macrocode}
%
% \end{macro}
%
%
% \subsection{Author and title handling}
%
% I'll redefine the |\author| and |\title| commands so that I get told
% whether I need to do it myself.
%
% \begin{macro}{\author}
%
% This is easy: I'll save the old meaning, and then redefine |\author| to
% do the old thing and redefine itself to then do nothing.
%
%    \begin{macrocode}
\let\mdw@author\author
\def\author{\let\author\@gobble\mdw@author}
%    \end{macrocode}
%
% \end{macro}
%
% \begin{macro}{\title}
%
% And oddly enough, I'll do exactly the same thing for the title, except
% that I'll also disable the |\mdw@buildtitle| command, which constructs
% the title automatically.
%
%    \begin{macrocode}
\let\mdw@title\title
\def\title{\let\title\@gobble\let\mdw@buildtitle\relax\mdw@title}
%    \end{macrocode}
%
% \end{macro}
%
% \begin{macro}{\date}
%
% This works in a very similar sort of way.
%
%    \begin{macrocode}
\def\date#1{\let\date\@gobble\def\today{#1}}
%    \end{macrocode}
%
% \end{macro}
%
% \begin{macro}{\datefrom}
%
% Saying \syntax{"\\datefrom{"<file-name>"}"} sets the document date from
% the given filename.
%
%    \begin{macrocode}
\def\datefrom#1{%
  \protected@edef\@tempa{\noexpand\date{\csname #1date\endcsname}}%
  \@tempa%
}
%    \end{macrocode}
%
% \end{macro}
%
% \begin{macro}{\docfile}
%
% Saying \syntax{"\\docfile{"<file-name>"}"} sets up the file name from which
% documentation will be read.
%
%    \begin{macrocode}
\def\docfile#1{%
  \def\@tempa##1.##2\@@{\def\@basefile{##1.##2}\def\@basename{##1}}%
  \edef\@tempb{\noexpand\@tempa#1\noexpand\@@}%
  \@tempb%
}
%    \end{macrocode}
%
% I'll set up a default value as well.
%
%    \begin{macrocode}
\docfile{\jobname.dtx}
%    \end{macrocode}
%
% \end{macro}
%
%
% \subsection{Building title strings}
%
% This is rather tricky.  For each list, I need to build a legible looking
% string.
%
% \begin{macro}{\mdw@addtotitle}
%
% By saying
%\syntax{"\\mdw@addtotitle{"<list>"}{"<command>"}{"<singular>"}{"<plural>"}"}
% I can add the contents of a list to the current title string in the
% |\mdw@title| macro.
%
%    \begin{macrocode}
\tdef\mdw@addtotitle#1#2#3#4{%
%    \end{macrocode}
%
% Now to get to work.  I need to keep one `lookahead' list item, and a count
% of the number of items read so far.  I'll keep the lookahead item in
% |\@nextitem| and the counter in |\count@|.
%
%    \begin{macrocode}
  \count@\z@%
%    \end{macrocode}
%
% Now I'll define what to do for each list item.  The |\protect| command is
% already set up appropriately for playing with |\edef| commands.
%
%    \begin{macrocode}
  \def\do##1{%
%    \end{macrocode}
%
% The first job is to add the previous item to the title string.  If this
% is the first item, though, I'll just add the appropriate \lit{The } or
% \lit{ and the } string to the title (this is stored in the |\@prefix|
% macro).
%
%    \begin{macrocode}
    \edef\mdw@title{%
      \mdw@title%
      \ifcase\count@\@prefix%
      \or\@nextitem%
      \else, \@nextitem%
      \fi%
    }%
%    \end{macrocode}
%
% That was rather easy.  Now I'll set up the |\@nextitem| macro for the
% next time around the loop.
%
%    \begin{macrocode}
    \edef\@nextitem{%
      \protect#2{##1}%
      \protect\footnote{%
        The \protect#2{##1} #3 is currently at version %
        \mdwfileinfo{##1}{version}, dated \mdwfileinfo{##1}{date}.%
      }\space%
    }%
%    \end{macrocode}
%
% Finally, I need to increment the counter.
%
%    \begin{macrocode}
    \advance\count@\@ne%
  }%
%    \end{macrocode}
%
% Now execute the list.
%
%    \begin{macrocode}
  #1%
%    \end{macrocode}
%
% I still have one item left over, unless the list was empty.  I'll add
% that now.
%
%    \begin{macrocode}
  \edef\mdw@title{%
    \mdw@title%
    \ifcase\count@%
    \or\@nextitem\space#3%
    \or\ and \@nextitem\space#4%
    \fi%
  }%
%    \end{macrocode}
%
% Finally, if $|\count@| \ne 0$, I must set |\@prefix| to \lit{ and the }.
%
%    \begin{macrocode}
  \ifnum\count@>\z@\def\@prefix{ and the }\fi%
}
%    \end{macrocode}
%
% \end{macro}
%
% \begin{macro}{\mdw@buildtitle}
%
% This macro will actually do the job of building the title string.
%
%    \begin{macrocode}
\tdef\mdw@buildtitle{%
%    \end{macrocode}
%
% First of all, I'll open a group to avoid polluting the namespace with
% my gubbins (although the code is now much tidier than it has been in
% earlier releases).
%
%    \begin{macrocode}
  \begingroup%
%    \end{macrocode}
%
% The title building stuff makes extensive use of |\edef|.  I'll set
% |\protect| appropriately.  (For those not in the know,
% |\@unexpandable@protect| expands to `|\noexpand\protect\noexpand|',
% which prevents expansion of the following macro, and inserts a |\protect|
% in front of it ready for the next |\edef|.)
%
%    \begin{macrocode}
  \let\@@protect\protect\let\protect\@unexpandable@protect%
%    \end{macrocode}
%
% Set up some simple macros ready for the main code.
%
%    \begin{macrocode}
  \def\mdw@title{}%
  \def\@prefix{The }%
%    \end{macrocode}
%
% Now build the title.  This is fun.
%
%    \begin{macrocode}
  \mdw@addtotitle\dopackages\package{package}{packages}%
  \mdw@addtotitle\doclasses\package{document class}{document classes}%
  \mdw@addtotitle\dootherfiles\texttt{file}{files}%
%    \end{macrocode}
%
% Now I want to end the group and set the title from my string.  The
% following hacking will do this.
%
%    \begin{macrocode}
  \edef\next{\endgroup\noexpand\title{\mdw@title}}%
  \next%
}
%    \end{macrocode}
%
% \end{macro}
%
%
% \subsection{Starting the main document}
%
% \begin{macro}{\mdwdoc}
%
% Once the document preamble has done all of its stuff, it calls the
% |\mdwdoc| command, which takes over and really starts the documentation
% going.
%
%    \begin{macrocode}
\def\mdwdoc{%
%    \end{macrocode}
%
% First, I'll construct the title string.
%
%    \begin{macrocode}
  \mdw@buildtitle%
  \author{Mark Wooding}%
%    \end{macrocode}
%
% Set up the date string based on the date of the package which shares
% the same name as the current file.
%
%    \begin{macrocode}
  \datefrom\@basename%
%    \end{macrocode}
%
% Set up verbatim characters after all the packages have started.
%
%    \begin{macrocode}
  \shortverb\|%
  \shortverb\"%
%    \end{macrocode}
%
% Start the document, and put the title in.
%
%    \begin{macrocode}
  \begin{document}
  \maketitle%
%    \end{macrocode}
%
% This is nasty.  It makes maths displays work properly in demo environments.
% \emph{The \LaTeX\ Companion} exhibits the bug which this hack fixes.  So
% ner.
%
%    \begin{macrocode}
  \abovedisplayskip\z@%
%    \end{macrocode}
%
% Now start the contents tables.  After starting each one, I'll make it
% be multicolumnar.
%
%    \begin{macrocode}
  \def\do##1##2{%
    ##2%
    \ifhave@multicol\addtocontents{##1}{%
      \protect\begin{multicols}{2}%
      \hbadness\@M%
    }\fi%
  }%
  \docontents%
%    \end{macrocode}
%
% Input the main file now.
%
%    \begin{macrocode}
  \DocInput{\@basefile}%
%    \end{macrocode}
%
% That's it.  I'm done.
%
%    \begin{macrocode}
  \end{document}
}
%    \end{macrocode}
%
% \end{macro}
%
%
% \subsection{And finally\dots}
%
% Right at the end I'll put a hook for the configuration file.
%
%    \begin{macrocode}
\ifx\mdwhook\@@undefined\else\expandafter\mdwhook\fi
%    \end{macrocode}
%
% That's all the code done now.  I'll change back to `user' mode, where
% all the magic control sequences aren't allowed any more.
%
%    \begin{macrocode}
\makeatother
%</mdwtools>
%    \end{macrocode}
%
% Oh, wait!  What if I want to typeset this documentation?  Aha.  I'll cope
% with that by comparing |\jobname| with my filename |mdwtools|.  However,
% there's some fun here, because |\jobname| contains category-12 letters,
% while my letters are category-11.  Time to play with |\string| in a messy
% way.
%
%    \begin{macrocode}
%<*driver>
\makeatletter
\edef\@tempa{\expandafter\@gobble\string\mdwtools}
\edef\@tempb{\jobname}
\ifx\@tempa\@tempb
  \describesfile*{mdwtools.tex}
  \docfile{mdwtools.tex}
  \makeatother
  \expandafter\mdwdoc
\fi
\makeatother
%</driver>
%    \end{macrocode}
%
% That's it.  Done!
%
% \hfill Mark Wooding, \today
%
% \Finale
%
\endinput
|
% \<declarations>
% |\mdwdoc|
% \end{listinglist}
% The initial |\input| reads in this file and sets up the various commands
% which may be needed.  The final |\mdwdoc| actually starts the document,
% inserting a title (which is automatically generated), a table of
% contents etc., and reads the documentation file in (using the |\DocInput|
% command from the \package{doc} package.
%
% \subsubsection{Describing packages}
%
% \DescribeMacro{\describespackage}
% \DescribeMacro{\describesclass}
% \DescribeMacro{\describesfile}
% \DescribeMacro{\describesfile*}
% The most important declarations are those which declare what the
% documentation describes.  Saying \syntax{"\\describespackage{<package>}"}
% loads the \<package> (if necessary) and adds it to the auto-generated
% title, along with a footnote containing version information.  Similarly,
% |\describesclass| adds a document class name to the title (without loading
% it -- the document itself must do this, with the |\documentclass| command).
% For files which aren't packages or classes, use the |\describesfile| or
% |\describesfile*| command (the $*$-version won't |\input| the file, which
% is handy for files like |mdwtools.tex|, which are already input).
%
% \DescribeMacro{\author}
% \DescribeMacro{\date}
% \DescribeMacro{\title}
% The |\author|, |\date| and |\title| declarations work slightly differently
% to normal -- they ensure that only the \emph{first} declaration has an
% effect.  (Don't you play with |\author|, please, unless you're using this
% program to document your own packages.)  Using |\title| suppresses the
% automatic title generation.
%
% \DescribeMacro{\docdate}
% The default date is worked out from the version string of the package or
% document class whose name is the same as that of the documentation file.
% You can choose a different `main' file by saying
% \syntax{"\\docdate{"<file>"}"}.
%
% \subsubsection{Contents handling}
%
% \DescribeMacro{\addcontents}
% A documentation file always has a table of contents.  Other
% contents-like lists can be added by saying
% \syntax{"\\addcontents{"<extension>"}{"<command>"}"}.  The \<extension>
% is the file extension of the contents file (e.g., \lit{lot} for the
% list of tables); the \<command> is the command to actually typeset the
% contents file (e.g., |\listoftables|).
%
% \subsubsection{Other declarations}
%
% \DescribeMacro{\implementation}
% The \package{doc} package wants you to say
% \syntax{"\\StopEventually{"<stuff>"}"}' before describing the package
% implementation.  Using |mdwtools.tex|, you just say |\implementation|, and
% everything works.  It will automatically read in the licence text (from
% |gpl.tex|, and wraps some other things up.
%
% 
% \subsection{Other commands}
%
% The |mdwtools.tex| file includes the \package{syntax} and \package{sverb}
% packages so that they can be used in documentation files.  It also defines
% some trivial commands of its own.
%
% \DescribeMacro{\<}
% Saying \syntax{"\\<"<text>">" is the same as "\\synt{"<text>"}"}; this
% is a simple abbreviation.
%
% \DescribeMacro{\smallf}
% Saying \syntax{"\\smallf" <number>"/"<number>} typesets a little fraction,
% like this: \smallf 3/4.  It's useful when you want to say that the default
% value of a length is 2 \smallf 1/2\,pt, or something like that.
%
%
% \subsection{Customisation}
%
% You can customise the way that the package documentation looks by writing
% a file called |mdwtools.cfg|.  You can redefine various commands (before
% they're defined here, even; |mdwtools.tex| checks most of the commands that
% it defines to make sure they haven't been defined already.
%
% \DescribeMacro{\indexing}
% If you don't want the prompt about whether to generate index files, you
% can define the |\indexing| command to either \lit{y} or \lit{n}.  I'd
% recommend that you use |\providecommand| for this, to allow further
% customisation from the command line.
%
% \DescribeMacro{\mdwdateformat}
% If you don't like my date format (maybe you're American or something),
% you can redefine the |\mdwdateformat| command.  It takes three arguments:
% the year, month and date, as numbers; it should expand to something which
% typesets the date nicely.  The default format gives something like
% `10 May 1996'.  You can produce something rather more exotic, like
% `10\textsuperscript{th} May \textsc{\romannumeral 1996}' by saying
%\begin{listing}
%\newcommand{\mdwdateformat}[3]{%
%  \number#3\textsuperscript{\numsuffix{#3}}\ %
%  \monthname{#2}\ %
%  \textsc{\romannumeral #1}%
%}
%\end{listing}
% \DescribeMacro{\monthname}
% \DescribeMacro{\numsuffix}
% Saying \syntax{"\\monthname{"<number>"}"} expands to the name of the
% numbered month (which can be useful when doing date formats).  Saying
% \syntax{"\\numsuffix{"<number>"}"} will expand to the appropriate suffix
% (`th' or `rd' or whatever) for the \<number>.  You'll have to superscript
% it yourself, if this is what you want to do.  Putting the year number
% in roman numerals is just pretentious |;-)|.
%
% \DescribeMacro{\mdwhook}
% After all the declarations in |mdwtools.tex|, the command |\mdwhook| is
% executed, if it exists.  This can be set up by the configuration file
% to do whatever you want.
%
% There are lots of other things you can play with; you should look at the
% implementation section to see what's possible.
%
% \implementation
%
% \section{Implementation}
%
%    \begin{macrocode}
%<*mdwtools>
%    \end{macrocode}
%
% The first thing is that I'm not a \LaTeX\ package or anything official
% like that, so I must enable `|@|' as a letter by hand.
%
%    \begin{macrocode}
\makeatletter
%    \end{macrocode}
%
% Now input the user's configuration file, if it exists.  This is fairly
% simple stuff.
%
%    \begin{macrocode}
\@input{mdwtools.cfg}
%    \end{macrocode}
%
% Well, that's the easy bit done.
%
%
% \subsection{Initialisation}
%
% Obviously the first thing to do is to obtain a document class.  Obviously,
% it would be silly to do this if a document class has already been loaded,
% either by the package documentation or by the configuration file.
%
% The only way I can think of for finding out if a document class is already
% loaded is by seeing if the |\documentclass| command has been redefined
% to raise an error.  This isn't too hard, really.
%
%    \begin{macrocode}
\ifx\documentclass\@twoclasseserror\else
  \documentclass[a4paper]{ltxdoc}
  \ifx\doneclasses\mdw@undefined\else\doneclasses\fi
\fi
%    \end{macrocode}
%
% As part of my standard environment, I'll load some of my more useful
% packages.  If they're already loaded (possibly with different options),
% I'll not try to load them again.
%
%    \begin{macrocode}
\@ifpackageloaded{doc}{}{\usepackage{doc}}
\@ifpackageloaded{syntax}{}{\usepackage[rounded]{syntax}}
\@ifpackageloaded{sverb}{}{\usepackage{sverb}}
%    \end{macrocode}
%
%
% \subsection{Some macros for interaction}
%
% I like the \LaTeX\ star-boxes, although it's a pain having to cope with
% \TeX's space-handling rules.  I'll define a new typing-out macro which
% makes spaces more significant, and has a $*$-version which doesn't put
% a newline on the end, and interacts prettily with |\read|.
%
% First of all, I need to make spaces active, so I can define things about
% active spaces.
%
%    \begin{macrocode}
\begingroup\obeyspaces
%    \end{macrocode}
%
% Now to define the main macro.  This is easy stuff.  Spaces must be
% carefully rationed here, though.
%
% I'll start a group, make spaces active, and make spaces expand to ordinary
% space-like spaces.  Then I'll look for a star, and pass either |\message|
% (which doesn't start a newline, and interacts with |\read| well) or
% |\immediate\write 16| which does a normal write well.
%
%    \begin{macrocode}
\gdef\mdwtype{%
\begingroup\catcode`\ \active\let \space%
\@ifstar{\mdwtype@i{\message}}{\mdwtype@i{\immediate\write\sixt@@n}}%
}
\endgroup
%    \end{macrocode}
%
% Now for the easy bit.  I have the thing to do, and the thing to do it to,
% so do that and end the group.
%
%    \begin{macrocode}
\def\mdwtype@i#1#2{#1{#2}\endgroup}
%    \end{macrocode}
%
%
% \subsection{Decide on indexing}
%
% A configuration file can decide on indexing by defining the |\indexing|
% macro to either \lit{y} or \lit{n}.  If it's not set, then I'll prompt
% the user.
%
% First of all, I want a switch to say whether I'm indexing.
%
%    \begin{macrocode}
\newif\ifcreateindex
%    \end{macrocode}
%
% Right: now I need to decide how to make progress.  If the macro's not set,
% then I want to set it, and start a row of stars.
%
%    \begin{macrocode}
\ifx\indexing\@@undefined
  \mdwtype{*****************************}
  \def\indexing{?}
\fi
%    \end{macrocode}
%
% Now enter a loop, asking the user whether to do indexing, until I get
% a sensible answer.
%
%    \begin{macrocode}
\loop
  \@tempswafalse
  \if y\indexing\@tempswatrue\createindextrue\fi
  \if Y\indexing\@tempswatrue\createindextrue\fi
  \if n\indexing\@tempswatrue\createindexfalse\fi
  \if N\indexing\@tempswatrue\createindexfalse\fi
  \if@tempswa\else
  \mdwtype*{* Create index files? (y/n) *}
  \read\sixt@@n to\indexing%
\repeat
%    \end{macrocode}
%
% Now, based on the results of that, display a message about the indexing.
%
%    \begin{macrocode}
\mdwtype{*****************************}
\ifcreateindex
  \mdwtype{* Creating index files      *}
  \mdwtype{* This may take some time   *}
\else
  \mdwtype{* Not creating index files  *}
\fi
\mdwtype{*****************************}
%    \end{macrocode}
%
% Now I can play with the indexing commands of the \package{doc} package
% to do whatever it is that the user wants.
%
%    \begin{macrocode}
\ifcreateindex
  \CodelineIndex
  \EnableCrossrefs
\else
  \CodelineNumbered
  \DisableCrossrefs
\fi
%    \end{macrocode}
%
% And register lots of plain \TeX\ things which shouldn't be indexed.
% This contains lots of |\if|\dots\ things which don't fit nicely in
% conditionals, which is a shame.  Still, it doesn't matter that much,
% really.
%
%    \begin{macrocode}
\DoNotIndex{\def,\long,\edef,\xdef,\gdef,\let,\global}
\DoNotIndex{\if,\ifnum,\ifdim,\ifcat,\ifmmode,\ifvmode,\ifhmode,%
            \iftrue,\iffalse,\ifvoid,\ifx,\ifeof,\ifcase,\else,\or,\fi}
\DoNotIndex{\box,\copy,\setbox,\unvbox,\unhbox,\hbox,%
            \vbox,\vtop,\vcenter}
\DoNotIndex{\@empty,\immediate,\write}
\DoNotIndex{\egroup,\bgroup,\expandafter,\begingroup,\endgroup}
\DoNotIndex{\divide,\advance,\multiply,\count,\dimen}
\DoNotIndex{\relax,\space,\string}
\DoNotIndex{\csname,\endcsname,\@spaces,\openin,\openout,%
            \closein,\closeout}
\DoNotIndex{\catcode,\endinput}
\DoNotIndex{\jobname,\message,\read,\the,\m@ne,\noexpand}
\DoNotIndex{\hsize,\vsize,\hskip,\vskip,\kern,\hfil,\hfill,\hss}
\DoNotIndex{\m@ne,\z@,\z@skip,\@ne,\tw@,\p@}
\DoNotIndex{\dp,\wd,\ht,\vss,\unskip}
%    \end{macrocode}
%
% Last bit of indexing stuff, for now: I'll typeset the index in two columns
% (the default is three, which makes them too narrow for my tastes).
%
%    \begin{macrocode}
\setcounter{IndexColumns}{2}
%    \end{macrocode}
%
%
% \subsection{Selectively defining things}
%
% I don't want to tread on anyone's toes if they redefine any of these
% commands and things in a configuration file.  The following definitions
% are fairly evil, but should do the job OK.
%
% \begin{macro}{\@gobbledef}
%
% This macro eats the following |\def|inition, leaving not a trace behind.
%
%    \begin{macrocode}
\def\@gobbledef#1#{\@gobble}
%    \end{macrocode}
%
% \end{macro}
%
% \begin{macro}{\tdef}
% \begin{macro}{\tlet}
%
% The |\tdef| command is a sort of `tentative' definition -- it's like
% |\def| if the control sequence named doesn't already have a definition.
% |\tlet| does the same thing with |\let|.
%
%    \begin{macrocode}
\def\tdef#1{
  \ifx#1\@@undefined%
    \expandafter\def\expandafter#1%
  \else%
    \expandafter\@gobbledef%
  \fi%
}
\def\tlet#1#2{\ifx#1\@@undefined\let#1=#2\fi}
%    \end{macrocode}
%
% \end{macro}
% \end{macro}
%
%
% \subsection{General markup things}
%
% Now for some really simple things.  I'll define how to typeset package
% names and environment names (both in the sans serif font, for now).
%
%    \begin{macrocode}
\tlet\package\textsf
\tlet\env\textsf
%    \end{macrocode}
%
% I'll define the |\<|\dots|>| shortcut for syntax items suggested in the
% \package{syntax} package.
%
%    \begin{macrocode}
\tdef\<#1>{\synt{#1}}
%    \end{macrocode}
%
% And because it's used in a few places (mainly for typesetting lengths),
% here's a command for typesetting fractions in text.
%
%    \begin{macrocode}
\tdef\smallf#1/#2{\ensuremath{^{#1}\!/\!_{#2}}}
%    \end{macrocode}
%
%
% \subsection{A table environment}
%
% \begin{environment}{tab}
%
% Most of the packages don't use the (obviously perfect) \package{mdwtab}
% package, because it's big, and takes a while to load.  Here's an
% environment for typesetting centred tables.  The first (optional) argument
% is some declarations to perform.  The mandatory argument is the table
% preamble (obviously).
%
%    \begin{macrocode}
\@ifundefined{tab}{%
  \newenvironment{tab}[2][\relax]{%
    \par\vskip2ex%
    \centering%
    #1%
    \begin{tabular}{#2}%
  }{%
    \end{tabular}%
    \par\vskip2ex%
  }
}{}
%    \end{macrocode}
%
% \end{environment}
%
%
% \subsection{Commenting out of stuff}
%
% \begin{environment}{meta-comment}
%
% Using |\iffalse|\dots|\fi| isn't much fun.  I'll define a gobbling
% environment using the \package{sverb} stuff.
%
%    \begin{macrocode}
\ignoreenv{meta-comment}
%    \end{macrocode}
%
% \end{environment}
%
%
% \subsection{Float handling}
%
% This gubbins will try to avoid float pages as much as possible, and (with
% any luck) encourage floats to be put on the same pages as text.
%
%    \begin{macrocode}
\def\textfraction{0.1}
\def\topfraction{0.9}
\def\bottomfraction{0.9}
\def\floatpagefraction{0.7}
%    \end{macrocode}
%
% Now redefine the default float-placement parameters to allow `here' floats.
%
%    \begin{macrocode}
\def\fps@figure{htbp}
\def\fps@table{htbp}
%    \end{macrocode}
%
%
% \subsection{Other bits of parameter tweaking}
%
% Make \env{grammar} environments look pretty, by indenting the left hand
% sides by a large amount.
%
%    \begin{macrocode}
\grammarindent1in
%    \end{macrocode}
%
% I don't like being told by \TeX\ that my paragraphs are hard to linebreak:
% I know this already.  This lot should shut \TeX\ up about most problems.
%
%    \begin{macrocode}
\sloppy
\hbadness\@M
\hfuzz10\p@
%    \end{macrocode}
%
% Also make \TeX\ shut up in the index.  The \package{multicol} package
% irritatingly plays with |\hbadness|.  This is the best hook I could find
% for playing with this setting.
%
%    \begin{macrocode}
\expandafter\def\expandafter\IndexParms\expandafter{%
  \IndexParms%
  \hbadness\@M%
}
%    \end{macrocode}
%
% The other thing I really don't like is `Marginpar moved' warnings.  This
% will get rid of them, and lots of other \LaTeX\ warnings at the same time.
%
%    \begin{macrocode}
\let\@latex@warning@no@line\@gobble
%    \end{macrocode}
%
% Put some extra space between table rows, please.
%
%    \begin{macrocode}
\def\arraystretch{1.2}
%    \end{macrocode}
%
% Most of the code is at guard level one, so typeset that in upright text.
%
%    \begin{macrocode}
\setcounter{StandardModuleDepth}{1}
%    \end{macrocode}
%
%
% \subsection{Contents handling}
%
% I use at least one contents file (the main table of contents) although
% I may want more.  I'll keep a list of contents files which I need to
% handle.
%
% There are two things I need to do to contents files here:
% \begin{itemize}
% \item I must typeset the table of contents at the beginning of the
%       document; and
% \item I want to typeset tables of contents in two columns (using the
%       \package{multicol} package).
% \end{itemize}
%
% The list consists of items of the form
% \syntax{"\\do{"<extension>"}{"<command>"}"}, where \<extension> is the
% file extension of the contents file, and \<command> is the command to
% typeset it.
%
% \begin{macro}{\docontents}
%
% This is where I keep the list of contents files.  I'll initialise it to
% just do the standard contents table.
%
%    \begin{macrocode}
\def\docontents{\do{toc}{\tableofcontents}}
%    \end{macrocode}
%
% \end{macro}
%
% \begin{macro}{\addcontents}
%
% By saying \syntax{"\\addcontents{"<extension>"}{"<command>"}"}, a document
% can register a new table of contents which gets given the two-column
% treatment properly.  This is really easy to implement.
%
%    \begin{macrocode}
\def\addcontents#1#2{%
  \toks@\expandafter{\docontents\do{#1}{#2}}%
  \edef\docontents{\the\toks@}%
}
%    \end{macrocode}
%
% \end{macro}
%
%
% \subsection{Finishing it all off}
%
% \begin{macro}{\finalstuff}
%
% The |\finalstuff| macro is a hook for doing things at the end of the
% document.  Currently, it inputs the licence agreement as an appendix.
%
%    \begin{macrocode}
\tdef\finalstuff{\appendix\part*{Appendix}% \iffalse <meta-comment>
%
% $Id$
%
% The GNU General Public Licence as a LaTeX section
%
% (c) 1989, 1991 Free Software Foundation, Inc.
%   LaTeX markup and minor formatting changes by Mark Wooding
%

%----- Revision history -----------------------------------------------------
%
% $Log$
% Revision 1.1  1998-09-21 10:19:01  michael
% Initial implementation
%
% Revision 1.1  1996/11/19 20:51:14  mdw
% Initial revision
%

% --- Chapter heading ---
%
% We don't know whether this ought to be a section or a chapter.  Easy.
% We'll see if chapters are possible.
%
% \fi

\begingroup
\makeatletter

\edef\next#1#2#3{\relax
  \ifx\chapter\@@undefined
    \ifx\documentclass\@notprerr#2\else#3\fi
  \else#1\fi
}

\expandafter\endgroup\next
{
  \let\gpltoplevel\chapter
  \let\gplsec\section
  \let\gplend\endinput
}{
  \let\gpltoplevel\section
  \let\gplsec\subsection
  \let\gplend\endinput
}{
  \documentclass[a4paper]{article}
  \def\gpltoplevel#1{%
    \vspace*{1in}%
    \hbox to\hsize{\hfil\LARGE\bfseries#1\hfil}%
    \vspace{1in}%
  }
  \let\gplsec\section
  \def\gplend{\end{document}}
  \advance\textwidth1in
  \advance\oddsidemargin-.5in
  \sloppy
  \begin{document}
}

%^^A-------------------------------------------------------------------------
\gpltoplevel{The GNU General Public Licence}


The following is the text of the GNU General Public Licence, under the terms
of which this software is distrubuted.

\vspace{12pt}

\begin{center}
\textbf{GNU GENERAL PUBLIC LICENSE} \\
Version 2, June 1991
\end{center}

\begin{center}
Copyright (C) 1989, 1991 Free Software Foundation, Inc. \\
675 Mass Ave, Cambridge, MA 02139, USA

Everyone is permitted to copy and distribute verbatim copies \\
of this license document, but changing it is not allowed.
\end{center}


\gplsec{Preamble}

The licenses for most software are designed to take away your freedom to
share and change it.  By contrast, the GNU General Public License is intended
to guarantee your freedom to share and change free software---to make sure
the software is free for all its users.  This General Public License applies
to most of the Free Software Foundation's software and to any other program
whose authors commit to using it.  (Some other Free Software Foundation
software is covered by the GNU Library General Public License instead.)  You
can apply it to your programs, too.

When we speak of free software, we are referring to freedom, not price.  Our
General Public Licenses are designed to make sure that you have the freedom
to distribute copies of free software (and charge for this service if you
wish), that you receive source code or can get it if you want it, that you
can change the software or use pieces of it in new free programs; and that
you know you can do these things.

To protect your rights, we need to make restrictions that forbid anyone to
deny you these rights or to ask you to surrender the rights.  These
restrictions translate to certain responsibilities for you if you distribute
copies of the software, or if you modify it.

For example, if you distribute copies of such a program, whether gratis or
for a fee, you must give the recipients all the rights that you have.  You
must make sure that they, too, receive or can get the source code.  And you
must show them these terms so they know their rights.

We protect your rights with two steps: (1) copyright the software, and (2)
offer you this license which gives you legal permission to copy, distribute
and/or modify the software.

Also, for each author's protection and ours, we want to make certain that
everyone understands that there is no warranty for this free software.  If
the software is modified by someone else and passed on, we want its
recipients to know that what they have is not the original, so that any
problems introduced by others will not reflect on the original authors'
reputations.

Finally, any free program is threatened constantly by software patents.  We
wish to avoid the danger that redistributors of a free program will
individually obtain patent licenses, in effect making the program
proprietary.  To prevent this, we have made it clear that any patent must be
licensed for everyone's free use or not licensed at all.

The precise terms and conditions for copying, distribution and modification
follow.


\gplsec{Terms and conditions for copying, distribution and modification}

\begin{enumerate}

\makeatletter \setcounter{\@listctr}{-1} \makeatother

\item [0.] This License applies to any program or other work which contains a
      notice placed by the copyright holder saying it may be distributed
      under the terms of this General Public License.  The ``Program'',
      below, refers to any such program or work, and a ``work based on the
      Program'' means either the Program or any derivative work under
      copyright law: that is to say, a work containing the Program or a
      portion of it, either verbatim or with modifications and/or translated
      into another language.  (Hereinafter, translation is included without
      limitation in the term ``modification''.)  Each licensee is addressed
      as ``you''.

      Activities other than copying, distribution and modification are not
      covered by this License; they are outside its scope.  The act of
      running the Program is not restricted, and the output from the Program
      is covered only if its contents constitute a work based on the Program
      (independent of having been made by running the Program).  Whether that
      is true depends on what the Program does.

\item [1.] You may copy and distribute verbatim copies of the Program's
      source code as you receive it, in any medium, provided that you
      conspicuously and appropriately publish on each copy an appropriate
      copyright notice and disclaimer of warranty; keep intact all the
      notices that refer to this License and to the absence of any warranty;
      and give any other recipients of the Program a copy of this License
      along with the Program.

      You may charge a fee for the physical act of transferring a copy, and
      you may at your option offer warranty protection in exchange for a fee.

\item [2.] You may modify your copy or copies of the Program or any portion
      of it, thus forming a work based on the Program, and copy and
      distribute such modifications or work under the terms of Section 1
      above, provided that you also meet all of these conditions:

      \begin{enumerate}

      \item [(a)] You must cause the modified files to carry prominent
            notices stating that you changed the files and the date of any
            change.

      \item [(b)] You must cause any work that you distribute or publish,
            that in whole or in part contains or is derived from the Program
            or any part thereof, to be licensed as a whole at no charge to
            all third parties under the terms of this License.

      \item [(c)] If the modified program normally reads commands
            interactively when run, you must cause it, when started running
            for such interactive use in the most ordinary way, to print or
            display an announcement including an appropriate copyright notice
            and a notice that there is no warranty (or else, saying that you
            provide a warranty) and that users may redistribute the program
            under these conditions, and telling the user how to view a copy
            of this License.  (Exception: if the Program itself is
            interactive but does not normally print such an announcement,
            your work based on the Program is not required to print an
            announcement.)

      \end{enumerate}

      These requirements apply to the modified work as a whole.  If
      identifiable sections of that work are not derived from the Program,
      and can be reasonably considered independent and separate works in
      themselves, then this License, and its terms, do not apply to those
      sections when you distribute them as separate works.  But when you
      distribute the same sections as part of a whole which is a work based
      on the Program, the distribution of the whole must be on the terms of
      this License, whose permissions for other licensees extend to the
      entire whole, and thus to each and every part regardless of who wrote
      it.

      Thus, it is not the intent of this section to claim rights or contest
      your rights to work written entirely by you; rather, the intent is to
      exercise the right to control the distribution of derivative or
      collective works based on the Program.

      In addition, mere aggregation of another work not based on the Program
      with the Program (or with a work based on the Program) on a volume of a
      storage or distribution medium does not bring the other work under the
      scope of this License.

\item [3.] You may copy and distribute the Program (or a work based on it,
      under Section 2) in object code or executable form under the terms of
      Sections 1 and 2 above provided that you also do one of the following:

      \begin{enumerate}

      \item [(a)] Accompany it with the complete corresponding
            machine-readable source code, which must be distributed under the
            terms of Sections 1 and 2 above on a medium customarily used for
            software interchange; or,

      \item [(b)] Accompany it with a written offer, valid for at least three
            years, to give any third party, for a charge no more than your
            cost of physically performing source distribution, a complete
            machine-readable copy of the corresponding source code, to be
            distributed under the terms of Sections 1 and 2 above on a medium
            customarily used for software interchange; or,

      \item [(c)] Accompany it with the information you received as to the
            offer to distribute corresponding source code.  (This alternative
            is allowed only for noncommercial distribution and only if you
            received the program in object code or executable form with such
            an offer, in accord with Subsection b above.)

      \end{enumerate}

      The source code for a work means the preferred form of the work for
      making modifications to it.  For an executable work, complete source
      code means all the source code for all modules it contains, plus any
      associated interface definition files, plus the scripts used to control
      compilation and installation of the executable.  However, as a special
      exception, the source code distributed need not include anything that
      is normally distributed (in either source or binary form) with the
      major components (compiler, kernel, and so on) of the operating system
      on which the executable runs, unless that component itself accompanies
      the executable.

      If distribution of executable or object code is made by offering access
      to copy from a designated place, then offering equivalent access to
      copy the source code from the same place counts as distribution of the
      source code, even though third parties are not compelled to copy the
      source along with the object code.

\item [4.] You may not copy, modify, sublicense, or distribute the Program
      except as expressly provided under this License.  Any attempt otherwise
      to copy, modify, sublicense or distribute the Program is void, and will
      automatically terminate your rights under this License.  However,
      parties who have received copies, or rights, from you under this
      License will not have their licenses terminated so long as such parties
      remain in full compliance.

\item [5.] You are not required to accept this License, since you have not
      signed it.  However, nothing else grants you permission to modify or
      distribute the Program or its derivative works.  These actions are
      prohibited by law if you do not accept this License.  Therefore, by
      modifying or distributing the Program (or any work based on the
      Program), you indicate your acceptance of this License to do so, and
      all its terms and conditions for copying, distributing or modifying the
      Program or works based on it.

\item [6.] Each time you redistribute the Program (or any work based on the
      Program), the recipient automatically receives a license from the
      original licensor to copy, distribute or modify the Program subject to
      these terms and conditions.  You may not impose any further
      restrictions on the recipients' exercise of the rights granted herein.
      You are not responsible for enforcing compliance by third parties to
      this License.

\item [7.] If, as a consequence of a court judgment or allegation of patent
      infringement or for any other reason (not limited to patent issues),
      conditions are imposed on you (whether by court order, agreement or
      otherwise) that contradict the conditions of this License, they do not
      excuse you from the conditions of this License.  If you cannot
      distribute so as to satisfy simultaneously your obligations under this
      License and any other pertinent obligations, then as a consequence you
      may not distribute the Program at all.  For example, if a patent
      license would not permit royalty-free redistribution of the Program by
      all those who receive copies directly or indirectly through you, then
      the only way you could satisfy both it and this License would be to
      refrain entirely from distribution of the Program.

      If any portion of this section is held invalid or unenforceable under
      any particular circumstance, the balance of the section is intended to
      apply and the section as a whole is intended to apply in other
      circumstances.

      It is not the purpose of this section to induce you to infringe any
      patents or other property right claims or to contest validity of any
      such claims; this section has the sole purpose of protecting the
      integrity of the free software distribution system, which is
      implemented by public license practices.  Many people have made
      generous contributions to the wide range of software distributed
      through that system in reliance on consistent application of that
      system; it is up to the author/donor to decide if he or she is willing
      to distribute software through any other system and a licensee cannot
      impose that choice.

      This section is intended to make thoroughly clear what is believed to
      be a consequence of the rest of this License.

\item [8.] If the distribution and/or use of the Program is restricted in
      certain countries either by patents or by copyrighted interfaces, the
      original copyright holder who places the Program under this License may
      add an explicit geographical distribution limitation excluding those
      countries, so that distribution is permitted only in or among countries
      not thus excluded.  In such case, this License incorporates the
      limitation as if written in the body of this License.

\item [9.] The Free Software Foundation may publish revised and/or new
      versions of the General Public License from time to time.  Such new
      versions will be similar in spirit to the present version, but may
      differ in detail to address new problems or concerns.

      Each version is given a distinguishing version number.  If the Program
      specifies a version number of this License which applies to it and
      ``any later version'', you have the option of following the terms and
      conditions either of that version or of any later version published by
      the Free Software Foundation.  If the Program does not specify a
      version number of this License, you may choose any version ever
      published by the Free Software Foundation.

\item [10.] If you wish to incorporate parts of the Program into other free
      programs whose distribution conditions are different, write to the
      author to ask for permission.  For software which is copyrighted by the
      Free Software Foundation, write to the Free Software Foundation; we
      sometimes make exceptions for this.  Our decision will be guided by the
      two goals of preserving the free status of all derivatives of our free
      software and of promoting the sharing and reuse of software generally.

\begin{center}
NO WARRANTY
\end{center}

\bfseries

\item [11.] Because the Program is licensed free of charge, there is no
      warranty for the Program, to the extent permitted by applicable law.
      except when otherwise stated in writing the copyright holders and/or
      other parties provide the program ``as is'' without warranty of any
      kind, either expressed or implied, including, but not limited to, the
      implied warranties of merchantability and fitness for a particular
      purpose.  The entire risk as to the quality and performance of the
      Program is with you.  Should the Program prove defective, you assume
      the cost of all necessary servicing, repair or correction.

\item [12.] In no event unless required by applicable law or agreed to in
      writing will any copyright holder, or any other party who may modify
      and/or redistribute the program as permitted above, be liable to you
      for damages, including any general, special, incidental or
      consequential damages arising out of the use or inability to use the
      program (including but not limited to loss of data or data being
      rendered inaccurate or losses sustained by you or third parties or a
      failure of the Program to operate with any other programs), even if
      such holder or other party has been advised of the possibility of such
      damages.

\end{enumerate}

\begin{center}
\textbf{END OF TERMS AND CONDITIONS}
\end{center}


\gplsec{Appendix: How to Apply These Terms to Your New Programs}

If you develop a new program, and you want it to be of the greatest possible
use to the public, the best way to achieve this is to make it free software
which everyone can redistribute and change under these terms.

To do so, attach the following notices to the program.  It is safest to
attach them to the start of each source file to most effectively convey the
exclusion of warranty; and each file should have at least the ``copyright''
line and a pointer to where the full notice is found.

\begin{verbatim}
<one line to give the program's name and a brief idea of what it does.>
Copyright (C) 19yy  <name of author>

This program is free software; you can redistribute it and/or modify
it under the terms of the GNU General Public License as published by
the Free Software Foundation; either version 2 of the License, or
(at your option) any later version.

This program is distributed in the hope that it will be useful,
but WITHOUT ANY WARRANTY; without even the implied warranty of
MERCHANTABILITY or FITNESS FOR A PARTICULAR PURPOSE.  See the
GNU General Public License for more details.

You should have received a copy of the GNU General Public License
along with this program; if not, write to the Free Software
Foundation, Inc., 675 Mass Ave, Cambridge, MA 02139, USA.
\end{verbatim}

Also add information on how to contact you by electronic and paper mail.

If the program is interactive, make it output a short notice like this when
it starts in an interactive mode:

\begin{verbatim}
Gnomovision version 69, Copyright (C) 19yy name of author
Gnomovision comes with ABSOLUTELY NO WARRANTY; for details type `show w'.
This is free software, and you are welcome to redistribute it
under certain conditions; type `show c' for details.
\end{verbatim}

The hypothetical commands `show w' and `show c' should show the appropriate
parts of the General Public License.  Of course, the commands you use may be
called something other than `show w' and `show c'; they could even be
mouse-clicks or menu items--whatever suits your program.

You should also get your employer (if you work as a programmer) or your
school, if any, to sign a ``copyright disclaimer'' for the program, if
necessary.  Here is a sample; alter the names:

\begin{verbatim}
Yoyodyne, Inc., hereby disclaims all copyright interest in the program
`Gnomovision' (which makes passes at compilers) written by James Hacker.

<signature of Ty Coon>, 1 April 1989
Ty Coon, President of Vice
\end{verbatim}

This General Public License does not permit incorporating your program into
proprietary programs.  If your program is a subroutine library, you may
consider it more useful to permit linking proprietary applications with the
library.  If this is what you want to do, use the GNU Library General Public
License instead of this License.

\gplend
}
%    \end{macrocode}
%
% \end{macro}
%
% \begin{macro}{\implementation}
%
% The |\implementation| macro starts typesetting the implementation of
% the package(s).  If we're not doing the implementation, it just does
% this lot and ends the input file.
%
% I define a macro with arguments inside the |\StopEventually|, which causes
% problems, since the code gets put through an extra level of |\def|fing
% depending on whether the implementation stuff gets typeset or not.  I'll
% store the code I want to do in a separate macro.
%
%    \begin{macrocode}
\def\implementation{\StopEventually{\attheend}}
%    \end{macrocode}
%
% Now for the actual activity.  First, I'll do the |\finalstuff|.  Then, if
% \package{doc}'s managed to find the \package{multicol} package, I'll add
% the end of the environment to the end of each contents file in the list.
% Finally, I'll read the index in from its formatted |.ind| file.
%
%    \begin{macrocode}
\tdef\attheend{%
  \finalstuff%
  \ifhave@multicol%
    \def\do##1##2{\addtocontents{##1}{\protect\end{multicols}}}%
    \docontents%
  \fi%
  \PrintIndex%
}
%    \end{macrocode}
%
% \end{macro}
%
%
% \subsection{File version information}
%
% \begin{macro}{\mdwpkginfo}
%
% For setting up the automatic titles, I'll need to be able to work out
% file versions and things.  This macro will, given a file name, extract
% from \LaTeX\ the version information and format it into a sensible string.
%
% First of all, I'll put the original string (direct from the
% |\Provides|\dots\ command).  Then I'll pass it to another macro which can
% parse up the string into its various bits, along with the original
% filename.
%
%    \begin{macrocode}
\def\mdwpkginfo#1{%
  \edef\@tempa{\csname ver@#1\endcsname}%
  \expandafter\mdwpkginfo@i\@tempa\@@#1\@@%
}
%    \end{macrocode}
%
% Now for the real business.  I'll store the string I build in macros called
% \syntax{"\\"<filename>"date", "\\"<filename>"version" and
% "\\"<filename>"info"}, which store the file's date, version and
% `information string' respectively.  (Note that the file extension isn't
% included in the name.)
%
% This is mainly just tedious playing with |\expandafter|.  The date format
% is defined by a separate macro, which can be modified from the
% configuration file.
%
%    \begin{macrocode}
\def\mdwpkginfo@i#1/#2/#3 #4 #5\@@#6.#7\@@{%
  \expandafter\def\csname #6date\endcsname%
    {\protect\mdwdateformat{#1}{#2}{#3}}%
  \expandafter\def\csname #6version\endcsname{#4}%
  \expandafter\def\csname #6info\endcsname{#5}%
}
%    \end{macrocode}
%
% \end{macro}
%
% \begin{macro}{\mdwdateformat}
%
% Given three arguments, a year, a month and a date (all numeric), build a
% pretty date string.  This is fairly simple really.
%
%    \begin{macrocode}
\tdef\mdwdateformat#1#2#3{\number#3\ \monthname{#2}\ \number#1}
\def\monthname#1{%
  \ifcase#1\or%
     January\or February\or March\or April\or May\or June\or%
     July\or August\or September\or October\or November\or December%
  \fi%
}
\def\numsuffix#1{%
  \ifnum#1=1 st\else%
  \ifnum#1=2 nd\else%
  \ifnum#1=3 rd\else%
  \ifnum#1=21 st\else%
  \ifnum#1=22 nd\else%
  \ifnum#1=23 rd\else%
  \ifnum#1=31 st\else%
  th%
  \fi\fi\fi\fi\fi\fi\fi%
}
%    \end{macrocode}
%
% \end{macro}
%
% \begin{macro}{\mdwfileinfo}
%
% Saying \syntax{"\\mdwfileinfo{"<file-name>"}{"<info>"}"} extracts the
% wanted item of \<info> from the version information for file \<file-name>.
%
%    \begin{macrocode}
\def\mdwfileinfo#1#2{\mdwfileinfo@i{#2}#1.\@@}
\def\mdwfileinfo@i#1#2.#3\@@{\csname#2#1\endcsname}
%    \end{macrocode}
%
% \end{macro}
%
%
% \subsection{List handling}
%
% There are several other lists I need to build.  These macros will do
% the necessary stuff.
%
% \begin{macro}{\mdw@ifitem}
%
% The macro \syntax{"\\mdw@ifitem"<item>"\\in"<list>"{"<true-text>"}"^^A
% "{"<false-text>"}"} does \<true-text> if the \<item> matches any item in
% the \<list>; otherwise it does \<false-text>.
%
%    \begin{macrocode}
\def\mdw@ifitem#1\in#2{%
  \@tempswafalse%
  \def\@tempa{#1}%
  \def\do##1{\def\@tempb{##1}\ifx\@tempa\@tempb\@tempswatrue\fi}%
  #2%
  \if@tempswa\expandafter\@firstoftwo\else\expandafter\@secondoftwo\fi%
}
%    \end{macrocode}
%
% \end{macro}
%
% \begin{macro}{\mdw@append}
%
% Saying \syntax{"\\mdw@append"<item>"\\to"<list>} adds the given \<item>
% to the end of the given \<list>.
%
%    \begin{macrocode}
\def\mdw@append#1\to#2{%
  \toks@{\do{#1}}%
  \toks\tw@\expandafter{#2}%
  \edef#2{\the\toks\tw@\the\toks@}%
}
%    \end{macrocode}
%
% \end{macro}
%
% \begin{macro}{\mdw@prepend}
%
% Saying \syntax{"\\mdw@prepend"<item>"\\to"<list>} adds the \<item> to the
% beginning of the \<list>.
%
%    \begin{macrocode}
\def\mdw@prepend#1\to#2{%
  \toks@{\do{#1}}%
  \toks\tw@\expandafter{#2}%
  \edef#2{\the\toks@\the\toks\tw@}%
}
%    \end{macrocode}
%
% \end{macro}
%
% \begin{macro}{\mdw@add}
%
% Finally, saying \syntax{"\\mdw@add"<item>"\\to"<list>} adds the \<item>
% to the list only if it isn't there already.
%
%    \begin{macrocode}
\def\mdw@add#1\to#2{\mdw@ifitem#1\in#2{}{\mdw@append#1\to#2}}
%    \end{macrocode}
%
% \end{macro}
%
%
% \subsection{Described file handling}
%
% I'l maintain lists of packages, document classes, and other files
% described by the current documentation file.
%
% First of all, I'll declare the various list macros.
%
%    \begin{macrocode}
\def\dopackages{}
\def\doclasses{}
\def\dootherfiles{}
%    \end{macrocode}
%
% \begin{macro}{\describespackage}
%
% A document file can declare that it describes a package by saying
% \syntax{"\\describespackage{"<package-name>"}"}.  I add the package to
% my list, read the package into memory (so that the documentation can
% offer demonstrations of it) and read the version information.
%
%    \begin{macrocode}
\def\describespackage#1{%
  \mdw@ifitem#1\in\dopackages{}{%
    \mdw@append#1\to\dopackages%
    \usepackage{#1}%
    \mdwpkginfo{#1.sty}%
  }%
}
%    \end{macrocode}
%
% \end{macro}
%
% \begin{macro}{\describesclass}
%
% By saying \syntax{"\\describesclass{"<class-name>"}"}, a document file
% can declare that it describes a document class.  I'll assume that the
% document class is already loaded, because it's much too late to load
% it now.
%
%    \begin{macrocode}
\def\describesclass#1{\mdw@add#1\to\doclasses\mdwpkginfo{#1.cls}}
%    \end{macrocode}
%
% \end{macro}
%
% \begin{macro}{\describesfile}
%
% Finally, other `random' files, which don't have the status of real \LaTeX\
% packages or document classes, can be described by saying \syntax{^^A
% "\\describesfile{"<file-name>"}" or "\\describesfile*{"<file-name>"}"}.
% The difference is that the starred version will not |\input| the file.
%
%    \begin{macrocode}
\def\describesfile{%
  \@ifstar{\describesfile@i\@gobble}{\describesfile@i\input}%
}
\def\describesfile@i#1#2{%
  \mdw@ifitem#2\in\dootherfiles{}{%
    \mdw@add#2\to\dootherfiles%
    #1{#2}%
    \mdwpkginfo{#2}%
  }%
}
%    \end{macrocode}
%
% \end{macro}
%
%
% \subsection{Author and title handling}
%
% I'll redefine the |\author| and |\title| commands so that I get told
% whether I need to do it myself.
%
% \begin{macro}{\author}
%
% This is easy: I'll save the old meaning, and then redefine |\author| to
% do the old thing and redefine itself to then do nothing.
%
%    \begin{macrocode}
\let\mdw@author\author
\def\author{\let\author\@gobble\mdw@author}
%    \end{macrocode}
%
% \end{macro}
%
% \begin{macro}{\title}
%
% And oddly enough, I'll do exactly the same thing for the title, except
% that I'll also disable the |\mdw@buildtitle| command, which constructs
% the title automatically.
%
%    \begin{macrocode}
\let\mdw@title\title
\def\title{\let\title\@gobble\let\mdw@buildtitle\relax\mdw@title}
%    \end{macrocode}
%
% \end{macro}
%
% \begin{macro}{\date}
%
% This works in a very similar sort of way.
%
%    \begin{macrocode}
\def\date#1{\let\date\@gobble\def\today{#1}}
%    \end{macrocode}
%
% \end{macro}
%
% \begin{macro}{\datefrom}
%
% Saying \syntax{"\\datefrom{"<file-name>"}"} sets the document date from
% the given filename.
%
%    \begin{macrocode}
\def\datefrom#1{%
  \protected@edef\@tempa{\noexpand\date{\csname #1date\endcsname}}%
  \@tempa%
}
%    \end{macrocode}
%
% \end{macro}
%
% \begin{macro}{\docfile}
%
% Saying \syntax{"\\docfile{"<file-name>"}"} sets up the file name from which
% documentation will be read.
%
%    \begin{macrocode}
\def\docfile#1{%
  \def\@tempa##1.##2\@@{\def\@basefile{##1.##2}\def\@basename{##1}}%
  \edef\@tempb{\noexpand\@tempa#1\noexpand\@@}%
  \@tempb%
}
%    \end{macrocode}
%
% I'll set up a default value as well.
%
%    \begin{macrocode}
\docfile{\jobname.dtx}
%    \end{macrocode}
%
% \end{macro}
%
%
% \subsection{Building title strings}
%
% This is rather tricky.  For each list, I need to build a legible looking
% string.
%
% \begin{macro}{\mdw@addtotitle}
%
% By saying
%\syntax{"\\mdw@addtotitle{"<list>"}{"<command>"}{"<singular>"}{"<plural>"}"}
% I can add the contents of a list to the current title string in the
% |\mdw@title| macro.
%
%    \begin{macrocode}
\tdef\mdw@addtotitle#1#2#3#4{%
%    \end{macrocode}
%
% Now to get to work.  I need to keep one `lookahead' list item, and a count
% of the number of items read so far.  I'll keep the lookahead item in
% |\@nextitem| and the counter in |\count@|.
%
%    \begin{macrocode}
  \count@\z@%
%    \end{macrocode}
%
% Now I'll define what to do for each list item.  The |\protect| command is
% already set up appropriately for playing with |\edef| commands.
%
%    \begin{macrocode}
  \def\do##1{%
%    \end{macrocode}
%
% The first job is to add the previous item to the title string.  If this
% is the first item, though, I'll just add the appropriate \lit{The } or
% \lit{ and the } string to the title (this is stored in the |\@prefix|
% macro).
%
%    \begin{macrocode}
    \edef\mdw@title{%
      \mdw@title%
      \ifcase\count@\@prefix%
      \or\@nextitem%
      \else, \@nextitem%
      \fi%
    }%
%    \end{macrocode}
%
% That was rather easy.  Now I'll set up the |\@nextitem| macro for the
% next time around the loop.
%
%    \begin{macrocode}
    \edef\@nextitem{%
      \protect#2{##1}%
      \protect\footnote{%
        The \protect#2{##1} #3 is currently at version %
        \mdwfileinfo{##1}{version}, dated \mdwfileinfo{##1}{date}.%
      }\space%
    }%
%    \end{macrocode}
%
% Finally, I need to increment the counter.
%
%    \begin{macrocode}
    \advance\count@\@ne%
  }%
%    \end{macrocode}
%
% Now execute the list.
%
%    \begin{macrocode}
  #1%
%    \end{macrocode}
%
% I still have one item left over, unless the list was empty.  I'll add
% that now.
%
%    \begin{macrocode}
  \edef\mdw@title{%
    \mdw@title%
    \ifcase\count@%
    \or\@nextitem\space#3%
    \or\ and \@nextitem\space#4%
    \fi%
  }%
%    \end{macrocode}
%
% Finally, if $|\count@| \ne 0$, I must set |\@prefix| to \lit{ and the }.
%
%    \begin{macrocode}
  \ifnum\count@>\z@\def\@prefix{ and the }\fi%
}
%    \end{macrocode}
%
% \end{macro}
%
% \begin{macro}{\mdw@buildtitle}
%
% This macro will actually do the job of building the title string.
%
%    \begin{macrocode}
\tdef\mdw@buildtitle{%
%    \end{macrocode}
%
% First of all, I'll open a group to avoid polluting the namespace with
% my gubbins (although the code is now much tidier than it has been in
% earlier releases).
%
%    \begin{macrocode}
  \begingroup%
%    \end{macrocode}
%
% The title building stuff makes extensive use of |\edef|.  I'll set
% |\protect| appropriately.  (For those not in the know,
% |\@unexpandable@protect| expands to `|\noexpand\protect\noexpand|',
% which prevents expansion of the following macro, and inserts a |\protect|
% in front of it ready for the next |\edef|.)
%
%    \begin{macrocode}
  \let\@@protect\protect\let\protect\@unexpandable@protect%
%    \end{macrocode}
%
% Set up some simple macros ready for the main code.
%
%    \begin{macrocode}
  \def\mdw@title{}%
  \def\@prefix{The }%
%    \end{macrocode}
%
% Now build the title.  This is fun.
%
%    \begin{macrocode}
  \mdw@addtotitle\dopackages\package{package}{packages}%
  \mdw@addtotitle\doclasses\package{document class}{document classes}%
  \mdw@addtotitle\dootherfiles\texttt{file}{files}%
%    \end{macrocode}
%
% Now I want to end the group and set the title from my string.  The
% following hacking will do this.
%
%    \begin{macrocode}
  \edef\next{\endgroup\noexpand\title{\mdw@title}}%
  \next%
}
%    \end{macrocode}
%
% \end{macro}
%
%
% \subsection{Starting the main document}
%
% \begin{macro}{\mdwdoc}
%
% Once the document preamble has done all of its stuff, it calls the
% |\mdwdoc| command, which takes over and really starts the documentation
% going.
%
%    \begin{macrocode}
\def\mdwdoc{%
%    \end{macrocode}
%
% First, I'll construct the title string.
%
%    \begin{macrocode}
  \mdw@buildtitle%
  \author{Mark Wooding}%
%    \end{macrocode}
%
% Set up the date string based on the date of the package which shares
% the same name as the current file.
%
%    \begin{macrocode}
  \datefrom\@basename%
%    \end{macrocode}
%
% Set up verbatim characters after all the packages have started.
%
%    \begin{macrocode}
  \shortverb\|%
  \shortverb\"%
%    \end{macrocode}
%
% Start the document, and put the title in.
%
%    \begin{macrocode}
  \begin{document}
  \maketitle%
%    \end{macrocode}
%
% This is nasty.  It makes maths displays work properly in demo environments.
% \emph{The \LaTeX\ Companion} exhibits the bug which this hack fixes.  So
% ner.
%
%    \begin{macrocode}
  \abovedisplayskip\z@%
%    \end{macrocode}
%
% Now start the contents tables.  After starting each one, I'll make it
% be multicolumnar.
%
%    \begin{macrocode}
  \def\do##1##2{%
    ##2%
    \ifhave@multicol\addtocontents{##1}{%
      \protect\begin{multicols}{2}%
      \hbadness\@M%
    }\fi%
  }%
  \docontents%
%    \end{macrocode}
%
% Input the main file now.
%
%    \begin{macrocode}
  \DocInput{\@basefile}%
%    \end{macrocode}
%
% That's it.  I'm done.
%
%    \begin{macrocode}
  \end{document}
}
%    \end{macrocode}
%
% \end{macro}
%
%
% \subsection{And finally\dots}
%
% Right at the end I'll put a hook for the configuration file.
%
%    \begin{macrocode}
\ifx\mdwhook\@@undefined\else\expandafter\mdwhook\fi
%    \end{macrocode}
%
% That's all the code done now.  I'll change back to `user' mode, where
% all the magic control sequences aren't allowed any more.
%
%    \begin{macrocode}
\makeatother
%</mdwtools>
%    \end{macrocode}
%
% Oh, wait!  What if I want to typeset this documentation?  Aha.  I'll cope
% with that by comparing |\jobname| with my filename |mdwtools|.  However,
% there's some fun here, because |\jobname| contains category-12 letters,
% while my letters are category-11.  Time to play with |\string| in a messy
% way.
%
%    \begin{macrocode}
%<*driver>
\makeatletter
\edef\@tempa{\expandafter\@gobble\string\mdwtools}
\edef\@tempb{\jobname}
\ifx\@tempa\@tempb
  \describesfile*{mdwtools.tex}
  \docfile{mdwtools.tex}
  \makeatother
  \expandafter\mdwdoc
\fi
\makeatother
%</driver>
%    \end{macrocode}
%
% That's it.  Done!
%
% \hfill Mark Wooding, \today
%
% \Finale
%
\endinput
|
% \<declarations>
% |\mdwdoc|
% \end{listinglist}
% The initial |\input| reads in this file and sets up the various commands
% which may be needed.  The final |\mdwdoc| actually starts the document,
% inserting a title (which is automatically generated), a table of
% contents etc., and reads the documentation file in (using the |\DocInput|
% command from the \package{doc} package.
%
% \subsubsection{Describing packages}
%
% \DescribeMacro{\describespackage}
% \DescribeMacro{\describesclass}
% \DescribeMacro{\describesfile}
% \DescribeMacro{\describesfile*}
% The most important declarations are those which declare what the
% documentation describes.  Saying \syntax{"\\describespackage{<package>}"}
% loads the \<package> (if necessary) and adds it to the auto-generated
% title, along with a footnote containing version information.  Similarly,
% |\describesclass| adds a document class name to the title (without loading
% it -- the document itself must do this, with the |\documentclass| command).
% For files which aren't packages or classes, use the |\describesfile| or
% |\describesfile*| command (the $*$-version won't |\input| the file, which
% is handy for files like |mdwtools.tex|, which are already input).
%
% \DescribeMacro{\author}
% \DescribeMacro{\date}
% \DescribeMacro{\title}
% The |\author|, |\date| and |\title| declarations work slightly differently
% to normal -- they ensure that only the \emph{first} declaration has an
% effect.  (Don't you play with |\author|, please, unless you're using this
% program to document your own packages.)  Using |\title| suppresses the
% automatic title generation.
%
% \DescribeMacro{\docdate}
% The default date is worked out from the version string of the package or
% document class whose name is the same as that of the documentation file.
% You can choose a different `main' file by saying
% \syntax{"\\docdate{"<file>"}"}.
%
% \subsubsection{Contents handling}
%
% \DescribeMacro{\addcontents}
% A documentation file always has a table of contents.  Other
% contents-like lists can be added by saying
% \syntax{"\\addcontents{"<extension>"}{"<command>"}"}.  The \<extension>
% is the file extension of the contents file (e.g., \lit{lot} for the
% list of tables); the \<command> is the command to actually typeset the
% contents file (e.g., |\listoftables|).
%
% \subsubsection{Other declarations}
%
% \DescribeMacro{\implementation}
% The \package{doc} package wants you to say
% \syntax{"\\StopEventually{"<stuff>"}"}' before describing the package
% implementation.  Using |mdwtools.tex|, you just say |\implementation|, and
% everything works.  It will automatically read in the licence text (from
% |gpl.tex|, and wraps some other things up.
%
% 
% \subsection{Other commands}
%
% The |mdwtools.tex| file includes the \package{syntax} and \package{sverb}
% packages so that they can be used in documentation files.  It also defines
% some trivial commands of its own.
%
% \DescribeMacro{\<}
% Saying \syntax{"\\<"<text>">" is the same as "\\synt{"<text>"}"}; this
% is a simple abbreviation.
%
% \DescribeMacro{\smallf}
% Saying \syntax{"\\smallf" <number>"/"<number>} typesets a little fraction,
% like this: \smallf 3/4.  It's useful when you want to say that the default
% value of a length is 2 \smallf 1/2\,pt, or something like that.
%
%
% \subsection{Customisation}
%
% You can customise the way that the package documentation looks by writing
% a file called |mdwtools.cfg|.  You can redefine various commands (before
% they're defined here, even; |mdwtools.tex| checks most of the commands that
% it defines to make sure they haven't been defined already.
%
% \DescribeMacro{\indexing}
% If you don't want the prompt about whether to generate index files, you
% can define the |\indexing| command to either \lit{y} or \lit{n}.  I'd
% recommend that you use |\providecommand| for this, to allow further
% customisation from the command line.
%
% \DescribeMacro{\mdwdateformat}
% If you don't like my date format (maybe you're American or something),
% you can redefine the |\mdwdateformat| command.  It takes three arguments:
% the year, month and date, as numbers; it should expand to something which
% typesets the date nicely.  The default format gives something like
% `10 May 1996'.  You can produce something rather more exotic, like
% `10\textsuperscript{th} May \textsc{\romannumeral 1996}' by saying
%\begin{listing}
%\newcommand{\mdwdateformat}[3]{%
%  \number#3\textsuperscript{\numsuffix{#3}}\ %
%  \monthname{#2}\ %
%  \textsc{\romannumeral #1}%
%}
%\end{listing}
% \DescribeMacro{\monthname}
% \DescribeMacro{\numsuffix}
% Saying \syntax{"\\monthname{"<number>"}"} expands to the name of the
% numbered month (which can be useful when doing date formats).  Saying
% \syntax{"\\numsuffix{"<number>"}"} will expand to the appropriate suffix
% (`th' or `rd' or whatever) for the \<number>.  You'll have to superscript
% it yourself, if this is what you want to do.  Putting the year number
% in roman numerals is just pretentious |;-)|.
%
% \DescribeMacro{\mdwhook}
% After all the declarations in |mdwtools.tex|, the command |\mdwhook| is
% executed, if it exists.  This can be set up by the configuration file
% to do whatever you want.
%
% There are lots of other things you can play with; you should look at the
% implementation section to see what's possible.
%
% \implementation
%
% \section{Implementation}
%
%    \begin{macrocode}
%<*mdwtools>
%    \end{macrocode}
%
% The first thing is that I'm not a \LaTeX\ package or anything official
% like that, so I must enable `|@|' as a letter by hand.
%
%    \begin{macrocode}
\makeatletter
%    \end{macrocode}
%
% Now input the user's configuration file, if it exists.  This is fairly
% simple stuff.
%
%    \begin{macrocode}
\@input{mdwtools.cfg}
%    \end{macrocode}
%
% Well, that's the easy bit done.
%
%
% \subsection{Initialisation}
%
% Obviously the first thing to do is to obtain a document class.  Obviously,
% it would be silly to do this if a document class has already been loaded,
% either by the package documentation or by the configuration file.
%
% The only way I can think of for finding out if a document class is already
% loaded is by seeing if the |\documentclass| command has been redefined
% to raise an error.  This isn't too hard, really.
%
%    \begin{macrocode}
\ifx\documentclass\@twoclasseserror\else
  \documentclass[a4paper]{ltxdoc}
  \ifx\doneclasses\mdw@undefined\else\doneclasses\fi
\fi
%    \end{macrocode}
%
% As part of my standard environment, I'll load some of my more useful
% packages.  If they're already loaded (possibly with different options),
% I'll not try to load them again.
%
%    \begin{macrocode}
\@ifpackageloaded{doc}{}{\usepackage{doc}}
\@ifpackageloaded{syntax}{}{\usepackage[rounded]{syntax}}
\@ifpackageloaded{sverb}{}{\usepackage{sverb}}
%    \end{macrocode}
%
%
% \subsection{Some macros for interaction}
%
% I like the \LaTeX\ star-boxes, although it's a pain having to cope with
% \TeX's space-handling rules.  I'll define a new typing-out macro which
% makes spaces more significant, and has a $*$-version which doesn't put
% a newline on the end, and interacts prettily with |\read|.
%
% First of all, I need to make spaces active, so I can define things about
% active spaces.
%
%    \begin{macrocode}
\begingroup\obeyspaces
%    \end{macrocode}
%
% Now to define the main macro.  This is easy stuff.  Spaces must be
% carefully rationed here, though.
%
% I'll start a group, make spaces active, and make spaces expand to ordinary
% space-like spaces.  Then I'll look for a star, and pass either |\message|
% (which doesn't start a newline, and interacts with |\read| well) or
% |\immediate\write 16| which does a normal write well.
%
%    \begin{macrocode}
\gdef\mdwtype{%
\begingroup\catcode`\ \active\let \space%
\@ifstar{\mdwtype@i{\message}}{\mdwtype@i{\immediate\write\sixt@@n}}%
}
\endgroup
%    \end{macrocode}
%
% Now for the easy bit.  I have the thing to do, and the thing to do it to,
% so do that and end the group.
%
%    \begin{macrocode}
\def\mdwtype@i#1#2{#1{#2}\endgroup}
%    \end{macrocode}
%
%
% \subsection{Decide on indexing}
%
% A configuration file can decide on indexing by defining the |\indexing|
% macro to either \lit{y} or \lit{n}.  If it's not set, then I'll prompt
% the user.
%
% First of all, I want a switch to say whether I'm indexing.
%
%    \begin{macrocode}
\newif\ifcreateindex
%    \end{macrocode}
%
% Right: now I need to decide how to make progress.  If the macro's not set,
% then I want to set it, and start a row of stars.
%
%    \begin{macrocode}
\ifx\indexing\@@undefined
  \mdwtype{*****************************}
  \def\indexing{?}
\fi
%    \end{macrocode}
%
% Now enter a loop, asking the user whether to do indexing, until I get
% a sensible answer.
%
%    \begin{macrocode}
\loop
  \@tempswafalse
  \if y\indexing\@tempswatrue\createindextrue\fi
  \if Y\indexing\@tempswatrue\createindextrue\fi
  \if n\indexing\@tempswatrue\createindexfalse\fi
  \if N\indexing\@tempswatrue\createindexfalse\fi
  \if@tempswa\else
  \mdwtype*{* Create index files? (y/n) *}
  \read\sixt@@n to\indexing%
\repeat
%    \end{macrocode}
%
% Now, based on the results of that, display a message about the indexing.
%
%    \begin{macrocode}
\mdwtype{*****************************}
\ifcreateindex
  \mdwtype{* Creating index files      *}
  \mdwtype{* This may take some time   *}
\else
  \mdwtype{* Not creating index files  *}
\fi
\mdwtype{*****************************}
%    \end{macrocode}
%
% Now I can play with the indexing commands of the \package{doc} package
% to do whatever it is that the user wants.
%
%    \begin{macrocode}
\ifcreateindex
  \CodelineIndex
  \EnableCrossrefs
\else
  \CodelineNumbered
  \DisableCrossrefs
\fi
%    \end{macrocode}
%
% And register lots of plain \TeX\ things which shouldn't be indexed.
% This contains lots of |\if|\dots\ things which don't fit nicely in
% conditionals, which is a shame.  Still, it doesn't matter that much,
% really.
%
%    \begin{macrocode}
\DoNotIndex{\def,\long,\edef,\xdef,\gdef,\let,\global}
\DoNotIndex{\if,\ifnum,\ifdim,\ifcat,\ifmmode,\ifvmode,\ifhmode,%
            \iftrue,\iffalse,\ifvoid,\ifx,\ifeof,\ifcase,\else,\or,\fi}
\DoNotIndex{\box,\copy,\setbox,\unvbox,\unhbox,\hbox,%
            \vbox,\vtop,\vcenter}
\DoNotIndex{\@empty,\immediate,\write}
\DoNotIndex{\egroup,\bgroup,\expandafter,\begingroup,\endgroup}
\DoNotIndex{\divide,\advance,\multiply,\count,\dimen}
\DoNotIndex{\relax,\space,\string}
\DoNotIndex{\csname,\endcsname,\@spaces,\openin,\openout,%
            \closein,\closeout}
\DoNotIndex{\catcode,\endinput}
\DoNotIndex{\jobname,\message,\read,\the,\m@ne,\noexpand}
\DoNotIndex{\hsize,\vsize,\hskip,\vskip,\kern,\hfil,\hfill,\hss}
\DoNotIndex{\m@ne,\z@,\z@skip,\@ne,\tw@,\p@}
\DoNotIndex{\dp,\wd,\ht,\vss,\unskip}
%    \end{macrocode}
%
% Last bit of indexing stuff, for now: I'll typeset the index in two columns
% (the default is three, which makes them too narrow for my tastes).
%
%    \begin{macrocode}
\setcounter{IndexColumns}{2}
%    \end{macrocode}
%
%
% \subsection{Selectively defining things}
%
% I don't want to tread on anyone's toes if they redefine any of these
% commands and things in a configuration file.  The following definitions
% are fairly evil, but should do the job OK.
%
% \begin{macro}{\@gobbledef}
%
% This macro eats the following |\def|inition, leaving not a trace behind.
%
%    \begin{macrocode}
\def\@gobbledef#1#{\@gobble}
%    \end{macrocode}
%
% \end{macro}
%
% \begin{macro}{\tdef}
% \begin{macro}{\tlet}
%
% The |\tdef| command is a sort of `tentative' definition -- it's like
% |\def| if the control sequence named doesn't already have a definition.
% |\tlet| does the same thing with |\let|.
%
%    \begin{macrocode}
\def\tdef#1{
  \ifx#1\@@undefined%
    \expandafter\def\expandafter#1%
  \else%
    \expandafter\@gobbledef%
  \fi%
}
\def\tlet#1#2{\ifx#1\@@undefined\let#1=#2\fi}
%    \end{macrocode}
%
% \end{macro}
% \end{macro}
%
%
% \subsection{General markup things}
%
% Now for some really simple things.  I'll define how to typeset package
% names and environment names (both in the sans serif font, for now).
%
%    \begin{macrocode}
\tlet\package\textsf
\tlet\env\textsf
%    \end{macrocode}
%
% I'll define the |\<|\dots|>| shortcut for syntax items suggested in the
% \package{syntax} package.
%
%    \begin{macrocode}
\tdef\<#1>{\synt{#1}}
%    \end{macrocode}
%
% And because it's used in a few places (mainly for typesetting lengths),
% here's a command for typesetting fractions in text.
%
%    \begin{macrocode}
\tdef\smallf#1/#2{\ensuremath{^{#1}\!/\!_{#2}}}
%    \end{macrocode}
%
%
% \subsection{A table environment}
%
% \begin{environment}{tab}
%
% Most of the packages don't use the (obviously perfect) \package{mdwtab}
% package, because it's big, and takes a while to load.  Here's an
% environment for typesetting centred tables.  The first (optional) argument
% is some declarations to perform.  The mandatory argument is the table
% preamble (obviously).
%
%    \begin{macrocode}
\@ifundefined{tab}{%
  \newenvironment{tab}[2][\relax]{%
    \par\vskip2ex%
    \centering%
    #1%
    \begin{tabular}{#2}%
  }{%
    \end{tabular}%
    \par\vskip2ex%
  }
}{}
%    \end{macrocode}
%
% \end{environment}
%
%
% \subsection{Commenting out of stuff}
%
% \begin{environment}{meta-comment}
%
% Using |\iffalse|\dots|\fi| isn't much fun.  I'll define a gobbling
% environment using the \package{sverb} stuff.
%
%    \begin{macrocode}
\ignoreenv{meta-comment}
%    \end{macrocode}
%
% \end{environment}
%
%
% \subsection{Float handling}
%
% This gubbins will try to avoid float pages as much as possible, and (with
% any luck) encourage floats to be put on the same pages as text.
%
%    \begin{macrocode}
\def\textfraction{0.1}
\def\topfraction{0.9}
\def\bottomfraction{0.9}
\def\floatpagefraction{0.7}
%    \end{macrocode}
%
% Now redefine the default float-placement parameters to allow `here' floats.
%
%    \begin{macrocode}
\def\fps@figure{htbp}
\def\fps@table{htbp}
%    \end{macrocode}
%
%
% \subsection{Other bits of parameter tweaking}
%
% Make \env{grammar} environments look pretty, by indenting the left hand
% sides by a large amount.
%
%    \begin{macrocode}
\grammarindent1in
%    \end{macrocode}
%
% I don't like being told by \TeX\ that my paragraphs are hard to linebreak:
% I know this already.  This lot should shut \TeX\ up about most problems.
%
%    \begin{macrocode}
\sloppy
\hbadness\@M
\hfuzz10\p@
%    \end{macrocode}
%
% Also make \TeX\ shut up in the index.  The \package{multicol} package
% irritatingly plays with |\hbadness|.  This is the best hook I could find
% for playing with this setting.
%
%    \begin{macrocode}
\expandafter\def\expandafter\IndexParms\expandafter{%
  \IndexParms%
  \hbadness\@M%
}
%    \end{macrocode}
%
% The other thing I really don't like is `Marginpar moved' warnings.  This
% will get rid of them, and lots of other \LaTeX\ warnings at the same time.
%
%    \begin{macrocode}
\let\@latex@warning@no@line\@gobble
%    \end{macrocode}
%
% Put some extra space between table rows, please.
%
%    \begin{macrocode}
\def\arraystretch{1.2}
%    \end{macrocode}
%
% Most of the code is at guard level one, so typeset that in upright text.
%
%    \begin{macrocode}
\setcounter{StandardModuleDepth}{1}
%    \end{macrocode}
%
%
% \subsection{Contents handling}
%
% I use at least one contents file (the main table of contents) although
% I may want more.  I'll keep a list of contents files which I need to
% handle.
%
% There are two things I need to do to contents files here:
% \begin{itemize}
% \item I must typeset the table of contents at the beginning of the
%       document; and
% \item I want to typeset tables of contents in two columns (using the
%       \package{multicol} package).
% \end{itemize}
%
% The list consists of items of the form
% \syntax{"\\do{"<extension>"}{"<command>"}"}, where \<extension> is the
% file extension of the contents file, and \<command> is the command to
% typeset it.
%
% \begin{macro}{\docontents}
%
% This is where I keep the list of contents files.  I'll initialise it to
% just do the standard contents table.
%
%    \begin{macrocode}
\def\docontents{\do{toc}{\tableofcontents}}
%    \end{macrocode}
%
% \end{macro}
%
% \begin{macro}{\addcontents}
%
% By saying \syntax{"\\addcontents{"<extension>"}{"<command>"}"}, a document
% can register a new table of contents which gets given the two-column
% treatment properly.  This is really easy to implement.
%
%    \begin{macrocode}
\def\addcontents#1#2{%
  \toks@\expandafter{\docontents\do{#1}{#2}}%
  \edef\docontents{\the\toks@}%
}
%    \end{macrocode}
%
% \end{macro}
%
%
% \subsection{Finishing it all off}
%
% \begin{macro}{\finalstuff}
%
% The |\finalstuff| macro is a hook for doing things at the end of the
% document.  Currently, it inputs the licence agreement as an appendix.
%
%    \begin{macrocode}
\tdef\finalstuff{\appendix\part*{Appendix}% \iffalse <meta-comment>
%
% $Id$
%
% The GNU General Public Licence as a LaTeX section
%
% (c) 1989, 1991 Free Software Foundation, Inc.
%   LaTeX markup and minor formatting changes by Mark Wooding
%

%----- Revision history -----------------------------------------------------
%
% $Log$
% Revision 1.1  1998-09-21 10:19:01  michael
% Initial implementation
%
% Revision 1.1  1996/11/19 20:51:14  mdw
% Initial revision
%

% --- Chapter heading ---
%
% We don't know whether this ought to be a section or a chapter.  Easy.
% We'll see if chapters are possible.
%
% \fi

\begingroup
\makeatletter

\edef\next#1#2#3{\relax
  \ifx\chapter\@@undefined
    \ifx\documentclass\@notprerr#2\else#3\fi
  \else#1\fi
}

\expandafter\endgroup\next
{
  \let\gpltoplevel\chapter
  \let\gplsec\section
  \let\gplend\endinput
}{
  \let\gpltoplevel\section
  \let\gplsec\subsection
  \let\gplend\endinput
}{
  \documentclass[a4paper]{article}
  \def\gpltoplevel#1{%
    \vspace*{1in}%
    \hbox to\hsize{\hfil\LARGE\bfseries#1\hfil}%
    \vspace{1in}%
  }
  \let\gplsec\section
  \def\gplend{\end{document}}
  \advance\textwidth1in
  \advance\oddsidemargin-.5in
  \sloppy
  \begin{document}
}

%^^A-------------------------------------------------------------------------
\gpltoplevel{The GNU General Public Licence}


The following is the text of the GNU General Public Licence, under the terms
of which this software is distrubuted.

\vspace{12pt}

\begin{center}
\textbf{GNU GENERAL PUBLIC LICENSE} \\
Version 2, June 1991
\end{center}

\begin{center}
Copyright (C) 1989, 1991 Free Software Foundation, Inc. \\
675 Mass Ave, Cambridge, MA 02139, USA

Everyone is permitted to copy and distribute verbatim copies \\
of this license document, but changing it is not allowed.
\end{center}


\gplsec{Preamble}

The licenses for most software are designed to take away your freedom to
share and change it.  By contrast, the GNU General Public License is intended
to guarantee your freedom to share and change free software---to make sure
the software is free for all its users.  This General Public License applies
to most of the Free Software Foundation's software and to any other program
whose authors commit to using it.  (Some other Free Software Foundation
software is covered by the GNU Library General Public License instead.)  You
can apply it to your programs, too.

When we speak of free software, we are referring to freedom, not price.  Our
General Public Licenses are designed to make sure that you have the freedom
to distribute copies of free software (and charge for this service if you
wish), that you receive source code or can get it if you want it, that you
can change the software or use pieces of it in new free programs; and that
you know you can do these things.

To protect your rights, we need to make restrictions that forbid anyone to
deny you these rights or to ask you to surrender the rights.  These
restrictions translate to certain responsibilities for you if you distribute
copies of the software, or if you modify it.

For example, if you distribute copies of such a program, whether gratis or
for a fee, you must give the recipients all the rights that you have.  You
must make sure that they, too, receive or can get the source code.  And you
must show them these terms so they know their rights.

We protect your rights with two steps: (1) copyright the software, and (2)
offer you this license which gives you legal permission to copy, distribute
and/or modify the software.

Also, for each author's protection and ours, we want to make certain that
everyone understands that there is no warranty for this free software.  If
the software is modified by someone else and passed on, we want its
recipients to know that what they have is not the original, so that any
problems introduced by others will not reflect on the original authors'
reputations.

Finally, any free program is threatened constantly by software patents.  We
wish to avoid the danger that redistributors of a free program will
individually obtain patent licenses, in effect making the program
proprietary.  To prevent this, we have made it clear that any patent must be
licensed for everyone's free use or not licensed at all.

The precise terms and conditions for copying, distribution and modification
follow.


\gplsec{Terms and conditions for copying, distribution and modification}

\begin{enumerate}

\makeatletter \setcounter{\@listctr}{-1} \makeatother

\item [0.] This License applies to any program or other work which contains a
      notice placed by the copyright holder saying it may be distributed
      under the terms of this General Public License.  The ``Program'',
      below, refers to any such program or work, and a ``work based on the
      Program'' means either the Program or any derivative work under
      copyright law: that is to say, a work containing the Program or a
      portion of it, either verbatim or with modifications and/or translated
      into another language.  (Hereinafter, translation is included without
      limitation in the term ``modification''.)  Each licensee is addressed
      as ``you''.

      Activities other than copying, distribution and modification are not
      covered by this License; they are outside its scope.  The act of
      running the Program is not restricted, and the output from the Program
      is covered only if its contents constitute a work based on the Program
      (independent of having been made by running the Program).  Whether that
      is true depends on what the Program does.

\item [1.] You may copy and distribute verbatim copies of the Program's
      source code as you receive it, in any medium, provided that you
      conspicuously and appropriately publish on each copy an appropriate
      copyright notice and disclaimer of warranty; keep intact all the
      notices that refer to this License and to the absence of any warranty;
      and give any other recipients of the Program a copy of this License
      along with the Program.

      You may charge a fee for the physical act of transferring a copy, and
      you may at your option offer warranty protection in exchange for a fee.

\item [2.] You may modify your copy or copies of the Program or any portion
      of it, thus forming a work based on the Program, and copy and
      distribute such modifications or work under the terms of Section 1
      above, provided that you also meet all of these conditions:

      \begin{enumerate}

      \item [(a)] You must cause the modified files to carry prominent
            notices stating that you changed the files and the date of any
            change.

      \item [(b)] You must cause any work that you distribute or publish,
            that in whole or in part contains or is derived from the Program
            or any part thereof, to be licensed as a whole at no charge to
            all third parties under the terms of this License.

      \item [(c)] If the modified program normally reads commands
            interactively when run, you must cause it, when started running
            for such interactive use in the most ordinary way, to print or
            display an announcement including an appropriate copyright notice
            and a notice that there is no warranty (or else, saying that you
            provide a warranty) and that users may redistribute the program
            under these conditions, and telling the user how to view a copy
            of this License.  (Exception: if the Program itself is
            interactive but does not normally print such an announcement,
            your work based on the Program is not required to print an
            announcement.)

      \end{enumerate}

      These requirements apply to the modified work as a whole.  If
      identifiable sections of that work are not derived from the Program,
      and can be reasonably considered independent and separate works in
      themselves, then this License, and its terms, do not apply to those
      sections when you distribute them as separate works.  But when you
      distribute the same sections as part of a whole which is a work based
      on the Program, the distribution of the whole must be on the terms of
      this License, whose permissions for other licensees extend to the
      entire whole, and thus to each and every part regardless of who wrote
      it.

      Thus, it is not the intent of this section to claim rights or contest
      your rights to work written entirely by you; rather, the intent is to
      exercise the right to control the distribution of derivative or
      collective works based on the Program.

      In addition, mere aggregation of another work not based on the Program
      with the Program (or with a work based on the Program) on a volume of a
      storage or distribution medium does not bring the other work under the
      scope of this License.

\item [3.] You may copy and distribute the Program (or a work based on it,
      under Section 2) in object code or executable form under the terms of
      Sections 1 and 2 above provided that you also do one of the following:

      \begin{enumerate}

      \item [(a)] Accompany it with the complete corresponding
            machine-readable source code, which must be distributed under the
            terms of Sections 1 and 2 above on a medium customarily used for
            software interchange; or,

      \item [(b)] Accompany it with a written offer, valid for at least three
            years, to give any third party, for a charge no more than your
            cost of physically performing source distribution, a complete
            machine-readable copy of the corresponding source code, to be
            distributed under the terms of Sections 1 and 2 above on a medium
            customarily used for software interchange; or,

      \item [(c)] Accompany it with the information you received as to the
            offer to distribute corresponding source code.  (This alternative
            is allowed only for noncommercial distribution and only if you
            received the program in object code or executable form with such
            an offer, in accord with Subsection b above.)

      \end{enumerate}

      The source code for a work means the preferred form of the work for
      making modifications to it.  For an executable work, complete source
      code means all the source code for all modules it contains, plus any
      associated interface definition files, plus the scripts used to control
      compilation and installation of the executable.  However, as a special
      exception, the source code distributed need not include anything that
      is normally distributed (in either source or binary form) with the
      major components (compiler, kernel, and so on) of the operating system
      on which the executable runs, unless that component itself accompanies
      the executable.

      If distribution of executable or object code is made by offering access
      to copy from a designated place, then offering equivalent access to
      copy the source code from the same place counts as distribution of the
      source code, even though third parties are not compelled to copy the
      source along with the object code.

\item [4.] You may not copy, modify, sublicense, or distribute the Program
      except as expressly provided under this License.  Any attempt otherwise
      to copy, modify, sublicense or distribute the Program is void, and will
      automatically terminate your rights under this License.  However,
      parties who have received copies, or rights, from you under this
      License will not have their licenses terminated so long as such parties
      remain in full compliance.

\item [5.] You are not required to accept this License, since you have not
      signed it.  However, nothing else grants you permission to modify or
      distribute the Program or its derivative works.  These actions are
      prohibited by law if you do not accept this License.  Therefore, by
      modifying or distributing the Program (or any work based on the
      Program), you indicate your acceptance of this License to do so, and
      all its terms and conditions for copying, distributing or modifying the
      Program or works based on it.

\item [6.] Each time you redistribute the Program (or any work based on the
      Program), the recipient automatically receives a license from the
      original licensor to copy, distribute or modify the Program subject to
      these terms and conditions.  You may not impose any further
      restrictions on the recipients' exercise of the rights granted herein.
      You are not responsible for enforcing compliance by third parties to
      this License.

\item [7.] If, as a consequence of a court judgment or allegation of patent
      infringement or for any other reason (not limited to patent issues),
      conditions are imposed on you (whether by court order, agreement or
      otherwise) that contradict the conditions of this License, they do not
      excuse you from the conditions of this License.  If you cannot
      distribute so as to satisfy simultaneously your obligations under this
      License and any other pertinent obligations, then as a consequence you
      may not distribute the Program at all.  For example, if a patent
      license would not permit royalty-free redistribution of the Program by
      all those who receive copies directly or indirectly through you, then
      the only way you could satisfy both it and this License would be to
      refrain entirely from distribution of the Program.

      If any portion of this section is held invalid or unenforceable under
      any particular circumstance, the balance of the section is intended to
      apply and the section as a whole is intended to apply in other
      circumstances.

      It is not the purpose of this section to induce you to infringe any
      patents or other property right claims or to contest validity of any
      such claims; this section has the sole purpose of protecting the
      integrity of the free software distribution system, which is
      implemented by public license practices.  Many people have made
      generous contributions to the wide range of software distributed
      through that system in reliance on consistent application of that
      system; it is up to the author/donor to decide if he or she is willing
      to distribute software through any other system and a licensee cannot
      impose that choice.

      This section is intended to make thoroughly clear what is believed to
      be a consequence of the rest of this License.

\item [8.] If the distribution and/or use of the Program is restricted in
      certain countries either by patents or by copyrighted interfaces, the
      original copyright holder who places the Program under this License may
      add an explicit geographical distribution limitation excluding those
      countries, so that distribution is permitted only in or among countries
      not thus excluded.  In such case, this License incorporates the
      limitation as if written in the body of this License.

\item [9.] The Free Software Foundation may publish revised and/or new
      versions of the General Public License from time to time.  Such new
      versions will be similar in spirit to the present version, but may
      differ in detail to address new problems or concerns.

      Each version is given a distinguishing version number.  If the Program
      specifies a version number of this License which applies to it and
      ``any later version'', you have the option of following the terms and
      conditions either of that version or of any later version published by
      the Free Software Foundation.  If the Program does not specify a
      version number of this License, you may choose any version ever
      published by the Free Software Foundation.

\item [10.] If you wish to incorporate parts of the Program into other free
      programs whose distribution conditions are different, write to the
      author to ask for permission.  For software which is copyrighted by the
      Free Software Foundation, write to the Free Software Foundation; we
      sometimes make exceptions for this.  Our decision will be guided by the
      two goals of preserving the free status of all derivatives of our free
      software and of promoting the sharing and reuse of software generally.

\begin{center}
NO WARRANTY
\end{center}

\bfseries

\item [11.] Because the Program is licensed free of charge, there is no
      warranty for the Program, to the extent permitted by applicable law.
      except when otherwise stated in writing the copyright holders and/or
      other parties provide the program ``as is'' without warranty of any
      kind, either expressed or implied, including, but not limited to, the
      implied warranties of merchantability and fitness for a particular
      purpose.  The entire risk as to the quality and performance of the
      Program is with you.  Should the Program prove defective, you assume
      the cost of all necessary servicing, repair or correction.

\item [12.] In no event unless required by applicable law or agreed to in
      writing will any copyright holder, or any other party who may modify
      and/or redistribute the program as permitted above, be liable to you
      for damages, including any general, special, incidental or
      consequential damages arising out of the use or inability to use the
      program (including but not limited to loss of data or data being
      rendered inaccurate or losses sustained by you or third parties or a
      failure of the Program to operate with any other programs), even if
      such holder or other party has been advised of the possibility of such
      damages.

\end{enumerate}

\begin{center}
\textbf{END OF TERMS AND CONDITIONS}
\end{center}


\gplsec{Appendix: How to Apply These Terms to Your New Programs}

If you develop a new program, and you want it to be of the greatest possible
use to the public, the best way to achieve this is to make it free software
which everyone can redistribute and change under these terms.

To do so, attach the following notices to the program.  It is safest to
attach them to the start of each source file to most effectively convey the
exclusion of warranty; and each file should have at least the ``copyright''
line and a pointer to where the full notice is found.

\begin{verbatim}
<one line to give the program's name and a brief idea of what it does.>
Copyright (C) 19yy  <name of author>

This program is free software; you can redistribute it and/or modify
it under the terms of the GNU General Public License as published by
the Free Software Foundation; either version 2 of the License, or
(at your option) any later version.

This program is distributed in the hope that it will be useful,
but WITHOUT ANY WARRANTY; without even the implied warranty of
MERCHANTABILITY or FITNESS FOR A PARTICULAR PURPOSE.  See the
GNU General Public License for more details.

You should have received a copy of the GNU General Public License
along with this program; if not, write to the Free Software
Foundation, Inc., 675 Mass Ave, Cambridge, MA 02139, USA.
\end{verbatim}

Also add information on how to contact you by electronic and paper mail.

If the program is interactive, make it output a short notice like this when
it starts in an interactive mode:

\begin{verbatim}
Gnomovision version 69, Copyright (C) 19yy name of author
Gnomovision comes with ABSOLUTELY NO WARRANTY; for details type `show w'.
This is free software, and you are welcome to redistribute it
under certain conditions; type `show c' for details.
\end{verbatim}

The hypothetical commands `show w' and `show c' should show the appropriate
parts of the General Public License.  Of course, the commands you use may be
called something other than `show w' and `show c'; they could even be
mouse-clicks or menu items--whatever suits your program.

You should also get your employer (if you work as a programmer) or your
school, if any, to sign a ``copyright disclaimer'' for the program, if
necessary.  Here is a sample; alter the names:

\begin{verbatim}
Yoyodyne, Inc., hereby disclaims all copyright interest in the program
`Gnomovision' (which makes passes at compilers) written by James Hacker.

<signature of Ty Coon>, 1 April 1989
Ty Coon, President of Vice
\end{verbatim}

This General Public License does not permit incorporating your program into
proprietary programs.  If your program is a subroutine library, you may
consider it more useful to permit linking proprietary applications with the
library.  If this is what you want to do, use the GNU Library General Public
License instead of this License.

\gplend
}
%    \end{macrocode}
%
% \end{macro}
%
% \begin{macro}{\implementation}
%
% The |\implementation| macro starts typesetting the implementation of
% the package(s).  If we're not doing the implementation, it just does
% this lot and ends the input file.
%
% I define a macro with arguments inside the |\StopEventually|, which causes
% problems, since the code gets put through an extra level of |\def|fing
% depending on whether the implementation stuff gets typeset or not.  I'll
% store the code I want to do in a separate macro.
%
%    \begin{macrocode}
\def\implementation{\StopEventually{\attheend}}
%    \end{macrocode}
%
% Now for the actual activity.  First, I'll do the |\finalstuff|.  Then, if
% \package{doc}'s managed to find the \package{multicol} package, I'll add
% the end of the environment to the end of each contents file in the list.
% Finally, I'll read the index in from its formatted |.ind| file.
%
%    \begin{macrocode}
\tdef\attheend{%
  \finalstuff%
  \ifhave@multicol%
    \def\do##1##2{\addtocontents{##1}{\protect\end{multicols}}}%
    \docontents%
  \fi%
  \PrintIndex%
}
%    \end{macrocode}
%
% \end{macro}
%
%
% \subsection{File version information}
%
% \begin{macro}{\mdwpkginfo}
%
% For setting up the automatic titles, I'll need to be able to work out
% file versions and things.  This macro will, given a file name, extract
% from \LaTeX\ the version information and format it into a sensible string.
%
% First of all, I'll put the original string (direct from the
% |\Provides|\dots\ command).  Then I'll pass it to another macro which can
% parse up the string into its various bits, along with the original
% filename.
%
%    \begin{macrocode}
\def\mdwpkginfo#1{%
  \edef\@tempa{\csname ver@#1\endcsname}%
  \expandafter\mdwpkginfo@i\@tempa\@@#1\@@%
}
%    \end{macrocode}
%
% Now for the real business.  I'll store the string I build in macros called
% \syntax{"\\"<filename>"date", "\\"<filename>"version" and
% "\\"<filename>"info"}, which store the file's date, version and
% `information string' respectively.  (Note that the file extension isn't
% included in the name.)
%
% This is mainly just tedious playing with |\expandafter|.  The date format
% is defined by a separate macro, which can be modified from the
% configuration file.
%
%    \begin{macrocode}
\def\mdwpkginfo@i#1/#2/#3 #4 #5\@@#6.#7\@@{%
  \expandafter\def\csname #6date\endcsname%
    {\protect\mdwdateformat{#1}{#2}{#3}}%
  \expandafter\def\csname #6version\endcsname{#4}%
  \expandafter\def\csname #6info\endcsname{#5}%
}
%    \end{macrocode}
%
% \end{macro}
%
% \begin{macro}{\mdwdateformat}
%
% Given three arguments, a year, a month and a date (all numeric), build a
% pretty date string.  This is fairly simple really.
%
%    \begin{macrocode}
\tdef\mdwdateformat#1#2#3{\number#3\ \monthname{#2}\ \number#1}
\def\monthname#1{%
  \ifcase#1\or%
     January\or February\or March\or April\or May\or June\or%
     July\or August\or September\or October\or November\or December%
  \fi%
}
\def\numsuffix#1{%
  \ifnum#1=1 st\else%
  \ifnum#1=2 nd\else%
  \ifnum#1=3 rd\else%
  \ifnum#1=21 st\else%
  \ifnum#1=22 nd\else%
  \ifnum#1=23 rd\else%
  \ifnum#1=31 st\else%
  th%
  \fi\fi\fi\fi\fi\fi\fi%
}
%    \end{macrocode}
%
% \end{macro}
%
% \begin{macro}{\mdwfileinfo}
%
% Saying \syntax{"\\mdwfileinfo{"<file-name>"}{"<info>"}"} extracts the
% wanted item of \<info> from the version information for file \<file-name>.
%
%    \begin{macrocode}
\def\mdwfileinfo#1#2{\mdwfileinfo@i{#2}#1.\@@}
\def\mdwfileinfo@i#1#2.#3\@@{\csname#2#1\endcsname}
%    \end{macrocode}
%
% \end{macro}
%
%
% \subsection{List handling}
%
% There are several other lists I need to build.  These macros will do
% the necessary stuff.
%
% \begin{macro}{\mdw@ifitem}
%
% The macro \syntax{"\\mdw@ifitem"<item>"\\in"<list>"{"<true-text>"}"^^A
% "{"<false-text>"}"} does \<true-text> if the \<item> matches any item in
% the \<list>; otherwise it does \<false-text>.
%
%    \begin{macrocode}
\def\mdw@ifitem#1\in#2{%
  \@tempswafalse%
  \def\@tempa{#1}%
  \def\do##1{\def\@tempb{##1}\ifx\@tempa\@tempb\@tempswatrue\fi}%
  #2%
  \if@tempswa\expandafter\@firstoftwo\else\expandafter\@secondoftwo\fi%
}
%    \end{macrocode}
%
% \end{macro}
%
% \begin{macro}{\mdw@append}
%
% Saying \syntax{"\\mdw@append"<item>"\\to"<list>} adds the given \<item>
% to the end of the given \<list>.
%
%    \begin{macrocode}
\def\mdw@append#1\to#2{%
  \toks@{\do{#1}}%
  \toks\tw@\expandafter{#2}%
  \edef#2{\the\toks\tw@\the\toks@}%
}
%    \end{macrocode}
%
% \end{macro}
%
% \begin{macro}{\mdw@prepend}
%
% Saying \syntax{"\\mdw@prepend"<item>"\\to"<list>} adds the \<item> to the
% beginning of the \<list>.
%
%    \begin{macrocode}
\def\mdw@prepend#1\to#2{%
  \toks@{\do{#1}}%
  \toks\tw@\expandafter{#2}%
  \edef#2{\the\toks@\the\toks\tw@}%
}
%    \end{macrocode}
%
% \end{macro}
%
% \begin{macro}{\mdw@add}
%
% Finally, saying \syntax{"\\mdw@add"<item>"\\to"<list>} adds the \<item>
% to the list only if it isn't there already.
%
%    \begin{macrocode}
\def\mdw@add#1\to#2{\mdw@ifitem#1\in#2{}{\mdw@append#1\to#2}}
%    \end{macrocode}
%
% \end{macro}
%
%
% \subsection{Described file handling}
%
% I'l maintain lists of packages, document classes, and other files
% described by the current documentation file.
%
% First of all, I'll declare the various list macros.
%
%    \begin{macrocode}
\def\dopackages{}
\def\doclasses{}
\def\dootherfiles{}
%    \end{macrocode}
%
% \begin{macro}{\describespackage}
%
% A document file can declare that it describes a package by saying
% \syntax{"\\describespackage{"<package-name>"}"}.  I add the package to
% my list, read the package into memory (so that the documentation can
% offer demonstrations of it) and read the version information.
%
%    \begin{macrocode}
\def\describespackage#1{%
  \mdw@ifitem#1\in\dopackages{}{%
    \mdw@append#1\to\dopackages%
    \usepackage{#1}%
    \mdwpkginfo{#1.sty}%
  }%
}
%    \end{macrocode}
%
% \end{macro}
%
% \begin{macro}{\describesclass}
%
% By saying \syntax{"\\describesclass{"<class-name>"}"}, a document file
% can declare that it describes a document class.  I'll assume that the
% document class is already loaded, because it's much too late to load
% it now.
%
%    \begin{macrocode}
\def\describesclass#1{\mdw@add#1\to\doclasses\mdwpkginfo{#1.cls}}
%    \end{macrocode}
%
% \end{macro}
%
% \begin{macro}{\describesfile}
%
% Finally, other `random' files, which don't have the status of real \LaTeX\
% packages or document classes, can be described by saying \syntax{^^A
% "\\describesfile{"<file-name>"}" or "\\describesfile*{"<file-name>"}"}.
% The difference is that the starred version will not |\input| the file.
%
%    \begin{macrocode}
\def\describesfile{%
  \@ifstar{\describesfile@i\@gobble}{\describesfile@i\input}%
}
\def\describesfile@i#1#2{%
  \mdw@ifitem#2\in\dootherfiles{}{%
    \mdw@add#2\to\dootherfiles%
    #1{#2}%
    \mdwpkginfo{#2}%
  }%
}
%    \end{macrocode}
%
% \end{macro}
%
%
% \subsection{Author and title handling}
%
% I'll redefine the |\author| and |\title| commands so that I get told
% whether I need to do it myself.
%
% \begin{macro}{\author}
%
% This is easy: I'll save the old meaning, and then redefine |\author| to
% do the old thing and redefine itself to then do nothing.
%
%    \begin{macrocode}
\let\mdw@author\author
\def\author{\let\author\@gobble\mdw@author}
%    \end{macrocode}
%
% \end{macro}
%
% \begin{macro}{\title}
%
% And oddly enough, I'll do exactly the same thing for the title, except
% that I'll also disable the |\mdw@buildtitle| command, which constructs
% the title automatically.
%
%    \begin{macrocode}
\let\mdw@title\title
\def\title{\let\title\@gobble\let\mdw@buildtitle\relax\mdw@title}
%    \end{macrocode}
%
% \end{macro}
%
% \begin{macro}{\date}
%
% This works in a very similar sort of way.
%
%    \begin{macrocode}
\def\date#1{\let\date\@gobble\def\today{#1}}
%    \end{macrocode}
%
% \end{macro}
%
% \begin{macro}{\datefrom}
%
% Saying \syntax{"\\datefrom{"<file-name>"}"} sets the document date from
% the given filename.
%
%    \begin{macrocode}
\def\datefrom#1{%
  \protected@edef\@tempa{\noexpand\date{\csname #1date\endcsname}}%
  \@tempa%
}
%    \end{macrocode}
%
% \end{macro}
%
% \begin{macro}{\docfile}
%
% Saying \syntax{"\\docfile{"<file-name>"}"} sets up the file name from which
% documentation will be read.
%
%    \begin{macrocode}
\def\docfile#1{%
  \def\@tempa##1.##2\@@{\def\@basefile{##1.##2}\def\@basename{##1}}%
  \edef\@tempb{\noexpand\@tempa#1\noexpand\@@}%
  \@tempb%
}
%    \end{macrocode}
%
% I'll set up a default value as well.
%
%    \begin{macrocode}
\docfile{\jobname.dtx}
%    \end{macrocode}
%
% \end{macro}
%
%
% \subsection{Building title strings}
%
% This is rather tricky.  For each list, I need to build a legible looking
% string.
%
% \begin{macro}{\mdw@addtotitle}
%
% By saying
%\syntax{"\\mdw@addtotitle{"<list>"}{"<command>"}{"<singular>"}{"<plural>"}"}
% I can add the contents of a list to the current title string in the
% |\mdw@title| macro.
%
%    \begin{macrocode}
\tdef\mdw@addtotitle#1#2#3#4{%
%    \end{macrocode}
%
% Now to get to work.  I need to keep one `lookahead' list item, and a count
% of the number of items read so far.  I'll keep the lookahead item in
% |\@nextitem| and the counter in |\count@|.
%
%    \begin{macrocode}
  \count@\z@%
%    \end{macrocode}
%
% Now I'll define what to do for each list item.  The |\protect| command is
% already set up appropriately for playing with |\edef| commands.
%
%    \begin{macrocode}
  \def\do##1{%
%    \end{macrocode}
%
% The first job is to add the previous item to the title string.  If this
% is the first item, though, I'll just add the appropriate \lit{The } or
% \lit{ and the } string to the title (this is stored in the |\@prefix|
% macro).
%
%    \begin{macrocode}
    \edef\mdw@title{%
      \mdw@title%
      \ifcase\count@\@prefix%
      \or\@nextitem%
      \else, \@nextitem%
      \fi%
    }%
%    \end{macrocode}
%
% That was rather easy.  Now I'll set up the |\@nextitem| macro for the
% next time around the loop.
%
%    \begin{macrocode}
    \edef\@nextitem{%
      \protect#2{##1}%
      \protect\footnote{%
        The \protect#2{##1} #3 is currently at version %
        \mdwfileinfo{##1}{version}, dated \mdwfileinfo{##1}{date}.%
      }\space%
    }%
%    \end{macrocode}
%
% Finally, I need to increment the counter.
%
%    \begin{macrocode}
    \advance\count@\@ne%
  }%
%    \end{macrocode}
%
% Now execute the list.
%
%    \begin{macrocode}
  #1%
%    \end{macrocode}
%
% I still have one item left over, unless the list was empty.  I'll add
% that now.
%
%    \begin{macrocode}
  \edef\mdw@title{%
    \mdw@title%
    \ifcase\count@%
    \or\@nextitem\space#3%
    \or\ and \@nextitem\space#4%
    \fi%
  }%
%    \end{macrocode}
%
% Finally, if $|\count@| \ne 0$, I must set |\@prefix| to \lit{ and the }.
%
%    \begin{macrocode}
  \ifnum\count@>\z@\def\@prefix{ and the }\fi%
}
%    \end{macrocode}
%
% \end{macro}
%
% \begin{macro}{\mdw@buildtitle}
%
% This macro will actually do the job of building the title string.
%
%    \begin{macrocode}
\tdef\mdw@buildtitle{%
%    \end{macrocode}
%
% First of all, I'll open a group to avoid polluting the namespace with
% my gubbins (although the code is now much tidier than it has been in
% earlier releases).
%
%    \begin{macrocode}
  \begingroup%
%    \end{macrocode}
%
% The title building stuff makes extensive use of |\edef|.  I'll set
% |\protect| appropriately.  (For those not in the know,
% |\@unexpandable@protect| expands to `|\noexpand\protect\noexpand|',
% which prevents expansion of the following macro, and inserts a |\protect|
% in front of it ready for the next |\edef|.)
%
%    \begin{macrocode}
  \let\@@protect\protect\let\protect\@unexpandable@protect%
%    \end{macrocode}
%
% Set up some simple macros ready for the main code.
%
%    \begin{macrocode}
  \def\mdw@title{}%
  \def\@prefix{The }%
%    \end{macrocode}
%
% Now build the title.  This is fun.
%
%    \begin{macrocode}
  \mdw@addtotitle\dopackages\package{package}{packages}%
  \mdw@addtotitle\doclasses\package{document class}{document classes}%
  \mdw@addtotitle\dootherfiles\texttt{file}{files}%
%    \end{macrocode}
%
% Now I want to end the group and set the title from my string.  The
% following hacking will do this.
%
%    \begin{macrocode}
  \edef\next{\endgroup\noexpand\title{\mdw@title}}%
  \next%
}
%    \end{macrocode}
%
% \end{macro}
%
%
% \subsection{Starting the main document}
%
% \begin{macro}{\mdwdoc}
%
% Once the document preamble has done all of its stuff, it calls the
% |\mdwdoc| command, which takes over and really starts the documentation
% going.
%
%    \begin{macrocode}
\def\mdwdoc{%
%    \end{macrocode}
%
% First, I'll construct the title string.
%
%    \begin{macrocode}
  \mdw@buildtitle%
  \author{Mark Wooding}%
%    \end{macrocode}
%
% Set up the date string based on the date of the package which shares
% the same name as the current file.
%
%    \begin{macrocode}
  \datefrom\@basename%
%    \end{macrocode}
%
% Set up verbatim characters after all the packages have started.
%
%    \begin{macrocode}
  \shortverb\|%
  \shortverb\"%
%    \end{macrocode}
%
% Start the document, and put the title in.
%
%    \begin{macrocode}
  \begin{document}
  \maketitle%
%    \end{macrocode}
%
% This is nasty.  It makes maths displays work properly in demo environments.
% \emph{The \LaTeX\ Companion} exhibits the bug which this hack fixes.  So
% ner.
%
%    \begin{macrocode}
  \abovedisplayskip\z@%
%    \end{macrocode}
%
% Now start the contents tables.  After starting each one, I'll make it
% be multicolumnar.
%
%    \begin{macrocode}
  \def\do##1##2{%
    ##2%
    \ifhave@multicol\addtocontents{##1}{%
      \protect\begin{multicols}{2}%
      \hbadness\@M%
    }\fi%
  }%
  \docontents%
%    \end{macrocode}
%
% Input the main file now.
%
%    \begin{macrocode}
  \DocInput{\@basefile}%
%    \end{macrocode}
%
% That's it.  I'm done.
%
%    \begin{macrocode}
  \end{document}
}
%    \end{macrocode}
%
% \end{macro}
%
%
% \subsection{And finally\dots}
%
% Right at the end I'll put a hook for the configuration file.
%
%    \begin{macrocode}
\ifx\mdwhook\@@undefined\else\expandafter\mdwhook\fi
%    \end{macrocode}
%
% That's all the code done now.  I'll change back to `user' mode, where
% all the magic control sequences aren't allowed any more.
%
%    \begin{macrocode}
\makeatother
%</mdwtools>
%    \end{macrocode}
%
% Oh, wait!  What if I want to typeset this documentation?  Aha.  I'll cope
% with that by comparing |\jobname| with my filename |mdwtools|.  However,
% there's some fun here, because |\jobname| contains category-12 letters,
% while my letters are category-11.  Time to play with |\string| in a messy
% way.
%
%    \begin{macrocode}
%<*driver>
\makeatletter
\edef\@tempa{\expandafter\@gobble\string\mdwtools}
\edef\@tempb{\jobname}
\ifx\@tempa\@tempb
  \describesfile*{mdwtools.tex}
  \docfile{mdwtools.tex}
  \makeatother
  \expandafter\mdwdoc
\fi
\makeatother
%</driver>
%    \end{macrocode}
%
% That's it.  Done!
%
% \hfill Mark Wooding, \today
%
% \Finale
%
\endinput

\describespackage{at}
\aton
\atlet p=\package
\atdef at{\package{at}}
\atdef={\mbox{-}}
\atdef-{@@@=}
\atlet.=\syntax
\mdwdoc
%</driver>
%
% \end{meta-comment}
%
% \section{User guide}
%
% The @at\ package is an attempt to remove a lot of tedious typing that
% ends up in \LaTeX\ documents, by expanding the number of short command
% names available.  The new command names begin with the `|@|' character,
% rather than the conventional `|\|', so you can tell them apart.
%
% The package provides some general commands for defining @-commands, and
% then uses them to define some fairly simple ones which will be useful to
% most people.
%
% The rules for @-command names aren't terribly complex:
% \begin{itemize}
% \item If the first character of the name is a letter, then the command name
%       consists of all characters up to, but not including, the first
%       nonletter.  Spaces following the command name are ignored.
% \item If the first character of the name is a backslash, then the @-command
%       name consists of the control sequence introduced by the backslash.
% \item Otherwise, the command name consists only of that first character.
%       Spaces following the name are not ignored, unless that character
%       was itself a space character.
% \end{itemize}
%
% Usually, digits are not considered to be letters.  However, the
% \package{at} package will consider digits to be letters if you give it the
% \textsf{digits} option in the |\usepackage| command.  (Note that this
% only affects the \package{at} package; it won't change the characters
% allowed in normal command names.)
%
% \DescribeMacro{\atallowdigits}
% \DescribeMacro{\atdisallowdigits}
% You can enable and disable digits being considered as letters dynamically.
% The |\atallowdigits| command allows digits to be used as letters;
% |\atdisallowdigits| prevents this.  Both declarations follow \LaTeX's
% usual scoping rules.  Both of these commands have corresponding
% environments with the same names (without the leading `|\|', obviously).
%
% \subsection{Defining @-commands}
%
% \DescribeMacro{\newatcommand}
% \DescribeMacro{\renewatcommand}
% The |\newatcommand| command will define a new @-command using a syntax
% similar to |\newcommand|.  For example, you could define
% \begin{listing}
%\newatcommand c[1]{\chapter{#1}}
% \end{listing}
% to make @.{"@c{"<name>"}"} equivalent to @.{"\\chapter{"<name>"}"}.
%
% A |\renewatcommand| is also provided to redefine existing commands, should
% the need arise.
%
% \DescribeMacro{\atdef}
% For \TeX\ hackers, the |\atdef| command defines @-commands using a syntax
% similar to \TeX's built-in |\def|.  
%
% As an example, the following command makes @.{"@/"<text>"/"} write its
% argument \<text> in italics:
% \begin{listing}
%\atdef/#1/{\textit{#1}}
% \end{listing}
% The real implementation of the |@/|\dots|/| command is a bit more
% complex, and is given in the next section.
%
% You can use all of \TeX's features for defining the syntax of your
% command.  (See chapter~20 of @/The \TeX book/ for more details.)
%
% \DescribeMacro{\atlet}
% Since |\atdef| is provided to behave similarly to |\def|, @at\ provides
% |\atlet| which works similarly to |\let|.  For example you can say
% \begin{listing}
%\atlet!=\index
% \end{listing}
% to allow the short |@!| to behave exactly like |\index|.
%
% Note that all commands defined using these commands are robust even if you
% use fragile commands in their definitions.  Unless you start doing very
% strange things, @-commands never need |\protect|ing.
%
% \subsection{Predefined @-commands}
%
% A small number of hopefully useful commands are provided by default.
% These are described in the table below:
%
% \bigskip \begin{center} \begin{tabular}{lp{3in}}                    \hline
% \bf Command        & \bf Meaning                                 \\ \hline
% @.{"@@"}           & Typesets an `@@' character.                 \\
% @.{"@/"<text>"/"}  & In text (LR or paragraph) mode, typesets its
%                      argument emphasised.  In maths mode, it
%                      always chooses italics.                     \\
% @.{"@*"<text>"*"}  & Typesets its argument \<text> in bold.      \\
% @.{"@i{"<text>"}"} & Equivalent to `@.{"\\index{"<text>"}"}'.    \\
% @.{"@I{"<text>"}"} & As for |@i|, but also writes its argument
%                      to the document.                            \\ \hline
% \end{tabular} \end{center} \bigskip
%
% Package writers should not rely on any predefined @-commands -- they're
% provided for users, and users should be able to redefine them without
% fear of messing anything up.  (This includes the `standard' commands
% provided by the @at\ package, by the way.  They're provided in the vague
% hope that they might be useful, and as examples.)
%
% \implementation
%
% \section{Implementation}
%
%    \begin{macrocode}
%<*package>
%    \end{macrocode}
%
% \subsection{Options handling}
%
% We need a switch to say whether digits should be allowed.  Since this
% is a user thing, I'll avoid |\newif| and just define the thing by hand.
%
%    \begin{macrocode}
\def\atallowdigits{\let\ifat@digits\iftrue}
\def\atdisallowdigits{\let\ifat@digits\iffalse}
%    \end{macrocode}
%
% Now define the options.
%
%    \begin{macrocode}
\DeclareOption{digits}{\atallowdigits}
\DeclareOption{nodigits}{\atdisallowdigits}
\ExecuteOptions{nodigits}
\ProcessOptions
%    \end{macrocode}
%
% \subsection{How the commands work}
%
% Obviously we make the `@@' character active.  It inspects the next
% character (or argument, actually -- it can be enclosed in braces for
% longer commands, although this is a bit futile), and builds the command
% name from that.
%
% The |\at| command is equivalent to the active `@@' character always.
%
%
% \subsection{Converting command names}
%
% We need to be able to read an @-command name, and convert it to a normal
% \TeX\ control sequence.  First, we declare some control sequences for
% braces, which we need later.
%
%    \begin{macrocode}
\begingroup
\catcode`\<1
\catcode`\>2
\catcode`\{12
\catcode`\}12
\gdef\at@lb<{>
\gdef\at@rb<}>
\gdef\at@spc< >
\endgroup
%    \end{macrocode}
%
% I'll set up some helper routines now, to help me read the command
% names.  The way this works is that we |\futurelet| the token into
% |\@let@token|.  These routines will then sort out what to do next.
%
% \begin{macro}{\at@test}
%
% Given an |\if|\dots\ test, does its first or second argument.
%
%    \begin{macrocode}
\def\at@test#1\then{%
  #1\expandafter\@firstoftwo\else\expandafter\@secondoftwo\fi%
}
%    \end{macrocode}
%
% \end{macro}
%
% \begin{macro}{\at@ifcat}
%
% Checks the category code of the current character.  If it matches the
% argument, it does its second argument, otherwise it does the third.
%
%    \begin{macrocode}
\def\at@ifcat#1{\at@test\ifcat#1\noexpand\@let@token\then}
%    \end{macrocode}
%
% \end{macro}
%
% \begin{macro}{\at@ifletter}
%
% This routine tests the token to see if it's a letter, and if so adds
% it to the token list and does the first argument; otherwise it does the
% second argument.  It accepts digits as letters if the switch is turned
% on.
%
% There's some fun later, so I'll describe this slowly.  First, we compare
% the category code to a letter, and if we have a match, we know we're done;
% we need to pick up the letter as an argument.  If the catcode is `other',
% we must compare with numbers to see if it's in range.
%
%    \begin{macrocode}
\def\at@ifletter#1#2{%
  \at@ifcat x%
    {\at@ifletter@ii{#1}}%
    {\at@ifcat 0%
      {\at@ifletter@i{#1}{#2}}%
      {#2}%
    }% 
}
%    \end{macrocode}
%
% Right.  It's `other' (so it's safe to handle as a macro argument) and we
% need to know if it's a digit.  This is a little tricky: I use |\if| to
% compare two characters.  The first character is~`1' or~`0' depending on the
% `digit' switch; the second is~`1' or~`x' depending on whether it's actually
% a digit.  They'll only match if everything's worked out OK.
%
%    \begin{macrocode}
\def\at@ifletter@i#1#2#3{%
  \at@test\if%
    \ifat@digits1\else0\fi%
    \ifnum`#3<`0x\else\ifnum`#3>`9x\else1\fi\fi%
  \then%
    {\at@ifletter@ii{#1}{#3}}%
    {#2#3}%
}
%    \end{macrocode}
%
% Right; we have the character, so add it to the list and carry on.
%
%    \begin{macrocode}
\def\at@ifletter@ii#1#2{\toks@\expandafter{\the\toks@#2}#1}
%    \end{macrocode}
%
% \end{macro}
%
% Now we define the command name reading routines.  We have @/almost/ the
% same behaviour as \TeX, although we can't support `|%|' characters for
% reasons to do with \TeX's tokenising algorithm.
%
% \begin{macro}{\at@read@name}
%
% The routine which actually reads the command name works as follows:
% \begin{enumerate}
% \item Have a peek at the next character.  If it's a left or right brace,
%       then use the appropriate character.
% \item If the character is not a letter, just use the character (or whole
%       control sequence.
% \item Finally, if it's a letter, keep reading letters until we find one
%       that wasn't.
% \end{enumerate}
%
% First, we do some setting up and read the first character
%
%    \begin{macrocode}
\def\at@read@name#1{%
  \let\at@next=#1%
  \toks@{}%
  \futurelet\@let@token\at@rn@i%
}
%    \end{macrocode}
%
% Next, sort out what to do, based on the category code.
%
%    \begin{macrocode}
\def\at@rn@i{%
  \def\@tempa{\afterassignment\at@rn@iv\let\@let@token= }%
  \at@ifletter%
    {\futurelet\@let@token\at@rn@iii}%
    {\at@ifcat\bgroup%
      {\toks@\expandafter{\at@lb}\@tempa}%
      {\at@ifcat\egroup%
        {\toks@\expandafter{\at@rb}\@tempa}%
        {\at@ifcat\at@spc%
          {\toks@{ }\@tempa}%
          {\at@rn@ii}%
        }%
      }%
    }%
}
%    \end{macrocode}
%
% Most types of tokens can be fiddled using |\string|.
%
%    \begin{macrocode}
\def\at@rn@ii#1{%
  \toks@\expandafter{\string#1}%
  \at@rn@iv%
}
%    \end{macrocode}
%
% We've found a letter, so we should check for another one.
%
%    \begin{macrocode}
\def\at@rn@iii{%
  \at@ifletter%
    {\futurelet\@let@token\at@rn@iii}%
    {\@ifnextchar.\at@rn@iv\at@rn@iv}%
}
%    \end{macrocode}
%
% Finally, we need to pass the real string, as an argument, to the
% macro.  We make |\@let@token| relax, since it might be something which will
% upset \TeX\ later, e.g., a |#| character.
%
%    \begin{macrocode}
\def\at@rn@iv{%
  \let\@let@token\relax%
  \expandafter\at@next\csname at.\the\toks@\endcsname%
}
%    \end{macrocode}
%
% \end{macro}
%
% \begin{macro}{\at@cmdname}
%
% Given a control sequence, work out which @-command it came from.
%
%    \begin{macrocode}
\def\at@cmdname#1{\expandafter\at@cmdname@i\string#1\@@foo}
%    \end{macrocode}
%
% Now extract the trailing bits.
%
%    \begin{macrocode}
\def\at@cmdname@i#1.#2\@@foo{#2}
%    \end{macrocode}
%
% \end{macro}
%
% \begin{macro}{\at@decode}
%
% The |\at@decode| macro takes an extracted @-command name, and tries to
% execute the correct control sequence derived from it.
%
%    \begin{macrocode}
\def\at@decode#1{%
  \at@test\ifx#1\relax\then{%
    \PackageError{at}{Unknown @-command `@\at@cmdname#1'}{%
      The @-command you typed wasn't recognised, so I've ignored it.
    }%
  }{%
    #1%
  }%
}
%    \end{macrocode}
%
% \end{macro}
%
% \begin{macro}{\@at}
%
% We'd like a measure of compatibility with @p{amsmath}.  The @-commands
% provided by @p{amsmath} work only in maths mode, so this gives us a way of
% distinguishing.  If the control sequence |\Iat| is defined, and we're in
% maths mode, we'll call that instead of doing our own thing.
%
%    \begin{macrocode}
\def\@at{%
  \def\@tempa{\at@read@name\at@decode}%
  \ifmmode\ifx\Iat\not@@defined\else%
    \let\@tempa\Iat%
  \fi\fi%
  \@tempa%
}
%    \end{macrocode}
%
% \end{macro}
%
%
% \subsection{Defining new commands}
%
% \begin{macro}{\at@buildcmd}
%
% First, we define a command to build these other commands:
%
%    \begin{macrocode}
\def\at@buildcmd#1#2{%
  \expandafter\def\csname\expandafter
    \@gobble\string#1@decode\endcsname##1{#2##1}%
  \edef#1{%
    \noexpand\at@read@name%
    \expandafter\noexpand%
      \csname\expandafter\@gobble\string#1@decode\endcsname%
  }%
}
%    \end{macrocode}
%
% \end{macro}
%
% \begin{macro}{\newatcommand}
% \begin{macro}{\renewatcommand}
% \begin{macro}{\provideatcommand}
% \begin{macro}{\atdef}
% \begin{macro}{\atshow}
%
% Now we define the various operations on @-commands.
%
%    \begin{macrocode}
\at@buildcmd\newatcommand\newcommand
\at@buildcmd\renewatcommand\renewcommand
\at@buildcmd\provideatcommand\providecommand
\at@buildcmd\atdef\def
\at@buildcmd\atshow\show
%    \end{macrocode}
%
% \end{macro}
% \end{macro}
% \end{macro}
% \end{macro}
% \end{macro}
%
% \begin{macro}{\atlet}
%
% |\atlet| is rather harder than the others, because we want to allow people
% to say things like @.{"\\atlet"<name>"=@"<name>}.  The following hacking
% does the trick.  I'm trying very hard to duplicate |\let|'s behaviour with
% respect to space tokens here, to avoid any surprises, although there
% probably will be some differences.  In particular, |\afterassignment|
% won't work in any sensible way.
%
% First, we read the name of the @-command we're defining.  We also open
% a group, to stop messing other people up, and make `@@' into an `other'
% token, so that it doesn't irritatingly look like its meaning as a control
% sequence.
%
%    \begin{macrocode}
\def\atlet{%
  \begingroup%
  \@makeother\@%
  \at@read@name\atlet@i%
}
%    \end{macrocode}
%
% Put the name into a scratch macro for later use.  Now see if there's an
% equals sign up ahead.  If not, this will gobble any spaces in between the
% @-command name and the argument.
%
%    \begin{macrocode}
\def\atlet@i#1{%
  \def\at@temp{#1}%
  \@ifnextchar=\atlet@ii{\atlet@ii=}%
}
%    \end{macrocode}
%
% Now we gobble the equals sign (whatever catcode it is), and peek at the
% next token up ahead using |\let| with no following space.
%
%    \begin{macrocode}
\def\atlet@ii#1{\afterassignment\atlet@iii\global\let\at@gnext=}
%    \end{macrocode}
%
% The control sequence |\at@gnext| is now |\let| to be whatever we want the
% @-command to be, unless it's picked up an `@@' sign.  If it has, we've
% eaten the |@| token, so just read the name and pass it on.  Otherwise,
% we can |\let| the @-command directly to |\at@gnext|.  There's some
% nastiness here to make |\the\toks@| expand before we close the group and
% restore its previous definition.
%
%    \begin{macrocode}
\def\atlet@iii{%
  \if @\noexpand\at@gnext%
    \expandafter\at@read@name\expandafter\atlet@iv%
  \else%
    \expandafter\endgroup%
    \expandafter\let\at@temp= \at@gnext%
  \fi%
}
%    \end{macrocode}
%
% We've read the source @-command name, so just copy the definitions over.
%
%    \begin{macrocode}
\def\atlet@iv#1{%
  \expandafter\endgroup%
  \expandafter\let\at@temp=#1%
}
%    \end{macrocode}
%
% \end{macro}
%
%
% \subsection{Robustness of @-commands}
%
% We want all @-commands to be robust.  We could leave them all being
% fragile, although making robust @-commands would then be almost impossible.
% There are two problems which we must face:
%
% \begin{itemize}
%
% \item The `|\@at|' command which scans the @-command name is (very)
%       fragile.  I could have used |\DeclareRobustCommand| for it (and in
%       fact I did in an earlier version), but that doesn't help the other
%       problem at all.
%
% \item The `name' of the @-command may contain active characters or control
%       sequences, which will be expanded at the wrong time unless we do
%       something about it now.
%
% \end{itemize}
%
% We must also be careful not to introduce extra space characters into any
% files written, because spaces are significant in @-commands.  Finally,
% we have a minor problem in that most auxiliary files are read in with
% the `@@' character set to be a letter.
%
% \begin{macro}{\at}
%
% Following the example of \LaTeX's `short' command handling, we'll define
% |\at| to decide what to do depending on what |\protect| looks like.  If
% we're typesetting, we just call |\@at| (above) and expect it to cope.
% Otherwise we call |\at@protect|, which scoops up the |\fi| and the |\@at|,
% and inserts other magic.
%
%    \begin{macrocode}
\def\at{\ifx\protect\@typeset@protect\else\at@protect\fi\@at}
%    \end{macrocode}
%
% \end{macro}
%
% \begin{macro}{\at@protect}
%
% Since we gobbled the |\fi| from the above, we must put that back.  We then 
% need to do things which are more complicated.  If |\protect| is behaving 
% like |\string|, then we do one sort of protection.  Otherwise, we assume
% that |\protect| is being like |\noexpand|.
%
%    \begin{macrocode}
\def\at@protect\fi#1{%
  \fi%
  \ifx\protect\string%
    \expandafter\at@protect@string%
  \else%
    \expandafter\at@protect@noexpand%
  \fi%
}
%    \end{macrocode}
%
% \end{macro}
%
% \begin{macro}{\at@protect@string}
%
% When |\protect| is |\string|, we don't need to be able to recover the
% original text particularly accurately -- it's for the user to look at.
% Therefore, we just output a $|@|_{11}$ and use |\string| on the next
% token.  This must be sufficient, since we only allow multi-token command
% names if the first token is a letter (code~11).
%
%    \begin{macrocode}
\def\at@protect@string{@\string}
%    \end{macrocode}
%
% \end{macro}
%
% \begin{macro}{\at@protect@noexpand}
%
% This is a little more complex, since we're still expecting to be executed
% properly at some stage.  However, there's a cheeky dodge we can employ
% since the |\at| command is thoroughly robustified (or at least it will be
% by the time we've finished this).  All |\@unexpandable@protect| does
% is confer repeated robustness on a fragile command.  Since our command
% is robust, we don't need this and we can get away with just using a
% single |\noexpand|, both for the |\@at@| command and the following token
% (which we must robustify, because no-one else can do it for us -- if
% anyone tries, they end up using the |@\protect| command which is rather
% embarassing).
%
% I'll give the definition, and then examine how this expands in various
% cases.
%
%    \begin{macrocode}
\def\at@protect@noexpand{\noexpand\@at@ @\noexpand}
\def\@at@#1{\at}
%    \end{macrocode}
%
% A few points, before we go into the main examination of the protection.
% I've inserted a $|@|_{11}$ token, which is gobbled by |\@at@| when the
% thing is finally expanded fully.  This prevents following space tokens
% in an |\input| file from being swallowed because they follow a control
% sequence.  (I can't use the normal $|@|_{13}$ token, because when files
% like the |.aux| file are read in, |@| is given code~11 by
% |\makeatletter|.)
%
% \setbox0\hbox{|@at@|}
% Now for a description of why this works.  When |\at| is expanded, it works
% out that |\protect| is either |\noexpand| or |\@unexpandable@protect|, and
% becomes |\at@protect@noexpand|.  Because of the |\noexpand| tokens, this
% stops being expanded once it reaches $\fbox{\box0}\,|@|_{11}\,x$ (where
% $x$ is the token immediately following the $|@|_{13}$ character).  If this
% is expanded again, for example in another |\edef|, or in a |\write| or a
% |\mark|, the |\@at@| wakes up, gobbles the following |@| (whatever catcode
% it is -- there may be intervening |\write| and |\input| commands) and
% becomes |\at|, and the whole thing can start over again.
%
% \end{macro}
%
%
% \subsection{Enabling and disabling @-commands}
%
% \begin{macro}{\aton}
%
% We define the |\aton| command to enable all of our magic.  We store
% the old catcode in the |\atoff| command, make `@@' active, and make it
% do the stuff.
%
%    \begin{macrocode}
\def\aton{%
  \ifnum\catcode`\@=\active\else%
    \edef\atoff{\catcode`\noexpand\@\the\catcode`\@}%
    \catcode`\@\active%
    \lccode`\~`\@%
    \lowercase{\let~\at}%
  \fi%
}
%    \end{macrocode}
%
% \end{macro}
%
% \begin{macro}{\atoff}
%
% The |\atoff| command makes `@@' do the stuff it's meant to.  We remember
% the old catcode and revert to it.  This is largely unnecessary.
%
%    \begin{macrocode}
\def\atoff{\catcode`\@12}
%    \end{macrocode}
%
% \end{macro}
%
% \begin{macro}{\makeatother}
%
% Now we make our active `@@' the default outside of package files.
%
%    \begin{macrocode}
\let\makeatother\aton
%    \end{macrocode}
%
% \end{macro}
%
% And we must make sure that the user can use all of our nice commands.
% Once the document starts, we allow @-commands.
%
%    \begin{macrocode}
\AtBeginDocument{\aton}
%    \end{macrocode}
%
% \begin{macro}{\dospecials}
% \begin{macro}{\@sanitize}
%
% We must add the `@@' character to the various specials lists.
%
%    \begin{macrocode}
\expandafter\def\expandafter\dospecials\expandafter{\dospecials\do\@}
\expandafter\def\expandafter\@sanitize\expandafter{%
  \@sanitize\@makeother\@}
%    \end{macrocode}
%
% \end{macro}
% \end{macro}
%
% \subsection{Default @-commands}
%
% We define some trivial examples to get the user going.
%
%    \begin{macrocode}
\expandafter\chardef\csname at.@\endcsname=`\@
\atdef*#1*{\ifmmode\mathbf{#1}\else\textbf{#1}\fi}
\atdef/#1/{\ifmmode\mathit{#1}\else\emph{#1}\fi}
\atlet i=\index
\atdef I#1{#1\index{#1}}
%</package>
%    \end{macrocode}
%
% \hfill Mark Wooding, \today
%
% \Finale
%
\endinput
