%
%   $Id$
%   This file is part of the FPC documentation.
%   Copyright (C) 1997, by Michael Van Canneyt
%
%   The FPC documentation is free text; you can redistribute it and/or
%   modify it under the terms of the GNU Library General Public License as
%   published by the Free Software Foundation; either version 2 of the
%   License, or (at your option) any later version.
%
%   The FPC Documentation is distributed in the hope that it will be useful,
%   but WITHOUT ANY WARRANTY; without even the implied warranty of
%   MERCHANTABILITY or FITNESS FOR A PARTICULAR PURPOSE.  See the GNU
%   Library General Public License for more details.
%
%   You should have received a copy of the GNU Library General Public
%   License along with the FPC documentation; see the file COPYING.LIB.  If not,
%   write to the Free Software Foundation, Inc., 59 Temple Place - Suite 330,
%   Boston, MA 02111-1307, USA.
%
%%%%%%%%%%%%%%%%%%%%%%%%%%%%%%%%%%%%%%%%%%%%%%%%%%%%%%%%%%%%%%%%%%%%%%%
% Preamble.
\input{preamble.inc}
\latex{%
  \ifpdf
  \pdfinfo{/Author(Michael Van Canneyt)
           /Title(Standard units Reference Guide)
           /Subject(Free Pascal Reference guide)
           /Keywords(Free Pascal, Language, System Unit)
           }
\fi}
%
% Settings
%
\makeindex
%
% Syntax style
%
\usepackage{syntax}
%
% Here we determine the style of the syntax diagrams.
%
% Define a 'boxing' environment
\newenvironment{diagram}[2]%
{\begin{quote}\rule{0.5pt}{1ex}%
\rule[1ex]{\linewidth}{0.5pt}%
\rule{0.5pt}{1ex}\\[-0.5ex]%
\textbf{#1}\\[-0.5ex]}%
{\rule{0.5pt}{1ex}%
\rule{\linewidth}{0.5pt}%
\rule{0.5pt}{1ex}\end{quote}}
%\newenvironment{diagram}[2]{}{}
% Define mysyntdiag for my style of diagrams
\makeatletter
% Under Tex4HT, the diagrams are rendered as pictures.
\@ifpackageloaded{tex4ht}{%
\newenvironment{mysyntdiag}%
{\Picture*{}\begin{syntdiag}\setlength{\sdmidskip}{.5em}\sffamily\sloppy}%
{\end{syntdiag}\EndPicture}%
}{%
\newenvironment{mysyntdiag}%
{\begin{syntdiag}\setlength{\sdmidskip}{.5em}\sffamily\sloppy}%
{\end{syntdiag}}%
}% 
\makeatother
% Finally, define a combination of the above two.
\newenvironment{psyntax}[2]{\begin{diagram}{#1}{#2}\begin{mysyntdiag}}%
{\end{mysyntdiag}\end{diagram}}
% Redefine the styles used in the diagram.
\latex{\renewcommand{\litleft}{\bfseries\ }
\renewcommand{\ulitleft}{\bfseries\ }
\renewcommand{\syntleft}{\ }
\renewcommand{\litright}{\ \rule[.5ex]{.5em}{2\sdrulewidth}}
\renewcommand{\ulitright}{\ \rule[.5ex]{.5em}{2\sdrulewidth}}
\renewcommand{\syntright}{\ \rule[.5ex]{.5em}{2\sdrulewidth}}
}
% Finally, a referencing command.
\newcommand{\seesy}[1]{see diagram}
%
%
% Start of document.
%
\begin{document}
\title{Free Pascal :\\ Reference guide.}
\docdescription{Reference guide for Free Pascal, version \fpcversion}
\docversion{1.8}
\input{date.inc}
\author{Micha\"el Van Canneyt}
\maketitle
\tableofcontents
\newpage
\listoftables
\newpage


%%%%%%%%%%%%%%%%%%%%%%%%%%%%%%%%%%%%%%%%%%%%%%%%%%%%%%%%%%%%%%%%%%%%%
% Introduction
%%%%%%%%%%%%%%%%%%%%%%%%%%%%%%%%%%%%%%%%%%%%%%%%%%%%%%%%%%%%%%%%%%%%%


%%%%%%%%%%%%%%%%%%%%%%%%%%%%%%%%%%%%%%%%%%%%%%%%%%%%%%%%%%%%%%%%%%%%%%%
% About this guide
\section*{About this guide}
This document describes all constants, types, variables, functions and
procedures as they are declared in the system unit.
Furthermore, it describes all pascal constructs supported by \fpc, and lists
all supported data types. It does not, however, give a detailed explanation
of the pascal language. The aim is to list which Pascal constructs are
supported, and to show where the \fpc implementation differs from the
Turbo Pascal implementation.
\subsection*{Notations}
Throughout this document, we will refer to functions, types and variables
with \var{typewriter} font. Functions and procedures have their own
subsections, and for each function or procedure we have the following
topics:
\begin{description}
\item [Declaration] The exact declaration of the function.
\item [Description] What does the procedure exactly do ?
\item [Errors] What errors can occur.
\item [See Also] Cross references to other related functions/commands.
\end{description}
The cross-references come in two flavours:
\begin{itemize}
\item References to other functions in this manual. In the printed copy, a
number will appear after this reference. It refers to the page where this
function is explained. In the on-line help pages, this is a hyperlink, on
which you can click to jump to the declaration.
\item References to Unix manual pages. (For linux related things only) they
are printed in \var{typewriter} font, and the number after it is the Unix
manual section.
\end{itemize}
\subsection*{Syntax diagrams}
All elements of the pascal language are explained in syntax diagrams.
Syntax diagrams are like flow charts. Reading a syntax diagram means that
you must get from the left side to the right side, following the arrows.
When you are at the right of a syntax diagram, and it ends with a single
arrow, this means the syntax diagram is continued on the next line. If
the line ends on 2 arrows pointing to each other, then the diagram is
ended.

Syntactical elements are written like this
\begin{mysyntdiag}
\synt{syntactical\ elements\ are\ like\ this}
\end{mysyntdiag}
Keywords you must type exactly as in the diagram:
\begin{mysyntdiag}
\lit*{keywords\ are\ like\ this}
\end{mysyntdiag}
When you can repeat something there is an arrow around it:
\begin{mysyntdiag}
\<[b] \synt{this\ can\ be\ repeated} \\ \>
\end{mysyntdiag}
When there are different possibilities, they are listed in columns:
\begin{mysyntdiag}
\(
\synt{First\ possibility} \\
\synt{Second\ possibility}
\)
\end{mysyntdiag}
Note, that one of the possibilities can be empty:
\begin{mysyntdiag}
\[
\synt{First\ possibility} \\
\synt{Second\ possibility}
\]
\end{mysyntdiag}
This means that both the first or second possibility are optional.
Of course, all these elements can be combined and nested.

\part{The Pascal language}

%%%%%%%%%%%%%%%%%%%%%%%%%%%%%%%%%%%%%%%%%%%%%%%%%%%%%%%%%%%%%%%%%%%%%%%
% The Pascal language

%%%%%%%%%%%%%%%%%%%%%%%%%%%%%%%%%%%%%%%%%%%%%%%%%%%%%%%%%%%%%%%%%%%%%%%
\chapter{Pascal Tokens}
In this chapter we describe all the pascal reserved words, as well as the
various ways to denote strings, numbers, identifiers etc.

%%%%%%%%%%%%%%%%%%%%%%%%%%%%%%%%%%%%%%%%%%%%%%%%%%%%%%%%%%%%%%%%%%%%%%%
% Symbols
\section{Symbols}
Free Pascal allows all characters, digits and some special ASCII symbols
in a Pascal source file.
\input{syntax/symbol.syn}
The following characters have a special meaning:
\begin{verbatim}
 + - * / = < > [ ] . , ( ) : ^ @ { } $ #
\end{verbatim}
and the following character pairs too:
\begin{verbatim}
<= >= := += -= *= /= (* *) (. .) //
\end{verbatim}
When used in a range specifier, the character pair \var{(.} is equivalent to
the left square bracket \var{[}. Likewise, the character pair \var{.)} is
equivalent to the right square bracket \var{]}.
When used for comment delimiters, the character pair \var{(*} is equivalent
to the  left brace \var{\{} and the character pair \var{*)} is equivalent
to the right brace \var{\}}.
These character pairs retain their normal meaning in string expressions.


%%%%%%%%%%%%%%%%%%%%%%%%%%%%%%%%%%%%%%%%%%%%%%%%%%%%%%%%%%%%%%%%%%%%%%%
% Comments
\section{Comments}
\fpc supports the use of nested comments. The following constructs are valid
comments:
\begin{verbatim}
(* This is an old style comment *)
{  This is a Turbo Pascal comment }
// This is a Delphi comment. All is ignored till the end of the line.
\end{verbatim}
The following are valid ways of nesting comments:
\begin{verbatim}
{ Comment 1 (* comment 2 *) }
(* Comment 1 { comment 2 } *)
{ comment 1 // Comment 2 }
(* comment 1 // Comment 2 *)
// comment 1 (* comment 2 *)
// comment 1 { comment 2 }
\end{verbatim}
The last two comments {\em must} be on one line. The following two will give
errors:
\begin{verbatim}
 // Valid comment { No longer valid comment !!
    }
\end{verbatim}
and
\begin{verbatim}
 // Valid comment (* No longer valid comment !!
    *)
\end{verbatim}
The compiler will react with a 'invalid character' error when it encounters
such constructs, regardless of the \var{-So} switch.

%%%%%%%%%%%%%%%%%%%%%%%%%%%%%%%%%%%%%%%%%%%%%%%%%%%%%%%%%%%%%%%%%%%%%%%
% Reserved words
\section{Reserved words}
Reserved words are part of the Pascal language, and cannot be redefined.
They will be denoted as {\sffamily\bfseries this} throughout the syntax
diagrams. Reserved words can be typed regardless of case, i.e. Pascal is
case insensitive.
We make a distinction between Turbo Pascal and Delphi reserved words, since
with the \var{-So} switch, only the Turbo Pascal reserved words are
recognised, and the Delphi ones can be redefined. By default, \fpc
recognises the Delphi reserved words.
\subsection{Turbo Pascal reserved words}
The following keywords exist in Turbo Pascal mode
\begin{multicols}{4}
\begin{verbatim}
absolute
and
array
asm
begin
break
case
const
constructor
continue
destructor
div
do
downto
else
end
file
for
function
goto
if
implementation
in
inherited
inline
interface
label
mod
nil
not
object
of
on
operator
or
packed
procedure
program
record
repeat
self
set
shl
shr
string
then
to
type
unit
until
uses
var
while
with
xor
\end{verbatim}
\end{multicols}
\subsection{Delphi reserved words}
The Delphi (II) reserved words are the same as the pascal ones, plus the
following ones:
\begin{multicols}{4}
\begin{verbatim}
as
class
except
exports
finalization
finally
initialization
is
library
on
property
raise
try
\end{verbatim}
\end{multicols}
\subsection{\fpc reserved words}
On top of the Turbo Pascal and Delphi reserved words, \fpc also considers
the following as reserved words:
\begin{multicols}{4}
\begin{verbatim}
dispose
exit
false
new
true
\end{verbatim}
\end{multicols}
\subsection{Modifiers}
The following is a list of all modifiers. Contrary to Delphi, \fpc doesn't
allow you to redefine these modifiers.
\begin{multicols}{4}
\begin{verbatim}
absolute
abstract
alias
assembler
cdecl
default
export
external
far
forward
index
name
near
override
pascal
popstack
private
protected
public
published
read
register
saveregisters
stdcall
virtual
write
\end{verbatim}
\end{multicols}
\begin{remark}
Predefined types such as \var{Byte}, \var{Boolean} and constants
such as \var{maxint} are {\em not} reserved words. They are
identifiers, declared in the system unit. This means that you can redefine
these types. You are, however, not encouraged to do this, as it will cause
a lot of confusion.
\end{remark}

%%%%%%%%%%%%%%%%%%%%%%%%%%%%%%%%%%%%%%%%%%%%%%%%%%%%%%%%%%%%%%%%%%%%%%%
% Identifiers
\section{Identifiers}
Identifiers denote constants, types, variables, procedures and functions,
units, and programs. All names of things that you define are identifiers.
An identifier consists of 255 significant characters (letters, digits and
the underscore character), from which the first must be an alphanumeric
character, or an underscore (\var{\_})
The following diagram gives the basic syntax for identifiers.
\input{syntax/identifier.syn}

%%%%%%%%%%%%%%%%%%%%%%%%%%%%%%%%%%%%%%%%%%%%%%%%%%%%%%%%%%%%%%%%%%%%%%%
% Numbers
\section{Numbers}
Numbers are denoted in decimal notation. Real (or decimal) numbers are
written using engeneering notation (e.g. \var{0.314E1}).
\fpc supports hexadecimal format the same way as Turbo Pascal does. To
specify a constant value in hexadecimal format, prepend it with a dollar
sign (\var{\$}). Thus, the hexadecimal \var{\$FF} equals 255 decimal.
In addition to the support for hexadecimal notation, \fpc also supports
binary notation. You can specify a binary number by preceding it with a
percent sign (\var{\%}). Thus, \var{255} can be specified in binary notation
as \var{\%11111111}.
The following diagrams show the syntax for numbers.
\input{syntax/numbers.syn}

%%%%%%%%%%%%%%%%%%%%%%%%%%%%%%%%%%%%%%%%%%%%%%%%%%%%%%%%%%%%%%%%%%%%%%%
% Labels
\section{Labels}
Labels can be digit sequences or identifiers.
\input{syntax/label.syn}
\begin{remark}
Note that you must specify the \var{-Sg} switch before you can use labels.
By default, \fpc doesn't support \var{label} and \var{goto} statements.
\end{remark}

%%%%%%%%%%%%%%%%%%%%%%%%%%%%%%%%%%%%%%%%%%%%%%%%%%%%%%%%%%%%%%%%%%%%%%%
% Character strings
\section{Character strings}
A character string (or string for short) is a sequence of zero or more
characters from the ASCII character set, enclosed by single quotes, and on 1
line of the program source.
A character set with nothing between the quotes (\var{'{}'}) is an empty
string.
\input{syntax/string.syn}
\chapter{Constants}
Just as in Turbo Pascal, \fpc supports both normal and typed constants.

%%%%%%%%%%%%%%%%%%%%%%%%%%%%%%%%%%%%%%%%%%%%%%%%%%%%%%%%%%%%%%%%%%%%%%%
% Ordinary constants
\section{Ordinary constants}
Ordinary constants declarations are not different from the Turbo Pascal or
Delphi implementation.
\input{syntax/const.syn}
The compiler must be able to evaluate the expression in a constant
declaration at compile time.  This means that most of the functions
in the Run-Time library cannot be used in a constant declaration.
Operators such as \var{+, -, *, /, not, and, or, div(), mod(), ord(), chr(),
sizeof} can be used, however. For more information on expressions, see
\seec{Expressions}.
You can only declare constants of the following types: \var{Ordinal types},
\var{Real types}, \var{Char}, and \var{String}.
The following are all valid constant declarations:
\begin{verbatim}
Const
  e = 2.7182818;  { Real type constant. }
  a = 2;          { Ordinal (Integer) type constant. }
  c = '4';        { Character type constant. }
  s = 'This is a constant string'; {String type constant.}
  s = chr(32)
  ls = SizeOf(Longint);
\end{verbatim}
Assigning a value to an ordinary constant is not permitted.
Thus, given the previous declaration, the following will result
in a compiler error:
\begin{verbatim}
  s := 'some other string';
\end{verbatim}

%%%%%%%%%%%%%%%%%%%%%%%%%%%%%%%%%%%%%%%%%%%%%%%%%%%%%%%%%%%%%%%%%%%%%%%
% Typed constants
\section{Typed constants}
Typed constants serve to provide a program with initialised variables.
Contrary to ordinary constants, they may be assigned to at run-time.
The difference with normal variables is that their value is initialised
when the program starts, whereas normal variables must be initialised
explicitly.
\input{syntax/tconst.syn}
Given the declaration:
\begin{verbatim}
Const
  S : String = 'This is a typed constant string';
\end{verbatim}
The following is a valid assignment:
\begin{verbatim}
 S := 'Result : '+Func;
\end{verbatim}
Where \var{Func} is a function that returns a \var{String}.
Typed constants also allow you to initialize arrays and records. For arrays,
the initial elements must be specified, surrounded by round brackets, and
separated by commas. The number of elements must be exactly the same as
the number of elements in the declaration of the type.
As an example:
\begin{verbatim}
Const
  tt : array [1..3] of string[20] = ('ikke', 'gij', 'hij');
  ti : array [1..3] of Longint = (1,2,3);
\end{verbatim}
For constant records, you should specify each element of the record, in the
form \var{Field : Value}, separated by commas, and surrounded by round
brackets.
As an example:
\begin{verbatim}
Type
  Point = record
    X,Y : Real
    end;
Const
  Origin : Point = (X:0.0; Y:0.0);
\end{verbatim}
The order of the fields in a constant record needs to be the same as in the type declaration,
otherwise you'll get a compile-time error.
%%%%%%%%%%%%%%%%%%%%%%%%%%%%%%%%%%%%%%%%%%%%%%%%%%%%%%%%%%%%%%%%%%%%%%%
% resource strings
\section{Resource strings}
\label{se:resourcestring}
A special kind of constant declaration part is the \var{Resourestring}
part. This part is like a \var{Const} section, but it only allows
to declare constant of type string. This part is only available in the
\var{Delphi} or \var{objfpc} mode.

The following is an example of a resourcestring definition:
\begin{verbatim}
Resourcestring

  FileMenu = '&File...';
  EditMenu = '&Edit...';
\end{verbatim}
All string constants defined in the resourcestring section are stored
in special tables, allowing to manipulate the values of the strings
at runtime with some special mechanisms.

Semantically, the strings are like constants; you cannot assign values to
them, except through the special mechanisms in the objpas unit. However,
you can use them in assignments or expressions as normal constants.
The main use of the resourcestring section is to provide an easy means
of internationalization.

More on the subject of resourcestrings can be found in the \progref, and
in the chapter on the \file{objpas} later in this manual.

%%%%%%%%%%%%%%%%%%%%%%%%%%%%%%%%%%%%%%%%%%%%%%%%%%%%%%%%%%%%%%%%%%%%%%%
% Types
%%%%%%%%%%%%%%%%%%%%%%%%%%%%%%%%%%%%%%%%%%%%%%%%%%%%%%%%%%%%%%%%%%%%%%%
\chapter{Types}
All variables have a type. \fpc supports the same basic types as Turbo
Pascal, with some extra types from Delphi.
You can declare your own types, which is in essence defining an identifier
that can be used to denote your custom type when declaring variables further
in the source code.
\input{syntax/typedecl.syn}
There are 7 major type classes :
\input{syntax/type.syn}
The last class, {\sffamily type identifier}, is just a means to give another
name to a type. This gives you a way to make types platform independent, by
only using your own types, and then defining these types for each platform
individually. The programmer that uses your units doesn't have to worry
about type size and so on. It also allows you to use shortcut names for
fully qualified type names. You can e.g. define \var{system.longint} as
\var{Olongint} and then redefine \var{longint}.

%%%%%%%%%%%%%%%%%%%%%%%%%%%%%%%%%%%%%%%%%%%%%%%%%%%%%%%%%%%%%%%%%%%%%%%
% Base types
\section{Base types}
The base or simple types of \fpc are the Delphi types.
We will discuss each separate.
\input{syntax/typesim.syn}
\subsection{Ordinal types}
With the exception of Real types, all base types are ordinal types.
Ordinal types have the following characteristics:
\begin{enumerate}
\item Ordinal types are countable and ordered, i.e. it is, in principle,
possible to start counting them one bye one, in a specified order.
This property allows the operation of functions as \seep{Inc}, \seef{Ord},
\seep{Dec}
on ordinal types to be defined.
\item Ordinal values have a smallest possible value. Trying to apply the
\seef{Pred} function on the smallest possible value will generate a range
check error if range checking is enabled.
\item Ordinal values have a largest possible value. Trying to apply the
\seef{Succ} function on the largest possible value will generate a range
check error if range checking is enabled.
\end{enumerate}
\subsubsection{Integers}
A list of pre-defined ordinal types is presented in \seet{ordinals}
\begin{FPCltable}{l}{Predefined ordinal types}{ordinals}
Name\\ \hline
Integer \\
Shortint \\
SmallInt \\
Longint \\
Int64 \\
Byte \\
Word \\
Cardinal \\
QWord \\
Boolean \\
ByteBool \\
LongBool \\
Char \\ \hline
\end{FPCltable}
The integer types, and their ranges and sizes, that are predefined in
\fpc are listed in \seet{integers}.
\begin{FPCltable}{lcr}{Predefined integer types}{integers}
Type & Range & Size in bytes \\ \hline
Byte & 0 .. 255 & 1 \\
Shortint & -128 .. 127 & 1\\
Integer & -32768 .. 32767 & 2\footnote{The integer type is redefined as
longint if you are in Delphi or ObjFPC mode, and then has size 4} \\
Word & 0 .. 65535 & 2 \\
Longint & -2147483648 .. 2147483647 & 4\\
Cardinal & 0..4294967295 & 4 \\
Int64 & -9223372036854775808 .. 9223372036854775807 & 8 \\
QWord & 0 .. 18446744073709551615 & 8 \\ \hline
\end{FPCltable}
\fpc does automatic type conversion in expressions where different kinds of
integer types are used.
\subsubsection{Boolean types}
\fpc supports the \var{Boolean} type, with its two pre-defined possible
values \var{True} and \var{False}. It also supports the \var{ByteBool},
\var{WordBool} and \var{LongBool} types. These are the only two values that can be
assigned to a \var{Boolean} type. Of course, any expression that resolves
to a \var{boolean} value, can also be assigned to a boolean type.
\begin{FPCltable}{lll}{Boolean types}{booleantypes}
Name & Size & Ord(True) \\ \hline
Boolean & 1 & 1 \\
ByteBool & 1 & Any nonzero value \\
WordBool & 2 & Any nonzero value \\
LongBool & 4 & Any nonzero value \\ \hline
\end{FPCltable}
Assuming \var{B} to be of type \var{Boolean}, the following are valid
assignments:
\begin{verbatim}
 B := True;
 B := False;
 B := 1<>2;  { Results in B := True }
\end{verbatim}
Boolean expressions are also used in conditions.

\begin{remark}
In \fpc, boolean expressions are always evaluated in such a
way that when the result is known, the rest of the expression will no longer
be evaluated (Called short-cut evaluation). In the following example, the function \var{Func} will never
be called, which may have strange side-effects.
\begin{verbatim}
 ...
 B := False;
 A := B and Func;
\end{verbatim}
Here \var{Func} is a function which returns a \var{Boolean} type.
\end{remark}

\begin{remark} The \var{WordBool}, \var{LongBool} and \var{ByteBool} types
were not supported by \fpc until version 0.99.6.
\end{remark}
\subsubsection{Enumeration types}
Enumeration types are supported in \fpc. On top of the Turbo Pascal
implementation, \fpc allows also a C-style extension of the
enumeration type, where a value is assigned to a particular element of
the enumeration list.
\input{syntax/typeenum.syn}
(see \seec{Expressions} for how to use expressions)
When using assigned enumerated types, the assigned elements must be in
ascending numerical order in the list, or the compiler will complain.
The expressions used in assigned enumerated elements must be known at
compile time.
So the following is a correct enumerated type declaration:
\begin{verbatim}
Type
  Direction = ( North, East, South, West );
\end{verbatim}
The C style enumeration type looks as follows:
\begin{verbatim}
Type
  EnumType = (one, two, three, forty := 40,fortyone);
\end{verbatim}
As a result, the ordinal number of \var{forty} is \var{40}, and not \var{3},
as it would be when the \var{':= 40'} wasn't present.
The ordinal value of \var{fortyone} is then {41}, and not \var{4}, as it
would be when the assignment wasn't present. After an assignment in an
enumerated definition the compiler adds 1 to the assigned value to assign to
the next enumerated value.
When specifying such an enumeration type, it is important to keep in mind
that you should keep the enumerated elements in ascending order. The
following will produce a compiler error:
\begin{verbatim}
Type
  EnumType = (one, two, three, forty := 40, thirty := 30);
\end{verbatim}
It is necessary to keep \var{forty} and \var{thirty} in the correct order.
When using enumeration types it is important to keep the following points
in mind:
\begin{enumerate}
\item You cannot use the \var{Pred} and \var{Succ} functions on
this kind of enumeration types. If you try to do that, you'll get a compiler
error.
\item Enumeration types are by default stored in 4 bytes. You can change
this behaviour with the \var{\{\$PACKENUM n\}} compiler directive, which
tells the compiler the minimal number of bytes to be used for enumeration
types.
For instance
\begin{verbatim}
Type
  LargeEnum = ( BigOne, BigTwo, BigThree );
{$PACKENUM 1}
  SmallEnum = ( one, two, three );
Var S : SmallEnum;
    L : LargeEnum;
begin
  WriteLn ('Small enum : ',SizeOf(S));
  WriteLn ('Large enum : ',SizeOf(L));
end.
\end{verbatim}
will, when run, print the following:
\begin{verbatim}
Small enum : 1
Large enum : 4
\end{verbatim}
\end{enumerate}
More information can be found in the \progref, in the compiler directives
section.
\subsubsection{Subrange types}
A subrange type is a range of values from an ordinal type (the {\em host}
type). To define a subrange type, one must specify it's limiting values: the
highest and lowest value of the type.
\input{syntax/typesubr.syn}
Some of the predefined \var{integer} types are defined as subrange types:
\begin{verbatim}
Type
  Longint  = $80000000..$7fffffff;
  Integer  = -32768..32767;
  shortint = -128..127;
  byte     = 0..255;
  Word     = 0..65535;
\end{verbatim}
But you can also define subrange types of enumeration types:
\begin{verbatim}
Type
  Days = (monday,tuesday,wednesday,thursday,friday,
          saturday,sunday);
  WorkDays = monday .. friday;
  WeekEnd = Saturday .. Sunday;
\end{verbatim}
\subsection{Real types}
\fpc uses the math coprocessor (or an emulation) for all its floating-point
calculations. The Real native type is processor dependant,
but it is either Single or Double. Only the IEEE floating point types are
supported, and these depend on the target processor and emulation options.
The true Turbo Pascal compatible types are listed in
\seet{Reals}.
 \begin{FPCltable}{lccr}{Supported Real types}{Reals}
Type & Range & Significant digits & Size\footnote{In Turbo Pascal.} \\ \hline
Single & 1.5E-45 .. 3.4E38 & 7-8 & 4 \\
Real & 5.0E-324 .. 1.7E308 & 15-16 & 8 \\
Double & 5.0E-324 .. 1.7E308 & 15-16 & 8 \\
Extended & 1.9E-4951 .. 1.1E4932 & 19-20 & 10\\
Comp & -2E64+1 .. 2E63-1 & 19-20 & 8  \\
\end{FPCltable}
Until version 0.9.1 of the compiler, all the \var{Real} types were mapped to
type \var{Double}, meaning that they all have size 8. The \seef{SizeOf} function
is your friend here. The \var{Real} type of turbo pascal is automatically
mapped to Double. The \var{Comp} type is, in effect, a 64-bit integer.

%%%%%%%%%%%%%%%%%%%%%%%%%%%%%%%%%%%%%%%%%%%%%%%%%%%%%%%%%%%%%%%%%%%%%%%
% Character types
\section{Character types}
\subsection{Char}
\fpc supports the type \var{Char}. A \var{Char} is exactly 1 byte in
size, and contains one character.
You can specify a character constant by enclosing the character in single
quotes, as follows : 'a' or 'A' are both character constants.
You can also specify a character by their ASCII
value, by preceding the ASCII value with the number symbol (\#). For example
specifying \var{\#65} would be the same as \var{'A'}.
Also, the caret character (\verb+^+) can be used in combination with a letter to
specify a character with ASCII value less than 27. Thus \verb+^G+ equals
\var{\#7} (G is the seventh letter in the alphabet.)
If you want to represent the single quote character, type it two times
successively, thus \var{''''} represents the single quote character.

\subsection{Strings}
\fpc supports the \var{String} type as it is defined in Turbo Pascal and
it supports ansistrings as in Delphi.
To declare a variable as a string, use the following type specification:
\input{syntax/sstring.syn}

The meaning of a string declaration statement is interpreted differently
depending on the \var{\{\$H\}} switch. The above declaration can declare an
ansistrng or a short string.

Whatever the actual type, ansistrings and short strings can be used
interchangeably. The compiler always takes care of the necessary type
coversions. Note, however, that the result of an expression that contains
ansistrings and short strings will always be an ansistring.

\subsection{Short strings}

A string declaration declares a short string in the following cases:

\begin{enumerate}
\item If the switch is off: \var{\{\$H-\}}, the string declaration
will always be a short string declaration.
\item If the switch is on \var{\{\$H+\}}, and there is a length
specifier, the declaration is a short string declaration.
\end{enumerate}

The predefined type \var{ShortString} is defined as a string of length 255:
\begin{verbatim}
 ShortString = String[255];
\end{verbatim}

For short strings \fpc reserves \var{Size+1} bytes for the string \var{S},
and in the zeroeth element of the string (\var{S[0]}) it will store the
length of the variable.
If you don't specify the size of the string, \var{255} is taken as a
default.
For example in
\begin{verbatim}
{$H-}

Type
   NameString = String[10];
   StreetString = String;
\end{verbatim}
\var{NameString} can contain maximum 10 characters. While
\var{StreetString} can contain 255 characters. The sizes of these variables
are, respectively, 11 and 256 bytes.

\subsection{Ansistrings}

If the \var{\{\$H\}} switch is on, then a string definition that doesn't
contain a length specifier, will be regarded as an ansistring.

Ansistrings are strings that have no length limit. They are reference
counted. Internally, an ansistring is treated as a pointer.

If the string is empty (\var{''}), then the pointer is nil.
If the string is not empty, then the pointer points to a structure in
heap memory that looks as in \seet{ansistrings}.

\begin{FPCltable}{rl}{AnsiString memory structure}{ansistrings}
Offset & Contains \\ \hline
-12  & Longint with maximum string size. \\
-8   & Longint with actual string size.\\
-4   & Longint with reference count.\\
0    & Actual string, null-terminated. \\ \hline
\end{FPCltable}

Because of this structure, it is possible to typecast an ansistring to a
pchar. If the string is empty (so the pointer is nil) then the compiler
makes sure that the typecasted pchar will point to a null byte.

AnsiStrings can be unlimited in length. Since the length is stored,
the length of an ansistring is available immediatly, providing for fast
access.

Assigning one ansistring to another doesn't involve moving the actual
string. A statement
\begin{verbatim}
  S2:=S1;
\end{verbatim}
results in the reference count of \var{S2} being decreased by one,
The referece count of \var{S1} is increased by one, and finally \var{S1}
(as a pointer) is copied to \var{S2}. This is a significant speed-up in
your code.

If a reference count reaches zero, then the memory occupied by the
string is deallocated automatically, so no memory leaks arise.

When an ansistring is declared, the \fpc compiler initially
allocates just memory for a pointer, not more. This pointer is guaranteed
to be nil, meaning that the string is initially empty. This is
true for local, global or part of a structure (arrays, records or objects).

This does introduce an overhead. For instance, declaring
\begin{verbatim}
Var
  A : Array[1..100000] of string;
\end{verbatim}
Will copy 100,000 times \var{nil} into \var{A}. When \var{A} goes out of scope, then
the 100,000 strings will be dereferenced one by one. All this happens
invisibly for the programmer, but when considering performance issues,
this is important.

Memory will be allocated only when the string is assigned a value.
If the string goes out of scope, then it is automatically dereferenced.

If you assign a value to a character of a string that has a reference count
greater than 1, such as in the following
statements:
\begin{verbatim}
  S:=T;  { reference count for S and T is now 2 }
  S[I]:='@';
\end{verbatim}
then a copy of the string is created before the assignment. This is known
as {\em copy-on-write} semantics.

It is impossible to access the length of an ansistring by referring to
the zeroeth character. The following statement will generate a compiler
error if S is an ansistring:
\begin{verbatim}
  Len:=S[0];
\end{verbatim}
Instead, you must use the \seef{Length} function to get the length of a
string.

To set the length of an ansistring, you can use the \seep{SetLength}
function.
Constant ansistrings have a reference count of -1 and are treated specially.

Ansistrings are converted to short strings by the compiler if needed,
this means that you can mix the use of ansistrings and short strings
without problems.

You can typecast ansistrings to \var{PChar} or \var{Pointer} types:
\begin{verbatim}
Var P : Pointer;
    PC : PChar;
    S : AnsiString;

begin
  S :='This is an ansistring';
  PC:=Pchar(S);
  P :=Pointer(S);
\end{verbatim}
There is a difference between the two typecasts. If you typecast an empty
ansistring to a pointer, the pointer wil be \var{Nil}. If you typecast an
empty ansistring to a \var{PChar}, then the result will be a pointer to a
zero byte (an empty string).

The result of such a typecast must be used with care. In general, it is best
to consider the result of such a typecast as read-only, i.e. suitable for
passing to a procedure that needs a constant pchar argument.

It is therefore NOT advisable to typecast one of the following:
\begin{enumerate}
\item expressions.
\item strings that have reference count larger than 0.
(call uniquestring if you want to ensure a string has reference count 1)
\end{enumerate}
\subsection{Constant strings}

To specify a constant string, you enclose the string in single-quotes, just
as a \var{Char} type, only now you can have more than one character.
Given that \var{S} is of type \var{String}, the following are valid assignments:
\begin{verbatim}
S := 'This is a string.';
S := 'One'+', Two'+', Three';
S := 'This isn''t difficult !';
S := 'This is a weird character : '#145' !';
\end{verbatim}
As you can see, the single quote character is represented by 2 single-quote
characters next to each other. Strange characters can be specified by their
ASCII value.
The example shows also that you can add two strings. The resulting string is
just the concatenation of the first with the second string, without spaces in
between them. Strings can not be substracted, however.

Whether the constant string is stored as an ansistring or a short string
depends on the settings of the \var{\{\$H\}} switch.


\subsection{PChar}
\fpc supports the Delphi implementation of the \var{PChar} type. \var{PChar}
is defined as a pointer to a \var{Char} type, but allows additional
operations.
The \var{PChar} type can be understood best as the Pascal equivalent of a
C-style null-terminated string, i.e. a variable of type \var{PChar} is a
pointer that points to an array of type \var{Char}, which is ended by a
null-character (\var{\#0}).
\fpc supports initializing of \var{PChar} typed constants, or a direct
assignment. For example, the following pieces of code are equivalent:
\begin{verbatim}
program one;
var p : PChar;
begin
  P := 'This is a null-terminated string.';
  WriteLn (P);
end.
\end{verbatim}
Results in the same as
\begin{verbatim}
program two;
const P : PChar = 'This is a null-terminated string.'
begin
  WriteLn (P);
end.
\end{verbatim}
These examples also show that it is possible to write {\em the contents} of
the string to a file of type \var{Text}.
The \seestrings unit contains procedures and functions that manipulate the
\var{PChar} type as you can do it in C.
Since it is equivalent to a pointer to a type \var{Char} variable, it  is
also possible to do the following:
\begin{verbatim}
Program three;
Var S : String[30];
    P : PChar;
begin
  S := 'This is a null-terminated string.'#0;
  P := @S[1];
  WriteLn (P);
end.
\end{verbatim}
This will have the same result as the previous two examples.
You cannot add null-terminated strings as you can do with normal Pascal
strings. If you want to concatenate two \var{PChar} strings, you will need
to use the unit \seestrings.
However, it is possible to do some pointer arithmetic. You can use the
operators \var{+} and \var{-} to do operations on \var{PChar} pointers.
In \seet{PCharMath}, \var{P} and \var{Q} are of type \var{PChar}, and
\var{I} is of type \var{Longint}.
\begin{FPCltable}{lr}{\var{PChar} pointer arithmetic}{PCharMath}
Operation & Result \\ \hline
\var{P + I} & Adds \var{I} to the address pointed to by \var{P}. \\
\var{I + P} & Adds \var{I} to the address pointed to by \var{P}. \\
\var{P - I} & Substracts \var{I} from the address pointed to by \var{P}. \\
\var{P - Q} & Returns, as an integer, the distance between 2 addresses \\
 & (or the number of characters between \var{P} and \var{Q}) \\
\hline
\end{FPCltable}

%%%%%%%%%%%%%%%%%%%%%%%%%%%%%%%%%%%%%%%%%%%%%%%%%%%%%%%%%%%%%%%%%%%%%%%
% Structured Types
\section{Structured Types}
A structured type is a type that can hold multiple values in one variable.
Stuctured types can be nested to unlimited levels.
\input{syntax/typestru.syn}
Unlike Delphi, \fpc does not support the keyword \var{Packed} for all
structured types, as can be seen in the syntax diagram. It will be mentioned
when a type supports the \var{packed} keyword.
In the following, each of the possible structured types is discussed.
\subsection{Arrays}
\fpc supports arrays as in Turbo Pascal, multi-dimensional arrays
and packed arrays are also supported:
\input{syntax/typearr.syn}
The following is a valid array declaration:
\begin{verbatim}
Type
  RealArray = Array [1..100] of Real;
\end{verbatim}
As in Turbo Pascal, if the array component type is in itself an array, it is
possible to combine the two arrays into one multi-dimensional array. The
following declaration:
\begin{verbatim}
Type
   APoints = array[1..100] of Array[1..3] of Real;
\end{verbatim}
is equivalent to the following declaration:
\begin{verbatim}
Type
   APoints = array[1..100,1..3] of Real;
\end{verbatim}
The functions \seef{High} and \seef{Low} return the high and low bounds of
the leftmost index type of the array. In the above case, this would be 100
and 1.
\subsection{Record types}
\fpc supports fixed records and records with variant parts.
The syntax diagram for a record type is
\input{syntax/typerec.syn}
So the following are valid record types declarations:
\begin{verbatim}
Type
  Point = Record
          X,Y,Z : Real;
          end;
  RPoint = Record
          Case Boolean of
          False : (X,Y,Z : Real);
          True : (R,theta,phi : Real);
          end;
  BetterRPoint = Record
          Case UsePolar : Boolean of
          False : (X,Y,Z : Real);
          True : (R,theta,phi : Real);
          end;
\end{verbatim}
The variant part must be last in the record. The optional identifier in the
case statement serves to access the tag field value, which otherwise would
be invisible to the programmer. It can be used to see which variant is
active at a certain time. In effect, it introduces a new field in the
record.
\begin{remark}
It is possible to nest variant parts, as in:
\begin{verbatim}
Type
  MyRec = Record
          X : Longint;
          Case byte of
            2 : (Y : Longint;
                 case byte of
                 3 : (Z : Longint);
                 );
          end;
\end{verbatim}
\end{remark}
The size of a record is the sum of the sizes of its fields, each size of a
field is rounded up to a power of two. If the record contains a variant part, the size
of the variant part is the size of the biggest variant, plus the size of the
tag field type {\em if an identifier was declared for it}. Here also, the size of
each part is first rounded up to two. So in the above example,
\seef{SizeOf} would return 24 for \var{Point}, 24 for \var{RPoint} and
26 for \var{BetterRPoint}. For \var{MyRec}, the value would be 12.
If you want to read a typed file with records, produced by
a Turbo Pascal program, then chances are that you will not succeed in
reading that file correctly.
The reason for this is that by default, elements of a record are aligned at
2-byte boundaries, for performance reasons. This default behaviour can be
changed with the \var{\{\$PackRecords n\}} switch. Possible values for
\var{n} are 1, 2, 4, 16 or \var{Default}.
This switch tells the compiler to align elements of a record or object or
class that have size larger than \var{n} on \var{n} byte boundaries.
Elements that have size smaller or equal than \var{n} are aligned on
natural boundaries, i.e. to the first power of two that is larger than or
equal to the size of the record element.
The keyword \var{Default} selects the default value for the platform
you're working on (currently, this is 2 on all platforms)
Take a look at the following program:
\begin{verbatim}
Program PackRecordsDemo;
type
   {$PackRecords 2}
     Trec1 = Record
       A : byte;
       B : Word;
     end;

     {$PackRecords 1}
     Trec2 = Record
       A : Byte;
       B : Word;
       end;
   {$PackRecords 2}
     Trec3 = Record
       A,B : byte;
     end;

    {$PackRecords 1}
     Trec4 = Record
       A,B : Byte;
       end;
   {$PackRecords 4}
     Trec5 = Record
       A : Byte;
       B : Array[1..3] of byte;
       C : byte;
     end;

     {$PackRecords 8}
     Trec6 = Record
       A : Byte;
       B : Array[1..3] of byte;
       C : byte;
       end;
   {$PackRecords 4}
     Trec7 = Record
       A : Byte;
       B : Array[1..7] of byte;
       C : byte;
     end;

     {$PackRecords 8}
     Trec8 = Record
       A : Byte;
       B : Array[1..7] of byte;
       C : byte;
       end;
Var rec1 : Trec1;
    rec2 : Trec2;
    rec3 : TRec3;
    rec4 : TRec4;
    rec5 : Trec5;
    rec6 : TRec6;
    rec7 : TRec7;
    rec8 : TRec8;

begin
  Write ('Size Trec1 : ',SizeOf(Trec1));
  Writeln (' Offset B : ',Longint(@rec1.B)-Longint(@rec1));
  Write ('Size Trec2 : ',SizeOf(Trec2));
  Writeln (' Offset B : ',Longint(@rec2.B)-Longint(@rec2));
  Write ('Size Trec3 : ',SizeOf(Trec3));
  Writeln (' Offset B : ',Longint(@rec3.B)-Longint(@rec3));
  Write ('Size Trec4 : ',SizeOf(Trec4));
  Writeln (' Offset B : ',Longint(@rec4.B)-Longint(@rec4));
  Write ('Size Trec5 : ',SizeOf(Trec5));
  Writeln (' Offset B : ',Longint(@rec5.B)-Longint(@rec5),
           ' Offset C : ',Longint(@rec5.C)-Longint(@rec5));
  Write ('Size Trec6 : ',SizeOf(Trec6));
  Writeln (' Offset B : ',Longint(@rec6.B)-Longint(@rec6),
           ' Offset C : ',Longint(@rec6.C)-Longint(@rec6));
  Write ('Size Trec7 : ',SizeOf(Trec7));
  Writeln (' Offset B : ',Longint(@rec7.B)-Longint(@rec7),
           ' Offset C : ',Longint(@rec7.C)-Longint(@rec7));
  Write ('Size Trec8 : ',SizeOf(Trec8));
  Writeln (' Offset B : ',Longint(@rec8.B)-Longint(@rec8),
           ' Offset C : ',Longint(@rec8.C)-Longint(@rec8));
end.
\end{verbatim}
The output of this program will be :
\begin{verbatim}
Size Trec1 : 4 Offset B : 2
Size Trec2 : 3 Offset B : 1
Size Trec3 : 2 Offset B : 1
Size Trec4 : 2 Offset B : 1
Size Trec5 : 8 Offset B : 4 Offset C : 7
Size Trec6 : 8 Offset B : 4 Offset C : 7
Size Trec7 : 12 Offset B : 4 Offset C : 11
Size Trec8 : 16 Offset B : 8 Offset C : 15
\end{verbatim}
And this is as expected. In \var{Trec1}, since \var{B} has size 2, it is
aligned on a 2 byte boundary, thus leaving an empty byte between \var{A}
and \var{B}, and making the total size 4. In \var{Trec2}, \var{B} is aligned
on a 1-byte boundary, right after \var{A}, hence, the total size of the
record is 3.
For \var{Trec3}, the sizes of \var{A,B} are 1, and hence they are aligned on 1
byte boundaries. The same is true for \var{Trec4}.
For \var{Trec5}, since the size of B -- 3 -- is smaller than 4, \var{B} will
be on a  4-byte boundary, as this is the first power of two that is
larger than it's size. The same holds for \var{Trec6}.
For \var{Trec7}, \var{B} is aligned on a 4 byte boundary, since it's size --
7 -- is larger than 4. However, in \var{Trec8}, it is aligned on a 8-byte
boundary, since 8 is the first power of two that is greater than 7, thus
making the total size of the record 16.
As from version 0.9.3, \fpc supports also the 'packed record', this is a
record where all the elements are byte-aligned.
Thus the two following declarations are equivalent:
\begin{verbatim}
     {$PackRecords 1}
     Trec2 = Record
       A : Byte;
       B : Word;
       end;
     {$PackRecords 2}
\end{verbatim}
and
\begin{verbatim}
     Trec2 = Packed Record
       A : Byte;
       B : Word;
       end;
\end{verbatim}
Note the \var{\{\$PackRecords 2\}} after the first declaration !
\subsection{Set types}
\fpc supports the set types as in Turbo Pascal. The prototype of a set
declaration is:
\input{syntax/typeset.syn}
Each of the elements of \var{SetType} must be of type \var{TargetType}.
\var{TargetType} can be any ordinal type with a range between \var{0} and
\var{255}. A set can contain maximally \var{255} elements.
The following are valid set declaration:
\begin{verbatim}
Type
    Junk = Set of Char;

    Days = (Mon, Tue, Wed, Thu, Fri, Sat, Sun);
    WorkDays : Set of days;
\end{verbatim}
Given this set declarations, the following assignment is legal:
\begin{verbatim}
WorkDays := [ Mon, Tue, Wed, Thu, Fri];
\end{verbatim}
The operators and functions for manipulations of sets are listed in
\seet{SetOps}.
\begin{FPCltable}{lr}{Set Manipulation operators}{SetOps}
Operation & Operator \\ \hline
Union & + \\
Difference & - \\
Intersection & * \\
Add element & \var{include} \\
Delete element & \var{exclude} \\ \hline
\end{FPCltable}
You can compare two sets with the \var{<>} and \var{=} operators, but not
(yet) with the \var{<} and \var{>} operators.
As of compiler version 0.9.5, the compiler stores small sets (less than 32
elements) in a Longint, if the type range allows it. This allows for faster
processing and decreases program size. Otherwise, sets are stored in 32
bytes.
\subsection{File types}
File types are types that store a sequence of some base type, which can be
any type except another file type. It can contain (in principle) an infinite
number of elements.
File types are used commonly to store data on disk. Nothing stops you,
however, from writing a file driver that stores it's data in memory.
Here is the type declaration for a file type:
\input{syntax/typefil.syn}
If no type identifier is given, then the file is an untyped file; it can be
considered as equivalent to a file of bytes. Untyped files require special
commands to act on them (see \seep{Blockread}, \seep{Blockwrite}).
The following declaration declares a file of records:
\begin{verbatim}
Type
   Point = Record
     X,Y,Z : real;
     end;
   PointFile = File of Point;
\end{verbatim}
Internally, files are represented by the \var{FileRec} record, which is
declared in the DOS unit.

A special file type is the \var{Text} file type, represented by the
\var{TextRec} record. A file of type \var{Text} uses special input-output
routines.

%%%%%%%%%%%%%%%%%%%%%%%%%%%%%%%%%%%%%%%%%%%%%%%%%%%%%%%%%%%%%%%%%%%%%%%
% Pointers
\section{Pointers}
\fpc supports the use of pointers. A variable of the pointer type
contains an address in memory, where the data of another variable may be
stored.
\input{syntax/typepoin.syn}
As can be seen from this diagram, pointers are typed, which means that
they point to a particular kind of data. The type of this data must be
known at compile time.
Dereferencing the pointer (denoted by adding \var{\^{}} after the variable
name) behaves then like a variable. This variable has the type declared in
the pointer declaration, and the variable is stored in the address that is
pointed to by the pointer variable.
Consider the following example:
\begin{verbatim}
Program pointers;
type
  Buffer = String[255];
  BufPtr = ^Buffer;
Var B  : Buffer;
    BP : BufPtr;
    PP : Pointer;
etc..
\end{verbatim}
In this example, \var{BP} {\em is a pointer to} a \var{Buffer} type; while \var{B}
{\em is} a variable of type \var{Buffer}. \var{B} takes 256 bytes memory,
and \var{BP} only takes 4 bytes of memory (enough to keep an adress in
memory).
\begin{remark} \fpc treats pointers much the same way as C does. This means
that you can treat a pointer to some type as being an array of this type.
The pointer then points to the zeroeth element of this array. Thus the
following pointer declaration
\begin{verbatim}
Var p : ^Longint;
\end{verbatim}
Can be considered equivalent to the following array declaration:
\begin{verbatim}
Var p : array[0..Infinity] of Longint;
\end{verbatim}
The difference is that the former declaration allocates memory for the
pointer only (not for the array), and the second declaration allocates
memory for the entire array. If you use the former, you must allocate memory
yourself, using the \seep{Getmem} function.
The reference \var{P\^{}} is then the same as \var{p[0]}. The following program
illustrates this maybe more clear:
\begin{verbatim}
program PointerArray;
var i : Longint;
    p : ^Longint;
    pp : array[0..100] of Longint;
begin
  for i := 0 to 100 do pp[i] := i; { Fill array }
  p := @pp[0];                     { Let p point to pp }
  for i := 0 to 100 do
    if p[i]<>pp[i] then
      WriteLn ('Ohoh, problem !')
end.
\end{verbatim}
\end{remark}
\fpc supports pointer arithmetic as C does. This means that, if \var{P} is a
typed pointer, the instructions
\begin{verbatim}
Inc(P);
Dec(P);
\end{verbatim}
Will increase, respectively descrease the address the pointer points to
with the size of the type \var{P} is a pointer to. For example
\begin{verbatim}
Var P : ^Longint;
...
 Inc (p);
\end{verbatim}
will increase \var{P} with 4.
You can also use normal arithmetic operators on pointers, that is, the
following are valid pointer arithmetic operations:
\begin{verbatim}
var  p1,p2 : ^Longint;
     L : Longint;
begin
  P1 := @P2;
  P2 := @L;
  L := P1-P2;
  P1 := P1-4;
  P2 := P2+4;
end.
\end{verbatim}
Here, the value that is added or substracted is {\em not} multiplied by the
size of the type the pointer points to.

%%%%%%%%%%%%%%%%%%%%%%%%%%%%%%%%%%%%%%%%%%%%%%%%%%%%%%%%%%%%%%%%%%%%%%%
% Procedural types
\section{Procedural types}
\fpc has support for procedural types, although it differs a little from
the Turbo Pascal implementation of them. The type declaration remains the
same, as can be seen in the following syntax diagram:
\input{syntax/typeproc.syn}
For a description of formal parameter lists, see \seec{Procedures}.
The two following examples are valid type declarations:
\begin{verbatim}
Type TOneArg = Procedure (Var X : integer);
     TNoArg = Function : Real;
var proc : TOneArg;
    func : TNoArg;
\end{verbatim}
One can assign the following values to a procedural type variable:
\begin{enumerate}
\item \var{Nil}, for both normal procedure pointers and method pointers.
\item A variable reference of a procedural type, i.e. another variable of
the same type.
\item A global procedure or function address, with matching function or
procedure header and calling convention.
\item A method address.
\end{enumerate}
Given these declarations, the following assignments are valid:
\begin{verbatim}
Procedure printit (Var X : Integer);
begin
  WriteLn (x);
end;
...
Proc := @printit;
Func := @Pi;
\end{verbatim}
From this example, the difference with Turbo Pascal is clear: In Turbo
Pascal it isn't necessary to use the address operator (\var{@})
when assigning a procedural type variable, whereas in \fpc it is required
(unless you use the \var{-So} switch, in which case you can drop the address
operator.)
\begin{remark} The modifiers concerning the calling conventions (\var{cdecl},
\var{pascal}, \var{stdcall} and \var{popstack} stick to the declaration;
i.e. the following code would give an error:
\begin{verbatim}
Type TOneArgCcall = Procedure (Var X : integer);cdecl;
var proc : TOneArgCcall;
Procedure printit (Var X : Integer);
begin
  WriteLn (x);
end;
begin
Proc := @printit;
end.
\end{verbatim}
Because the \var{TOneArgCcall} type is a procedure that uses the cdecl
calling convention.
\end{remark}

%%%%%%%%%%%%%%%%%%%%%%%%%%%%%%%%%%%%%%%%%%%%%%%%%%%%%%%%%%%%%%%%%%%%%%%
% Objects

%%%%%%%%%%%%%%%%%%%%%%%%%%%%%%%%%%%%%%%%%%%%%%%%%%%%%%%%%%%%%%%%%%%%%%%
\chapter{Objects}
\label{ch:Objects}

%%%%%%%%%%%%%%%%%%%%%%%%%%%%%%%%%%%%%%%%%%%%%%%%%%%%%%%%%%%%%%%%%%%%%%%
% Declaration
\section{Declaration}
\fpc supports object oriented programming. In fact, most  of the compiler is
written using objects. Here we present some technical questions regarding
object oriented programming in \fpc.
Objects should be treated as a special kind of record. The record contains
all the fields that are declared in the objects definition, and pointers
to the methods that are associated to the objects' type.

An object is declared just as you would declare a record; except that you
can now declare procedures and functions as if they were part of the record.
Objects can ''inherit'' fields and methods from ''parent'' objects. This means
that you can use these fields and methods as if they were included in the
objects you declared as a ''child'' object.

Furthermore, you can declare fields, procedures and functions as \var{public}
or \var{private}. By default, fields and methods are \var{public}, and are
exported outside the current unit. Fields or methods that are declared
\var{private} are only accessible in the current unit.
The prototype declaration of an object is as follows:
\input{syntax/typeobj.syn}
As you can see, you can repeat as many \var{private} and \var{public}
blocks as you want.
\var{Method definitions} are normal function or procedure declarations.
You cannot put fields after methods in the same block, i.e. the following
will generate an error when compiling:
\begin{verbatim}
Type MyObj = Object
       Procedure Doit;
       Field : Longint;
     end;
\end{verbatim}
But the following will be accepted:
\begin{verbatim}
Type MyObj = Object
      Public
       Procedure Doit;
      Private
       Field : Longint;
     end;
\end{verbatim}
because the field is in a different section.

\begin{remark}
\fpc also supports the packed object. This is the same as an object, only
the elements (fields) of the object are byte-aligned, just as in the packed
record.
The declaration of a packed object is similar to the declaration
of a packed record :
\begin{verbatim}
Type
  TObj = packed object;
   Constructor init;
   ...
   end;
  Pobj = ^TObj;
Var PP : Pobj;
\end{verbatim}
Similarly, the \var{\{\$PackRecords \}} directive acts on objects as well.
\end{remark}
%%%%%%%%%%%%%%%%%%%%%%%%%%%%%%%%%%%%%%%%%%%%%%%%%%%%%%%%%%%%%%%%%%%%%%%
% Fields
\section{Fields}
Object Fields are like record fields. They are accessed in the same way as
you would access a record field : by using a qualified identifier. Given the
following declaration:
\begin{verbatim}
Type TAnObject = Object
       AField : Longint;
       Procedure AMethod;
       end;
Var AnObject : TAnObject;
\end{verbatim}
then the following would be a valid assignment:
\begin{verbatim}
  AnObject.AField := 0;
\end{verbatim}
Inside methods, fields can be accessed using the short identifier:
\begin{verbatim}
Procedure TAnObject.AMethod;
begin
  ...
  AField := 0;
  ...
end;
\end{verbatim}
Or, one can use the \var{self} identifier. The \var{self} identifier refers
to the current instance of the object:
\begin{verbatim}
Procedure TAnObject.AMethod;
begin
  ...
  Self.AField := 0;
  ...
end;
\end{verbatim}
You cannot access fields that are in a private section of an object from
outside the objects' methods. If you do, the compiler will complain about
an unknown identifier.
It is also possible to use the \var{with} statement with an object instance:
\begin{verbatim}
With AnObject do
  begin
  Afield := 12;
  AMethod;
  end;
\end{verbatim}
In this example, between the \var{begin} and \var{end}, it is as if
\var{AnObject} was prepended to the \var{Afield} and \var{Amethod}
identifiers. More about this in \sees{With}

%%%%%%%%%%%%%%%%%%%%%%%%%%%%%%%%%%%%%%%%%%%%%%%%%%%%%%%%%%%%%%%%%%%%%%%
% Constructors and destructors
\section{Constructors and destructors }
\label{se:constructdestruct}
As can be seen in the syntax diagram for an object declaration, \fpc supports
constructors and destructors. You are responsible for calling the
constructor and the destructor explicitly when using objects.
The declaration of a constructor or destructor is as follows:
\input{syntax/construct.syn}
A constructor/destructor pair is {\em required} if you use virtual methods.
In the declaration of the object type, you should use a simple identifier
for the name of the constuctor or destructor. When you implement the
constructor or destructor, you should use a qulified method identifier,
i.e. an identifier of the form \var{objectidentifier.methodidentifier}.
\fpc supports also the extended syntax of the \var{New} and \var{Dispose}
procedures. In case you want to allocate a dynamic variable of an object
type, you can specify the constructor's name in the call to \var{New}.
The \var{New} is implemented as a function which returns a pointer to the
instantiated object. Consider the following declarations:
\begin{verbatim}
Type
  TObj = object;
   Constructor init;
   ...
   end;
  Pobj = ^TObj;
Var PP : Pobj;
\end{verbatim}
Then the following 3 calls are equivalent:
\begin{verbatim}
 pp := new (Pobj,Init);
\end{verbatim}
and
\begin{verbatim}
  new(pp,init);
\end{verbatim}
and also
\begin{verbatim}
  new (pp);
  pp^.init;
\end{verbatim}
In the last case, the compiler will issue a warning that you should use the
extended syntax of \var{new} and \var{dispose} to generate instances of an
object. You can ignore this warning, but it's better programming practice to
use the extended syntax to create instances of an object.
Similarly, the \var{Dispose} procedure accepts the name of a destructor. The
destructor will then be called, before removing the object from the heap.

In view of the compiler warning remark, the following chapter presents the
Delphi approach to object-oriented programming, and may be considered a
more natural way of object-oriented programming.

%%%%%%%%%%%%%%%%%%%%%%%%%%%%%%%%%%%%%%%%%%%%%%%%%%%%%%%%%%%%%%%%%%%%%%%
% Methods
\section{Methods}
Object methods are just like ordinary procedures or functions, only they
have an implicit extra parameter : \var{self}. Self points to the object
with which the method was invoked.
When implementing methods, the fully qualified identifier must be given
in the function header. When declaring methods, a normal identifier must be
given.

%%%%%%%%%%%%%%%%%%%%%%%%%%%%%%%%%%%%%%%%%%%%%%%%%%%%%%%%%%%%%%%%%%%%%%%
% Method invocation
\section{Method invocation}
Methods are called just as normal procedures are called, only they have an
object instance identifier prepended to them (see also \seec{Statements}).
To determine which method is called, it is necessary to know the type of
the method. We treat the different types in what follows.
\subsubsection{Static methods}
Static methods are methods that have been declared without a \var{abstract}
or \var{virtual} keyword. When calling a static method, the declared (i.e.
compile time) method of the object is used.
For example, consider the following declarations:
\begin{verbatim}
Type
  TParent = Object
    ...
    procedure Doit;
    ...
    end;
  PParent = ^TParent;
  TChild = Object(TParent)
    ...
    procedure Doit;
    ...
    end;
  PChild = ^TChild;
\end{verbatim}
As it is visible, both the parent and child objects have a method called
\var{Doit}. Consider now the following declarations and calls:
\begin{verbatim}
Var ParentA,ParentB : PParent;
    Child           : PChild;
   ParentA := New(PParent,Init);
   ParentB := New(PChild,Init);
   Child := New(PChild,Init);
   ParentA^.Doit;
   ParentB^.Doit;
   Child^.Doit;
\end{verbatim}
Of the three invocations of \var{Doit}, only the last one will call
\var{TChild.Doit}, the other two calls will call \var{TParent.Doit}.
This is because for static methods, the compiler determines at compile
time which method should be called. Since \var{ParentB} is of type
\var{TParent}, the compiler decides that it must be called with
\var{TParent.Doit}, even though it will be created as a \var{TChild}.
There may be times when you want the method that is actually called to
depend on the actual type of the object at run-time. If so, the method
cannot be a static method, but must be a virtual method.
\subsubsection{Virtual methods}
To remedy the situation in the previous section, \var{virtual} methods are
created. This is simply done by appending the method declaration with the
\var{virtual} modifier.
Going back to the previous example, consider the following alternative
declaration:
\begin{verbatim}
Type
  TParent = Object
    ...
    procedure Doit;virtual;
    ...
    end;
  PParent = ^TParent;
  TChild = Object(TParent)
    ...
    procedure Doit;virtual;
    ...
    end;
  PChild = ^TChild;
\end{verbatim}
As it is visible, both the parent and child objects have a method called
\var{Doit}. Consider now the following declarations and calls :
\begin{verbatim}
Var ParentA,ParentB : PParent;
    Child           : PChild;
   ParentA := New(PParent,Init);
   ParentB := New(PChild,Init);
   Child := New(PChild,Init);
   ParentA^.Doit;
   ParentB^.Doit;
   Child^.Doit;
\end{verbatim}
Now, different methods will be called, depending on the actual run-time type
of the object. For \var{ParentA}, nothing changes, since it is created as
a \var{TParent} instance. For \var{Child}, the situation also doesn't
change: it is again created as an instance of \var{TChild}.
For \var{ParentB} however, the situation does change: Even though it was
declared as a \var{TParent}, it is created as an instance of \var{TChild}.
Now, when the program runs, before calling \var{Doit}, the program
checks what the actual type of \var{ParentB} is, and only then decides which
method must be called. Seeing that \var{ParentB} is of type \var{TChild},
\var{TChild.Doit} will be called.
The code for this run-time checking of the actual type of an object is
inserted by the compiler at compile time.
The \var{TChild.Doit} is said to {\em override} the \var{TParent.Doit}.
It is possible to acces the \var{TParent.Doit} from within the
var{TChild.Doit}, with the \var{inherited} keyword:
\begin{verbatim}
Procedure TChild.Doit;
begin
  inherited Doit;
  ...
end;
\end{verbatim}
In the above example, when \var{TChild.Doit} is called, the first thing it
does is call \var{TParent.Doit}. You cannot use the inherited keyword on
static methods, only on virtual methods.
\subsubsection{Abstract methods}
An abstract method is a special kind of virtual method. A method can not be
abstract if it is not virtual (this is not obvious from the syntax diagram).
You cannot create an instance of an object that has an abstract method.
The reason is obvious: there is no method where the compiler could jump to !
A method that is declared \var{abstract} does not have an implementation for
this method. It is up to inherited objects to override and implement this
method. Continuing our example, take a look at this:
\begin{verbatim}
Type
  TParent = Object
    ...
    procedure Doit;virtual;abstract;
    ...
    end;
  PParent=^TParent;
  TChild = Object(TParent)
    ...
    procedure Doit;virtual;
    ...
    end;
  PChild = ^TChild;
\end{verbatim}
As it is visible, both the parent and child objects have a method called
\var{Doit}. Consider now the following declarations and calls :
\begin{verbatim}
Var ParentA,ParentB : PParent;
    Child           : PChild;
   ParentA := New(PParent,Init);
   ParentB := New(PChild,Init);
   Child := New(PChild,Init);
   ParentA^.Doit;
   ParentB^.Doit;
   Child^.Doit;
\end{verbatim}
First of all, Line 3 will generate a compiler error, stating that you cannot
generate instances of objects with abstract methods: The compiler has
detected that \var{PParent} points to an object which has an abstract
method. Commenting line 3 would allow compilation of the program.
\begin{remark}
If you override an abstract method, you cannot call the parent
method with \var{inherited}, since there is no parent method; The compiler
will detect this, and complain about it, like this:
\begin{verbatim}
testo.pp(32,3) Error: Abstract methods can't be called directly
\end{verbatim}
If, through some mechanism, an abstract method is called at run-time,
then a run-time error will occur. (run-time error 211, to be precise)
\end{remark}

%%%%%%%%%%%%%%%%%%%%%%%%%%%%%%%%%%%%%%%%%%%%%%%%%%%%%%%%%%%%%%%%%%%%%%%
% Visibility
\section{Visibility}
For objects, 3 visibility specifiers exist : \var{private}, \var{protected} and
\var{public}. If you don't specify a visibility specifier, \var{public}
is assumed.
Both methods and fields can be hidden from a programmer by putting them
in a \var{private} section. The exact visibility rule is as follows:
\begin{description}
\item [Private\ ] All fields and methods that are in a \var{private} block,
can  only be accessed in the module (i.e. unit or program) that contains
the object definition.
They can be accessed from inside the object's methods or from outside them
e.g. from other objects' methods, or global functions.
\item [Protected\ ] Is the same as \var{Private}, except that the members of
a \var{Protected} section are also accessible to descendent types, even if
they are implemented in other modules.
\item [Public\ ] sections are always accessible, from everywhere.
Fields and metods in a \var{public} section behave as though they were part
of an ordinary \var{record} type.
\end{description}


%%%%%%%%%%%%%%%%%%%%%%%%%%%%%%%%%%%%%%%%%%%%%%%%%%%%%%%%%%%%%%%%%%%%%%%
% Classes

%%%%%%%%%%%%%%%%%%%%%%%%%%%%%%%%%%%%%%%%%%%%%%%%%%%%%%%%%%%%%%%%%%%%%%%
\chapter{Classes}
\label{ch:Classes}
In the Delphi approach to Object Oriented Programming, everything revolves
around  the concept of 'Classes'.  A class can be seen as a pointer to an
object, or a pointer to a record.

\begin{remark}
In earlier versions of \fpc it was necessary, in order to use classes,
to put the \file{objpas} unit in the uses clause of your unit or program.
{\em This is no longer needed} as of version 0.99.12. As of version 0.99.12
the \file{system} unit contains the basic  definitions of \var{TObject}
and  \var{TClass}, as well as some auxiliary methods for using classes.
The \file{objpas} unit still exists, and contains some redefinitions of
basic types, so they coincide with Delphi types. The unit will be loaded
automatically if you specify the \var{-S2} or \var{-Sd} options.
\end{remark}

%%%%%%%%%%%%%%%%%%%%%%%%%%%%%%%%%%%%%%%%%%%%%%%%%%%%%%%%%%%%%%%%%%%%%%%
% Class definitions
\section{Class definitions}
The prototype declaration of a class is as follows :
\input{syntax/typeclas.syn}
Again, You can repeat as many \var{private}, \var{protected}, \var{published}
and \var{public} blocks as you want.
Methods are normal function or procedure declarations.
As you can see, the declaration of a class is almost identical to the
declaration of an object. The real difference between objects and classes
is in the way they are created (see further in this chapter).
The visibility of the different sections is as follows:
\begin{description}
\item [Private\ ] All fields and methods that are in a \var{private} block, can
only be accessed in the module (i.e. unit) that contains the class definition.
They can be accessed from inside the classes' methods or from outside them
(e.g. from other classes' methods)
\item [Protected\ ] Is the same as \var{Private}, except that the members of
a \var{Protected} section are also accessible to descendent types, even if
they are implemented in other modules.
\item [Public\ ] sections are always accessible.
\item [Published\ ] Is the same as a \var{Public} section, but the compiler
generates also type information that is needed for automatic streaming of
these classes. Fields defined in a \var{published} section must be of class type.
Array peroperties cannot be in a \var{published} section.
\end{description}


%%%%%%%%%%%%%%%%%%%%%%%%%%%%%%%%%%%%%%%%%%%%%%%%%%%%%%%%%%%%%%%%%%%%%%%
% Class instantiation
\section{Class instantiation}
Classes must be created using their constructor. Remember that a class is a
pointer to an object, so when you declare a variable of some class, the
compiler just allocates a pointer, not the entire object. The constructor of
a class returns a pointer to an initialized instance of the object.
So, to initialize an instance of some class, you would do the following :
\begin{verbatim}
  ClassVar := ClassType.ConstructorName;
\end{verbatim}
You cannot use the extended syntax of \var{new} and \var{dispose} to
instantiate and destroy class instances.
That construct is reserved for use with objects only.
Calling the constructor will provoke a call to \var{getmem}, to allocate
enough space to hold the class instance data.
After that, the constuctor's code is executed.
The constructor has a pointer to it's data, in \var{self}.

\begin{remark}
\begin{itemize}
\item The \var{\{\$PackRecords \}} directive also affects classes.
i.e. the alignment in memory of the different fields depends on the
value of  the \var{\{\$PackRecords \}} directive.
\item Just as for objects and records, you can declare a packed class.
This has the same effect as on an object, or record, namely that the
elements are aligned on 1-byte boundaries. i.e. as close as possible.
\item \var{SizeOf(class)} will return 4, since a class is but a pointer to
an object. To get the size of the class instance data, use the
\var{TObject.InstanceSize} method.
\end{itemize}
\end{remark}

%%%%%%%%%%%%%%%%%%%%%%%%%%%%%%%%%%%%%%%%%%%%%%%%%%%%%%%%%%%%%%%%%%%%%%%
% Methods
\section{Methods}
\subsection{invocation}
Method invocaticn for classes is no different than for objects. The
following is a valid method invocation:
\begin{verbatim}
Var  AnObject : TAnObject;
begin
  AnObject := TAnObject.Create;
  ANobject.AMethod;
\end{verbatim}
\subsection{Virtual methods}
Classes have virtual methods, just as objects do. There is however a
difference between the two. For objects, it is sufficient to redeclare the
same method in a descendent object with the keyword \var{virtual} to
override it. For classes, the situation is different: you {\em must}
override virtual methods with the \var{override} keyword. Failing to do so,
will start a {\em new} batch of virtual methods, hiding the previous
one.  The \var{Inherited} keyword will not jump to the inherited method, if
virtual was used.

The following code is {\em wrong}:
\begin{verbatim}
Type ObjParent = Class
        Procedure MyProc; virtual;
     end;
     ObjChild  = Class(ObjPArent)
       Procedure MyProc; virtual;
     end;
\end{verbatim}
The compiler will produce a warning:
\begin{verbatim}
Warning: An inherited method is hidden by OBJCHILD.MYPROC
\end{verbatim}
The compiler will compile it, but using \var{Inherited} can
produce strange effects.

The correct declaration is as follows:
\begin{verbatim}
Type ObjParent = Class
        Procedure MyProc; virtual;
     end;
     ObjChild  = Class(ObjPArent)
       Procedure MyProc; override;
     end;
\end{verbatim}
This will compile and run without warnings or errors.

\subsection{Message methods}
New in classes are \var{message} methods. Pointers to message methods are
stored in a special table, together with the integer or string cnstant that
they were declared with. They are primarily intended to ease programming of
callback functions in several \var{GUI} toolkits, such as \var{Win32} or
\var{GTK}. In difference with Delphi, \fpc also accepts strings as message
identifiers.

Message methods that are declared with an integer constant can take only one
var argument (typed or not):
\begin{verbatim}
 Procedure TMyObject.MyHandler(Var Msg); Message 1;
\end{verbatim}
The method implementation of a message function is no different from an
ordinary method. It is also possible to call a message method directly,
but you should not do this. Instead use the \var{TObject.Dispatch} method.

The \var{TOBject.Dispatch} method can be used to call a \var{message}
handler. It is declared in the \file{system} unit and will accept a var
parameter  which must have at the first position a cardinal with the
message ID that should be called. For example:
\begin{verbatim}
Type
  TMsg = Record
    MSGID : Cardinal
    Data : Pointer;
Var
  Msg : TMSg;

MyObject.Dispatch (Msg);
\end{verbatim}
In this example, the \var{Dispatch} method will look at the object and all
it's ancestors (starting at the object, and searching up the class tree),
to see if a message method with message \var{MSGID} has been
declared. If such a method is found, it is called, and passed the
\var{Msg} parameter.

If no such method is found, \var{DefaultHandler} is called.
\var{DefaultHandler} is a virtual method of \var{TObject} that doesn't do
anything, but which can be overridden to provide any processing you might
need. \var{DefaultHandler} is declared as follows:
\begin{verbatim}
   procedure defaulthandler(var message);virtual;
\end{verbatim}

In addition to the message method with a \var{Integer} identifier,
\fpc also supports a messae method with a string identifier:
\begin{verbatim}
 Procedure TMyObject.MyStrHandler(Var Msg); Message 'OnClick';
\end{verbatim}

The working of the string message handler is the same as the ordinary
integer message handler:

The \var{TOBject.DispatchStr} method can be used to call a \var{message}
handler. It is declared in the system unit and will accept one parameter
which must have at the first position a string with the message ID that
should be called. For example:
\begin{verbatim}
Type
  TMsg = Record
    MsgStr : String[10]; // Arbitrary length up to 255 characters.
    Data : Pointer;
Var
  Msg : TMSg;

MyObject.DispatchStr (Msg);
\end{verbatim}
In this example, the \var{DispatchStr} method will look at the object and
all it's ancestors (starting at the object, and searching up the class tree),
to see if a message method with message \var{MsgStr} has been
declared. If such a method is found, it is called, and passed the
\var{Msg} parameter.

If no such method is found, \var{DefaultHandlerStr} is called.
\var{DefaultHandlerStr} is a virtual method of \var{TObject} that doesn't do
anything, but which can be overridden to provide any processing you might
need. \var{DefaultHandlerStr} is declared as follows:
\begin{verbatim}
   procedure DefaultHandlerStr(var message);virtual;
\end{verbatim}

In addition to this mechanism, a string message method accepts a \var{self}
parameter:
\begin{verbatim}
  TMyObject.StrMsgHandler(Data : Pointer; Self : TMyObject);Message 'OnClick';
\end{verbatim}
When encountering such a method, the compiler will generate code that loads
the \var{Self} parameter into the object instance pointer. The result of
this is that it is possible to pass \var{Self} as a parameter to such a
method.

\begin{remark}
The type of the \var{Self} parameter must be of the same class
as the class you define the method for.
\end{remark}

%%%%%%%%%%%%%%%%%%%%%%%%%%%%%%%%%%%%%%%%%%%%%%%%%%%%%%%%%%%%%%%%%%%%%%%
% Properties
\section{Properties}
Classes can contain properties as part of their fields list. A property
acts like a normal field, i.e. you can get or set it's value, but
allows to redirect the access of the field through functions and
procedures. They provide a means to associate an action with an assignment
of or a reading from a class 'field'. This allows for e.g. checking that a
value is valid when assigning, or, when reading, it allows to construct the
value on the fly. Moreover, properties can be read-only or write only.
The prototype declaration of a property is as follows:
\input{syntax/property.syn}
A \var{read specifier} is either the name of a field that contains the
property, or the name of a method function that has the same return type as
the property type. In the case of a simple type, this
function must not accept an argument. A \var{read specifier} is optional,
making the property write-only.
A \var{write specifier} is optional: If there is no \var{write specifier}, the
property is read-only. A write specifier is either the name of a field, or
the name of a method procedure that accepts as a sole argument a variable of
the same type as the property.
The section (\var{private}, \var{published}) in which the specified function or
procedure resides is irrelevant. Usually, however, this will be a protected
or private method.
Example:
Given the following declaration:
\begin{verbatim}
Type
  MyClass = Class
    Private
    Field1 : Longint;
    Field2 : Longint;
    Field3 : Longint;
    Procedure  Sety (value : Longint);
    Function Gety : Longint;
    Function Getz : Longint;
    Public
    Property X : Longint Read Field1 write Field2;
    Property Y : Longint Read GetY Write Sety;
    Property Z : Longint Read GetZ;
    end;
Var MyClass : TMyClass;
\end{verbatim}
The following are valid statements:
\begin{verbatim}
WriteLn ('X : ',MyClass.X);
WriteLn ('Y : ',MyClass.Y);
WriteLn ('Z : ',MyClass.Z);
MyClass.X := 0;
MyClass.Y := 0;
\end{verbatim}
But the following would generate an error:
\begin{verbatim}
MyClass.Z := 0;
\end{verbatim}
because Z is a read-only property.
What happens in the above statements is that when a value needs to be read,
the compiler inserts a call to the various \var{getNNN} methods of the
object, and the result of this call is used. When an assignment is made,
the compiler passes the value that must be assigned as a paramater to
the various \var{setNNN} methods.
Because of this mechanism, properties cannot be passed as var arguments to a
function or procedure, since there is no known address of the property (at
least, not always).
If the property definition contains an index, then the read and write
specifiers must be a function and a procedure. Moreover, these functions
require an additional parameter : An integer parameter. This allows to read
or write several properties with the same function. For this, the properties
must have the same type.
The following is an example of a property with an index:
\begin{verbatim}
{$mode objfpc}
Type TPoint = Class(TObject)
       Private
       FX,FY : Longint;
       Function GetCoord (Index : Integer): Longint;
       Procedure SetCoord (Index : Integer; Value : longint);
       Public
       Property X : Longint index 1 read GetCoord Write SetCoord;
       Property Y : Longint index 2 read GetCoord Write SetCoord;
       Property Coords[Index : Integer] Read GetCoord;
       end;
Procedure TPoint.SetCoord (Index : Integer; Value : Longint);
begin
  Case Index of
   1 : FX := Value;
   2 : FY := Value;
  end;
end;
Function TPoint.GetCoord (INdex : Integer) : Longint;
begin
  Case Index of
   1 : Result := FX;
   2 : Result := FY;
  end;
end;
Var P : TPoint;
begin
  P := TPoint.create;
  P.X := 2;
  P.Y := 3;
  With P do
    WriteLn ('X=',X,' Y=',Y);
end.
\end{verbatim}
When the compiler encounters an assignment to \var{X}, then \var{SetCoord}
is called with as first parameter the index (1 in the above case) and with
as a second parameter the value to be set.
Conversely, when reading the value of \var{X}, the compiler calls
\var{GetCoord} and passes it index 1.
Indexes can only be integer values.
You can also have array properties. These are properties that accept an
index, just as an array does. Only now the index doesn't have to be an
ordinal type, but can be any type.

A \var{read specifier} for an array property is the name method function
that has the same return type as  the property type.
The function must accept as a sole arguent a variable of the same type as
the index type. For an array property, you cannot specify fields as \var{read
specifiers}.

A \var{write specifier} for an array property is the name of a method
procedure that accepts two arguments: The first argument has the same
type as the index, and the second argument is a parameter of the same
type as the property type.
As an example, see the following declaration:
\begin{verbatim}
Type TIntList = Class
      Private
      Function GetInt (I : Longint) : longint;
      Function GetAsString (A : String) : String;
      Procedure SetInt (I : Longint; Value : Longint;);
      Procedure SetAsString (A : String; Value : String);
      Public
      Property Items [i : Longint] : Longint Read GetInt
                                             Write SetInt;
      Property StrItems [S : String] : String Read GetAsString
                                              Write SetAsstring;
      end;
Var AIntList : TIntList;
\end{verbatim}
Then the following statements would be valid:
\begin{verbatim}
AIntList.Items[26] := 1;
AIntList.StrItems['twenty-five'] := 'zero';
WriteLn ('Item 26 : ',AIntList.Items[26]);
WriteLn ('Item 25 : ',AIntList.StrItems['twenty-five']);
\end{verbatim}
While the following statements would generate errors:
\begin{verbatim}
AIntList.Items['twenty-five'] := 1;
AIntList.StrItems[26] := 'zero';
\end{verbatim}
Because the index types are wrong.
Array properties can be declared as \var{default} properties. This means that
it is not necessary to specify the property name when assigning or reading
it. If, in the previous example, the definition of the items property would
have been
\begin{verbatim}
 Property Items[i : Longint]: Longint Read GetInt
                                      Write SetInt; Default;
\end{verbatim}
Then the assignment
\begin{verbatim}
AIntList.Items[26] := 1;
\end{verbatim}
Would be equivalent to the following abbreviation.
\begin{verbatim}
AIntList[26] := 1;
\end{verbatim}
You can have only one default property per class, and descendent classes
cannot redeclare the default property.


%%%%%%%%%%%%%%%%%%%%%%%%%%%%%%%%%%%%%%%%%%%%%%%%%%%%%%%%%%%%%%%%%%%%%%%
% Expressions

%%%%%%%%%%%%%%%%%%%%%%%%%%%%%%%%%%%%%%%%%%%%%%%%%%%%%%%%%%%%%%%%%%%%%%%
\chapter{Expressions}
\label{ch:Expressions}
Expressions occur in assignments or in tests. Expressions produce a value,
of a certain type.
Expressions are built with two components: Operators and their operands.
Usually an operator is binary, i.e. it requires 2 operands. Binary operators
occur always between the operands (as in \var{X/Y}). Sometimes an
operator is unary, i.e. it requires only one argument. A unary operator
occurs always before the operand, as in \var{-X}.

When using multiple operands in an expression, the precedence rules of
\seet{OperatorPrecedence} are used.
\begin{FPCltable}{lll}{Precedence of operators}{OperatorPrecedence}
Operator & Precedence & Category \\ \hline
\var{Not, @} & Highest (first) & Unary operators\\
\var{* / div mod and shl shr as} & Second & Multiplying operators\\
\var{+ - or xor} & Third & Adding operators \\
\var{< <> < > <= >= in is} & Lowest (Last) & relational operators \\
\hline
\end{FPCltable}
When determining the precedence, the compiler uses the following rules:
\begin{enumerate}
\item In operations with unequal precedences the operands belong to the
operater with the highest precedence. For example, in \var{5*3+7}, the
multiplication is higher in precedence than the addition, so it is
executed first. The result would be 22.
\item If parentheses are used in an epression, their contents is evaluated
first. Thus, \var {5*(3+7)} would result in 50.
\end{enumerate}

\begin{remark}
The order in which expressions of the same precedence are evaluated is not
guaranteed to be left-to-right. In general, no assumptions on which expression
is evaluated first should be made in such a case.
The compiler will decide which expression to evaluate first based on
optimization rules. Thus, in the following expression:
\begin{verbatim}
  a := g(3) + f(2);
\end{verbatim}
\var{f(2)} may be executed before \var{g(3)}. This behaviour is distinctly
different from \delphi or \tp.

If one expression {\em must} be executed before the other, it is necessary
to split up the statement using temporary results:
\begin{verbatim}
  e1 := g(3);
  a  := e1 + f(2);
\end{verbatim}
\end{remark}

%%%%%%%%%%%%%%%%%%%%%%%%%%%%%%%%%%%%%%%%%%%%%%%%%%%%%%%%%%%%%%%%%%%%%%%
% Expression syntax
\section{Expression syntax}
An expression applies relational operators to simple expressions. Simple
expressions are a series of terms (what a term is, is explained below), joined by
adding operators.
\input{syntax/expsimpl.syn}
The following are valid expressions:
\begin{verbatim}
GraphResult<>grError
(DoItToday=Yes) and (DoItTomorrow=No);
Day in Weekend
\end{verbatim}
And here are some simple expressions:
\begin{verbatim}
A + B
-Pi
ToBe or NotToBe
\end{verbatim}
Terms consist of factors, connected by multiplication operators.
\input{syntax/expterm.syn}
Here are some valid terms:
\begin{verbatim}
2 * Pi
A Div B
(DoItToday=Yes) and (DoItTomorrow=No);
\end{verbatim}
Factors are all other constructions:
\input{syntax/expfact.syn}

%%%%%%%%%%%%%%%%%%%%%%%%%%%%%%%%%%%%%%%%%%%%%%%%%%%%%%%%%%%%%%%%%%%%%%%
% Function calls
\section{Function calls}
Function calls are part of expressions (although, using extended syntax,
they can be statements too). They are constructed as follows:
\input{syntax/fcall.syn}
The \synt{variable reference} must be a procedural type variable reference.
A method designator can only be used inside the method of an object. A
qualified method designator can be used outside object methods too.
The function that will get called is the function with a declared parameter
list that matches the actual parameter list. This means that
\begin{enumerate}
\item The number of actual parameters must equal the number of declared
parameters.
\item The types of the parameters must be compatible. For variable
reference parameters, the parameter types must be exactly the same.
\end{enumerate}
If no matching function is found, then the compiler will generate an error.
Depending on the fact of the function is overloaded (i.e. multiple functions
with the same name, but different parameter lists) the error will be
different.
There are cases when the compiler will not execute the function call in an
expression. This is the case when you are assigning a value to a procedural
type variable, as in the following example:
\begin{verbatim}
Type
  FuncType = Function: Integer;
Var A : Integer;
Function AddOne : Integer;
begin
  A := A+1;
  AddOne := A;
end;
Var F : FuncType;
    N : Integer;
begin
  A := 0;
  F := AddOne; { Assign AddOne to F, Don't call AddOne}
  N := AddOne; { N := 1 !!}
end.
\end{verbatim}
In the above listing, the assigment to F will not cause the function AddOne
to be called. The assignment to N, however, will call AddOne.
A problem with this syntax is the following construction:
\begin{verbatim}
If F = AddOne Then
  DoSomethingHorrible;
\end{verbatim}
Should the compiler compare the addresses of \var{F} and \var{AddOne},
or should it call both functions, and compare the result ? \fpc solves this
by deciding that a procedural variable is equivalent to a pointer. Thus the
compiler will give a type mismatch error, since AddOne is considered a
call to a function with integer result, and F is a pointer, Hence a type
mismatch occurs.
How then, should one compare whether \var{F} points to the function
\var{AddOne} ? To do this, one should use the address operator \var{@}:
\begin{verbatim}
If F = @AddOne Then
  WriteLn ('Functions are equal');
\end{verbatim}
The left hand side of the boolean expression is an address. The right hand
side also, and so the compiler compares 2 addresses.
How to compare the values that both functions return ? By adding an empty
parameter list:
\begin{verbatim}
  If F()=Addone then
    WriteLn ('Functions return same values ');
\end{verbatim}
Remark that this behaviour is not compatible with Delphi syntax.

%%%%%%%%%%%%%%%%%%%%%%%%%%%%%%%%%%%%%%%%%%%%%%%%%%%%%%%%%%%%%%%%%%%%%%%
% Set constructors
\section{Set constructors}
When you want to enter a set-type constant in an expression, you must give a
set constructor. In essence this is the same thing as when you define a set
type, only you have no identifier to identify the set with.
A set constructor is a comma separated list of expressions, enclosed in
square brackets.
\input{syntax/setconst.syn}
All set groups and set elements must be of the same ordinal type.
The empty set is denoted by \var{[]}, and it can be assigned to any type of
set. A set group with a range  \var{[A..Z]} makes all values in the range a
set element. If the first range specifier has a bigger ordinal value than
the second the set is empty, e.g., \var{[Z..A]} denotes an empty set.
The following are valid set constructors:
\begin{verbatim}
[today,tomorrow]
[Monday..Friday,Sunday]
[ 2, 3*2, 6*2, 9*2 ]
['A'..'Z','a'..'z','0'..'9']
\end{verbatim}

%%%%%%%%%%%%%%%%%%%%%%%%%%%%%%%%%%%%%%%%%%%%%%%%%%%%%%%%%%%%%%%%%%%%%%%
% Value typecasts
\section{Value typecasts}
Sometimes it is necessary to change the type of an expression, or a part of
the expression, to be able to be assignment compatible. This is done through
a value typecast. The syntax diagram for a value typecast is as follows:
\input{syntax/tcast.syn}
Value typecasts cannot be used on the left side of assignments, as variable
typecasts.
Here are some valid typecasts:
\begin{verbatim}
Byte('A')
Char(48)
boolean(1)
longint(@Buffer)
\end{verbatim}
The type size of the expression and the size of the type cast must be the
same. That is, the following doesn't work:
\begin{verbatim}
Integer('A')
Char(4875)
boolean(100)
Word(@Buffer)
\end{verbatim}
This is different from Delphi or Turbo Pascal behaviour.


%%%%%%%%%%%%%%%%%%%%%%%%%%%%%%%%%%%%%%%%%%%%%%%%%%%%%%%%%%%%%%%%%%%%%%%
% The @ operator
\section{The @ operator}
The address operator \var{@} returns the address of a variable, procedure
or function. It is used as follows:
\input{syntax/address.syn}
The \var{@} operator returns a typed pointer if the \var{\$T} switch is on.
If the \var{\$T} switch is off then the address operator returns an untyped
pointer, which is assigment compatible with all pointer types. The type of
the pointer is \var{\^{}T}, where \var{T} is the type of the variable
reference.
For example, the following will compile
\begin{verbatim}
Program tcast;
{$T-} { @ returns untyped pointer }

Type art = Array[1..100] of byte;
Var Buffer : longint;
    PLargeBuffer : ^art;

begin
 PLargeBuffer := @Buffer;
end.
\end{verbatim}
Changing the \var{\{\$T-\}} to \var{\{\$T+\}} will prevent the compiler from
compiling this. It will give a type mismatch error.
By default, the address operator returns an untyped pointer.
Applying the address operator to a function, method, or procedure identifier
will give a pointer to the entry point of that function. The result is an
untyped pointer.
By default, you must use the address operator if you want to assign a value
to a procedural type variable. This behaviour can be avoided by using the
\var{-So} or \var{-S2} switches, which result in a more compatible Delphi or
Turbo Pascal syntax.

%%%%%%%%%%%%%%%%%%%%%%%%%%%%%%%%%%%%%%%%%%%%%%%%%%%%%%%%%%%%%%%%%%%%%%%
% Operators
\section{Operators}
Operators can be classified according to the type of expression they
operate on. We will discuss them type by type.
\subsection{Arithmetic operators}
Arithmetic operators occur in arithmetic operations, i.e. in expressions
that contain integers or reals. There are 2 kinds of operators : Binary and
unary arithmetic operators.
Binary operators are listed in \seet{binaroperators}, unary operators are
listed in \seet{unaroperators}.
\begin{FPCltable}{ll}{Binary arithmetic operators}{binaroperators}
Operator & Operation \\ \hline
\var{+} & Addition\\
\var{-} & Subtraction\\
\var{*} & Multiplication \\
\var{/} & Division \\
\var{Div} & Integer division \\
\var{Mod} & Remainder \\ \hline
\end{FPCltable}
With the exception of \var{Div} and \var{Mod}, which accept only integer
expressions as operands, all operators accept real and integer expressions as
operands.
For binary operators, the result type will be integer if both operands are
integer type expressions. If one of the operands is a real type expression,
then  the result is real.
As an exception : division (\var{/}) results always in real values.
\begin{FPCltable}{ll}{Unary arithmetic operators}{unaroperators}
Operator & Operation \\ \hline
\var{+} & Sign identity\\
\var{-} & Sign inversion \\ \hline
\end{FPCltable}
For unary operators, the result type is always equal to the expression type.
The division (\var{/}) and \var{Mod} operator will cause run-time errors if
the second argument is zero.
The sign of the result of a \var{Mod} operator is the same as the sign of
the left side operand of the \var{Mod} operator. In fact, the \var{Mod}
operator is equivalent to the following operation :
\begin{verbatim}
  I mod J = I - (I div J) * J
\end{verbatim}
but it executes faster than the right hand side expression.
\subsection{Logical operators}
Logical operators act on the individual bits of ordinal expressions.
Logical operators require operands that are of an integer type, and produce
an integer type result. The possible logical operators are listed in
\seet{logicoperations}.
\begin{FPCltable}{ll}{Logical operators}{logicoperations}
Operator & Operation \\ \hline
\var{not} & Bitwise negation (unary) \\
\var{and} & Bitwise and \\
\var{or}  & Bitwise or \\
\var{xor} & Bitwise xor \\
\var{shl} & Bitwise shift to the left \\
\var{shr} & Bitwise shift to the right \\ \hline
\end{FPCltable}
The following are valid logical expressions:
\begin{verbatim}
A shr 1  { same as A div 2, but faster}
Not 1    { equals -2 }
Not 0    { equals -1 }
Not -1   { equals 0  }
B shl 2  { same as B * 2 for integers }
1 or 2   { equals 3 }
3 xor 1  { equals 2 }
\end{verbatim}
\subsection{Boolean operators}
Boolean operators can be considered logical operations on a type with 1 bit
size. Therefore the \var{shl} and \var{shr} operations have little sense.
Boolean operators can only have boolean type operands, and the resulting
type is always boolean. The possible operators are listed in
\seet{booleanoperators}
\begin{FPCltable}{ll}{Boolean operators}{booleanoperators}
Operator & Operation \\ \hline
\var{not} & logical negation (unary) \\
\var{and} & logical and \\
\var{or}  & logical or \\
\var{xor} & logical xor \\ \hline
\end{FPCltable}
\begin{remark} Boolean expressions are ALWAYS evaluated with short-circuit
evaluation. This means that from the moment the result of the complete
expression is known, evaluation is stopped and the result is returned.
For instance, in the following expression:
\begin{verbatim}
 B := True or MaybeTrue;
\end{verbatim}
The compiler will never look at the value of \var{MaybeTrue}, since it is
obvious that the expression will always be true. As a result of this
strategy, if \var{MaybeTrue} is a function, it will not get called !
(This can have surprising effects when used in conjunction with properties)
\end{remark}

\subsection{String operators}
There is only one string operator : \var{+}. It's action is to concatenate
the contents of the two strings (or characters) it stands between.
You cannot use \var{+} to concatenate null-terminated (\var{PChar}) strings.
The following are valid string operations:
\begin{verbatim}
  'This is ' + 'VERY ' + 'easy !'
  Dirname+'\'
\end{verbatim}
The following is not:
\begin{verbatim}
Var Dirname = Pchar;
...
  Dirname := Dirname+'\';
\end{verbatim}
Because \var{Dirname} is a null-terminated string.
\subsection{Set operators}
The following operations on sets can be performed with operators:
Union, difference and intersection. The operators needed for this are listed
in \seet{setoperators}.
\begin{FPCltable}{ll}{Set operators}{setoperators}
Operator & Action \\ \hline
\var{+} & Union \\
\var{-} & Difference \\
\var{*} & Intersection \\ \hline
\end{FPCltable}
The set type of the operands must be the same, or an error will be
generated by the compiler.
\subsection{Relational operators}
The relational operators are listed in \seet{relationoperators}
\begin{FPCltable}{ll}{Relational operators}{relationoperators}
Operator & Action \\ \hline
\var{=} & Equal \\
\var{<>} & Not equal \\
\var{<} & Stricty less than\\
\var{>} & Strictly greater than\\
\var{<=} & Less than or equal \\
\var{>=} & Greater than or equal \\
\var{in} & Element of \\ \hline
\end{FPCltable}
Left and right operands must be of the same type. You can only mix integer
and real types in relational expressions.
Comparing strings is done on the basis of their ASCII code representation.
When comparing pointers, the addresses to which they point are compared.
This also is true for \var{PChar} type pointers. If you want to compare the
strings the \var{Pchar} points to, you must use the \var{StrComp} function
from the \file{strings} unit.
The \var{in} returns \var{True} if the left operand (which must have the same
ordinal type as the set type) is an element of the set which is the right
operand, otherwise it returns \var{False}
\chapter{Statements}
\label{ch:Statements}
The heart of each algorithm are the actions it takes. These actions are
contained in the statements of your program or unit. You can label your
statements, and jump to them (within certain limits) with \var{Goto}
statements.
This can be seen in the following syntax diagram:
\input{syntax/statement.syn}
A label can be an identifier or an integer digit.

%%%%%%%%%%%%%%%%%%%%%%%%%%%%%%%%%%%%%%%%%%%%%%%%%%%%%%%%%%%%%%%%%%%%%%%
% Simple statements
\section{Simple statements}
A simple statement cannot be decomposed in separate statements. There are
basically 4 kinds of simple statements:
\input{syntax/simstate.syn}
Of these statements, the {\em raise statement} will be explained in the
chapter on Exceptions (\seec{Exceptions})
\subsection{Assignments}
Assignments give a value to a variable, replacing any previous value the
variable might have had:
\input{syntax/assign.syn}
In addition to the standard Pascal assignment operator (\var{ := }), which
simply replaces the value of the varable with the value resulting from the
expression on the right of the { := } operator, \fpc
supports some c-style constructions. All available constructs are listed in
\seet{assignments}.
\begin{FPCltable}{lr}{Allowed C constructs in \fpc}{assignments}
Assignment & Result \\ \hline
a += b & Adds \var{b} to \var{a}, and stores the result in \var{a}.\\
a -= b & Substracts \var{b} from \var{a}, and stores the result in
\var{a}. \\
a *= b & Multiplies \var{a} with \var{b}, and stores the result in
\var{a}. \\
a /= b & Divides \var{a} through \var{b}, and stores the result in
\var{a}. \\ \hline
\end{FPCltable}
For these constructs to work, you should specify the \var{-Sc}
command-line switch.

\begin{remark}
These constructions are just for typing convenience, they
don't generate different code.
Here are some examples of valid assignment statements:
\begin{verbatim}
X := X+Y;
X+=Y;      { Same as X := X+Y, needs -Sc command line switch}
X/=2;      { Same as X := X/2, needs -Sc command line switch}
Done := False;
Weather := Good;
MyPi := 4* Tan(1);
\end{verbatim}
\end{remark}

\subsection{Procedure statements}
Procedure statements are calls to subroutines. There are
different possibilities for procedure calls: A normal procedure call, an
object method call (fully qualified or not), or even a call to a procedural
type variable. All types are present in the following diagram.
\input{syntax/procedure.syn}
The \fpc compiler will look for a procedure with the same name as given in
the procedure statement, and with a declared parameter list that matches the
actual parameter list.
The following are valid procedure statements:
\begin{verbatim}
Usage;
WriteLn('Pascal is an easy language !');
Doit();
\end{verbatim}
\subsection{Goto statements}
\fpc supports the \var{goto} jump statement. Its prototype syntax is
\input{syntax/goto.syn}
When using \var{goto} statements, you must keep the following in mind:
\begin{enumerate}
\item The jump label must be defined in the same block as the \var{Goto}
statement.
\item Jumping from outside a loop to the inside of a loop or vice versa can
 have strange effects.
\item To be able to use the \var{Goto} statement, you need to specify the
\var{-Sg} compiler switch.
\end{enumerate}
\var{Goto} statements are considered bad practice and should be avoided as
much as possible. It is always possible to replace a \var{goto} statement by a
construction that doesn't need a \var{goto}, although this construction may
not be as clear as a goto statement.
For instance, the following is an allowed goto statement:
\begin{verbatim}
label
  jumpto;
...
Jumpto :
  Statement;
...
Goto jumpto;
...
\end{verbatim}

%%%%%%%%%%%%%%%%%%%%%%%%%%%%%%%%%%%%%%%%%%%%%%%%%%%%%%%%%%%%%%%%%%%%%%%
% Structured statements
\section{Structured statements}
Structured statements can be broken into smaller simple statements, which
should be executed repeatedly, conditionally  or sequentially:
\input{syntax/struct.syn}
Conditional statements come in 2 flavours :
\input{syntax/conditio.syn}
Repetitive statements come in 3 flavours:
\input{syntax/repetiti.syn}
The following sections deal with each of these statements.
\subsection{Compound statements}
Compound statements are a group of statements, separated by semicolons,
that are surrounded by the keywords \var{Begin} and \var{End}. The
Last statement doesn't need to be followed by a semicolon, although it is
allowed. A compound statement is a way of grouping statements together,
executing the statements sequentially. They are treated as one statement
in cases where Pascal syntax expects 1 statement, such as in
\var{if ... then} statements.
\input{syntax/compound.syn}
\subsection{The \var{Case} statement}
\fpc supports the \var{case} statement. Its syntax diagram is
\input{syntax/case.syn}
The constants appearing in the various case parts must be known at
compile-time, and can be of the following types : enumeration types,
Ordinal types (except boolean), and chars. The expression must be also of
this type, or a compiler error will occur. All case constants must
have the same type.
The compiler will evaluate the expression. If one of the case constants
values matches the value of the expression, the statement that follows
this constant is executed. After that, the program continues after the final
\var{end}.
If none of the case constants match the expression value, the statement
after the \var{else} keyword is executed. This can be an empty statement.
If no else part is present, and no case constant matches the expression
value, program flow continues after the final \var{end}.
The case statements can be compound statements
(i.e. a \var{begin..End} block).

\begin{remark}
Contrary to Turbo Pascal, duplicate case labels are not
allowed in \fpc, so the following code will generate an error when
compiling:
\begin{verbatim}
Var i : integer;
...
Case i of
 3 : DoSomething;
 1..5 : DoSomethingElse;
end;
\end{verbatim}
The compiler will generate a \var{Duplicate case label} error when compiling
this, because the 3 also appears (implicitly) in the range \var{1..5}. This
is similar to Delhpi syntax.
\end{remark}
The following are valid case statements:
\begin{verbatim}
Case C of
 'a' : WriteLn ('A pressed');
 'b' : WriteLn ('B pressed');
 'c' : WriteLn ('C pressed');
else
  WriteLn ('unknown letter pressed : ',C);
end;
\end{verbatim}
Or
\begin{verbatim}
Case C of
 'a','e','i','o','u' : WriteLn ('vowel pressed');
 'y' : WriteLn ('This one depends on the language');
else
  WriteLn ('Consonant pressed');
end;
\end{verbatim}
\begin{verbatim}
Case Number of
 1..10   : WriteLn ('Small number');
 11..100 : WriteLn ('Normal, medium number');
else
 WriteLn ('HUGE number');
end;
\end{verbatim}
\subsection{The \var{If..then..else} statement}
The \var{If .. then .. else..} prototype syntax is
\input{syntax/ifthen.syn}
The expression between the \var{if} and \var{then} keywords must have a
boolean return type. If the expression evaluates to \var{True} then the
statement following \var{then} is executed.

If the expression evaluates to \var{False}, then the statement following
\var{else} is executed, if it is present.

Be aware of the fact that the boolean expression will be short-cut evaluated.
(Meaning that the evaluation will be stopped at the point where the
 outcome is known with certainty)
Also, before the \var {else} keyword,  no semicolon (\var{;}) is allowed,
but all statements can be compound statements.
In nested \var{If.. then .. else} constructs, some ambiguity may araise as
to which  \var{else} statement pairs with which \var{if} statement. The rule
is that the \var{else } keyword matches the first \var{if} keyword not
already matched by an \var{else} keyword.
For example:
\begin{verbatim}
If exp1 Then
  If exp2 then
    Stat1
else
  stat2;
\end{verbatim}
Despite it's appearance, the statement is syntactically equivalent to
\begin{verbatim}
If exp1 Then
   begin
   If exp2 then
      Stat1
   else
      stat2
   end;
\end{verbatim}
and not to
\begin{verbatim}
{ NOT EQUIVALENT }
If exp1 Then
   begin
   If exp2 then
      Stat1
   end
else
   stat2
\end{verbatim}
If it is this latter construct you want, you must explicitly put the
\var{begin} and \var{end} keywords. When in doubt, add them, they don't
hurt.

The following is a valid statement:
\begin{verbatim}
If Today in [Monday..Friday] then
  WriteLn ('Must work harder')
else
  WriteLn ('Take a day off.');
\end{verbatim}
\subsection{The \var{For..to/downto..do} statement}
\fpc supports the \var{For} loop construction. A for loop is used in case
one wants to calculated something a fixed number of times.
The prototype syntax is as follows:
\input{syntax/for.syn}
\var{Statement} can be a compound statement.
When this statement is encountered, the control variable is initialized with
the initial value, and is compared with the final value.
What happens next depends on whether \var{to} or \var{downto} is used:
\begin{enumerate}
\item In the case \var{To} is used, if the initial value larger than the final
value then \var{Statement} will never be executed.
\item In the case \var{DownTo} is used, if the initial value larger than the final
value then \var{Statement} will never be executed.
\end{enumerate}
After this check, the statement after \var{Do} is executed. After the
execution of the statement, the control variable is increased or decreased
with 1, depending on whether \var{To} or \var{Downto} is used.
The control variable must be an ordinal type, no other
types can be used as counters in a loop.

\begin{remark}
Contrary to ANSI pascal specifications, \fpc first initializes
the counter variable, and only then calculates the upper bound.
\end{remark}

The following are valid loops:
\begin{verbatim}
For Day := Monday to Friday do Work;
For I := 100 downto 1 do
  WriteLn ('Counting down : ',i);
For I := 1 to 7*dwarfs do KissDwarf(i);
\end{verbatim}

If the statement is a compound statement, then  the \seep{Break} and
\seep{Continue} reserved words can be used to jump to the end or just
after the end of the \var{For} statement.


\subsection{The \var{Repeat..until} statement}
The \var{repeat} statement is used to execute a statement until a certain
condition is reached. The statement will be executed at least once.
The prototype syntax of the \var{Repeat..until} statement is
\input{syntax/repeat.syn}
This will execute the statements between \var{repeat} and \var{until} up to
the moment when \var{Expression} evaluates to \var{True}.
Since the \var{expression} is evaluated {\em after} the execution of the
statements, they are executed at least once.
Be aware of the fact that the boolean expression \var{Expression} will be
short-cut evaluated. (Meaning that the evaluation will be stopped at the
point where the outcome is known with certainty)
The following are valid \var{repeat} statements
\begin{verbatim}
repeat
  WriteLn ('I =',i);
  I := I+2;
until I>100;
repeat
 X := X/2
until x<10e-3
\end{verbatim}
The \seep{Break} and \seep{Continue} reserved words can be used to jump to
the end or just after the end of the \var{repeat .. until } statement.

\subsection{The \var{While..do} statement}
A \var{while} statement is used to execute a statement as long as a certain
condition holds. This may imply that the statement is never executed.
The prototype syntax of the \var{While..do} statement is
\input{syntax/while.syn}
This will execute \var{Statement} as long as \var{Expression} evaluates to
\var{True}. Since \var{Expression} is evaluated {\em before} the execution
of \var{Statement}, it is possible that \var{Statement} isn't executed at
all. \var{Statement} can be a compound statement.
Be aware of the fact that the boolean expression \var{Expression} will be
short-cut evaluated. (Meaning that the evaluation will be stopped at the
point where the outcome is known with certainty)
The following are valid \var{while} statements:
\begin{verbatim}
I := I+2;
while i<=100 do
  begin
  WriteLn ('I =',i);
  I := I+2;
  end;
X := X/2;
while x>=10e-3 do
  X := X/2;
\end{verbatim}
They correspond to the example loops for the \var{repeat} statements.

If the statement is a compound statement, then  the \seep{Break} and
\seep{Continue} reserved words can be used to jump to the end or just
after the end of the \var{While} statement.

\subsection{The \var{With} statement}
\label{se:With}
The \var{with} statement serves to access the elements of a record\footnote{
The \var{with} statement does not work correctly when used with
objects or classes until version 0.99.6}
or object or class, without having to specify the name of the each time.
The syntax for a \var{with} statement is
\input{syntax/with.syn}
The variable reference must be a variable of a record, object or class type.
In the \var{with} statement, any variable reference, or method reference is
checked to see if it is a field or method of the record or object or class.
If so, then that field is accessed, or that method is called.
Given the declaration:
\begin{verbatim}
Type Passenger = Record
       Name : String[30];
       Flight : String[10];
       end;
Var TheCustomer : Passenger;
\end{verbatim}
The following statements are completely equivalent:
\begin{verbatim}
TheCustomer.Name := 'Michael';
TheCustomer.Flight := 'PS901';
\end{verbatim}
and
\begin{verbatim}
With TheCustomer do
  begin
  Name := 'Michael';
  Flight := 'PS901';
  end;
\end{verbatim}
The statement
\begin{verbatim}
With A,B,C,D do Statement;
\end{verbatim}
is equivalent to
\begin{verbatim}
With A do
 With B do
  With C do
   With D do Statement;
\end{verbatim}
This also is a clear example of the fact that the variables are tried {\em last
to first}, i.e., when the compiler encounters a variable reference, it will
first check if it is a field or method of the last variable. If not, then it
will check the last-but-one, and so on.
The following example shows this;
\begin{verbatim}
Program testw;
Type AR = record
      X,Y : Longint;
     end;
     PAR = Record;

Var S,T : Ar;
begin
  S.X := 1;S.Y := 1;
  T.X := 2;T.Y := 2;
  With S,T do
    WriteLn (X,' ',Y);
end.
\end{verbatim}
The output of this program is
\begin{verbatim}
2 2
\end{verbatim}
Showing thus that the \var{X,Y} in the \var{WriteLn} statement match the
\var{T} record variable.

\begin{remark}
If you use a \var{With} statement with a pointer, or a class, it is not
permitted to change the pointer or the class in the \var{With} block.
With the definitions of the previous example, the following illiustrates
what it is about:
\begin{verbatim}

Var p : PAR;

begin
  With P^ do
   begin
   // Do some operations
   P:=OtherP;
   X:=0.0;  // Wrong X will be used !!
   end;
\end{verbatim}
The reason the pointer cannot be changed is that the address is stored
by the compiler in a temporary register. Changing the pointer won't change
the temporary address. The same is true for classes.
\end{remark}

\subsection{Exception Statements}
As of version 0.99.7, \fpc supports exceptions. Exceptions provide a
convenient way to program error and error-recovery mechanisms, and are
closely related to classes.
Exception support is explained in \seec{Exceptions}

%%%%%%%%%%%%%%%%%%%%%%%%%%%%%%%%%%%%%%%%%%%%%%%%%%%%%%%%%%%%%%%%%%%%%%%
% Assembler statements
\section{Assembler statements}
An assembler statement allows you to insert assembler code right in your
pascal code.
\input{syntax/statasm.syn}
More information about assembler blocks can be found in the \progref.
The register list is used to indicate the registers that are modified by an
assembler statement in your code. The compiler stores certain results in the
registers. If you modify the registers in an assembler statement, the compiler
should, sometimes, be told about it. The registers are denoted with their
Intel names for the I386 processor, i.e., \var{'EAX'}, \var{'ESI'} etc...
As an example, consider the following assembler code:
\begin{verbatim}
asm
  Movl $1,%ebx
  Movl $0,%eax
  addl %eax,%ebx
end; ['EAX','EBX'];
\end{verbatim}
This will tell the compiler that it should save and restore the contents of
the \var{EAX} and \var{EBX} registers when it encounters this asm statement.

\fpc supports various styles of assembler syntax. By default, \var{AT\&T}
syntax is assumed. You can change the default assembler style with the
\var{\{\$asmmode xxx\}} switch in your code, or the \var{-R} command-line
option. More about this can be found in the \progref.


%%%%%%%%%%%%%%%%%%%%%%%%%%%%%%%%%%%%%%%%%%%%%%%%%%%%%%%%%%%%%%%%%%%%%%%
% Using functions and procedures.
%%%%%%%%%%%%%%%%%%%%%%%%%%%%%%%%%%%%%%%%%%%%%%%%%%%%%%%%%%%%%%%%%%%%%%%
\chapter{Using functions and procedures}
\label{ch:Procedures}
\fpc supports the use of functions and procedures, but with some extras:
Function overloading is supported, as well as \var{Const} parameters and
open arrays.

\begin{remark} In many of the subsequent paragraphs the words \var{procedure}
and \var{function} will be used interchangeably. The statements made are
valid for both, except when indicated otherwise.
\end{remark}

%%%%%%%%%%%%%%%%%%%%%%%%%%%%%%%%%%%%%%%%%%%%%%%%%%%%%%%%%%%%%%%%%%%%%%%
% Procedure declaration
\section{Procedure declaration}
A procedure declaration defines an identifier and associates it with a
block of code. The procedure can then be called with a procedure statement.
\input{syntax/procedur.syn}
See \sees{Parameters} for the list of parameters.
A procedure declaration that is followed by a block implements the action of
the procedure in that block.
The following is a valid procedure :
\begin{verbatim}
Procedure DoSomething (Para : String);
begin
  Writeln ('Got parameter : ',Para);
  Writeln ('Parameter in upper case : ',Upper(Para));
end;
\end{verbatim}
Note that it is possible that a procedure calls itself.

%%%%%%%%%%%%%%%%%%%%%%%%%%%%%%%%%%%%%%%%%%%%%%%%%%%%%%%%%%%%%%%%%%%%%%%
% Function declaration
\section{Function declaration}
A function declaration defines an identifier and associates it with a
block of code. The block of code will return a result.
The function can then be called inside an expression, or with a procedure
statement, if extended syntax is on.
\input{syntax/function.syn}
The result type of a function can be any previously declared type.
contrary to Turbo pascal, where only simple types could be returned.

%%%%%%%%%%%%%%%%%%%%%%%%%%%%%%%%%%%%%%%%%%%%%%%%%%%%%%%%%%%%%%%%%%%%%%%
% Parameter lists
\section{Parameter lists}
\label{se:Parameters}
When you need to pass arguments to a function or procedure, these parameters
must be declared in the formal parameter list of that function or procedure.
The parameter list is a declaration of identifiers that can be referred to
only in that procedure or function's block.
\input{syntax/params.syn}
Constant parameters and variable parameters can also be \var{untyped}
parameters if they have no type identifier.
\subsection{Value parameters}
Value parameters are declared as follows:
\input{syntax/paramval.syn}
When you declare parameters as value parameters, the procedure gets {\em
a copy} of the parameters that the calling block passes. Any modifications
to these parameters are purely local to the procedure's block, and do not
propagate back to the calling block.
A block that wishes to call a procedure with value parameters must pass
assignment compatible parameters to the procedure. This means that the types
should not match exactly, but can be converted (conversion code is inserted
by the compiler itself)

Take care that using value parameters makes heavy use of the stack,
especially if you pass large parameters. The total size of all parameters in
the formal parameter list should be below 32K for portability's sake (the
Intel version limits this to 64K).

You can pass open arrays as value parameters. See \sees{openarray} for
more information on using open arrays.
\subsection{Variable parameters}
\label{se:varparams}
Variable parameters are declared as follows:
\input{syntax/paramvar.syn}
When you declare parameters as variable parameters, the procedure or
function accesses immediatly the variable that the calling block passed in
its parameter list. The procedure gets a pointer to the variable that was
passed, and uses this pointer to access the variable's value.
From this, it follows that any changes that you make to the parameter, will
proagate back to the calling block. This mechanism can be used to pass
values back in procedures.
Because of this, the calling block must pass a parameter of {\em exactly}
the same type as the declared parameter's type. If it does not, the compiler
will generate an error.

Variable parameters can be untyped. In that case the variable has no type,
and hence is incompatible with all other types. However, you can use the
address operator on it, or you can pass it to a function that has also an
untyped parameter. If you want to use an untyped parameter in an assigment,
or you want to assign to it, you must use a typecast.

File type variables must always be passed as variable parameters.

You can pass open arrays as variable parameters. See \sees{openarray} for
more information on using open arrays.
\subsection{Constant parameters}
In addition to variable parameters and value parameters \fpc also supports
Constant parameters. You can specify a constant parameter as follows:
\input{syntax/paramcon.syn}
A constant argument is passed by reference if it's size is larger than a
longint. It is passed by value if the size equals 4 or less.
This means that the function or procedure receives a pointer to the passed
argument, but you are not allowed to assign to it, this will result in a
compiler error. Likewise, you cannot pass a const parameter on to another
function that requires a variable parameter.
The main use for this is reducing the stack size, hence improving
performance, and still retaining the semantics of passing by value...

Constant parameters can also be untyped. See \sees{varparams} for more
information about untyped parameters.

You can pass open arrays as constant parameters. See \sees{openarray} for
more information on using open arrays.
\subsection{Open array parameters}
\label{se:openarray}
\fpc supports the passing of open arrays, i.e. you can declare a procedure
with an array of unspecified length as a parameter, as in Delphi.
Open array parameters can be accessed in the procedure or function as an
array that is declared with starting index 0, and last element
index \var{High(paremeter)}.
For example, the parameter
\begin{verbatim}
Row : Array of Integer;
\end{verbatim}
would be equivalent to
\begin{verbatim}
Row : Array[0..N-1] of Integer;
\end{verbatim}
Where  \var{N} would be the actual size of the array that is passed to the
function. \var{N-1} can be calculated as \var{High(Row)}.
Open parameters can be passed by value, by reference or as a constant
parameter. In the latter cases the procedure receives a pointer to the
actual array. In the former case, it receives a copy of the array.
In a function or procedure, you can pass open arrays only to functions which
are also declared with open arrays as parameters, {\em not} to functions or
procedures which accept arrays of fixed length.
The following is an example of a function using an open array:
\begin{verbatim}
Function Average (Row : Array of integer) : Real;
Var I : longint;
    Temp : Real;
begin
  Temp := Row[0];
  For I := 1 to High(Row) do
    Temp := Temp + Row[i];
  Average := Temp / (High(Row)+1);
end;
\end{verbatim}
%%%%%%%%%%%%%%%%%%%%%%%%%%%%%%%%%%%%%%%%%%%%%%%%%%%%%%%%%%%%%%%%%%%%%%%
% The array of const construct
\subsection{Array of const}
In Object Pascal or Delphi mode, \fpc supports the \var{Array of Const}
construction to pass parameters to a subroutine.

This is a special case of the \var{Open array} construction, where you
are allowed to pass any expression in an array to a function or procedure.

In the procedure, passed the arguments can be examined using a special
record:
\begin{verbatim}
Type
   PVarRec = ^TVarRec;
   TVarRec = record
     case VType : Longint of
       vtInteger    : (VInteger: Longint);
       vtBoolean    : (VBoolean: Boolean);
       vtChar       : (VChar: Char);
       vtExtended   : (VExtended: PExtended);
       vtString     : (VString: PShortString);
       vtPointer    : (VPointer: Pointer);
       vtPChar      : (VPChar: PChar);
       vtObject     : (VObject: TObject);
       vtClass      : (VClass: TClass);
       vtAnsiString : (VAnsiString: Pointer);
       vtWideString : (VWideString: Pointer);
       vtInt64      : (VInt64: PInt64);
   end;
\end{verbatim}
Inside the procedure body, the array of const is equivalent to
an open array of TVarRec:
\begin{verbatim}
Procedure Testit (Args: Array of const);

Var I : longint;

begin
  If High(Args)<0 then
    begin
    Writeln ('No aguments');
    exit;
    end;
  Writeln ('Got ',High(Args)+1,' arguments :');
  For i:=0 to High(Args) do
    begin
    write ('Argument ',i,' has type ');
    case Args[i].vtype of
      vtinteger    :
        Writeln ('Integer, Value :',args[i].vinteger);
      vtboolean    :
        Writeln ('Boolean, Value :',args[i].vboolean);
      vtchar       :
        Writeln ('Char, value : ',args[i].vchar);
      vtextended   :
        Writeln ('Extended, value : ',args[i].VExtended^);
      vtString     :
        Writeln ('ShortString, value :',args[i].VString^);
      vtPointer    :
        Writeln ('Pointer, value : ',Longint(Args[i].VPointer));
      vtPChar      :
        Writeln ('PCHar, value : ',Args[i].VPChar);
      vtObject     :
        Writeln ('Object, name : ',Args[i].VObject.Classname);
      vtClass      :
        Writeln ('Class reference, name :',Args[i].VClass.Classname);
      vtAnsiString :
        Writeln ('AnsiString, value :',AnsiString(Args[I].VAnsiStr
    else
        Writeln ('(Unknown) : ',args[i].vtype);
    end;
    end;
end;
\end{verbatim}
In your code, it is possible to pass an arbitrary array of elements
to this procedure:
\begin{verbatim}
  S:='Ansistring 1';
  T:='AnsiString 2';
  Testit ([]);
  Testit ([1,2]);
  Testit (['A','B']);
  Testit ([TRUE,FALSE,TRUE]);
  Testit (['String','Another string']);
  Testit ([S,T])  ;
  Testit ([P1,P2]);
  Testit ([@testit,Nil]);
  Testit ([ObjA,ObjB]);
  Testit ([1.234,1.234]);
  TestIt ([AClass]);
\end{verbatim}

If the procedure is declared with the \var{cdecl} modifier, then the
compiler will pass the array as a C compiler would pass it. This, in effect,
emulates the C construct of a varable number of arguments, as the following
example will show:
\begin{verbatim}
program testaocc;
{$mode objfpc}

Const
  P : Pchar = 'example';
  Fmt : PChar =
        'This %s uses printf to print numbers (%d) and strings.'#10;

// Declaration of standard C function printf:
procedure printf (fm : pchar; args : array of const);cdecl; external 'c';

begin
 printf(Fmt,[P,123]);
end.
\end{verbatim}
Remark that this is not true for Delphi, so code relying on this feature
will not be portable.

%%%%%%%%%%%%%%%%%%%%%%%%%%%%%%%%%%%%%%%%%%%%%%%%%%%%%%%%%%%%%%%%%%%%%%%
% Function overloading
\section{Function overloading}
Function overloading simply means that you can define the same function more
than once, but each time with a different formal parameter list.
The parameter lists must differ at least in one of it's elements type.
When the compiler encounters a function call, it will look at the function
parameters to decide which one of the defined functions it should call.
This can be useful if you want to define the same function for different
types. For example, in the RTL, the  \var{Dec} procedure is
is defined as:
\begin{verbatim}
...
Dec(Var I : Longint;decrement : Longint);
Dec(Var I : Longint);
Dec(Var I : Byte;decrement : Longint);
Dec(Var I : Byte);
...
\end{verbatim}
When the compiler encounters a call to the dec function, it will first search
which function it should use. It therefore checks the parameters in your
function call, and looks if there is a function definition which matches the
specified parameter list. If the compiler finds such a function, a call is
inserted to that function. If no such function is found, a compiler error is
generated.
You cannot have overloaded functions that have a \var{cdecl} or \var{export}
modifier (Technically, because these two modifiers prevent the mangling of
the function name by the compiler).

%%%%%%%%%%%%%%%%%%%%%%%%%%%%%%%%%%%%%%%%%%%%%%%%%%%%%%%%%%%%%%%%%%%%%%%
% forward defined functions
\section{Forward defined functions}
You can define a function without having it followed by it's implementation,
by having it followed by the \var{forward} procedure. The effective
implementation of that function must follow later in the module.
The function can be used after a \var{forward} declaration as if it had been
implemented already.
The following is an example of a forward declaration.
\begin{verbatim}
Program testforward;
Procedure First (n : longint); forward;
Procedure Second;
begin
  WriteLn ('In second. Calling first...');
  First (1);
end;
Procedure First (n : longint);
begin
  WriteLn ('First received : ',n);
end;
begin
  Second;
end.
\end{verbatim}
You cannot define a function twice as forward (nor is there any reason why
you would want to do that).
Likewise, in units, you cannot have a forward declared function of a
function that has been declared in the interface part. The interface
declaration counts as a \var{forward} declaration.
The following unit will give an error when compiled:
\begin{verbatim}
Unit testforward;
interface
Procedure First (n : longint);
Procedure Second;
implementation
Procedure First (n : longint); forward;
Procedure Second;
begin
  WriteLn ('In second. Calling first...');
  First (1);
end;
Procedure First (n : longint);
begin
  WriteLn ('First received : ',n);
end;
end.
\end{verbatim}

%%%%%%%%%%%%%%%%%%%%%%%%%%%%%%%%%%%%%%%%%%%%%%%%%%%%%%%%%%%%%%%%%%%%%%%
% External functions
\section{External functions}
\label{se:external}
The \var{external} modifier can be used to declare a function that resides in
an external object file. It allows you to use the function in
your code, and at linking time, you must link the object file containing the
implementation of the function or procedure.
\input{syntax/external.syn}
It replaces, in effect, the function or procedure code block. As such, it
can be present only in an implementation block of a unit, or in a program.
As an example:
\begin{verbatim}
program CmodDemo;
{$Linklib c}
Const P : PChar = 'This is fun !';
Function strlen (P : PChar) : Longint; cdecl; external;
begin
  WriteLn ('Length of (',p,') : ',strlen(p))
end.
\end{verbatim}
\begin{remark}
The parameters in our declaration of the \var{external} function
should match exactly the ones in the declaration in the object file.
\end{remark}
If the \var{external} modifier is followed by a string constant:
\begin{verbatim}
external 'lname';
\end{verbatim}
Then this tells the compiler that the function resides in library
'lname'. The compiler will then automatically link this library to
your program.

You can also specify the name that the function has in the library:
\begin{verbatim}
external 'lname' name Fname;
\end{verbatim}
This tells the compiler that the function resides in library 'lname',
but with name 'Fname'. The compiler will then automatically link this
library to your program, and use the correct name for the function.
Under \windows and \ostwo, you can also use the following form:
\begin{verbatim}
external 'lname' Index Ind;
\end{verbatim}
This tells the compiler that the function resides in library 'lname',
but with index \var{Ind}. The compiler will then automatically
link this library to your program, and use the correct index for the
function.

%%%%%%%%%%%%%%%%%%%%%%%%%%%%%%%%%%%%%%%%%%%%%%%%%%%%%%%%%%%%%%%%%%%%%%%
% Assembler functions
\section{Assembler functions}
Functions and procedures can be completely implemented in assembly
language. To indicate this, you use the \var{assembler} keyword:
\input{syntax/asm.syn}
Contrary to Delphi, the assembler keyword must be present to indicate an
assembler function.
For more information about assembler functions, see the chapter on using
assembler in the \progref.


%%%%%%%%%%%%%%%%%%%%%%%%%%%%%%%%%%%%%%%%%%%%%%%%%%%%%%%%%%%%%%%%%%%%%%%
% Modifiers
\section{Modifiers}
A function or procedure declaration can contain modifiers. Here we list the
various possibilities:
\input{syntax/modifiers.syn}
\fpc doesn't support all Turbo Pascal modifiers, but
does support a number of additional modifiers. They are used mainly for assembler and
reference to C object files. More on the use of modifiers can be found in
the \progref.

\subsection{Public}
The \var{Public} keyword is used to declare a function globally in a unit.
This is useful if you don't want a function to be accessible from the unit
file, but you do want the function to be accessible from the object file.
as an example:
\begin{verbatim}
Unit someunit;
interface
Function First : Real;
Implementation
Function First : Real;
begin
  First := 0;
end;
Function Second : Real; [Public];
begin
  Second := 1;
end;
end.
\end{verbatim}
If another program or unit uses this unit, it will not be able to use the
function \var{Second}, since it isn't declared in the interface part.
However, it will be possible to access the function \var{Second} at the
assembly-language level, by using it's mangled name (see the \progref).

\subsection{cdecl}
\label{se:cdecl}
The \var{cdecl} modifier can be used to declare a function that uses a C
type calling convention. This must be used if you wish to acces functions in
an object file generated by a C compiler. It allows you to use the function in
your code, and at linking time, you must link the object file containing the
\var{C} implementation of the function or procedure.
As an example:
\begin{verbatim}
program CmodDemo;
{$LINKLIB c}
Const P : PChar = 'This is fun !';
Function strlen (P : PChar) : Longint; cdecl; external;
begin
  WriteLn ('Length of (',p,') : ',strlen(p))
end.
\end{verbatim}
When compiling this, and linking to the C-library, you will be able to call
the \var{strlen} function throughout your program. The \var{external}
directive tells the compiler that the function resides in an external
object filebrary (see \ref{se:external}).
\begin{remark}
The parameters in our declaration of the \var{C} function should
match exactly the ones in the declaration in \var{C}. Since \var{C} is case
sensitive, this means also that the name of the
function must be exactly the same. the \fpc compiler will use the name {\em
exactly} as it is typed in the declaration.
\end{remark}

\subsection{popstack}
\label{se:popstack}
Popstack does the same as \var{cdecl}, namely it tells the \fpc compiler
that a function uses the C calling convention. In difference with the
\var{cdecl} modifier, it still mangles the name of the function as it would
for a normal pascal function.
With \var{popstack} you could access functions by their pascal names in a
library.

\subsection{Export}
Sometimes you must provide a callback function for a C library, or you want
your routines to be callable from a C program. Since \fpc and C use
different calling schemes for functions and procedures\footnote{More
techically: In C the calling procedure must clear the stack. In \fpc, the
subroutine clears the stack.}, the compiler must be told to generate code
that can be called from a C routine. This is where the \var{Export} modifier
comes in. Contrary to the other modifiers, it must be specified separately,
as follows:
\begin{verbatim}
function DoSquare (X : Longint) : Longint; export;
begin
...
end;
\end{verbatim}
The square brackets around the modifier are not allowed in this case.
\begin{remark}
as of version 0.9.8, \fpc supports the Delphi \var{cdecl} modifier.
This modifier works in the same way as the \var{export} modifier.
More information about these modifiers can be found in the \progref, in the
section on the calling mechanism and the chapter on linking.
\end{remark}

\subsection{StdCall}
As of version 0.9.8, \fpc supports the Delphi \var{stdcall} modifier.
This modifier does actually nothing, since the \fpc compiler by default
pushes parameters from right to left on the stack, which is what the
modifier does under Delphi (which pushes parameters on the stack from left to
right).
More information about this modifier can be found in the \progref, in the
section on the calling mechanism and the chapter on linking.

\subsection{saveregisters}
As of version 0.99.15, \fpc has the \var{saveregisters} modifier. If this
modifier is specified after a procedure or function, then the \fpc compiler
will save all registers on procedure entry, and restore them when the
procedure exits (except for registers where return values are stored).

You should not need this modifier, except maybe when calling assembler code.

\subsection{Alias}
The \var{Alias} modifier allows you to specify a different name for a
procedure or function. This is mostly useful for referring to this procedure
from assembly language constructs. As an example, consider the following
program:
\begin{verbatim}
Program Aliases;
Procedure Printit; [Alias : 'DOIT'];
begin
  WriteLn ('In Printit (alias : "DOIT")');
end;
begin
  asm
  call DOIT
  end;
end.
\end{verbatim}
\begin{remark} the specified alias is inserted straight into the assembly
code, thus it is case sensitive.
\end{remark}
The \var{Alias} modifier, combined with the \var{Public} modifier, make a
powerful tool for making externally accessible object files.

%%%%%%%%%%%%%%%%%%%%%%%%%%%%%%%%%%%%%%%%%%%%%%%%%%%%%%%%%%%%%%%%%%%%%%%
% Unsupported Turbo Pascal modifiers
\section{Unsupported Turbo Pascal modifiers}
The modifiers that exist in Turbo pascal, but aren't supported by \fpc, are
listed in \seet{Modifs}.
\begin{FPCltable}{lr}{Unsupported modifiers}{Modifs}
Modifier & Why not supported ? \\ \hline
Near & \fpc is a 32-bit compiler.\\
Far & \fpc is a 32-bit compiler. \\
%External & Replaced by \var{C} modifier. \\ \hline
\end{FPCltable}

%%%%%%%%%%%%%%%%%%%%%%%%%%%%%%%%%%%%%%%%%%%%%%%%%%%%%%%%%%%%%%%%%%%%%%%
% Operator overloading
%%%%%%%%%%%%%%%%%%%%%%%%%%%%%%%%%%%%%%%%%%%%%%%%%%%%%%%%%%%%%%%%%%%%%%%
\chapter{Operator overloading}
\label{ch:operatoroverloading}

\section{Introduction}
\fpc supports operator overloading. This means that it is possible to
define the action of some operators on self-defined types, and thus allow
the use of these types in mathematical expressions.

Defining the action of an operator is much like the definition of a
function or procedure, only there are some restrictions on the possible
definitions, as will be shown in the subsequent.

Operator overloading is, in essence, a powerful notational tool;
but it is also not more than that, since the same results can be
obtained with regular function calls. When using operator overloading,
It is important to keep in mind that some implicit rules may produce
some unexpected results. This will be indicated.

\section{Operator declarations}
To define the action of an operator is much like defining a function:
\input{syntax/operator.syn}
The parameter list for a comparision operator or an arithmetic operator
must always contain 2 parameters. The result type of the comparision
operator must be \var{Boolean}.

\begin{remark}
When compiling in \var{Delphi} mode or \var{Objfpc} mode, the result
identifier may be dropped. The result can then be accessed through
the standard \var{Result} symbol.

If the result identifier is dropped and the compiler is not in one
of these modes, a syntax error will occur.
\end{remark}

The statement block contains the necessary statements to determine the
result of the operation. It can contain arbitrary large pieces of code;
it is executed whenever the operation is encountered in some expression.
The result of the statement block must always be defined; error conditions
are not checked by the compiler, and the code must take care of all possible
cases, throwing a run-time error if some error condition is encountered.

In the following, the three types of operator definitions will be examined.
As an example, throughout this chapter the following type will be used to
define overloaded operators on :
\begin{verbatim}
type
  complex = record
    re : real;
    im : real;
  end;
\end{verbatim}
this type will be used in all examples.

The sources of the Run-Time Library contain a unit \file{ucomplex},
which contains a complete calculus for complex numbers, based on
operator overloading.

\section{Assignment operators}

The assignment operator defines the action of a assignent of one type of
variable to another. The result type must match the type of the variable
at the left of the assignment statement, the single parameter to the
assignment operator must have the same type as the expression at the
right of the assignment operator.

This system can be used to declare a new type, and define an assignment for
that type. For instance, to be able to assign a newly defined type 'Complex'
\begin{verbatim}
Var
  C,Z : Complex; // New type complex

begin
  Z:=C;  // assignments between complex types.
end;
\end{verbatim}
You would have to define the following assignment operator:
\begin{verbatim}
Operator := (C : Complex) z : complex;
\end{verbatim}


To be able to assign a real type to a complex type as follows:
\begin{verbatim}
var
  R : real;
  C : complex;

begin
  C:=R;
end;
\end{verbatim}
the following assignment operator must be defined:
\begin{verbatim}
Operator := (r : real) z : complex;
\end{verbatim}
As can be seen from this statement, it defines the action of the operator
\var{:=} with at the right a real expression, and at the left a complex
expression.

an example implementation of this could be as follows:
\begin{verbatim}
operator := (r : real) z : complex;

begin
  z.re:=r;
  z.im:=0.0;
end;
\end{verbatim}
As can be seen in the example, the result identifier (\var{z} in this case)
is used to store the result of the assignment. When compiling in Delphi mode
or objfpc mode, the use of the special identifier \var{Result} is also
allowed, and can be substituted for the \var{z}, so the above would be
equivalent to
\begin{verbatim}
operator := (r : real) z : complex;

begin
  Result.re:=r;
  Result.im:=0.0;
end;
\end{verbatim}

The assignment operator is also used to convert types from one type to
another. The compiler will consider all overloaded assignment operators
till it finds one that matches the types of the left hand and right hand
expressions. If no such operator is found, a 'type mismatch' error
is given.

\begin{remark}
The assignment operator is not commutative; the compiler will never reverse
the role of the two arguments. in other words, given the above definition of
the assignment operator, the following is {\em not} possible:
\begin{verbatim}
var
  R : real;
  C : complex;

begin
  R:=C;
end;
\end{verbatim}
if the reverse assignment should be possible (this is not so for reals and
complex numbers) then the assigment operator must be defined for that as well.
\end{remark}

\begin{remark}
The assignment operator is also used in implicit type conversions. This can
have unwanted effects. Consider the following definitions:
\begin{verbatim}
operator := (r : real) z : complex;
function exp(c : complex) : complex;
\end{verbatim}
then the following assignment will give a type mismatch:
\begin{verbatim}
Var
  r1,r2 : real;

begin
  r1:=exp(r2);
end;
\end{verbatim}
because the compiler will encounter the definition of the \var{exp} function
with the complex argument. It implicitly converts r2 to a complex, so it can
use the above \var{exp} function. The result of this function is a complex,
which cannot be assigned to r1, so the compiler will give a 'type mismatch'
error. The compiler will not look further for another \var{exp} which has
the correct arguments.

It is possible to avoid this particular problem by specifying
\begin{verbatim}
  r1:=system.exp(r2);
\end{verbatim}
An experimental solution for this problem exists in the compiler, but is
not enabled by default. Maybe someday it will be.
\end{remark}

\section{Arithmetic operators}

Arithmetic operators define the action of a binary operator. Possible
operations are:
\begin{description}
\item[multiplication] to multiply two types, the \var{*} multiplication
operator must be overloaded.
\item[division] to divide two types, the \var{/} division
operator must be overloaded.
\item[addition] to add two types, the \var{+} addition
operator must be overloaded.
\item[substraction] to substract two types, the \var{-} substraction
operator must be overloaded.
\item[exponentiation] to exponentiate two types, the \var{**} exponentiation
operator must be overloaded.
\end{description}

The definition of an arithmetic operator takes two parameters. The first
parameter must be of the type that occurs at the left of the operator,
the second parameter must be of the type that is at the right of the
arithmetic operator. The result type must match the type that results
after the arithmetic operation.

To compile an expression as
\begin{verbatim}
var
  R : real;
  C,Z : complex;

begin
  C:=R*Z;
end;
\end{verbatim}
one needs a definition of the multiplication operator as:
\begin{verbatim}
Operator * (r : real; z1 : complex) z : complex;

begin
  z.re := z1.re * r;
  z.im := z1.im * r;
end;
\end{verbatim}
As can be seen, the first operator is a real, and the second is
a complex. The result type is complex.

Multiplication and addition of reals and complexes are commutative
operations. The compiler, however, has no notion of this fact so even
if a multiplication between a real and a complex is defined, the
compiler will not use that definition when it encounters a complex
and a real (in that order). It is necessary to define both operations.

So, given the above definition of the multiplication,
the compiler will not accept the following statement:
\begin{verbatim}
var
  R : real;
  C,Z : complex;

begin
  C:=Z*R;
end;
\end{verbatim}
since the types of \var{Z} and \var{R} don't match the types in the
operator definition.

The reason for this behaviour is that it is possible that a multiplication
is not always commutative. e.g. the multiplication of a \var{(n,m)} with a
\var{(m,n)} matrix will result in a \var{(n,n)} matrix, while the
mutiplication of a \var{(m,n)} with a \var{(n,m)} matrix is a \var{(m,m)}
matrix, which needn't be the same in all cases.

\section{Comparision operator}
The comparision operator can be overloaded to compare two different types
or to compare two equal types that are not basic types. The result type of
a comparision operator is always a boolean.

The comparision operators that can be overloaded are:
\begin{description}
\item[equal to] (=) to determine if two variables are equal.
\item[less than] ($<$) to determine if one variable is less than another.
\item[greater than] ($>$) to determine if one variable is greater than another.
\item[greater than or equal to] ($>=$) to determine if one variable is greater than
or equal to another.
\item[less than or equal to] ($<=$) to determine if one variable is greater
than or equal to another.
\end{description}
There is no separate operator for {\em unequal to} ($<>$). To evaluate a
statement that contans the {\em unequal to} operator, the compiler uses the
{\em equal to} operator (=), and negates the result.


As an example, the following opetrator allows to compare two complex
numbers:
\begin{verbatim}
operator = (z1, z2 : complex) b : boolean;
\end{verbatim}
the above definition allows comparisions of the following form:
\begin{verbatim}
Var
  C1,C2 : Complex;

begin
  If C1=C2 then
    Writeln('C1 and C2 are equal');
end;
\end{verbatim}

The comparision operator definition needs 2 parameters, with the types that
the operator is meant to compare. Here also, the compiler doesn't apply
commutativity; if the two types are different, then it necessary to
define 2 comparision operators.

In the case of complex numbers, it is, for instance necessary to define
2 comparsions: one with the complex type first, and one with the real type
first.

Given the definitions
\begin{verbatim}
operator = (z1 : complex;r : real) b : boolean;
operator = (r : real; z1 : complex) b : boolean;
\end{verbatim}
the following two comparisions are possible:
\begin{verbatim}
Var
  R,S : Real;
  C : Complex;

begin
  If (C=R) or (S=C) then
   Writeln ('Ok');
end;
\end{verbatim}
Note that the order of the real and complex type in the two comparisions
is reversed.

%%%%%%%%%%%%%%%%%%%%%%%%%%%%%%%%%%%%%%%%%%%%%%%%%%%%%%%%%%%%%%%%%%%%%%%
% Programs, Units, Blocks
%%%%%%%%%%%%%%%%%%%%%%%%%%%%%%%%%%%%%%%%%%%%%%%%%%%%%%%%%%%%%%%%%%%%%%%

\chapter{Programs, units, blocks}
A Pascal program consists of modules called \var{units}. A unit can be used
to group pieces of code together, or to give someone code without giving
the sources.
Both programs and units consist of code blocks, which are mixtures of
statements, procedures, and variable or type declarations.

%%%%%%%%%%%%%%%%%%%%%%%%%%%%%%%%%%%%%%%%%%%%%%%%%%%%%%%%%%%%%%%%%%%%%%%
% Programs
\section{Programs}
A pascal program consists of the program header, followed possibly by a
'uses' clause, and a block.
\input{syntax/program.syn}
The program header is provided for backwards compatibility, and is ignored
by the compiler.
The uses clause serves to identify all units that are needed by the program.
The system unit doesn't have to be in this list, since it is always loaded
by the compiler.
The order in which the units appear is significant, it determines in
which order they are initialized. Units are initialized in the same order
as they appear in the uses clause. Identifiers are searched in the opposite
order, i.e. when the compiler searches for an identifier, then it looks
first in the last unit in the uses clause, then the last but one, and so on.
This is important in case two units declare different types with the same
identifier.
When the compiler looks for unit files, it adds the extension \file{.ppu}
(\file{.ppw} for Win32 platforms) to the name of the unit. On \linux, unit names
are converted to all lowercase when looking for a unit.

If a unit name is longer than 8 characters, the compiler will first look for
a unit name with this length, and then it will truncate the name to 8
characters and look for it again. For compatibility reasons, this is also
true on platforms that suport long file names.

%%%%%%%%%%%%%%%%%%%%%%%%%%%%%%%%%%%%%%%%%%%%%%%%%%%%%%%%%%%%%%%%%%%%%%%
% Units
\section{Units}
A unit contains a set of declarations, procedures and functions that can be
used by a program or another unit.
The syntax for a unit is as follows:
\input{syntax/unit.syn}
The interface part declares all identifiers that must be exported from the
unit. This can be constant, type or variable identifiers, and also procedure
or function identifier declarations. Declarations inside the
implementation part are {\em not} accessible outside the unit. The
implementation must contain a function declaration for each function or
procedure that is declared in the interface part. If a function is declared
in the interface part, but no declaration of that function is present in the
implementation part, then the compiler will give an error.

When a program uses a unit (say \file{unitA}) and this units uses a second
unit, say \file{unitB}, then the program depends indirectly also on
\var{unitB}. This means that the compiler must have access to \file{unitB} when
trying to compile the program. If the unit is not present at compile time,
an error occurs.

Note that the identifiers from a unit on which a program depends indirectly,
are not accessible to the program. To have access to the identifiers of a
unit, you must put that unit in the uses clause of the program or unit where
you want to yuse the identifier.

Units can be mutually dependent, that is, they can reference each other in
their uses clauses. This is allowed, on the condition that at least one of
the references is in the implementation section of the unit. This also holds
for indirect mutually dependent units.

If it is possible to start from one interface uses clause of a unit, and to return
there via uses clauses of interfaces only, then there is circular unit
dependence, and the compiler will generate an error.
As and example : the following is not allowed:
\begin{verbatim}
Unit UnitA;
interface
Uses UnitB;
implementation
end.

Unit UnitB
interface
Uses UnitA;
implementation
end.
\end{verbatim}
But this is allowed :
\begin{verbatim}
Unit UnitA;
interface
Uses UnitB;
implementation
end.
Unit UnitB
implementation
Uses UnitA;
end.
\end{verbatim}
Because \file{UnitB} uses \file{UnitA} only in it's implentation section.
In general, it is a bad idea to have circular unit dependencies, even if it is
only in implementation sections.

%%%%%%%%%%%%%%%%%%%%%%%%%%%%%%%%%%%%%%%%%%%%%%%%%%%%%%%%%%%%%%%%%%%%%%%
% Blocks
\section{Blocks}
Units and programs are made of blocks. A block is made of declarations of
labels, constants, types variables and functions or procedures. Blocks can
be nested in certain ways, i.e., a procedure or function declaration can
have blocks in themselves.
A block looks like the following:
\input{syntax/block.syn}
Labels that can be used to identify statements in a block are declared in
the label declaration part of that block. Each label can only identify one
statement.
Constants that are to be used only in one block should be declared in that
block's constant declaration part.
Variables that are to be used only in one block should be declared in that
block's constant declaration part.
Types that are to be used only in one block should be declared in that
block's constant declaration part.
Lastly, functions and procedures that will be used in that block can be
declared in the procedure/function declaration part.
After the different declaration parts comes the statement part. This
contains any actions that the block should execute.
All identifiers declared before the statement part can be used in that
statement part.

%%%%%%%%%%%%%%%%%%%%%%%%%%%%%%%%%%%%%%%%%%%%%%%%%%%%%%%%%%%%%%%%%%%%%%%
% Scope
\section{Scope}
Identifiers are valid from the point of their declaration until the end of
the block in which the declaration occurred. The range where the identifier
is known is the {\em scope} of the identifier. The exact scope of an
identifier depends on the way it was defined.
\subsection{Block scope}
The {\em scope} of a variable declared in the declaration part of a block,
is valid from the point of declaration until the end of the block.
If a block contains a second block, in which the identfier is
redeclared, then inside this block, the second declaration will be valid.
Upon leaving the inner block, the first declaration is valid again.
Consider the following example:
\begin{verbatim}
Program Demo;
Var X : Real;
{ X is real variable }
Procedure NewDeclaration
Var X : Integer;  { Redeclare X as integer}
begin
 // X := 1.234; {would give an error when trying to compile}
 X := 10; { Correct assigment}
end;
{ From here on, X is Real again}
begin
 X := 2.468;
end.
\end{verbatim}
In this example, inside the procedure, X denotes an integer variable.
It has it's own storage space, independent of the variable \var{X} outside
the procedure.
\subsection{Record scope}
The field identifiers inside a record definition are valid in the following
places:
\begin{enumerate}
\item to the end of the record definition.
\item field designators of a variable of the given record type.
\item identifiers inside a \var{With} statement that operates on a variable
of the given record type.
\end{enumerate}
\subsection{Class scope}
A component identifier is valid in the following places:
\begin{enumerate}
\item From the point of declaration to the end of the class definition.
\item In all descendent types of this class, unless it is in the private
part of the class declaration.
\item In all method declaration blocks of this class and descendent classes.
\item In a with statement that operators on a variable of the given class's
definition.
\end{enumerate}
Note that method designators are also considered identifiers.
\subsection{Unit scope}
All identifiers in the interface part of a unit are valid from the point of
declaration, until the end of the unit. Furthermore, the identifiers are
known in programs or units that have the unit in their uses clause.
Identifiers from indirectly dependent units are {\em not} available.
Identifiers declared in the implementation part of a unit are valid from the
point of declaration to the end of the unit.
The system unit is automatically used in all units and programs.
It's identifiers are therefore always known, in each program or unit
you make.
The rules of unit scope implie that you can redefine an identifier of a
unit. To have access to an identifier of another unit that was redeclared in
the current unit, precede it with that other units name, as in the following
example:
\begin{verbatim}
unit unitA;
interface
Type
  MyType = Real;
implementation
end.
Program prog;
Uses UnitA;

{ Redeclaration of MyType}
Type MyType = Integer;
Var A : Mytype;      { Will be Integer }
    B : UnitA.MyType { Will be real }
begin
end.
\end{verbatim}
This is especially useful if you redeclare the system unit's identifiers.

%%%%%%%%%%%%%%%%%%%%%%%%%%%%%%%%%%%%%%%%%%%%%%%%%%%%%%%%%%%%%%%%%%%%%%%
% Libraries
\section{Libraries}
\fpc supports making of dynamic libraries (DLLs under Win32 and \ostwo) trough
the use of the \var{Library} keyword.

A Library is just like a unit or a program:
\input{syntax/library.syn}

By default, functions and procedures that are declared and implemented in
library are not available to a programmer that wishes to use your library.

In order to make functions or procedures available from the library,
you must export them in an export clause:

\input{syntax/exports.syn}

Under Win32, an index clause can be added to an exports entry.
an index entry must be a positive number larger or equal than 1.
It is best to use low index values, although nothing forces you to
do this.

Optionally, an exports entry can have a name specifier. If present, the name
specifier gives the exact name (case sensitive) of the function in the
library.

If neither of these constructs is present, the functions or procedures
are exported with the exact names as specified in the exports clause.



%%%%%%%%%%%%%%%%%%%%%%%%%%%%%%%%%%%%%%%%%%%%%%%%%%%%%%%%%%%%%%%%%%%%%%%
% Exceptions

%%%%%%%%%%%%%%%%%%%%%%%%%%%%%%%%%%%%%%%%%%%%%%%%%%%%%%%%%%%%%%%%%%%%%%%
\chapter{Exceptions}
\label{ch:Exceptions}
As of version 0.99.7, \fpc supports exceptions. Exceptions provide a
convenient way to program error and error-recovery mechanisms, and are
closely related to classes.
Exception support is based on 3 constructs:
\begin{description}
\item [Raise\ ] statements. To raise an exeption. This is usually done to signal an
error condition.
\item [Try ... Except\ ] blocks. These block serve to catch exceptions
raised within the scope of the block, and to provide exception-recovery
code.
\item [Try ... Finally\ ] blocks. These block serve to force code to be
executed irrespective of an exception occurrence or not. They generally
serve to clean up memory or close files in case an exception occurs.
The compiler generates many implicit \var{Try ... Finally} blocks around
procedure, to force memory consistence.
\end{description}


%%%%%%%%%%%%%%%%%%%%%%%%%%%%%%%%%%%%%%%%%%%%%%%%%%%%%%%%%%%%%%%%%%%%%%%
% The raise statement
\section{The raise statement}
The \var{raise} statement is as follows:
\input{syntax/raise.syn}
This statement will raise an exception. If it is specified, the exception
instance must be an initialized instance of a class, which is the raise
type. The address exception is optional. If itis not specified, the compiler
will provide the address by itself.
If the exception instance is omitted, then the current exception is
re-raised. This construct can only be used in an exception handling
block (see further).

\begin{remark} Control {\em never} returns after an exception block. The
control is transferred to the first \var{try...finally} or
\var{try...except} statement that is encountered when unwinding the stack.
If no such statement is found, the \fpc Run-Time Library will generate a
run-time error 217 (see also \sees{exceptclasses}).
\end{remark}

As an example: The following division checks whether the denominator is
zero, and if so, raises an exception of type \var{EDivException}
\begin{verbatim}
Type EDivException = Class(Exception);
Function DoDiv (X,Y : Longint) : Integer;
begin
  If Y=0 then
    Raise EDivException.Create ('Division by Zero would occur');
  Result := X Div Y;
end;
\end{verbatim}
The class \var{Exception} is defined in the \file{Sysutils} unit of the rtl.
(\sees{exceptclasses})

%%%%%%%%%%%%%%%%%%%%%%%%%%%%%%%%%%%%%%%%%%%%%%%%%%%%%%%%%%%%%%%%%%%%%%%
% The try...except statement
\section{The try...except statement}
A \var{try...except} exception handling block is of the following form :
\input{syntax/try.syn}
If no exception is raised during the execution of the \var{statement list},
then all statements in the list will be executed sequentially, and the
except block will be skipped, transferring program flow to the statement
after the final \var{end}.

If an exception occurs during the execution of the \var{statement list}, the
program flow will be transferred to the except block. Statements in the
statement list between the place where the exception was raised and the
exception block are ignored.

In the exception handling block, the type of the exception is checked,
and if there is an exception handler where the class type matches the
exception object type, or is a parent type of
the exception object type, then the statement following the corresponding
\var{Do} will be executed. The first matching type is used. After the
\var{Do} block was executed, the program continues after the \var{End}
statement.

The identifier in an exception handling statement is optional, and declares
an exception object. It can be used to manipulate the exception object in
the exception handling code. The scope of this declaration is the statement
block foillowing the \var{Do} keyword.

If none of the \var{On} handlers matches the exception object type, then the
statement list after \var{else} is executed. If no such list is
found, then the exception is automatically re-raised. This process allows
to nest \var{try...except} blocks.

If, on the other hand, the exception was caught, then the exception object is
destroyed at the end of the exception handling block, before program flow
continues. The exception is destroyed through a call to the object's
\var{Destroy} destructor.

As an example, given the previous declaration of the \var{DoDiv} function,
consider the following
\begin{verbatim}
Try
  Z := DoDiv (X,Y);
Except
  On EDivException do Z := 0;
end;
\end{verbatim}
If \var{Y} happens to be zero, then the DoDiv function code will raise an
exception. When this happens, program flow is transferred to the except
statement, where the Exception handler will set the value of \var{Z} to
zero. If no exception is raised, then program flow continues past the last
\var{end} statement.
To allow error recovery, the \var{Try ... Finally} block is supported.
A \var{Try...Finally} block ensures that the statements following the
\var{Finally} keyword are guaranteed to be executed, even if an exception
occurs.


%%%%%%%%%%%%%%%%%%%%%%%%%%%%%%%%%%%%%%%%%%%%%%%%%%%%%%%%%%%%%%%%%%%%%%%
% The try...finally statement
\section{The try...finally statement}
A \var{Try..Finally} statement has the following form:
\input{syntax/finally.syn}
If no exception occurs inside the \var{statement List}, then the program
runs as if the \var{Try}, \var{Finally} and \var{End} keywords were not
present.

If, however, an exception occurs, the program flow is immediatly
transferred from the point where the excepion was raised to the first
statement of the \var{Finally statements}.

All statements after the finally keyword will be executed, and then
the exception will be automatically re-raised. Any statements between the
place where the exception was raised and the first statement of the
\var{Finally Statements} are skipped.

As an example consider the following routine:
\begin{verbatim}
Procedure Doit (Name : string);
Var F : Text;
begin
  Try
    Assign (F,Name);
    Rewrite (name);
    ... File handling ...
  Finally
    Close(F);
  end;
\end{verbatim}
If during the execution of the file handling an execption occurs, then
program flow will continue at the \var{close(F)} statement, skipping any
file operations that might follow between the place where the exception
was raised, and the \var{Close} statement.
If no exception occurred, all file operations will be executed, and the file
will be closed at the end.


%%%%%%%%%%%%%%%%%%%%%%%%%%%%%%%%%%%%%%%%%%%%%%%%%%%%%%%%%%%%%%%%%%%%%%%
% Exception handling nesting
\section{Exception handling nesting}
It is possible to nest \var{Try...Except} blocks with \var{Try...Finally}
blocks. Program flow will be done according to a \var{lifo} (last in, first
out) principle: The code of the last encountered \var{Try...Except} or
 \var{Try...Finally} block will be executed first. If the exception is not
caught, or it was a finally statement, program flow will be transferred to
the last-but-one block, {\em ad infinitum}.

If an exception occurs, and there is no exception handler present, then a
runerror 217 will be generated. If you use the \file{sysutils} unit, a default
handler is installed which will show the exception object message, and the
address where the exception occurred, after which the program will exit with
a \var{Halt} instruction.


%%%%%%%%%%%%%%%%%%%%%%%%%%%%%%%%%%%%%%%%%%%%%%%%%%%%%%%%%%%%%%%%%%%%%%%
% Exception classes
\section{Exception classes}
\label{se:exceptclasses}
The \file{sysutils} unit contains a great deal of exception handling.
It defines the following exception types:
\begin{verbatim}
       Exception = class(TObject)
        private
          fmessage : string;
          fhelpcontext : longint;
        public
          constructor create(const msg : string);
          constructor createres(indent : longint);
          property helpcontext : longint read fhelpcontext write fhelpcontext;
          property message : string read fmessage write fmessage;
       end;
       ExceptClass = Class of Exception;
       { mathematical exceptions }
       EIntError = class(Exception);
       EDivByZero = class(EIntError);
       ERangeError = class(EIntError);
       EIntOverflow = class(EIntError);
       EMathError = class(Exception);
\end{verbatim}
The sysutils unit also installs an exception handler. If an exception is
unhandled by any exception handling block, this handler is called by the
Run-Time library. Basically, it prints the exception address, and it prints
the message of the Exception object, and exits with a exit code of 217.
If the exception object is not a descendent object of the \var{Exception}
object, then the class name is printed instead of the exception message.

It is recommended to use the \var{Exception} object or a descendant class for
all \var{raise} statements, since then you can use the message field of the
exception object.

%%%%%%%%%%%%%%%%%%%%%%%%%%%%%%%%%%%%%%%%%%%%%%%%%%%%%%%%%%%%%%%%%%%%%%%
% Using Assembler
%%%%%%%%%%%%%%%%%%%%%%%%%%%%%%%%%%%%%%%%%%%%%%%%%%%%%%%%%%%%%%%%%%%%%%%
\chapter{Using assembler}
\fpc supports the use of assembler in your code, but not inline
assembler macros.  To have more information on the processor
specific assembler syntax and its limitations, see the \progref.

%%%%%%%%%%%%%%%%%%%%%%%%%%%%%%%%%%%%%%%%%%%%%%%%%%%%%%%%%%%%%%%%%%%%%%%
% Assembler statements
\section{Assembler statements }
The following is an example of assembler inclusion in your code.
\begin{verbatim}
 ...
 Statements;
 ...
 Asm
   your asm code here
   ...
 end;
 ...
 Statements;
\end{verbatim}
The assembler instructions between the \var{Asm} and \var{end} keywords will
be inserted in the assembler generated by the compiler.
You can still use conditionals in your assembler, the compiler will
recognise it, and treat it as any other conditionals.

\begin{remark}
Before version 0.99.1, \fpc did not support reference to variables by
their names in the assembler parts of your code.
\end{remark}

%%%%%%%%%%%%%%%%%%%%%%%%%%%%%%%%%%%%%%%%%%%%%%%%%%%%%%%%%%%%%%%%%%%%%%%
% Assembler procedures and functions
\section{Assembler procedures and functions}
Assembler procedures and functions are declared using the
\var{Assembler} directive. The \var{Assembler} keyword is supported
as of version 0.9.7. This permits the code generator to make a number
of code generation optimizations.

The code generator does not generate any stack frame (entry and exit
code for the routine) if it contains no local variables and no
parameters. In the case of functions, ordinal values must be returned
in the accumulator. In the case of floating point values, these depend
on the target processor and emulation options.

\begin{remark} From version 0.99.1 to 0.99.5 (\emph{excluding}
FPC 0.99.5a), the \var{Assembler} directive did not have the
same effect as in Turbo Pascal, so beware! The stack frame would be
omitted if there were no local variables, in this case if the assembly
routine had any parameters, they would be referenced directly via the stack
pointer. This was \emph{ NOT} like Turbo Pascal where the stack frame is only
omitted if there are no parameters \emph{ and } no local variables. As
stated earlier, starting from version 0.99.5a, \fpc now has the same
behaviour as Turbo Pascal.
\end{remark}

%
% System unit reference guide.
%

\part{Reference : The System unit}

%%%%%%%%%%%%%%%%%%%%%%%%%%%%%%%%%%%%%%%%%%%%%%%%%%%%%%%%%%%%%%%%%%%%%%%
% The system unit
%%%%%%%%%%%%%%%%%%%%%%%%%%%%%%%%%%%%%%%%%%%%%%%%%%%%%%%%%%%%%%%%%%%%%%%
\chapter{The system unit}
\label{ch:refchapter}
\FPCexampledir{refex}
The system unit contains the standard supported functions of \fpc. It is the
same for all platforms. Basically it is the same as the system unit provided
with Borland or Turbo Pascal.

Functions are listed in alphabetical order. Arguments of functions or
procedures that are optional are put between square brackets.

The pre-defined constants and variables are listed in the first section.
The second section contains an overview of all functions, grouped by
functionality, and the last section contains the supported functions
and procedures.

%%%%%%%%%%%%%%%%%%%%%%%%%%%%%%%%%%%%%%%%%%%%%%%%%%%%%%%%%%%%%%%%%%%%%%%
% Types, Constants and Variables
\section{Types, Constants and Variables}

\subsection{Types}
The following integer types are defined in the System unit:
\begin{verbatim}
Shortint = -128..127;
SmallInt = -32768..32767;
Longint  = $80000000..$7fffffff;
byte     = 0..255;
word     = 0..65535;
dword    = cardinal;
longword = cardinal;
Integer  = smallint;
\end{verbatim}
The following types are used for the functions that need compiler magic
such as \seep{Val} or \seep{Str}:
\begin{verbatim}
StrLenInt = LongInt;
ValSInt = Longint;
ValUInt = Cardinal;
ValReal = Extended;
\end{verbatim}
The following character types are defined for Delphi compatibility:
\begin{verbatim}
TAnsiChar   = Char;
AnsiChar    = TAnsiChar;
\end{verbatim}
And the following pointer types:
\begin{verbatim}
  PChar = ^char;
  pPChar = ^PChar;
  PAnsiChar   = PChar;
  PQWord      = ^QWord;
  PInt64      = ^Int64;
  pshortstring = ^shortstring;
  plongstring  = ^longstring;
  pansistring  = ^ansistring;
  pwidestring  = ^widestring;
  pextended    = ^extended;
  ppointer     = ^pointer;
\end{verbatim}
For the \seef{SetJmp} and \seep{LongJmp} calls, the following jump bufer
type is defined (for the I386 processor):
\begin{verbatim}
  jmp_buf = record
    ebx,esi,edi : Longint;
    bp,sp,pc : Pointer;
    end;
  PJmp_buf = ^jmp_buf;
\end{verbatim}
The following records and pointers can be used if you want to scan the
entries in the string message handler tables:
\begin{verbatim}
  tmsgstrtable = record
     name : pshortstring;
     method : pointer;
  end;
  pmsgstrtable = ^tmsgstrtable;

  tstringmessagetable = record
     count : dword;
     msgstrtable : array[0..0] of tmsgstrtable;
  end;
  pstringmessagetable = ^tstringmessagetable;
\end{verbatim}

The base class for all classes is defined as:
\begin{verbatim}
Type
  TObject = Class
  Public
    constructor create;
    destructor destroy;virtual;
    class function newinstance : tobject;virtual;
    procedure freeinstance;virtual;
    function safecallexception(exceptobject : tobject;
      exceptaddr : pointer) : longint;virtual;
    procedure defaulthandler(var message);virtual;
    procedure free;
    class function initinstance(instance : pointer) : tobject;
    procedure cleanupinstance;
    function classtype : tclass;
    class function classinfo : pointer;
    class function classname : shortstring;
    class function classnameis(const name : string) : boolean;
    class function classparent : tclass;
    class function instancesize : longint;
    class function inheritsfrom(aclass : tclass) : boolean;
    class function inheritsfrom(aclass : tclass) : boolean;
    class function stringmessagetable : pstringmessagetable;
    procedure dispatch(var message);
    procedure dispatchstr(var message);
    class function methodaddress(const name : shortstring) : pointer;
    class function methodname(address : pointer) : shortstring;
    function fieldaddress(const name : shortstring) : pointer;
    procedure AfterConstruction;virtual;
    procedure BeforeDestruction;virtual;
    procedure DefaultHandlerStr(var message);virtual;
  end;
  TClass = Class Of TObject;
  PClass = ^TClass;
\end{verbatim}
Unhandled exceptions can be treated using a constant of the
\var{TExceptProc} type:
\begin{verbatim}
TExceptProc = Procedure (Obj : TObject; Addr,Frame: Pointer);
\end{verbatim}
\var{Obj} is the exception object that was used to raise the exception,
\var{Addr} and \var{Frame} contain the exact address and stack frame
where the exception was raised.

The \var{TVarRec} type is used to access the elements passed in a \var{Array
of Const} argument to a function or procedure:
\begin{verbatim}
Type
  PVarRec = ^TVarRec;
  TVarRec = record
    case VType : Longint of
    vtInteger    : (VInteger: Longint);
    vtBoolean    : (VBoolean: Boolean);
    vtChar       : (VChar: Char);
    vtExtended   : (VExtended: PExtended);
    vtString     : (VString: PShortString);
    vtPointer    : (VPointer: Pointer);
    vtPChar      : (VPChar: PChar);
    vtObject     : (VObject: TObject);
    vtClass      : (VClass: TClass);
    vtAnsiString : (VAnsiString: Pointer);
    vtWideString : (VWideString: Pointer);
    vtInt64      : (VInt64: PInt64);
  end;
\end{verbatim}
The heap manager uses the \var{TMemoryManager} type:
\begin{verbatim}
  PMemoryManager = ^TMemoryManager;
  TMemoryManager = record
    Getmem      : Function(Size:Longint):Pointer;
    Freemem     : Function(var p:pointer):Longint;
    FreememSize : Function(var p:pointer;Size:Longint):Longint;
    AllocMem    : Function(Size:longint):Pointer;
    ReAllocMem  : Function(var p:pointer;Size:longint):Pointer;
    MemSize     : function(p:pointer):Longint;
    MemAvail    : Function:Longint;
    MaxAvail    : Function:Longint;
    HeapSize    : Function:Longint;
  end;
\end{verbatim}
More information on using this record can be found in \progref.

\subsection{Constants}
The following constants define the maximum values that can be used with
various types:
\begin{verbatim}
  MaxSIntValue = High(ValSInt);
  MaxUIntValue = High(ValUInt);
  maxint   = maxsmallint;
  maxLongint  = $7fffffff;
  maxSmallint = 32767;
\end{verbatim}
The following constants for file-handling are defined in the system unit:
\begin{verbatim}
Const
  fmclosed = $D7B0;
  fminput  = $D7B1;
  fmoutput = $D7B2;
  fminout  = $D7B3;
  fmappend = $D7B4;
  filemode : byte = 2;
\end{verbatim}
Further, the following non processor specific general-purpose constants
are also defined:
\begin{verbatim}
const
  erroraddr : pointer = nil;
  errorcode : word = 0;
 { max level in dumping on error }
  max_frame_dump : word = 20;
\end{verbatim}
\begin{remark}
Processor specific global constants are named Testxxxx where xxxx
represents the processor number (such as Test8086, Test68000),
and are used to determine on what generation of processor the program
is running on.
\end{remark}
The following constants are defined to access VMT entries:
\begin{verbatim}
   vmtInstanceSize         = 0;
   vmtParent               = 8;
   vmtClassName            = 12;
   vmtDynamicTable         = 16;
   vmtMethodTable          = 20;
   vmtFieldTable           = 24;
   vmtTypeInfo             = 28;
   vmtInitTable            = 32;
   vmtAutoTable            = 36;
   vmtIntfTable            = 40;
   vmtMsgStrPtr            = 44;
   vmtMethodStart          = 48;
   vmtDestroy              = vmtMethodStart;
   vmtNewInstance          = vmtMethodStart+4;
   vmtFreeInstance         = vmtMethodStart+8;
   vmtSafeCallException    = vmtMethodStart+12;
   vmtDefaultHandler       = vmtMethodStart+16;
   vmtAfterConstruction    = vmtMethodStart+20;
   vmtBeforeDestruction    = vmtMethodStart+24;
   vmtDefaultHandlerStr    = vmtMethodStart+28;
\end{verbatim}
You should always use the constant names, and never their values, because
the VMT table can change, breaking your code.

The following constants will be used for the planned \var{variant} support:
\begin{verbatim}
  varEmpty     = $0000;
  varNull      = $0001;
  varSmallint  = $0002;
  varInteger   = $0003;
  varSingle    = $0004;
  varDouble    = $0005;
  varCurrency  = $0006;
  varDate      = $0007;
  varOleStr    = $0008;
  varDispatch  = $0009;
  varError     = $000A;
  varBoolean   = $000B;
  varVariant   = $000C;
  varUnknown   = $000D;
  varByte      = $0011;
  varString    = $0100;
  varAny       = $0101;
  varTypeMask  = $0FFF;
  varArray     = $2000;
  varByRef     = $4000;
\end{verbatim}
The following constants are used in the \var{TVarRec} record:
\begin{verbatim}
vtInteger    = 0;
vtBoolean    = 1;
vtChar       = 2;
vtExtended   = 3;
vtString     = 4;
vtPointer    = 5;
vtPChar      = 6;
vtObject     = 7;
vtClass      = 8;
vtWideChar   = 9;
vtPWideChar  = 10;
vtAnsiString = 11;
vtCurrency   = 12;
vtVariant    = 13;
vtInterface  = 14;
vtWideString = 15;
vtInt64      = 16;
vtQWord      = 17;
\end{verbatim}
The \var{ExceptProc} is called when an unhandled exception occurs:
\begin{verbatim}
Const
  ExceptProc : TExceptProc = Nil;
\end{verbatim}
It is set in the \file{objpas} unit, but you can set it yourself to change
the default exception handling.

\subsection{Variables}
The following variables are defined and initialized in the system unit:
\begin{verbatim}
var
  output,input,stderr : text;
  exitproc : pointer;
  exitcode : word;
  stackbottom : Longint;
  loweststack : Longint;
\end{verbatim}
The variables \var{ExitProc}, \var{exitcode} are used in the \fpc exit
scheme. It works similarly to the one in Turbo Pascal:

When a program halts (be it through the call of the \var{Halt} function or
\var{Exit} or through a run-time error), the exit mechanism checks the value
of \var{ExitProc}. If this one is non-\var{Nil}, it is set to \var{Nil}, and
the procedure is called. If the exit procedure exits, the value of ExitProc
is checked again. If it is non-\var{Nil} then the above steps are repeated.
So if you want to install your exit procedure, you should save the old value
of \var{ExitProc} (may be non-\var{Nil}, since other units could have set it before
you did). In your exit procedure you then restore the value of
\var{ExitProc}, such that if it was non-\var{Nil} the exit-procedure can be
called.

\FPCexample{ex98}

The \var{ErrorAddr} and \var{ExitCode} can be used to check for
error-conditions. If \var{ErrorAddr} is non-\var{Nil}, a run-time error has
occurred. If so, \var{ExitCode} contains the error code. If \var{ErrorAddr} is
\var{Nil}, then {ExitCode} contains the argument to \var{Halt} or 0 if the
program terminated normally.

\var{ExitCode} is always passed to the operating system as the exit-code of
your process.

\begin{remark}
The maximum error code under \linux is 127.
\end{remark}

Under \file{GO32}, the following constants are also defined :
\begin{verbatim}
const
   seg0040 = $0040;
   segA000 = $A000;
   segB000 = $B000;
   segB800 = $B800;
\end{verbatim}
These constants allow easy access to the bios/screen segment via mem/absolute.

The randomize function uses a seed stored in the \var{RandSeed} variable:
\begin{verbatim}
  RandSeed    : Cardinal;
\end{verbatim}
This variable is initialized in the initialization code of the system unit.

%%%%%%%%%%%%%%%%%%%%%%%%%%%%%%%%%%%%%%%%%%%%%%%%%%%%%%%%%%%%%%%%%%%%%%%
% Functions and Procedures by category
\section{Function list by category}
What follows is a listing of the available functions, grouped by category.
For each function there is a reference to the page where you can find the
function.
\subsection{File handling}
Functions concerning input and output from and to file.
\begin{funclist}
\procref{Append}{Open a file in append mode}
\procref{Assign}{Assign a name to a file}
\procref{Blockread}{Read data from a file into memory}
\procref{Blockwrite}{Write data from memory to a file}
\procref{Close}{Close a file}
\funcref{Eof}{Check for end of file}
\funcref{Eoln}{Check for end of line}
\procref{Erase}{Delete  file from disk}
\funcref{Filepos}{Position in file}
\funcref{Filesize}{Size of file}
\procref{Flush}{Write file buffers to disk}
\funcref{IOresult}{Return result of last file IO operation}
\procref{Read}{Read from file into variable}
\procref{Readln}{Read from file into variable and goto next line}
\procref{Rename}{Rename file on disk}
\procref{Reset}{Open file for reading}
\procref{Rewrite}{Open file for writing}
\procref{Seek}{Set file position}
\funcref{SeekEof}{Set file position to end of file}
\funcref{SeekEoln}{Set file position to end of line}
\procref{SetTextBuf}{Set size of file buffer}
\procref{Truncate}{Truncate the file at position}
\procref{Write}{Write variable to file}
\procref{WriteLn}{Write variable to file and append newline}
\end{funclist}

\subsection{Memory management}
Functions concerning memory issues.
\begin{funclist}
\funcref{Addr}{Return address of variable}
\funcref{Assigned}{Check if a pointer is valid}
\funcref{CompareByte}{Compare 2 memory buffers byte per byte}
\funcref{CompareChar}{Compare 2 memory buffers byte per byte}
\funcref{CompareDWord}{Compare 2 memory buffers byte per byte}
\funcref{CompareWord}{Compare 2 memory buffers byte per byte}
\funcref{CSeg}{Return code segment}
\procref{Dispose}{Free dynamically allocated memory}
\funcref{DSeg}{Return data segment}
\procref{FillByte}{Fill memory region with 8-bit pattern}
\procref{Fillchar}{Fill memory region with certain character}
\procref{FillDWord}{Fill memory region with 32-bit pattern}
\procref{Fillword}{Fill memory region with 16-bit pattern}
\procref{Freemem}{Release allocated memory}
\procref{Getmem}{Allocate new memory}
\procref{GetMemoryManager}{Return current memory manager}
\funcref{High}{Return highest index of open array or enumerated}
\funcref{IsMemoryManagerSet}{Is the memory manager set}
\funcref{Low}{Return lowest index of open array or enumerated}
\procref{Mark}{Mark current memory position}
\funcref{Maxavail}{Return size of largest free memory block}
\funcref{Memavail}{Return total available memory}
\procref{Move}{Move data from one location in memory to another}
\procrefl{MoveChar0}{MoveCharNull}{Move data till first zero character}
\procref{New}{Dynamically allocate memory for variable}
\funcref{Ofs}{Return offset of variable}
\funcref{Ptr}{Combine segmant and offset to pointer}
\procref{Release}{Release memory above mark point}
\funcref{Seg}{Return segment}
\procref{SetMemoryManager}{Set a memory manager}
\funcref{Sptr}{Return current stack pointer}
\funcref{SSeg}{Return ESS register value}
\end{funclist}

\subsection{Mathematical routines}
Functions connected to calculating and coverting numbers.
\begin{funclist}
\funcref{Abs}{Calculate absolute value}
\funcref{Arctan}{Calculate inverse tangent}
\funcref{Cos}{Calculate cosine of angle}
\procref{Dec}{Decrease value of variable}
\funcref{Exp}{Exponentiate}
\funcref{Frac}{Return fractional part of floating point value}
\funcref{Hi}{Return high byte/word of value}
\procref{Inc}{Increase value of variable}
\funcref{Int}{Calculate integer part of floating point value}
\funcref{Ln}{Calculate logarithm}
\funcref{Lo}{Return low byte/word of value}
\funcref{Odd}{Is a value odd or even ? }
\funcref{Pi}{Return the value of pi}
\funcref{Power}{Raise float to integer power}
\funcref{Random}{Generate random number}
\procref{Randomize}{Initialize random number generator}
\funcref{Round}{Round floating point value to nearest integer number}
\funcref{Sin}{Calculate sine of angle}
\funcref{Sqr}{Calculate the square of a value}
\funcref{Sqrt}{Calculate the square root of a value}
\funcref{Swap}{Swap high and low bytes/words of a variable}
\funcref{Trunc}{Truncate a floating point value}
\end{funclist}

\subsection{String handling}
All things connected to string handling.
\begin{funclist}
\funcref{BinStr}{Construct binary representation of integer}
\funcref{Chr}{Convert ASCII code to character}
\funcref{Concat}{Concatenate two strings}
\funcref{Copy}{Copy part of a string}
\procref{Delete}{Delete part of a string}
\funcref{HexStr}{Construct hexadecimal representation of integer}
\procref{Insert}{Insert one string in another}
\funcref{Length}{Return length of string}
\funcref{Lowercase}{Convert string to all-lowercase}
\funcref{Pos}{Calculate position of one string in another}
\procref{SetLength}{Set length of a string}
\procref{Str}{Convert number to string representation}
\funcref{StringOfChar}{Create string consisting of a number of characters}
\funcref{Upcase}{Convert string to all-uppercase}
\procref{Val}{Convert string to number}
\end{funclist}

\subsection{Operating System functions}
Functions that are connected to the operating system.
\begin{funclist}
\procref{Chdir}{Change working directory}
\procref{Getdir}{Return current working directory}
\procref{Halt}{Halt program execution}
\funcref{Paramcount}{Number of parameters with which program was called}
\funcref{Paramstr}{Retrieve parameters with which program was called}
\procref{Mkdir}{Make a directory}
\procref{Rmdir}{Remove a directory}
\procref{Runerror}{Abort program execution with error condition}
\end{funclist}

\subsection{Miscellaneous functions}
Functions that do not belong in one of the other categories.
\begin{funclist}
\procref{Break}{Abort current loop}
\procref{Continue}{Next cycle in current loop}
\procref{Exit}{Exit current function or procedure}
\procref{LongJmp}{Jump to execution point}
\funcref{Ord}{Return ordinal value of enumerated type}
\funcref{Pred}{Return previous value of ordinal type}
\funcref{SetJmp}{Mark execution point for jump}
\funcref{SizeOf}{Return size of variable or type}
\funcref{Succ}{Return next value of ordinal type}
\end{funclist}


%%%%%%%%%%%%%%%%%%%%%%%%%%%%%%%%%%%%%%%%%%%%%%%%%%%%%%%%%%%%%%%%%%%%%%%
% Functions and Procedures
\section{Functions and Procedures}

\begin{function}{Abs}
\Declaration
Function Abs (X : Every numerical type) : Every numerical type;
\Description
\var{Abs} returns the absolute value of a variable. The result of the
function has the same type as its argument, which can be any numerical
type.
\Errors
None.
\SeeAlso
\seef{Round}
\end{function}

\FPCexample{ex1}

\begin{function}{Addr}
\Declaration
Function Addr (X : Any type) : Pointer;

\Description
\var{Addr} returns a pointer to its argument, which can be any type, or a
function or procedure name. The returned pointer isn't typed.
The same result can be obtained by the \var{@} operator, which can return a
typed pointer (\progref).
\Errors
None
\SeeAlso
\seef{SizeOf}
\end{function}

\FPCexample{ex2}

\begin{procedure}{Append}
\Declaration
Procedure Append (Var F : Text);

\Description
\var{Append} opens an existing file in append mode. Any data written to
\var{F} will be appended to the file. If the file didn't exist, it will be
created, contrary to the Turbo Pascal implementation of \var{Append}, where
a file needed to exist in order to be opened by
\var{Append}.
Only text files can be opened in append mode.

\Errors
If the file can't be created, a run-time error will be generated.
\SeeAlso
\seep{Rewrite},\seep{Close}, \seep{Reset}
\end{procedure}

\FPCexample{ex3}

\begin{function}{Arctan}
\Declaration
Function Arctan (X : Real) : Real;

\Description
\var{Arctan} returns the Arctangent of \var{X}, which can be any Real type.
The resulting angle is in radial units.
\Errors
None
\SeeAlso
\seef{Sin}, \seef{Cos}
\end{function}

\FPCexample{ex4}

\begin{procedure}{Assign}
\Declaration
Procedure Assign (Var F; Name : String);

\Description
\var{Assign} assigns a name to \var{F}, which can be any file type.
This call doesn't open the file, it just assigns a name to a file variable,
and marks the file as closed.
\Errors
None.
\SeeAlso
\seep{Reset}, \seep{Rewrite}, \seep{Append}
\end{procedure}

\FPCexample{ex5}

\begin{function}{Assigned}
\Declaration
Function Assigned (P : Pointer) : Boolean;
\Description
\var{Assigned} returns \var{True} if \var{P} is non-nil
and retuns \var{False} of \var{P} is nil.
The main use of Assigned is that Procedural variables, method variables and
class-type variables also can be passed to \var{Assigned}.
\Errors
None
\SeeAlso
\seep{New}
\end{function}

\FPCexample{ex96}

\begin{function}{BinStr}
\Declaration
Function BinStr (Value : longint; cnt : byte) : String;

\Description
\var{BinStr} returns a string with the binary representation
of \var{Value}. The string has at most \var{cnt} characters.
(i.e. only the \var{cnt} rightmost bits are taken into account)
To have a complete representation of any longint-type value, you need 32
bits, i.e. \var{cnt=32}

\Errors
None.
\SeeAlso
\seep{Str},\seep{Val},\seef{HexStr}
\end{function}

\FPCexample{ex82}

\begin{procedure}{Blockread}
\Declaration
Procedure Blockread (Var F : File; Var Buffer; Var Count : Longint [; var
Result : Longint]);

\Description
\var{Blockread} reads \var{count} or less records from file \var{F}. A
record is a block of bytes with size specified by the \seep{Rewrite} or
\seep{Reset} statement.

The result is placed in \var{Buffer}, which must contain enough room for
\var{Count} records. The function cannot read partial records.
If \var{Result} is specified, it contains the number of records actually
read. If \var{Result} isn't specified, and less than \var{Count} records were
read, a run-time error is generated. This behavior can be controlled by the
\var{\{\$i\}} switch.
\Errors
If \var{Result} isn't specified, then a run-time error is generated if less
than \var{count} records were read.
\SeeAlso
\seep{Blockwrite}, \seep{Close}, \seep{Reset}, \seep{Assign}
\end{procedure}

\FPCexample{ex6}

\begin{procedure}{Blockwrite}
\Declaration
Procedure Blockwrite (Var F : File; Var Buffer; Var Count : Longint);

\Description
\var{BlockWrite} writes \var{count} records from \var{buffer} to the file
 \var{F}.A record is a block of bytes with size specified by the \seep{Rewrite} or
\seep{Reset} statement.

If the records couldn't be written to disk, a run-time error is generated.
This behavior can be controlled by the \var{\{\$i\}} switch.

\Errors
A run-time error is generated if, for some reason, the records couldn't be
written to disk.
\SeeAlso
\seep{Blockread},\seep{Close}, \seep{Rewrite}, \seep{Assign}
\end{procedure}

For the example, see \seep{Blockread}.

\begin{procedure}{Break}
\Declaration
Procedure Break;
\Description
\var{Break} jumps to the statement following the end of the current
repetitive statement. The code between the \var{Break} call and
the end of the repetitive statement is skipped.
The condition of the repetitive statement is NOT evaluated.

This can be used with \var{For}, var{repeat} and \var{While} statements.

Note that while this is a procedure, \var{Break} is a reserved word
and hence cannot be redefined.
\Errors
None.
\SeeAlso
\seep{Continue}, \seep{Exit}
\end{procedure}

\FPCexample{ex87}

\begin{procedure}{Chdir}
\Declaration
Procedure Chdir (const S : string);
\Description
\var{Chdir} changes the working directory of the process to \var{S}.
\Errors
If the directory \var{S} doesn't exist, a run-time error is generated.
\SeeAlso
\seep{Mkdir}, \seep{Rmdir}
\end{procedure}

\FPCexample{ex7}

\begin{function}{Chr}
\Declaration
Function Chr (X : byte) : Char;
\Description
\var{Chr} returns the character which has ASCII value \var{X}.
\Errors
None.
\SeeAlso
\seef{Ord}, \seep{Str}
\end{function}

\FPCexample{ex8}

\begin{procedure}{Close}
\Declaration
Procedure Close (Var F : Anyfiletype);

\Description
\var{Close} flushes the buffer of the file \var{F} and closes \var{F}.
After a call to \var{Close}, data can no longer be read from or written to
\var{F}.
To reopen a file closed with \var{Close}, it isn't necessary to assign the
file again. A call to \seep{Reset} or \seep{Rewrite} is sufficient.
\Errors
None.
\SeeAlso
\seep{Assign}, \seep{Reset}, \seep{Rewrite}, \seep{Flush}
\end{procedure}

\FPCexample{ex9}

\begin{function}{CompareByte}
\Declaration
function CompareByte(var buf1,buf2;len:longint):longint;
\Description
\var{CompareByte} compares two memory regions \var{buf1},\var{buf2} on a
byte-per-byte basis for a total of \var{len} bytes.

The function returns one of the following values:
\begin{description}
\item[-1] if \var{buf1} and \var{buf2} contain different bytes
in the first \var{len} bytes, and the first such byte is smaller in \var{buf1}
than the byte at the same position in \var{buf2}.
\item[0]  if the first \var{len} bytes in \var{buf1} and \var{buf2} are
equal.
\item [1] if \var{buf1} and \var{buf2} contain different bytes
in the first \var{len} bytes, and the first such byte is larger in \var{buf1}
than the byte at the same position in \var{buf2}.
\end{description}
\Errors
None.
\SeeAlso
\seef{CompareChar},\seef{CompareWord},\seef{CompareDWord}
\end{function}

\FPCexample{ex99}

\begin{function}{CompareChar}
\Declaration
function  CompareChar(var buf1,buf2;len:longint):longint;
function  CompareChar0(var buf1,buf2;len:longint):longint;
\Description
\var{CompareChar} compares two memory regions \var{buf1},\var{buf2} on a
character-per-character basis for a total of \var{len} characters.

The \var{CompareChar0} variant compares \var{len} bytes, or until
a zero character is found.

The function returns one of the following values:
\begin{description}
\item[-1] if \var{buf1} and \var{buf2} contain different characters
in the first \var{len} positions, and the first such character is smaller in \var{buf1}
than the character at the same position in \var{buf2}.
\item[0]  if the first \var{len} characters in \var{buf1} and \var{buf2} are
equal.
\item [1] if \var{buf1} and \var{buf2} contain different characters
in the first \var{len} positions, and the first such character is larger in
\var{buf1} than the character at the same position in \var{buf2}.
\end{description}

\Errors
None.
\SeeAlso
\seef{CompareByte},\seef{CompareWord},\seef{CompareDWord}
\end{function}

\FPCexample{ex100}

\begin{function}{CompareDWord}
\Declaration
function  CompareDWord(var buf1,buf2;len:longint):longint;
\Description
\var{CompareDWord} compares two memory regions \var{buf1},\var{buf2} on a
DWord-per-DWord basis for a total of \var{len} DWords. (A DWord is 4 bytes).

The function returns one of the following values:
\begin{description}
\item[-1] if \var{buf1} and \var{buf2} contain different DWords
in the first \var{len} DWords, and the first such DWord is smaller in \var{buf1}
than the DWord at the same position in \var{buf2}.
\item[0]  if the first \var{len} DWords in \var{buf1} and \var{buf2} are
equal.
\item [1] if \var{buf1} and \var{buf2} contain different DWords
in the first \var{len} DWords, and the first such DWord is larger in \var{buf1}
than the DWord at the same position in \var{buf2}.
\end{description}
\Errors
None.
\SeeAlso
\seef{CompareChar},\seef{CompareByte},\seef{CompareWord},
\end{function}

\FPCexample{ex101}

\begin{function}{CompareWord}
\Declaration
function  CompareWord(var buf1,buf2;len:longint):longint;
\Description
\var{CompareWord} compares two memory regions \var{buf1},\var{buf2} on a
Word-per-Word basis for a total of \var{len} Words. (A Word is 2 bytes).

The function returns one of the following values:
\begin{description}
\item[-1] if \var{buf1} and \var{buf2} contain different Words
in the first \var{len} Words, and the first such Word is smaller in \var{buf1}
than the Word at the same position in \var{buf2}.
\item[0]  if the first \var{len} Words in \var{buf1} and \var{buf2} are
equal.
\item [1] if \var{buf1} and \var{buf2} contain different Words
in the first \var{len} Words, and the first such Word is larger in \var{buf1}
than the Word at the same position in \var{buf2}.
\end{description}
\Errors
None.
\SeeAlso
\seef{CompareChar},\seef{CompareByte},\seef{CompareWord},
\end{function}

\FPCexample{ex102}

\begin{function}{Concat}
\Declaration
Function Concat (S1,S2 [,S3, ... ,Sn]) : String;

\Description
\var{Concat} concatenates the strings \var{S1},\var{S2} etc. to one long
string. The resulting string is truncated at a length of 255 bytes.
The same operation can be performed with the \var{+} operation.
\Errors
None.
\SeeAlso
\seef{Copy}, \seep{Delete}, \seep{Insert}, \seef{Pos}, \seef{Length}
\end{function}

\FPCexample{ex10}

\begin{procedure}{Continue}
\Declaration
Procedure Continue;
\Description
\var{Continue} jumps to the end of the current repetitive statement.
The code between the \var{Continue} call and the end of the repetitive
statement is skipped. The condition of the repetitive statement is then
checked again.

This can be used with \var{For}, var{repeat} and \var{While} statements.

Note that while this is a procedure, \var{Continue} is a reserved word
and hence cannot be redefined.
\Errors
None.
\SeeAlso
\seep{Break}, \seep{Exit}
\end{procedure}

\FPCexample{ex86}


\begin{function}{Copy}
\Declaration
Function Copy (Const S : String;Index : Integer;Count : Byte) : String;

\Description
\var{Copy} returns a string which is a copy if the \var{Count} characters
in \var{S}, starting at position \var{Index}. If \var{Count} is larger than
the length of the string \var{S}, the result is truncated.
If \var{Index} is larger than the length of the string \var{S}, then an
empty string is returned.
\Errors
None.
\SeeAlso
\seep{Delete}, \seep{Insert}, \seef{Pos}
\end{function}

\FPCexample{ex11}

\begin{function}{Cos}
\Declaration
Function Cos (X : Real) : Real;

\Description
\var{Cos} returns the cosine of \var{X}, where X is an angle, in radians.

If the absolute value of the argument is larger than \var{2\^{}63}, then the
result is undefined.
\Errors
None.
\SeeAlso
\seef{Arctan}, \seef{Sin}
\end{function}

\FPCexample{ex12}

\begin{function}{CSeg}
\Declaration
Function CSeg  : Word;

\Description
\var{CSeg} returns the Code segment register. In \fpc, it returns always a
zero, since \fpc is a 32 bit compiler.
\Errors
None.
\SeeAlso
\seef{DSeg}, \seef{Seg}, \seef{Ofs}, \seef{Ptr}
\end{function}

\FPCexample{ex13}

\begin{procedure}{Dec}
\Declaration
Procedure Dec (Var X : Any ordinal type[; Decrement : Longint]);

\Description
\var{Dec} decreases the value of \var{X} with \var{Decrement}.
If \var{Decrement} isn't specified, then 1 is taken as a default.
\Errors
A range check can occur, or an underflow error, if you try to decrease \var{X}
below its minimum value.
\SeeAlso
\seep{Inc}
\end{procedure}

\FPCexample{ex14}

\begin{procedure}{Delete}
\Declaration
Procedure Delete (var S : string;Index : Integer;Count : Integer);

\Description
\var{Delete} removes \var{Count} characters from string \var{S}, starting
at position \var{Index}. All characters after the delected characters are
shifted \var{Count} positions to the left, and the length of the string is adjusted.

\Errors
None.
\SeeAlso
\seef{Copy},\seef{Pos},\seep{Insert}
\end{procedure}

\FPCexample{ex15}

\begin{procedure}{Dispose}
\Declaration
Procedure Dispose (P : pointer);\\
Procedure Dispiose (P : Typed Pointer; Des : Procedure);
\Description
The first form \var{Dispose} releases the memory allocated with a call to
\seep{New}. The pointer \var{P} must be typed. The released memory is
returned to the heap.

The second form of \var{Dispose} accepts as a first parameter a pointer
to an object type, and as a second parameter the name of a destructor
of this object. The destructor will be called, and the memory allocated
for the object will be freed.
\Errors
An error will occur if the pointer doesn't point to a location in the
heap.
\SeeAlso
\seep{New}, \seep{Getmem}, \seep{Freemem}
\end{procedure}

\FPCexample{ex16}

\begin{function}{DSeg}
\Declaration
Function DSeg  : Word;

\Description
\var{DSeg} returns the data segment register. In \fpc, it returns always a
zero, since \fpc is a 32 bit compiler.
\Errors
None.
\SeeAlso
\seef{CSeg}, \seef{Seg}, \seef{Ofs}, \seef{Ptr}
\end{function}

\FPCexample{ex17}

\begin{function}{Eof}
\Declaration
Function Eof [(F : Any file type)] : Boolean;

\Description
\var{Eof} returns \var{True} if the file-pointer has reached the end of the
file, or if the file is empty. In all other cases \var{Eof} returns
\var{False}.
If no file \var{F} is specified, standard input is assumed.
\Errors
None.
\SeeAlso
\seef{Eoln}, \seep{Assign}, \seep{Reset}, \seep{Rewrite}
\end{function}

\FPCexample{ex18}

\begin{function}{Eoln}
\Declaration
Function Eoln [(F : Text)] : Boolean;

\Description
\var{Eof} returns \var{True} if the file pointer has reached the end of a
line, which is demarcated by a line-feed character (ASCII value 10), or if
the end of the file is reached.
In all other cases \var{Eof} returns \var{False}.
If no file \var{F} is specified, standard input is assumed.
It can only be used on files of type \var{Text}.
\Errors
None.
\SeeAlso
\seef{Eof}, \seep{Assign}, \seep{Reset}, \seep{Rewrite}
\end{function}

\FPCexample{ex19}

\begin{procedure}{Erase}
\Declaration
Procedure Erase (Var F : Any file type);

\Description
\var{Erase} removes an unopened file from disk. The file should be
assigned with \var{Assign}, but not opened with \var{Reset} or \var{Rewrite}
\Errors
A run-time error will be generated if the specified file doesn't exist, or
is opened by the program.
\SeeAlso
\seep{Assign}
\end{procedure}

\FPCexample{ex20}

\begin{procedure}{Exit}
\Declaration
Procedure Exit ([Var X : return type )];

\Description
\var{Exit} exits the current subroutine, and returns control to the calling
routine. If invoked in the main program routine, exit stops the program.
The optional argument \var{X} allows to specify a return value, in the case
\var{Exit} is invoked in a function. The function result will then be
equal to \var{X}.
\Errors
None.
\SeeAlso
\seep{Halt}
\end{procedure}

\FPCexample{ex21}

\begin{function}{Exp}
\Declaration
Function Exp (Var X : Real) : Real;

\Description
\var{Exp} returns the exponent of \var{X}, i.e. the number \var{e} to the
power \var{X}.
\Errors
None.
\SeeAlso
\seef{Ln}, \seef{Power}
\end{function}

\FPCexample{ex22}

\begin{function}{Filepos}
\Declaration
Function Filepos (Var F : Any file type) : Longint;

\Description
\var{Filepos} returns the current record position of the file-pointer in file
\var{F}. It cannot be invoked with a file of type \var{Text}. If you try to
do this, a compiler error will be generated.
\Errors
None.
\SeeAlso
\seef{Filesize}
\end{function}

\FPCexample{ex23}

\begin{function}{Filesize}
\Declaration
Function Filesize (Var F : Any file type) : Longint;
\Description
\var{Filesize} returns the total number of records in file \var{F}.
It cannot be invoked with a file of type \var{Text}. (under \linux, this
also means that it cannot be invoked on pipes.)
If \var{F} is empty, 0 is returned.
\Errors
None.
\SeeAlso
\seef{Filepos}
\end{function}

\FPCexample{ex24}

\begin{procedure}{FillByte}
\Declaration
Procedure FillByte(var X;Count:longint;Value:byte);
\Description
\var{FillByte} fills the memory starting at \var{X} with \var{Count} bytes
with value equal to \var{Value}.

This is useful for quickly zeroing out a memory location. If you know
that the size of the memory location to be filled out is a multiple of
2 bytes, it is better to use \seep{Fillword}, and if it is a multiple
of 4 bytes it's better to use \seep{FillDWord}, these routines are
optimized for their respective sizes.

\Errors
No checking on the size of \var{X} is done.
\SeeAlso
\seep{Fillchar}, \seep{FillDWord}, \seep{Fillword}, \seep{Move}
\end{procedure}

\FPCexample{ex102}

\begin{procedure}{Fillchar}
\Declaration
Procedure Fillchar (Var X;Count : Longint;Value : char or byte);;

\Description
\var{Fillchar} fills the memory starting at \var{X} with \var{Count} bytes
or characters with value equal to \var{Value}.

\Errors
No checking on the size of \var{X} is done.
\SeeAlso
\seep{Fillword}, \seep{Move}, \seep{FillByte}, \seep{FillDWord}
\end{procedure}

\FPCexample{ex25}

\begin{procedure}{FillDWord}
\Declaration
Procedure FillDWord (Var X;Count : Longint;Value : DWord);;
\Description
\var{Fillword} fills the memory starting at \var{X} with \var{Count} DWords
with value equal to \var{Value}. A DWord is 4 bytes in size.

\Errors
No checking on the size of \var{X} is done.
\SeeAlso
\seep{FillByte}, \seep{Fillchar}, \seep{Fillword}, \seep{Move}
\end{procedure}

\FPCexample{ex103}

\begin{procedure}{Fillword}
\Declaration
Procedure Fillword (Var X;Count : Longint;Value : Word);;
\Description
\var{Fillword} fills the memory starting at \var{X} with \var{Count} words
with value equal to \var{Value}. A word is 2 bytes in size.
\Errors
No checking on the size of \var{X} is done.
\SeeAlso
\seep{Fillchar}, \seep{Move}
\end{procedure}

\FPCexample{ex76}

\begin{procedure}{Flush}
\Declaration
Procedure Flush (Var F : Text);

\Description
\var{Flush} empties the internal buffer of an opened file \var{F} and writes the
contents to disk. The file is \textit{not} closed as a result of this call.
\Errors
If the disk is full, a run-time error will be generated.
\SeeAlso
\seep{Close}
\end{procedure}

\FPCexample{ex26}

\begin{function}{Frac}
\Declaration
Function Frac (X : Real) : Real;

\Description
\var{Frac} returns the non-integer part of \var{X}.
\Errors
None.
\SeeAlso
\seef{Round}, \seef{Int}
\end{function}

\FPCexample{ex27}

\begin{procedure}{Freemem}
\Declaration
Procedure Freemem (Var P : pointer; Count : Longint);

\Description
\var{Freemem} releases the memory occupied by the pointer \var{P}, of size
\var{Count} (in bytes), and returns it to the heap. \var{P} should point to the memory
allocated to a dynamical variable.
\Errors
An error will occur when \var{P} doesn't point to the heap.
\SeeAlso
\seep{Getmem}, \seep{New}, \seep{Dispose}
\end{procedure}

\FPCexample{ex28}

\begin{procedure}{Getdir}
\Declaration
Procedure Getdir (drivenr : byte;var dir : string);

\Description
\var{Getdir} returns in \var{dir} the current directory on the drive
\var{drivenr}, where {drivenr} is 1 for the first floppy drive, 3 for the
first hard disk etc. A value of 0 returns the directory on the current disk.
On \linux, \var{drivenr} is ignored, as there is only one directory tree.
\Errors
An error is returned under \dos, if the drive requested isn't ready.
\SeeAlso
\seep{Chdir}
\end{procedure}

\FPCexample{ex29}

\begin{procedure}{Getmem}
\Declaration
Procedure Getmem (var p : pointer;size : Longint);

\Description
\var{Getmem} reserves \var{Size} bytes memory on the heap, and returns a
pointer to this memory in \var{p}. If no more memory is available, nil is
returned.
\Errors
None.
\SeeAlso
\seep{Freemem}, \seep{Dispose}, \seep{New}
\end{procedure}
For an example, see \seep{Freemem}.

\begin{procedure}{GetMemoryManager}
\Declaration
procedure GetMemoryManager(var MemMgr: TMemoryManager);
\Description
\var{GetMemoryManager} stores the current Memory Manager record in
\var{MemMgr}.
\Errors
None.
\SeeAlso
\seep{SetMemoryManager}, \seef{IsMemoryManagerSet}.
\end{procedure}

For an example, see \progref.

\begin{procedure}{Halt}
\Declaration
Procedure Halt [(Errnum : byte)];
\Description
\var{Halt} stops program execution and returns control to the calling
program. The optional argument \var{Errnum} specifies an exit value. If
omitted, zero is returned.
\Errors
None.
\SeeAlso
\seep{Exit}
\end{procedure}

\FPCexample{ex30}

\begin{function}{HexStr}
\Declaration
Function HexStr (Value : longint; cnt : byte) : String;
\Description
\var{HexStr} returns a string with the hexadecimal representation
of \var{Value}. The string has at most \var{cnt} charaters.
 (i.e. only the \var{cnt} rightmost nibbles are taken into account)
To have a complete representation of a Longint-type value, you need 8
nibbles, i.e. \var{cnt=8}.

\Errors
None.
\SeeAlso
\seep{Str}, \seep{Val}, \seef{BinStr}
\end{function}

\FPCexample{ex81}

\begin{function}{Hi}
\Declaration
Function Hi (X : Ordinal type) : Word or byte;

\Description
\var{Hi} returns the high byte or word from \var{X}, depending on the size
of X. If the size of X is 4, then the high word is returned. If the size is
2 then the high byte is returned.
\var{Hi} cannot be invoked on types of size 1, such as byte or char.
\Errors
None
\SeeAlso
\seef{Lo}
\end{function}

\FPCexample{ex31}

\begin{function}{High}
\Declaration
Function High (Type identifier or variable reference) : Ordinal;

\Description
 The return value of \var{High} depends on it's argument:
\begin{enumerate}
\item If the argument is an ordinal type, \var{High} returns the lowest
 value in the range of the given ordinal type.
\item If the argument is an array type or an array type variable then
\var{High} returns the highest possible value of it's index.
\item If the argument is an open array identifier in a function or
procedure, then \var{High} returns the highest index of the array, as if the
array has a zero-based index.
\end{enumerate}
The return type is always the same type as the type of the argument
(This can lead to some nasty surprises !).
\Errors
None.
\SeeAlso
\seef{Low}, \seef{Ord}, \seef{Pred}, \seef{Succ}
\end{function}

\FPCexample{ex80}

\begin{procedure}{Inc}
\Declaration
Procedure Inc (Var X : Any ordinal type[; Increment : Longint]);

\Description
\var{Inc} increases the value of \var{X} with \var{Increment}.
If \var{Increment} isn't specified, then 1 is taken as a default.
\Errors
If range checking is on, then A range check can occur, or an overflow
error, if you try to increase \var{X} over its maximum value.
\SeeAlso
\seep{Dec}
\end{procedure}

\FPCexample{ex32}

\begin{function}{IndexByte}
\Declaration
function  IndexByte(var buf;len:longint;b:byte):longint;
\Description
\var{IndexByte} searches the memory at \var{buf} for maximally \var{len}
positions for the byte \var{b} and returns it's position if it found one.
If \var{b} is not found then -1 is returned.

The position is zero-based.
\Errors
\var{Buf} and \var{Len} are not checked to see if they are valid values.
\SeeAlso
\seef{IndexChar}, \seef{IndexDWord}, \seef{IndexWord}, \seef{CompareByte}
\end{function}

\FPCexample{ex105}

\begin{function}{IndexChar}
\Declaration
function  IndexChar(var buf;len:longint;b:char):longint;
function  IndexChar0(var buf;len:longint;b:char):longint;
\Description
\var{IndexChar} searches the memory at \var{buf} for maximally \var{len}
positions for the character \var{b} and returns it's position if it found one.
If \var{b} is not found then -1 is returned.

The position is zero-based. The \var{IndexChar0} variant stops looking if
a null character is found, and returns -1 in that case.
\Errors
\var{Buf} and \var{Len} are not checked to see if they are valid values.
\SeeAlso
\seef{IndexByte}, \seef{IndexDWord}, \seef{IndexWord}, \seef{CompareChar}
\end{function}

\FPCexample{ex108}

\begin{function}{IndexDWord}
\Declaration
function  IndexDWord(var buf;len:longint;DW:DWord):longint;
\var{IndexChar} searches the memory at \var{buf} for maximally \var{len}
positions for the DWord \var{DW} and returns it's position if it found one.
If \var{DW} is not found then -1 is returned.

The position is zero-based.
\Errors
\var{Buf} and \var{Len} are not checked to see if they are valid values.
\SeeAlso
\seef{IndexByte}, \seef{IndexChar}, \seef{IndexWord}, \seef{CompareDWord}
\end{function}

\FPCexample{ex106}

\begin{function}{IndexWord}
\Declaration
function  IndexWord(var buf;len:longint;W:word):longint;
\var{IndexChar} searches the memory at \var{buf} for maximally \var{len}
positions for the Word \var{W} and returns it's position if it found one.
If \var{W} is not found then -1 is returned.
\Errors
\var{Buf} and \var{Len} are not checked to see if they are valid values.
\SeeAlso
\seef{IndexByte}, \seef{IndexDWord}, \seef{IndexChar}, \seef{CompareWord}
\end{function}

\FPCexample{ex107}

\begin{procedure}{Insert}
\Declaration
Procedure Insert (Const Source : String;var S : String;Index : Longint);

\Description
\var{Insert} inserts string \var{Source} in string \var{S}, at position
\var{Index}, shifting all characters after \var{Index} to the right. The
resulting string is truncated at 255 characters, if needed. (i.e. for
shortstrings)
\Errors
None.
\SeeAlso
\seep{Delete}, \seef{Copy}, \seef{Pos}
\end{procedure}

\FPCexample{ex33}

\begin{function}{IsMemoryManagerSet}
\Declaration
function  IsMemoryManagerSet: Boolean;
\Description
\var{IsMemoryManagerSet} will return \var{True} if the memory manager has
been set to another value than the system heap manager, it will return
\var{False} otherwise.
\Errors
None.
\SeeAlso
\seep{SetMemoryManager}, \seep{GetMemoryManager}
\end{function}

\begin{function}{Int}
\Declaration
Function Int (X : Real) : Real;

\Description
\var{Int} returns the integer part of any Real \var{X}, as a Real.
\Errors
None.
\SeeAlso
\seef{Frac}, \seef{Round}
\end{function}

\FPCexample{ex34}

\begin{function}{IOresult}
\Declaration
Function IOresult  : Word;

\Description
IOresult contains the result of any input/output call, when the
\var{\{\$i-\}} compiler directive is active, disabling IO checking.
When the flag is read, it is reset to zero.
If \var{IOresult} is zero, the operation completed successfully. If
non-zero, an error occurred. The following errors can occur:

\dos errors :
\begin{description}
\item [2\ ] File not found.
\item [3\ ] Path not found.
\item [4\ ] Too many open files.
\item [5\ ] Access denied.
\item [6\ ] Invalid file handle.
\item [12\ ] Invalid file-access mode.
\item [15\ ] Invalid disk number.
\item [16\ ] Cannot remove current directory.
\item [17\ ] Cannot rename across volumes.
\end{description}
I/O errors :
\begin{description}
\item [100\ ] Error when reading from disk.
\item [101\ ] Error when writing to disk.
\item [102\ ] File not assigned.
\item [103\ ] File not open.
\item [104\ ] File not opened for input.
\item [105\ ] File not opened for output.
\item [106\ ] Invalid number.
\end{description}
Fatal errors :
\begin{description}
\item [150\ ] Disk is write protected.
\item [151\ ] Unknown device.
\item [152\ ] Drive not ready.
\item [153\ ] Unknown command.
\item [154\ ] CRC check failed.
\item [155\ ] Invalid drive specified..
\item [156\ ] Seek error on disk.
\item [157\ ] Invalid media type.
\item [158\ ] Sector not found.
\item [159\ ] Printer out of paper.
\item [160\ ] Error when writing to device.
\item [161\ ] Error when reading from device.
\item [162\ ] Hardware failure.
\end{description}

\Errors
None.
\SeeAlso
All I/O functions.
\end{function}

\FPCexample{ex35}

\begin{function}{Length}
\Declaration
Function Length (S : String) : Byte;

\Description
\var{Length} returns the length of the string \var{S}, which is limited
to 255 for shortstrings. If the strings \var{S} is empty, 0 is returned.
{\em Note:} The length of the string \var{S} is stored in \var{S[0]} for
shortstrings only. Ansistrings have their length stored elsewhere,
the \var{Length} fuction should always be used on ansistrings.

\Errors
None.
\SeeAlso
\seef{Pos}
\end{function}

\FPCexample{ex36}

\begin{function}{Ln}
\Declaration
Function Ln (X : Real) : Real;

\Description

\var{Ln} returns the natural logarithm of the Real parameter \var{X}.
\var{X} must be positive.

\Errors
An run-time error will occur when \var{X} is negative.
\SeeAlso
\seef{Exp}, \seef{Power}
\end{function}

\FPCexample{ex37}

\begin{function}{Lo}
\Declaration
Function Lo (O : Word or Longint) : Byte or Word;

\Description
\var{Lo} returns the low byte of its argument if this is of type
\var{Integer} or
\var{Word}. It returns the low word of its argument if this is of type
\var{Longint} or \var{Cardinal}.
\Errors
None.
\SeeAlso
\seef{Ord}, \seef{Chr}, \seef{Hi}
\end{function}

\FPCexample{ex38}

\begin{procedure}{LongJmp}
\Declaration
Procedure LongJmp (Var env : Jmp\_Buf; Value : Longint);

\Description

\var{LongJmp} jumps to the adress in the \var{env} \var{jmp\_buf},
and resores the registers that were stored in it at the corresponding
\seef{SetJmp} call.
In effect, program flow will continue at the \var{SetJmp} call, which will
return \var{value} instead of 0. If you pas a \var{value} equal to zero, it will be
converted to 1 before passing it on. The call will not return, so it must be
used with extreme care.
This can be used for error recovery, for instance when a segmentation fault
occurred.
\Errors
None.
\SeeAlso
\seef{SetJmp}
\end{procedure}
For an example, see \seef{SetJmp}

\begin{function}{Low}
\Declaration
Function Low (Type identifier or variable reference) : Longint;

\Description
 The return value of \var{Low} depends on it's argument:
\begin{enumerate}
\item If the argument is an ordinal type, \var{Low} returns the lowest
value in the range of the given ordinal type.
\item If the argument is an array type or an array type variable then
\var{Low} returns the lowest possible value of it's index.
\end{enumerate}
The return type is always the same type as the type of the argument
\Errors
None.
\SeeAlso
\seef{High}, \seef{Ord}, \seef{Pred}, \seef{Succ}
\end{function}
for an example, see \seef{High}.
\begin{function}{Lowercase}
\Declaration
Function Lowercase (C : Char or String) : Char or String;

\Description
\var{Lowercase} returns the lowercase version of its argument \var{C}.
If its argument is a string, then the complete string is converted to
lowercase. The type of the returned value is the same as the type of the
argument.
\Errors
None.
\SeeAlso
\seef{Upcase}
\end{function}

\FPCexample{ex73}

\begin{procedure}{Mark}
\Declaration
Procedure Mark (Var P : Pointer);

\Description
\var{Mark} copies the current heap-pointer to \var{P}.
\Errors
None.
\SeeAlso
\seep{Getmem}, \seep{Freemem}, \seep{New}, \seep{Dispose}, \seef{Maxavail}
\end{procedure}

\FPCexample{ex39}

\begin{function}{Maxavail}
\Declaration
Function Maxavail  : Longint;

\Description
\var{Maxavail} returns the size, in bytes, of the biggest free memory block in
the heap.
\begin{remark}
The heap grows dynamically if more memory is needed than is available.
\end{remark}
\Errors
None.
\SeeAlso
\seep{Release}, \seef{Memavail},\seep{Freemem}, \seep{Getmem}
\end{function}

\FPCexample{ex40}

\begin{function}{Memavail}
\Declaration
Function Memavail  : Longint;

\Description
\var{Memavail} returns the size, in bytes, of the free heap memory.
\begin{remark}
The heap grows dynamically if more memory is needed than is available.
\end{remark}
\Errors
None.
\SeeAlso
\seef{Maxavail},\seep{Freemem}, \seep{Getmem}
\end{function}

\FPCexample{ex41}

\begin{procedure}{Mkdir}
\Declaration
Procedure Mkdir (const S : string);

\Description
\var{Mkdir} creates a new  directory \var{S}.
\Errors
If a parent-directory of directory \var{S} doesn't exist, a run-time error is generated.
\SeeAlso
\seep{Chdir}, \seep{Rmdir}
\end{procedure}
For an example, see \seep{Rmdir}.
\begin{procedure}{Move}
\Declaration
Procedure Move (var Source,Dest;Count : Longint);
\Description
\var{Move} moves \var{Count} bytes from \var{Source} to \var{Dest}.
\Errors
If either \var{Dest} or \var{Source} is outside the accessible memory for
the process, then a run-time error will be generated. With older versions of
the compiler, a segmentation-fault will occur.
\SeeAlso
\seep{Fillword}, \seep{Fillchar}
\end{procedure}

\FPCexample{ex42}

\begin{procedurel}{MoveChar0}{MoveCharNull}
\Declaration
procedure MoveChar0(var Src,Dest;Count:longint);
\Description
\var{MoveChar0} moves \var{Count} bytes from \var{Src} to \var{Dest}, and
stops moving if a zero character is found.
\Errors
No checking is done to see if \var{Count} stays within the memory allocated
to the process.
\SeeAlso
\seep{Move}
\end{procedurel}

\FPCexample{ex109}

\begin{procedure}{New}
\Declaration
Procedure New (Var P : Pointer[, Constructor]);

\Description
\var{New} allocates a new instance of the type pointed to by \var{P}, and
puts the address in \var{P}.
If P is an object, then it is possible to
specify the name of the constructor with which the instance will be created.
\Errors
If not enough memory is available, \var{Nil} will be returned.
\SeeAlso
\seep{Dispose}, \seep{Freemem}, \seep{Getmem}, \seef{Memavail},
\seef{Maxavail}
\end{procedure}
For an example, see \seep{Dispose}.
\begin{function}{Odd}
\Declaration
Function Odd (X : Longint) : Boolean;

\Description
\var{Odd} returns \var{True} if \var{X} is odd, or \var{False} otherwise.
\Errors
None.
\SeeAlso
\seef{Abs}, \seef{Ord}
\end{function}

\FPCexample{ex43}

\begin{function}{Ofs}
\Declaration
Function Ofs Var X : Longint;

\Description
\var{Ofs} returns the offset of the address of a variable.
This function is only supported for compatibility. In \fpc, it
returns always the complete address of the variable, since \fpc is a 32 bit
compiler.

\Errors
None.
\SeeAlso
\seef{DSeg}, \seef{CSeg}, \seef{Seg}, \seef{Ptr}
\end{function}

\FPCexample{ex44}

\begin{function}{Ord}
\Declaration
Function Ord (X : Any ordinal type) : Longint;

\Description
\var{Ord} returns the Ordinal value of a ordinal-type variable \var{X}.
\Errors
None.
\SeeAlso
\seef{Chr}, \seef{Succ}, \seef{Pred}, \seef{High}, \seef{Low}
\end{function}

\FPCexample{ex45}

\begin{function}{Paramcount}
\Declaration
Function Paramcount  : Longint;

\Description
\var{Paramcount} returns the number of command-line arguments. If no
arguments were given to the running program, \var{0} is returned.

\Errors
None.
\SeeAlso
\seef{Paramstr}
\end{function}

\FPCexample{ex46}

\begin{function}{Paramstr}
\Declaration
Function Paramstr (L : Longint) : String;

\Description
\var{Paramstr} returns the \var{L}-th command-line argument. \var{L} must
be between \var{0} and \var{Paramcount}, these values included.
The zeroth argument is the name with which the program was started.

In all cases, the command-line will be truncated to a length of 255,
even though the operating system may support bigger command-lines. If you
want to access the complete command-line, you must use the \var{argv} pointer
to access the Real values of the command-line parameters.
\Errors
None.
\SeeAlso
\seef{Paramcount}
\end{function}
For an example, see \seef{Paramcount}.
\begin{function}{Pi}
\Declaration
Function Pi  : Real;

\Description
\var{Pi} returns the value of Pi (3.1415926535897932385).
\Errors
None.
\SeeAlso
\seef{Cos}, \seef{Sin}
\end{function}

\FPCexample{ex47}

\begin{function}{Pos}
\Declaration
Function Pos (Const Substr : String;Const S : String) : Byte;
\Description
\var{Pos} returns the index of \var{Substr} in \var{S}, if \var{S} contains
\var{Substr}. In case \var{Substr} isn't found, \var{0} is returned.
The search is case-sensitive.
\Errors
None
\SeeAlso
\seef{Length}, \seef{Copy}, \seep{Delete}, \seep{Insert}
\end{function}

\FPCexample{ex48}

\begin{function}{Power}
\Declaration
Function Power (base,expon : Real) : Real;
\Description

\var{Power} returns the value of \var{base} to the power \var{expon}.
\var{Base} and \var{expon} can be of type Longint, in which case the
result will also be a Longint.

The function actually returns \var{Exp(expon*Ln(base))}
\Errors
None.
\SeeAlso
\seef{Exp}, \seef{Ln}
\end{function}

\FPCexample{ex78}

\begin{function}{Pred}
\Declaration
Function Pred (X : Any ordinal type) : Same type;
\Description
 \var{Pred} returns the element that precedes the element that was passed
to it. If it is applied to the first value of the ordinal type, and the
program was compiled with range checking on (\var{\{\$R+\}}, then a run-time
error will be generated.
\Errors
Run-time error 201 is generated when the result is out of
range.
\SeeAlso
\seef{Ord}, \seef{Pred}, \seef{High}, \seef{Low}
\end{function}

for an example, see \seef{Ord}

\begin{function}{Ptr}
\Declaration
Function Ptr (Sel,Off : Longint) : Pointer;
\Description
\var{Ptr} returns a pointer, pointing to the address specified by
segment \var{Sel} and offset \var{Off}.

 \begin{remark}
\begin{enumerate}
\item In the 32-bit flat-memory model supported by \fpc, this
function is obsolete.
\item The returned address is simply the offset. If you recompile
the RTL with \var{-dDoMapping} defined, then the compiler returns the
following : \var{ptr := pointer(\$e0000000+sel shl 4+off)} under \dos, or
\var{ptr := pointer(sel shl 4+off)} on other OSes.
\end{enumerate}
\end{remark}
\Errors
None.
\SeeAlso
\seef{Addr}
\end{function}

\FPCexample{ex59}

\begin{function}{Random}
\Declaration
Function Random [(L : Longint)] : Longint or Real;

\Description
\var{Random} returns a random number larger or equal to \var{0} and
strictly less than \var{L}.
If the argument \var{L} is omitted, a Real number between 0 and 1 is returned.
(0 included, 1 excluded)
\Errors
None.
\SeeAlso
\seep{Randomize}
\end{function}

\FPCexample{ex49}

\begin{procedure}{Randomize}
\Declaration
Procedure Randomize ;

\Description
\var{Randomize} initializes the random number generator of \fpc, by giving
a value to \var{Randseed}, calculated with the system clock.

\Errors
None.
\SeeAlso
\seef{Random}
\end{procedure}
For an example, see \seef{Random}.
\begin{procedure}{Read}
\Declaration
Procedure Read ([Var F : Any file type], V1 [, V2, ... , Vn]);

\Description
\var{Read} reads one or more values from a file \var{F}, and stores the
result in \var{V1}, \var{V2}, etc.; If no file \var{F} is specified, then
standard input is read.
If \var{F} is of type \var{Text}, then the variables \var{V1, V2} etc. must be
of type \var{Char}, \var{Integer}, \var{Real}, \var{String} or \var{PChar}.
If \var{F} is a typed file, then each of the variables must be of the type
specified in the declaration of \var{F}. Untyped files are not allowed as an
argument.
\Errors
If no data is available, a run-time error is generated. This behavior can
be controlled with the \var{\{\$i\}} compiler switch.
\SeeAlso
\seep{Readln}, \seep{Blockread}, \seep{Write}, \seep{Blockwrite}
\end{procedure}

\FPCexample{ex50}

\begin{procedure}{Readln}
\Declaration
Procedure Readln [Var F : Text], V1 [, V2, ... , Vn]);

\Description
\var{Read} reads one or more values from a file \var{F}, and stores the
result in \var{V1}, \var{V2}, etc. After that it goes to the next line in
the file (defined by the \var{LineFeed (\#10)} character).
If no file \var{F} is specified, then standard input is read.
The variables \var{V1, V2} etc. must be of type \var{Char}, \var{Integer},
\var{Real}, \var{String} or \var{PChar}.

\Errors
If no data is available, a run-time error is generated. This behavior can
be controlled with the \var{\{\$i\}} compiler switch.
\SeeAlso
\seep{Read}, \seep{Blockread}, \seep{Write}, \seep{Blockwrite}
\end{procedure}
For an example, see \seep{Read}.
\begin{procedure}{Release}
\Declaration
Procedure Release (Var P : pointer);

\Description
\var{Release} sets the top of the Heap to the location pointed to by
\var{P}. All memory at a location higher than \var{P} is marked empty.
\Errors
A run-time error will be generated if \var{P} points to memory outside the
heap.
\SeeAlso
\seep{Mark}, \seef{Memavail}, \seef{Maxavail}, \seep{Getmem}, \seep{Freemem}
\seep{New}, \seep{Dispose}
\end{procedure}
For an example, see \seep{Mark}.
\begin{procedure}{Rename}
\Declaration
Procedure Rename (Var F : Any Filetype; Const S : String);
\Description
\var{Rename} changes the name of the assigned file \var{F} to \var{S}.
\var{F}
must be assigned, but not opened.
\Errors
A run-time error will be generated if \var{F} isn't assigned,
or doesn't exist.
\SeeAlso
\seep{Erase}
\end{procedure}

\FPCexample{ex77}

\begin{procedure}{Reset}
\Declaration
Procedure Reset (Var F : Any File Type[; L : Longint]);
\Description
\var{Reset} opens a file \var{F} for reading. \var{F} can be any file type.
If \var{F} is an untyped or typed file, then it is opened for reading and
writing. If \var{F} is an untyped file, the record size can be specified in
the optional parameter \var{L}. Default a value of 128 is used.
\Errors
If the file cannot be opened for reading, then a run-time error is
generated. This behavior can be changed by the \var{\{\$i\} } compiler switch.
\SeeAlso
\seep{Rewrite}, \seep{Assign}, \seep{Close}, \seep{Append}
\end{procedure}

\FPCexample{ex51}

\begin{procedure}{Rewrite}
\Declaration
Procedure Rewrite (Var F : Any File Type[; L : Longint]);
\Description
\var{Rewrite} opens a file \var{F} for writing. \var{F} can be any file type.
If \var{F} is an untyped or typed file, then it is opened for reading and
writing. If \var{F} is an untyped file, the record size can be specified in
the optional parameter \var{L}. Default a value of 128 is used.
if \var{Rewrite} finds a file with the same name as \var{F}, this file is
truncated to length \var{0}. If it doesn't find such a file, a new file is
created.

Contrary to \tp, \fpc opens the file with mode \var{fmoutput}. If you want
to get it in \var{fminout} mode, you'll need to do an extra call to
\seep{Reset}.

\Errors
If the file cannot be opened for writing, then a run-time error is
generated. This behavior can be changed by the \var{\{\$i\} } compiler switch.
\SeeAlso
\seep{Reset}, \seep{Assign}, \seep{Close}, \seep{Flush}, \seep{Append}
\end{procedure}

\FPCexample{ex52}

\begin{procedure}{Rmdir}
\Declaration
Procedure Rmdir (const S : string);

\Description
\var{Rmdir} removes the directory \var{S}.
\Errors
If \var{S} doesn't exist, or isn't empty, a run-time error is generated.

\SeeAlso
\seep{Chdir}, \seep{Mkdir}
\end{procedure}

\FPCexample{ex53}

\begin{function}{Round}
\Declaration
Function Round (X : Real) : Longint;

\Description
\var{Round} rounds \var{X} to the closest integer, which may be bigger or
smaller than \var{X}.
\Errors
None.
\SeeAlso
\seef{Frac}, \seef{Int}, \seef{Trunc}
\end{function}

\FPCexample{ex54}

\begin{procedure}{Runerror}
\Declaration
Procedure Runerror (ErrorCode : Word);

\Description
\var{Runerror} stops the execution of the program, and generates a
run-time error \var{ErrorCode}.
\Errors
None.
\SeeAlso
\seep{Exit}, \seep{Halt}
\end{procedure}

\FPCexample{ex55}

\begin{procedure}{Seek}
\Declaration
Procedure Seek (Var F; Count : Longint);

\Description
\var{Seek} sets the file-pointer for file \var{F} to record Nr. \var{Count}.
The first record in a file has \var{Count=0}. F can be any file type, except
\var{Text}. If \var{F} is an untyped file, with no record size specified in
\seep{Reset} or \seep{Rewrite}, 128 is assumed.
\Errors
A run-time error is generated if \var{Count} points to a position outside
the file, or the file isn't opened.
\SeeAlso
\seef{Eof}, \seef{SeekEof}, \seef{SeekEoln}
\end{procedure}

\FPCexample{ex56}

\begin{function}{SeekEof}
\Declaration
Function SeekEof [(Var F : text)] : Boolean;

\Description
\var{SeekEof} returns \var{True} is the file-pointer is at the end of the
file. It ignores all whitespace.
Calling this function has the effect that the file-position is advanced
until the first non-whitespace character or the end-of-file marker is
reached.
If the end-of-file marker is reached, \var{True} is returned. Otherwise,
False is returned.
If the parameter \var{F} is omitted, standard \var{Input} is assumed.

\Errors
A run-time error is generated if the file \var{F} isn't opened.
\SeeAlso
\seef{Eof}, \seef{SeekEoln}, \seep{Seek}
\end{function}

\FPCexample{ex57}

\begin{function}{SeekEoln}
\Declaration
Function SeekEoln [(Var F : text)] : Boolean;

\Description
\var{SeekEoln} returns \var{True} is the file-pointer is at the end of the
current line. It ignores all whitespace.
Calling this function has the effect that the file-position is advanced
until the first non-whitespace character or the end-of-line marker is
reached.
If the end-of-line marker is reached, \var{True} is returned. Otherwise,
False is returned.
The end-of-line marker is defined as \var{\#10}, the LineFeed character.
If the parameter \var{F} is omitted, standard \var{Input} is assumed.
\Errors
A run-time error is generated if the file \var{F} isn't opened.
\SeeAlso
\seef{Eof}, \seef{SeekEof}, \seep{Seek}
\end{function}

\FPCexample{ex58}

\begin{function}{Seg}
\Declaration
Function Seg Var X : Longint;

\Description
\var{Seg} returns the segment of the address of a variable.
This function is only supported for compatibility. In \fpc, it
returns always 0, since \fpc is a 32 bit compiler, segments have no meaning.

\Errors
None.
\SeeAlso
\seef{DSeg}, \seef{CSeg}, \seef{Ofs}, \seef{Ptr}
\end{function}

\FPCexample{ex60}

\begin{procedure}{SetMemoryManager}
\Declaration
procedure SetMemoryManager(const MemMgr: TMemoryManager);
\Description
\var{SetMemoryManager} sets the current memory manager record to
\var{MemMgr}.
\Errors
None.
\SeeAlso
\seep{GetMemoryManager}, \seef{IsMemoryManagerSet}
\end{procedure}

For an example, see \progref.

\begin{function}{SetJmp}
\Declaration
Function SetJmp (Var Env : Jmp\_Buf) : Longint;

\Description

\var{SetJmp} fills \var{env} with the necessary data for a jump back to the
point where it was called. It returns zero if called in this way.
If the function returns nonzero, then it means that a call to \seep{LongJmp}
with \var{env} as an argument was made somewhere in the program.

\Errors
None.
\SeeAlso
\seep{LongJmp}
\end{function}

\FPCexample{ex79}


\begin{procedure}{SetLength}
\Declaration
Procedure SetLength(var S : String; Len : Longint);
\Description
\var{SetLength} sets the length of the string \var{S} to \var{Len}. \var{S}
can be an ansistring or a short string.
For \var{ShortStrings}, \var{Len} can maximally be 255. For \var{AnsiStrings}
it can have any value. For \var{AnsiString} strings, \var{SetLength} {\em
must} be used to set the length of the string.
\Errors
None.
\SeeAlso
\seef{Length}
\end{procedure}


\FPCexample{ex85}


\begin{procedure}{SetTextBuf}
\Declaration
Procedure SetTextBuf (Var f : Text; Var Buf[; Size : Word]);

\Description
\var{SetTextBuf} assigns an I/O buffer to a text file. The new buffer is
located at \var{Buf} and is \var{Size} bytes long. If \var{Size} is omitted,
then \var{SizeOf(Buf)} is assumed.
The standard buffer of any text file is 128 bytes long. For heavy I/0
operations this may prove too slow. The \var{SetTextBuf} procedure allows
you to set a bigger buffer for your application, thus reducing the number of
system calls, and thus reducing the load on the system resources.
The maximum size of the newly assigned buffer is 65355 bytes.
\begin{remark}
\begin{itemize}
\item Never assign a new buffer to an opened file. You can assign a
new buffer immediately after a call to \seep{Rewrite}, \seep{Reset} or
\var{Append}, but not after you read from/wrote to the file. This may cause
loss of data. If you still want to assign a new buffer after read/write
operations have been performed, flush the file first. This will ensure that
the current buffer is emptied.
\item Take care that the buffer you assign is always valid. If you
assign a local variable as a buffer, then after your program exits the local
program block, the buffer will no longer be valid, and stack problems may
occur.
\end{itemize}
\end{remark}
\Errors
No checking on \var{Size} is done.
\SeeAlso
\seep{Assign}, \seep{Reset}, \seep{Rewrite}, \seep{Append}
\end{procedure}

\FPCexample{ex61}

\begin{function}{Sin}
\Declaration
Function Sin (X : Real) : Real;

\Description
\var{Sin} returns the sine of its argument \var{X}, where \var{X} is an
angle in radians.

If the absolute value of the argument is larger than \var{2\^{}63}, then the
result is undefined.
\Errors
None.
\SeeAlso
\seef{Cos}, \seef{Pi}, \seef{Exp}, \seef{Ln}
\end{function}

\FPCexample{ex62}

\begin{function}{SizeOf}
\Declaration
Function SizeOf (X : Any Type) : Longint;

\Description
\var{SizeOf} returns the size, in bytes, of any variable or type-identifier.
\begin{remark}
This isn't really a RTL function. It's result is calculated at
compile-time, and hard-coded in your executable.
\end{remark}
\Errors
None.
\SeeAlso
\seef{Addr}
\end{function}

\FPCexample{ex63}

\begin{function}{Sptr}
\Declaration
Function Sptr  : Pointer;

\Description
\var{Sptr} returns the current stack pointer.

\Errors
None.
\SeeAlso
\seef{SSeg}
\end{function}

\FPCexample{ex64}

\begin{function}{Sqr}
\Declaration
Function Sqr (X : Real) : Real;

\Description
\var{Sqr} returns the square of its argument \var{X}.
\Errors
None.
\SeeAlso
\seef{Sqrt}, \seef{Ln}, \seef{Exp}
\end{function}

\FPCexample{ex65}

\begin{function}{Sqrt}
\Declaration
Function Sqrt (X : Real) : Real;

\Description
\var{Sqrt} returns the square root of its argument \var{X}, which must be
positive.
\Errors
If \var{X} is negative, then a run-time error is generated.
\SeeAlso
\seef{Sqr}, \seef{Ln}, \seef{Exp}
\end{function}

\FPCexample{ex66}

\begin{function}{SSeg}
\Declaration
Function SSeg  : Longint;

\Description
 \var{SSeg} returns the Stack Segment. This function is only
 supported for compatibility reasons, as \var{Sptr} returns the
correct contents of the stackpointer.
\Errors
None.
\SeeAlso
\seef{Sptr}
\end{function}

\FPCexample{ex67}

\begin{procedure}{Str}
\Declaration
Procedure Str (Var X[:NumPlaces[:Decimals]]; Var S : String);

\Description
\var{Str} returns a string which represents the value of X. X can be any
numerical type.
The optional \var{NumPLaces} and \var{Decimals} specifiers control the
formatting of the string.
\Errors
None.
\SeeAlso
\seep{Val}
\end{procedure}

\FPCexample{ex68}

\begin{function}{StringOfChar}
\Declaration
Function StringOfChar(c : char;l : longint) : AnsiString;
\Description
\var{StringOfChar} creates a new Ansistring of length \var{l} and fills
it with the character \var{c}.

It is equivalent to  the following calls:
\begin{verbatim}
SetLength(StringOfChar,l);
FillChar(Pointer(StringOfChar)^,Length(StringOfChar),c);
\end{verbatim}
\Errors
None.
\SeeAlso
\seep{SetLength}
\end{function}

\FPCexample{ex97}

\begin{function}{Succ}
\Declaration
Function Succ (X : Any ordinal type) : Same type;

\Description
 \var{Succ} returns the element that succeeds the element that was passed
to it. If it is applied to the last value of the ordinal type, and the
program was compiled with range checking on (\var{\{\$R+\}}), then a run-time
error will be generated.

\Errors
Run-time error 201 is generated when the result is out of
range.
\SeeAlso
\seef{Ord}, \seef{Pred}, \seef{High}, \seef{Low}
\end{function}
for an example, see \seef{Ord}.
\begin{function}{Swap}
\Declaration
Function Swap (X) : Type of X;

\Description
\var{Swap} swaps the high and low order bytes of \var{X} if \var{X} is of
type \var{Word} or \var{Integer}, or swaps the high and low order words of
\var{X} if \var{X} is of type \var{Longint} or \var{Cardinal}.
The return type is the type of \var{X}
\Errors
None.
\SeeAlso
\seef{Lo}, \seef{Hi}
\end{function}

\FPCexample{ex69}

\begin{function}{Trunc}
\Declaration
Function Trunc (X : Real) : Longint;

\Description
\var{Trunc} returns the integer part of \var{X},
which is always smaller than (or equal to) \var{X} in absolute value.
\Errors
None.
\SeeAlso
\seef{Frac}, \seef{Int}, \seef{Round}
\end{function}

\FPCexample{ex70}

\begin{procedure}{Truncate}
\Declaration
Procedure Truncate (Var F : file);

\Description
\var{Truncate} truncates the (opened) file \var{F} at the current file
position.

\Errors
Errors are reported by IOresult.
\SeeAlso
\seep{Append}, \seef{Filepos},
\seep{Seek}
\end{procedure}

\FPCexample{ex71}

\begin{function}{Upcase}
\Declaration
Function Upcase (C : Char or string) : Char or String;

\Description
\var{Upcase} returns the uppercase version of its argument \var{C}.
If its argument is a string, then the complete string is converted to
uppercase. The type of the returned value is the same as the type of the
argument.
\Errors
None.
\SeeAlso
\seef{Lowercase}
\end{function}

\FPCexample{ex72}

\begin{procedure}{Val}
\Declaration
Procedure Val (const S : string;var V;var Code : word);

\Description
\var{Val} converts the value represented in the string \var{S} to a numerical
value, and stores this value in the variable \var{V}, which
can be of type \var{Longint}, \var{Real} and \var{Byte}.
If the conversion isn't succesfull, then the parameter \var{Code} contains
the index of the character in \var{S} which prevented the conversion.
The string \var{S} isn't allowed to contain spaces.
\Errors
If the conversion doesn't succeed, the value of \var{Code} indicates the
position where the conversion went wrong.
\SeeAlso
\seep{Str}
\end{procedure}

\FPCexample{ex74}

\begin{procedure}{Write}
\Declaration
Procedure Write ([Var F : Any filetype;] V1 [; V2; ... , Vn)];

\Description
\var{Write} writes the contents of the variables \var{V1}, \var{V2} etc. to
the file \var{F}. \var{F} can be a typed file, or a \var{Text} file.
If \var{F} is a typed file, then the variables \var{V1}, \var{V2} etc. must
be of the same type as the type in the declaration of \var{F}. Untyped files
are not allowed.
If the parameter \var{F} is omitted, standard output is assumed.
If \var{F} is of type \var{Text}, then the necessary conversions are done
such that the output of the variables is in human-readable format.
This conversion is done for all numerical types. Strings are printed exactly
as they are in memory, as well as \var{PChar} types.
The format of the numerical conversions can be influenced through
the following modifiers:
\var{ OutputVariable : NumChars [: Decimals ]  }
This will print the value of \var{OutputVariable} with a minimum of
\var{NumChars} characters, from which \var{Decimals} are reserved for the
decimals. If the number cannot be represented with \var{NumChars} characters,
\var{NumChars} will be increased, until the representation fits. If the
representation requires less than \var{NumChars} characters then the output
is filled up with spaces, to the left of the generated string, thus
resulting in a right-aligned representation.
If no formatting is specified, then the number is written using its natural
length, with nothing in front of it if it's positive, and a minus sign if
it's negative.
Real numbers are, by default, written in scientific notation.

\Errors
If an error occurs, a run-time error is generated. This behavior can be
controlled with the \var{\{\$i\}} switch.
\SeeAlso
\seep{WriteLn}, \seep{Read}, \seep{Readln}, \seep{Blockwrite}
\end{procedure}
\begin{procedure}{WriteLn}
\Declaration
Procedure WriteLn [([Var F : Text;] [V1 [; V2; ... , Vn)]];

\Description
\var{WriteLn} does the same as \seep{Write} for text files, and emits a
Carriage Return - LineFeed character pair after that.
If the parameter \var{F} is omitted, standard output is assumed.
If no variables are specified, a Carriage Return - LineFeed character pair
is emitted, resulting in a new line in the file \var{F}.
\begin{remark}
Under \linux, the Carriage Return character is omitted, as
customary in Unix environments.
\end{remark}

\Errors
If an error occurs, a run-time error is generated. This behavior can be
controlled with the \var{\{\$i\}} switch.
\SeeAlso
\seep{Write}, \seep{Read}, \seep{Readln}, \seep{Blockwrite}
\end{procedure}
\FPCexample{ex75}

%%%%%%%%%%%%%%%%%%%%%%%%%%%%%%%%%%%%%%%%%%%%%%%%%%%%%%%%%%%%%%%%%%%%%%%
% The objpas unit
%%%%%%%%%%%%%%%%%%%%%%%%%%%%%%%%%%%%%%%%%%%%%%%%%%%%%%%%%%%%%%%%%%%%%%%
\chapter{The OBJPAS unit}
The \file{objpas} unit is meant for compatibility with Object Pascal as
implemented by Delphi. The unit is loaded automatically by the \fpc compiler
whenever the \var{Delphi} or \var{objfpc} more is entered, either through
the command line switches \var{-Sd} or \var{-Sh} or with the \var{\{\$MODE
DELPHI\}} or \var{\{\$MODE OBJFPC\}} directives.

It redefines some basic pascal types, introduces some functions for
compatibility with Delphi's system unit, and introduces some methods for the
management of the resource string tables.

%%%%%%%%%%%%%%%%%%%%%%%%%%%%%%%%%%%%%%%%%%%%%%%%%%%%%%%%%%%%%%%%%%%%%%%
% Tytpes
\section{Types}
The \file{objpas} unit redefines two integer types, for compatibity with
Delphi:
\begin{verbatim}
type
  smallint = system.integer;
  integer  = system.longint;
\end{verbatim}
The resource string tables can be managed with a callback function which the
user must provide: \var{TResourceIterator}.
\begin{verbatim}
Type
   TResourceIterator =
      Function (Name,Value : AnsiString;Hash : Longint):AnsiString;
\end{verbatim}

%%%%%%%%%%%%%%%%%%%%%%%%%%%%%%%%%%%%%%%%%%%%%%%%%%%%%%%%%%%%%%%%%%%%%%%
% Functions and procedures
\section{Functions and Procedures}

\begin{procedure}{AssignFile}
\Declaration
Procedure AssignFile(Var f: FileType;Name: Character type);
\Description
\var{AssignFile} is completely equivalent to the system unit's \seep{Assign}
function: It assigns \var{Name} to a function of any type (\var{FileType}
can be \var{Text} or a typed or untyped \var{File} variable). \var{Name} can
be a string, a single character or a \var{PChar}.

It is most likely introduced to avoid confusion between the regular
\seep{Assign} function and the \var{Assign} method of \var{TPersistent}
in the Delphi VCL.
\Errors
None.
\SeeAlso
\seep{CloseFile}, \seep{Assign}, \seep{Reset}, \seep{Rewrite}, \seep{Append}
\end{procedure}

\FPCexample{ex88}

\begin{procedure}{CloseFile}
\Declaration
Procedure CloseFile(Var F: FileType);
\Description
\var{CloseFile} flushes and closes a file \var{F} of any file type.
\var{F} can be  \var{Text} or a typed or untyped \var{File} variable.
After a call to \var{CloseFile}, any attempt to write to the file \var{F}
will result in an error.

It is most likely introduced to avoid confusion between the regular
\seep{Close} function and the \var{Close} method of \var{TForm}
in the Delphi VCL.

\Errors
None.
\SeeAlso
\seep{Close}, \seep{AssignFile}, \seep{Reset}, \seep{Rewrite}, \seep{Append}
\end{procedure}

for an example, see \seep{AssignFile}.

\begin{procedurel}{Freemem}{objpasfreemem}
\Declaration
Procedure FreeMem(Var p:pointer[;Size:Longint]);
\Description
\var{FreeMem} releases the memory reserved by a call to
\seepl{GetMem}{objpasgetmem}. The (optional) \var{Size} parameter is
ignored, since the object pascal version of \var{GetMem} stores the amount
of memory that was requested.

be sure not to release memory that was not obtained with the \var{Getmem}
call in \file{Objpas}. Normally, this should not happen, since objpas
changes the default memory manager to it's own memory manager.
\Errors
None.
\SeeAlso
\seep{Freemem}, \seepl{GetMem}{objpasgetmem}, \seep{Getmem}
\end{procedurel}

\FPCexample{ex89}

\begin{procedurel}{Getmem}{objpasgetmem}
\Declaration
Procedure Getmem(Var P:pointer;Size:Longint);
\Description
\var{GetMem} reserves \var{Size} bytes of memory on the heap and returns
a pointer to it in \var{P}. \var{Size} is stored at offset -4 of the
result, and is used to release the memory again. \var{P} can be a typed or
untyped pointer.

Be sure to release this memory with the \seepl{FreeMem}{objpasfreemem} call
defined in the \file{objpas} unit.
\Errors
In case no more memory is available, and no more memory could be obtained
from the system a run-time error is triggered.
\SeeAlso
\seepl{FreeMem}{objpasfreemem}, \seep{Getmem}.
\end{procedurel}

For an example, see \seepl{FreeMem}{objpasfreemem}.

\begin{function}{GetResourceStringCurrentValue}
\Declaration
Function GetResourceStringCurrentValue(TableIndex,StringIndex : Longint) : AnsiString;
\Description
\var{GetResourceStringCurrentValue} returns the current value of the
resourcestring in table \var{TableIndex} with index \var{StringIndex}.

The current value depends on the system of internationalization that was
used, and which language is selected when the program is executed.
\Errors
If either \var{TableIndex} or \var{StringIndex} are out of range, then
a empty string is returned.
\SeeAlso
\seep{SetResourceStrings},
\seef{GetResourceStringDefaultValue},
\seef{GetResourceStringHash},
\seef{GetResourceStringName},
\seef{ResourceStringTableCount},
\seef{ResourceStringCount}
\end{function}

\FPCexample{ex90}

\begin{function}{GetResourceStringDefaultValue}
\Declaration
Function GetResourceStringDefaultValue(TableIndex,StringIndex : Longint) : AnsiString
\Description
\var{GetResourceStringDefaultValue} returns the default value of the
resourcestring in table \var{TableIndex} with index \var{StringIndex}.

The default value is the value of the string that appears in the source code
of the programmer, and is compiled into the program.
\Errors
If either \var{TableIndex} or \var{StringIndex} are out of range, then
a empty string is returned.
\Errors
\SeeAlso
\seep{SetResourceStrings},
\seef{GetResourceStringCurrentValue},
\seef{GetResourceStringHash},
\seef{GetResourceStringName},
\seef{ResourceStringTableCount},
\seef{ResourceStringCount}
\end{function}

\FPCexample{ex91}

\begin{function}{GetResourceStringHash}
\Declaration
Function GetResourceStringHash(TableIndex,StringIndex : Longint) : Longint;
\Description
\var{GetResourceStringHash} returns the hash value associated with the
resource string in table \var{TableIndex}, with index \var{StringIndex}.

The hash value is calculated from the default value of the resource string
in a manner that gives the same result as the GNU \file{gettext} mechanism.
It is stored in the resourcestring tables, so retrieval is faster than
actually calculating the hash for each string.
\Errors
If either \var{TableIndex} or \var{StringIndex} is zero, 0 is returned.
\SeeAlso
\seef{Hash}
\seep{SetResourceStrings},
\seef{GetResourceStringDefaultValue},
\seef{GetResourceStringHash},
\seef{GetResourceStringName},
\seef{ResourceStringTableCount},
\seef{ResourceStringCount}
\end{function}

For an example, see \seef{Hash}.

\begin{function}{GetResourceStringName}
\Declaration
Function GetResourceStringName(TableIndex,StringIndex : Longint) : Ansistring;
\Description
\var{GetResourceStringName} returns the name of the resourcestring in table
\var{TableIndex} with index \var{StringIndex}. The name of the string is
always the unit name in which the string was declared, followed by a period
and the name of the constant, all in lowercase.

If a unit \file{MyUnit} declares a resourcestring \var{MyTitle} then the
name returned will be \var{myunit.mytitle}. A resourcestring in the program file
will have the name of the program prepended.

The name returned by this function is also the name that is stored in the
resourcestring file generated by the compiler.

Strictly speaking, this information isn't necessary for the functioning
of the program, it is provided only as a means to easier translation of
strings.
\Errors
If either \var{TableIndex} or \var{StringIndex} is zero, an empty string
is returned.
\SeeAlso
\seep{SetResourceStrings},
\seef{GetResourceStringDefaultValue},
\seef{GetResourceStringHash},
\seef{GetResourceStringName},
\seef{ResourceStringTableCount},
\seef{ResourceStringCount}
\end{function}

\FPCexample{ex92}


\begin{function}{Hash}
\Declaration
Function Hash(S : AnsiString) : longint;
\Description
\var{Hash} calculates the hash value of the string \var{S} in a manner that
is compatible with the GNU gettext hash value for the string. It is the same
value that is stored in the Resource string tables, and which can be
retrieved with the \seef{GetResourceStringHash} function call.
\Errors
 None. In case the calculated hash value should be 0, the returned result
will be -1.
\SeeAlso
\seef{GetResourceStringHash},
\end{function}

\FPCexample{ex93}

\begin{functionl}{Paramstr}{objpasparamstr}
\Declaration
Function ParamStr(Param : Integer) : Ansistring;
\Description
\var{ParamStr} returns the \var{Param}-th command-line parameter as an
AnsiString. The system unit \seef{Paramstr} function limits the result to
255 characters.

The zeroeth command-line parameter contains the path of the executable,
except on \linux, where it is the command as typed on the command-line.
\Errors
In case \var{Param} is an invalid value, an empty string is returned.
\SeeAlso
\seef{Paramstr}
\end{functionl}

For an example, see \seef{Paramstr}.

\begin{procedure}{ResetResourceTables}
\Declaration
Procedure ResetResourceTables;
\Description
\var{ResetResourceTables} resets all resource strings to their default
(i.e. as in the source code) values.

Normally, this should never be called from a user's program. It is called
in the initialization code of the \file{objpas} unit. However, if the
resourcetables get messed up for some reason, this procedure will fix them
again.
\Errors
None.
\SeeAlso
\seep{SetResourceStrings},
\seef{GetResourceStringDefaultValue},
\seef{GetResourceStringHash},
\seef{GetResourceStringName},
\seef{ResourceStringTableCount},
\seef{ResourceStringCount}
\end{procedure}

\begin{function}{ResourceStringCount}
\Declaration
Function ResourceStringCount(TableIndex : longint) : longint;
\Description
\var{ResourceStringCount} returns the number of resourcestrings in
the table with index \var{TableIndex}. The strings in a particular table
are numbered from \var{0} to \var{ResourceStringCount-1}, i.e. they're zero
based.
\Errors
If an invalid \var{TableIndex} is given, \var{-1} is returned.
\SeeAlso
\seep{SetResourceStrings},
\seef{GetResourceStringCurrentValue},
\seef{GetResourceStringDefaultValue},
\seef{GetResourceStringHash},
\seef{GetResourceStringName},
\seef{ResourceStringTableCount},
\end{function}

For an example, see \seef{GetResourceStringDefaultValue}

\begin{function}{ResourceStringTableCount}
\Declaration
Function ResourceStringTableCount : Longint;
\Description
\var{ResourceStringTableCount} returns the number of resource string tables;
this may be zero if no resource strings are used in a program.

The tables are numbered from 0 to \var{ResourceStringTableCount-1}, i.e.
they're zero based.
\Errors
\SeeAlso
\seep{SetResourceStrings},
\seef{GetResourceStringDefaultValue},
\seef{GetResourceStringHash},
\seef{GetResourceStringName},
\seef{ResourceStringCount}
\end{function}

For an example, see \seef{GetResourceStringDefaultValue}

\begin{procedure}{SetResourceStrings}
\Declaration
TResourceIterator =  Function (Name,Value : AnsiString;Hash : Longint):AnsiString;

Procedure SetResourceStrings (SetFunction :  TResourceIterator);
\Description
\var{SetResourceStrings} calls \var{SetFunction} for all resourcestrings
in the resourcestring tables and sets the resourcestring's current value
to the value returned by \var{SetFunction}.

The \var{Name},\var{Value} and \var{Hash} parameters passed to the iterator
function are the values stored in the tables.
\Errors
None.
\SeeAlso
\seef{GetResourceStringCurrentValue},
\seef{GetResourceStringDefaultValue},
\seef{GetResourceStringHash},
\seef{GetResourceStringName},
\seef{ResourceStringTableCount},
\seef{ResourceStringCount}
\end{procedure}

\FPCexample{ex95}

\begin{function}{SetResourceStringValue}
\Declaration
Function SetResourceStringValue(TableIndex,StringIndex : longint; Value : Ansistring) : Boolean;
\Description
\var{SetResourceStringValue} assigns \var{Value} to the resource string in
table \var{TableIndex} with index \var{StringIndex}.
\Errors
\SeeAlso
\seep{SetResourceStrings},
\seef{GetResourceStringCurrentValue},
\seef{GetResourceStringDefaultValue},
\seef{GetResourceStringHash},
\seef{GetResourceStringName},
\seef{ResourceStringTableCount},
\seef{ResourceStringCount}
\end{function}

\FPCexample{ex94}


%
% The index.
%
\printindex
\end{document}
