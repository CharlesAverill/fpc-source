%
%   $Id$
%   This file is part of the FPC documentation.
%   Copyright (C) 1997, by Michael Van Canneyt
%
%   The FPC documentation is free text; you can redistribute it and/or
%   modify it under the terms of the GNU Library General Public License as
%   published by the Free Software Foundation; either version 2 of the
%   License, or (at your option) any later version.
%
%   The FPC Documentation is distributed in the hope that it will be useful,
%   but WITHOUT ANY WARRANTY; without even the implied warranty of
%   MERCHANTABILITY or FITNESS FOR A PARTICULAR PURPOSE.  See the GNU
%   Library General Public License for more details.
%
%   You should have received a copy of the GNU Library General Public
%   License along with the FPC documentation; see the file COPYING.LIB.  If not,
%   write to the Free Software Foundation, Inc., 59 Temple Place - Suite 330,
%   Boston, MA 02111-1307, USA. 
%
\documentclass{report}
%
% Preamble
%
\usepackage{a4}
\usepackage{makeidx}
\usepackage{html}
\latex{\usepackage{fpc}}
\html{%
%   $Id$
%   This file is part of the FPC documentation.
%   Copyright (C) 1997, by Michael Van Canneyt
%
%   The FPC documentation is free text; you can redistribute it and/or
%   modify it under the terms of the GNU Library General Public License as
%   published by the Free Software Foundation; either version 2 of the
%   License, or (at your option) any later version.
%
%   The FPC Documentation is distributed in the hope that it will be useful,
%   but WITHOUT ANY WARRANTY; without even the implied warranty of
%   MERCHANTABILITY or FITNESS FOR A PARTICULAR PURPOSE.  See the GNU
%   Library General Public License for more details.
%
%   You should have received a copy of the GNU Library General Public
%   License along with the FPC documentation; see the file COPYING.LIB.  If not,
%   write to the Free Software Foundation, Inc., 59 Temple Place - Suite 330,
%   Boston, MA 02111-1307, USA. 
%
% Dummy
\newenvironment{FPCList}{\begin{description}}{\end{description}}
%
%
\newcommand{\Declaration}{\item[Declaration]\ttfamily}
\newcommand{\Description}{\item[Description]\rmfamily}
\newcommand{\Errors}{\item[Errors]\rmfamily}
\newcommand{\SeeAlso}{\item[See also]\rmfamily}
%
%  The environments
%
\newenvironment{functionl}[2]{\subsection{#1}%
\index{#1}\label{fu:#2}\begin{FPCList}}{\end{FPCList}}
\newenvironment{procedurel}[2]{\subsection{#1}%
\index{#1}\label{pro:#2}\begin{FPCList}}{\end{FPCList}}
\newenvironment{function}[1]{\begin{functionl}{#1}{#1}}{\end{functionl}}
\newenvironment{procedure}[1]{\begin{procedurel}{#1}{#1}}{\end{procedurel}}
\newcommand{\seefl}[2]{
\htmlref{#1}{fu:#2}
}
\newcommand{\seepl}[2]{
\htmlref{#1}{pro:#2}
}
%
% Now the ones without label.
%
\newcommand{\seef}[1]{\seefl{#1}{#1}}
\newcommand{\seep}[1]{\seepl{#1}{#1}}
%
\newcommand{\seet}[1]{
\htmlref{#1}{sec:types}
}
\newcommand{\seem}[2] {\texttt{#1} (#2) }
\newcommand{\var}[1]{\texttt {#1}}
\newcommand{\file}[1]{\textsf {#1}}
%
% Abbreviations
%
\newcommand{\linux}{\textsc{LinuX} }
\newcommand{\dos}  {\textsc{dos} }
\newcommand{\msdos}{\textsc{ms-dos} }
\newcommand{\ostwo}{\textsc{os/2} }
\newcommand{\windowsnt}{\textsc{WindowsNT} }
\newcommand{\windows}{\textsc{Windows} }
\newcommand{\docdescription}[1]{}
\newcommand{\docversion}[1]{}
\newcommand{\unitdescription}[1]{}
\newcommand{\unitversion}[1]{}
\newcommand{\fpc}{Free Pascal }
\newcommand{\gnu}{gnu }
%
% Useful references.
%
\newcommand{\progref}{\htmladdnormallink{Programmer's guide}{../prog/prog.html}\ }
\newcommand{\refref}{\htmladdnormallink{Reference guide}{../ref/ref.html}\ }
\newcommand{\userref}{\htmladdnormallink{Users' guide}{../user/user.html}\ }
\newcommand{\unitsref}{\htmladdnormallink{Unit reference}{../units/units.html}\ }
\newcommand{\seecrt}{\htmladdnormallink{CRT}{../crt/crt.html}}
\newcommand{\seelinux}{\htmladdnormallink{Linux}{../linux/linux.html}}
\newcommand{\seestrings}{\htmladdnormallink{strings}{../strings/strings.html}}
\newcommand{\seedos}{\htmladdnormallink{DOS}{../dos/dos.html}}
\newcommand{\seegetopts}{\htmladdnormallink{getopts}{../getopts/getopts.html}}
\newcommand{\seeobjects}{\htmladdnormallink{objects}{../objects/objects.html}}
\newcommand{\seegraph}{\htmladdnormallink{graph}{../graph/graph.html}}
\newcommand{\seeprinter}{\htmladdnormallink{printer}{../printer/printer.html}}
\newcommand{\seego}{\htmladdnormallink{GO32}{../go32/go32.html}}
%
% Nice environments
%
% For Code examples (complete programs only)
\newenvironment{CodEx}{}{}
% For Tables.
\newenvironment{FPCtable}[2]{\begin{table}\caption{#2}\begin{center}\begin{tabular}{#1}}{\end{tabular}\end{center}\end{table}}
% The same, but with label in third argument (tab:#3)
\newenvironment{FPCltable}[3]{\begin{table}\caption{#2}\label{tab:#3}\begin{center}\begin{tabular}{#1}}{\end{tabular}\end{center}\end{table}}
%
% Commands to reference these things.
%
\newcommand{\seet}[1]{table (\ref{tab:#1}) }
\newcommand{\seec}[1]{chapter (\ref{ch:#1}) }
\newcommand{\sees}[1]{section (\ref{se:#1}) }
}
\makeindex
%
% start of document.
%
\begin{document}
\title{Free Pascal :\\ Reference guide.}
\docdescription{Reference guide for Free Pascal.}
\docversion{1.4}
\date{March 1998}
\author{Micha\"el Van Canneyt
% \\ Florian Kl\"ampfl
}
\maketitle
\tableofcontents
\newpage
\listoftables
\newpage
\section*{About this guide}
This document describes all constants, types, variables, functions and
procedures as they are declared in the system unit.

Furthermore, it describes all pascal constructs supported by \fpc, and lists
all supported data types. It does not, however, give a detailed explanation
of the pascal language. The aim is to list which Pascal constructs are
supported, and to show where the \fpc implementation differs from the
Turbo Pascal implementation.

Throughout this document, we will refer to functions, types and variables
with \var{typewriter} font. Functions and procedures gave their own
subsections, and for each function or procedure we have the following 
topics:
\begin{description}
\item [Declaration] The exact declaration of the function.
\item [Description] What does the procedure exactly do ?
\item [Errors] What errors can occur.
\item [See Also] Cross references to other related functions/commands.
\end{description}
The cross-references come in two flavours:
\begin{itemize}
\item References to other functions in this manual. In the printed copy, a
number will appear after this reference. It refers to the page where this
function is explained. In the on-line help pages, this is a hyperlink, on
which you can click to jump to the declaration.
\item References to Unix manual pages. (For linux related things only) they
are printed in \var{typewriter} font, and the number after it is the Unix
manual section.
\end{itemize}
%
% The Pascal language
%
\chapter{Supported Pascal language constructs}
In this chapter we describe the pascal constructs supported by \fpc, as well
as the supported data types.

This is not intended as an introduction to the Pascal language, although all
language constructs will be covered. The main goal is to explain what is
supported by \fpc, and where the Free implementation differs from the Turbo
Pascal one.
\section{Data types}
\fpc supports the same data types as Turbo Pascal, with some extensions from
Delphi.
\subsection{Integer types}
The integer types predefined in \fpc are listed in \seet{integers}.

\begin{FPCltable}{lcr}{Predefined integer types}{integers}
Type & Range & Size in bytes \\ \hline
Byte & 0 .. 255 & 1 \\
Shortint & -127 .. 127 & 1\\
Integer & -32768 .. 32767 & 2 \\
Word & 0 .. 65535 & 2 \\
Longint & -2147483648 .. 2147483648 & 4\\
Cardinal\footnote{The cardinal type support is buggy until version 0.9.3} & 0..4294967296 & 4 \\ \hline
\end{FPCltable}

\fpc does automatic type conversion in expressions where different kinds of
integer types are used.

\fpc supports hexadecimal format the same way as Turbo Pascal does. To
specify a constant value in hexadecimal format, prepend it with a dollar
sign (\var{\$}). Thus, the hexadecimal \var{\$FF} equals 255 decimal.

In addition to the support for hexadecimal notation, \fpc also supports
binary notation. You can specify a binary number by preceding it with a
percent sign (\var{\%}). Thus, \var{255} can be specified in binary notation
as \var{\%11111111}.

\subsection{Real types}
\fpc uses the math coprocessor (or an emulation) for al its floating-point 
calculations. The native type for the coprocessor is \var{Double}. Other
than that, all Turbo Pascal real types are supported. They're listed in
\seet{Reals}.
 \begin{FPCltable}{lccr}{Supported Real types}{Reals}
Type & Range & Significant digits & Size\footnote{In Turbo Pascal.} \\ \hline
Real & 2.9E-39 .. 1.7E38 & 11-12 & 6 \\
Single & 1.5E-45 .. 3.4E38 & 7-8 & 4 \\
Double & 5.0E-324 .. 1.7E308 & 15-16 & 8 \\
Extended & 1.9E-4951 .. 1.1E4932 & 19-20 & 10\\
%Comp\footnote{\var{Comp} only holds integer values.} & -2E64+1 .. 2E63-1 & 19-20 & 8  \\
\end{FPCltable}

Until version 0.9.1 of the compiler, all the real types are mapped to type
\var{Double}, meaning that they all have size 8. From version 0.9.3, the
\var{Extended} and \var{single} types are defined with the same suze as in
Turbo Pascal. The \seef{SizeOf} function is your friend here.

\subsection{Character types}
\subsubsection{Char}
\fpc supports the type \var{Char}. A \var{Char} is exactly 1 byte in
size, and contains one character. 

You can specify a character constant by enclosing the character in single 
quotes, as follows : 'a' or 'A' are both character constants. 

You can also specify a character by their ASCII
value, by preceding the ASCII value with the number symbol (\#). For example
specifying \var{\#65} would be the same as \var{'A'}.

Also, the caret character (\verb+^+) can be used in combination with a letter to
specify a character with ASCII value less than 27. Thus \verb+^G+ equals
\var{\#7} (G is the seventh letter in the alphabet.)

If you want to represent the single quote character, type it two times
successively, thus \var{''''} represents the single quote character.

\subsubsection{Strings}

\fpc supports the \var{String} type as it is defined in Turbo Pascal.
To declare a variable as a string, use the following declaration:
\begin{verbatim}
Var
   S : String[Size];
\end{verbatim}
This will declare \var{S} as a variable of type \var{String}, with maximum
length \var{Size}. \var{Size} can be any value from \var{1} to \var{255}.

\fpc reserves \var{Size+1} bytes for the string \var{S}, and in the zeroeth
element of the string (\var{S[0]}) it will store the length of the variable.

If you don't specify the size of the string, \var{255} is taken as a
default.

To specify a constant string, you enclose the string in single-quotes, just
as a \var{Char} type, only now you can have more than one character.
Given that \var{S} is of type \var{String}, the following are valid assignments: 
\begin{verbatim}
S:='This is a string.';
S:='One'+', Two'+', Three';
S:='This isn''t difficult !';
S:='This is a weird character : '#145' !';
\end{verbatim}
As you can see, the single quote character is represented by 2 single-quote
characters next to each other. Strange characters can be specified by their
ASCII value.

The example shows also that you can add two strings. The resulting string is
just the concatenation of the first with the second string, without spaces in
between them. Strings can not be substracted, however.

\subsubsection{PChar}

\fpc supports the Delphi implementation of the \var{PChar} type. \var{PChar}
is defined as a pointer to a \var{Char} type, but allows additional
operations. 

The \var{PChar} type can be understood best as the Pascal equivalent of a
C-style null-terminated string, i.e. a variable of type \var{PChar} is a pointer
that points to an array of type \var{Char}, which is ended by a
null-character (\var{\#0}).

\fpc supports initializing of \var{PChar} typed constants, or a direct
assignment. For example, the following pieces of code are equivalent:

\begin{CodEx}
\begin{verbatim}
program one;

var p : pchar;

begin
  P:='This is a null-terminated string.';
  writeln (P);
end.
\end{verbatim}
\end{CodEx}
Results in the same as
\begin{CodEx}
\begin{verbatim}
program two;

const P : PChar = 'This is a null-terminated string.'

begin
  Writeln (P);
end.
\end{verbatim}
\end{CodEx}
These examples also show that it is possible to write {\em the contents} of
the string to a file of type \var{Text}.

The \seestrings\_ unit contains procedures and functions that manipulate the
\var{PChar} type as you can do it in C.

Since it is equivalent to a pointer to a type \var{Char} variable, it  is
also possible to do the following:
\begin{CodEx}
\begin{verbatim}
Program three;

Var S : String[30];
    P : Pchar;

begin
  S:='This is a null-terminated string.'#0;
  P:=@S[1];
  writeln (P);
end.
\end{verbatim}
\end{CodEx}
This will have the same result as the previous two examples.

You cannot add null-terminated strings as you can do with normal Pascal
strings. If you want to concatenate two \var{PChar} strings, you will need
to use the \seestrings unit.

However, it is possible to do some pointer arithmetic. You can use the
operators \var{+} and \var{-} to do operations on \var{PChar} pointers.
In \seet{PCharMath}, \var{P} and \var{Q} are of type \var{PChar}, and
\var{I} is of type \var{Longint}.
\begin{FPCltable}{lr}{\var{PChar} pointer arithmetic}{PCharMath}
Operation & Result \\ \hline
\var{P + I} & Adds \var{I} to the address pointed to by \var{P}. \\
\var{I + P} & Adds \var{I} to the address pointed to by \var{P}. \\
\var{P - I} & Substracts \var{I} from the address pointed to by \var{P}. \\
\var{P - Q} & Returns, as an integer, the distance between 2 addresses \\
 & (or the number of characters between \var{P} and \var{Q}) \\
\hline
\end{FPCltable}

\subsection{Booleans}
\fpc supports the \var{Boolean} type, with its two pre-defined possible
values \var{True} and \var{False}. These are the only two values that can be
assigned to a \var{Boolean} type. Of course, any expression that resolves
to a \var{boolean} value, can also be assigned to a boolean type.

Assuming \var{B} to be of type \var{Boolean}, the following are valid
assignments:
\begin{verbatim}
 B:=True;
 B:=False;
 B:=1<>2;  { Results in B:=True }
\end{verbatim}
Boolean expressions are also used in conditions.

{\em Remark:} In \fpc, boolean expressions are always evaluated in such a
way that when the result is known, the rest of the expression will no longer
be evaluated (Called short-cut evaluation). In the following example, the function \var{Func} will never
be called, which may have strange side-effects.
\begin{verbatim}
 ...
 B:=False;
 A := B and Func;
\end{verbatim} 
Here \var{Func} is a function which returns a \var{Boolean} type.

\subsection{Arrays}
\fpc supports arrays as in Turbo Pascal, multi-dimensional arrays 
and packed arrays are also supported.

\subsection{Pointers}
\fpc supports the use of pointers. A variable of the type \var{Pointer}
contains an address in memory, where the data of another variable may be 
stored.

Pointers can be typed, which means that they point to a particular kind of
data. The type of this data is known at compile time.

Consider the following example:
\begin{CodEx}
\begin{verbatim}
Program pointers;

type 
  Buffer = String[255];
  BufPtr = ^Buffer;

Var B  : Buffer;
    BP : BufPtr;
    PP : Pointer;

etc..
\end{verbatim}
\end{CodEx}
In this example, \var{BP} {\em is a pointer to} a \var{Buffer} type; while \var{B}
{\em is} a variable of type \var{Buffer}. \var{B} takes 256 bytes memory,
and \var{BP} only takes 4 bytes of memory (enough to keep an adress in
memory).

{\em Remark:} \fpc treats pointers much the same way as C does. This means
that you can treat a pointer to some type as being an array of this type.  
The pointer then points to the zeroeth element of this array. Thus the
following pointer declaration 
\begin{verbatim}
Var p : ^Longint;
\end{verbatim}
Can be considered equivalent to the following array declaration:
\begin{verbatim}
Var p : array[0..Infinity] of Longint;
\end{verbatim}
The reference \verb+P^+ is then the same as \var{p[0]}. The following program
illustrates this maybe more clear:
\begin{CodEx}
\begin{verbatim}
program PointerArray;

var i : longint;
    p : ^longint;
    pp : array[0..100] of longint;  

begin
  for i:=0 to 100 do pp[i]:=i; { Fill array }
  p:=@pp[0];                   { Let p point to pp }
  for i:=0 to 100 do if p[i]<>pp[i] then writeln ('Ohoh, problem !')
end.
\end{verbatim}
\end{CodEx}
\fpc doesn't support pointer arithmetic as C does, however.

\subsection{Procedural types}
\fpc has support for procedural types, although it differs from the Turbo
Pascal implementation of them.

The type declaration remains the same. The two following examples are valid
type declarations:
\begin{verbatim}
Type TOneArg = Procedure (Var X : integer);
     TNoArg = Function : Real;

var proc : TOneArg;
    func : TNoArg;
\end{verbatim}
Given these declarations, the following assignments are valid:
\begin{verbatim}
Procedure printit (Var X : Integer);

begin
  writeln (x);
end;
...

P:=@printit;
Func:=@Pi;
\end{verbatim}
From this example, the difference with Turbo Pascal is clear: In Turbo
Pascal it isn't necessary to use the address operator (\var{@}) 
when assigning a procedural type variable, whereas in \fpc it is required.

\subsection{Records}

\fpc supports records. The prototype type definition of a record is:
\begin{verbatim}
Type
  RecType = Record
    Element1 : type1;
    Element2,Element3 : type2;
    ...
    Elementn ; Typen;
  end;
\end{verbatim}
Variant records are also supported:
\begin{verbatim}
Type
  RecType = Record
    Element1 : type1;
    Case [PivotElmt:] Type Identifier of
      Value1 : (VarElt1, Varelt2 : Vartype1);
      Value2 : (VarElt3, Varelt4 : Vartype2);
  end;
\end{verbatim}
The variant part must be last in the record. The optional \var{PivotElmt}
can be used to see which variant is active at a certain time.

{\em Remark:} If you want to read a typed file with records, produced by
a Turbo Pascal program, then chances are that you will not succeed in
reading that file correctly. 

The reason for this is that by default, elements of a record are aligned at
2-byte boundaries, for performance reasons. This default behaviour can be
changed with the \var{\{\$PackRecords n\}} switch. Possible values for
\var{n} are 1, 2 and 4. This switch tells the compiler to align elements of
a record or object or class on 1,2 or 4 byte boundaries. 

Take a look at the following program:
\begin{CodEx}
\begin{verbatim}
Program PackRecordsDemo;

type {$PackRecords 2}
     Trec1 = Record
       A : byte;
       B : Word;
       
     end;
     
     {$PACKRECORDS 1}
     Trec2 = Record
       A : Byte;
       B : Word;
       end;

begin
  Writeln ('Size Trec1 : ',SizeOf(Trec1));
  Writeln ('Size Trec2 : ',SizeOf(Trec2));
end.
\end{verbatim}
\end{CodEx}
The output of this program will be :
\begin{verbatim}
Size Trec1 : 4
Size Trec2 : 3
\end{verbatim}
And this is as expected. In \var{Trec1}, each of the elements \var{A} and
\var{B} takes 2 bytes of memory, and in  \var{Trec1}, \var{A} takes only 1
byte of memory.

{\em Remark:} As from version 0.9.3 (a developers' version), \fpc supports also the
'packed record', this is a record where all the elements are byte-aligned.

Thus the two following declarations are equivalent:
\begin{verbatim}
     {$PACKRECORDS 1}
     Trec2 = Record
       A : Byte;
       B : Word;
       end;
     {$PACKRECORDS 2}
\end{verbatim}
and
\begin{verbatim}
     Trec2 = Packed Record
       A : Byte;
       B : Word;
       end;
\end{verbatim}
Note the \var{\{\$PACKRECORDS 2\}} after the first declaration !

\subsection{Set types}

\fpc supports the set types as in Turbo Pascal. The prototype of a set
declaration is: 
\begin{verbatim}
SetType = Set of TargetType;
\end{verbatim}

Each of the elements of \var{SetType} must be of type \var{TargetType}.
\var{TargetType} can be any ordinal type with a range between \var{0} and
\var{255}. A set can contain maximally \var{255} elements.

The following is a valid set declaration:
\begin{verbatim}
Type Days = (Mon, Tue, Wed, Thu, Fri, Sqt, Sun);

Var WeekDays : Set of days;
\end{verbatim}
Given this set declaration, the follwing assignment is legal:
\begin{verbatim}
WeekDays := [ Mon, Tue, Wed, Thu, Fri];
\end{verbatim}
The operators for manipulations of sets are listed in \seet{SetOps}.
\begin{FPCltable}{lr}{Set Manipulation operators}{SetOps}
Operation & Operator \\ \hline
Union & + \\
Difference & - \\
Intersection & * \\ \hline
\end{FPCltable}

You can compare two sets with the \var{<>} and \var{=} operators, but not
(yet) with the \var{<} and \var{>} operators. 

From compiler version 0.9.5, the compiler stores small sets (less than 32
elements) in a longint, if the type range allows it. This allows for faster
processing and decreases program size.

\subsection{Enumeration types}

Enumeration types are supported in \fpc. On top of the Turbo Pascal
implementation, \fpc allows the following C-style extension of the
enumeration type.
\begin{verbatim}
Type
  EnumType = (one, two, three, forty := 40);
\end{verbatim}
As a result, the ordinal number of \var{forty} is \var{40}, and not \var{4},
as it would be when the \var{'= 40'} wasn't present.

When specifying such an enumeration type, it is important to keep in mind
that you should keep initialized set elements in ascending order. The
following will produce a compiler error:
\begin{verbatim}
Type
  EnumType = (one, two, three, forty := 40, thirty:=30);
\end{verbatim}
It is necessary to keep \var{forty} and \var{Thirty} in the correct order.

{\em Remark :} You cannot use the \var{Pred} and \var{Succ} functions on
this kind of enumeration types. If you try to do that, you'll get a compiler
error.

\section{Constants}

Just as in Turbo Pascal, \fpc supports both normal and typed constants.
\subsection{Ordinary constants}
Ordinary constants declarations are no different from the TP implementation. 
You can only declare constants of the following types: \var{Ordinal types},
\var{Real types}, \var{Char}, and \var{String}. 
The following are all valid constant declarations:
\begin{verbatim}
Const
  e = 2.7182818;  { Real type constant. }
  a = 2;          { Integer type constant. }
  c = '4';        { Character type constant. }
  s = 'This is a constant string'; {String type constant.}
\end{verbatim}
Assigning a value to a constant is not permitted. Thus, given the previous
declaration, the following will result in a compiler error:
\begin{verbatim}
  s:='some other string';
\end{verbatim}

\subsection{Typed constants}
Typed constants serve to provide a program with initialized variables.
Contrary to ordinary constants, they may be assigned to at run-time.
The difference with normal variables is that their value is initialised
when the program starts, whereas normal variables must be initialised
explicitly.

The prototype of a typed constant declaration is:
\begin{verbatim}
Const
  SomeConst : SomeType = SomeValue;
\end{verbatim}
After that, the constant \var{SomeConst} will be of type \var{SomeType}, and
have initial value \var{SomeValue}.

Given the declaration:
\begin{verbatim}
Const
  S : String = 'This is a typed constant string';
\end{verbatim}
The following is a valid assignment:
\begin{verbatim}
 S:='Result : '+Func;
\end{verbatim}
Where \var{Func} is a function that returns a \var{String}.

Typed constants also allow you to initialize arrays and records. For arrays,
the initial elements must be specified, surrounded by round brackets, and
separated by commas. The number of elements must be exactly the same as
number of elements in the declaration of the type. 

As an example:
\begin{verbatim}
Const 
  tt : array [1..3] of string[20] = ('ikke','gij', 'hij');
  ti : array [1..3] of longint = (1,2,3);
\end{verbatim}

For constant records, you should specify each element of the record, in the
form \var{Field : Value}, separated by commas, and surrounded by round
brackets.

As an example:
\begin{verbatim}
Type 
  Point = record
    X,Y : Real
    end;

Const
  Origin : Point = (X:0.0 , Y:0.0); 
\end{verbatim}
The order of the fields in a constant record needs to be the same as in the type declaration,
otherwise you'll get a compile-time error.

\section{Objects}
\fpc supports object oriented programming. In fact, part of the compiler is
written using objects. Here we present some technical questions regarding
object oriented programming in \fpc.

\fpc supports 2 programming models for object-oriented programming.
You can choose to program object oriented using the Turbo Pascal approach,
or you can prefer the Delphi approach.

\subsection{The Turbo Pascal approach}
In the Turbo Pascal approach, Objects should be treated as a special kind of
record. The record contains all the fields that are declared in the objects
definition, and pointers to the methods that are associated to the objects'
type.

An object is declared just as you would declare a record; except that you
can now declare procedures and fuctions as of they were part of the record.

Objects can ''inherit'' fields and methods from ''parent'' objects. This means
that you can use these fields and methods as if the were included in the
objects you declared as a ''child'' object. 

Furthermore, you can declare fields, procedures and functions as \var{public}
or \var{private}. By default, fields and methods are \var{public}, and are
exported outside the current unit. Fields or methods that are declared
\var{private} are only accessible in the current unit.

The prototype declaration of an object is as follows :
\begin{verbatim}
TObj = Object [(ParentObjectType)]
  [Constructor ConstructorName;]
  [Destructor DestructorName;]
  Field1 : Type1;
  ...
  Fieldn : Typen;
  Method1;
  Method2;
  [private
  PrField1 : PrType1;
  ...
  PrFieldn : PrTypen;
  PrMethod1;
  ...
  PrMethodn;]
  [public
  PuField1 : PuType1;
  ..
  Pufield1 : PuTypen;
  PuMethod1;
  ...
  PuMethodn;]
  end;
\end{verbatim}
You can repeat as many \var{private} and \var{public} blocks as you want.
\var{Method}s are normal function or procedure declarations. 

As can be seen in the prototype object declaration, \fpc supports
constructors and destructors. You are responsible for calling the 
destructor and constructor explicitly when using objects.

\fpc supports also the extended syntax of the \var{New} and \var{Dispose}
procedures. In case you want to allocate a dynamic varible of an object
type, you can specify the constructor's name in the call to \var{New}.
The \var{New} is implemented as a function which returns a pointer to the
instantiated object. Given the following declarations :
\begin{verbatim}
Type
  TObj = object;
   Constructor init;
   ...
   end;
  Pobj = ^TObj;

Var PP : Pobj;
\end{verbatim}  
Then the following 3 calls are equivalent :
\begin{verbatim}
 pp:=new (Pobj,Init);
\end{verbatim}
and
\begin{verbatim}
  new(pp,init);
\end{verbatim}
and also
\begin{verbatim}
  new (pp);
  pp^.init;
\end{verbatim}
In the last case, the compiler will issue a warning that you should use the
extended syntax of \var{new} and \var{dispose} to generate instances of an
object. You can ignore this warning, but it's better programming practice to
use the extended syntax to create instances of an object.

Similarly, the \var{Dispose} procedure accepts the name of a destructor. The
destructor will then be called, before removing the object from the heap.

In view of the compiler warning remark, the now following Delphi approach may 
be considered a more natural way of object-oriented programming.

{\em Remark:}
\fpc also supports the packed object. This is the same as an object, only 
the elements (fields) of the object are byte-aligned, just as in the packed
record.

The declaration of a packed object is similar to the declaration
of a packed record :
\begin{verbatim}
Type
  TObj = packed object;
   Constructor init;
   ...
   end;
  Pobj = ^TObj;

Var PP : Pobj;
\end{verbatim}  
Similarly, the \var{\{\$PACKRECORDS \}} directive acts on objects as well.

\subsection{The Delphi approach}
In the Delphi approach to Object Oriented Programming, everything revolves
around  the concept of 'Classes'. 
A class can be seen as a pointer to an object, or a pointer to a record. 

The prototype declaration of a class is as follows :
\begin{verbatim}
TObj = Class [(ParentClassType)]
  [Constructor ConstructorName;]
  [Destructor DestructorName;]
  Field1 : Type1;
  ...
  Fieldn : Typen;
  Method1;
  Method2;
  [private
  PrField1 : PrType1;
  ...
  PrFieldn : PrTypen;
  PrMethod1;
  ...
  PrMethodn;]
  [public
  PuField1 : PuType1;
  ..
  Pufield1 : PuTypen;
  PuMethod1;
  ...
  PuMethodn;]
  end;
\end{verbatim}
You can repeat as many \var{private} and \var{public} blocks as you want.
\var{Method}s are normal function or procedure declarations. 

As you can see, the declaration of a class is almost identical to the
declaration of an object. The real difference between objects and classes
is in the way they are created;

Classes must be created using their constructor. Remember that A class is a
pointer to an object, so when you declare a variable of some class, the
compiler just allocates a pointer, not the entire object. The constructor of
a class returns a pointer to an initialized instance of the object.

So, to initialize an instance of some class, you do the following :
\begin{verbatim}
  ClassVar:=ClassType.ConstructorName;
\end{verbatim}

{\em Remark :}
\begin{itemize}
\item \fpc doesn't support the concept of properties yet.
\item The \var{\{\$Packrecords \}} directive also affects classes.
i.e. the alignment in memory of the different fields depends on the
value of  the \var{\{\$Packrecords \}} directive.
\item Just as for objects and records, you can declare a packed class.
This has the same effect as on an object, or record, namely that the
elements are aligned on 1-byte boundaries. i.e. as close as possible.
\end{itemize}

\section{Statements controlling program flow.}

\subsection{Assignments}
In addition to the standard Pascal assignment operator (\var{:=}), \fpc
supports some c-style constructions. All available constructs are listed in
\seet{assignments}.
\begin{FPCltable}{lr}{Allowed C constructs in \fpc}{assignments}
Assignment & Result \\ \hline
a += b & Adds \var{b} to \var{a}, and stores the result in \var{a}.\\
a -= b & Substracts \var{b} from \var{a}, and stores the result in
\var{a}. \\
a *= b & Multiplies \var{a} with \var{b}, and stores the result in
\var{a}. \\
a /= b & Divides \var{a} through \var{b}, and stores the result in
\var{a}. \\ \hline
\end{FPCltable}
For these connstructs to work, you should specify the \var{-Sc} 
command-line switch. 

{\em Remark:} These constructions are just for typing convenience, they
don't generate different code.

\fpc also supports typed assignments. This means that an assignment
statement has a definite type, and hence can be assigned to another
variable. The type of the assignment \var{a:=b} is the type of \var{a}
(or, in this case, of \var{b}), and this can be assigned to another
variable : \var{c:=a:=b;}.
To summarize: the construct
\begin{verbatim}
 a:=b:=c;
\end{verbatim}
results in both \var{a} and \var{b} being assign the value of \var{c}, which
may be an expression.

For this construct to be allowed, it is necessary to specify the \var{-Sa4}
switch on the command line.

\subsection{The \var{Case} statement}
\fpc supports the \var{case} statement. Its prototype is
\begin{verbatim}
Case Pivot of
  Label1 : Statement1;
  Label2 : Statement2;
  ...
  Labeln : Statementn;
[Else
  AlternativeStatement]
end;
\end{verbatim}
\var{label1} until \var{Labeln} must be known at compile-time, and can be of
the following types : enumeration types, Ordinal types (except boolean), and
chars. \var{Pivot} must also be one of these types.

The statements \var{Statement1} etc., can be compound statements (i.e. a
\var{begin..End} block).

{\em Remark:} Contrary to Turbo Pascal, duplicate case labels are not
allowed in \fpc, so the following code will generate an error when
compiling:

\begin{verbatim}
Var i : integer;
...

Case i of
 3 : DoSomething;
 1..5 : DoSomethingElse;
end;
\end{verbatim}
The compiler will generate a \var{Duplicate case label} error when compiling
this, because the 3 also appears (implicitly) in the range \var{1..5}

{\em Remark:} In versions earlier than 0.9.7, there was an incompatibility here 
with Turbo Pascal. Where in Turbo Pascal you could do the following:
\begin{verbatim}
case Pivot of
  ...
Else
  begin
  Statement1
  Statement2
  end;
\end{verbatim}
You needed to do the following in \fpc :
\begin{verbatim}
case Pivot of
  ...
Else
  begin
  Statement1
  Statement2
  end;
end;
\end{verbatim}
So there's an extra \var{end} keyword at the end. But from version 0.9.7
this has been fixed.
\subsection{The \var{For..to/downto..do} statement}
\fpc supports the \var{For} loop construction. The prototypes are:
\begin{verbatim}
For Counter:=Lowerbound to Upperbound Do Statement;

or 

For Counter:=Upperbound downto Lowerbound Do Statement;
\end{verbatim}
\var{Statement} can be a compound statement. In the first case, if
\var{Lowerbound} is larger than \var{Upperbound} then \var{Statement} will
never be executed.
\subsection{The \var{Goto} statement}
\fpc supports the \var{goto} jump statement. Its prototype is
\begin{verbatim}

var
  jumpto : label
...
Jumpto : 
  Statement;
...
Goto jumpto;
...
\end{verbatim}
The jump label must be defined in the same block as the \var{Goto}
statement.
To be able to use the \var{Goto} statement, you need to specify the \var{-Sg}
compiler switch.
\subsection{The \var{If..then..else} statement}
The \var{If .. then .. else..} prototype is:
\begin{verbatim}
If Expression1 Then Statement1;

or 

If Expression2 then 
   Statement2
else
   Statement3;
\end{verbatim}
Be aware of the fact that the boolean expressions \var{Expression1} and
\var{Expression2} will be short-cut evaluated. (Meaning that the evaluation
will be stopped at the point where the outcome is known with certainty)

Also, after \var{Statement2}, no semicolon (\var{;}) is alllowed.

All statements can be compound statements.
\subsection{The \var{Repeat..until} statement}
The prototype of the \var{Repeat..until} statement is
\begin{verbatim}
Repeat
  Statement1;
  Statement2;
Until Expression;
\end{verbatim}
This will execute \var{Statement1} etc. until \var{Expression} evaluates to
\var{True}. Since \var{Expression} is evaluated {\em after} the execution of the
statements, they are executed at least once.

Be aware of the fact that the boolean expressions \var{Expression1} and
\var{Expression2} will be short-cut evaluated. (Meaning that the evaluation
will be stopped at the point where the outcome is known with certainty)

\subsection{The \var{While..do} statement}
The prototype of the \var{While..do} statement is
\begin{verbatim}
While Expression Do
  Statement;
\end{verbatim}
This will execute \var{Statement} as long as \var{Expression} evaluates to
\var{True}. Since \var{Expression} is evaluated {\em before} the execution
of \var{Statement}, it is possible that \var{Statement} isn't executed at
all.

\var{Statement} can be a compound statement.

Be aware of the fact that the boolean expressions \var{Expression1} and
\var{Expression2} will be short-cut evaluated. (Meaning that the evaluation
will be stopped at the point where the outcome is known with certainty)

\subsection{The \var{With} statement}
The with statement serves to access the elements of a record, without
having to specify the name of the record. Given the declaration:
\begin{verbatim}
Type Passenger = Record
       Name : String[30];
       Flight : String[10];
       end;

Var TheCustomer : Passenger;
\end{verbatim}
The following statements are completely equivalent:
\begin{verbatim}
TheCustomer.Name:='Michael';
TheCustomer.Flight:='PS901';
\end{verbatim}
and
\begin{verbatim}
With TheCustomer do
  begin
  Name:='Michael';
  Flight:='PS901';
  end;
\end{verbatim}
 
\subsection{Compound statements}
Compound statements are a group of statements, separated by semicolons, 
that are surrounded by the keywords \var{Begin} and \var{End}. The
Last statement doesn't need to be followed by a semicolon, although it is
allowed.

\section{Using functions and procedures}
\fpc supports the use of functions and procedures, but with some extras:
Function overloading is supported, as well as \var{Const} parameters and
open arrays.

{\em remark:} In the subsequent paragraph the word \var{procedure} and
\var{function} will be used interchangeably. The statements made are
valid for both.

\subsection{Function overloading}
Function overloading simply means that you can define the same function more
than once, but each time with a different set of arguments.

When the compiler encounters a unction call, it will look at the function
parameters to decide which od the defined function
This can be useful if you want to define the same function for different
types. For example, if the RTL, the  \var{Dec} procedure is
is defined as:
\begin{verbatim}
...
Dec(Var I : longint;decrement : longint);
Dec(Var I : longint);
Dec(Var I : Byte;decrement : longint);
Dec(Var I : Byte);
...
\end{verbatim}
When the compiler encounters a call to the dec function, it wil first search
which function it should use. It therefore checks the parameters in your
function call, and looks if there is a function definition which maches the
specified parameter list. If the compiler finds such a function, a call is
inserted to that function. If no such function is found, a compiler error is
generated.

\subsection{\var{Const} parameters}
In addition to \var{var} parameters and normal parameters (call by value,
call by reference), \fpc also supports \var{Const} parameters. You can
specify a \var{Const} parameter as follows:
\begin{verbatim}
Function Name (Const S: Type_Of_S) : ResultType
\end{verbatim}
A constant argument is passed by refenence 
(i.e. the function or procedure receives a pointer to the passed , 
but you are not allowed to assign to it, this will result in a compiler error.

The main use for this is reducing the stack size, hence improving
performance.

\subsection{Open array parameters}
\fpc supports the passing of open arrays, i.e. You can declare a procedure
with an array of unspecified length as a parameter, as in Delphi.

The prototype declaration for open array parameters is:
\begin{verbatim}
Function Func ( ... [Var|Const] Ident : Array of Type ...) : ReturnType;

ProcedureFunction Func (... [Var|Const] Ident : Array of Type ...);
\end{verbatim}
The \var{[Var|Const]} means that open parameters can be passed by reference
or as a constant parameter.

In a function or procedure, you can pass open arrays only to functions which 
are also declared with open arrays as parameters, {\em not} to functions or 
procedures which accept arrays of fixed length.

\section{Using assembler in your code}
\fpc supports the use of assembler in your code, but not inline
assembler. assembly functions (i.e. functions declared with the
\var{Assembler} keyword) are supported as of version 0.9.7.

{\em Remark :}
\fpc issues AT\&T assembly language, as understood by most
unix assemblers (most notably : GNU \var{as}). 
Intel assembler syntax is available as of version 0.9.8 but the Intel
support isn't complete in the sense that it is converted to AT\&T syntax,
and some constructions aren't supported by the conversion mechanism (see
\progref for more information about this).
Therefore all examples of assembly language will be given in AT\&T syntax,
as it is the 'native' assembly from \fpc.

The following is an example of assembler inclusion in your code.
\begin{verbatim}
 ...
 Statements;
 ...
 Asm
   Movl 0,%eax
   ...
 end;
 ...
 Statements;
\end{verbatim}
The assembler instructions between the \var{Asm} and \var{end} keywords will
be inserted in the assembler generated by the compiler.

You can still use comditionals in your assembler, the compiler will
recognise it, and treat it as any other conditionals.

Contrary to Turbo Pascal, it isn't possible (yet) to reference variables by 
their names in the assembler parts of your code.

\section{Modifiers}
\fpc doesn't support all Turbo Pascal modifiers, but
does support a number of additional modifiers. They are used mainly for assembler and
reference to C object files. 

\subsection{Public}
The \var{Public} keyword is used to declare a function globally in a unit.
This is useful if you don't want a function to be accessible from the unit
file, but you do want the function to be accessible from the object file.

as an example:
\begin{CodEx}
\begin{verbatim}
Unit someunit;

interface

Function First : Real;

Implementation

Function First : Real;
begin
  First:=0;
end;

Function Second : Real; [Public];

begin
  Second:=1;
end;

end.
\end{verbatim} 
\end{CodEx}
If another program or unit uses this unit, it will not be able to use the
function \var{Second}, since it isn't declared in the interface part.
However, it will be possible to access the function \var{Second} at the
assembly-language level, by using it's mangled name (\progref).

\subsection{cdecl}
\label{se:cdecl}
The \var{cdecl} modifier can be used to declare a function that uses a C
type calling convention. This must be used if you wish to acces functions in
an object file generated by a C compiler. It allows you to use the function in
your code, and at linking time, you must link the object file containing the
\var{C} implementation of the function or procedure.

As an example:
\begin{CodEx}
\begin{verbatim}
program CmodDemo;

{$LINKLIB c}

Const P : Pchar = 'This is fun !';

Function strlen (P : Pchar) : Longint; cdecl; external;

begin
  Writeln ('Length of (',p,') : ',strlen(p))
end.
\end{verbatim}
\end{CodEx}
When compiling this, and linking to the C-library, you will be able to call
the \var{strlen} function throughout your program. The \var{external}
directive tells the compiler that the function resides in an external
object filebrary (see \ref{se:external}). 

{\em Remark} The parameters in our declaration of the \var{C} function should 
match exactly the ones in the declaration in \var{C}. Since \var{C} is case 
sensitive, this means also that the name of the
function must be exactly the same. the \fpc compiler will use the name {\em
exactly} as it is typed in the declaration.

\subsection{popstack}
\label{se:popstack}
Popstack does the same as \var{cdecl}, namely it tells the \fpc compiler
that a function uses the C calling convention. In difference with the
\var{cdecl} modifier, it still mangles the name of the function as it would 
for a normal pascal function.

With \var{popstack} you could access functions by their pascal names in a
library.

\subsection{external}
\label{se:external}
The \var{external} modifier can be used to declare a function that resides in
an external object file. It allows you to use the function in
your code, and at linking time, you must link the object file containing the
implementation of the function or procedure.

As an example:
\begin{CodEx}
\begin{verbatim}
program CmodDemo;

{$Linklib c}

Const P : Pchar = 'This is fun !';

Function strlen (P : Pchar) : Longint; cdecl; external;

begin
  Writeln ('Length of (',p,') : ',strlen(p))
end.
\end{verbatim}
\end{CodEx}

{\em Remark} The parameters in our declaration of the \var{external} function 
should match exactly the ones in the declaration in the object file.
Since \var{C} is case sensitive, this means also that the name of the
function must be exactly the same.

The \var{external} modifier has also an extended syntax:
\begin{enumerate}
\item

\begin{verbatim}
external 'lname';
\end{verbatim}
Tells the compiler that the function resides in library 'lname'. The
compiler will the automatically link this library to your program.

\item
\begin{verbatim}
external 'lname' name Fname;
\end{verbatim}
Tells the compiler that the function resides in library 'lname', but with
name 'Fname'. The compiler will the automatically link this library to your 
program, and use the correct name for the function.

\item \windows and \ostwo only:
\begin{verbatim}
external 'lname' Index Ind;
\end{verbatim}
Tells the compiler that the function resides in library 'lname', but with
indexname \var{Ind}. The compiler will the automatically link this library to your 
program, and use the correct index for the function.
\end{enumerate}

\subsection{Export}
Sometimes you must provide a callback function for a C library, or you want
your routines to be callable from a C program. Since \fpc and C use
different calling schemes for functions and procedures\footnote{More
techically: In C the calling procedure must clear the stack. In \fpc, the
subroutine clears the stack.}, the compiler must be told to generate code
that can be called from a C routine. This is where the \var{Export} modifier
comes in. Contrary to the other modifiers, it must be specified separately,
as follows:
\begin{verbatim}
function DoSquare (X : longint) : longint; export;

begin
...
end;
\end{verbatim} 
The square brackets around the modifier are not allowed in this case.

{\em Remark:} You cannot call an exported function from within \fpc programs.
If you try to do so, the compiler will complain when compiling your source
code.

If you do want to call an exported procedure from pascal, you must use a
dummy function:
\begin{verbatim}
Procedure RealDoSomething;
begin
...
end;

Procedure DoSomething; export;

begin
  RealDoSomething;
end; 
\end{verbatim}
In this example, from your \fpc code, you can call the \var{RealDoSomething}
procedure. If someone wants to link to your code from a C program, he can
call the \var{DoSomething} procedure. Both calls will have the same effect.

{\em Remark:}
as of version 0.9.8, \fpc supports the Delphi \var{cdecl} modifier. 
This modifier works in the same way as the \var{export} modifier.

More information about these modifiers can be found in the \progref, in the
section on the calling mechanism and the chapter on linking.

\subsection{StdCall}
As of version 0.9.8, \fpc supports the Delphi \var{stdcall} modifier.
This modifier does actually nothing, since the \fpc compiler by default 
pushes parameters from right to left on the stack, which is what the 
modifier does under Delphi (which pushes parameters on the stack from left to 
right).

More information about this modifier can be found in the \progref, in the
section on the calling mechanism and the chapter on linking.

\subsection{Alias}
The \var{Alias} modifier allows you to specify a different name for a
procedure or function. This is mostly useful for referring to this procedure
from assembly language constructs. As an example, consider the following
program:

\begin{CodEx}
\begin{verbatim}
Program Aliases;

Procedure Printit; [Alias : 'DOIT'];

begin
  Writeln ('In Printit (alias : "DOIT")');
end;

begin
  asm
  call DOIT
  end;
end.
\end{verbatim}
\end{CodEx}
{\rm Remark:} the specified alias is inserted straight into the assembly
code, thus it is case sensitive.

The \var{Alias} modifier, combined with the \var{Public} modifier, make a
powerful tool for making externally accessible object files.

\subsection{[RegisterList]}
This modifier list is used to indicate the registers that are modified by an
assembler block in your code. The compiler stores certain results in the
registers. If you modify theregisters in an assembly block, the compiler
should, sometimes, be told about it.
The prototype syntax of the \var{Registerlist} modifier is:
\begin{verbatim}
asm
  statements
end; ['register1','register2',...,'registern'];
\end{verbatim}
Where \var{'register'} is one of \var{'EAX',EBX',ECX','EDX'} etc.


\subsection{Unsupported Turbo Pascal modifiers}
The modifiers that exist in Turbo pascal, but aren't supported by \fpc, are
listed in \seet{Modifs}.
\begin{FPCltable}{lr}{Unsupported modifiers}{Modifs}
Modifier & Why not supported ? \\ \hline
Near & \fpc is a 32-bit compiler.\\
Far & \fpc is a 32-bit compiler. \\
External & Replaced by \var{C} modifier. \\ \hline
\end{FPCltable}

%
% System unit reference guide.
%

\chapter{Reference : The system unit}
The system unit contains the standard supported functions of \fpc. It is the
same for all platforms. Basically it is the same as the system unit provided
with Borland or Turbo Pascal. 

Functions are listed in alphabetical order.
Arguments to functions or procedures that are optional are put between
square brackets.

The pre-defined constants and variables are listed in the first section. The
second section contains the supported functions and procedures.
\section{Types, Constants and Variables}
\subsection{Types}
The following integer types are defined in the System unit:
\begin{verbatim}
shortint = -128..127;
longint  = $80000000..$7fffffff;
integer  = -32768..32767;
byte     = 0..255;
word     = 0..65535;
\end{verbatim}

The following Real types are declared: 
\begin{verbatim}
double = real;
{$ifdef VER0_6}
  extended = real;
  single = real;
{$endif VER0_6}
\end{verbatim}

And the following pointer types:
\begin{verbatim}
  pchar = ^char;
  ppchar = ^pchar;
\end{verbatim}

\subsection{Constants}
The following constants for file-handling are defined in the system unit:
\begin{verbatim}
Const
  fmclosed = $D7B0;
  fminput  = $D7B1;
  fmoutput = $D7B2;
  fminout  = $D7B3;
  fmappend = $D7B4;

  filemode : byte = 2;
\end{verbatim}
Further, the following general-purpose constants are also defined:
\begin{verbatim} 
const
  test8086 : byte = 2; { always i386 or newer }
  test8087 : byte = 3; { Always 387 or higher. emulated if needed. }
  erroraddr : pointer = nil;
  errorcode : word = 0;
 { max level in dumping on error }
  max_frame_dump : word = 20;
\end{verbatim}

\subsection{Variables}
The following variables are defined and initialized in the system unit:
\begin{verbatim}
var
  output,input,stderr : text;
  exitproc : pointer;
  exitcode : word;
  stackbottom : longint;
  loweststack : longint;
\end{verbatim}
The variables \var{ExitProc}, \var{exitcode} are used in the \fpc exit
scheme. It works similarly to the on in Turbo Pascal:

When a program halts (be it through the call of the \var{Halt} function or
\var{Exit} or through a run-time error), the exit mechanism checks the value
of \var{ExitProc}. If this one is non-\var{Nil}, it is set to \var{Nil}, and
the procedure is called. If the exit procedure exits, the value of ExitProc
is checked again. If it is non-\var{Nil} then the above steps are repeated.

So if you want to install your exit procedure, you should save the old value
of \var{ExitProc} (may be non-\var{Nil}, since other units could have set it before 
you did). In your exit procedure you then restore the value of
\var{ExitProc}, such that if it was non-\var{Nil} the exit-procedure can be
called.

The \var{ErrorAddr} and \var{ExitCode} can be used to check for
error-conditions. If \var{ErrorAddr} is non-\var{Nil}, a run-time error has
occurred. If so, \var{ExitCode} contains the error code. If \var{ErrorAddr} is
\var{Nil}, then {ExitCode} contains the argument to \var{Halt} or 0 if the
program terminated normally.

\var{ExitCode} is always passed to the operating system as the exit-code of
your process.

Under \file{GO32}, the following constants are also defined :
\begin{verbatim}
const
   seg0040 = $0040;
   segA000 = $A000;
   segB000 = $B000;
   segB800 = $B800;
\end{verbatim}
These constants allow easy access to the bios/screen segment via mem/absolute.

\section{Functions and Procedures}
\function{Abs}{(X : Every numerical type)}{Every numerical type}
{\var{Abs} returns the absolute value of a variable. The result of the
function has the same type as its argument, which can be any numerical
type.}
{None.}
{\seef{Round}}

\input{refex/ex1.tex}

\function{Addr}{(X : Any type)}{Pointer}
{\var{Addr} returns a pointer to its argument, which can be any type, or a
function or procedure name. The returned pointer isn't typed.
The same result can be obtained by the \var{@} operator, which can return a
typed pointer (\progref). }
{None}
{\seef{SizeOf}}

\input{refex/ex2.tex}

\procedure{Append}{(Var F : Text)}
{\var{Append} opens an existing file in append mode. Any data written to
\var{F} will be appended to the file. If the file didn't exist, it will be
created, contrary to the Turbo Pascal implementation of \var{Append}, where
a file needed to exist in order to be opened by
append.

Only text files can be opened in append mode.
}
{If the file can't be created, a run-time error will be generated.}
{\seep{Rewrite},\seep{Append}, \seep{Reset}}

\input{refex/ex3.tex}

\function{Arctan}{(X : Real)}{Real}
{\var{Arctan} returns the Arctangent of \var{X}, which can be any real type.
The resulting angle is in radial units.}{None}{\seef{Sin}, \seef{Cos}}

\input{refex/ex4.tex}

\procedure{Assign}{(Var F; Name : String)}
{\var{Assign} assigns a name to \var{F}, which can be any file type.
This call doesn't open the file, it just assigns a name to a file variable,
and marks the file as closed.}
{None.}
{\seep{Reset}, \seep{Rewrite}, \seep{Append}}

\input{refex/ex5.tex}

\procedure{Blockread}{(Var F : File; Var Buffer; Var Count : Longint [; var
Result : Longint])}
{\var{Blockread} reads \var{count} or less records from file \var{F}. The
result is placed in \var{Buffer}, which must contain enough room for
\var{Count} records. The function cannot read partial records. 

If \var{Result} is specified, it contains the number of records actually
read. If \var{Result} isn't specified, and less than \var{Count} records were
read, a run-time error is generated. This behavior can be controlled by the
\var{\{\$i\}} switch. }
{If \var{Result} isn't specified, then a run-time error is generated if less
than \var{count} records were read.}
{\seep{Blockwrite},\seep{Reset}, \seep{Assign}}

\input{refex/ex6.tex}

\procedure{Blockwrite}{(Var F : File; Var Buffer; Var Count : Longint)}
{\var{Blockread} writes \var{count} records from \var{buffer} to the file
 \var{F}. 
If the records couldn't be written to disk, a run-time error is generated.
This behavior can be controlled by the \var{\{\$i\}} switch. 
}
{A run-time error is generated if, for some reason, the records couldn't be
written to disk.}
{\seep{Blockread},\seep{Reset}, \seep{Assign}}

For the example, see \seep{Blockread}.

\procedure{Chdir}{(const S : string)}
{\var{Chdir} changes the working directory of the process to \var{S}.}
{If the directory \var{S} doesn't exist, a run-time error is generated.}
{\seep{Mkdir}, \seep{Rmdir}}

\input{refex/ex7.tex}

\function{Chr}{(X : byte)}{Char}
{\var{Chr} returns the character which has ASCII value \var{X}.}
{None.}
{\seef{Ord},\seep{Str}}

\input{refex/ex8.tex}

\procedure{Close}{(Var F : Anyfiletype)}
{\var{Close} flushes the buffer of the file \var{F} and closes \var{F}.
After a call to \var{Close}, data can no longer be read from or written to
\var{F}.

To reopen a file closed with \var{Close}, it isn't necessary to assign the
file again. A call to \seep{Reset} or \seep{Rewrite} is sufficient.}
{None.}{\seep{Assign}, \seep{Reset}, \seep{Rewrite}}

\input{refex/ex9.tex}

\function{Concat}{(S1,S2 [,S3, ... ,Sn])}{String}
{\var{Concat} concatenates the strings \var{S1},\var{S2} etc. to one long
string. The resulting string is truncated at a length of 255 bytes.

The same operation can be performed with the \var{+} operation.}
{None.}
{\seef{Copy}, \seep{Delete}, \seep{Insert}, \seef{Pos}, \seef{Length}}

\input{refex/ex10.tex}

\function{Copy}{(Const S : String;Index : Integer;Count : Byte)}{String}
{\var{Copy} returns a string which is a copy if the \var{Count} characters
in \var{S}, starting at position \var{Index}. If \var{Count} is larger than
the length of the string \var{S}, the result is truncated. 

If \var{Index} is larger than the length of the string \var{S}, then an
empty string is returned.}
{None.}
{\seep{Delete}, \seep{Insert}, \seef{Pos}}

\input{refex/ex11.tex}

\function{Cos}{(X : real)}{Real}
{\var{Cos} returns the cosine of \var{X}, where X is an angle, in radians.}
{None.}
{\seef{Arctan}, \seef{Sin}}

\input{refex/ex12.tex}

\Function{CSeg}{Word}
{\var{CSeg} returns the Code segment register. In \fpc, it returns always a
zero, since \fpc is a 32 bit compiler.}
{None.}
{\seef{DSeg}, \seef{Seg}, \seef{Ofs}, \seef{Ptr}}

\input{refex/ex13.tex}

\procedure{Dec}{(Var X : Any ordinal type[; Decrement : Longint])}
{\var{Dec} decreases the value of \var{X} with \var{Decrement}.
If \var{Decrement} isn't specified, then 1 is taken as a default.}
{A range check can occur, or an underflow error, if you try to decrease \var{X}
below its minimum value.}
{\seep{Inc}}

\input{refex/ex14.tex}

\procedure{Delete}{(var S : string;Index : Integer;Count : Integer)}
{\var{Delete} removes \var{Count} characters from string \var{S}, starting
at position \var{Index}. All remaining characters are shifted \var{Count} 
positions to the left, and the length of the string is adjusted.
}
{None.}
{\seef{Copy},\seef{Pos},\seep{Insert}}

\input{refex/ex15.tex}

\procedure{Dispose}{(P : pointer)}
{\var{Dispose} releases the memory allocated with a call to \seep{New}.
The pointer \var{P} must be typed. The released memory is returned to the
heap.}
{An error will occur if the pointer doesn't point to a location in the
heap.}
{\seep{New}, \seep{Getmem}, \seep{Freemem}}

\input{refex/ex16.tex}

\Function{DSeg}{Word}
{\var{DSeg} returns the data segment register. In \fpc, it returns always a
zero, since \fpc is a 32 bit compiler.}
{None.}
{\seef{CSeg}, \seef{Seg}, \seef{Ofs}, \seef{Ptr}}

\input{refex/ex17.tex}

\function{Eof}{[(F : Any file type)]}{Boolean}
{\var{Eof} returns \var{True} if the file-pointer has reached the end of the
file, or if the file is empty. In all other cases \var{Eof} returns
\var{False}.

If no file \var{F} is specified, standard input is assumed.}
{None.}
{\seef{Eoln}, \seep{Assign}, \seep{Reset}, \seep{Rewrite}}

\input{refex/ex18.tex}

\function{Eoln}{[(F : Text)]}{Boolean}
{\var{Eof} returns \var{True} if the file pointer has reached the end of a
line, which is demarcated by a line-feed character (ASCII value 10), or if
the end of the file is reached.
In all other cases \var{Eof} returns \var{False}.

If no file \var{F} is specified, standard input is assumed.
It can only be used on files of type \var{Text}.}
{None.}
{\seef{Eof}, \seep{Assign}, \seep{Reset}, \seep{Rewrite}}

\input{refex/ex19.tex}

\procedure{Erase}{(Var F : Any file type)}
{\var{Erase} removes an unopened file from disk. The file should be
assigned with \var{Assign}, but not opened with \var{Reset} or \var{Rewrite}}
{A run-time error will be generated if the specified file doesn't exist.}
{\seep{Assign}}

\input{refex/ex20.tex}

\procedure{Exit}{([Var X : return type )]}
{\var{Exit} exits the current subroutine, and returns control to the calling
routine. If invoked in the main program routine, exit stops the program.

The optional argument \var{X} allows to specify a return value, in the case
\var{Exit} is invoked in a function. The function result will then be
equal to \var{X}.}
{None.}
{\seep{Halt}}

\input{refex/ex21.tex}

\function{Exp}{(Var X : real)}{Real}
{\var{Exp} returns the exponent of \var{X}, i.e. the number \var{e} to the
power \var{X}.}
{None.}{\seef{Ln}}

\input{refex/ex22.tex}

\function{Filepos}{(Var F : Any file type)}{Longint}
{\var{Filepos} returns the current record position of the file-pointer in file
\var{F}. It cannot be invoked with a file of type \var{Text}.}
{None.}
{\seef{Filesize}}

\input{refex/ex23.tex}

\function{Filesize}{(Var F : Any file type)}{Longint}
{\var{Filepos} returns the total number of records in file \var{F}. 
It cannot be invoked with a file of type \var{Text}. (under \linux, this
also means that it cannot be invoked on pipes.)

If \var{F} is empty, 0 is returned.
}
{None.}
{\seef{Filepos}}

\input{refex/ex24.tex}

\procedure{Fillchar}{(Var X;Count : Longint;Value : char or byte);}
{\var{Fillchar} fills the memory starting at \var{X} with \var{Count} bytes
or characters with value equal to \var{Value}.
}
{No checking on the size of \var{X} is done.}
{\seep{Fillword}, \seep{Move}}

\input{refex/ex25.tex}

\procedure{Fillword}{(Var X;Count : Longint;Value : Word);}
{\var{Fillword} fills the memory starting at \var{X} with \var{Count} words
with value equal to \var{Value}.
}
{No checking on the size of \var{X} is done.}
{\seep{Fillword}, \seep{Move}}

\input{refex/ex76.tex}

\procedure{Flush}{(Var F : Text)}
{\var{Flush} empties the internal buffer of file \var{F} and writes the
contents to disk. The file is \textit{not} closed as a result of this call.}
{If the disk is full, a run-time error will be generated.}
{\seep{Close}}

\input{refex/ex26.tex}

\function{Frac}{(X : real)}{Real}
{\var{Frac} returns the non-integer part of \var{X}.}
{None.}
{\seef{Round}, \seef{Int}}

\input{refex/ex27.tex}

\procedure{Freemem}{(Var P : pointer; Count : longint)}
{\var{Freemem} releases the memory occupied by the pointer \var{P}, of size
\var{Count}, and returns it to the heap. \var{P} should point to the memory
allocated to a dynamical variable.}
{An error will occur when \var{P} doesn't point to the heap.}
{\seep{Getmem}, \seep{New}, \seep{Dispose}}

\input{refex/ex28.tex}

\procedure{Getdir}{(drivenr : byte;var dir : string)}
{\var{Getdir} returns in \var{dir} the current directory on the drive
\var{drivenr}, where {drivenr} is 1 for the first floppy drive, 3 for the
first hard disk etc. A value of 0 returns the directory on the current disk.

On \linux, \var{drivenr} is ignored, as there is only one directory tree.}
{An error is returned under \dos, if the drive requested isn't ready.}
{\seep{Chdir}}

\input{refex/ex29.tex}

\procedure{Getmem}{(var p : pointer;size : longint)}
{\var{Getmem} reserves \var{Size} bytes memory on the heap, and returns a
pointer to this memory in \var{p}. If no more memory is available, nil is
returned.}
{None.}
{\seep{Freemem}, \seep{Dispose}, \seep{New}}

For an example, see \seep{Freemem}.

\procedure{Halt}{[(Errnum : byte]}
{\var{Halt} stops program execution and returns control to the calling
program. The optional argument \var{Errnum} specifies an exit value. If
omitted, zero is returned.}
{None.}
{\seep{Exit}}

\input{refex/ex30.tex}

\function{Hi}{(X : Ordinal type)}{Word or byte}
{\var{Hi} returns the high byte or word from \var{X}, depending on the size
of X. If the size of X is 4, then the high word is returned. If the size is
2 then the high byte is retuned. 
\var{hi} cannot be invoked on types of size 1, such as byte or char.}
{None}
{\seef{Lo}}

\input{refex/ex31.tex}

\procedure{Inc}{(Var X : Any ordinal type[; Increment : Longint])}
{\var{Inc} increases the value of \var{X} with \var{Increment}.
If \var{Increment} isn't specified, then 1 is taken as a default.}
{A range check can occur, or an overflow error, if you try to increase \var{X}
over its maximum value.}
{\seep{Dec}}

\input{refex/ex32.tex}

\procedure{Insert}{(Var Source : String;var S : String;Index : integer)}
{\var{Insert} inserts string \var{S} in string \var{Source}, at position
\var{Index}, shifting all characters after \var{Index} to the right. The
resulting string is truncated at 255 characters, if needed.}
{None.}
{\seep{Delete}, \seef{Copy}, \seef{Pos}}

\input{refex/ex33.tex}

\function{Int}{(X : real)}{Real}
{\var{Int} returns the integer part of any real \var{X}, as a real.}
{None.}
{\seef{Frac}, \seef{Round}}

\input{refex/ex34.tex}

\Function{IOresult}{Word}
{IOresult contains the result of any input/output call, when the
\var{\{\$i-\}} compiler directive is active, and IO checking is disabled. When the
flag is read, it is reset to zero.

If \var{IOresult} is zero, the operation completed successfully. If
non-zero, an error occurred. The following errors can occur:

\dos errors :

\begin{description}
\item [2\ ] File not found.
\item [3\ ] Path not found.
\item [4\ ] Too many open files.
\item [5\ ] Access denied.
\item [6\ ] Invalid file handle.
\item [12\ ] Invalid file-access mode.
\item [15\ ] Invalid disk number.
\item [16\ ] Cannot remove current directory.
\item [17\ ] Cannot rename across volumes.
\end{description}

I/O errors :

\begin{description}
\item [100\ ] Error when reading from disk.
\item [101\ ] Error when writing to disk.
\item [102\ ] File not assigned.
\item [103\ ] File not open.
\item [104\ ] File not opened for input.
\item [105\ ] File not opened for output.
\item [106\ ] Invalid number.
\end{description}

Fatal errors :

\begin{description}
\item [150\ ] Disk is write protected.
\item [151\ ] Unknown device.
\item [152\ ] Drive not ready.
\item [153\ ] Unknown command.
\item [154\ ] CRC check failed.
\item [155\ ] Invalid drive specified..
\item [156\ ] Seek error on disk.
\item [157\ ] Invalid media type.
\item [158\ ] Sector not found.
\item [159\ ] Printer out of paper.
\item [160\ ] Error when writing to device.
\item [161\ ] Error when reading from device.
\item [162\ ] Hardware failure.
\end{description}
}
{None.}
{All I/O functions.}

\input{refex/ex35.tex}

\function{Length}{(S : String)}{Byte}
{\var{Length} returns the length of the string \var{S},
which is limited to 255. If the strings \var{S} is empty, 0 is returned.

{\em Note:} The length of the string \var{S} is stored in \var{S[0]}.
}
{None.}
{\seef{Pos}}

\input{refex/ex36.tex}

\function{Ln}{(X : real)}{Real}
{
\var{Ln} returns the natural logarithm of the real parameter \var{X}.
\var{X} must be positive.
}
{An run-time error will occur when \var{X} is negative.}
{\seef{Exp}}

\input{refex/ex37.tex}

\function{Lo}{(O : Word or Longint)}{Byte or Word}
{\var{Lo} returns the low byte of its argument if this is of type
\var{Integer} or
\var{Word}. It returns the low word of its argument if this is of type 
\var{Longint} or \var{Cardinal}.}
{None.}
{\seef{Ord}, \seef{Chr}}

\input{refex/ex38.tex}

\function{Lowercase}{(C : Char or String)}{Char or String}
{\var{Lowercase} returns the lowercase version of its argument \var{C}.
If its argument is a string, then the complete string is converted to
lowercase. The type of the returned value is the same as the type of the
argument.}
{None.}
{\seef{Upcase}}

\input{refex/ex73.tex}

\procedure{Mark}{(Var P : Pointer)}
{\var{Mark} copies the current heap-pointer to \var{P}.}
{None.}
{\seep{Getmem}, \seep{Freemem}, \seep{New}, \seep{Dispose}, \seef{Maxavail}}

\input{refex/ex39.tex}

\Function{Maxavail}{Longint}
{\var{Maxavail} returns the size, in bytes, of the biggest free memory block in
the heap.

{\em Remark:} The heap grows dynamically if more memory is needed than is
available.}
{None.}
{\seep{Release}, \seef{Memavail},\seep{Freemem}, \seep{Getmem}}

\input{refex/ex40.tex}

\Function{Memavail}{Longint}
{\var{Memavail} returns the size, in bytes, of the free heap memory.

{\em Remark:} The heap grows dynamically if more memory is needed than is
available.}
{None.}
{\seef{Maxavail},\seep{Freemem}, \seep{Getmem}}

\input{refex/ex41.tex}

\procedure{Mkdir}{(const S : string)}
{\var{Chdir} creates a new  directory \var{S}.}
{If a parent-directory of directory \var{S} doesn't exist, a run-time error is generated.}
{\seep{Chdir}, \seep{Rmdir}}

For an example, see \seep{Rmdir}.

\procedure{Move}{(var Source,Dest;Count : longint)}
{\var{Move} moves \var{Count} bytes from \var{Source} to \var{Dest}.}
{If either \var{Dest} or \var{Source} is outside the accessible memory for
the process, then a run-time error will be generated. With older versions of
the compiler, a segmentation-fault will occur. }
{\seep{Fillword}, \seep{Fillchar}}

\input{refex/ex42.tex}


\procedure{New}{(Var P : Pointer[, Constructor])}
{\var{New} allocates a new instance of the type pointed to by \var{P}, and
puts the address in \var{P}. 

If P is an object, then it is possible to
specify the name of the constructor with which the instance will be created.}
{If not enough memory is available, \var{Nil} will be returned.}
{\seep{Dispose}, \seep{Freemem}, \seep{Getmem}, \seef{Memavail},
\seef{Maxavail}}

For an example, see \seep{Dispose}.

\function{Odd}{(X : longint)}{Boolean}
{\var{Odd} returns \var{True} if \var{X} is odd, or \var{False} otherwise.}
{None.}
{\seef{Abs}, \seef{Ord}}


\input{refex/ex43.tex}

\function{Ofs}{Var X}{Longint}
{\var{Ofs} returns the offset of the address of a variable. 

This function is only supported for compatibility. In \fpc, it 
returns always the complete address of the variable, since \fpc is a 32 bit 
compiler.
}
{None.}
{\seef{DSeg}, \seef{CSeg}, \seef{Seg}, \seef{Ptr}}


\input{refex/ex44.tex}


\function{Ord}{(X : Ordinal type)}{Byte}
{\var{Ord} returns the Ordinal value of a ordinal-type variable \var{X}.}
{None.}
{\seef{Chr}}


\input{refex/ex45.tex}

\Function{Paramcount}{Longint}
{\var{Paramcount} returns the number of command-line arguments. If no
arguments were given to the running program, \var{0} is returned.
}
{None.}
{\seef{Paramstr}}


\input{refex/ex46.tex}

\function{Paramstr}{(L : Longint)}{String}
{\var{Paramstr} returns the \var{L}-th command-line argument. \var{L} must
be between \var{0} and \var{Paramcount}, these values included.
The zeroth argument is the name with which the program was started.
}
{Under Linux, command-line arguments may be longer than 255 characters. In
that case, the string is truncated. If you want to access the complete
string, you must use the \var{argv} pointer to access the real values of the
command-line parameters.}
{\seef{Paramcount}}

For an example, see \seef{Paramcount}.

\Function{Pi}{Real}
{\var{Pi} returns the value of Pi (3.1415926535897932385).}
{None.}
{\seef{Cos}, \seef{Sin}}


\input{refex/ex47.tex}

\function{Pos}{(Const Substr : String;Const S : String)}{Byte}
{\var{Pos} returns the index of \var{Substr} in \var{S}, if \var{S} contains
\var{Substr}. In case \var{Substr} isn't found, \var{0} is returned.

The search is case-sensitive.
}
{None}
{\seef{Length}, \seef{Copy}, \seep{Delete}, \seep{Insert}}


\input{refex/ex48.tex}

\function{Ptr}{(Sel,Off : Longint)}{Pointer}
{\var{Ptr} returns a pointer, pointing to the address specified by
segment{Sel} and offset \var{Off}.

{\em Remark 1:} In the 32-bit flat-memory model supported by \fpc, this
function is obsolete.}

{\em Remark 2:} The returned address is simply the offset. If you recompile
the RTL with \var{-dDoMapping} defined, then the compiler returns the
following : \verb|ptr:=pointer($e0000000+sel shl 4+off)| under \dos, or
\verb|ptr:=pointer(sel shl 4+off)| on other OSes.
{None.}
{\seef{Addr}}
\input{refex/ex59.tex}

\function{Random}{[(L : longint)]}{Longint or Real}
{\var{Random} returns a random number larger or equal to \var{0} and
strictly less than \var{L}.

If the argument \var{L} is omitted, a real number between 0 and 1 is returned.
(0 included, 1 excluded)}
{None.}
{\seep{Randomize}}


\input{refex/ex49.tex}

\Procedure{Randomize}
{\var{Randomize} initializes the random number generator of \fpc, by giving
a value to \var{Randseed}, calculated with the system clock.
}
{None.}
{\seef{Random}}

For an example, see \seef{Random}.

\procedure{Read}{([Var F : Any file type], V1 [, V2, ... , Vn])}
{\var{Read} reads one or more values from a file \var{F}, and stores the
result in \var{V1}, \var{V2}, etc.; If no file \var{F} is specified, then
standard input is read.

If \var{F} is of type \var{Text}, then the variables \var{V1, V2} etc. must be
of type \var{Char}, \var{Integer}, \var{Real} or \var{String}.

If \var{F} is a typed file, then each of the variables must be of the type
specified in the declaration of \var{F}. Untyped files are not allowed as an
argument.}
{If no data is available, a run-time error is generated. This behavior can

be controlled with the \var{\{\$i\}} compiler switch.}
{\seep{Readln}, \seep{Blockread}, \seep{Write}, \seep{Blockwrite}}


\input{refex/ex50.tex}

\procedure{Readln}{[Var F : Text], V1 [, V2, ... , Vn])}
{\var{Read} reads one or more values from a file \var{F}, and stores the
result in \var{V1}, \var{V2}, etc. After that it goes to the next line in
the file (defined by the \var{LineFeed (\#10)} character). 
If no file \var{F} is specified, then standard input is read.

The variables \var{V1, V2} etc. must be of type \var{Char}, \var{Integer}, 
\var{Real}, \var{String} or \var{PChar}.
}
{If no data is available, a run-time error is generated. This behavior can
be controlled with the \var{\{\$i\}} compiler switch.}
{\seep{Read}, \seep{Blockread}, \seep{Write}, \seep{Blockwrite}}

For an example, see \seep{Read}.

\procedure{Release}{(Var P : pointer)}
{\var{Release} sets the top of the Heap to the location pointed to by
\var{P}. All memory at a location higher than \var{P} is marked empty.}
{A run-time error will be generated if \var{P} points to memory outside the
heap.}
{\seep{Mark}, \seef{Memavail}, \seef{Maxavail}, \seep{Getmem}, \seep{Freemem}
\seep{New}, \seep{Dispose}}

For an example, see \seep{Mark}.

\procedure{Rename}{(Var F : Any Filetype; Const S : String)}
{\var{Rename} changes the name of the assigned file \var{F} to \var{S}.
\var{F}
must be assigned, but not opened.}
{A run-time error will be generated if \var{F} isn't assigned, 
or doesn't exist.}
{\seep{Erase}}

\input{refex/ex77.tex}

\procedure{Reset}{(Var F : Any File Type[; L : longint])}
{\var{Reset} opens a file \var{F} for reading. \var{F} can be any file type.
If \var{F} is an untyped or typed file, then it is opened for reading and 
writing. If \var{F} is an untyped file, the record size can be specified in 
the optional parameter \var{L}. Default a value of 128 is used.}
{If the file cannot be opened for reading, then a run-time error is
generated. This behavior can be changed by the \var{\{\$i\} } compiler switch.}
{\seep{Rewrite}, \seep{Assign}, \seep{Close}}


\input{refex/ex51.tex}

\procedure{Rewrite}{(Var F : Any File Type[; L : longint])}
{\var{Rewrite} opens a file \var{F} for writing. \var{F} can be any file type.
If \var{F} is an untyped or typed file, then it is opened for reading and 
writing. If \var{F} is an untyped file, the record size can be specified in 
the optional parameter \var{L}. Default a value of 128 is used.

if \var{Rewrite} finds a file with the same name as \var{F}, this file is
truncated to length \var{0}. If it doesn't find such a file, a new file is 
created.
}
{If the file cannot be opened for writing, then a run-time error is
generated. This behavior can be changed by the \var{\{\$i\} } compiler switch.}
{\seep{Reset}, \seep{Assign}, \seep{Close}}


\input{refex/ex52.tex}

\procedure{Rmdir}{(const S : string)}
{\var{Rmdir} removes the  directory \var{S}.}
{If \var{S} doesn't exist, or isn't empty, a run-time error is generated.
}
{\seep{Chdir}, \seep{Rmdir}}

\input{refex/ex53.tex}

\function{Round}{(X : real)}{Longint}
{\var{Round} rounds \var{X} to the closest integer, which may be bigger or
smaller than \var{X}.}
{None.}
{\seef{Frac}, \seef{Int}, \seef{Trunc}}

\input{refex/ex54.tex}

\procedure{Runerror}{(ErrorCode : Word)}
{\var{Runerror} stops the execution of the program, and generates a
run-time error \var{ErrorCode}.}
{None.}
{\seep{Exit}, \seep{Halt}}

\input{refex/ex55.tex}

\procedure{Seek}{(Var F; Count : Longint)}
{\var{Seek} sets the file-pointer for file \var{F} to record Nr. \var{Count}.
The first record in a file has \var{Count=0}. F can be any file type, except
\var{Text}. If \var{F} is an untyped file, with no specified record size, 128
is assumed.}
{A run-time error is generated if \var{Count} points to a position outside
the file, or the file isn't opened.}
{\seef{Eof}, \seef{SeekEof}, \seef{SeekEoln}}

\input{refex/ex56.tex}

\function{SeekEof}{[(Var F : text)]}{Boolean}
{\var{SeekEof} returns \var{True} is the file-pointer is at the end of the
file. It ignores all whitespace.

Calling this function has the effect that the file-position is advanced
until the first non-whitespace character or the end-of-file marker is
reached.
If the end-of-file marker is reached, \var{True} is returned. Otherwise,
False is returned.

If the parameter \var{F} is omitted, standard \var{Input} is assumed.
}
{A run-time error is generated if the file \var{F} isn't opened.}
{\seef{Eof}, \seef{SeekEoln}, \seep{Seek}}

\input{refex/ex57.tex}

\function{SeekEoln}{[(Var F : text)]}{Boolean}
{\var{SeekEoln} returns \var{True} is the file-pointer is at the end of the
current line. It ignores all whitespace.

Calling this function has the effect that the file-position is advanced
until the first non-whitespace character or the end-of-line marker is
reached.
If the end-of-line marker is reached, \var{True} is returned. Otherwise,
False is returned.

The end-of-line marker is defined as \var{\#10}, the LineFeed character.

If the parameter \var{F} is omitted, standard \var{Input} is assumed.}
{A run-time error is generated if the file \var{F} isn't opened.}
{\seef{Eof}, \seef{SeekEof}, \seep{Seek}}

\input{refex/ex58.tex}

\function{Seg}{Var X}{Longint}
{\var{Seg} returns the segment of the address of a variable. 

This function is only supported for compatibility. In \fpc, it 
returns always 0, since \fpc is a 32 bit compiler, segments have no meaning.
}
{None.}
{\seef{DSeg}, \seef{CSeg}, \seef{Ofs}, \seef{Ptr}}

\input{refex/ex60.tex}


\procedure{SetTextBuf}{(Var f : Text; Var Buf[; Size : Word])}
{\var{SetTextBuf} assigns an I/O buffer to a text file. The new buffer is
located at \var{Buf} and is \var{Size} bytes long. If \var{Size} is omitted,
then \var{SizeOf(Buf)} is assumed.

The standard buffer of any text file is 128 bytes long. For heavy I/0
operations this may prove too slow. The \var{SetTextBuf} procedure allows
you to set a bigger buffer for your application, thus reducing the number of
system calls, and thus reducing the load on the system resources.

The maximum size of the newly assigned buffer is 65355 bytes.

{\em Remark 1:} Never assign a new buffer to an opened file. You can assign a
new buffer immediately after a call to \seep{Rewrite}, \seep{Reset} or
\var{Append}, but not after you read from/wrote to the file. This may cause
loss of data. If you still want to assign a new buffer after read/write
operations have been performed, flush the file first. This will ensure that
the current buffer is emptied.

{\em Remark 2:} Take care that the buffer you assign is always valid. If you
assign a local variable as a buffer, then after your program exits the local
program block, the buffer will no longer be valid, and stack problems may
occur.
}
{No checking on \var{Size} is done.}
{\seep{Assign}, \seep{Reset}, \seep{Rewrite}, \seep{Append}}

\input{refex/ex61.tex}

\function{Sin}{(X : real)}{Real}
{\var{Sin} returns the sine of its argument \var{X}, where \var{X} is an
angle in radians.}
{None.}
{\seef{Cos}, \seef{Pi}, \seef{Exp}}

\input{refex/ex62.tex}

\function{SizeOf}{(X : Any Type)}{Longint}
{\var{SizeOf} Returns the size, in bytes, of any variable or type-identifier.

 {\em Remark:} this isn't really a RTL function. Its result is calculated at
compile-time, and hard-coded in your executable.}
{None.}
{\seef{Addr}}

\input{refex/ex63.tex}

\Function{Sptr}{Pointer}
{\var{Sptr} returns the current stack pointer.
}{None.}{}

\input{refex/ex64.tex}

\function{Sqr}{(X : Real)}{Real}
{\var{Sqr} returns the square of its argument \var{X}.}
{None.}
{\seef{Sqrt}, \seef{Ln}, \seef{Exp}}

\input{refex/ex65.tex}

\function{Sqrt}{(X : Real)}{Real}
{\var{Sqrt} returns the square root of its argument \var{X}, which must be
positive.}
{If \var{X} is negative, then a run-time error is generated.}
{\seef{Sqr}, \seef{Ln}, \seef{Exp}}

\input{refex/ex66.tex}

\Function{SSeg}{Longint}
{ \var{SSeg} returns the Stack Segment. This function is only 
 supported for compatibolity reasons, as \var{Sptr} returns the
correct contents of the stackpointer.}
{None.}{\seef{Sptr}}

\input{refex/ex67.tex}


\procedure{Str}{(Var X[:NumPlaces[:Decimals]]; Var S : String)}
{\var{Str} returns a string which represents the value of X. X can be any
numerical type.

The optional \var{NumPLaces} and \var{Decimals} specifiers control the
formatting of the string.}
{None.}
{\seep{Val}}

\input{refex/ex68.tex}

\function{Swap}{(X)}{Type of X}
{\var{Swap} swaps the high and low order bytes of \var{X} if \var{X} is of
type \var{Word} or \var{Integer}, or swaps the high and low order words of
\var{X} if \var{X} is of type \var{Longint} or \var{Cardinal}.

The return type is the type of \var{X}}
{None.}{\seef{Lo}, \seef{Hi}}

\input{refex/ex69.tex}

\function{Trunc}{(X : real)}{Longint}
{\var{Trunc} returns the integer part of \var{X}, 
which is always smaller than (or equal to)  \var{X}.}
{None.}
{\seef{Frac}, \seef{Int}, \seef{Trunc}}

\input{refex/ex70.tex}

\procedure{Truncate}{(Var F : file)}
{\var{Truncate} truncates the (opened) file \var{F} at the current file
position.
}{Errors are reported by IOresult.}{\seep{Append}, \seef{Filepos},
\seep{Seek}}

\input{refex/ex71.tex}

\function{Upcase}{(C : Char or string)}{Char or String}
{\var{Upcase} returns the uppercase version of its argument \var{C}.
If its argument is a string, then the complete string is converted to 
uppercase. The type of the returned value is the same as the type of the
argument.}
{None.}
{\seef{Lowercase}}

\input{refex/ex72.tex}

\procedure{Val}{(const S : string;var V;var Code : word)}
{\var{Val} converts the value represented in the string \var{S} to a numerical
value, and stores this value in the variable \var{V}, which 
can be of type \var{Longint}, \var{real} and \var{Byte}.

If the conversion isn't succesfull, then the parameter \var{Code} contains
the index of the character in \var{S} which prevented the conversion.

The string \var{S} isn't allow to contain spaces.}
{If the conversion doesn't succeed, the value of \var{Code} indicates the
position where the conversion went wrong.}
{\seep{Str}}

\input{refex/ex74.tex}

\procedure{Write}{([Var F : Any filetype;] V1 [; V2; ... , Vn)]}
{\var{Write} writes the contents of the variables \var{V1}, \var{V2} etc. to
the file \var{F}. \var{F} can be a typed file, or a \var{Text} file.

If \var{F} is a typed file, then the variables \var{V1}, \var{V2} etc. must
be of the same type as the type in the declaration of \var{F}. Untyped files
are not allowed.

If the parameter \var{F} is omitted, standard output is assumed. 

If \var{F} is of type \var{Text}, then the necessary conversions are done
such that the output of the variables is in human-readable format.
This conversion is done for all numerical types. Strings are printed exactly
as they are in memory, as well as \var{PChar} types. 
The format of the numerical conversions can be influenced through
the following modifiers:

\var{ OutputVariable : NumChars [: Decimals ]  }

This will print the value of \var{OutputVariable} with a minimum of
\var{NumChars} characters, from which \var{Decimals} are reserved for the
decimals. If the number cannot be represented with \var{NumChars} characters,
\var{NumChars} will be increased, until the representation fits. If the
representation requires less than \var{NumChars} characters then the output
is filled up with spaces, to the left of the generated string, thus
resulting in a right-aligned representation.

If no formatting is specified, then the number is written using its natural
length, with a space in front of it if it's positive, and a minus sign if
it's negative.

Real numbers are, by default, written in scientific notation.
}
{If an error occurs, a run-time error is generated. This behavior can be
controlled with the \var{\{\$i\}} switch. }
{\seep{Writeln}, \seep{Read}, \seep{Readln}, \seep{Blockwrite} }

\procedure{Writeln}{[([Var F : Text;] [V1 [; V2; ... , Vn)]]}
{\var{Writeln} does the same as \seep{Write} for text files, and emits a
Carriage Return - LineFeed character pair after that.

If the parameter \var{F} is omitted, standard output is assumed. 

If no variables are specified, a Carriage Return - LineFeed character pair
is emitted, resulting in a new line in the file \var{F}.

{\em Remark:} Under \linux, the Carriage Return character is omitted, as
customary in Unix environments.
}
{If an error occurs, a run-time error is generated. This behavior can be
controlled with the \var{\{\$i\}} switch. }
{\seep{Write}, \seep{Read}, \seep{Readln}, \seep{Blockwrite}}

\input{refex/ex75.tex}

%
% The index.
% 
\printindex
\end{document}
