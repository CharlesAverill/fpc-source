%
%   $Id$
%   This file is part of the FPC documentation.
%   Copyright (C) 1997, by Michael Van Canneyt
%
%   The FPC documentation is free text; you can redistribute it and/or
%   modify it under the terms of the GNU Library General Public License as
%   published by the Free Software Foundation; either version 2 of the
%   License, or (at your option) any later version.
%
%   The FPC Documentation is distributed in the hope that it will be useful,
%   but WITHOUT ANY WARRANTY; without even the implied warranty of
%   MERCHANTABILITY or FITNESS FOR A PARTICULAR PURPOSE.  See the GNU
%   Library General Public License for more details.
%
%   You should have received a copy of the GNU Library General Public
%   License along with the FPC documentation; see the file COPYING.LIB.  If not,
%   write to the Free Software Foundation, Inc., 59 Temple Place - Suite 330,
%   Boston, MA 02111-1307, USA. 
%
\chapter{The GO32 unit.}
This chapter describes the \var{GO32} unit for Free Pascal under \dos.
The unit was first written for \dos by Florian kl\"ampfl. 

This chapter is divided in three sections. 
\begin{itemize}
\item The first section is an introduction to the GO32 unit.
\item The second section lists the pre-defined constants, types and variables. 
\item The third section describes the functions which appear in the
interface part of the GO32 unit.
\end{itemize}
Many function descriptions were made by Thomas Schatzl\footnote{
E-Mail: \var{tom\_at\_work@geocities.com}}, for which my thanks.

\section{Introduction}

The GO32 provides an interface to the \dos extender, \dos memory access,
I/O ports and processor access. Some of this functions work with all modes
of the extender, other only with DPMI.

\subsection{DOS memory access}

The \dos memory access is done by the \var{dosmem}
functions and it's \textbf {strongly recommended } to use these functions. 

Example:
\begin{verbatim} 
function shift_state : byte;
  
  begin
     { $40:$17 contains the current contents of the shift, 
       alt and strg keys} 
     dosmemget($40,$17,shift_state,1);
  end;
\end{verbatim}
\subsection{I/O port access}

The I/O port access is done by the \var{inport} and \var{outport} functions. 
It's not necessary to use these functions but it makes life easier. \fpc 
\textbf {doesn't} support the \var{PORT} array, as under Turbo Pascal. 

\subsection{Processor access}
There are some functions to access the segment registers, which makes
your work easier. 

\subsection{Interrupt redirection}

The \file{GO32} unit helps you to redirect interrupts. \var{SetIntVec}
and \var{GetIntVec} don't work with \fpc. This is now done via the functions
\var{set\_pm\_interrupt} and \var{get\_pm\_interrupt}.

As an example we show how to redirect the interrupt \var{8h}.
\begin{verbatim} 
{ the unit CRT _is_ needed because the program doesn't get 
  an interrupt while DOS is active }
uses
   go32,crt;

var
   timer : longint;
   ds : word;

procedure s; { interrupt;}
             { comes with versions > 0.9.2 of FPCPascal }

  begin
     asm
        { save used registers }
        pushl %eax
        pushw %ds
        { load ds }
        { prefix for cs }
        .byte 0x2e
        movw ALAB,%ax
        movw %ax,%ds
     end;
     inc(timer);
     asm
        { restore processor state }
        popw %ds
        popl %eax
        leave
        { call old interrupt }
        ljmp %cs:OLDINT
        iret
        { we need some data in the code segment, }
        {  since only CS is set in the }
        { entry point of the procedure }
     ALAB:
        .word 0
        { old vector as 48 bit pointer (16:32) }
     OLDINT:
        .long 0
        .word 0
     end;
  end;

  var
     oldint,myint : tseginfo; 
     i : longint;

  begin 
     timer:=0;
     { save old interrupt }
     get_pm_interrupt(8,oldint); 
     ds:=get_ds; 
     asm 
        { copy some data to the code segment }
        movw _DS,%ax
        movw %ax,ALAB
        movl _OLDINT,%eax
        movl %eax,OLDINT
        movw _OLDINT+4,%ax
        movw %ax,OLDINT+4
     end;
     { new handler }
     myint.segment:=get_cs;
     myint.offset:=@s;
     { install the handler }
     set_pm_interrupt(8,myint);
     { do something }
     for i:=1 to 10000 do
       writeln(timer);
     { install the old handler }
     set_pm_interrupt(8,oldint);
  end.
\end{verbatim}

\section{Types, Variables and Constants}
\subsection{Constants}
\begin{verbatim}
Const
rm_unknown = 0; 
rm_raw = 1;  { raw (without HIMEM) } 
rm_xms = 2;  { XMS (for example with HIMEM, without EMM386) } 
rm_vcpi = 3; { VCPI (for example HIMEM and EMM386) } 
rm_dpmi = 4; { DPMI (for example DOS box or 386Max) }
\end{verbatim}\label{co:rmmode}
These constants can be returned by the \htmlref{get\_run\_mode}{GetRunMode} function.
\subsection{Types}
\begin{verbatim}
type 
   tmeminfo = record 
      available_memory : longint;
      available_pages : longint;
      available_lockable_pages : longint;
      linear_space : longint;
      unlocked_pages : longint;
      available_physical_pages : longint;
      total_physical_pages : longint;
      free_linear_space : longint;
      max_pages_in_paging_file : longint;
      reserved : array[0..2] of longint;
   end;
\end{verbatim}\label{ty:tmeminfo}
Returns information about the memory allocation etc.

\textbf {NOTE:} The
value of a field is zero if the value is unknown, it's only guaranteed,
that \var{available\_memory} contains a valid value.
\begin{verbatim}
type
trealregs = record
  case integer of
   1: { 32-bit } (EDI, ESI, EBP, Res, EBX, EDX, ECX, EAX: longint;
                  Flags, ES, DS, FS, GS, IP, CS, SP, SS: word);
   2: { 16-bit } (DI, DI2, SI, SI2, BP, BP2, R1, R2: word;
                  BX, BX2, DX, DX2, CX, CX2, AX, AX2: word);
   3: { 8-bit }  (stuff: array[1..4] of longint;
                  BL, BH, BL2, BH2, DL, DH, DL2, DH2,
                  CL, CH, CL2, CH2, AL, AH, AL2, AH2: byte);
   4: { Compat } (RealEDI, RealESI, RealEBP, RealRES,
                  RealEBX, RealEDX, RealECX, RealEAX: longint;
                  RealFlags,
                  RealES, RealDS, RealFS, RealGS,
                  RealIP, RealCS, RealSP, RealSS: word);
end;

registers = trealregs;
\end{verbatim}\label{ty:trealregs}
This data structure is used to pass register values to an real mode
interrupt handler.
\begin{verbatim}
type
   tseginfo = record 
      offset : pointer;
      segment : word;
   end;
\end{verbatim}\label{ty:tseginfo}
This record is used to store a 48-bit pointer.
\subsection{Variables.}
\begin{verbatim}
var
   { puts count bytes from data to ptr(seg:ofs) of the DOS memory }
   dosmemput : procedure(seg : word;ofs : word;var data;count : longint);

   { gets count bytes from ptr(seg:ofs) of the DOS memory to data }
   dosmemget : procedure(seg : word;ofs : word;var data;count : longint);

   { moves count bytes from ptr(sseg:sofs) to ptr(dseg:dofs) }
   dosmemmove : procedure(sseg,sofs,dseg,dofs : word;count : longint);

   { fills count bytes beginning with ptr(seg:ofs) with c }
   dosmemfillchar : procedure(seg,ofs : word;count : longint;c : char);

   { fills count words beginning with ptr(seg:ofs) with w }
   { this function is especially used by the CRT unit.   }
   dosmemfillword : procedure(seg,ofs : word;count : longint;w : word);
\end{verbatim}
These procedure variables give you access to the \dos memory in each mode
of the \dos extender. It is strongly recommended to use these functions.

The procedural variables are assigned by the startup code of the \var{GO32}
unit to the correct procedures.
\begin{verbatim}
var
   dosmemselector : word;
\end{verbatim}
Selector to the \dos memory. The whole \dos memory is mapped to a
single segment. This function will only work in DPMI mode.

\section{Functions and Procedures}
\procedure {Disable}{}{
Clears the interrupt flag with CLD and disables the interrupts.
}
{None.}{\seep{Enable}}
\par {\bf NOTE: }This
function works only in DPMI mode\par 

\procedure{Enable}{}{
Sets the interrupt flag with STI and allows the processor to handle
interrupts.
}{None.}{\seep{Disable}}

\procedurel{get\_meminfo}{GetMeminfo}{(var meminfo : tmeminfo)}{
Returns the current state of memory allocation of the \dos extender.

\textbf{NOTE: }This procedure has nothing to do with the Pascal function 
\var{maxavail} and \var{memavail}.}{None.} 
{\htmlref{tmeminfo}{ty:tmeminfo}} 

\procedurel{get\_pm\_interrupt}{GetPmInterrupt}{(vector : byte;var intaddr : tseginfo)}
{
Returns the address of the current protected mode handler for the interrupt
\var{vector}.}{None.}
{\seep{SetPmInterrupt},\htmlref{tseginfo}{ty:tseginfo}}

\functionl{get\_run\_mode}{GetRunMode}{}{word}{
This function returns the mode which the extender is currently running. 
The function is mostly used to determine if DPMI is supported. 


}{None.}
{\htmlref{rm}{co:rmmode}}
Example :
\begin{verbatim}
uses
   go32;

begin
   if get_run_mode=rm_dpmi then
     writeln('DPMI available')
   else
     writeln('No DPMI available');
end.
\end{verbatim}

\functionl{get\_cs}{GetCs}{}{word}{Returns the value of the CS
register.}{None.}{\seef{GetDs},\seef{GetSs}}

\functionl{get\_ds}{GetDs}{}{word}{Returns the value of the DS
register.}{None.}{\seef{GetCs},\seef{GetSs}}

\functionl{get\_ss}{GetSs}{word}{Returns the value of the SS
register.}{None.}{\seef{GetDs},\seef{GetCs}}

\function{inportb}{(port : word)}{byte}{Reads a byte from the given I/O
port.}{None.}{\seep{outportb},\seef{inportw},\seef{inportl}}

\function{inportw}{(port : word)}{word}{Reads a word from the given I/O
port.}{None.}{\seep{outportw},\seef{inportb},\seef{inportl}}

\function{inportl}{(port : word)}{longint}{Reads a longint from the given I/O
port.}{None.}{\seep{outportl},\seef{inportw},\seef{inportb}}

\procedure{outportb}{(port : word;data : byte)}{Writes a byte to the
specified port.}{None.}{\seef{inportb},\seep{outportw},\seep{outportl}}

\procedure{outportw}{(port : word;data : word)}{Writes a word to the
specified port.}{None.}{\seef{inportw},\seep{outportb},\seep{outportl}}

\procedure{outportl}{(port : word;data : longint)}{Writes a longint to the
specified port.}{None.}{\seef{inportl},\seep{outportb},\seep{outportw}}

\procedure{realintr}{(intnr : word;var regs : trealregs)}{
\textbf {NOTE: }This procedure works only in DPMI mode.} 
{None.}{}

\procedurel{seg\_fillchar}{SegFillChar}{(seg : word;ofs : longint;count : longint;c :
char)}
{Fills a memory area specified by a 48 bit pointer with the given number
of chars.

\textbf {NOTE:} Be careful using this function in non-DPMI mode. 
}{None.}{\seep{SegFillWord}, \seep{SegMove}}

\procedurel {seg\_fillword}{SegFillWord}{(seg : word;ofs : longint;count : longint;w :
word)}{Fills a memory area specified by a 48 bit pointer with the given number
of words.

\textbf {NOTE:} Be careful using this function in non-DPMI mode. 
}{None.}{\seep{SegFillChar}, \seep{SegMove}}

\procedurel{seg\_move}{SegMove}{(sseg : word;source : longint;dseg : word;dest : longint;count :
longint)}
{This procedure copies data when the source and destination are specified
by 48 bit pointers. For example, this function is used by the DPMI version
of \var{dosmemget} and \var{dosmemput}. }{The procedure checks only
for overlapping if source selector equals the destination selector. 
Be also careful using this function in non-DPMI
mode.}{\seep{SegFillWord},\seep{SegFillChar}}

\procedurel{set\_pm\_interrupt}{SetPmInterrupt}{(vector : byte;const intaddr : tseginfo)}
{Sets a new protected mode handler for the interrupt \var{vector}.}
{None.}{\seep{GetPmInterrupt}, \htmlref{tseginfo}{ty:tseginfo}} 

\functionl{Allocate\_Ldt\_Descriptors}{AllocateLdtDescriptors}{(Count :
word)}{word}
{\var{Allocate\_ldt\_descriptors} allocates \var{Count} descriptors in the 
Local Descriptor Table (LDT).
The descriptors allocated must be initialized by the application with
other function calls.

The function returns a base descriptor with a limit and size value set to 
zero.

{\em Notes:}
\begin{itemize}
\item Only works with real DPMI.
\item If more than one descriptor was requested, the function returns a base
selector referencing the first of a contiguous array of descriptors. The
selector values for subsequent descriptors in the array can be
calculated by adding the value returned by
\var{get\_next\_selector\_increment\_value.}
\end{itemize}
}{None.}{
\seep{SetSegmentBaseAddress},
\seep{SetSegmentLimit},
\seef{GetLinearAddr},
\seep{FreeLdtDescriptor},
\seef{GetNextSelectorIncrementValue}
}
\begin{FPCList}
\item[Example]
\begin{verbatim}
uses go32;

var VGAsel : word;
    r : trealregs;

begin
  {...}
  r.realeax := $13; realintr($10, r); 
  { set VGA mode }
  {...}
  VGAsel := allocate_ldt_descriptors(1); 
  { allocate one descriptor to the VGA }
  set_segment_base_address(VGAsel, 
                           get_linear_address($A0000,
                           $FFFF)); 
  { set the base address to the VGA }
  set_segment_limit(VGAsel, $FFFF); 
  { set the limit of the descriptor }
  {...}
  seg_fillchar(VGAsel, 100*320+6, 1, 15); 
  { put a pixel at (6/100) in color 15 }
  readln;
  {...}
  free_ldt_descriptor(sel);
  r.realeax := $3; realintr($10, r); 
  { set textmode again }
  {...}
end.
\end{verbatim}
\end{FPCList}

\procedurel{Free\_Ldt\_Descriptor}{FreeLdtDescriptor}{(Sel : word)}
{
\var{Free\_Ldt\_Descriptor} frees a previously allocated selector
with descriptor \var{Sel}

{\em Notes:}
\begin{itemize}
\item Only works with real DPMI.
\item After this call this selector is invalid and must not be used for any
memory operations anymore.
\item Each descriptor allocated with \var{allocate\_ltd\_descriptor} must be 
freed
individually with this function, even if it was previously allocated as
a part of a contiguous array of descriptors.
\end{itemize}
}
{None.}
{\seef{AllocateLdtDescriptors}}

For an example, see \seef{AllocateLdtDescriptors}.

\functionl{Segment\_To\_Descriptor}{SegmentToDescriptor}{(Seg : Word)}{Word}
{\var{Segment\_To\_Descriptor} Maps a real mode segment (paragraph) address 
(in \var{Seg}) onto an descriptor that can be used by a protected mode 
program to access  the same memory.

The function returns a selector to the DOS real-mode segment.

{\em Notes:}
\begin{itemize}
\item Only works with real DPMI.
\item The descriptors limit will be set to 64KB.
\item multiple calls to this function with the same segment address will
  return the same selector.
\item Descriptors created by this function can never be modified or freed.
For this reason this function shouldn't be used too often. Programs
which need to examine various real mode addresses using the same
selector should use the function \var{allocate\_ldt\_descriptors} and change
the base address as necessary.
\end{itemize}
}
{None.}
{}
\begin{FPCList}
\item[Example]
\begin{verbatim}
uses go32;

var r : trealregs;
    VGAsel : word;

begin
     r.realeax := $13; realintr($10, r); 
     { set VGA mode 13h }
     VGASel := segment_to_descriptor($A000);
     {...}
     seg_fillchar(VGAsel, 100*320+6, 1, 15); 
     { put a pixel at (6/100) in color 15 }
     readln;
     {...} 
     r.realeax := $3; realintr($10, r);
end.
\end{verbatim}
\end{FPCList}


\Functionl{Get\_Next\_Selector\_Increment\_Value}
{GetNextSelectorIncrementValue}{word}
{\var{Get\_Next\_Selector\_Increment\_Value} returns the selector increment 
value when allocating multiple subsequent descriptors

The function \var{allocate\_ldt\_descriptors} can allocate an array of
contiguous descriptors, but only return the selector for the first. The
value returned by this function can be used to calculate the selectors
for subsequent descriptors in the array.

{\em Notes:}
\begin{itemize}
\item Only works under real DPMI.
\item the increment is always a power of 2.
\end{itemize}
}
{None.}
{\seef{AllocateLdtDescriptors}}

\functionl{Get\_Segment\_Base\_Address}{GetSegmentBaseAddress}{(Sel: word)}{Longint}
{
\var{Get\_Segment\_Base\_Address} returns the 32-bit linear base address 
from the LDT descriptor for the specified segment (\var{Sel}).


{\em Notes:}
\begin{itemize}
\item Only works under real DPMI.
\end{itemize}
}
{None.}
{\seep{SetSegmentBaseAddress}}

\begin{FPCList}
\item[Example:]
\begin{verbatim}
uses go32;

begin
     Writeln(get_segment_base_address(get_ds));
end.
\end{verbatim}
\end{FPCList}

\procedurel{Set\_Segment\_Base\_Address}{SetSegmentBaseAddress}
{(var Des : word;Sel : longint)}
{\var{{Set\_Segment\_Base\_Address}} sets the 32-bit linear base address 
of the descriptor \var{Des} for the specified selector \var{Sel}.

{\em Notes:}
\begin{itemize}
\item only works under real DPMI
\end{itemize}
}
{None.}
{ \seef{AllocateLdtDescriptors}, \seep{SetSegmentLimit}}

For an example, see \seef{AllocateLdtDescriptors}.

\procedurel{Set\_Segment\_Limit}{SetSegmentLimit}{(Des : word;Len : longint)}
{\var{Set\_Segment\_Limit} sets the limit of the descriptor \var{Des} 
to the specified length \var{Len}

{\em Notes:}
\begin{itemize}
\item Only works under real DPMI.
\item The new limit is the byte length of the segment-1.
\item Segment limits bigger than or equal to 1MB must be page aligned, they
must have the lower 12 bits set.
\end{itemize}
}
{None}
{ \seep{SetSegmentBaseAddress}, \seef{AllocateLdtDescriptors}}

For an example, see \seef{AllocateLdtDescriptors}.

\functionl{Create\_Code\_Segment\_Alias\_Descriptor}
{CreateCodeSegmentAliasDescriptor}{(Des : Word)}{Word}
{\var{Create\_Code\_Segment\_Alias\_Descriptor}
Creates a new descriptor that has the same base and limit as the
specified descriptor.  In effect, the function returns a copy of the
descriptor.

{\em Notes:}
\begin{itemize}
\item Only works under real DPMI.
\item The descriptor alias returned by this function will not track changes
to the original descriptor. In other words, if an alias is created with
this function, and the base or limit of the original segment is then
changed, the two descriptors will no longer map the same memory.
\end{itemize}
}
{None.}
{}
\begin{FPCList}
\item[Example]
\begin{verbatim}
uses go32;

var copysel : word;

begin
     copysel := create_code_segment_alias_descriptor(get_ds);
     {...}
     free_ldt_descriptor(copysel);
end.
\end{verbatim}
\end{FPCList}

\functionl{Get\_Linear\_Addr}{GetLinearAddr}{(PhysAddr : longint;Size : longint)}{longint}
{\var{Get\_Linear\_Addr} converts a physical address \var{PhysAddr} into 
a linear address.

{\em Notes:}
\begin{itemize}
\item Only works under real DPMI.
\end{itemize}
}{None.}{}

For an example, see \seef{AllocateLdtDescriptors}.

\functionl{Global\_Dos\_Alloc}{GlobalDosAlloc}{(Len : longint)}{longint}
{\var{Global\_Dos\_Alloc}
allocates a block of memory with length \var{Len} from the \dos memory pool, 
i.e. memory below the 1 MB boundary that is controlled by \dos. 
Such memory blocks are typically used to exchange data with real mode 
programs, TSRs, or device drivers. 

The function returns both the real mode segment base address of
the block and one or more descriptors that can be used by protected mode
applications to access the block.
The high word is the selector to this block, and the low word the
\dos real mode segment. The offset of this block is always zero.

{\em Notes:}
\begin{itemize}
\item Should only used for temporary buffers to get real mode information
(e.g. interrupts that need a data structure in ES:DI), because every
single block needs an unique selector.
\end{itemize}
}{None.}{\seep{GlobalDosFree}}

\begin{FPCList}
\item[Example]
\begin{verbatim}
uses go32;

procedure dosalloc (var selector : word; 
                    var segment : word; 
                    size : longint);
var result : longint;

begin
     result := global_dos_alloc(size);
     selector := word(result);
     segment := word(result shr 16);
end;

procedure dosfree(selector : word);
begin
     global_dos_free(selector);
end;

var selector : word;
    segment : word;
    r : trealregs;

begin
     fillchar(r, sizeof(r), 0);
     fillchar(any_record, sizeof(any_record), 0);
     dosalloc(selector, segment, sizeof(VBE_vgainfo)); 
     { allocate a real mode memory block }
     any_record.any_entry := 1000;
     dosmemput(segment, 0, any_record, sizeof(any_record)); 
     { copy the record to real mode memory } 
     r.realeax := $0000; 
     r.reales := segment; r.realedi := 0;
     realintr(IntNr, r);  { do our interrupt }
     dosmemget(segment, 0, any_record, sizeof(any_record)); 
     { get the record from real mode memory }
     dosfree(selector); { free selector afterwards }
end.
\end{verbatim}
\end{FPCList}

\procedurel{Global\_Dos\_Free}{GlobalDosFree}{(Sel : word)}
{var{Global\_Dos\_Free} frees a previously allocated \dos memory 
block, described by selector \var{Sel}.
}
{None.}
{\seef{GlobalDosAlloc}}

For an example, see \seef{GlobalDosAlloc}.

