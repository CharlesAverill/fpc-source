\chapter{The GO32 unit}
This chapter of the documentation describe the GO32 unit for the Free Pascal
compiler under \dos. It was donated by Thomas Schatzl
(tom\_at\_work@geocities.com), for which my thanks.
This unit was first written for \dos by Florian Klaempfl.
This chapter is divided in three sections. The first section is an
introduction to the GO32 unit. The second section lists the pre-defined
constants, types and variables. The third section describes the functions
which appear in the interface part of the GO32 unit.
\section{Introduction}
These docs contain information about the GO32 unit. Only the GO32V2 DPMI
mode is discussed by me here due to the fact that new applications shouldn't
be created with the older GO32V1 model. The former is much more advanced and
better. Additionally a lot of functions only work in DPMI mode anyway.
I hope the following explanations and introductions aren't too confusing at
all. If you notice an error or bug send it to the FPC mailing list or
directly to me.
So let's get started and happy and error free coding I wish you....
\hfill Thomas Schatzl, 25. August 1998
\section{Protected mode memory organization}
\subsection{What is DPMI}
The \dos Protected Mode Interface helps you with various aspects of protected
mode programming. These are roughly divided into descriptor handling, access
to \dos memory, management of interrupts and exceptions, calls to real mode
functions and other stuff. Additionally it automatically provides swapping
to disk for memory intensive applications.
A DPMI host (either a Windows \dos box or CWSDPMI.EXE) provides these
functions for your programs.
\subsection{Selectors and descriptors}
Descriptors are a bit like real mode segments; they describe (as the name
implies) a memory area in protected mode. A descriptor contains information
about segment length, its base address and the attributes of it (i.e. type,
access rights, ...).
These descriptors are stored internally in a so-called descriptor table,
which is basically an array of such descriptors.
Selectors are roughly an index into this table.
Because these 'segments' can be up to 4 GB in size, 32 bits aren't
sufficient anymore to describe a single memory location like in real mode.
48 bits are now needed to do this, a 32 bit address and a 16 bit sized
selector. The GO32 unit provides the tseginfo record to store such a
pointer.
But due to the fact that most of the time data is stored and accessed in the
\%ds selector, FPC assumes that all pointers point to a memory location of
this selector. So a single pointer is still only 32 bits in size. This value
represents the offset from the data segment base address to this memory
location.
\subsection{FPC specialities}
The \%ds and \%es selector MUST always contain the same value or some system
routines may crash when called. The \%fs selector is preloaded with the
DOSMEMSELECTOR variable at startup, and it MUST be restored after use,
because again FPC relys on this for some functions. Luckily we asm
programmers can still use the \%gs selector for our own purposes, but for how
long ?
See also:
% tseginfo, dosmemselector, \dos memory access,
 \seefl{get\_cs}{getcs}, 
 \seefl{get\_ds}{getds},
 \seefl{gett\_ss}{getss}, 
 \seefl{allocate\_ldt\_descriptors}{allocateldtdescriptors},
 \seefl{free\_ldt\_descriptor}{freeldtdescriptor},
 \seefl{segment\_to\_descriptor}{segmenttodescriptor},
 \seefl{get\_next\_selector\_increment\_value}{getnextselectorincrementvalue},
 \seefl{get\_segment\_base\_address}{getsegmentbaseaddress},
 \seefl{set\_segment\_base\_address}{setsegmentbaseaddress},
 \seefl{set\_segment\_limit}{setsegmentlimit},
 \seefl{create\_code\_segment\_alias\_descriptor}{createcodesegmentaliasdescriptor} 
\subsection{\dos memory access}
\dos memory is accessed by the predefined \var{dosmemselector} selector; 
the GO32 unit additionally provides some functions to help you with standard tasks,
like copying memory from heap to \dos memory and the likes. Because of this
it is strongly recommened to use them, but you are still free to use the
provided standard memory accessing functions which use 48 bit pointers. The
third, but only thought for compatibility purposes, is using the
\var{mem[]}-arrays. These arrays map the whole 1 Mb \dos space. They shouldn't be
used within new programs.
To convert a segment:offset real mode address to a protected mode linear
address you have to multiply the segment by 16 and add its offset. This
linear address can be used in combination with the DOSMEMSELECTOR variable.
See also: 
\seep{dosmemget},
\seepl{dosmemput}{dosmemput},
\seepl{dosmemmove}{dosmemmove},
\seepl{dosmemfillchar}{dosmemfillchar},
\seepl{dosmemfillword}{dosmemfillword},
mem[]-arrays, 
\seepl{seg\_move}{segmove},
\seepl{seg\_fillchar}{segfillchar},
\seepl{seg\_fillword}{segfillword}. 
\subsection{I/O port access}
The I/O port access is done via the various \seef{inportb}, \seep{outportb}
functions
which are available. Additionally Free Pascal supports the Turbo Pascal
PORT[]-arrays but it is by no means recommened to use them, because they're
only for compatibility purposes.
See also: \seep{outportb}, \seef{inportb}, PORT[]-arrays
\subsection{Processor access}
These are some functions to access various segment registers (\%cs, \%ds, \%ss)
which makes your work a bit easier.
See also: \seefl{get\_cs}{getcs}, \seefl{get\_ds}{getds}, 
\seefl{get\_ss}{getss} 
\subsection{Interrupt redirection}
Interrupts are program interruption requests, which in one or another way
get to the processor; there's a distinction between software and hardware
interrupts. The former are explicitely called by an 'int' instruction and
are a bit comparable to normal functions. Hardware interrupts come from
external devices like the keyboard or mouse. These functions are called
handlers.
\subsection{Handling interrupts with DPMI}
The interrupt functions are real-mode procedures; they normally can't be
called in protected mode without the risk of an protection fault. So the
DPMI host creates an interrupt descriptor table for the application.
Initially all software interrupts (except for int 31h, 2Fh and 21h function
4Ch) or external hardware interrupts are simply directed to a handler that
reflects the interrupt in real-mode, i.e. the DPMI host's default handlers
switch the CPU to real-mode, issue the interrupt and switch back to
protected mode. The contents of general registers and flags are passed to
the real mode handler and the modified registers and flags are returned to
the protected mode handler. Segment registers and stack pointer are not
passed between modes.
\subsection{Protected mode interrupts vs. Real mode interrupts}
As mentioned before, there's a distinction between real mode interrupts and
protected mode interrupts; the latter are protected mode programs, while the
former must be real mode programs. To call a protected mode interrupt
handler, an assembly 'int' call must be issued, while the other is called
via the realintr() or intr() function. Consequently, a real mode interrupt
then must either reside in \dos memory (<1MB) or the application must
allocate a real mode callback address via the get\_rm\_callback() function.
\subsection{Creating own interrupt handlers}
Interrupt redirection with FPC pascal is done via the set\_pm\_interrupt() for
protected mode interrupts or via the set\_rm\_interrupt() for real mode
interrupts.
\subsection{Disabling interrupts}
The GO32 unit provides the two procedures disable() and enable() to disable
and enable all interrupts.
\subsection{Hardware interrupts}
Hardware interrupts are generated by hardware devices when something unusual
happens; this could be a keypress or a mouse move or any other action. This
is done to minimize CPU time, else the CPU would have to check all installed
hardware for data in a big loop (this method is called 'polling') and this
would take much time.
A standard IBM-PC has two interrupt controllers, that are responsible for
these hardware interrupts: both allow up to 8 different interrupt sources
(IRQs, interrupt requests). The second controller is connected to the first
through IRQ 2 for compatibility reasons, e.g. if controller 1 gets an IRQ 2,
he hands the IRQ over to controller 2. Because of this up to 15 different
hardware interrupt sources can be handled.
IRQ 0 through IRQ 7 are mapped to interrupts 8h to Fh and the second
controller (IRQ 8 to 15) is mapped to interrupt 70h to 77h.
All of the code and data touched by these handlers MUST be locked (via the
various locking functions) to avoid page faults at interrupt time. Because
hardware interrupts are called (as in real mode) with interrupts disabled,
the handler has to enable them before it returns to normal program
execution. Additionally a hardware interrupt must send an EOI (end of
interrupt) command to the responsible controller; this is acomplished by
sending the value 20h to port 20h (for the first controller) or A0h (for the
second controller).
The following example shows how to redirect the keyboard interrupt.
\latex{\inputlisting{go32ex/keyclick.pp}}
\html{\begin{FPCList}
\item[Example]
\begin{verbatim}
Program Keyclick;

uses crt, 
     go32;

const kbdint = $9; 

var oldint9_handler : tseginfo;
    newint9_handler : tseginfo;

    clickproc : pointer; 

{$ASMMODE DIRECT}
procedure int9_handler; assembler;
asm
   cli
   pushal
   movw %cs:INT9_DS, %ax
   movw %ax, %ds
   movw %ax, %es
   movw U_GO32_DOSMEMSELECTOR, %ax
   movw %ax, %fs
   call *_CLICKPROC
   popal

   ljmp %cs:OLDHANDLER 

INT9_DS: .word 0
OLDHANDLER:
         .long 0
         .word 0
end;

procedure int9_dummy; begin end;

procedure clicker;
begin
     sound(500); delay(10); nosound;
end;

procedure clicker_dummy; begin end;

procedure install_click;
begin
     clickproc := @clicker;
     lock_data(clickproc, sizeof(clickproc));
     lock_data(dosmemselector, sizeof(dosmemselector));

     lock_code(@clicker, 
               longint(@clicker_dummy)-longint(@clicker));
     lock_code(@int9_handler, 
               longint(@int9_dummy)
                - longint(@int9_handler));
     newint9_handler.offset := @int9_handler;
     newint9_handler.segment := get_cs;
     get_pm_interrupt(kbdint, oldint9_handler);
     asm
        movw %ds, %ax
        movw %ax, INT9_DS
        movl _OLDINT9_HANDLER, %eax
        movl %eax, OLDHANDLER
        movw 4+_OLDINT9_HANDLER, %ax
        movw %ax, 4+OLDHANDLER
     end;
     set_pm_interrupt(kbdint, newint9_handler);
end;

procedure remove_click;
begin
     set_pm_interrupt(kbdint, oldint9_handler);
     unlock_data(dosmemselector, sizeof(dosmemselector));
     unlock_data(clickproc, sizeof(clickproc));
     unlock_code(@clicker, 
                 longint(@clicker_dummy)
                  - longint(@clicker));
     unlock_code(@int9_handler, 
                 longint(@int9_dummy)
                  - longint(@int9_handler));
end;

var ch : char;

begin
     install_click;
     Writeln('Enter any message.',
             ' Press return when finished');
     while (ch <> #13) do begin
           ch := readkey; write(ch);
     end;
     remove_click;
end.
\end{verbatim}
\end{FPCList}}
\subsection{Software interrupts}
Ordinarily, a handler installed with
\seefl{set\_pm\_interrupt}{setpminterrupt} only services software
interrupts that are executed in protected mode; real mode software
interrupts can be redirected by \seefl{set\_rm\_interrupt}{setrminterrupt}.
See also \seefl{set\_rm\_interrupt}{setrminterrupt}, 
\seefl{get\_rm\_interrupt}{getrminterrupt},
\seefl{set\_pm\_interrupt}{setpminterrupt},
\seefl{get\_pm\_interrupt}{getpminterrupt}, 
\seefl{lock\_data}{lockdata}, 
\seefl{lock\_code}{lockcode}, 
\seep{enable}, 
\seep{disable},
\seepl{outportb}{outportb} 
Executing software interrupts
Simply execute a realintr() call with the desired interrupt number and the
supplied register data structure.
But some of these interrupts require you to supply them a pointer to a
buffer where they can store data to or obtain data from in memory. These
interrupts are real mode functions and so they only can access the first Mb
of linear address space, not FPC's data segment.
For this reason FPC supplies a pre-initialized \dos memory location within
the GO32 unit. This buffer is internally used for \dos functions too and so
it's contents may change when calling other procedures. It's size can be
obtained with \seefl{tb\_size}{tbsize} and it's linear address via 
\seefl{transfer\_buffer}{transferbuffer}.
Another way is to allocate a completely new \dos memory area via the
\seefl{global\_dos\_alloc}{globaldosalloc} function for your use and 
supply its real mode address.
See also:
\seefl{tb\_size}{tbsize},
\seefl{transfer\_buffer}{transferbuffer}.
\seefl{global\_dos\_alloc}{globaldosalloc},
\seefl{global\_dos\_free}{globaldosfree},
\seef{realintr}
The following examples illustrate the use of software interrupts.
\latex{\inputlisting{go32ex/softint.pp}}
\html{\begin{FPCList}
\item[Example]
\begin{verbatim}
Program softint;

uses go32;

var r : trealregs;

begin
     r.al := $01;
     realintr($21, r);
     Writeln('DOS v', r.al,'.',r.ah, ' detected');
end.\end{verbatim}
\end{FPCList}}
\latex{\inputlisting{go32ex/rmpmint.pp}}
\html{\input{go32ex/rmpmint.tex}}
\subsection{Real mode callbacks}
The callback mechanism can be thought of as the converse of calling a real
mode procedure (i.e. interrupt), which allows your program to pass
information to a real mode program, or obtain services from it in a manner
that's transparent to the real mode program.
In order to make a real mode callback available, you must first get the real
mode callback address of your procedure and the selector and offset of a
register data structure. This real mode callback address (this is a
segment:offset address) can be passed to a real mode program via a software
interrupt, a \dos memory block or any other convenient mechanism.
When the real mode program calls the callback (via a far call), the DPMI
host saves the registers contents in the supplied register data structure,
switches into protected mode, and enters the callback routine with the
following conditions:
\begin{itemize}
\item interrupts disabled
\item \var{\%CS:\%EIP} = 48 bit pointer specified in the original call to 
\seefl{get\_rm\_callback}{getrmcallback}
\item \var{\%DS:\%ESI} = 48 bit pointer to to real mode \var{SS:SP}
\item \var{\%ES:\%EDI} = 48 bit pointer of real mode register data
structure. 
\item \var{\%SS:\%ESP} = locked protected mode stack
\item  All other registers undefined
\end{itemize}
The callback procedure can then extract its parameters from the real mode
register data structure and/or copy parameters from the real mode stack to
the protected mode stack. Recall that the segment register fields of the
real mode register data structure contain segment or paragraph addresses
that are not valid in protected mode. Far pointers passed in the real mode
register data structure must be translated to virtual addresses before they
can be used with a protected mode program.
The callback procedure exits by executing an IRET with the address of the
real mode register data structure in \var{\%ES:\%EDI}, passing information back to
the real mode caller by modifying the contents of the real mode register
data structure and/or manipulating the contents of the real mode stack. The
callback procedure is responsible for setting the proper address for
resumption of real mode execution into the real mode register data
structure; typically, this is accomplished by extracting the return address
from the real mode stack and placing it into the \var{\%CS:\%EIP} fields of the real
mode register data structure. After the IRET, the DPMI host switches the CPU
back into real mode, loads ALL registers with the contents of the real mode
register data structure, and finally returns control to the real mode
program.
All variables and code touched by the callback procedure MUST be locked to
prevent page faults.
See also: \seefl{get\_rm\_callback}{getrmcallback},
\seefl{free\_rm\_callback}{freermcallback}, 
\seefl{lock\_code}{lockcode}, 
\seefl{lock\_data}{lockdata} 
\section{Types, Variables and Constants}
\subsection{Constants}
\subsubsection{Constants returned by get\_run\_mode}
Tells you under what memory environment (e.g. memory manager) the program
currently runs.
\begin{verbatim}
rm_unknown = 0; { unknown }
rm_raw     = 1; { raw (without HIMEM) } 
rm_xms     = 2; { XMS (for example with HIMEM, without EMM386) } 
rm_vcpi    = 3; { VCPI (for example HIMEM and EMM386) } 
rm_dpmi    = 4; { DPMI (for example \dos box or 386Max) }
\end{verbatim}
Note: GO32V2 {\em always} creates DPMI programs, so you need a suitable DPMI
host like \file{CWSDPMI.EXE} or a Windows \dos box. So you don't need to check it,
these constants are only useful in GO32V1 mode.
\subsubsection{Processor flags constants}
They are provided for a simple check with the flags identifier in the
trealregs type. To check a single flag, simply do an AND operation with the
flag you want to check. It's set if the result is the same as the flag
value.
\begin{verbatim}
const carryflag = $001; 
parityflag      = $004; 
auxcarryflag    = $010; 
zeroflag        = $040; 
signflag        = $080; 
trapflag        = $100; 
interruptflag   = $200;
directionflag   = $400; 
overflowflag    = $800;
\end{verbatim}
\subsection{Predefined types}
\begin{verbatim}
type tmeminfo = record
            available_memory : Longint; 
            available_pages : Longint;
            available_lockable_pages : Longint; 
            linear_space : Longint;
            unlocked_pages : Longint; 
            available_physical_pages : Longint;
            total_physical_pages : Longint; 
            free_linear_space : Longint;
            max_pages_in_paging_file : Longint; 
            reserved : array[0..2] of Longint;
   end;
\end{verbatim}
Holds information about the memory allocation, etc.
\begin{tabular}{ll}
Record entry & Description \\ \hline
\var{available\_memory} & Largest available free block in bytes. \\
\var{available\_pages} & Maximum unlocked page allocation in pages \\
\var{available\_lockable\_pages} &  Maximum locked page allocation in pages. \\
\var{linear\_space} &  Linear address space size in pages. \\
\var{unlocked\_pages} & Total number of unlocked pages. \\
\var{available\_physical\_pages} &  Total number of free pages.\\
\var{total\_physical\_pages} &  Total number of physical pages. \\
\var{free\_linear\_space} & Free linear address space in pages.\\
\var{max\_pages\_in\_paging\_file} &  Size of paging file/partition in
pages. \\
\end{tabular}
NOTE: The value of a field is -1 (0ffffffffh) if the value is unknown, it's
only guaranteed, that \var{available\_memory} contains a valid value.
The size of the pages can be determined by the get\_page\_size() function.
\begin{verbatim}
type 
trealregs = record
  case Integer of 
    1: { 32-bit } 
      (EDI, ESI, EBP, Res, EBX, EDX, ECX, EAX: Longint; 
       Flags, ES, DS, FS, GS, IP, CS, SP, SS: Word); 
    2: { 16-bit } 
      (DI, DI2, SI, SI2, BP, BP2, R1, R2: Word;
       BX, BX2, DX, DX2, CX, CX2, AX, AX2: Word);
    3: { 8-bit } 
      (stuff: array[1..4] of Longint;
       BL, BH, BL2, BH2, DL, DH, DL2, DH2, CL,
       CH, CL2, CH2, AL, AH, AL2, AH2: Byte);
    4: { Compat } 
      (RealEDI, RealESI, RealEBP, RealRES, RealEBX, 
       RealEDX, RealECX, RealEAX: Longint; 
       RealFlags, RealES, RealDS, RealFS, RealGS, 
       RealIP, RealCS, RealSP, RealSS: Word);
    end;
    registers = trealregs;
\end{verbatim}
These two types contain the data structure to pass register values to a
interrupt handler or real mode callback.
\begin{verbatim}
type tseginfo = record
             offset : Pointer; segment : Word; end;
\end{verbatim}
This record is used to store a full 48-bit pointer. This may be either a
protected mode selector:offset address or in real mode a segment:offset
address, depending on application.
See also: Selectors and descriptors, \dos memory access, Interrupt
redirection
\subsection{Variables.}
\begin{verbatim}
var dosmemselector : Word;
\end{verbatim}
Selector to the \dos memory. The whole \dos memory is automatically mapped to
this single descriptor at startup. This selector is the recommened way to
access \dos memory.
\begin{verbatim}
  var int31error : Word;
\end{verbatim}
This variable holds the result of a DPMI interrupt call. Any nonzero value
must be treated as a critical failure.
\section{Functions and Procedures}
\begin{functionl}{allocate\_ldt\_descriptors}{allocateldtdescriptors}
\Declaration
Function allocate\_ldt\_descriptors (count : Word) : Word;

\Description
Allocates a number of new descriptors.
Parameters: 
\begin{description}
\item[count:\ ] specifies the number of requested unique descriptors.
\end{description}
Return value: The base selector.
Notes: The descriptors allocated must be initialized by the application with
other function calls. This function returns descriptors with a limit and
size value set to zero. If more than one descriptor was requested, the
function returns a base selector referencing the first of a contiguous array
of descriptors. The selector values for subsequent descriptors in the array
can be calculated by adding the value returned by the
\seefl{get\_next\_selector\_increment\_value}{getnextselectorincrementvalue} 
function.

\Errors
 Check the \var{int31error} variable. 
\SeeAlso

\seefl{free\_ldt\_descriptor}{freeldtdescriptor},
\seefl{get\_next\_selector\_increment\_value}{getnextselectorincrementvalue},
\seefl{segment\_to\_descriptor}{segmenttodescriptor},
\seefl{create\_code\_segment\_alias\_descriptor}{createcodesegmentaliasdescriptor},
\seefl{set\_segment\_limit}{setsegmentlimit},
\seefl{set\_segment\_base\_address}{setsegmentbaseaddress} 

\end{functionl}
\latex{\inputlisting{go32ex/seldes.pp}}
\html{\input{go32ex/seldes.tex}}
\begin{functionl}{allocate\_memory\_block}{allocatememoryblock}
\Declaration
Function allocate\_memory\_block (size:Longint) : Longint;

\Description
Allocates a block of linear memory.
Parameters: 
\begin{description}
\item[size:\ ] Size of requested linear memory block in bytes.
\end{description}
Returned values: blockhandle - the memory handle to this memory block. Linear
address of the requested memory.
Notes: WARNING: According to my DPMI docs this function is not implemented
correctly. Normally you should also get a blockhandle to this block after
successful operation. This handle is used to free the memory block
afterwards or use this handle for other purposes. So this block can't be
deallocated and is henceforth unusuable !
This function doesn't allocate any descriptors for this block, it's the
applications resposibility to allocate and initialize for accessing this
memory.

\Errors
 Check the \var{int31error} variable.
\SeeAlso
 \seefl{free\_memory\_block}{freememoryblock} 
\end{functionl}
\begin{procedure}{copyfromdos}
\Declaration
Procedure copyfromdos (var addr; len : Longint);

\Description

Copies data from the pre-allocated \dos memory transfer buffer to the heap.
Parameters:
\begin{description}
\item[addr:\ ] data to copy to.
\item[len:\ ] number of bytes to copy to heap.
\end{description}
Notes:
Can only be used in conjunction with the \dos memory transfer buffer.

\Errors
Check the \var{int31error} variable.
\SeeAlso
\seefl{tb\_size}{tbsize}, \seefl{transfer\_buffer}{transferbuffer}, 
\seep{copytodos}
\end{procedure}
\begin{procedure}{copytodos}
\Declaration
Procedure copytodos (var addr; len : Longint);

\Description
Copies data from heap to the pre-allocated \dos memory buffer.
Parameters:
\begin{description}
\item[addr:\ ] data to copy from.
\item[len:\ ] number of bytes to copy to \dos memory buffer.
\end{description}
Notes: This function fails if you try to copy more bytes than the transfer
buffer is in size. It can only be used in conjunction with the transfer
buffer.

\Errors
 Check the \var{int31error} variable.
\SeeAlso
\seefl{tb\_size}{tbsize}, \seefl{transfer\_buffer}{transferbuffer}, 
\seep{copyfromdos}
\end{procedure}
\begin{functionl}{create\_code\_segment\_alias\_descriptor}{createcodesegmentaliasdescriptor}
\Declaration
Function create\_code\_segment\_alias\_descriptor (seg : Word) : Word;

\Description

Creates a new descriptor that has the same base and limit as the specified
descriptor.
Parameters: 
\begin{description}
\item[seg:\ ] selector.
\end{description}
Return values: The data selector (alias).
Notes: In effect, the function returns a copy of the descriptor. The
descriptor alias returned by this function will not track changes to the
original descriptor. In other words, if an alias is created with this
function, and the base or limit of the original segment is then changed, the
two descriptors will no longer map the same memory.

\Errors
Check the \var{int31error} variable.
\SeeAlso
 
\seefl{allocate\_ldt\_descriptors}{allocateldtdescriptors},
\seefl{set\_segment\_limit}{setsegmentlimit}, 
\seefl{set\_segment\_base\_address}{setsegmentbaseaddress} 
\end{functionl}
\begin{procedure}{disable}
\Declaration
Procedure disable ;

\Description
Disables all hardware interrupts by execution a CLI instruction.
Parameters: None.

\Errors
None.
\SeeAlso
\seep{enable}
\end{procedure}
\begin{procedure}{dosmemfillchar}
\Declaration
Procedure dosmemfillchar (seg, ofs : Word; count : Longint; c : char);

\Description
Sets a region of \dos memory to a specific byte value.
Parameters:
\begin{description}
\item[seg:\ ] real mode segment.
\item[ofs:\ ] real mode offset. 
\item[count:\ ] number of bytes to set.
\item[c:\ ] value to set memory to.
\end{description}
Notes: No range check is performed.

\Errors
 None.
\SeeAlso
 
\seep{dosmemput},
\seep{dosmemget}, 
\seep{dosmemmove}{dosmemmove}, 
\seepl{dosmemfillword}{dosmemfillword}, 
\seepl{seg\_move}{segmove},
\seepl{seg\_fillchar}{segfillchar},
\seepl{seg\_fillword}{segfillword} 
\end{procedure}
\latex{\inputlisting{go32ex/textmess.pp}}
\html{\begin{FPCList}
\item[Example]
\begin{verbatim}
Program textmess;

uses crt, go32;

const columns = 80;
      rows = 25; 
      screensize = rows*columns*2;

      text = '! Hello world !'; 

var textofs : Longint;
    save_screen : array[0..screensize-1] of byte;
    curx, cury : Integer;

begin
     randomize;
     dosmemget($B800, 0, save_screen, screensize);
     curx := wherex; cury := wherey;
     gotoxy(1, 1); Write(text);
     textofs := screensize + length(text)*2;
     dosmemmove($B800, 0, $B800, textofs, length(text)*2);
     dosmemfillchar($B800, 0, screensize, #0);
     while (not keypressed) do 
       begin
       dosmemfillchar($B800, 
                      textofs + random(length(text))*2 + 1,
                      1, char(random(255)));
       dosmemmove($B800, textofs, $B800,
                  random(columns)*2+random(rows)*columns*2,
                  length(text)*2);
           delay(1);
     end;
     readkey;
     readkey;
     dosmemput($B800, 0, save_screen, screensize);
     gotoxy(curx, cury);
end.\end{verbatim}
\end{FPCList}}
\begin{procedure}{dosmemfillword}
\Declaration
Procedure dosmemfillword (seg,ofs : Word; count : Longint; w : Word);

\Description
Sets a region of \dos memory to a specific word value.
Parameters: 
\begin{description}
\item[seg:\ ] real mode segment.
\item[ofs:\ ] real mode offset. 
\item[count:\ ] number of words to set.
\item[w:\ ] value to set memory to.
\end{description}
Notes: No range check is performed.

\Errors
 None.
\SeeAlso
 
\seep{dosmemput},
\seepl{dosmemget}{dosmemget}, 
\seepl{dosmemmove}{dosmemmove}, 
\seepl{dosmemfillchar}{dosmemfillchar}, 
\seepl{seg\_move}{segmove}, 
\seepl{seg\_fillchar}{segfillchar},
\seepl{seg\_fillword}{segfillword} 
\end{procedure}
\begin{procedure}{dosmemget}
\Declaration
Procedure dosmemget (seg : Word; ofs : Word; var data; count : Longint);

\Description
Copies data from the \dos memory onto the heap.
Parameters:
\begin{description}
\item[seg:\ ] source real mode segment.
\item[ofs:\ ] source real mode offset.
\item[data:\ ] destination. 
\item[count:\ ] number of bytes to copy.
\end{description}
Notes: No range checking is performed.

\Errors
 None. 
\SeeAlso
\seep{dosmemput},
\seep{dosmemmove},
\seep{dosmemfillchar},
\seep{dosmemfillword},
\seepl{seg\_move}{segmove},
\seepl{seg\_fillchar}{segfillchar}, 
\seepl{seg\_fillword}{segfillword}  
\end{procedure}
For an example, see \seefl{global\_dos\_alloc}{globaldosalloc}.
\begin{procedure}{dosmemmove}
\Declaration
Procedure dosmemmove (sseg, sofs, dseg, dofs : Word; count : Longint);

\Description
Copies count bytes of data between two \dos real mode memory locations.
Parameters:
\begin{description}
\item[sseg:\ ] source real mode segment. 
\item[sofs:\ ] source real mode offset.
\item[dseg:\ ] destination real mode segment. 
\item[dofs:\ ] destination real mode offset.
\item[count:\ ] number of bytes to copy.
\end{description}
Notes: No range check is performed in any way.

\Errors
 None.
\SeeAlso
\seep{dosmemput}, 
\seep{dosmemget},
\seep{dosmemfillchar}, 
\seep{dosmemfillword}
\seepl{seg\_move}{segmove}, 
\seepl{seg\_fillchar}{segfillchar},
\seepl{seg\_fillword}{segfillword} 
\end{procedure}
For an example, see \seepl{seg\_fillchar}{segfillchar}.
\begin{procedure}{dosmemput}
\Declaration
Procedure dosmemput (seg : Word; ofs : Word; var data; count : Longint);

\Description
Copies heap data to \dos real mode memory.
Parameters:
\begin{description}
\item[seg:\ ] destination real mode segment.
\item[ofs:\ ] destination real mode offset. 
\item[data:\ ] source. 
\item[count:\ ] number of bytes to copy.
\end{description}
Notes: No range checking is performed.

\Errors
 None. 
\SeeAlso
\seep{dosmemget}, 
\seep{dosmemmove},
\seep{dosmemfillchar},
\seep{dosmemfillword},
\seepl{seg\_move}{segmove},
\seepl{seg\_fillchar}{segfillchar},
\seepl{seg\_fillword}{segfillword} 
\end{procedure}
For an example, see \seefl{global\_dos\_alloc}{globaldosalloc}.
\begin{procedure}{enable}
\Declaration
Procedure enable ;

\Description

Enables all hardware interrupts by executing a STI instruction.
Parameters: None.

\Errors
None.
\SeeAlso
 \seep{disable} 
\end{procedure}
\begin{functionl}{free\_ldt\_descriptor}{freeldtdescriptor}
\Declaration
Function free\_ldt\_descriptor (des : Word) : boolean;

\Description
Frees a previously allocated descriptor.
Parameters:
\begin{description}
\item[des:\ ] The descriptor to be freed.
\end{description}
Return value: \var{True} if successful, \var{False} otherwise.
Notes: After this call this selector is invalid and must not be used for any
memory operations anymore. Each descriptor allocated with
\seefl{allocate\_ldt\_descriptors}{allocateldtdescriptors} must be freed 
individually with this function,
even if it was previously allocated as a part of a contiguous array of
descriptors.

\Errors
Check the \var{int31error} variable.
\SeeAlso

\seefl{allocate\_ldt\_descriptors}{allocateldtdescriptors},
\seefl{get\_next\_selector\_increment\_value}{getnextselectorincrementvalue} 

\end{functionl}
For an example, see 
\seefl{allocate\_ldt\_descriptors}{allocateldtdescriptors}.
\begin{functionl}{free\_memory\_block}{freememoryblock}
\Declaration
Function free\_memory\_block (blockhandle :
Longint) : boolean;

\Description
Frees a previously allocated memory block.
Parameters: 
\begin{description}
\item{blockhandle:} the handle to the memory area to free.
\end{description}
Return value: \var{True} if successful, \var{false} otherwise.
Notes: Frees memory that was previously allocated with
\seefl{allocate\_memory\_block}{allocatememoryblock} . 
This function doesn't free any descriptors mapped to this block, 
it's the application's responsibility.

\Errors
 Check \var{int31error} variable.
\SeeAlso
\seefl{allocate\_memory\_block}{allocatememoryblock} 
\end{functionl}
\begin{functionl}{free\_rm\_callback}{freermcallback}
\Declaration
Function free\_rm\_callback (var intaddr : tseginfo) : boolean;

\Description

Releases a real mode callback address that was previously allocated with the
\seefl{get\_rm\_callback}{getrmcallback}  function.
Parameters: 
\begin{description}
\item[intaddr:\ ] real mode address buffer returned by 
\seefl{get\_rm\_callback}{getrmcallback} .
\end{description}
Return values: \var{True} if successful, \var{False} if not

\Errors
 Check the \var{int31error} variable.
\SeeAlso

\seefl{set\_rm\_interrupt}{setrminterrupt},
\seefl{get\_rm\_callback}{getrmcallback}

\end{functionl}
For an example, see \seefl{get\_rm\_callback}{getrmcallback}.
\begin{functionl}{get\_cs}{getcs}
\Declaration
Function get\_cs  : Word;

\Description

Returns the cs selector.
Parameters: None.
Return values: The content of the cs segment register.

\Errors
None.
\SeeAlso
 \seefl{get\_ds}{getds}, \seefl{get\_ss}{getss}
\end{functionl}
For an example, see \seefl{set\_pm\_interrupt}{setpminterrupt}.
\begin{functionl}{get\_descriptor\_access\_rights}{getdescriptoraccessrights}
\Declaration
Function get\_descriptor\_access\_rights (d : Word) : Longint;

\Description
Gets the access rights of a descriptor.
Parameters: 
\begin{description}
\item{d} selector to descriptor.
\end{description}
Return value: Access rights bit field.

\Errors
Check the \var{int31error} variable.
\SeeAlso
 
\seefl{set\_descriptor\_access\_rights}{setdescriptoraccessrights}
\end{functionl}
\begin{functionl}{get\_ds}{getds}
\Declaration
Function get\_ds  : Word;

\Description
Returns the ds selector.
Parameters: None.
Return values: The content of the ds segment register.

\Errors
 None.
\SeeAlso
 \seefl{get\_cs}{getcs}, \seefl{get\_ss}{getss}
\end{functionl}
\begin{functionl}{get\_linear\_addr}{getlinearaddr}
\Declaration
Function get\_linear\_addr (phys\_addr : Longint; size : Longint) : Longint;

\Description
Converts a physical address into a linear address.
Parameters: 
\begin{description}
\item [phys\_addr:\ ] physical address of device.
\item [size:\ ] Size of region to map in bytes.
\end{description}
Return value: Linear address that can be used to access the physical memory.
Notes: It's the applications resposibility to allocate and set up a
descriptor for access to the memory. This function shouldn't be used to map
real mode addresses.

\Errors
 Check the \var{int31error} variable.
\SeeAlso
 
\seefl{allocate\_ldt\_descriptors}{allocateldtdescriptors}, \seefl{set\_segment\_limit}{setsegmentlimit},
\seefl{set\_segment\_base\_address}{setsegmentbaseaddress} 
\end{functionl}
\begin{functionl}{get\_meminfo}{getmeminfo}
\Declaration
Function get\_meminfo (var meminfo : tmeminfo) : boolean;

\Description
 Returns information about the amount of available physical memory, linear
address space, and disk space for page swapping.
Parameters:
\begin{description}
\item[meminfo:\ ] buffer to fill memory information into.
\end{description}
Return values: Due to an implementation bug this function always returns
\var{False}, but it always succeeds.
Notes: Only the first field of the returned structure is guaranteed to
contain a valid value. Any fields that are not supported by the DPMI host
will be set by the host to \var{-1 (0FFFFFFFFH)} to indicate that the information
is not available. The size of the pages used by the DPMI host can be
obtained with the \seefl{get\_page\_size}{getpagesize}  function.

\Errors
Check the \var{int31error} variable.
\SeeAlso
\seefl{get\_page\_size}{getpagesize} 
\end{functionl}
\latex{\inputlisting{go32ex/meminfo.pp}}
\html{\begin{FPCList}
\item[Example]
\begin{verbatim}
Program meminf;

uses go32;

var meminfo : tmeminfo;

begin
get_meminfo(meminfo);
if (int31error <> 0)  then 
 begin
 Writeln('Error getting DPMI memory information... Halting');
 Writeln('DPMI error number : ', int31error);
 end 
else 
 with meminfo do 
   begin
   Writeln('Largest available free block : ', 
           available_memory div 1024, ' kbytes');
   if (available_pages <> -1) then
     Writeln('Maximum available unlocked pages : ', 
              available_pages);
   if (available_lockable_pages <> -1) then
     Writeln('Maximum lockable available pages : ', 
              available_lockable_pages);
   if (linear_space <> -1) then
     Writeln('Linear address space size : ', 
             linear_space*get_page_size div 1024, 
             ' kbytes');
   if (unlocked_pages <> -1) then
     Writeln('Total number of unlocked pages : ', 
             unlocked_pages);
   if (available_physical_pages <> -1) then
     Writeln('Total number of free pages : ', 
             available_physical_pages);
   if (total_physical_pages <> -1) then
     Writeln('Total number of physical pages : ', 
             total_physical_pages);
   if (free_linear_space <> -1) then
     Writeln('Free linear address space : ', 
             free_linear_space*get_page_size div 1024,
             ' kbytes');
   if (max_pages_in_paging_file <> -1) then
     Writeln('Maximum size of paging file : ', 
              max_pages_in_paging_file*get_page_size div 1024, 
              ' kbytes');
  end;
end.\end{verbatim}
\end{FPCList}}
\begin{functionl}{get\_next\_selector\_increment\_value}{getnextselectorincrementvalue}
\Declaration
Function get\_next\_selector\_increment\_value  : Word;

\Description
Returns the selector increment value when allocating multiple subsequent
descriptors via \seefl{allocate\_ldt\_descriptors}{allocateldtdescriptors}.
Parameters: None.
Return value: Selector increment value.
Notes: Because \seefl{allocate\_ldt\_descriptors}{allocateldtdescriptors} only returns the selector for the
first descriptor and so the value returned by this function can be used to
calculate the selectors for subsequent descriptors in the array.

\Errors
 Check the \var{int31error} variable.
\SeeAlso
 \seefl{allocate\_ldt\_descriptors}{allocateldtdescriptors}, 
\seefl{free\_ldt\_descriptor}{freeldtdescriptor} 
\end{functionl}
\begin{functionl}{get\_page\_size}{getpagesize}
\Declaration
Function get\_page\_size  :  Longint;

\Description
 Returns the size of a single memory page.
Return value: Size of a single page in bytes.
Notes: The returned size is typically 4096 bytes.

\Errors
 Check the \var{int31error} variable.
\SeeAlso
 \seefl{get\_meminfo}{getmeminfo} 
\end{functionl}
For an example, see \seefl{get\_meminfo}{getmeminfo}.
\begin{functionl}{get\_pm\_interrupt}{getpminterrupt}
\Declaration
Function get\_pm\_interrupt (vector : byte; var intaddr : tseginfo) : boolean;

\Description
Returns the address of a current protected mode interrupt handler.
Parameters:
\begin{description}
\item[vector:\ ] interrupt handler number you want the address to.
\item[intaddr:\ ] buffer to store address.
\end{description}
Return values: \var{True} if successful, \var{False} if not.
Notes: The returned address is a protected mode selector:offset address.

\Errors
 Check the \var{int31error} variable.
\SeeAlso
 \seefl{set\_pm\_interrupt}{setpminterrupt},
\seefl{set\_rm\_interrupt}{setrminterrupt}, \seefl{get\_rm\_interrupt}{getrminterrupt} 
\end{functionl}
For an example, see \seefl{set\_pm\_interrupt}{setpminterrupt}.
\begin{functionl}{get\_rm\_callback}{getrmcallback}
\Declaration
Function get\_rm\_callback (pm\_func : pointer; const reg : trealregs; var rmcb: tseginfo) : boolean;

\Description

Returns a unique real mode \var{segment:offset} address, known as a "real mode
callback," that will transfer control from real mode to a protected mode
procedure.
Parameters:
\begin{description}
\item[pm\_func:\ ]  pointer to the protected mode callback function.
\item[reg:\ ] supplied registers structure.
\item[rmcb:\ ] buffer to real mode address of callback function.
\end{description}
Return values: \var{True} if successful, otherwise \var{False}.
Notes: Callback addresses obtained with this function can be passed by a
protected mode program for example to an interrupt handler, device driver,
or TSR, so that the real mode program can call procedures within the
protected mode program or notify the protected mode program of an event. The
contents of the supplied regs structure is not valid after function call,
but only at the time of the actual callback.

\Errors
Check the \var{int31error} variable.
\SeeAlso
\seefl{free\_rm\_callback}{freermcallback} 
\end{functionl}
\latex{\inputlisting{go32ex/callback.pp}}
\html{\begin{FPCList}
\item[Example]
\begin{verbatim}
Program callback;

uses crt,
     go32;

const mouseint = $33;          

var mouse_regs    : trealregs;
    mouse_seginfo : tseginfo;

var mouse_numbuttons : longint;

    mouse_action : word;
    mouse_x, mouse_y : Word;
    mouse_b : Word;

    userproc_installed : Longbool;
    userproc_length : Longint;
    userproc_proc : pointer;

{$ASMMODE DIRECT}
procedure callback_handler; assembler;
asm
   pushw %es
   pushw %ds
   pushl %edi
   pushl %esi
   cmpl $1, _USERPROC_INSTALLED
   je .LNoCallback
   pushal
   movw %es, %ax 
   movw %ax, %ds
   movw U_GO32_DOSMEMSELECTOR, %ax
   movw %ax, %fs  
   call *_USERPROC_PROC
   popal
.LNoCallback:

   popl %esi
   popl %edi
   popw %ds
   popw %es

   movl (%esi), %eax
   movl %eax, %es: 42(%edi) 
   addw $4, %es: 46(%edi)
   iret
end;

procedure mouse_dummy; begin end;

procedure textuserproc;
begin
     mouse_b := mouse_regs.bx;
     mouse_x := (mouse_regs.cx shr 3) + 1;
     mouse_y := (mouse_regs.dx shr 3) + 1;
end;

procedure install_mouse (userproc : pointer; 
                         userproclen : longint);
var r : trealregs;
begin
     r.eax := $0; realintr(mouseint, r);
     if (r.eax <> $FFFF) then begin
        Writeln('No Mircosoft compatible mouse found');
        Write('A Mircosoft compatible mouse driver is');
        writeln(' necessary to run this example');
        halt;
     end;
     if (r.bx = $ffff) then mouse_numbuttons := 2
     else mouse_numbuttons := r.bx;
     Writeln(mouse_numbuttons, 
             ' button Mircosoft compatible mouse found.');
     if (userproc <> nil) then begin
        userproc_proc := userproc;
        userproc_installed := true;
        userproc_length := userproclen;
        lock_code(userproc_proc, userproc_length);
     end else begin
         userproc_proc := nil;
         userproc_length := 0;
         userproc_installed := false;
     end;
     lock_data(mouse_x, sizeof(mouse_x));
     lock_data(mouse_y, sizeof(mouse_y));
     lock_data(mouse_b, sizeof(mouse_b));
     lock_data(mouse_action, sizeof(mouse_action));

     lock_data(userproc_installed, sizeof(userproc_installed));
     lock_data(@userproc_proc, sizeof(userproc_proc));

     lock_data(mouse_regs, sizeof(mouse_regs));
     lock_data(mouse_seginfo, sizeof(mouse_seginfo));
     lock_code(@callback_handler, 
               longint(@mouse_dummy) 
                - longint(@callback_handler));
     get_rm_callback(@callback_handler, mouse_regs, mouse_seginfo);
     r.eax := $0c; r.ecx := $7f; r.edx := longint(mouse_seginfo.offset);
     r.es := mouse_seginfo.segment;
     realintr(mouseint, r);
     r.eax := $01;
     realintr(mouseint, r);
end;

procedure remove_mouse;
var r : trealregs;
begin
     r.eax := $02; realintr(mouseint, r);
     r.eax := $0c; r.ecx := 0; r.edx := 0; r.es := 0;
     realintr(mouseint, r);
     free_rm_callback(mouse_seginfo);
     if (userproc_installed) then begin
        unlock_code(userproc_proc, userproc_length);
        userproc_proc := nil;
        userproc_length := 0;
        userproc_installed := false;
     end;
     unlock_data(mouse_x, sizeof(mouse_x));
     unlock_data(mouse_y, sizeof(mouse_y));
     unlock_data(mouse_b, sizeof(mouse_b));
     unlock_data(mouse_action, sizeof(mouse_action));

     unlock_data(@userproc_proc, sizeof(userproc_proc));
     unlock_data(userproc_installed, 
                 sizeof(userproc_installed));

     unlock_data(mouse_regs, sizeof(mouse_regs));
     unlock_data(mouse_seginfo, sizeof(mouse_seginfo));
     unlock_code(@callback_handler, 
                 longint(@mouse_dummy)
                  - longint(@callback_handler));
     fillchar(mouse_seginfo, sizeof(mouse_seginfo), 0);
end;


begin
     install_mouse(@textuserproc, 400);
     Writeln('Press any key to exit...');
     while (not keypressed) do begin
           { write mouse state info }
           gotoxy(1, wherey);
           write('MouseX : ', mouse_x:2, 
                 ' MouseY : ', mouse_y:2, 
                 ' Buttons : ', mouse_b:2);
     end;
     remove_mouse;
end.\end{verbatim}
\end{FPCList}}
\begin{functionl}{get\_rm\_interrupt}{getrminterrupt}
\Declaration
Function get\_rm\_interrupt (vector : byte; var intaddr :
tseginfo) : boolean;

\Description
Returns the contents of the current machine's real mode interrupt vector for
the specified interrupt.
Parameters:
\begin{description}
\item[vector:\ ] interrupt vector number. 
\item[intaddr:\ ] buffer to store real mode \var{segment:offset} address.
\end{description}
Return values: \var{True} if successful, \var{False} otherwise.
Notes: The returned address is a real mode segment address, which isn't
valid in protected mode.

\Errors
 Check the \var{int31error} variable.
\SeeAlso
 \seefl{set\_rm\_interrupt}{setrminterrupt}, 
\seefl{set\_pm\_interrupt}{setpminterrupt}, 
\seefl{get\_pm\_interrupt}{getpminterrupt} 
\end{functionl}
\begin{functionl}{get\_run\_mode}{getrunmode}
\Declaration
Function get\_run\_mode  : Word;

\Description
Returns the current mode your application runs with.
Return values: One of the constants used by this function.

\Errors
None. 
\SeeAlso
 constants returned by \seefl{get\_run\_mode}{getrunmode}  
\end{functionl}
\latex{\inputlisting{go32ex/getrunmd.pp}}
\html{\begin{FPCList}
\item[Example]
\begin{verbatim}
Program getrunmd;

uses go32;

begin
{
  depending on the detected environment,
  we simply write another message
}
case (get_run_mode) of
  rm_unknown : 
    Writeln('Unknown environment found');
  rm_raw     : 
    Writeln('You are currently running in raw mode',
            ' (without HIMEM)');
  rm_xms     : 
    Writeln('You are currently using HIMEM.SYS only');
  rm_vcpi    : 
    Writeln('VCPI server detected.',
            ' You''re using HIMEM and EMM386');
  rm_dpmi    : 
    Writeln('DPMI detected.',
            ' You''re using a DPMI host like ',
            'a windows DOS box or CWSDPMI');
end;
end.\end{verbatim}
\end{FPCList}}
\begin{functionl}{get\_segment\_base\_address}{getsegmentbaseaddress}
\Declaration
Function get\_segment\_base\_address  
(d : Word) : Longint;

\Description
 Returns the 32-bit linear base address from the descriptor table for the
specified segment.
Parameters: 
\begin{description}
\item[d:\ ] selector of the descriptor you want the base address.
\end{description}
Return values: Linear base address of specified descriptor.

\Errors
 Check the \var{int31error} variable.
\SeeAlso

\seefl{allocate\_ldt\_descriptors}{allocateldtdescriptors},
\seefl{set\_segment\_base\_address}{setsegmentbaseaddress}, 
\seefl{allocate\_ldt\_descriptors}{allocateldtdescriptors},
\seefl{set\_segment\_limit}{setsegmentlimit},
\seefl{get\_segment\_limit}{getsegmentlimit} 

\end{functionl}
For an example, see 
\seefl{allocate\_ldt\_descriptors}{allocateldtdescriptors}.
\begin{functionl}{get\_segment\_limit}{getsegmentlimit}
\Declaration
Function get\_segment\_limit (d : Word) : Longint;

\Description
Returns a descriptors segment limit.
Parameters:
\begin{description}
\item [d:\ ] selector.
\end{description}
Return value: Limit of the descriptor in bytes.

\Errors
 Returns zero if descriptor is invalid. 
\SeeAlso
\seefl{allocate\_ldt\_descriptors}{allocateldtdescriptors},
\seefl{set\_segment\_limit}{setsegmentlimit}, 
\seefl{set\_segment\_base\_address}{setsegmentbaseaddress},
\seefl{get\_segment\_base\_address}{getsegmentbaseaddress}, 

\end{functionl}
\begin{functionl}{get\_ss}{getss}
\Declaration
Function get\_ss  : Word;

\Description

Returns the ss selector.
Parameters: None.
Return values: The content of the ss segment register.

\Errors
 None.
\SeeAlso
 \seefl{get\_ds}{getds}, \seefl{get\_cs}{getcs}
\end{functionl}
\begin{functionl}{global\_dos\_alloc}{globaldosalloc}
\Declaration
Function global\_dos\_alloc (bytes : Longint) : Longint;

\Description
Allocates a block of \dos real mode memory.
Parameters: 
\begin{description}
\item [bytes:\ ] size of requested real mode memory.
\end{description}
Return values: The low word of the returned value contains the selector to
the allocated \dos memory block, the high word the corresponding real mode
segment value. The offset value is always zero.
This function allocates memory from \dos memory pool, i.e. memory below the 1
MB boundary that is controlled by \dos. Such memory blocks are typically used
to exchange data with real mode programs, TSRs, or device drivers. The
function returns both the real mode segment base address of the block and
one descriptor that can be used by protected mode applications to access the
block. This function should only used for temporary buffers to get real mode
information (e.g. interrupts that need a data structure in ES:(E)DI),
because every single block needs an unique selector. The returned selector
should only be freed by a \seefl{global\_dos\_free}{globaldosfree}  call.

\Errors
 Check the \var{int31error} variable.
\SeeAlso
 \seefl{global\_dos\_free}{globaldosfree} 
\end{functionl}
\latex{\inputlisting{go32ex/buffer.pp}}
\html{\begin{FPCList}
\item[Example]
\begin{verbatim}
Program buffer;

uses go32;

procedure dosalloc(var selector : word; var segment : word; size : longint);
var res : longint;
begin
     res := global_dos_alloc(size);
     selector := word(res);
     segment := word(res shr 16);
end;

procedure dosfree(selector : word);
begin
     global_dos_free(selector);
end;

type VBEInfoBuf = record
                Signature : array[0..3] of char; 
                Version : Word;
                reserved : array[0..505] of byte; 
     end;

var selector,       
    segment : Word; 

    r : trealregs;  
    infobuf : VBEInfoBuf;

begin
     fillchar(r, sizeof(r), 0);
     fillchar(infobuf, sizeof(VBEInfoBuf), 0);
     dosalloc(selector, segment, sizeof(VBEInfoBuf));
     if (int31error<>0) then begin
        Writeln('Error while allocating real mode memory, halting');
        halt;
     end;
     infobuf.Signature := 'VBE2';
     dosmemput(segment, 0, infobuf, sizeof(infobuf));
     r.ax := $4f00; r.es := segment;
     realintr($10, r);
     dosmemget(segment, 0, infobuf, sizeof(infobuf));
     dosfree(selector);
     if (r.ax <> $4f) then begin
        Writeln('VBE BIOS extension not available, function call failed');
        halt;
     end;
     if (infobuf.signature[0] = 'V') and (infobuf.signature[1] = 'E') and
        (infobuf.signature[2] = 'S') and (infobuf.signature[3] = 'A') then begin
        Writeln('VBE version ', hi(infobuf.version), '.', lo(infobuf.version), ' detected');
     end;
end.\end{verbatim}
\end{FPCList}}
\begin{functionl}{global\_dos\_free}{globaldosfree}
\Declaration
Function global\_dos\_free (selector :Word) : boolean;

\Description
Frees a previously allocated \dos memory block.
Parameters:
\begin{description}
\item[selector:\ ] selector to the \dos memory block.
\end{description}
Return value: \var{True} if successful, \var{False} otherwise.
Notes: The descriptor allocated for the memory block is automatically freed
and hence invalid for further use. This function should only be used for
memory allocated by \seefl{global\_dos\_alloc}{globaldosalloc}.

\Errors
 Check the \var{int31error} variable.
\SeeAlso
\seefl{global\_dos\_alloc}{globaldosalloc} 
\end{functionl}
For an example, see \seefl{global\_dos\_alloc}{globaldosalloc}.
\begin{function}{inportb}
\Declaration
Function inportb (port : Word) : byte;

\Description
Reads 1 byte from the selected I/O port.
Parameters: 
\begin{description}
\item[port:\ ] the I/O port number which is read.
\end{description}
Return values: Current I/O port value.

\Errors
 None. 
\SeeAlso
\seep{outportb}, \seef{inportw}, \seef{inportl}
\end{function}
\begin{function}{inportl}
\Declaration
Function inportl (port : Word) : Longint;

\Description

Reads 1 longint from the selected I/O port.
Parameters: 
\begin{description}
\item[port:\ ] the I/O port number which is read.
\end{description}
Return values: Current I/O port value.

\Errors
None. 
\SeeAlso
\seep{outportb}, \seef{inportb}, \seef{inportw} 
\end{function}
\begin{function}{inportw}
\Declaration
Function inportw (port : Word) : Word;

\Description

Reads 1 word from the selected I/O port.
Parameters: 
\begin{description}
\item[port:\ ] the I/O port number which is read.
\end{description}
Return values: Current I/O port value.

\Errors
 None. 
\SeeAlso
\seep{outportw} \seef{inportb}, \seef{inportl} 
\end{function}
\begin{functionl}{lock\_code}{lockcode}
\Declaration
Function lock\_code (functionaddr : pointer; size : Longint) : boolean;

\Description
Locks a memory range which is in the code segment selector.
Parameters: 
\begin{description}
\item[functionaddr:\ ] address of the function to be locked.
\item[size:\ ] size in bytes to be locked.
\end{description}
Return values: \var{True} if successful, \var{False} otherwise.

\Errors
Check the \var{int31error} variable.
\SeeAlso
 
\seefl{lock\_linear\_region}{locklinearregion},
\seefl{lock\_data}{lockdata},
\seefl{unlock\_linear\_region}{unlocklinearregion},
\seefl{unlock\_data}{unlockdata},
\seefl{unlock\_code}{unlockcode} 
\end{functionl}
For an example, see \seefl{get\_rm\_callback}{getrmcallback}.
\begin{functionl}{lock\_data}{lockdata}
\Declaration
Function lock\_data (var data; size : Longint) : boolean;

\Description
Locks a memory range which resides in the data segment selector.
Parameters:
\begin{description}
\item[data:\ ] address of data to be locked.
\item[size:\ ] length of data to be locked.
\end{description}
Return values: \var{True} if successful, \var{False} otherwise.

\Errors
 Check the \var{int31error} variable.
\SeeAlso

\seefl{lock\_linear\_region}{locklinearregion},
\seefl{lock\_code}{lockcode},
\seefl{unlock\_linear\_region}{unlocklinearregion},
\seefl{unlock\_data}{unlockdata},
\seefl{unlock\_code}{unlockcode} 
\end{functionl}
For an example, see \seefl{get\_rm\_callback}{getrmcallback}.
\begin{functionl}{lock\_linear\_region}{locklinearregion}
\Declaration
Function lock\_linear\_region (linearaddr, size : Longint) : boolean;

\Description
Locks a memory region to prevent swapping of it.
Parameters: 
\begin{description}
\item[linearaddr:\ ] the linear address of the memory are to be locked.
\item[size:\ ] size in bytes to be locked.
\end{description}
Return value: \var{True} if successful, False otherwise.

\Errors
 Check the \var{int31error} variable.
\SeeAlso

\seefl{lock\_data}{lockdata},
\seefl{lock\_code}{lockcode},
\seefl{unlock\_linear\_region}{unlocklinearregion},
\seefl{unlock\_data}{unlockdata},
\seefl{unlock\_code}{unlockcode}
\end{functionl}
\begin{procedure}{outportb}
\Declaration
Procedure outportb (port : Word; data : byte);

\Description
Sends 1 byte of data to the specified I/O port.
Parameters: 
\begin{description}
\item[port:\ ] the I/O port number to send data to.
\item[data:\ ] value sent to I/O port.
\end{description}
Return values: None.

\Errors
 None. 
\SeeAlso
\seef{inportb}, \seep{outportl}, \seep{outportw} 
\end{procedure}
\latex{\inputlisting{go32ex/outport.pp}}
\html{\begin{FPCList}
\item[Example]
\begin{verbatim}
program outport;

uses crt, go32;

begin
 { turn on speaker }
 outportb($61, $ff);
 { wait a little bit }
 delay(50);
 { turn it off again }
 outportb($61, $0);
end.\end{verbatim}
\end{FPCList}}
\begin{procedure}{outportl}
\Declaration
Procedure outportl (port : Word; data : Longint);

\Description
Sends 1 longint of data to the specified I/O port.
Parameters: 
\begin{description}
\item[port:\ ] the I/O port number to send data to.
\item[data:\ ] value sent to I/O port.
\end{description}
Return values: None.

\Errors
None. 
\SeeAlso
\seef{inportl}, \seep{outportw}, \seep{outportb}
\end{procedure}
For an example, see \seep{outportb}.
\begin{procedure}{outportw}
\Declaration
Procedure outportw (port : Word; data : Word);

\Description
Sends 1 word of data to the specified I/O port.
Parameters: 
\begin{description}
\item[port:\ ] the I/O port number to send data to.
\item[data:\ ] value sent to I/O port.
\end{description}
Return values: None.

\Errors
 None. 
\SeeAlso
\seef{inportw}, \seep{outportl}, \seep{outportb}
\end{procedure}
For an example, see \seep{outportb}.
\begin{function}{realintr}
\Declaration
Function realintr (intnr: Word; var regs : trealregs) :  boolean;

\Description
Simulates an interrupt in real mode.
Parameters:
\begin{description}
\item[intnr:\ ] interrupt number to issue in real mode.
\item[regs:\ ] registers data structure.
\end{description}
Return values: The supplied registers data structure contains the values
that were returned by the real mode interrupt. \var{True} if successful, \var{False} if
not.
Notes: The function transfers control to the address specified by the real
mode interrupt vector of intnr. The real mode handler must return by
executing an IRET.

\Errors
 Check the \var{int31error} variable.
\SeeAlso

\end{function}
\latex{\inputlisting{go32ex/flags.pp}}
\html{\begin{FPCList}
\item[Example]
\begin{verbatim}
Program flags;

uses go32;

var r : trealregs;

begin
     r.ax := $5300;
     r.bx := 0;
     realintr($15, r);
     { check if carry clear and write a suited message }
     if ((r.flags and carryflag)=0) then begin
        Writeln('APM v',(r.ah and $f), 
                '.', (r.al shr 4), (r.al and $f), 
                ' detected');
     end else
         Writeln('APM not present');
end.\end{verbatim}
\end{FPCList}}
\begin{procedurel}{seg\_fillchar}{segfillchar}
\Declaration
Procedure seg\_fillchar (seg : Word; ofs : Longint; count : Longint; c : char);

\Description

Sets a memory area to a specific value.
Parameters:
\begin{description}
\item[seg:\ ] selector to memory area.
\item[ofs:\ ] offset to memory.
\item[count:\ ] number of bytes to set.
\item[c:\ ] byte data which is set.
\end{description}
Return values: None.
Notes: No range check is done in any way.

\Errors
 None. 
\SeeAlso
\seepl{seg\_move}{segmove},
\seepl{seg\_fillword}{segfillword},
\seepl{dosmemfillchar}{dosmemfillchar},
\seepl{dosmemfillword}{dosmemfillword},
\seepl{dosmemget}{dosmemget},
\seepl{dosmemput}{dosmemput},
\seepl{dosmemmove}{dosmemmove} 
\end{procedurel}
\latex{\inputlisting{go32ex/vgasel.pp}}
\html{\begin{FPCList}
\item[Example]
\begin{verbatim}
Program svgasel;

uses go32;

var vgasel : Word;
    r : trealregs;

begin
  r.eax := $13; realintr($10, r);
  vgasel := segment_to_descriptor($A000);
  { simply fill the screen memory with color 15 }
  seg_fillchar(vgasel, 0, 64000, #15);
  readln;
 { back to text mode }
  r.eax := $3; 
  realintr($10, r);
end.\end{verbatim}
\end{FPCList}}
\begin{procedurel}{seg\_fillword}{segfillword}
\Declaration
Procedure seg\_fillword (seg : Word; ofs : Longint; count : Longint; w :Word);

\Description

Sets a memory area to a specific value.
Parameters:
\begin{description}
\item[seg:\ ] selector to memory area.
\item[ofs:\ ] offset to memory.
\item[count:\ ] number of words to set.
\item[w:\ ] word data which is set.
\end{description}
Return values: None.
Notes: No range check is done in any way.

\Errors
None. 
\SeeAlso
 
\seepl{seg\_move}{segmove},
\seepl{seg\_fillchar}{segfillchar}, 
\seepl{dosmemfillchar}{dosmemfillchar}, 
\seepl{dosmemfillword}{dosmemfillword},
\seepl{dosmemget}{dosmemget},
\seepl{dosmemput}{dosmemput},
\seepl{dosmemmove}{dosmemmove} 
\end{procedurel}
For an example, see 
\seefl{allocate\_ldt\_descriptors}{allocateldtdescriptors}.
\begin{functionl}{segment\_to\_descriptor}{segmenttodescriptor}
\Declaration
Function segment\_to\_descriptor (seg : Word) : Word;

\Description

Maps a real mode segment (paragraph) address onto an descriptor that can be
used by a protected mode program to access the same memory.
Parameters: 
\begin{description}
\item [seg:\ ] the real mode segment you want the descriptor to.
\end{description}
Return values: Descriptor to real mode segment address.
Notes: The returned descriptors limit will be set to 64 kB. Multiple calls
to this function with the same segment address will return the same
selector. Descriptors created by this function can never be modified or
freed. Programs which need to examine various real mode addresses using the
same selector should use the function 
\seefl{allocate\_ldt\_descriptors}{allocateldtdescriptors} and change
the base address as necessary.

\Errors
 Check the \var{int31error} variable. 
\SeeAlso
\seefl{allocate\_ldt\_descriptors}{allocateldtdescriptors},
\seefl{free\_ldt\_descriptor}{freeldtdescriptor},
\seefl{set\_segment\_base\_address}{setsegmentbaseaddress} 

\end{functionl}
For an example, see \seepl{seg\_fillchar}{segfillchar}.
\begin{procedurel}{seg\_move}{segmove}
\Declaration
Procedure seg\_move (sseg : Word; source : Longint; dseg : Word; dest :
Longint; count : Longint);

\Description
Copies data between two memory locations.
Parameters: 
\begin{description}
\item[sseg:\ ] source selector. 
\item[source:\ ] source offset. 
\item[dseg:\ ] destination selector.
\item[dest:\ ] destination offset.
\item[count:\ ] size in bytes to copy.
\end{description}
Return values: None.
Notes: Overlapping is only checked if the source selector is equal to the
destination selector. No range check is done.

\Errors
 None.
\SeeAlso
 
\seepl{seg\_fillchar}{segfillchar},
\seepl{seg\_fillword}{segfillword},
\seepl{dosmemfillchar}{dosmemfillchar},
\seepl{dosmemfillword}{dosmemfillword},
\seepl{dosmemget}{dosmemget},
\seepl{dosmemput}{dosmemput},
\seepl{dosmemmove}{dosmemmove} 
\end{procedurel}
For an example, see 
\seefl{allocate\_ldt\_descriptors}{allocateldtdescriptors}.
\begin{functionl}{set\_descriptor\_access\_rights}{setdescriptoraccessrights}
\Declaration
Function set\_descriptor\_access\_rights (d : Word; w : Word) : Longint;

\Description

Sets the access rights of a descriptor.
Parameters: 
\begin{description}
\item[d:\ ] selector.
\item[w:\ ] new descriptor access rights.
\end{description}
Return values: This function doesn't return anything useful.

\Errors
 Check the \var{int31error} variable.
\SeeAlso

\seefl{get\_descriptor\_access\_rights}{getdescriptoraccessrights} 
\end{functionl}
\begin{functionl}{set\_pm\_interrupt}{setpminterrupt}
\Declaration
Function set\_pm\_interrupt (vector : byte; const intaddr : tseginfo) : boolean;

\Description
Sets the address of the protected mode handler for an interrupt.
Parameters: 
\begin{description}
\item[vector:\ ] number of protected mode interrupt to set.
\item[intaddr:\ ] selector:offset address to the interrupt vector.
\end{description}
Return values: \var{True} if successful, \var{False} otherwise.
Notes: The address supplied must be a valid \var{selector:offset} 
protected mode address.

\Errors
 Check the \var{int31error} variable.
\SeeAlso
\seefl{get\_pm\_interrupt}{getpminterrupt}, 
\seefl{set\_rm\_interrupt}{setrminterrupt},
\seefl{get\_rm\_interrupt}{getrminterrupt} 
\end{functionl}
\latex{\inputlisting{go32ex/intpm.pp}}
\html{\input{go32ex/intpm.tex}}
\begin{functionl}{set\_rm\_interrupt}{setrminterrupt}
\Declaration
Function set\_rm\_interrupt (vector : byte; const intaddr :
tseginfo) : boolean;

\Description
Sets a real mode interrupt handler.
Parameters:
\begin{description}
\item[vector:\ ] the interrupt vector number to set.
\item[intaddr:\ ] address of new interrupt vector.
\end{description}
Return values: \var{True} if successful, otherwise \var{False}.
Notes: The address supplied MUST be a real mode segment address, not a
\var{selector:offset} address. So the interrupt handler must either reside in \dos
memory (below 1 Mb boundary) or the application must allocate a real mode
callback address with \seefl{get\_rm\_callback}{getrmcallback}.

\Errors
 Check the \var{int31error} variable.
\SeeAlso
 
\seefl{get\_rm\_interrupt}{getrminterrupt}, 
\seefl{set\_pm\_interrupt}{setpminterrupt}, \seefl{get\_pm\_interrupt}{getpminterrupt}, 
\seefl{get\_rm\_callback}{getrmcallback} 
\end{functionl}
\begin{functionl}{set\_segment\_base\_address}{setsegmentbaseaddress}
\Declaration
Function set\_segment\_base\_address (d : Word; s : Longint) : boolean;

\Description
Sets the 32-bit linear base address of a descriptor.
Parameters: 
\begin{description}
\item[d:\ ] selector.
\item[s:\ ] new base address of the descriptor.
\end{description}

\Errors
 Check the \var{int31error} variable.
\SeeAlso

\seefl{allocate\_ldt\_descriptors}{allocateldtdescriptors},
\seefl{get\_segment\_base\_address}{getsegmentbaseaddress}, 
\seefl{allocate\_ldt\_descriptors}{allocateldtdescriptors}, 
\seefl{set\_segment\_limit}{setsegmentlimit},
\seefl{get\_segment\_base\_address}{getsegmentbaseaddress},
\seefl{get\_segment\_limit}{getsegmentlimit} 

\end{functionl}
\begin{functionl}{set\_segment\_limit}{setsegmentlimit}
\Declaration
Function set\_segment\_limit (d : Word; s : Longint) : boolean;

\Description
Sets the limit of a descriptor.
Parameters: 
\begin{description}
\item[d:\ ] selector.
\item[s:\ ] new limit of the descriptor.
\end{description}
Return values: Returns \var{True} if successful, else \var{False}.
Notes: The new limit specified must be the byte length of the segment - 1.
Segment limits bigger than or equal to 1MB must be page aligned, they must
have the lower 12 bits set.

\Errors
 Check the \var{int31error} variable.
\SeeAlso
\seefl{allocate\_ldt\_descriptors}{allocateldtdescriptors},
\seefl{set\_segment\_base\_address}{setsegmentbaseaddress},
\seefl{get\_segment\_limit}{getsegmentlimit}, 
\seefl{set\_segment\_limit}{setsegmentlimit} 

\end{functionl}
For an example, see 
\seefl{allocate\_ldt\_descriptors}{allocateldtdescriptors}.
\begin{functionl}{tb\_size}{tbsize}
\Declaration
Function tb\_size  : Longint;

\Description
Returns the size of the pre-allocated \dos memory buffer.
Parameters: None.
Return values: The size of the pre-allocated \dos memory buffer.
Notes:
This block always seems to be 16k in size, but don't rely on this.

\Errors
None.
\SeeAlso
\seefl{transfer\_buffer}{transferbuffer}, \seep{copyfromdos}
\seep{copytodos}
\end{functionl}

\begin{functionl}{transfer\_buffer}{transferbuffer}
\Declaration
Function transfer\_buffer : Longint;
\Description
\var{transfer\_buffer} returns the offset of the transfer buffer.
\Errors
None.
\SeeAlso
\seefl{tb\_size}{tbsize}
\end{functionl}

\begin{functionl}{unlock\_code}{unlockcode}
\Declaration
Function unlock\_code (functionaddr : pointer; size : Longint) : boolean;

\Description
Unlocks a memory range which resides in the code segment selector.
Parameters:
\begin{description}
\item[functionaddr:\ ] address of function to be unlocked. 
\item[size:\ ] size bytes to be unlocked.
\end{description}
Return value: \var{True} if successful, \var{False} otherwise.

\Errors
 Check the \var{int31error} variable.
\SeeAlso
\seefl{unlock\_linear\_region}{unlocklinearregion},
 \seefl{unlock\_data}{unlockdata},
\seefl{lock\_linear\_region}{locklinearregion},
\seefl{lock\_data}{lockdata},
\seefl{lock\_code}{lockcode} 
\end{functionl}
For an example, see \seefl{get\_rm\_callback}{getrmcallback}.
\begin{functionl}{unlock\_data}{unlockdata}
\Declaration
Function unlock\_data (var data; size : Longint) : boolean;

\Description
Unlocks a memory range which resides in the data segment selector.
Paramters:
\begin{description}
\item[data:\ ] address of memory to be unlocked. 
\item[size:\ ] size bytes to be unlocked.
\end{description}
Return values: \var{True} if successful, \var{False} otherwise.

\Errors
 Check the \var{int31error} variable.
\SeeAlso
\seefl{unlock\_linear\_region}{unlocklinearregion},
\seefl{unlock\_code}{unlockcode},
\seefl{lock\_linear\_region}{locklinearregion},
\seefl{lock\_data}{lockdata},
\seefl{lock\_code}{lockcode} 
\end{functionl}
For an example, see \seefl{get\_rm\_callback}{getrmcallback}.
\begin{functionl}{unlock\_linear\_region}{unlocklinearregion}
\Declaration
Function unlock\_linear\_region (linearaddr, size : Longint) : boolean;

\Description
Unlocks a previously locked linear region range to allow it to be swapped
out again if needed.
Parameters:
\begin{description}
\item[linearaddr:\ ] linear address of the memory to be unlocked. 
\item[size:\ ] size bytes to be unlocked.
\end{description}
Return values: \var{True} if successful, \var{False} otherwise.

\Errors
 Check the \var{int31error} variable.
\SeeAlso

\seefl{unlock\_data}{unlockdata},
\seefl{unlock\_code}{unlockcode},
\seefl{lock\_linear\_region}{locklinearregion},
\seefl{lock\_data}{lockdata},
\seefl{lock\_code}{lockcode}
\end{functionl}

