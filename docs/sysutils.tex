%
%   $Id$
%   This file is part of the FPC documentation.
%   Copyright (C) 1999, by Michael Van Canneyt
%
%   The FPC documentation is free text; you can redistribute it and/or
%   modify it under the terms of the GNU Library General Public License as
%   published by the Free Software Foundation; either version 2 of the
%   License, or (at your option) any later version.
%
%   The FPC Documentation is distributed in the hope that it will be useful,
%   but WITHOUT ANY WARRANTY; without even the implied warranty of
%   MERCHANTABILITY or FITNESS FOR A PARTICULAR PURPOSE.  See the GNU
%   Library General Public License for more details.
%
%   You should have received a copy of the GNU Library General Public
%   License along with the FPC documentation; see the file COPYING.LIB.  If not,
%   write to the Free Software Foundation, Inc., 59 Temple Place - Suite 330,
%   Boston, MA 02111-1307, USA. 
%
\chapter{The SYSUTILS unit.}

This chapter describes the \file{sysutils} unit. The \var{sysutils} unit 
was largely written by Gertjan Schouten, and completed by michael Van Canneyt. 
It aims to be compatible to the Delphi sysutils unit, but in contrast with 
the latter, it is designed to work on multiple platforms.

This chapter starts out with a definition of all types and constants 
that are defined, followed by a complete explanation of each function.

\section{Constants and types}

The following general-purpose constants are defined:
\begin{verbatim}
const
   SecsPerDay = 24 * 60 * 60; // Seconds and milliseconds per day
   MSecsPerDay = SecsPerDay * 1000;
   DateDelta = 693594;        // Days between 1/1/0001 and 12/31/1899
   Eoln = #10;
\end{verbatim}
The following types are used frequently in date and time functions.
They are the same on all platforms.
\begin{verbatim}
type
   TSystemTime = record
      Year, Month, Day: word;
      Hour, Minute, Second, MilliSecond: word;
   end ;

   TDateTime = double;

   TTimeStamp = record
      Time: integer;   { Number of milliseconds since midnight }
      Date: integer;   { One plus number of days since 1/1/0001 }
   end ;
\end{verbatim}
The following type is used in the \seef{FindFirst},\seef{FindNext} 
and \seepl{FindClose}{FindCloseSys} functions. The \var{win32} version differs from 
the other versions. If code is to be portable, that part  shouldn't 
be used.
\begin{verbatim}
Type 
  THandle = Longint; 
  TSearchRec = Record
    Time,Size, Attr : Longint;
    Name : TFileName;
    ExcludeAttr : Longint;
    FindHandle : THandle;
    {$ifdef Win32}
    FindData : TWin32FindData;        
    {$endif}
    end;
\end{verbatim}
The following constants are file-attributes that need to be matched in the 
findfirst call.
\begin{verbatim}
Const 
  faReadOnly  = $00000001; 
  faHidden    = $00000002;
  faSysFile   = $00000004;
  faVolumeId  = $00000008;
  faDirectory = $00000010;
  faArchive   = $00000020;
  faAnyFile   = $0000003f;
\end{verbatim}
The following constants can be used in the \seef{FileOpen} call.
\begin{verbatim}
Const
  fmOpenRead       = $0000;
  fmOpenWrite      = $0001;
  fmOpenReadWrite  = $0002;
\end{verbatim}
The following constants can be used in the \seef{FileSeek} call.
\begin{verbatim}
Const
  fsFromBeginning = 0;
  fsFromCurrent   = 1;
  fsFromEnd       = 2;

\end{verbatim}
The following variables are used in the case translation routines.
\begin{verbatim}
type
   TCaseTranslationTable = array[0..255] of char;
var
   UpperCaseTable: TCaseTranslationTable;
   LowerCaseTable: TCaseTranslationTable;
\end{verbatim}
The initialization code of the \file{sysutils} unit fills these 
tables with the appropriate values. For the win32 and go32v2
versions, this information is obtained from the operating system.

The following constants control the formatting of dates.
For the Win32 version of the \file{sysutils} unit, these 
constants are set according to the internationalization 
settings of Windows by the initialization code of the unit.
\begin{verbatim}
Const 
   DateSeparator: char = '-';
   ShortDateFormat: string = 'd/m/y';
   LongDateFormat: string = 'dd" "mmmm" "yyyy';
   ShortMonthNames: array[1..12] of string[128] =
     ('Jan','Feb','Mar','Apr','May','Jun',
      'Jul','Aug','Sep','Oct','Nov','Dec');
   LongMonthNames: array[1..12] of string[128] =
     ('January','February','March','April',
      'May','June','July','August',
      'September','October','November','December');
   ShortDayNames: array[1..7] of string[128] =
     ('Sun','Mon','Tue','Wed','Thu','Fri','Sat');
   LongDayNames: array[1..7] of string[128] =
     ('Sunday','Monday','Tuesday','Wednesday',
       'Thursday','Friday','Saturday');
\end{verbatim}  

The following constants control the formatting of times.
For the Win32 version of the \file{sysutils} unit, these 
constants are set according to the internationalization 
settings of Windows by the initialization code of the unit.
\begin{verbatim}
Const
   ShortTimeFormat: string = 'hh:nn';
   LongTimeFormat: string = 'hh:nn:ss';
   TimeSeparator: char = ':';
   TimeAMString: string[7] = 'AM';
   TimePMString: string[7] = 'PM';
\end{verbatim}

The following constants control the formatting of currencies 
and numbers. For the Win32 version of the \file{sysutils} unit, 
these  constants are set according to the internationalization 
settings of Windows by the initialization code of the unit.
\begin{verbatim}
Const
  DecimalSeparator : Char = '.';
  ThousandSeparator : Char = ',';
  CurrencyDecimals : Byte = 2;
  CurrencyString : String[7] = '$';
  { Format to use when formatting currency :
    0 = $1        1 = 1$         2 = $ 1      3 = 1 $
    4 = Currency string replaces decimal indicator. 
        e.g. 1$50 
   }
  CurrencyFormat : Byte = 1;
  { Same as above, only for negative currencies:
    0 = ($1)
    1 = -$1
    2 = $-1
    3 = $1-
    4 = (1$)
    5 = -1$
    6 = 1-$
    7 = 1$-
    8 = -1 $
    9 = -$ 1
    10 = $ 1-
   }
  NegCurrFormat : Byte = 5;
\end{verbatim}
The following types are used in various string functions.
\begin{verbatim}
type
   PString = ^String;
   TFloatFormat = (ffGeneral, ffExponent, ffFixed, ffNumber, ffCurrency);
\end{verbatim}
The following constants are used in the file name handling routines. Do not
use a slash of backslash character directly as a path separator; instead 
use the \var{OsDirSeparator} character.
\begin{verbatim}
Const
  DirSeparators : set of char = ['/','\'];  
{$ifdef Linux}
  OSDirSeparator = '/';
{$else}
  OsDirSeparator = '\';
{$endif}
\end{verbatim}

\section{Date and time functions}

\subsection{Date and time formatting characters}
\label{se:formatchars}

Various date and time formatting routines accept a format string.
to format the date and or time. The following characters can be used
to control the date and time formatting:
\begin{description}
\item[c] : shortdateformat + ' ' + shorttimeformat
\item[d] : day of month
\item[dd] : day of month (leading zero)
\item[ddd] : day of week (abbreviation)
\item[dddd] : day of week (full)
\item[ddddd] : shortdateformat
\item[dddddd] : longdateformat
\item[m] : month
\item[mm] : month (leading zero)
\item[mmm] : month (abbreviation)
\item[mmmm] : month (full)
\item[y] : year (four digits)
\item[yy] : year (two digits)
\item[yyyy] : year (with century)
\item[h] : hour
\item[hh] : hour (leading zero)
\item[n] : minute
\item[nn] : minute (leading zero)
\item[s] : second
\item[ss] : second (leading zero)
\item[t] : shorttimeformat
\item[tt] : longtimeformat
\item[am/pm] : use 12 hour clock and display am and pm accordingly
\item[a/p] : use 12 hour clock and display a and p accordingly
\item[/] : insert date seperator
\item[:] : insert time seperator
\item["xx"] : literal text
\item['xx'] : literal text
\end{description}

\begin{type}{TDateTime}
\Declaration
  TDateTime = Double;
\Description
Many functions return or require a \var{TDateTime} type, which contains
a date and time in encoded form. The date and time are converted to a double
as follows:
\end{type}

\begin{function}{Date}
\Declaration
Function Date: TDateTime;
\Description
\var{Date} returns the current date in \var{TDateTime} format. 
For more information about the \var{TDateTime} type, see \seetype{TDateTime}.
\Errors
None.
\SeeAlso
\seef{Time},\seef{Now}, \seetype{TDateTime}.
\end{function}

\latex{\lstinputlisting{sysutex/ex1.pp}}
\html{\input{sysutex/ex1.tex}}

\begin{function}{DateTimeToFileDate}
\Declaration
Function DateTimeToFileDate(DateTime : TDateTime) : Longint;
\Description
\var{DateTimeToFileDate} function converts a date/time indication in
\var{TDateTime} format to a filedate function, such as returned for 
instance by the \seef{FileAge} function.
\Errors
None.
\SeeAlso
\seef{Time}, \seef{Date}, \seef{FileDateToDateTime},
\seep{DateTimeToSystemTime}, \seef{DateTimeToTimeStamp}
\end{function}

\latex{\lstinputlisting{sysutex/ex2.pp}}
\html{\input{sysutex/ex2.tex}}
 
\begin{function}{DateTimeToStr}
\Declaration
Function DateTimeToStr(DateTime: TDateTime): string;
\Description
\var{DateTimeToStr} returns a string representation of 
\var{DateTime} using the formatting specified in
\var{ShortDateTimeFormat}. It corresponds to a call to 
\var{FormatDateTime('c',DateTime)} (see \sees{formatchars}).
\Errors
None.
\SeeAlso
\seef{FormatDateTime}, \seetype{TDateTime}.
\end{function}

\latex{\lstinputlisting{sysutex/ex3.pp}}
\html{\input{sysutex/ex3.tex}}
 
\begin{procedure}{DateTimeToString}
\Declaration
Procedure DateTimeToString(var Result: string; const FormatStr: string; const DateTime: TDateTime);
\Description
\var{DateTimeToString} returns in \var{Result} a string representation of 
\var{DateTime} using the formatting specified in \var{FormatStr}. 

for a list of characters that can be used in the \var{FormatStr} formatting
string, see \sees{formatchars}.
\Errors
In case a wrong formatting character is found, an \var{EConvertError} is
raised.
\SeeAlso
\seef{FormatDateTime}, \sees{formatchars}.
\end{procedure}

\latex{\lstinputlisting{sysutex/ex4.pp}}
\html{\input{sysutex/ex4.tex}}
 
\begin{procedure}{DateTimeToSystemTime}
\Declaration
Procedure DateTimeToSystemTime(DateTime: TDateTime; var SystemTime: TSystemTime);
\Description
\var{DateTimeToSystemTime} converts a date/time pair in \var{DateTime}, with
\var{TDateTime} format to a system time \var{SystemTime}.
\Errors
None.
\SeeAlso
\seef{DateTimeToFileDate}, \seef{SystemTimeToDateTime},
\seef{DateTimeToTimeStamp}
\end{procedure}

\latex{\lstinputlisting{sysutex/ex5.pp}}
\html{\input{sysutex/ex5.tex}}

\begin{function}{DateTimeToTimeStamp}
\Declaration
Function DateTimeToTimeStamp(DateTime: TDateTime): TTimeStamp;
\Description
\var{DateTimeToSystemTime} converts a date/time pair in \var{DateTime}, with
\var{TDateTime} format to a \var{TTimeStamp} format.
\Errors
None.
\SeeAlso
\seef{DateTimeToFileDate}, \seef{SystemTimeToDateTime},
\seep{DateTimeToSystemTime}
\end{function}

\latex{\lstinputlisting{sysutex/ex6.pp}}
\html{\input{sysutex/ex6.tex}}

\begin{function}{DateToStr}
\Declaration
Function DateToStr(Date: TDateTime): string;
\Description
\var{DateToStr} converts \var{Date} to a string representation. It uses
\var{ShortDateFormat} as it's formatting string. It is hence completely
equivalent to a \var{FormatDateTime('ddddd', Date)}.
\Errors
None.
\SeeAlso
\seef{TimeToStr}, \seef{DateTimeToStr}, \seef{FormatDateTime},
\seef{StrToDate}
\end{function}


\latex{\lstinputlisting{sysutex/ex7.pp}}
\html{\input{sysutex/ex7.tex}}
 
\begin{function}{DayOfWeek}
\Declaration
Function DayOfWeek(DateTime: TDateTime): integer;
\Description
\var{DayOfWeek} returns the day of the week from \var{DateTime}.
\var{Sunday} is counted as day 1, \var{Saturday} is counted as 
day 7. The result of \var{DayOfWeek} can serve as an index to 
the \var{LongDayNames} constant array, to retrieve the name of 
the day.
\Errors
None.
\SeeAlso
\seef{Date}, \seef{DateToStr}
\end{function}


\latex{\lstinputlisting{sysutex/ex8.pp}}
\html{\input{sysutex/ex8.tex}}

\begin{procedure}{DecodeDate}
\Declaration
Procedure DecodeDate(Date: TDateTime; var Year, Month, Day: word);
\Description
\var{DecodeDate} decodes the Year, Month and Day stored in \var{Date},
and returns them in the \var{Year}, \var{Month} and \var{Day} variables.
\Errors
None.
\SeeAlso
\seef{EncodeDate}, \seep{DecodeTime}.
\end{procedure}

\latex{\lstinputlisting{sysutex/ex9.pp}}
\html{\input{sysutex/ex9.tex}}

 
\begin{procedure}{DecodeTime}
\Declaration
Procedure DecodeTime(Time: TDateTime; var Hour, Minute, Second, MilliSecond: word);
\Description
\var{DecodeDate} decodes the hours, minutes, second and milliseconds stored 
in \var{Time}, and returns them in the \var{Hour}, \var{Minute} and
\var{Second} and \var{MilliSecond} variables.
\Errors
None.
\SeeAlso
\seef{EncodeTime}, \seep{DecodeDate}.
\end{procedure}

\latex{\lstinputlisting{sysutex/ex10.pp}}
\html{\input{sysutex/ex10.tex}}
 
\begin{function}{EncodeDate}
\Declaration
Function EncodeDate(Year, Month, Day :word): TDateTime;
\Description
\var{EncodeDate} encodes the \var{Year}, \var{Month} and \var{Day} variables to 
a date in \var{TDateTime} format. It does the opposite of the
\seep{DecodeDate} procedure.

The parameters must lie withing valid ranges (boundaries included):
\begin{description}
\item[Year] must be between 1 and 9999.
\item[Month] must be within the range 1-12.
\item[Day] msut be between 1 and 31.
\end{description}
\Errors
In case one of the parameters is out of it's valid range, 0 is returned.
\SeeAlso
\seef{EncodeTime}, \seep{DecodeDate}.
\end{function}

\latex{\lstinputlisting{sysutex/ex11.pp}}
\html{\input{sysutex/ex11.tex}}

\begin{function}{EncodeTime}
\Declaration
Function EncodeTime(Hour, Minute, Second, MilliSecond:word): TDateTime;
\Description
\var{EncodeTime} encodes the \var{Hour}, \var{Minute}, \var{Second},
\var{MilliSecond} variables to a \var{TDateTime} format result.
It does the opposite of the \seep{DecodeTime} procedure.

The parameters must have a valid range (boundaries included):
\begin{description}
\item[Hour] must be between 0 and 23.
\item[Minute,second] must both be between 0 and 59.
\item[Millisecond] must be between 0 and 999.
\end{description}
\Errors
In case one of the parameters is outside of it's valid range, 0 is returned.
\SeeAlso
\seef{EncodeDate}, \seep{DecodeTime}.
\end{function}

\latex{\lstinputlisting{sysutex/ex12.pp}}
\html{\input{sysutex/ex12.tex}}

 
\begin{function}{FileDateToDateTime}
\Declaration
Function FileDateToDateTime(Filedate : Longint) : TDateTime;
\Description
\var{FileDateToDateTime} converts the date/time encoded in \var{filedate}
to a \var{TDateTime} encoded form. It can be used to convert date/time values 
returned by the \seef{FileAge} or \seef{FindFirst}/\seef{FindNext} 
functions to \var{TDateTime} form.
\Errors
None.
\SeeAlso
\seef{DateTimeToFileDate}
\end{function}

\latex{\lstinputlisting{sysutex/ex13.pp}}
\html{\input{sysutex/ex13.tex}}
 
\begin{function}{FormatDateTime}
\Declaration
Function FormatDateTime(FormatStr: string; DateTime: TDateTime):string;
\Description
\var{FormatDateTime} formats the date and time encoded in \var{DateTime}
according to the formatting given in \var{FormatStr}. The complete list 
of formatting characters can be found in \sees{formatchars}.
\Errors
On error (such as an invalid character in the formatting string), and
\var{EConvertError} exception is raised.
\SeeAlso
\seef{DateTimeToStr}, \seef{DateToStr}, \seef{TimeToStr},
\seef{StrToDateTime}
\end{function}

\latex{\lstinputlisting{sysutex/ex14.pp}}
\html{\input{sysutex/ex14.tex}}

 
\begin{function}{IncMonth}
\Declaration
Function IncMonth(const DateTime: TDateTime; NumberOfMonths: integer): TDateTime;
\Description
\var{IncMonth} increases the month number in \var{DateTime} with
\var{NumberOfMonths}. It wraps the result as to get a month between 1 and
12, and updates the year accordingly. \var{NumberOfMonths} can be negative,
and can be larger than 12 (in absolute value).
\Errors
None.
\SeeAlso
\seef{Date}, \seef{Time}, \seef{Now}
\end{function}


\latex{\lstinputlisting{sysutex/ex15.pp}}
\html{\input{sysutex/ex15.tex}}
 
\begin{function}{IsLeapYear}
\Declaration
Function IsLeapYear(Year: Word): boolean;
\Description
\var{IsLeapYear} returns \var{True} if \var{Year} is a leap year,
\var{False} otherwise.
\Errors
None.
\SeeAlso
\seef{IncMonth}, \seef{Date}
\end{function}

\latex{\lstinputlisting{sysutex/ex16.pp}}
\html{\input{sysutex/ex16.tex}}
 
\begin{function}{MSecsToTimeStamp}
\Declaration
Function MSecsToTimeStamp(MSecs: Comp): TTimeStamp;
\Description
\var{MSecsTiTimeStamp} converts the given number of milliseconds to
a \var{TTimeStamp} date/time notation.

Use \var{TTimeStamp} variables if you need to keep very precise track of
time.
\Errors
None.
\SeeAlso
\seef{TimeStampToMSecs}, \seef{DateTimeToTimeStamp}, 
\end{function}

\latex{\lstinputlisting{sysutex/ex17.pp}}
\html{\input{sysutex/ex17.tex}}
 
\begin{function}{Now}
\Declaration
Function Now: TDateTime;
\Description
\var{Now} returns the current date and time. It is equivalent to 
\var{Date+Time}.
\Errors
None.
\SeeAlso
\seef{Date}, \seef{Time}
\end{function}

\latex{\lstinputlisting{sysutex/ex18.pp}}
\html{\input{sysutex/ex18.tex}}
 
\begin{function}{StrToDate}
\Declaration
Function StrToDate(const S: string): TDateTime;
\Description
\var{StrToDate} converts the string \var{S} to a \var{TDateTime} date 
value. The Date must consist of 1 to three digits, separated by the 
\var{DateSeparator} character. If two numbers are given, they
are supposed to form the day and month of the current year. If only 
one number is given, it is supposed to represent the day of the 
current month. (This is \em{not} supported in Delphi)

The order of the digits (y/m/d, m/d/y, d/m/y) is determined from the 
\var{ShortDateFormat} variable.
\Errors
On error (e.g. an invalid date or invalid character), 
an \var{EConvertError} exception is raised.
\SeeAlso
\seef{StrToTime}, \seef{DateToStr}n \seef{TimeToStr}.
\end{function}

\latex{\lstinputlisting{sysutex/ex19.pp}}
\html{\input{sysutex/ex19.tex}}

\begin{function}{StrToDateTime}
\Declaration
Function StrToDateTime(const S: string): TDateTime;
\Description
\var{StrToDateTime} converts the string \var{S} to a \var{TDateTime} date 
and time value. The Date must consist of 1 to three digits, separated by the 
\var{DateSeparator} character. If two numbers are given, they
are supposed to form the day and month of the current year. If only 
one number is given, it is supposed to represent the day of the 
current month. (This is \em{not} supported in Delphi)

The order of the digits (y/m/d, m/d/y, d/m/y) is determined from the 
\var{ShortDateFormat} variable.
\Errors
On error (e.g. an invalid date or invalid character), 
an \var{EConvertError} exception is raised.
\SeeAlso
\seef{StrToDate}, \seef{StrToTime}, \seef{DateTimeToStr}
\end{function}

\latex{\lstinputlisting{sysutex/ex20.pp}}
\html{\input{sysutex/ex20.tex}}
 
\begin{function}{StrToTime}
\Declaration
Function StrToTime(const S: string): TDateTime;
\Description
\var{StrToTime} converts the string \var{S} to a \var{TDateTime} time 
value. The time must consist of 1 to 4 digits, separated by the 
\var{TimeSeparator} character. If two numbers are given, they
are supposed to form the hour and minutes. 
\Errors
On error (e.g. an invalid date or invalid character), 
an \var{EConvertError} exception is raised.
\SeeAlso
\seef{StrToDate}, \seef{StrToDateTime}, \seef{TimeToStr}
\end{function}

\latex{\lstinputlisting{sysutex/ex21.pp}}
\html{\input{sysutex/ex21.tex}}

\begin{function}{SystemTimeToDateTime}
\Declaration
Function SystemTimeToDateTime(const SystemTime: TSystemTime): TDateTime;
\Description
\var{SystemTimeToDateTime} converts a \var{TSystemTime} record to a
\var{TDateTime} style date/time indication.
\Errors
None.
\SeeAlso
\seep{DateTimeToSystemTime}
\end{function}

\latex{\lstinputlisting{sysutex/ex22.pp}}
\html{\input{sysutex/ex22.tex}}

\begin{function}{Time}
\Declaration
Function Time: TDateTime;
\Description
\var{Time} returns the current time in \var{TDateTime} format. The date
part of the \var{TDateTimeValue} is set to zero.
\Errors
None.
\SeeAlso
\seef{Now}, \seef{Date}
\end{function}


\latex{\lstinputlisting{sysutex/ex23.pp}}
\html{\input{sysutex/ex23.tex}}
 
\begin{function}{TimeStampToDateTime}
\Declaration
Function TimeStampToDateTime(const TimeStamp: TTimeStamp): TDateTime;
\Description
\var{TimeStampToDateTime} converts \var{TimeStamp} to a \var{TDateTime}
format variable. It is the inverse operation of \seef{DateTimeToTimeStamp}.
\Errors
None.
\SeeAlso
\seef{DateTimeToTimeStamp}, \seef{TimeStampToMSecs}
\end{function}

\latex{\lstinputlisting{sysutex/ex24.pp}}
\html{\input{sysutex/ex24.tex}}

\begin{function}{TimeStampToMSecs}
\Declaration
Function TimeStampToMSecs(const TimeStamp: TTimeStamp): comp;
\Description
\var{TimeStampToMSecs} converts {TimeStamp} to the number of seconds
since \var{1/1/0001}.

Use \var{TTimeStamp} variables if you need to keep very precise track of
time.
\Errors
None.
\SeeAlso
\seef{MSecsToTimeStamp}, \seef{TimeStampToDateTime}
\end{function}

For an example, see \seef{MSecsToTimeStamp}.

\begin{function}{TimeToStr}
\Declaration
Function TimeToStr(Time: TDateTime): string;
\Description
\var{TimeToStr} converts the time in \var{Time} to a string. It uses
the \var{ShortTimeFormat} variable to see what formatting needs to be
applied. It is therefor entirely equivalent to a
\var{FormatDateTime('t',Time)} call.
\Errors
None.
\SeeAlso
\end{function}

\latex{\lstinputlisting{sysutex/ex25.pp}}
\html{\input{sysutex/ex25.tex}}
 

\section{Disk functions}

\begin{functionl}{AddDisk (Linux only)}{AddDisk}
\Declaration
Function AddDisk (Const PAth : String) : Longint;
\Description
On Linux  both the \seef{DiskFree} and \seef{DiskSize} functions need a 
file on the specified drive, since is required for the statfs system call.

These filenames are set in drivestr[0..26], and the first 4 have been 
preset to :
\begin{description}
\item[Disk 0]  \var{'.'} default drive - hence current directory is used.
\item[Disk 1]  \var{'/fd0/.'} floppy drive 1.
\item[Disk 2]  \var{'/fd1/.'} floppy drive 2.
\item[Disk 3]  \var{'/'} \file{C:} equivalent of DOS is the root partition.
\end{description}
Drives 4..26 can be set by your own applications with the \var{AddDisk} call.

The \var{AddDisk} call adds \var{Path} to the names of drive files, and
returns the number of the disk that corresponds to this drive. If you
add more than 21 drives, the count is wrapped to 4.
\Errors
None. 
\SeeAlso
\seefl{DiskFree}{DiskFreeSys}, \seefl{DiskSize}{DiskSizeSys}
\end{functionl}

\begin{function}{CreateDir}
\Declaration
Function CreateDir(Const NewDir : String) : Boolean;
\Description
\var{CreateDir} creates a new directory with name \var{NewDir}.
If the directory doesn't contain an absolute path, then the directory is
created below the current working directory.

The function returns \var{True} if the directory was successfully 
created, \var{False} otherwise.
\Errors
In case of an error, the function returns \var{False}.
\SeeAlso
\seef{RemoveDir}
\end{function}

\latex{\lstinputlisting{sysutex/ex26.pp}}
\html{\input{sysutex/ex26.tex}}
 
\begin{functionl}{DiskFree}{DiskFreeSys}
\Declaration
Function DiskFree(Drive : Byte) : Longint;
\Description
\var{DiskFree} returns the free space (in bytes) on disk \var{Drive}.
Drive is the number of the disk drive: 
\begin{description}
\item[0] for the current drive.
\item[1] for the first floppy drive.
\item[2] for the second floppy drive.
\item[3] for the first hard-disk parttion.
\item[4-26] for all other drives and partitions.
\end{description}

{\em Remark} Under \linux, and Unix in general, the concept of disk is
different than the \dos one, since the filesystem is seen as one big
directory tree. For this reason, the \var{DiskFree} and \seef{DiskSize}
functions must be mimicked using filenames that reside on the partitions.
For more information, see \seef{AddDisk}
\Errors
On error, \var{-1} is returned.
\SeeAlso
\seefl{DiskSize}{DiskSizeSys}, \seef{AddDisk}
\end{functionl}

\latex{\lstinputlisting{sysutex/ex27.pp}}
\html{\input{sysutex/ex27.tex}}
 
\begin{functionl}{DiskSize}{DiskSizeSys}
\Declaration
Function DiskSize(Drive : Byte) : Longint;
\Description
\var{DiskSize} returns the size (in bytes) of disk \var{Drive}.
Drive is the number of the disk drive: 
\begin{description}
\item[0] for the current drive.
\item[1] for the first floppy drive.
\item[2] for the second floppy drive.
\item[3] for the first hard-disk parttion.
\item[4-26] for all other drives and partitions.
\end{description}

{\em Remark} Under \linux, and Unix in general, the concept of disk is
different than the \dos one, since the filesystem is seen as one big
directory tree. For this reason, the \seef{DiskFree} and \var{DiskSize}
functions must be mimicked using filenames that reside on the partitions.
For more information, see \seef{AddDisk}
\Errors
On error, \var{-1} is returned.
\SeeAlso
\seefl{DiskFree}{DiskFreeSys}, \seef{AddDisk}
\end{functionl}

For an example, see \seefl{DiskFree}{DiskFreeSys}.
 
\begin{function}{GetCurrentDir}
\Declaration
Function GetCurrentDir : String;
\Description
\var{GetCurrentDir} returns the current working directory.
\Errors
None.
\SeeAlso
\seef{SetCurrentDir}, \seef{DiskFree}, \seef{DiskSize}
\end{function}

\latex{\lstinputlisting{sysutex/ex28.pp}}
\html{\input{sysutex/ex28.tex}}
 
\begin{function}{RemoveDir}
\Declaration
Function RemoveDir(Const Dir : String) : Boolean;
\Description
\var{RemoveDir} removes directory \var{Dir} from the disk.
If the directory is not absolue, it is appended to the current working 
directory.
\Errors
In case of error (e.g. the directory isn't empty) the function returns 
\var{False}. If successful, \var{True} is returned.
\SeeAlso
\end{function}

For an example, see \seef{CreateDir}.
 
\begin{function}{SetCurrentDir}
\Declaration
Function SetCurrentDir(Const NewDir : String) : Boolean;
\Description
\var{SetCurrentDir} sets the current working directory of your program
to \var{NewDir}. It returns \var{True} if the function was successfull,
\var{False} otherwise.
\Errors
In case of error, \var{False} is returned.
\SeeAlso
\seef{GetCurrentDir}
\end{function}

\latex{\lstinputlisting{sysutex/ex29.pp}}
\html{\input{sysutex/ex29.tex}}

\section{File handling functions}

\begin{function}{ChangeFileExt}
\Declaration
Function ChangeFileExt(const FileName, Extension: string): string;
\Description
\var{ChangeFileExt} changes the file extension in \var{FileName} to
\var{Extension}. 
The extension \var{Extension} includes the starting \var{.} (dot).
The previous extension of \var{FileName} are all characters after the
last \var{.}, the \var{.} character included.

If \var{FileName} doesn't have an extension, \var{Extension} is just
appended.
\Errors
None.
\SeeAlso
\seef{ExtractFileName}, \seef{ExtractFilePath}, \seef{ExpandFileName}
\end{function}


\begin{function}{DeleteFile}
\Declaration
Function DeleteFile(Const FileName : String) : Boolean;
\Description
\var{DeleteFile} deletes file \var{FileName} from disk. The function
returns \var{True} if the file was successfully removed, \var{False}
otherwise.
\Errors
On error, \var{False} is returned.
\SeeAlso
\seef{FileCreate}, \seef{FileExists}
\end{function}

\latex{\lstinputlisting{sysutex/ex31.pp}}
\html{\input{sysutex/ex31.tex}}
 
\begin{procedure}{DoDirSeparators}
\Declaration
Procedure DoDirSeparators(Var FileName : String);
\Description
This function replaces all directory separators \var{'\' and '/'}
to the directory separator character for the current system.
\Errors
None.
\SeeAlso
\seef{ExtractFileName}, \seef{ExtractFilePath}
\end{procedure}

\latex{\lstinputlisting{sysutex/ex32.pp}}
\html{\input{sysutex/ex32.tex}}

\begin{function}{ExpandFileName}
\Declaration
Function ExpandFileName(Const FileName : string): String;
\Description
\var{ExpandFileName} expands the filename to an absolute filename.
It changes all directory separator characters to the one appropriate for the
system first.
\Errors
None.
\SeeAlso
\seef{ExtractFileName}, \seef{ExtractFilePath}, \seef{ExtractFileDir},
\seef{ExtractFileDrive}, \seef{ExtractFileExt}, \seef{ExtractRelativePath}
\end{function}

\latex{\lstinputlisting{sysutex/ex33.pp}}
\html{\input{sysutex/ex33.tex}}

 
\begin{function}{ExpandUNCFileName}
\Declaration
Function ExpandUNCFileName(Const FileName : string): String;
\Description
\var{ExpandUNCFileName} runs \seef{ExpandFileName} on \var{FileName}
and then attempts to replace the driveletter by the name of a shared disk.
\Errors
\SeeAlso
\seef{ExtractFileName}, \seef{ExtractFilePath}, \seef{ExtractFileDir},
\seef{ExtractFileDrive}, \seef{ExtractFileExt}, \seef{ExtractRelativePath}
\end{function}

 
\begin{function}{ExtractFileDir}
\Declaration
Function ExtractFileDir(Const FileName : string): string;
\Description
\var{ExtractFileDir} returns only the directory part of \var{FileName},
not including a driveletter. The directory name has NO ending directory
separator, in difference with \seef{ExtractFilePath}. 
\Errors
None.
\SeeAlso
\seef{ExtractFileName}, \seef{ExtractFilePath}, \seef{ExtractFileDir},
\seef{ExtractFileDrive}, \seef{ExtractFileExt}, \seef{ExtractRelativePath}
\end{function}

\latex{\lstinputlisting{sysutex/ex34.pp}}
\html{\input{sysutex/ex34.tex}}

\begin{function}{ExtractFileDrive}
\Declaration
Function ExtractFileDrive(const FileName: string): string;
\Description
\var{Extract}
\Errors
\SeeAlso
\seef{ExtractFileName}, \seef{ExtractFilePath}, \seef{ExtractFileDir},
\seef{ExtractFileDrive}, \seef{ExtractFileExt}, \seef{ExtractRelativePath}
\end{function}

For an example, see \seef{ExtractFileDir}.
 
\begin{function}{ExtractFileExt}
\Declaration
Function ExtractFileExt(const FileName: string): string; 
\Description
\var{ExtractFileExt} returns the extension (including the 
\var{.}(dot) character) of \var{FileName}.
\Errors
None.
\SeeAlso
\seef{ExtractFileName}, \seef{ExtractFilePath}, \seef{ExtractFileDir},
\seef{ExtractFileDrive}, \seef{ExtractFileExt}, \seef{ExtractRelativePath}
\end{function}

For an example, see \seef{ExtractFileDir}.
 
\begin{function}{ExtractFileName}
\Declaration
Function ExtractFileName(const FileName: string): string;
\Description
\var{ExtractFileName} returns the filename part from \var{FileName}.
The filename consists of all characters after the last directory separator
character ('/' or '\') or drive letter. 

The full filename can always be reconstucted by concatenating the result
of \seef{ExtractFilePath} and \var{ExtractFileName}.
\Errors
None.
\SeeAlso
\seef{ExtractFileName}, \seef{ExtractFilePath}, \seef{ExtractFileDir},
\seef{ExtractFileDrive}, \seef{ExtractFileExt},\seef{ExtractRelativePath}
\end{function}

For an example, see \seef{ExtractFileDir}.
 
\begin{function}{ExtractFilePath}
\Declaration
Function ExtractFilePath(const FileName: string): string;
\Description
\var{ExtractFilePath} returns the path part (including driveletter) from 
\var{FileName}. The path consists of all characters before the last 
directory separator character ('/' or '\'), including the directory
separator itself. 
In case there is only a drive letter, that will be returned.

The full filename can always be reconstucted by concatenating the result
of \var{ExtractFilePath} and \seef{ExtractFileName}.
\Errors
None.
\SeeAlso
\seef{ExtractFileName}, \seef{ExtractFilePath}, \seef{ExtractFileDir},
\seef{ExtractFileDrive}, \seef{ExtractFileExt}, \seef{ExtractRelativePath}
\end{function}

For an example, see \seef{ExtractFileDir}.
 
\begin{function}{ExtractRelativePath}
\Declaration
Function ExtractRelativePath(Const BaseName,DestNAme : String): String;
\Description
\var{ExtractRelativePath} constructs a relative path to go from
\var{BaseName} to \var{DestName}. If \var{DestName} is on another drive
(Not on Linux) then the whole \var{Destname} is returned.

{\em Note:} This function does not exist in the Delphi unit.
\Errors
None.
\SeeAlso
\seef{ExtractFileName}, \seef{ExtractFilePath}, \seef{ExtractFileDir},
\seef{ExtractFileDrive}, \seef{ExtractFileExt},
\end{function}

\latex{\lstinputlisting{sysutex/ex35.pp}}
\html{\input{sysutex/ex35.tex}}

\begin{function}{FileAge}
\Declaration
Function FileAge(Const FileName : String): Longint;
\Description
\var{FileAge} returns the last modification time of file \var{FileName}.
The FileDate format can be transformed to \var{TDateTime} format with the
\seef{FileDateToDateTime} function.
\Errors
In case of errors, \var{-1} is returned.
\SeeAlso
\seef{FileDateToDateTime}, \seef{FileExists}, \seef{FileGetAttr}
\end{function}

\latex{\lstinputlisting{sysutex/ex36.pp}}
\html{\input{sysutex/ex36.tex}}

 
\begin{procedure}{FileClose}
\Declaration
Procedure FileClose(Handle : Longint);
\Description
\var{FileClose} closes the file handle \var{Handle}. After this call,
attempting to read or write from the handle will result in an error.
\Errors
None.
\SeeAlso
\seef{FileCreate}, \seef{FileWrite}, \seef{FileOpen}, \seef{FileRead},
\seef{FileTruncate}, \seef{FileSeek}
\end{procedure}

For an example, see \seef{FileCreate}

\begin{function}{FileCreate}
\Declaration
Function FileCreate(Const FileName : String) : Longint;
\Description
\var{FileCreate} creates a new file with name \var{FileName} on the disk and
returns a file handle which can be used to read or write from the file with
the \seef{FileRead} and \seef{FileWrite} functions.

If a file with name \var{FileName} already existed on the disk, it is
overwritten.
\Errors
If an error occurs (e.g. disk full or non-existent path), the function
returns \var{-1}.
\SeeAlso
\seep{FileClose}, \seef{FileWrite}, \seef{FileOpen}, \seef{FileRead},
\seef{FileTruncate}, \seef{FileSeek}
\end{function}

\latex{\lstinputlisting{sysutex/ex37.pp}}
\html{\input{sysutex/ex37.tex}}

\begin{function}{FileExists}
\Declaration
Function FileExists(Const FileName : String) : Boolean;
\Description
\var{FileExists} returns \var{True} if a file with name \var{FileName}
exists on the disk, \var{False} otherwise.
\Errors
None.
\SeeAlso
\seef{FileAge}, \seef{FileGetAttr}, \seef{FileSetAttr}
\end{function}


\latex{\lstinputlisting{sysutex/ex38.pp}}
\html{\input{sysutex/ex38.tex}}

 
\begin{function}{FileGetAttr}
\Declaration
Function FileGetAttr(Const FileName : String) : Longint;
\Description
\var{FileGetAttr} returns the attribute settings of file 
\var{FileName}. The attribute is a \var{OR}-ed combination
of the following constants: 
\begin{description}
\item[faReadOnly] The file is read-only.
\item[faHidden] The file is hidden. (On \linux, this means that the filename
starts with a dot)
\item[faSysFile] The file is a system file (On \linux, this means that the
file is a character, block or FIFO file).
\item[faVolumeId] Volume Label. Not possible under \linux.
\item[faDirectory] File is a directory.
\item[faArchive] file is an archive. Not possible on \linux.
\end{description}
\Errors
In case of error, -1 is returned.
\SeeAlso
\seef{FileSetAttr}, \seef{FileAge}, \seef{FileGetDate}.
\end{function}

\latex{\lstinputlisting{sysutex/ex40.pp}}
\html{\input{sysutex/ex40.tex}}

\begin{function}{FileGetDate}
\Declaration
Function FileGetDate(Handle : Longint) : Longint;
\Description
\var{FileGetdate} returns the filetime of the opened file with filehandle
\var{Handle}. It is the same as \seef{FileAge}, with this difference that
\var{FileAge} only needs the file name, while \var{FilegetDate} needs an
open file handle.
\Errors
On error, -1 is returned.
\SeeAlso
\seef{FileAge}
\end{function}

\latex{\lstinputlisting{sysutex/ex39.pp}}
\html{\input{sysutex/ex39.tex}}
 
\begin{function}{FileOpen}
\Declaration
Function FileOpen(Const FileName : string; Mode : Integer) : Longint;
\Description
\var{FileOpen} opens a file with name \var{FileName} with mode \var{Mode}.
\var{Mode} can be one of the following constants:
\begin{description}
\item[fmOpenRead] The file is opened for reading.
\item[fmOpenWrite] The file is opened for writing.
\item[fmOpenReadWrite] The file is opened for reading and writing.
\end{description}
If the file has been successfully opened, it can be read  from or written to
(depending on the \var{Mode} parameter) with the \seef{FileRead} and
\var{FileWrite} functions.

Remark that you cannot open a file if it doesn't exist yet, i.e. it will not
be created for you. If you want tp create a new file, or overwrite an old
one, use the \seef{FileCreate} function.
\Errors
On Error, -1 is returned.
\SeeAlso
\seep{FileClose}, \seef{FileWrite}, \seef{FileCreate}, \seef{FileRead},
\seef{FileTruncate}, \seef{FileSeek}
\end{function}

For an example, see \seef{FileRead}
 
\begin{function}{FileRead}
\Declaration
Function FileRead(Handle : Longint; Var Buffer; Count : longint) : Longint;
\Description
\var{FileRead} reads \var{Count} bytes from file-handle \var{Handle} and
stores them into \var{Buffer}. Buffer must be at least \var{Count} bytes
long. No checking on this is performed, so be careful not to overwrite any
memory.  \var{Handle} must be the result of a \seef{FileOpen} call.
\Errors
On error, -1 is returned.
\SeeAlso
\seep{FileClose}, \seef{FileWrite}, \seef{FileCreate}, \seef{FileOpen},
\seef{FileTruncate}, \seef{FileSeek}
\end{function}

For an example, see \seef{FileOpen}

\begin{function}{FileSearch}
\Declaration
Function FileSearch(Const Name, DirList : String) : String;
\Description
\var{FileSearch} looks for the file \var{Name} in \var{DirList}, where
dirlist is a list of directories, separated by semicolons or colons.
It returns the full filename of the first match found.
\Errors
On error, an empty string is returned.
\SeeAlso
\seef{ExpandFileName}, \seef{FindFirst}
\end{function}

\latex{\lstinputlisting{sysutex/ex41.pp}}
\html{\input{sysutex/ex41.tex}}

\begin{function}{FileSeek}
\Declaration
Function FileSeek(Handle,Offset,Origin : Longint) : Longint;
\Description
\var{FileSeek} sets the file pointer on position \var{Offset}, starting from 
\var{Origin}. Origin can be one of the following values: 
\begin{description}
\item[fsFromBeginning]  \var{Offset} is relative to the first byte of the file. This
position is zero-based. i.e. the first byte is at offset 0.
\item[fsFromCurrent]  \var{Offset} is relative to the current position.
\item[fsFromEnd] \var{Offset} is relative to the end of the file. This means
that \var{Offset} can only be zero or negative in this case.
\end{description}
If successfull, the function returns the new file position, relative to the
beginning of the file.

{\em Remark:} The abovementioned constants do not exist in Delphi.
\Errors
On error, -1 is returned.
\SeeAlso
\seep{FileClose}, \seef{FileWrite}, \seef{FileCreate}, \seef{FileOpen}
\seef{FileRead}, \seef{FileTruncate}
\end{function}

\latex{\lstinputlisting{sysutex/ex42.pp}}
\html{\input{sysutex/ex42.tex}}

For an example, see \seef{FileCreate}
 
\begin{functionl}{FileSetAttr (Not on Linux)}{FileSetAttr}
\Declaration
Function FileSetAttr(Const Filename : String; Attr: longint) : Longint;
\Description
\var{FileSetAttr} sets the attributes of \var{FileName} to \var{Attr}.
If the function was successful, 0 is returned, -1 otherwise.

\var{Attr} can be set to an OR-ed combination of the pre-defined 
\var{faXXX} constants.
\Errors
On error, -1 is returned (always on linux). 
\SeeAlso
\seef{FileGetAttr}, \seef{FileGetDate}, \seef{FileSetDate}.
\end{functionl}

 
\begin{functionl}{FileSetDate (Not on Linux)}{FileSetDate}
\Declaration
Function FileSetDate(Handle,Age : Longint) : Longint;
\Description
\var{FileSetDate} sets the file date of the file with handle \var{Handle} 
to \var{Age}, where \var{Age} is a DOS date-and-time stamp value.

The function returns zero of successfull.
\Errors
On Linux, -1 is always returned, since this is impossible to implement.
On Windows and DOS, a negative error code is returned.
\SeeAlso
\end{functionl}

 
\begin{function}{FileTruncate}
\Declaration
Function FileTruncate(Handle,Size: Longint) : boolean;
\Description
\var{FileTruncate} truncates the file with handle \var{Handle} to
\var{Size} bytes. The file must have been opened for writing prior
to this call. The function returns \var{True} is successful, \var{False}
otherwise.
\Errors
On error, the function returns \var{False}.
\SeeAlso
\seep{FileClose}, \seef{FileWrite}, \seef{FileCreate}, \seef{FileOpen}
\seef{FileRead}, \seef{FileSeek}
\end{function}

For an example, see \seef{FileCreate}.
 
\begin{function}{FileWrite}
\Declaration
Function FileWrite(Handle : Longint; Var Buffer; Count : Longint) : Longint;
\Description
\var{FileWrite} writes \var{Count} bytes from \var{Buffer} to the file with
handle \var{Handle}. Prior to this call, the file must have been opened 
for writing. \var{Buffer} must be at least \var{Count} bytes large, or 
a memory access error may occur.

The function returns the number of bytes written, or -1 in case of an
error.
\Errors
In case of error, -1 is returned.
\SeeAlso
\seep{FileClose}, \seef{FileCreate}, \seef{FileOpen}
\seef{FileRead}, \seef{FileTruncate}, \seef{FileSeek}
\end{function}

For an example, see \seef{FileCreate}. 

\begin{procedurel}{FindClose}{FindCloseSys}
\Declaration
Procedure FindClose(Var F : TSearchrec);
\Description
\var{FindClose} ends a series of \seef{FindFirst}/\seef{FindNext} calls,
and frees any memory used by these calls. It is {\em absolutely} necessary
to do this call, or huge memory losses may occur.
\Errors
None.
\SeeAlso
\seef{FindFirst}, \seef{FindNext}.
\end{procedurel}

For an example, see \seef{FindFirst}.
 
\begin{function}{FindFirst}
\Declaration
Function FindFirst(Const Path : String; Attr : Longint; Var Rslt : TSearchRec) : Longint;
\Description
\var{FindFirst} looks for files that match the name (possibly with
wildcards) in \var{Path} and attributes \var{Attr}. It then fills up the
\var{Rslt} record with data gathered about the file. It returns 0 if a file
matching the specified criteria is found, a nonzero value (-1 on linux) 
otherwise.

The \var{Rslt} record can be fed to subsequent calls to \var{FindNext}, in
order to find other files matching the specifications.

{\em remark:} A \var{FindFirst} call must {\em always} be followed by a
\seepl{FindClose}{FindCloseSys} call with the same \var{Rslt} record. Failure to do so will
result in memory loss.
\Errors
On error the function returns -1 on linux, a nonzero error code on Windows.
\SeeAlso
\seep{FindClose}{FindCloseSys}, \seef{FindNext}.
\end{function}

\latex{\lstinputlisting{sysutex/ex43.pp}}
\html{\input{sysutex/ex43.tex}}

\begin{function}{FindNext}
\Declaration
Function FindNext(Var Rslt : TSearchRec) : Longint;
\Description
\var{FindNext} finds a next occurrence of a search sequence initiated by
\var{FindFirst}. If another record matching the criteria in Rslt is found, 0
is returned, a nonzero constant is returned otherwise.

{\em remark:} The last \var{FindNext} call must {\em always} be followed by a
\var{FindClose} call with the same \var{Rslt} record. Failure to do so will
result in memory loss.
\Errors
On error (no more file is found), a nonzero constant is returned.
\SeeAlso
\seef{FindFirst}, \seep{FindClose}
\end{function}

For an example, see \seef{FindFirst}
 
\begin{function}{GetDirs}
\Declaration
Function GetDirs(Var DirName : String; Var Dirs : Array of pchar) : Longint; 
\Description
\var{GetDirs} splits DirName in a null-byte separated list of directory names,
\var{Dirs} is an array of \var{PChars}, pointing to these directory names.
The function returns the number of directories found, or -1 if none were found.
DirName must contain only OSDirSeparator as Directory separator chars.
\Errors
None.
\SeeAlso
\seef{ExtractRelativePath}
\end{function}

\latex{\lstinputlisting{sysutex/ex45.pp}}
\html{\input{sysutex/ex45.tex}}
 
\begin{function}{RenameFile}
\Declaration
Function RenameFile(Const OldName, NewName : String) : Boolean;
\Description
\var{RenameFile} renames a file from \var{OldName} to \var{NewName}. The
function returns \var{True} if successful, \var{False} otherwise.

{\em Remark:} you cannot rename across disks or partitions. 
\Errors
On Error, \var{False} is returned.
\SeeAlso
\seef{DeleteFile}
\end{function}

\latex{\lstinputlisting{sysutex/ex44.pp}}
\html{\input{sysutex/ex44.tex}}

\begin{function}{SetDirSeparators}
\Declaration
Function SetDirSeparators(Const FileName : String) : String;
\Description
\var{SetDirSeparators} returns \var{FileName} with all possible
DirSeparators replaced by \var{OSDirSeparator}.
\Errors
None.
\SeeAlso
\seef{ExpandFileName}, \seef{ExtractFilePath}, \seef{ExtractFileDir}
\end{function}

\latex{\lstinputlisting{sysutex/ex47.pp}}
\html{\input{sysutex/ex47.tex}}

\section{PChar functions}

\subsection{Introduction}

Most PChar functions are the same as their counterparts in the \file{STRINGS}
unit. The following functions are the same :

\begin{enumerate}
\item \seef{StrCat} : Concatenates two \var{PChar} strings.
\item \seef{StrComp} : Compares two \var{PChar} strings.
\item \seef{StrCopy} : Copies a \var{PChar} string.
\item \seef{StrECopy} : Copies a \var{PChar} string and returns a pointer to
the terminating null byte.
\item \seef{StrEnd} : Returns a pointer to the terminating null byte.
\item \seef{StrIComp} : Case insensitive compare of 2 \var{PChar} strings.
\item \seef{StrLCat} : Appends at most L characters from one \var{PChar} to
another \var{PChar}.
\item \seef{StrLComp} : Case sensitive compare of at most L characters of 2
 \var{PChar} strings.
\item \seef{StrLCopy} : Copies at most L characters from one \var{PChar} to
another.
\item \seef{StrLen} : Returns the length (exclusive terminating null byte)
of a \var{PChar} string.
\item \seef{StrLIComp} : Case insensitive compare of at most L characters of 2
 \var{PChar} strings.  
\item \seef{StrLower} : Converts a \var{PChar} to all lowercase letters.
\item \seef{StrMove} : Moves one \var{PChar} to another.
\item \seef{StrNew} : Makes a copy of a \var{PChar} on the heap, and returns
a pointer to this copy.
\item \seef{StrPos} : Returns the position of one \var{PChar} string in
another?
\item \seef{StrRScan} : returns a pointer to the last occurrence of on 
 \var{PChar} string in another one.
\item \seef{StrScan} : returns a pointer to the first occurrence of on 
 \var{PChar} string in another one.
\item \seef{StrUpper} : Converts a \var{PChar} to all uppercase letters.
\end{enumerate}
The subsequent functions are different from their counterparts in
\file{STRINGS}, although the same examples can be used.


\begin{functionl}{StrAlloc}{StrAllocSys}
\Declaration
Function StrAlloc(Size: cardinal): PChar;
\Description
\var{StrAlloc} reserves memory on the heap for a string with length \var{Len},
terminating \var{\#0} included, and returns a pointer to it.

Additionally, \var{StrAlloc} allocates 4 extra bytes to store the size of
the allocated memory. Therefore this function is NOT compatible with the 
\seef{StrAlloc} function of the \var{Strings} unit.
\Errors
None.
\SeeAlso
\seef{StrBufSize}, \seepl{StrDispose}{StrDisposeSys}, \seef{StrAlloc}
\end{functionl}

For an example, see \seef{StrBufSize}.
 
\begin{function}{StrBufSize}
\Declaration
Function StrBufSize(var Str: PChar): cardinal;
\Description
\var{StrBufSize} returns the memory allocated for \var{Str}. This function
ONLY gives the correct result if \var{Str} was allocated using
\seefl{StrAlloc}{StrAllocSys}.
\Errors
If no more memory is available, a runtime error occurs.
\SeeAlso
\seefl{StrAlloc}{StrAllocSys}.\seepl{StrDispose}{StrDisposeSys}.
\end{function}

\latex{\lstinputlisting{sysutex/ex46.pp}}
\html{\input{sysutex/ex46.tex}}

 
\begin{procedurel}{StrDispose}{StrDisposeSys}
\Declaration
Procedure StrDispose(var Str: PChar);
\Description
\var{StrDispose} frees any memory allocated for \var{Str}. This function
will only function correctly if \var{Str} has been allocated using
\seefl{StrAlloc}{StrAllocSys} from the \file{SYSUTILS} unit.
\Errors
If an invalid pointer is passed, or a pointer not allocated with
\var{StrAlloc}, an error may occur.
\SeeAlso
\seef{StrBufSize}, \seefl{StrAlloc}{StrAllocSys}, \seep{StrDispose}
\end{procedurel}

For an example, see \seef{StrBufSize}.
 
\begin{functionl}{StrPCopy}{StrPCopySys}
\Declaration
Function StrPCopy(Dest: PChar; Source: string): PChar;
\Description
\var{StrPCopy} Converts the Ansistring in \var{Source} to a Null-terminated 
string, and copies it to \var{Dest}. \var{Dest} needs enough room to contain
the string \var{Source}, i.e. \var{Length(Source)+1} bytes.
\Errors
No checking is performed to see whether \var{Dest} points to enough memory
to contain \var{Source}.
\SeeAlso
\seefl{StrPLCopy}{StrPLCopySys}, \seef{StrPCopy}
\end{functionl}

For an example, see \seef{StrPCopy}.

\begin{functionl}{StrPLCopy}{StrPLCopySys}
\Declaration
Function StrPLCopy(Dest: PChar; Source: string; MaxLen: cardinal): PChar;
\Description
\var{StrPLCopy} Converts maximally \var{MaxLen} characters of the 
Ansistring in \var{Source} to a Null-terminated  string, and copies 
it to \var{Dest}. \var{Dest} needs enough room to contain
the  characters.
\Errors
No checking is performed to see whether \var{Dest} points to enough memory
to contain L characters of \var{Source}.
\Errors
\SeeAlso
\seefl{StrPCopy}{StrPCopySys}.
\end{functionl}


\begin{functionl}{StrPas}{StrPasSys}
\Declaration
Function StrPas(Str: PChar): string;
\Description
Converts a null terminated string in \var{Str} to an Ansitring, and returns
this string. This string is NOT truncated at 255 characters as is the 
\Errors
None.
\SeeAlso
\seef{StrPas}.
\end{functionl}

For an example, see \seef{StrPas}.

\section{String handling functions}

\begin{function}{AdjustLineBreaks}
\Declaration
Function AdjustLineBreaks(const S: string): string;
\Description
\var{AdjustLineBreaks} will change all \var{\#13} characters with
\var{\#13\#10} on \windowsnt and \dos. On \linux, all \var{\#13\#10}
character pairs are converted to \var{\#10} and single \var{\#13}
characters also.
\Errors
None.
\SeeAlso
\seef{AnsiCompareStr}, \seef{AnsiCompareText}
\end{function}

\latex{\lstinputlisting{sysutex/ex48.pp}}
\html{\input{sysutex/ex48.tex}}
 
\begin{function}{AnsiCompareStr}
\Declaration
Function AnsiCompareStr(const S1, S2: string): integer;
\Description
\var{AnsiCompareStr} compares two strings and returns the following
result:
\begin{description}
\item[<0]  if \var{S1<S2}.
\item[0]  if \var{S1=S2}.
\item[>0] if \var{S1>S2}.
\end{description}
the comparision takes into account Ansi characters, i.e. it takes
care of strange accented characters. Contrary to \seef{AnsiCompareText}, 
the comparision is case sensitive.
\Errors
None.
\SeeAlso
\seef{AdjustLineBreaks}, \seef{AnsiCompareText}
\end{function}

\latex{\lstinputlisting{sysutex/ex49.pp}}
\html{\input{sysutex/ex49.tex}}
 
\begin{function}{AnsiCompareText}
\Declaration
Function AnsiCompareText(const S1, S2: string): integer;
\Description
\Description
\var{AnsiCompareText} compares two strings and returns the following
result:
\begin{description}
\item[<0]  if \var{S1<S2}.
\item[0]  if \var{S1=S2}.
\item[>0] if \var{S1>S2}.
\end{description}
the comparision takes into account Ansi characters, i.e. it takes
care of strange accented characters. Contrary to \seef{AnsiCompareStr}, 
the comparision is case insensitive.
\Errors
None.
\SeeAlso
\seef{AdjustLineBreaks}, \seef{AnsiCompareText}
\end{function}

\latex{\lstinputlisting{sysutex/ex50.pp}}
\html{\input{sysutex/ex50.tex}}

\begin{function}{AnsiExtractQuotedStr}
\Declaration
Function AnsiExtractQuotedStr(var Src: PChar; Quote: Char): string;
\Description
\var{AnsiExtractQuotedStr} Returns \var{Src} as a string,, with \var{Quute} 
characters removed from the beginning and end of the string, and double 
\var{Quote} characters replaced by a single \var{Quote} characters. 
As such, it revereses the action of \seef{AnsiQuotedStr}.
\Errors
None.
\SeeAlso
\seef{AnsiQuotedStr}
\end{function}

\latex{\lstinputlisting{sysutex/ex51.pp}}
\html{\input{sysutex/ex51.tex}}
 
\begin{function}{AnsiLastChar}
\Declaration
Function AnsiLastChar(const S: string): PChar;
\Description
This function returns a pointer to the last character of \var{S}.
Since multibyte characters are not yet supported, this is the same
as \var{@S[Length(S)])}.
\Errors
None.
\SeeAlso
\seef{AnsiStrLastChar}
\end{function}

\latex{\lstinputlisting{sysutex/ex52.pp}}
\html{\input{sysutex/ex52.tex}}
 
\begin{function}{AnsiLowerCase}
\Declaration
Function AnsiLowerCase(const s: string): string;
\Description
\var{AnsiLowerCase} converts the string \var{S} to lowercase characters 
and returns the resulting string. 
It takes into account the operating system language
settings when doing this, so spcial characters are converted correctly as
well.

{\em Remark} On linux, no language setting is taken in account yet.
\Errors
None.
\SeeAlso
\seef{AnsiUpperCase}, \seef{AnsiStrLower}, \seef{AnsiStrUpper}
\end{function}

\latex{\lstinputlisting{sysutex/ex53.pp}}
\html{\input{sysutex/ex53.tex}}
 
\begin{function}{AnsiQuotedStr}
\Declaration
Function AnsiQuotedStr(const S: string; Quote: char): string;
\Description
\var{AnsiQuotedString} quotes the string \var{S} and returns the result.
This means that it puts the \var{Quote} character at both the beginning and
end of the string and replaces any occurrence of \var{Quote} in \var{S} 
with 2 \var{Quote} characters. The action of \var{AnsiQuotedString} can be
reversed by \seef{AnsiExtractQuotedStr}.
\Errors
None.
\SeeAlso
\seef{AnsiExtractQuotedStr}
\end{function}

For an example, see \seef{AnsiExtractQuotedStr}
 
\begin{function}{AnsiStrComp}
\Declaration
Function AnsiStrComp(S1, S2: PChar): integer;
\Description
\var{AnsiStrComp} compares 2 \var{PChar} strings, and returns the following
result:
\begin{description}
\item[<0]  if \var{S1<S2}.
\item[0]  if \var{S1=S2}.
\item[>0]  if \var{S1>S2}.
\end{description}
The comparision of the two strings is case-sensitive.
The function does not yet take internationalization settings into account.
\Errors
None.
\SeeAlso
\seef{AnsiCompareText}, \seef{AnsiCompareStr}
\end{function}

\latex{\lstinputlisting{sysutex/ex54.pp}}
\html{\input{sysutex/ex54.tex}}
 
\begin{function}{AnsiStrIComp}
\Declaration
Function AnsiStrIComp(S1, S2: PChar): integer;
\Description
\var{AnsiStrIComp} compares 2 \var{PChar} strings, and returns the following
result:
\begin{description}
\item[<0]  if \var{S1<S2}.
\item[0]  if \var{S1=S2}.
\item[>0]  if \var{S1>S2}.
\end{description}
The comparision of the two strings is case-insensitive.
The function does not yet take internationalization settings into account.
\Errors
None.
\SeeAlso
\seef{AnsiCompareText}, \seef{AnsiCompareStr}
\end{function}

\latex{\lstinputlisting{sysutex/ex55.pp}}
\html{\input{sysutex/ex55.tex}}

\begin{function}{AnsiStrLastChar}
\Declaration
function AnsiStrLastChar(Str: PChar): PChar;
\Declaration
\var{AnsiStrLastChar} returns a pointer to the last character of \var{Str}.
Since multibyte characters are not yet supported, this is the same
as \var{StrEnd(Str)-1}.
\Errors
None.
\SeeAlso
\seef{AnsiLastChar}
\end{function}

\latex{\lstinputlisting{sysutex/ex58.pp}}
\html{\input{sysutex/ex58.tex}}

\begin{function}{AnsiStrLComp}
\Declaration
Function AnsiStrLComp(S1, S2: PChar; MaxLen: cardinal): integer;
\Description
\var{AnsiStrLComp} compares the first \var{Maxlen} characters of
2 \var{PChar} strings, \var{S1} and \var{S2}, and returns the following
result:
\begin{description}
\item[<0]  if \var{S1<S2}.
\item[0]  if \var{S1=S2}.
\item[>0]  if \var{S1>S2}.
\end{description}
The comparision of the two strings is case-sensitive.
The function does not yet take internationalization settings into account.
\Errors
None.
\SeeAlso
\seef{AnsiCompareText}, \seef{AnsiCompareStr}
\end{function}

\latex{\lstinputlisting{sysutex/ex56.pp}}
\html{\input{sysutex/ex56.tex}}

\begin{function}{AnsiStrLIComp}
\Declaration
Function AnsiStrLIComp(S1, S2: PChar; MaxLen: cardinal): integer;
\Description
\var{AnsiStrLIComp} compares the first \var{Maxlen} characters of
2 \var{PChar} strings, \var{S1} and \var{S2}, and returns the following
result:
\begin{description}
\item[<0]  if \var{S1<S2}.
\item[0]  if \var{S1=S2}.
\item[>0]  if \var{S1>S2}.
\end{description}
The comparision of the two strings is case-insensitive.
The function does not yet take internationalization settings into account.
\Errors
None.
\SeeAlso
\seef{AnsiCompareText}, \seef{AnsiCompareStr}
\end{function}
 
\latex{\lstinputlisting{sysutex/ex57.pp}}
\html{\input{sysutex/ex57.tex}}



\begin{function}{AnsiStrLower}
\Declaration
Function AnsiStrLower(Str: PChar): PChar;
\Description
\var{AnsiStrLower} converts the PChar \var{Str} to lowercase characters 
and returns the resulting pchar. Note that \var{Str} itself is modified,
not a copy, as in the case of \seef{AnsiLowerCase}.
It takes into account the operating system language
settings when doing this, so spcial characters are converted correctly as
well.

{\em Remark} On linux, no language setting is taken in account yet.
\Errors
None.
\SeeAlso
\seef{AnsiStrUpper}, \seef{AnsiLowerCase}
\end{function}

\latex{\lstinputlisting{sysutex/ex59.pp}}
\html{\input{sysutex/ex59.tex}}
 
\begin{function}{AnsiStrUpper}
\Declaration
Function AnsiStrUpper(Str: PChar): PChar;
\Description
\var{AnsiStrUpper} converts the \var{PChar} \var{Str} to uppercase characters 
and returns the resulting string. Note that \var{Str} itself is modified,
not a copy, as in the case of \seef{AnsiUpperCase}.
It takes into account the operating system language
settings when doing this, so spcial characters are converted correctly as
well.

{\em Remark} On linux, no language setting is taken in account yet.
\Errors
None.
\SeeAlso
\seef{AnsiUpperCase}, \seef{AnsiStrLower}, \seef{AnsiLowerCase}
\end{function}

\latex{\lstinputlisting{sysutex/ex60.pp}}
\html{\input{sysutex/ex60.tex}}
 
\begin{function}{AnsiUpperCase}
\Declaration
Function AnsiUpperCase(const s: string): string;
\Description
\var{AnsiUpperCase} converts the string \var{S} to uppercase characters 
and returns the resulting string. 
It takes into account the operating system language
settings when doing this, so spcial characters are converted correctly as
well.

{\em Remark} On linux, no language setting is taken in account yet.
\Errors
None.
\SeeAlso
\seef{AnsiStrUpper}, \seef{AnsiStrLower}, \seef{AnsiLowerCase}
\end{function}

\latex{\lstinputlisting{sysutex/ex61.pp}}
\html{\input{sysutex/ex61.tex}}
 
\begin{procedure}{AppendStr}
\Declaration
Procedure AppendStr(var Dest: String; const S: string);
\Description
\var{AppendStr} appends \var{S} to Dest. 

This function is provided for Delphi
compatibility only, since it is completely equivalent to \var{Dest:=Dest+S}.
\Errors
None.
\SeeAlso
\seep{AssignStr},\seef{NewStr}, \seep{DisposeStr}
\end{procedure}

\latex{\lstinputlisting{sysutex/ex62.pp}}
\html{\input{sysutex/ex62.tex}}
 
\begin{procedure}{AssignStr}
\Declaration
Procedure AssignStr(var P: PString; const S: string);
\Description
\var{AssignStr} allocates \var{S} to P. The old value of \var{P} is
disposed of.

This function is provided for Delphi compatibility only. \var{AnsiStrings}
are managed on the heap and should be preferred to the mechanism of
dynamically allocated strings.
\Errors
None.
\SeeAlso
\seef{NewStr}, \seep{AppendStr}, \seep{DisposeStr}
\end{procedure}

\latex{\lstinputlisting{sysutex/ex63.pp}}
\html{\input{sysutex/ex63.tex}}
 
\begin{function}{BCDToInt}
\Declaration
Function BCDToInt(Value: integer): integer;
\Description
\var{BCDToInt} converts a \var{BCD} coded integer to a normal integer.
\Errors
None.
\SeeAlso
\seef{StrToInt}, \seef{IntToStr}
\end{function}

\latex{\lstinputlisting{sysutex/ex64.pp}}
\html{\input{sysutex/ex64.tex}}

 
\begin{function}{CompareMem}
\Declaration
Function CompareMem(P1, P2: Pointer; Length: cardinal): integer;
\Description
\var{CompareMem} compares, byte by byte,  2 memory areas pointed 
to by \var{P1} and \var{P2}, for a length of \var{L} bytes. 

It returns the following values:
\begin{description}
\item[<0] if at some position the byte at \var{P1} is less than the byte at the
same postion at \var{P2}.
\item[0] if all \var{L} bytes are the same.
\item[3]
\end{description}
\Errors
\SeeAlso
\end{function}

 
\begin{function}{CompareStr}
\Declaration
Function CompareStr(const S1, S2: string): Integer;
\Description
\var{CompareStr} compares two strings, \var{S1} and \var{S2}, 
and returns the following
result:
\begin{description}
\item[<0]  if \var{S1<S2}.
\item[0]  if \var{S1=S2}.
\item[>0]  if \var{S1>S2}.
\end{description}
The comparision of the two strings is case-sensitive.
The function does not take internationalization settings into account, it
simply compares ASCII values.
\Errors
None.
\SeeAlso
\seef{AnsiCompareText}, \seef{AnsiCompareStr}, \seef{CompareText}
\end{function}

\latex{\lstinputlisting{sysutex/ex65.pp}}
\html{\input{sysutex/ex65.tex}}

\begin{function}{CompareText}
\Declaration
Function CompareText(const S1, S2: string): integer;
\Description
\var{CompareText} compares two strings, \var{S1} and \var{S2}, 
and returns the following
result:
\begin{description}
\item[<0]  if \var{S1<S2}.
\item[0]  if \var{S1=S2}.
\item[>0]  if \var{S1>S2}.
\end{description}
The comparision of the two strings is case-insensitive.
The function does not take internationalization settings into account, it
simply compares ASCII values.
\Errors
None.
\SeeAlso
\seef{AnsiCompareText}, \seef{AnsiCompareStr}, \seef{CompareStr}
\end{function}

\latex{\lstinputlisting{sysutex/ex66.pp}}
\html{\input{sysutex/ex66.tex}}


 
\begin{procedurel}{DisposeStr}{DisposeStrSys}
\Declaration
Procedure DisposeStr(S: PString);
\Description
\var{DisposeStr} removes the dynamically allocated string \var{S} from the
heap, and releases the occupied memory.

This function is provided for Delphi compatibility only. \var{AnsiStrings}
are managed on the heap and should be preferred to the mechanism of
dynamically allocated strings.
\Errors
None.
\SeeAlso
\seef{NewStr}, \seep{AppendStr}, \seep{AssignStr}
\end{procedurel}

For an example, see \seep{DisposeStr}.
 
\begin{function}{FloatToStr}
\Declaration
Function FloatToStr(Value: Extended): String;
\Description
\var{FloatToStr} converts the floating point variable \var{Value} to a 
string representation.  It will choose the shortest possible notation of the
two following formats:
\begin{description}
\item[Fixed format] will represent the string in fixed notation,
\item[Decimal format] will represent the string in scientific notation.
\end{description}
(more information on these formats can be found in \seef{FloatToStrF})
\var{FloatToStr} is completely equivalent to a \var{FloatToStrF(Value, ffGeneral,
15, 0);} call.
\Errors
None.
\SeeAlso
\seef{FloatToStrF}
%, \seef{FormatFloat}
\end{function}

\latex{\lstinputlisting{sysutex/ex67.pp}}
\html{\input{sysutex/ex67.tex}}

\begin{function}{FloatToStrF}
\Declaration
Function FloatToStrF(Value: Extended; format: TFloatFormat; Precision, Digits: Integer): String;
\Description
\var{FloatToStrF} converts the floating point number \var{value} to a string
representation, according to the settings of the parameters \var{Format},
\var{Precision} and \var{Digits}.

The meaning of the \var{Precision} and \var{Digits} parameter depends on the
\var{Format} parameter. The format is controlled mainly by the \var{Format} 
parameter. It can have one of the following values:
\begin{description}
\item[ffcurrency] Money format. \var{Value} is converted to a string using
the global variables \var{CurrencyString}, \var{CurrencyFormat} and
\var{NegCurrencyFormat}. The \var{Digits} paramater specifies the number of digits
following the decimal point and should be in the range -1 to 18. If Digits
equals \var{-1}, \var{CurrencyDecimals} is assumed. The \var{Precision} parameter is ignored.
%
\item[ffExponent] Scientific format. \var{Value} is converted to a 
string using scientific notation: 1 digit before the decimal point, possibly 
preceded by a minus sign if \var{Value} is negative. The number of
digits after the decimal point is controlled by \var{Precision} and must lie
in the range 0 to 15.  
%
\item[ffFixed] Fixed point format. \var{Value} is converted to a string
using fixed point notation. The result is composed of all digits of the 
integer part of \var{Value}, preceded by a minus sign if \var{Value} is
negative. Following the integer part is \var{DecimalSeparator} and then the
fractional part of \var{Value}, rounded off to \var{Digits} numbers.
If the number is too large then the result will be in scientific notation.
%
\item[ffGeneral] General number format. The argument is converted to a
string using \var{ffExponent} or \var{ffFixed} format, depending on wich one
gives the shortest string. There will be no trailing zeroes. If \var{Value}
is less than \var{0.00001} or if the number of decimals left of the decimal
point is larger than \var{Precision} then scientific notation is used, and
\var{Digits} is the minimum number of digits in the exponent. Otherwise
\var{Digits} is ignored.
\item[ffnumber] Is the same as \var{ffFixed}, except that thousand separators
are inserted in the resultig string.
\end{description}
\Errors
None.
\SeeAlso
\seef{FloatToStr}, \seef{FloatToText}
\end{function}

\latex{\lstinputlisting{sysutex/ex68.pp}}
\html{\input{sysutex/ex68.tex}}

\begin{function}{FloatToText}
\Declaration
Function FloatToText(Buffer : Pchar;Value: Extended; Format: TFloatFormat; Precision, Digits: Integer): Longint;
\Description
\var{FloatToText} converts the floating point variable \var{Value} to a 
string representation and stores it in \var{Buffer}.  The conversion is
giverned by \var{format}, \var{Precisison} and \var{Digits}.
more information on these parameters can be found in \seef{FloatToStrF}.
\var{Buffer} should point to enough space to hold the result. No checking on
this is performed.

The result is the number of characters that was copied in \var{Buffer}.
\Errors
None.
\SeeAlso
\seef{FloatToStr}, \seef{FloatToStrF}
\end{function}
 
\latex{\lstinputlisting{sysutex/ex69.pp}}
\html{\input{sysutex/ex69.tex}}

\begin{procedure}{FmtStr}
\Declaration
Procedure (Var Res: String; Const Fmt : String; Const args: Array of const);
\Description
\var{FmtStr} calls \seef{Format} with \var{Fmt} and \var{Args} as arguments,
and stores the result in \var{Res}. For more information on how the
resulting string is composed, see \seef{Format}.
\Errors
In case of error, a \var{EConvertError} exception is raised.
\SeeAlso
\seef{Format}, \seef{FormatBuf}.
\end{procedure}

\latex{\lstinputlisting{sysutex/ex70.pp}}
\html{\input{sysutex/ex70.tex}}

\begin{function}{Format}
\Declaration
Function Format(Const Fmt : String; const Args : Array of const) : String;
\Description
Format replaces all placeholders in\var{Fmt} with the arguments passed in
\var{Args} and returns the resulting string. A placeholder looks as follows:
\begin{verbatim}
'%' [Index':'] ['-'] [Width] ['.' Precision] ArgType
\end{verbatim}
elements between single quotes must be typed as shown without the quotes,
and elements between square brackets \var{[ ]} are optional. The meaning 
of the different elements is shown below:
\begin{description}
\item['\%'] starts the placeholder. If you want to insert a literal
\var{\%} character, then you must insert two of them : \var{\%\%}.
\item[Index ':'] takes the \var{Index}-th element in the argument array 
as the element to insert.
\item['-'] tells \var{Format} to left-align the inserted text. The default
behaviour is to right-align inserted text. This can only take effect if the
\var{Width} element is also specified.
\item[Width] the inserted string must have at least have \var{Width}
characters. If not, the inserted string will be padded with spaces. By
default, the string is left-padded, resulting in a right-aligned string.
This behaviour can be changed by the \var{'-'} character.
\item['.' Precision] Indicates the precision to be used when converting
the argument. The exact meaning of this parameter depends on \var{ArgType}.
\end{description}
The \var{Index}, \var{Width} and \var{Precision} parameters can be replaced
by \var{*}, in which case their value will be read from the next element in
the \var{Args} array. This value must be an integer, or an
\var{EConvertError} exception will be raised.

The argument type is determined from \var{ArgType}. It can have one of the
following values (case insensitive):
\begin{description}
\item[D] Decimal format. The next argument in the \var{Args} array should be
an integer. The argument is converted to a decimal string,. If precision is
specified, then the string will have at least \var{Precision} digits in it.
If needed, the string is (left) padded with zeroes.
\item[E] scientific format. The next argument in the \var{Args} array should
be a Floating point value. The argument is converted to a decimal string
using scientific notation, using \seef{FloatToStrF}, where the optional
precision is used to specify the total number of decimals. (defalt a valueof
15 is used). The exponent is formatted using maximally 3 digits.

In short, the \var{E} specifier formats it's arguument as follows:
\begin{verbatim}
FloatToStrF(Argument,ffexponent,Precision,3)
\end{verbatim}

\item[F] fixed point format. The next argument in the \var{Args} array
should be a floating point value. The argument is converted to a 
decimal string, using fixed notation (see \seef{FloatToStrF}). 
\var{Precision} indicates the number of digits following the 
decimal point.

In short, the \var{F} specifier formats it's arguument as follows:
\begin{verbatim}
FloatToStrF(Argument,ffFixed,ffixed,9999,Precision)
\end{verbatim}

\item[G] General number format. The next argument in the \var{Args} array 
should be a floating point value. The argument is converted to a decimal
string using fixed point notation or scientific notation, depending on which
gives the shortest result. \var{Precision} is used to determine the number
of digits after the decimal point.

In short, the \var{G} specifier formats it's arguument as follows:
\begin{verbatim}
FloatToStrF(Argument,ffGeneral,Precision,3)
\end{verbatim}

\item[M] Currency format. the next argument in the var{Args} array must 
be a floating point value. The argument is converted to a decimal string
using currency notation. This means that fixed-point notation is used, but 
that the currency symbol is appended. If precision is specified, then 
then it overrides the \var{CurrencyDecimals} global variable used in the 
\seef{FloatToStrF}

In short, the \var{M} specifier formats it's arguument as follows:
\begin{verbatim}
FloatToStrF(Argument,ffCurrency,9999,Precision)
\end{verbatim}

\item[N] Number format. This is the same as fixed point format, except that
thousand separators are inserted in the resulting string.

\item[P] Pointer format. The next argument in the \var{Args} array must be a
pointer (typed or untyped). The pointer value is converted to a string of
length 8, representing the hexadecimal value of the pointer. 

\item[S] String format. The next argument in the \var{Args} array must be
a string. The argument is simply copied to the result string. If
\var{Precision} is specified, then only \var{Precision} characters are
copied to the result string.

\item[X] hexadecimal format. The next argument in the \var{Args} array must
be an integer. The argument is converted to a hexadecimal string with just
enough characters to contain the value of the integer. If \var{Precision}
is specified then the resulting hexadecimal representation will have at
least \var{Precision} characters in it (with a maximum value of 32).
\end{description}
\Errors
In case of error, an \var{EConversionError} exception is raised. Possible
errors are:
\begin{enumerate}
\item Errors in the format specifiers. 
\item The next argument is not of the type needed by a specifier.
\item The number of arguments is not sufficient for all format specifiers.
\end{enumerate}
\SeeAlso
\seef{FormatBuf}
\end{function}

\latex{\lstinputlisting{sysutex/ex71.pp}}
\html{\input{sysutex/ex71.tex}}
 
\begin{function}{FormatBuf}
\Declaration
Function FormatBuf(Var Buffer; BufLen : Cardinal; Const Fmt; fmtLen : Cardinal; Const Args : Array of const) : Cardinal;
\Description
\var{Format}
\Errors
\SeeAlso
\end{function}

\latex{\lstinputlisting{sysutex/ex72.pp}}
\html{\input{sysutex/ex72.tex}}

\begin{function}{IntToHex}
\Declaration
Function IntToHex(Value: integer; Digits: integer): string;
\Description
\var{IntToHex} converts \var{Value} to a hexadecimal string
representation. The result will contain at least \var{Digits} 
characters. If \var{Digits} is less than the needed number of characters,
the string will NOT be truncated. If \var{Digits} is larger than the needed
number of characters, the result is padded with zeroes.
\Errors
None.
\SeeAlso
\seef{IntToStr}, \var{StrToInt}
\end{function}

\latex{\lstinputlisting{sysutex/ex73.pp}}
\html{\input{sysutex/ex73.tex}}

\begin{function}{IntToStr}
\Declaration
Function IntToStr(Value: integer): string;
\Description
\var{IntToStr} coverts \var{Value} to it's string representation.
The resulting string has only as much characters as needed to represent
the value. If the value is negative a minus sign is prepended to the 
string.
\Errors
None.
\SeeAlso
\seef{IntToHex}, \seef{StrToInt}
\end{function}

\latex{\lstinputlisting{sysutex/ex74.pp}}
\html{\input{sysutex/ex74.tex}}
 
\begin{function}{IsValidIdent}
\Declaration
Function IsValidIdent(const Ident: string): boolean;
\Description
\var{IsValidIdent} returns \var{True} if \var{Ident} can be used as a
compoent name. It returns \var{False} otherwise. \var{Ident} must consist of
a letter or underscore, followed by a combination of letters, numbers or
underscores to be a valid identifier.
\Errors
None.
\SeeAlso
\end{function}

\latex{\lstinputlisting{sysutex/ex75.pp}}
\html{\input{sysutex/ex75.tex}}
 
\begin{function}{LeftStr}
\Declaration
Function LeftStr(const S: string; Count: integer): string;
\Description
\var{LeftStr} returns the \var{Count} leftmost characters of \var{S}.
It is equivalent to a call to \var{Copy(S,1,Count)}.
\Errors
None.
\SeeAlso
\seef{RightStr}, \seef{TrimLeft}, \seef{TrimRight}, \seef{Trim}
\end{function}

 \latex{\lstinputlisting{sysutex/ex76.pp}}
\html{\input{sysutex/ex76.tex}}

\begin{function}{LoadStr}
\Declaration
Function LoadStr(Ident: integer): string;
\Description
This function is not yet implemented. resources are not yet supported.
\Errors
\SeeAlso
\end{function}

\begin{function}{LowerCase}
\Declaration
Function LowerCase(const s: string): string;
\Description
\var{LowerCase} returns the lowercase equivalent of \var{S}. Ansi characters
are not taken into account, only ASCII codes below 127 are converted. It is 
completely equivalent to the lowercase function of the system unit, and is
provided for compatiibility only.
\Errors
None.
\SeeAlso
\seef{AnsiLowerCase}, \seef{UpperCase}, \seef{AnsiUpperCase}
\end{function}

\latex{\lstinputlisting{sysutex/ex77.pp}}
\html{\input{sysutex/ex77.tex}}
 
\begin{functionl}{NewStr}{NewStrSys}
\Declaration
Function NewStr(const S: string): PString;
\Description
\var{NewStr} assigns a new dynamic string on the heap, copies \var{S} into
it, and returns a pointer to the newly assigned string.

This function is obsolete, and shouldn't be used any more. The
\var{AnsiString} mechanism also allocates ansistrings on the heap, and
should be preferred over this mechanism.
\Errors
If not enough memory is present, an EOutOfMemory exception will be raised.
\SeeAlso
\seep{AssignStr}, \seepl{DisposeStr}{DisposeStrSys}
\end{functionl}

For an example, see \seep{AssignStr}.
 
\begin{function}{QuotedStr}
\Declaration
Function QuotedStr(const S: string): string;
\Description
\var{QuotedStr} returns the string \var{S}, quoted with single quotes. This means
that \var{S} is enclosed in single quotes, and every single quote in \var{S} 
is doubled. It is equivalent to a call to \var{AnsiQuotedStr(s, '''')}.
\Errors
None.
\SeeAlso
\seef{AnsiQuotedStr}, \seef{AnsiExtractQuotedStr}.
\end{function}

\latex{\lstinputlisting{sysutex/ex78.pp}}
\html{\input{sysutex/ex78.tex}}
 
 
\begin{function}{RightStr}
\Declaration
Function RightStr(const S: string; Count: integer): string;
\Description
\var{RightStr} returns the \var{Count} rightmost characters of \var{S}.
It is equivalent to a call to \var{Copy(S,Length(S)+1-Count,Count)}.

If \var{Count} is larger than the actual length of \var{S} only the real
length will be used.
\Errors
None.
\SeeAlso
\seef{LeftStr},\seef{Trim}, \seef{TrimLeft}, \seef{TrimRight}
\end{function}

\latex{\lstinputlisting{sysutex/ex79.pp}}
\html{\input{sysutex/ex79.tex}}
 
\begin{function}{StrFmt}
\Declaration
Function StrFmt(Buffer,Fmt : PChar; Const args: Array of const) : Pchar;
\Description
\var{StrFmt} will format \var{fmt} with \var{Args}, as the \seef{Format} 
function does, and it will store the result in \var{Buffer}. The function
returns \var{Buffer}. \var{Buffer} should point to enough space to contain
the whole result.
\Errors
for a list of errors, see \seef{Format}.
\SeeAlso
\seef{StrLFmt}, \seep{FmtStr}, \seef{Format}, \seef{FormatBuf}
\end{function}

\latex{\lstinputlisting{sysutex/ex80.pp}}
\html{\input{sysutex/ex80.tex}}
 
\begin{function}{StrLFmt}
\Declaration
Function StrLFmt(Buffer : PCHar; Maxlen : Cardinal;Fmt : PChar; Const args: Array of const) : Pchar;
\Description
\var{StrLFmt} will format \var{fmt} with \var{Args}, as the \seef{Format} 
function does, and it will store maximally \var{Maxlen characters} of the 
result in \var{Buffer}. The function returns \var{Buffer}. \var{Buffer} 
should point to enough space to contain \var{MaxLen} characters.
\Errors
for a list of errors, see \seef{Format}.
\SeeAlso
\seef{StrFmt}, \seep{FmtStr}, \seef{Format}, \seef{FormatBuf}
\end{function}

\latex{\lstinputlisting{sysutex/ex81.pp}}
\html{\input{sysutex/ex81.tex}}
 
\begin{function}{StrToInt}
\Declaration
Function StrToInt(const s: string): integer;
\Description
\var{StrToInt} will convert the string \var{S}to an integer. 
If the string contains invalid characters or has an invalid format, 
then an \var{EConvertError} is raised. 

To be successfully converted, a string can contain a combination
of \var{numerical} characters, possibly preceded by a minus sign (\var{-}).
Spaces are not allowed.
\Errors
In case of error, an \var{EConvertError} is raised.
\SeeAlso
\seef{IntToStr}, \seef{StrToIntDef}
\end{function}

\latex{\lstinputlisting{sysutex/ex82.pp}}
\html{\input{sysutex/ex82.tex}}
 
\begin{function}{StrToIntDef}
\Declaration
Function StrToIntDef(const S: string; Default: integer): integer;
\Description
\var{StrToIntDef} will convert a string to an integer. If the string contains
invalid characters or has an invalid format, then \var{Default} is returned.

To be successfully converted, a string can contain a combination of 
\var{numerical} characters, possibly preceded by a minus sign (\var{-}).
Spaces are not allowed.
\Errors
None.
\SeeAlso
\seef{IntToStr}, \seef{StrToInt}
\end{function}

\latex{\lstinputlisting{sysutex/ex83.pp}}
\html{\input{sysutex/ex83.tex}}
 
\begin{function}{Trim}
\Declaration
Function Trim(const S: string): string;
\Description
\var{Trim} strips blank characters (spaces) at the beginning and end of \var{S}
and returns the resulting string. Only \var{\#32} characters are stripped.

If the string contains only spaces, an empty string is returned.
\Errors
None.
\SeeAlso
\seef{TrimLeft}, \seef{TrimRight}
\end{function}

\latex{\lstinputlisting{sysutex/ex84.pp}}
\html{\input{sysutex/ex84.tex}}
 
\begin{function}{TrimLeft}
\Declaration
Function TrimLeft(const S: string): string;
\Description
\var{TrimLeft} strips blank characters (spaces) at the beginning of \var{S}
and returns the resulting string. Only \var{\#32} characters are stripped.

If the string contains only spaces, an empty string is returned.
\Errors
None.
\SeeAlso
\seef{Trim}, \seef{TrimRight}
\end{function}

\latex{\lstinputlisting{sysutex/ex85.pp}}
\html{\input{sysutex/ex85.tex}}
 
\begin{function}{TrimRight}
\Declaration
Function TrimRight(const S: string): string;
\Description
\var{Trim} strips blank characters (spaces) at the end of \var{S}
and returns the resulting string. Only \var{\#32} characters are stripped.

If the string contains only spaces, an empty string is returned.
\Errors
None.
\SeeAlso
\seef{Trim}, \seef{TrimLeft}
\end{function}

\latex{\lstinputlisting{sysutex/ex86.pp}}
\html{\input{sysutex/ex86.tex}}

 
\begin{function}{UpperCase}
\Declaration
Function UpperCase(const s: string): string;
\Description
\var{UpperCase} returns the uppercase equivalent of \var{S}. Ansi characters
are not taken into account, only ASCII codes below 127 are converted. It is 
completely equivalent to the \var{UpCase} function of the system unit, and is
provided for compatiibility only.
\Errors
None.
\SeeAlso
\seef{AnsiLowerCase}, \seef{LowerCase}, \seef{AnsiUpperCase}
\Errors
\SeeAlso
\end{function}

\latex{\lstinputlisting{sysutex/ex87.pp}}
\html{\input{sysutex/ex87.tex}}
 
