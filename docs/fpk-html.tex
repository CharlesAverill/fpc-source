%
%   $Id$
%   This file is part of the FPC documentation.
%   Copyright (C) 1997, by Michael Van Canneyt
%
%   The FPC documentation is free text; you can redistribute it and/or
%   modify it under the terms of the GNU Library General Public License as
%   published by the Free Software Foundation; either version 2 of the
%   License, or (at your option) any later version.
%
%   The FPC Documentation is distributed in the hope that it will be useful,
%   but WITHOUT ANY WARRANTY; without even the implied warranty of
%   MERCHANTABILITY or FITNESS FOR A PARTICULAR PURPOSE.  See the GNU
%   Library General Public License for more details.
%
%   You should have received a copy of the GNU Library General Public
%   License along with the FPC documentation; see the file COPYING.LIB.  If not,
%   write to the Free Software Foundation, Inc., 59 Temple Place - Suite 330,
%   Boston, MA 02111-1307, USA. 
%
% Dummy
\newenvironment{FPKList}{\begin{description}}{\end{description}}
\newcommand{\functionl}[7]{
\subsection{#1}
\label{fu:#2}
\index{#1}
\subsubsection*{Declaration:}
\texttt {Function #1  #3  : #4;}
\subsubsection*{Description:}
#5
\subsubsection*{Errors:}
#6
\subsubsection*{See also:}
#7
}
\newcommand{\procedurel}[6]{
\subsection{#1}
\label{pro:#2}
\index{#1}
\subsubsection*{Declaration:}
\texttt {Procedure #1  #3 ;}
\subsubsection*{Description:}
#4
\subsubsection*{Errors:}
#5
\subsubsection*{See also:}
#6
}
\newcommand{\seefl}[2]{
\htmlref{#1}{fu:#2}
}
\newcommand{\seepl}[2]{
\htmlref{#1}{pro:#2}
}
%
% Now the ones without label.
%
\newcommand{\seef}[1]{\seefl{#1}{#1}}
\newcommand{\seep}[1]{\seepl{#1}{#1}}
%
\newcommand{\seet}[1]{
\htmlref{#1}{sec:types}
}
\newcommand{\seem}[2] {\texttt{#1} (#2) }
\newcommand{\var}[1]{\texttt {#1}}
\newcommand{\file}[1]{\textsf {#1}}
%
% procedures without args
%
\newcommand{\Procedurel}[5]{\procedurel{#1}{#2}{}{#3}{#4}{#5}}
\newcommand{\Functionl}[6]{\functionl{#1}{#2}{}{#3}{#4}{#5}{#6}}
%
% Procedures without labels.
%
\newcommand{\procedure}[5]{\procedurel{#1}{#1}{#2}{#3}{#4}{#5}}
\newcommand{\function}[6]{\functionl{#1}{#1}{#2}{#3}{#4}{#5}{#6}}
\newcommand{\Procedure}[4]{\procedure{#1}{}{#2}{#3}{#4}}
\newcommand{\Function}[5]{\function{#1}{}{#2}{#3}{#4}{#5}}
\newcommand{\linux}{\textsc{LinuX} }
\newcommand{\dos}  {\textsc{dos} }
\newcommand{\msdos}{\textsc{ms-dos} }
\newcommand{\ostwo}{\textsc{os/2} }
\newcommand{\windowsnt}{\textsc{WindowsNT} }
\newcommand{\docdescription}[1]{}
\newcommand{\docversion}[1]{}
\newcommand{\unitdescription}[1]{}
\newcommand{\unitversion}[1]{}
\newcommand{\fpk}{Free Pascal }
\newcommand{\gnu}{gnu }
%
% Useful references.
%
\newcommand{\progref}{\htmladdnormallink{Programmer's guide}{../prog/prog.html}\ }
\newcommand{\refref}{\htmladdnormallink{Reference guide}{../ref/ref.html}\ }
\newcommand{\userref}{\htmladdnormallink{Users' guide}{../user/user.html}\ }
\newcommand{\seecrt}{\htmladdnormallink{CRT}{../crt/crt.html}}
\newcommand{\seelinux}{\htmladdnormallink{Linux}{../linux/linux.html}}
\newcommand{\seestrings}{\htmladdnormallink{strings}{../strings/strings.html}}
\newcommand{\seedos}{\htmladdnormallink{DOS}{../dos/dos.html}}
\newcommand{\seegetopts}{\htmladdnormallink{getopts}{../getopts/getopts.html}}
\newcommand{\seeobjects}{\htmladdnormallink{objects}{../objects/objects.html}}
\newcommand{\seegraph}{\htmladdnormallink{graph}{../graph/graph.html}}
\newcommand{\seeprinter}{\htmladdnormallink{printer}{../printer/printer.html}}
\newcommand{\seego}{\htmladdnormallink{GO32}{../go32/go32.html}}
%
% Nice environments
%
% For Code examples (complete programs only)
\newenvironment{CodEx}{}{}
% For Tables.
\newenvironment{FPKtable}[2]{\begin{table}\caption{#2}\begin{center}\begin{tabular}{#1}}{\end{tabular}\end{center}\end{table}}
% The same, but with label in third argument (tab:#3)
\newenvironment{FPKltable}[3]{\begin{table}\caption{#2}\label{tab:#3}\begin{center}\begin{tabular}{#1}}{\end{tabular}\end{center}\end{table}}
%
% Commands to reference these things.
%
\newcommand{\seet}[1]{table (\ref{tab:#1}) }
\newcommand{\seec}[1]{chapter (\ref{ch:#1}) }
\newcommand{\sees}[1]{section (\ref{se:#1}) }
