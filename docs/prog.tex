%
%   $Id$
%   This file is part of the FPC documentation.
%   Copyright (C) 1997, by Michael Van Canneyt
%
%   The FPC documentation is free text; you can redistribute it and/or
%   modify it under the terms of the GNU Library General Public License as
%   published by the Free Software Foundation; either version 2 of the
%   License, or (at your option) any later version.
%
%   The FPC Documentation is distributed in the hope that it will be useful,
%   but WITHOUT ANY WARRANTY; without even the implied warranty of
%   MERCHANTABILITY or FITNESS FOR A PARTICULAR PURPOSE.  See the GNU
%   Library General Public License for more details.
%
%   You should have received a copy of the GNU Library General Public
%   License along with the FPC documentation; see the file COPYING.LIB.  If not,
%   write to the Free Software Foundation, Inc., 59 Temple Place - Suite 330,
%   Boston, MA 02111-1307, USA. 
%
\documentclass{report}
\usepackage{a4}
\usepackage{html}
\latex{\usepackage{multicol}}
\latex{\usepackage{fpcman}}
\usepackage{fancyheadings}
\pagestyle{fancy}
\renewcommand{\chaptermark}[1]{\markboth{#1}{}}
\html{%
%   $Id$
%   This file is part of the FPC documentation.
%   Copyright (C) 1997, by Michael Van Canneyt
%
%   The FPC documentation is free text; you can redistribute it and/or
%   modify it under the terms of the GNU Library General Public License as
%   published by the Free Software Foundation; either version 2 of the
%   License, or (at your option) any later version.
%
%   The FPC Documentation is distributed in the hope that it will be useful,
%   but WITHOUT ANY WARRANTY; without even the implied warranty of
%   MERCHANTABILITY or FITNESS FOR A PARTICULAR PURPOSE.  See the GNU
%   Library General Public License for more details.
%
%   You should have received a copy of the GNU Library General Public
%   License along with the FPC documentation; see the file COPYING.LIB.  If not,
%   write to the Free Software Foundation, Inc., 59 Temple Place - Suite 330,
%   Boston, MA 02111-1307, USA. 
%
% Dummy
\newenvironment{FPCList}{\begin{description}}{\end{description}}
%
%
\newcommand{\Declaration}{\item[Declaration]\ttfamily}
\newcommand{\Description}{\item[Description]\rmfamily}
\newcommand{\Errors}{\item[Errors]\rmfamily}
\newcommand{\SeeAlso}{\item[See also]\rmfamily}
%
%  The environments
%
\newenvironment{functionl}[2]{\subsection{#1}%
\index{#1}\label{fu:#2}\begin{FPCList}}{\end{FPCList}}
\newenvironment{procedurel}[2]{\subsection{#1}%
\index{#1}\label{pro:#2}\begin{FPCList}}{\end{FPCList}}
\newenvironment{function}[1]{\begin{functionl}{#1}{#1}}{\end{functionl}}
\newenvironment{procedure}[1]{\begin{procedurel}{#1}{#1}}{\end{procedurel}}
\newcommand{\seefl}[2]{
\htmlref{#1}{fu:#2}
}
\newcommand{\seepl}[2]{
\htmlref{#1}{pro:#2}
}
%
% Now the ones without label.
%
\newcommand{\seef}[1]{\seefl{#1}{#1}}
\newcommand{\seep}[1]{\seepl{#1}{#1}}
%
\newcommand{\seet}[1]{
\htmlref{#1}{sec:types}
}
\newcommand{\seem}[2] {\texttt{#1} (#2) }
\newcommand{\var}[1]{\texttt {#1}}
\newcommand{\file}[1]{\textsf {#1}}
%
% Abbreviations
%
\newcommand{\linux}{\textsc{LinuX} }
\newcommand{\dos}  {\textsc{dos} }
\newcommand{\msdos}{\textsc{ms-dos} }
\newcommand{\ostwo}{\textsc{os/2} }
\newcommand{\windowsnt}{\textsc{WindowsNT} }
\newcommand{\windows}{\textsc{Windows} }
\newcommand{\docdescription}[1]{}
\newcommand{\docversion}[1]{}
\newcommand{\unitdescription}[1]{}
\newcommand{\unitversion}[1]{}
\newcommand{\fpc}{Free Pascal }
\newcommand{\gnu}{gnu }
%
% Useful references.
%
\newcommand{\progref}{\htmladdnormallink{Programmer's guide}{../prog/prog.html}\ }
\newcommand{\refref}{\htmladdnormallink{Reference guide}{../ref/ref.html}\ }
\newcommand{\userref}{\htmladdnormallink{Users' guide}{../user/user.html}\ }
\newcommand{\unitsref}{\htmladdnormallink{Unit reference}{../units/units.html}\ }
\newcommand{\seecrt}{\htmladdnormallink{CRT}{../crt/crt.html}}
\newcommand{\seelinux}{\htmladdnormallink{Linux}{../linux/linux.html}}
\newcommand{\seestrings}{\htmladdnormallink{strings}{../strings/strings.html}}
\newcommand{\seedos}{\htmladdnormallink{DOS}{../dos/dos.html}}
\newcommand{\seegetopts}{\htmladdnormallink{getopts}{../getopts/getopts.html}}
\newcommand{\seeobjects}{\htmladdnormallink{objects}{../objects/objects.html}}
\newcommand{\seegraph}{\htmladdnormallink{graph}{../graph/graph.html}}
\newcommand{\seeprinter}{\htmladdnormallink{printer}{../printer/printer.html}}
\newcommand{\seego}{\htmladdnormallink{GO32}{../go32/go32.html}}
%
% Nice environments
%
% For Code examples (complete programs only)
\newenvironment{CodEx}{}{}
% For Tables.
\newenvironment{FPCtable}[2]{\begin{table}\caption{#2}\begin{center}\begin{tabular}{#1}}{\end{tabular}\end{center}\end{table}}
% The same, but with label in third argument (tab:#3)
\newenvironment{FPCltable}[3]{\begin{table}\caption{#2}\label{tab:#3}\begin{center}\begin{tabular}{#1}}{\end{tabular}\end{center}\end{table}}
%
% Commands to reference these things.
%
\newcommand{\seet}[1]{table (\ref{tab:#1}) }
\newcommand{\seec}[1]{chapter (\ref{ch:#1}) }
\newcommand{\sees}[1]{section (\ref{se:#1}) }
}
% define the version number here, and not in the fpc.sty !!!
\newcommand{\remark}[1]{\par$\rightarrow$\textbf{#1}\par}
\newcommand{\olabel}[1]{\label{option:#1}}
% We should change this to something better. See \seef etc.
\begin{document}
\title{Free Pascal \\ Programmers' manual}
\docdescription{Programmers' manual for \fpc, version \fpcversion}
\docversion{1.4}
\date{July 1998}
\author{Micha\"el Van Canneyt}
\maketitle
\tableofcontents
\newpage

%%%%%%%%%%%%%%%%%%%%%%%%%%%%%%%%%%%%%%%%%%%%%%%%%%%%%%%%%%%%%%%%%%%%%
% Introduction
%%%%%%%%%%%%%%%%%%%%%%%%%%%%%%%%%%%%%%%%%%%%%%%%%%%%%%%%%%%%%%%%%%%%%
\section*{About this document}
This is the programmer's manual for \fpc.

It describes some of the peculiarities of the \fpc compiler, and provides a
glimpse of how the compiler generates its code, and how you can change the
generated code. It will not, however, provide you with a detailed account of
the inner workings of the compiler, nor will it tell you how to use the
compiler (described in the \userref). It also will not describe the inner
workings of the Run-Time Library (RTL). The best way to learn about the way
the RTL is implemented is from the sources themselves.

The things described here are useful if you want to do things which need
greater flexibility than the standard Pascal language constructs.
(described in the \refref)

Since the compiler is continuously under development, this document may get
out of date. Wherever possible, the information in this manual will be
updated. If you find something which isn't correct, or you think something
 is missing, feel free to contact me\footnote{at
\var{michael@tfdec1.fys.kuleuven.ac.be}}.

%%%%%%%%%%%%%%%%%%%%%%%%%%%%%%%%%%%%%%%%%%%%%%%%%%%%%%%%%%%%%%%%%%%%%
% Compiler switches
%%%%%%%%%%%%%%%%%%%%%%%%%%%%%%%%%%%%%%%%%%%%%%%%%%%%%%%%%%%%%%%%%%%%%
\chapter{Compiler directives}
\label{ch:CompSwitch}

\fpc supports compiler directives in your source file. They are not the same
as Turbo Pascal directives, although some are supported for compatibility.
There is a distinction between local and global directives; local directives
take effect from the moment they are encountered, global directives have an
effect on all of the compiled code.

Many switches have a long form also. If they do, then the name of the
long form is given also. For long switches, the + or - character to switch
the option on or off, may be replaced by \var{ON} or \var{OFF} keywords.

Thus \verb|{$I+}| is equivalent to \verb|{$IOCHECKS ON}| or 
\verb|{$IOCHECKS +}| and 
\verb|{$C-}| is equivalent to \verb|{$ASSERTIONS OFF}| or
\verb|{$ASSERTIONS -}|

The long forms of the switches are the same as their Delphi 
counterparts.

%%%%%%%%%%%%%%%%%%%%%%%%%%%%%%%%%%%%%%%%%%%%%%%%%%%%%%%%%%%%%%%%%%%%%
% Local switches
\section{Local directives}
\label{se:LocalSwitch}
Local directives have no command-line counterpart. They influence the
compiler's behaviour from the moment they're encountered until the moment
another switch annihilates their behaviour, or the end of the unit or
program is reached.

\subsection{\var{\$DEFINE} : Define a symbol}

The directive
\begin{verbatim}
{$DEFINE name}
\end{verbatim}
defines the symbol \var{name}. This symbol remains defined until the end of
the current module, or until a \var{\$UNDEF name} directive is encountered.

If \var{name} is already defined, this has no effect. \var{Name} is case
insensitive.


\subsection{\var{\$ELSE} : Switch conditional compilation}

The \var{\{\$ELSE \}} switches between compiling and ignoting the source
text delimited by the preceding \var{\{\$IFxxx\}} and following 
\var{\{\$ENDIF\}}. Any text after the \var{ELSE} keyword but before the 
brace is ignored: 
\begin{verbatim}
{$ELSE some ignored text}
\end{verbatim}
is the same as
\begin{verbatim}
{$ELSE}
\end{verbatim}
This is useful for indication what switch is meant.

\subsection{\var{\$ENDIF} : End conditional compilation}

The \var{\{\$ENDIF\}} directive ends the conditional compilation initiated by the
last \var{\{\$IFxxx\}} directive. Any text after the \var{ENDIF} keyword but
before the closing brace is ignored:
\begin{verbatim}
{$ENDIF some ignored text}
\end{verbatim}
is the same as
\begin{verbatim}
{$ENDIF}
\end{verbatim}
This is useful for indication what switch is meant to be ended.

\subsection{\var{\$ERROR} : Generate error message}

The following code
\begin{verbatim}
{$ERROR This code is erroneous !}
\end{verbatim}
will display an error message when the compiler encounters it, 
and increase the error count of the compiler. 
The compiler will continue to compile, but no code will be emitted.

\subsection{\var{\$F} : Far or near functions}
This directive is recognized for compatibility with Turbo Pascal. Under the
32-bit programming model, the concept of near and far calls have no meaning,
hence the directive is ignored. A warning is printed to the screen, telling
you so.

As an example, : the following piece of code :
\begin{verbatim}
{$F+}

Procedure TestProc;

begin
 Writeln ('Hello From TestProc');
end;

begin
 testProc
end.
\end{verbatim}
Generates the following compiler output:
\begin{verbatim}
malpertuus: >pp -vw testf
Compiler: ppc386
Units are searched in: /home/michael;/usr/bin/;/usr/lib/ppc/0.9.1/linuxunits
Target OS: Linux
Compiling testf.pp
testf.pp(1) Warning: illegal compiler switch
7739 kB free
Calling assembler...
Assembled...
Calling linker...
12 lines compiled,
 1.00000000000000E+0000
\end{verbatim}
You can see that the verbosity level was set to display warnings.

If you declare a function as \var{Far} (this has the same effect as setting it
between \var{\{\$F+\}...\{\$F-\}} directives), the compiler also generates a
warning :
\begin{verbatim}
testf.pp(3) Warning: FAR ignored
\end{verbatim}

The same story is true for procedures declared as \var{Near}. The warning
displayed in that case is:
\begin{verbatim}
testf.pp(3) Warning: NEAR ignored
\end{verbatim}

\subsection{\var{\$FATAL} : Generate fatal error message}

The following code
\begin{verbatim}
{$FATAL This code is erroneous !}
\end{verbatim}
will display an error message when the compiler encounters it, and trigger
and increase the error count of the compiler. 
The compiler will immediatly stop the compilation process.


\subsection{\var{\$H} or \var{\$LONGSTRINGS} : Use AnsiStrings}

If \var{\{\$LONGSTRINGS ON\}} is specified, the keyword \var{String} (no
length specifier) will be treated as \var{AnsiString}, and the compiler 
will treat the corresponding varible as an ansistring, and will 
generate corresponding code. 

By default, the use of ansistrings is off, corresponding to \var{\{\$H-\}}.

This feature is still experimental, and should be used with caution for the
time being.

\subsection{\var{\$HINT} : Generate hint message}

If the generation of hints is turned on, through the \var{-vh} command-line
option or the \var{\{\$HINTS ON\}} directive, then
\begin{verbatim}
{$Hint This code should be optimized }
\end{verbatim}
will display a hint message when the compiler encounters it.

\subsection{\var{\$HINTS} : Emit hints}

\var{\{\$HINTS ON\}} switches the generation of hints on. 
\var{\{\$HINTS OFF\}} switches the generation of hints off. 
Contrary to the command-line option \var{-vh} this is a local switch, 
this is useful for checking parts of your code.

\subsection{\var{\$IF} : Start conditional compilation}

The directive \var{\{\$IF expr\}} will continue the compilation
if the boolean expression \var{expr} evaluates to \var{true}. If the 
compilation evaluates to false, then the source are skipped to the first
\var{\{\$ELSE\}} or \var{\{\$ENDIF\}} directive.

The compiler must be able to evaluate the expression at compile time.
This means that you cannot use variables or constants that are defined in
the source. Macros and symbols may be used, however.   

More information on this can be found in the section about 
conditionals.

\subsection{\var{\$IFDEF} : Start conditional compilation}

The \var{\{\$IFDEF name\}} will skip the compilation of the text that
follows it if the symbol \var{name} is not defined. If it is defined, then
compilation continues as if the directive wasn't there. 

\subsection{\var{\$IFNDEF} : Start conditional compilation}

The \var{\{\$IFNDEF name\}} will skip the compilation of the text that
follows it if the symbol \var{name} is defined. If it is not defined, then
compilation continues as if the directive wasn't there. 

\subsection{\var{\$IFOPT} : Start conditional compilation}

The \var{\{\$IFOPT switch\}} will skip the compilation of the text that
follows it if the switch \var{switch} is currently not in the specified 
state.  If it is in the specified state, then compilation continues as if the directive 
wasn't there. 

As an example:
\begin{verbatim}
{$IFOPT M+}
  Writeln ('Compiled with type information');
{$ENDIF}
\end{verbatim}
Will compile the writeln statement if generation of type information is on.

\subsection{\var{\$INFO} : Generate info message}

If the generation of info is turned on, through the \var{-vi} command-line
option, then 
\begin{verbatim}
{$INFO This was coded on a rainy day by Bugs Bunny }
\end{verbatim}
will display an info message when the compiler encounters it.


\subsection{\var{\$I} or \var{\$IOCHECK} : Input/Output checking}
The \var{\{\$I-\}} or \var{\{\$IOCHECK OFF\}} directive tells the compiler 
not to generate input/output checking code in your program. By default, the
compiler does not generate this code, you must switch it on using the
\var{-Ci} command-lne switch.

If you compile using the \var{-Ci} compiler switch, the \fpc compiler inserts input/output
checking code after every input/output call in your program. If an error
occurred during input or output, then a run-time error will be generated.
Use this switch if you wish to avoid this behavior.
If you still want to check if something went wrong, you can use the
\var{IOResult} function to see if everything went without problems.

Conversely, \var{\{\$I+\}} will turn error-checking back on, until another
directive is encountered which turns it off again.

The most common use for this switch is to check if the opening of a file
went without problems, as in the following piece of code:
\begin{verbatim}
...
assign (f,'file.txt');
{$I-}
rewrite (f);
{$I+}
if IOResult<>0 then
  begin
  Writeln ('Error opening file : "file.txt"');
  exit
  end;
...
\end{verbatim}

\subsection{\var{\$I} or \var{\$INCLUDE} : Include file }
The \var{\{\$I filename\}} or \var{\{\$INCLUDE filename\}} directive 
tells the compiler to read further statements from the file \var{filename}. 
The statements read there will be inserted as if they occurred in the 
current file.

The compiler will append the \file{.pp} extension to the file if you don't
specify an extension yourself. Do not put the filename between quotes, as
they will be regarded as part of the file's name.

You can nest included files, but not infinitely deep. The number of files is
restricted to the number of file descriptors available to the \fpc compiler.

Contrary to Turbo Pascal, include files can cross blocks. I.e. you can start
a block in one file (with a \var{Begin} keyword) and end it in another (with
a \var{End} keyword). The smallest entity in an include file must be a token,
i.e. an identifier, keyword or operator.

The compiler will look for the file to include in the following places:

\begin{enumerate}
\item It will look in the path specified in the incude file name. 
\item It will look in the directory where the current source file is.
\item it will look in all directories specified in the include file search
path.
\end{enumerate}
You can add files to the include file search path with the \var{-I} 
command-line option.

\subsection{\var{\$I} or \var{\$INCLUDE} : Include compiler info}

In this form:
\begin{verbatim}
{$INCLUDE %xxx%}
\end{verbatim}
where \var{xxx} is one of \var{TIME}, \var{DATE}, \var{FPCVERSION} or
\var{FPCTARGET}, will generate a macro with the value of these things.
If \var{xxx} is none of the above, then it is assumed to be the value of
an environment variable. It's value will be fetched.

% Assembler type
\subsection{\var{\$I386\_XXX} : Specify assembler format}
This switch can only be used in the i386 assembler.

This switch informs the compiler what kind of assembler it can expect in an
\var{asm} block. The \var{XXX} should be replaced by one of the following:
\begin{description}
\item [att\ ] Indicates that \var{asm} blocks contain AT\&T syntax assembler.
\item [intel\ ] Indicates that \var{asm} blocks contain Intel syntax
assembler.
\item [direct\ ] Tells the compiler that asm blocks should be copied
directly to the assembler file.
\end{description}
These switches are local, and retain their value to the end of the unit that
is compiled, unless they are replaced by another directive of the same type.
The command-line switch that corresponds to this switch is \var{-R}.

\subsection{\var{\$L} or \var{\$LINK} : Link object file}
The \var{\{\$L filename\}} or \var{\{\$LINK filename\}} directive 
tells the compiler that the file \file{filename} should be linked to 
your program. 

the compiler will look for this file in the following way:

\begin{enumerate}
\item It will look in the path specified in the object file name. 
\item It will look in the directory where the current source file is.
\item it will look in all directories specified in the object file search path.
\end{enumerate}
You can add files to the object file search path with the \var{-Fo}
option.

On \linux systems, the name is case sensitive, and must be typed 
exactly as it appears on your system.

{\em Remark :} Take care that the object file you're linking is in a
format the linker understands. Which format this is, depends on the platform
you're on. Typing \var{ld} on the command line gives a list of formats
\var{ld} knows about.

You can pass other files and options to the linker using the \var{-k}
command-line option. You can specify more than one of these options, and
they will be passed to the linker, in the order that you specified them on
the command line, just before the names of the object files that must be
linked.

\subsection{\var{\$LINKLIB} : Link to a library}
The \var{\{\$LINKLIB name\}} will link to a library \file{name}.
This has the effect of passing \var{-lname} to the linker. 

As an example, consider the following unit:
\begin{verbatim}
unit getlen;

interface
{$LINKLIB c}

function strlen (P : pchar) : longint;cdecl;

implementation

function strlen (P : pchar) : longint;cdecl;external;

end.
\end{verbatim}
If one would issue the command the command
\begin{verbatim}
ppc386 foo.pp
\end{verbatim}
where foo.pp has the above unit in its \var{uses} clause, 
then the compiler would link your program to the c library, by passing the
linker the \var{-lc} option.

The same effect could be obtained by removing the linklib directive in the
above unit, and specify \var{-k-lc} on the command-line:
\begin{verbatim}
ppc386 -k-lc foo.pp
\end{verbatim}

\subsection{\var{\$M} or \var{\$TYPEINFO} : Generate type info}

This switch is recognized for Delphi compatibility only since the generation
of type information isn't fully implemented yet.
 
\subsection{\var{\$MMX} : Intel MMX support}
As of version 0.9.8, \fpc supports optimization for the \textbf{MMX} Intel
processor (see also \ref{ch:MMXSupport}). This optimizes certain code parts for the \textbf{MMX} Intel
processor, thus greatly improving speed. The speed is noticed mostly when
moving large amounts of data. Things that change are
\begin{itemize}
\item Data with a size that is a multiple of 8 bytes is moved using the
\var{movq} assembler instruction, which moves 8 bytes at a time
\end{itemize}

When \textbf{MMX} support is on, you aren't allowed to do floating point
arithmetic. You are allowed to move floating point data, but no arithmetic
can be done. If you wish to do floating point math anyway, you must first
switch of \textbf{MMX} support and clear the FPU using the \var{emms}
function of the \file{cpu} unit.

The following example will make this more clear:
\begin{verbatim}
Program MMXDemo;

uses cpu;

var
   d1 : double;
   a : array[0..10000] of double;
   i : longint;

begin
   d1:=1.0;
{$mmx+}
   { floating point data is used, but we do _no_ arithmetic }
   for i:=0 to 10000 do
     a[i]:=d2;  { this is done with 64 bit moves }
{$mmx-}
   emms;   { clear fpu }
   { now we can do floating point arithmetic }
   ....
end.
\end{verbatim}
See, however, the chapter on MMX (\ref{ch:MMXSupport}) for more information
on this topic.

\subsection{\var{\$NOTE} : Generate note message}

If the generation of notes is turned on, through the \var{-vn} command-line
option or the \var{\{\$NOTES ON\}} directive, then
\begin{verbatim}
{$NOTE Ask Santa Claus to look at this code }
\end{verbatim}
will display a note message when the compiler encounters it.


\subsection{\var{\$NOTES} : Emit notes}

\var{\{\$NOTES ON\}} switches the generation of notes on. 
\var{\{\$NOTES OFF\}} switches the generation of notes off.
 Contrary to the command-line option \var{-vn} this
is a local switch, this is useful for checking parts of your code.

\subsection{\var{\$OUTPUT\_FORMAT} : Specify the output format}
\var{\{\$OUTPUT\_FORMAT format\}} has the same functionality as the \var{-A}
command-line option : It tells the compiler what kind of object file must be
generated. You can specify this switch \textbf{only} befor the \var{Program}
or \var{Unit} clause in your source file. The different kinds of formats are
shown in \seet{Formats}.

\begin{FPCltable}{ll}{Formats generated by the x86 compiler}{Formats} \hline
Switch value & Generated format \\ \hline
att  & AT\&T assembler file. \\
o    & Unix object file.\\
obj  & OMF file.\\
wasm & assembler for the Watcom assembler. \\ \hline
\end{FPCltable}

\subsection{\var{\$P} or \var{\$OPENSTRINGS} : Use open strings}

This switch is provided for compatibility only, as open strings aren't
implemented yet.

\subsection{\var{\$PACKENUM} : Minimum enumeration type size}

This directive tells the compiler the minimum number of bytes it should
use when storing enumerated types. It is of the following form:
\begin{verbatim}
{$PACKENUM xxx}
{$MINENUMSIZE xxx}
\end{verbatim}
Where the form with \var{\$MINENUMSIZE} is for Delphi compatibility.
var{xxx} can be one of \var{1,2} or \var{4}, or \var{NORMAL} or
\var{DEFAULT}, corresponding to the default value of 4.


As an alternative form one can use \var{\{\$Z1\}}, \var{\{\$Z2\}} 
\var{\{\$Z4\}}. Contrary to Delphi, the default size is 4 bytes
(\var{\{\$Z4\}}).

So the follwoing code
\begin{verbatim}
{$PACKENUM 1}
Type
  Days = (monday, tuesday, wednesday, thursday, friday, 
          saturday, sunday);
\end{verbatim}
will use 1 byte to store a variable of type \var{Days}, wheras it nomally
would use 4 bytes. The above code is equivalent to
\begin{verbatim}
{$Z1}
Type
  Days = (monday, tuesday, wednesday, thursday, friday, 
          saturday, sunday);
\end{verbatim}

\subsection{\var{\$PACKRECORDS} : Alignment of record elements}

This directive controls the byte alignment of the elements in a record,
object or class type definition.

It is of the following form:
\begin{verbatim}
{$PACKRECORDS xx}
\end{verbatim}

Where \var{xxx} is one of 1,2,4,16 or \var{NORMAL} or \var{DEFAULT}. 
This means that the elements of a record will be aligned on 1,2, 4 or 
16 byte boundaries. Thus, the size of a record will always be a multiple of
the alignment size.
The default alignment (which can be selected with \var{DEFAULT}) is 2, 
contrary to Turbo Pascal, where it is 1.

More information of this can be found in the reference guide, in the section
about record types.

\subsection{\var{\$STOP} : Generate fatal error message}

The following code
\begin{verbatim}
{$STOP This code is erroneous !}
\end{verbatim}
will display an error message when the compiler encounters it.
The compiler will immediatly stop the compilation process.

It has the same effect as the \var{\{\$FATAL\}} directive.


\subsection{\var{\$UNDEF} : Undefine a symbol}

The directive
\begin{verbatim}
{$UNDEF name}
\end{verbatim}
un-defines the symbol \var{name} if it was previously defined.
\var{Name} is case insensitive.


\subsection{\var{\$V} or \var{\$VARSTRINGCHECKS} : Var-string checking}

When in the \var{+} or \var{ON} state, the compiler checks that strings 
passed as parameters are of the same, identical, string type as the declared
parameters of the procedure.

\subsection{\var{\$WAIT} : Wait for enter key press}

If the compiler encaounters a
\begin{verbatim}
{$WAIT }
\end{verbatim}
directive, it will resume compiling only after the user has pressed the
enter key. If the generation of info messages is turned on, then the compiler 
will display the follwing message:
\begin{verbatim}
Press <return> to continue
\end{verbatim}
before waiting for a keypress. Careful ! this may interfere with automatic
compilation processes. It should be used for debuggig purposes only.

\subsection{\var{\$WARNING} : Generate warning message}

If the generation of warnings is turned on, through the \var{-vw} 
command-line option or the \var{\{\$WARNINGS ON\}} directive, then
\begin{verbatim}
{$WARNING This is dubious code }
\end{verbatim}
will display a warning message when the compiler encounters it.

\subsection{\var{\$WARNINGS} : Emit warnings}

\var{\{\$WARNINGS ON\}} switches the generation of warnings on. 
\var{\{\$WARNINGS OFF\}} switches the generation of warnings off. 
Contrary to the command-line option \var{-vw} this
is a local switch, this is useful for checking parts of your code.

%%%%%%%%%%%%%%%%%%%%%%%%%%%%%%%%%%%%%%%%%%%%%%%%%%%%%%%%%%%%%%%%%%%%%
% Global switches
\section{Global directives}
\label{se:GlobalSwitch}
Global directives affect the whole of the compilation process. That is why
they also have a command - line counterpart. The command-line counterpart is
given for each of the directives.

\subsection{\var{\$A} or \var{\$ALIGN}: Align Data}

This switch is recognized for Turbo Pascal Compatibility, but is not
yet implemented. The alignment of data will be different in any case, since
\fpc is a 32-bit compiler.

\subsection{\var{\$B} or \var{\$BOOLEVAL}: Complete boolean evaluation}

This switch is understood by the \fpc compiler, but is ignored. The compiler
always uses shortcut evaluation, i.e. the evaluation of a boolean expression
is stopped once the result of the total exression is known with certainty.

So, in the following example, the function \var{Bofu}, which has a boolean
result, will never get called.
\begin{verbatim}
If False and Bofu then
  ...
\end{verbatim}

\subsection{\var{\$D} or \var{\$DEBUGINFO}: Debugging symbols}

When this switch is on (\var{\{\$DEBUGINFO ON\}}), 
the compiler inserts GNU debugging information in
the executable. The effect of this switch is the same as the command-line
switch \var{-g}. By default, insertion of debugging information is off.

\subsection{\var{\$E} : Emulation of coprocessor}

This directive controls the emulation of the coprocessor. There is no
command-line counterpart for this directive.

\subsubsection{ Intel x86 version }

When this switch is enabled, all floating point instructions
which are not supported by standard coprocessor emulators will give out
a warning.

The compiler itself doesn't do the emulation of the coprocessor.

To use coprocessor emulation under \dos go32v1 there is nothing special
required, as it is handled automatically.

To use coprocessor emulation under \dos go32v2 you must use the
emu387 unit, which contains correct initialization code for the
emulator.

Under \linux, the kernel takes care of the coprocessor support.

\subsubsection{ Motorola 680x0 version }

When the switch is on, no floating point opcodes are emitted
by the code generator. Instead, internal run-time library routines
are called to do the necessary calculations. In this case all
real types are mapped to the single IEEE floating point type.

\emph{ Remark : } By default, emulation is on. It is possible to
intermix emulation code with real floating point opcodes, as
long as the only type used is single or real.


\subsection{\var{\$G} : Generate 80286 code}

This option is recognised for Turbo Pascal compatibility, but is ignored,

\subsection{\var{\$L} or \var{\$LOCALSYMBOLS}: Local symbol information}

This switch (not to be confused with the \var{\{\$L file\}} file linking
directive) is recognised for Turbo Pascal compatibility, but is ignored.
generation of symbol information is controlled by the \var{\$D} switch.

\subsection{\var{\$M} or \var{\$MEMORY}: Memory sizes}

This switch can be used to set the heap and stacksize. It's format is as
follows:
\begin{verbatim}
{$M StackSize,HeapSize}
\end{verbatim}
Wher \var{StackSize} and \var{HeapSize} should be two integer values,
greater than 1024. The first number sets the size of the stack, and the
second the size of the heap. (Stack setting is ignored under \linux).
The two numbers can be set on the command line using the \var{-Ch}
(and \var{-Cs} switches.
 
\subsection{\var{\$N} : Numeric processing }

This switch is recognised for Turbo Pascal compatibility, but is otherwise
ignored, since the compiler always uses the coprocessor for floating point
mathematics.

\subsection{\var{\$O} : Overlay code generation }

This switch is recognised for Turbo Pascal compatibility, but is otherwise
ignored.

\subsection{\var{\$Q} \var{\$OVERFLOWCHECKS}: Overflow checking}
The \var{\{\$Q+\}} or \var{\{\$OVERFLOWCHECKS ON\}} directive turns on 
integer overflow checking. This means that the compiler inserts code 
to check for overflow when doing computations with integers.
When an overflow occurs, the run-time library will print a message
\var{Overflow at xxx}, and exit the program with exit code 215.

\emph{ Remark: } Overflow checking behaviour is not the same as in
Turbo Pascal since all arithmetic operations are done via 32-bit
values. Furthermore, the Inc() and Dec() standard system procedures
\emph{ are } checked for overflow in \fpc, while in Turbo Pascal they
are not.

Using the \var{\{\$Q-\}} switch switches off the overflow checking code
generation.

The generation of overflow checking code can also be controlled
using the \var{-Co} command line compiler option (see \userref).

\subsection{\var{\$R} or \var{\$RANGECHECKS} : Range checking}
By default, the computer doesn't generate code to check the ranges of array
indices, enumeration types, subrange types, etc. Specifying the
\var{\{\$R+\}} switch tells the computer to generate code to check these
indices. If, at run-time, an index or enumeration type is specified that is
out of the declared range of the compiler, then a run-time error is
generated, and the program exits with exit code 201.

The \var{\{\$RANGECHECKS OFF\}} switch tells the compiler not to generate range checking
code. This may result in faulty program behaviour, but no run-time errors
will be generated.

{\em Remark: } Range checking for sets and enumerations are not yet fully
implemented.

\subsection{\var{\$S} : Stack checking}
The \var{\{\$S+\}} directive tells the compiler to generate stack checking
code. This generates code to check if a stack overflow occurred, i.e. to see
whether the stack has grown beyond its maximally allowed size. If the stack
grows beyond the maximum size, then a run-time error is generated, and the
program will exit with exit code 202.

Specifying \var{\{\$S-\}} will turn generation of stack-checking code off.

The command-line compiler switch \var{-Ct} has the same effect as the
\var{\{\$S+\}} directive.

\subsection{\var{\$T} or \var{\$TYPEDADDRESS} : Typed address operator (@)}

In the \var{\{\$T+\}} or \var{\{\$TYPEDADDRESS ON\}} state the @ operator,
when applied to a variable, returns a result of type \var{\^{}T}, if the 
type of the variable is \var{T}. In the \var{\{\$T-\}} state, the result is
always an untyped pointer, which is assignment compatible with all other 
pointer types.

\subsection{\var{\$X} or \var{\$EXTENDEDSYNTAX} : Extended syntax}
Extended syntax allows you to drop the result of a function. This means that
you can use a function call as if it were a procedure. Standard this feature
is on. You can switch it off using the \var{\{\$X-\}} or
\var{\{\$EXTENDEDSYNTAX OFF\}}directive.

The following, for instance, will not compile :
\begin{verbatim}
function Func (var Arg : sometype) : longint;
begin
...          { declaration of Func }
end;

...

{$X-}
Func (A);
\end{verbatim}
The reason this construct is supported is that you may wish to call a
function for certain side-effects it has, but you don't need the function
result. In this case you don't need to assign the function result, saving
you an extra variable.

The command-line compiler switch \var{-Sa1} has the same effect as the
\var{\{\$X+\}} directive.

\subsection{\var{\$Y} or \var{\$REFERENCEINFO} : Insert Browser information}

This switch controls the generation of browser inforation. It is recognized
for compatibility with Turbo Pascal and Delphi only, as Browser information
generation is not yet fully supported.


%%%%%%%%%%%%%%%%%%%%%%%%%%%%%%%%%%%%%%%%%%%%%%%%%%%%%%%%%%%%%%%%%%%%%
% Using conditionals and macros
%%%%%%%%%%%%%%%%%%%%%%%%%%%%%%%%%%%%%%%%%%%%%%%%%%%%%%%%%%%%%%%%%%%%%
\chapter{Using conditionals, Messages and macros}
\label{ch:CondMessageMacro}
The \fpc compiler supports conditionals as in normal Turbo Pascal. It does,
however, more than that. It allows you to make macros which can be used in
your code, and it allows you to define messages or errors which will be
displayed when compiling.

%%%%%%%%%%%%%%%%%%%%%%%%%%%%%%%%%%%%%%%%%%%%%%%%%%%%%%%%%%%%%%%%%%%%%
% Conditionals
\section{Conditionals}
\label{se:Conditionals}
The rules for using conditional symbols are the same as under Turbo Pascal.
Defining a symbol goes as follows:
\begin{verbatim}
{$Define Symbol }
\end{verbatim}
From this point on in your code, the compiler know the symbol \var{Symbol}
Symbols are, like the Pascal language, case insensitive.

You can also define a symbol on the command line. the \var{-dSymbol} option
defines the symbol \var{Symbol}. You can specify as many symbols on the
command line as you want.

Undefining an existing symbol is done in a similar way:
\begin{verbatim}
{$Undef Symbol }
\end{verbatim}
If the symbol didn't exist yet, this doesn't do anything. If the symbol
existed previously, the symbol will be erased, and will not be recognized
any more in the code following the \verb|{$Undef ...}| statement.

You can also undefine symbols from the command line with the \var{-u}
command-line switch..

To compile code conditionally, depending on whether a symbol is defined or
not, you can enclose the code in a \verb|{$ifdef Symbol}| .. \verb|{$endif}|
pair. For instance the following code will never be compiled :
\begin{verbatim}
{$Undef MySymbol}
{$ifdef Mysymbol}
  DoSomething;
  ...
{$endif}
\end{verbatim}

Similarly, you can enclose your code in a \verb|{$Ifndef Symbol}| .. \verb|{$endif}|
pair. Then the code between the pair will only be compiled when the used
symbol doesn't exist. For example, in the following example, the call to the
\var{DoSomething} will always be compiled:
\begin{verbatim}
{$Undef MySymbol}
{$ifndef Mysymbol}
  DoSomething;
  ...
{$endif}
\end{verbatim}

You can combine the two alternatives in one structure, namely as follows
\begin{verbatim}
{$ifdef Mysymbol}
  DoSomething;
{$else}
  DoSomethingElse
{$endif}
\end{verbatim}
In this example, if \var{MySymbol} exists, then the call to \var{DoSomething}
will be compiled. If it doesn't exist, the call to \var{DoSomethingElse} is
compiled.

The \fpc compiler defines some symbols before starting to compile your
program or unit. You can use these symbols to differentiate between
different versions of the compiler, and between different compilers.
In \seet{Symbols}, a list of pre-defined symbols is given\footnote{Remark:
The \var{FPK} symbol is still defined for compatibility with older versions.}. In that table,
you should change \var{v} with the version number of the compiler
you're using, \var{r} with the release number and \var{p}
with the patch-number of the compiler. 'OS' needs to be changed by the type
of operating system. Currently this can be one of \var{DOS}, \var{GO32V2},
\var{LINUX}, \var{OS2}, \var{WIN32}, \var{MACOS}, \var{AMIGA} or \var{ATARI}. This symbol is undefined if you
specify a target that is different from the platform you're compiling on.
the \var{-TSomeOS} option on the command line will define the \var{SomeOS} symbol,
and will undefined the existing platform symbol\footnote{In versions prior to
0.9.4, this didn't happen, thus making Cross-compiling impossible.}.

\begin{FPCltable}{c}{Symbols defined by the compiler.}{Symbols} \hline
Free \\
VER\var{v} \\
VER\var{v}\_\var{r} \\
VER\var{v}\_\var{r}\_\var{p} \\
OS \\ \hline
\end{FPCltable}

As an example : Version 0.9.1 of the compiler, running on a Linux system,
defines the following symbols before reading the command line arguments:
\var{FPC}, \var{VER0}, \var{VER0\_9}, \var{VER0\_9\_1} and \var{LINUX}.
Specifying \var{-TOS2} on the command-line will undefine the \var{LINUX}
symbol, and will define the \var{OS2} symbol. 

{\em Remark: } Symbols, even when they're defined in the interface part of 
a unit, are not available outside that unit.

\fpc supports the \var{\{\$IFOPT \}} directive for Turbo Pascal
compatibility, but doesn't act on it. It always rejects the condition, so
code between \var{\{\$IFOPT \}} and \var{\{\$Endif\}} is never compiled.

Except for the Turbo Pascal constructs, from version 0.9.8 and higher,
the \fpc compiler also supports a stronger conditional compile mechanism:
The \var{\{\$If \}} construct. 

The prototype of this construct is as follows :
\begin{verbatim}
{$If expr}
  CompileTheseLines;
{$else}
  BetterCompileTheseLines;
{$endif}
\end{verbatim}
In this directive \var{expr} is a Pascal expression which is evaluated using
strings, unless both parts of a comparision can be evaluated as numbers, 
in which case they are evaluated using numbers\footnote{Otherwise
\var{\{\$If 8>54} would evaluate to \var{True}}.
If the complemete expression evaluates to \var{'0'}, then it is considered 
false and rejected. Otherwise it is considered true and accepted. This may
have unsexpected consequences :
\begin{verbatim}
{$If 0}
\end{verbatim}
Will evaluate to \var{False} and be rejected, while
\begin{verbatim}
{$If 00}
\end{verbatim}
Will evaluate to \var{True}.

You can use any Pascal operator to construct your expression : \var{=, <>,
>, <, >=, <=, AND, NOT, OR} and you can use round brackets to change the
precedence of the operators.

The following example shows you many of the possibilities:
\begin{verbatim}
{$ifdef fpc}

var
   y : longint;
{$else fpc}

var
   z : longint;
{$endif fpc}

var
   x : longint;

begin

{$if (fpc_version=0) and (fpc_release>6) and (fpc_patch>4)}
{$info At least this is version 0.9.5}
{$else}
{$fatalerror Problem with version check}
{$endif}

{$define x:=1234}
{$if x=1234}
{$info x=1234}
{$else}
{$fatalerror x should be 1234}
{$endif}

{$if 12asdf and 12asdf}
{$info $if 12asdf and 12asdf is ok}
{$else}
{$fatalerror $if 12asdf and 12asdf rejected}
{$endif}

{$if 0 or 1}
{$info $if 0 or 1 is ok}
{$else}
{$fatalerror $if 0 or 1 rejected}
{$endif}

{$if 0}
{$fatalerror $if 0 accepted}
{$else}
{$info $if 0 is ok}
{$endif}

{$if 12=12}
{$info $if 12=12 is ok}
{$else}
{$fatalerror $if 12=12 rejected}
{$endif}

{$if 12<>312}
{$info $if 12<>312 is ok}
{$else}
{$fatalerror $if 12<>312 rejected}
{$endif}


{$if 12<=312}
{$info $if 12<=312 is ok}
{$else}
{$fatalerror $if 12<=312 rejected}
{$endif}

{$if 12<312}
{$info $if 12<312 is ok}
{$else}
{$fatalerror $if 12<312 rejected}
{$endif}

{$if a12=a12}
{$info $if a12=a12 is ok}
{$else}
{$fatalerror $if a12=a12 rejected}
{$endif}

{$if a12<=z312}
{$info $if a12<=z312 is ok}
{$else}
{$fatalerror $if a12<=z312 rejected}
{$endif}


{$if a12<z312}
{$info $if a12<z312 is ok}
{$else}
{$fatalerror $if a12<z312 rejected}
{$endif}

{$if not(0)}
{$info $if not(0) is OK}
{$else}
{$fatalerror $if not(0) rejected}
{$endif}

{$info *************************************************}
{$info * Now have to follow at least 2 error messages: *}
{$info *************************************************}

{$if not(0}
{$endif}

{$if not(<}
{$endif}

end.
\end{verbatim}
As you can see from the example, this construct isn't useful when used  
with normal symbols, but it is if you use macros, which are explained in 
\sees{Macros}, they can be very useful. When trying this example, you must
switch on macro support, with the \var{-Sm} command-line switch.

%%%%%%%%%%%%%%%%%%%%%%%%%%%%%%%%%%%%%%%%%%%%%%%%%%%%%%%%%%%%%%%%%%%%%
% Macros
\section{Messages}
\label{se:Messages}
\fpc lets you define normal, warning and error messages in your code. 
Messages can be used to display useful information, such as copyright
notices, a list of symbols that your code reacts on etc.

Warnings can be used if you think some part of your code is still buggy, or
if you think that a certain combination of symbols isn't useful. In general
anything which may cause problems when compiling.

Error messages can be useful if you need a certain symbol to be defined
to warn that a certain variable isn't defined or so, or when the compiler
version isn't suitable for your code.

The compiler treats these messages as if they were generated by the
compiler. This means that if you haven't turned on warning messages, the
warning will not e displayed. Errors are always displayed, and the compiler
stops as if an error had occurred.

For messages, the syntax is as follows  :
\begin{verbatim}
{$Message Message text }
\end{verbatim}
Or 
\begin{verbatim}
{$Info Message text }
\end{verbatim}
For notes:
\begin{verbatim}
{$Note Message text }
\end{verbatim}
For warnings:
\begin{verbatim}
{$Warning Warning Message text }
\end{verbatim}
For errors :
\begin{verbatim}
{$Error  Error Message text }
\end{verbatim}
Lastly, for fatal errors :
\begin{verbatim}
{$FatalError  Error Message text }
\end{verbatim}
or
\begin{verbatim}
{$Stop  Error Message text }
\end{verbatim}
The difference between \var{\$Error} and \var{\$FatalError} or \var{\$Stop}
messages is that when the compiler encounters an error, it still continues
to compile. With a fatal error, the compiler stops.

{\em Remark :} You cannot use the '\var{\}}' character in your message, since
this will be treated as the closing brace of the message.

As an example, the following piece of code will generate an error when 
the symbol \var{RequiredVar} isn't defined:
\begin{verbatim}
{$ifndef RequiredVar}
{$Error Requiredvar isn't defined !}
{$endif}
\end{verbatim} 
But the compiler will continue to compile. It will not, however, generate a
unit file or a program (since an error occurred).

%%%%%%%%%%%%%%%%%%%%%%%%%%%%%%%%%%%%%%%%%%%%%%%%%%%%%%%%%%%%%%%%%%%%%
% Macros
\section{Macros}
\label{se:Macros}
Macros are very much like symbols in their syntax, the difference is that
macros have a value whereas a symbol simply is defined or is not defined.
If you want macro support, you need to specify the \var{-Sm} command-line
switch, otherwise your macro will be regarded as a symbol.

Defining a macro in your program is done in the same way as defining a symbol; 
in a \var{\{\$define \}} preprocessor statement\footnote{In compiler
versions older than 0.9.8, the assignment operator for a macros wasn't
\var{:=}, but \var{=}}:
\begin{verbatim}
{$define ident:=expr}
\end{verbatim}
If the compiler encounters \var{ident} in the rest of the source file, it
will be replaced immediately by \var{expr}. This replacement works
recursive, meaning that when the compiler expanded one of your macros, it
will look at the resulting expression again to see if another replacement
can be made. You need to be careful with this, because an infinite loop can 
occur in this manner.

Here are two examples which illustrate the use of macros:
\begin{verbatim}
{$define sum:=a:=a+b;}
...
sum          { will be expanded to 'a:=a+b;' 
               remark the absence of the semicolon}
...
{$define b:=100}
sum          { Will be expanded recursively to a:=a+100; }                   
...
\end{verbatim}
The previous example could go wrong :
\begin{verbatim}
{$define sum:=a:=a+b;}
...
sum          { will be expanded to 'a:=a+b;' 
               remark the absence of the semicolon}
...
{$define b=sum} { DON'T do this !!!}
sum          { Will be infinitely recursively expanded... }                   
...
\end{verbatim}
On my system, the last example results in a heap error, causing the compiler
to exit with a run-time error 203.

{\em Remark: } Macros defined in the interface part of a unit are not
available outside that unit ! They can just be used as a notational
convenience, or in conditional compiles.

By default, from version 0.9.8 of the compiler on, the compiler predefines three
macros, containing the version number, the release number and the patch
number. They are listed in \seet{DefMacros}.
\begin{FPCltable}{ll}{Predefined macros}{DefMacros} \hline
Symbol & Contains \\ \hline
\var{FPC\_VERSION} & The version number of the compiler. \\
\var{FPC\_RELEASE} & The release number of the compiler. \\
\var{FPC\_PATCH} & The patch number of the compiler. \\
\hline
\end{FPCltable}

{\em Remark: } Don't forget that macros support isn't on by default. You
need to compile with the \var{-Sm} command-line switch.

%%%%%%%%%%%%%%%%%%%%%%%%%%%%%%%%%%%%%%%%%%%%%%%%%%%%%%%%%%%%%%%%%%%%%
% Using assembly language
%%%%%%%%%%%%%%%%%%%%%%%%%%%%%%%%%%%%%%%%%%%%%%%%%%%%%%%%%%%%%%%%%%%%%
\chapter{Using Assembly language}
\label{ch:AsmLang}
\fpc supports inserting of assembler instructions in your code. The
mechanism for this is the same as under Turbo Pascal. There are, however
some substantial differences, as will be explained in the following.

%%%%%%%%%%%%%%%%%%%%%%%%%%%%%%%%%%%%%%%%%%%%%%%%%%%%%%%%%%%%%%%%%%%%%
% Intel syntax
\section{Intel syntax}
\label{se:Intel}

As of version 0.9.7, \fpc supports Intel syntax for the Intel family of Ix86
processors  in it's \var{asm} blocks.

The Intel syntax in your \var{asm} block is converted to AT\&T syntax by the
compiler, after which it is inserted in the compiled source.
The supported assembler constructs are a subset of the normal assembly
syntax. In what follows we specify what constructs are not supported in
\fpc, but which exist in Turbo Pascal:

\begin{itemize}
\item  The \var{TBYTE} qualifier is not supported.
\item  The \var{\&} identifier override is not supported.
\item  The \var{HIGH} operator is not supported.
\item  The \var{LOW} operator is not supported.
\item  The \var{OFFSET} and \var{SEG} operators are not supported.
     use \var{LEA} and the various \var{Lxx} instructions instead.
\item  Expressions with constant strings are not allowed.
\item  Access to record fields via parenthesis is not allowed
\item  Typecasts with normal pascal types are not allowed, only
    recognized assembler typecasts are allowed.\\ Example:
\begin{verbatim}
mov al, byte ptr MyWord       -- allowed,
mov al, byte(MyWord)          -- allowed,
mov al, shortint(MyWord)      -- not allowed.
\end{verbatim}
\item  Pascal type typecasts on constants are not allowed. \\
Example:
\begin{verbatim}
const s= 10; const t = 32767;
\end{verbatim}
in Turbo Pascal:
\begin{verbatim}
mov al, byte(s)            -- useless typecast.
mov al, byte(t)            -- syntax error!
\end{verbatim}
In this parser, either of those  cases will give out a syntax error.
\item  Constant references expressions with constants only are not
   allowed (in all cases they do not work in protected mode,
    under linux i386). \\ Examples:
\begin{verbatim}
mov al,byte ptr ['c']      -- not allowed.
mov  al,byte ptr [100h]    -- not allowed.
\end{verbatim}
 (This is due to the limitation of Turbo Assembler).
\item  Brackets within brackets are not allowed
\item  Expressions with segment overrides fully in brackets are
presently not supported, but they can easily be implemented
in BuildReference if requested. \\ Example:
\begin{verbatim}
mov al,[ds:bx]     -- not allowed
\end{verbatim}
use instead:
\begin{verbatim}
mov al,ds:[bx]
\end{verbatim}
\item  Possible allowed indexing are as follows:
\begin{itemize}
\item    \var{Sreg:[REG+REG*SCALING+/-disp]}
\item    \var{SReg:[REG+/-disp]}
\item    \var{SReg:[REG]}
\item    \var{SReg:[REG+REG+/-disp]}
\item    \var{SReg:[REG+REG*SCALING]}
\end{itemize}
Where \var{Sreg} is optional and specifies the segment override.
{\em Notes:}
\begin{enumerate}
\item The order of terms is important contrary to Turbo Pascal.
\item The Scaling value must be a value, and not an identifier
to a symbol.\\  Examples:
\begin{verbatim}
const myscale = 1;
...
mov al,byte ptr [esi+ebx*myscale] -- not allowed.
\end{verbatim}
use:
\begin{verbatim}
mov al, byte ptr [esi+ebx*1]
\end{verbatim}
\end{enumerate}
\item  Possible variable identifier syntax is as follows:
 (Id = Variable or typed constant identifier.)
\begin{enumerate}
\item \var{ID}
\item \var{[ID]}
\item \var{[ID+expr]}
\item \var{ID[expr]}
\end{enumerate}
 Possible fields are as follow:
\begin{enumerate}
\item  \var{ID.subfield.subfield ...}
\item  \var{[ref].ID.subfield.subfield ...}
\item  \var{[ref].typename.subfield ...}
\end{enumerate}
\item  Local Labels: Contrary to Turbo Pascal, local labels, must
at least contain one character after the local symbol indicator.\\
Example:
\begin{verbatim}
@:               -- not allowed
\end{verbatim}
  use instead, for example:
\begin{verbatim}
@1:             -- allowed
\end{verbatim}
\item  Contrary to Turbo Pascal local references cannot be used as references,
   only as displacements. \\   example:
\begin{verbatim}
lds si,@mylabel   -- not allowed
\end{verbatim}
\item  Contrary to Turbo Pascal, \var{SEGCS}, \var{SEGDS}, \var{SEGES} and
\var{SEGSS} segment overrides are presently not supported.
   (This is a planned addition though).
\item  Contrary to Turbo Pascal where memory sizes specifiers can
   be practically anywhere, the \fpc Intel inline assembler requires
   memory size specifiers to be outside the brackets. \\
      example:
\begin{verbatim}
mov al,[byte ptr myvar]    -- not allowed.
\end{verbatim}
 use:
\begin{verbatim}
mov al,byte ptr [myvar]    -- allowed.
\end{verbatim}
\item  Base and Index registers must be 32-bit registers.
     (limitation of the GNU Assembler).
\item  \var{XLAT} is equivalent to \var{XLATB}.
\item  Only Single and Double FPU opcodes are supported.
\item  Floating point opcodes are currently not supported
   (except those which involve only floating point registers).
\end{itemize}

The Intel inline assembler supports the following macros :
\begin{description}
\item [@Result] represents the function result return value.
\item [Self] represents the object method pointer in methods.
\end{description}

%%%%%%%%%%%%%%%%%%%%%%%%%%%%%%%%%%%%%%%%%%%%%%%%%%%%%%%%%%%%%%%%%%%%%
% AT&T syntax
\section{AT\&T Syntax}
\label{se:AttSyntax}
\fpc uses the \gnu \var{as} assembler to generate its object files for
the Intel Ix86 processors . Since
the \gnu assembler uses AT\&T assembly syntax, the code you write should
use the same syntax. The differences between AT\&T and Intel syntax as used
in Turbo Pascal are summarized in the following:
\begin{itemize}
\item The opcode names include the size of the operand. In general, one can
say that the AT\&T opcode name is the Intel opcode name, suffixed with a
'\var{l}', '\var{w}' or '\var{b}' for, respectively, longint (32 bit),
word (16 bit) and byte (8 bit) memory or register references. As an example,
the Intel construct \mbox{'\var{mov al bl}} is equivalent to the AT\&T style '\var{movb
\%bl,\%al}' instruction.
\item AT\&T immediate operands are designated with '\$', while Intel syntax
doesn't use a prefix for immediate operands. Thus the Intel construct
'\var{mov ax, 2}' becomes '\var{movb \$2, \%al}' in AT\&T syntax.
\item AT\&T register names are preceded by a '\var{\%}' sign. 
They are undelimited in Intel syntax. 
\item AT\&T indicates absolute jump/call operands with '\var{*}', Intel
syntax doesn't delimit these addresses.
\item The order of the source and destination operands are switched. AT\&T
syntax uses '\var{Source, Dest}', while Intel syntax features '\var{Dest,
Source}'. Thus the Intel construct '\var{add eax, 4}' transforms to
'\var{addl \$4, \%eax}' in the AT\&T dialect.
\item Immediate long jumps are prefixed with the '\var{l}' prefix. Thus the
Intel '\var{call/jmp section:offset'} is transformed to '\var{lcall/ljmp
\$section,\$offset}'. Similarly the far return is '\var{lret}', instead of the
Intel '\var{ret far}'.
\item Memory references are specified differently in AT\&T and Intel
assembly. The Intel indirect memory reference
\begin{quote}
\var{Section:[Base + Index*Scale + Offs]}
\end{quote}
is written in AT\&T syntax as :
\begin{quote}
\var{Section:Offs(Base,Index,Scale)}
\end{quote}
Where \var{Base} and \var{Index} are optional 32-bit base and index
registers, and \var{Scale} is used to multiply \var{Index}. It can take the
values 1,2,4 and 8. The \var{Section} is used to specify an optional section
register for the memory operand.
\end{itemize}

More information about the AT\&T syntax can be found in the \var{as} manual,
although the following differences with normal AT\&T assembly must be taken
into account :
\begin{itemize}
\item  Only the following directives are presently supported:
 \begin{description}
\item[.byte]
\item[.word]
\item[.long]
\item[.ascii]
\item[.asciz]
\item[.globl]
\end{description}
\item  The following directives are recognized but are not
   supported:
\begin{description}
\item[.align]
\item[.lcomm]
\end{description} 
Eventually they will be supported.
\item Directives are case sensitive, other identifiers are not case sensitive.
\item  Contrary to GAS local labels/symbols {\em must} start with \var{.L}
\item  The nor operator \var{'!'} is not supported.
\item  String expressions in operands are not supported.
\item  Constant expressions which represent memory references are not 
allowed even though constant immediate value expressions are supported. \\
examples:
\begin{verbatim}
const myid = 10;
...
movl $myid,%eax       -- allowed
movl myid(%esi),%eax  -- not allowed.
\end{verbatim}
\item When the \var{.globl} directive is found, the symbol following
    it is made public and is immediately emitted.
    Therefore label names with this name will be ignored.
\item  Only Single and Double FPU opcodes are supported.
\end{itemize} 

The AT\&T inline assembler supports the following macros :
\begin{description}
\item [\_\_RESULT] represents the function result return value.
\item [\_\_SELF]   represents the object method pointer in methods.
\item [\_\_OLDEBP] represents the old base pointer in recusrive routines.
\end{description}



%%%%%%%%%%%%%%%%%%%%%%%%%%%%%%%%%%%%%%%%%%%%%%%%%%%%%%%%%%%%%%%%%%%%%
% Calling mechanism
\section{Calling mechanism}
\label{se:Calling}
Procedures and Functions are called with their parameters on the stack.
Contrary to Turbo Pascal, {\em all} parameters are pushed on the stack, and
they are pushed {\em right} to {\em left}, instead of left to right for
Turbo Pascal. This is especially important if you have some assembly
subroutines in Turbo Pascal which you would like to translate to \fpc.

Function results are returned in the accumulator, if they fit in the
register.

The registers are {\em not} saved when calling a function or procedure. If
you want to call a procedure or function from assembly language, you must
save any registers you wish to preserve.

The first thing a procedure does is saving the base pointer, and setting the
base pointer equal to the stack pointer. References to the pushed parameters
and local variables are constructed using the base pointer.

When the procedure or function exits, it clears the stack.

When you want your code to be called by a C library or used in a C
program, you will run into trouble because of this calling mechanism. In C,
the calling procedure is expected to clear the stack, not the called
procedure. In other words, the arguments still are on the stack when the
procedure exits. To avoid this problem, \fpc supports the \var{export}
modifier. Procedures that are defined using the export modifier, use a
C-compatible calling mechanism. This means that they can be called from a
C program or library, or that you can use them as a callback function.

This also means that you cannot call this procedure or function from your
own program, since your program uses the Pascal calling convention.
However, in the exported function, you can of course call other Pascal
routines.

As of version 0.9.8, the \fpc compiler supports also the \var{cdecl} and
\var{stdcall} modifiers, as found in Delphi. The \var{cdecl} modifier does
the same as the \var{export} modifier, and \var{stdcall} does nothing, since
\fpc pushes the paramaters from right to left by default.
In addition to the Delphi \var{cdecl} construct, \fpc also supports the
\var{popstack} directive; it is nearly the same a the \var{cdecl} directive,
only it still mangles the name, i.e. makes it into a name such as the
compiler uses internally.

All this is summarized in \seet{Calling}. The first column lists the
modifier you specify for a procedure declaration. The second one lists the
order the paramaters are pushed on the stack. The third column specifies who
is responsible for cleaning the stack: the caller or the called function.
Finally, the last column specifies if registers are used to pass parameters
to the function.

\begin{FPCltable}{llll}{Calling mechanisms in \fpc}{Calling}\hline
Modifier & Pushing order & Stack cleaned by & Parameters in registers \\
\hline
(none)  & Right-to-left & Function & No \\
cdecl   & Right-to-left & Caller   & No \\
export  & Right-to-left & Caller   & No \\
stdcall & Right-to-left & Function & No \\
popstack & Right-to-left & Caller  & No \\ \hline
\end{FPCltable}

More about this can be found in \seec{Linking} on linking.


\subsection{ Ix86 calling conventions }

Standard entry code for procedures and functions is as follows on the
x86 architecture:
\begin{verbatim}
   pushl   %ebp
   movl    %esp,%ebp
\end{verbatim}

The generated exit sequence for procedure and functions looks as follows:
\begin{verbatim}
  leave
  ret  $xx
\end{verbatim}

Where \var{xx} is the total size of the pushed parameters.

To have more information on function return values take a look at the
\seec{RegConvs} section.


\subsection{ M680x0 calling conventions }

Standard entry code for procedures and functions is as follows on the
680x0 architecture:
\begin{verbatim}
   move.l  a6,-(sp)
   move.l  sp,a6
\end{verbatim}

The generated exit sequence for procedure and functions looks as follows:
\begin{verbatim}
  unlk   a6
  move.l (sp)+,a0     ; Get return address
  add.l  #xx,sp       ; Remove allocated stack
  move.l a0,-(sp)     ; Put back return address on top of the stack
\end{verbatim}

Where \var{xx} is the total size of the pushed parameters.

To have more information on function return values take a look at the
\seec{RegConvs} section.



%%%%%%%%%%%%%%%%%%%%%%%%%%%%%%%%%%%%%%%%%%%%%%%%%%%%%%%%%%%%%%%%%%%%%
% Telling the compiler what registers have changed
\section{Signalling changed registers}
\label{se:RegChanges}
When the compiler uses variables, it sometimes stores them, or the result of
some calculations, in the processor registers. If you insert assembler code
in your program that modifies the processor registers, then this may
interfere with the compiler's idea about the registers. To avoid this
problem, \fpc allows you to tell the compiler which registers have changed.
The compiler will then avoid using these registers. Telling the compiler
which registers have changed, is done by specifying a set of register names
behind an assembly block, as follows:
\begin{verbatim}
asm
  ...
end ['R1',...,'Rn'];
\end{verbatim}
Here \var{R1} to \var{Rn} are the names of the 32-bit registers you
modify in your assembly code.

As an example :
\begin{verbatim}
   asm
   movl BP,%eax
   movl 4(%eax),%eax
   movl %eax,__RESULT
   end ['EAX'];
\end{verbatim}
This example tells the compiler that the \var{EAX} register was modified.

%%%%%%%%%%%%%%%%%%%%%%%%%%%%%%%%%%%%%%%%%%%%%%%%%%%%%%%%%%%%%%%%%%%%%
% Register conventions
%%%%%%%%%%%%%%%%%%%%%%%%%%%%%%%%%%%%%%%%%%%%%%%%%%%%%%%%%%%%%%%%%%%%%
\section{Register Conventions}
\label{se:RegConvs}

The compiler has different register conventions, depending on the
target processor used.

\subsection{ Intel x86 version }

When optimizations are on, no register can be freely modified, without
first being saved and then restored. Otherwise, EDI is usually used as
a scratch register and can be freely used in assembler blocks.

\subsection{ Motorola 680x0 version }

Registers which can be freely modified without saving are registers
D0, D1, D6, A0, A1, and floating point registers FP2 to FP7. All other
registers are to be considered reserved and should be saved and then
restored when used in assembler blocks.

%%%%%%%%%%%%%%%%%%%%%%%%%%%%%%%%%%%%%%%%%%%%%%%%%%%%%%%%%%%%%%%%%%%%%
% Linking issues
%%%%%%%%%%%%%%%%%%%%%%%%%%%%%%%%%%%%%%%%%%%%%%%%%%%%%%%%%%%%%%%%%%%%%
\chapter{Linking issues}
\label{ch:Linking}
When you only use Pascal code, and Pascal units, then you will not see much
of the part that the linker plays in creating your executable.
The linker is only called when you compile a program. When compiling units,
the linker isn't invoked.

However, there are times that you want to C libraries, or to external
object files that are generated using a C compiler (or even another pascal
compiler). The \fpc compiler can generate calls to a C function,
and can generate functions that can be called from C (exported functions).
However, these exported functions cannot be called from
inside Pascal anymore. More on these calling conventions can be found in
\sees{Calling}.

In general, there are 2 things you must do to use a function that resides in
an external library or object file:
\begin{enumerate}
\item You must make a pascal declaration of the function or procedure you
want to use.
\item You must tell the compiler where the function resides, i.e. in what
object file or what library, so the compiler can link the necessary code in.
\end{enumerate}
The following sections attempt to explain how to do this.

%%%%%%%%%%%%%%%%%%%%%%%%%%%%%%%%%%%%%%%%%%%%%%%%%%%%%%%%%%%%%%%%%%%%%
% Declaring an external function or procedure
\section{Declaring an external function or procedure}
\label{se:ExternalDeclaration}

The first step in using external code blocks is declaring the function you
want to use. \fpc supports Delphi syntax, i.e. you must use the
\var{external} directive.

There exist four variants of the external direcive :
\begin{enumerate}
\item A simple external declaration:
\begin{verbatim}
Procedure ProcName (Args : TPRocArgs); external;
\end{verbatim}
The \var{external} directive tells the compiler that the function resides in
an external block of code. You can use this together with the \var{\{\$L \}}
or \var{\{\$LinkLib \}} directives to link to a function or procedure in a
library or external object file.

\item You can give the \var{external} directive a library name as an
argument:
\begin{verbatim}
Procedure ProcName (Args : TPRocArgs); external 'Name';
\end{verbatim}
This tells the compiler that the procedure resides in a library with name
\var{'Name'}. This method is equivalent to the following:
\begin{verbatim}
Procedure ProcName (Args : TPRocArgs);external;
{$LinkLib 'Name'}
\end{verbatim}
\item The \var{external} can also be used with two arguments:
\begin{verbatim}
Procedure ProcName (Args : TPRocArgs); external 'Name'
                                       name 'OtherProcName';
\end{verbatim}
This has the same meaning as the previous declaration, only the compiler
will use the name \var{'OtherProcName'} when linking to the library. This
can be used to give different names to procedures and functions in an
external library.

This method is equivalent to the following code:
\begin{verbatim}
Procedure OtherProcName (Args : TProcArgs); external;
{$LinkLib 'Name'}

Procedure ProcName (Args : TPRocArgs);

begin
  OtherProcName (Args);
end;
\end{verbatim}
\item Lastly, onder \windows and \ostwo, there is a fourth possibility
to specify an external function: In \file{.DLL} files, functionas also have
a unique number (their index). It is possible to refer to these fuctions
using their index:
\begin{verbatim}
Procedure ProcName (Args : TPRocArgs); external 'Name' Index SomeIndex;
\end{verbatim}
This tells the compiler that the procedure \var{ProcName} resides in a
dynamic link library, with index {SomeIndex}.

\em{Remark:} Note that this is ONLY available under \windows and \ostwo.
\end{enumerate}

In earlier versions of the \fpc compiler, the following construct was
also possible :
\begin{verbatim}
Procedure ProcName (Args : TPRocArgs); [ C ];
\end{verbatim}
This method is equivalent to the following statement:
\begin{verbatim}
Procedure ProcName (Args : TPRocArgs); cdecl; external;
\end{verbatim}
However, the \var{[ C ]} directive is no longer supoerted as of version
0.99.5  of \fpc, therefore you should use the
\var{external} directive, with the \var{cdecl} directive, if needed.

%%%%%%%%%%%%%%%%%%%%%%%%%%%%%%%%%%%%%%%%%%%%%%%%%%%%%%%%%%%%%%%%%%%%%
% Linking an object file in your program
\section{Linking to an object file}
\label{se:LinkIn}

Having declared the external function that resides in an object file,
you can use it as if it was defined in your own program or unit.
To produce an executable, you must still link the object file in.
This can be done with the \var{\{\$L 'file.o'\}} directive.

This will cause the linker to link in the object file \file{file.o}. On
\linux systems, this filename is case sensitive. Under \dos, case isn't
important. Note that \var{file.o} must be in the current directory if you
don't specify a path. The linker will not search for \file{file.o} if it
isn't found.

You cannot specify libraries in this way, it is for object files only.

Here we present an example. Consider that you have some assembly routine that
calculates the nth Fibonacci number :
\begin{verbatim}
.text
	.align 4
.globl Fibonacci
	.type Fibonacci,@function
Fibonacci:
	pushl %ebp
	movl %esp,%ebp
	movl 8(%ebp),%edx
	xorl %ecx,%ecx
	xorl %eax,%eax
	movl $1,%ebx
	incl %edx
loop:
	decl %edx
	je endloop
	movl %ecx,%eax
	addl %ebx,%eax
	movl %ebx,%ecx
	movl %eax,%ebx
	jmp loop
endloop:
	movl %ebp,%esp
        popl %ebp
	ret
\end{verbatim}
Then you can call this function with the following Pascal Program:
\begin{verbatim}
Program FibonacciDemo;

var i : longint;

Function Fibonacci (L : longint):longint;cdecl;external;

{$L fib.o}

begin
  For I:=1 to 40 do
    writeln ('Fib(',i,') : ',Fibonacci (i));
end.
\end{verbatim}
With just two commands, this can be made into a program :
\begin{verbatim}
as -o fib.o fib.s
pp fibo.pp
\end{verbatim}
This example supposes that you have your assembler routine in \file{fib.s},
and your Pascal program in \file{fibo.pp}.

%%%%%%%%%%%%%%%%%%%%%%%%%%%%%%%%%%%%%%%%%%%%%%%%%%%%%%%%%%%%%%%%%%%%%
% Linking your program to a library
\section{Linking to a library}
\label{se:LinkOut}

To link your program to a library, the procedure depends on how you declared
the external procedure. If you used thediffers a little from the
procedure when you link in an object file. although the declaration step
remains the same (see \ref{se:ExternalDeclaration} on how to do that).

In case you used the follwing syntax to declare your procedure:
\begin{verbatim}
Procedure ProcName (Args : TPRocArgs); external 'Name';
\end{verbatim}
You don't need to take additional steps to link your file in, the compiler
will do all that is needed for you.

In case you used
\begin{verbatim}
Procedure ProcName (Args : TPRocArgs); external;
\end{verbatim}
You still need to explicity link to the library. This can be done in 2 ways:
\begin{enumerate}
\item You can tell the compiler in the source file what library to link to
using the \var{\{\$LinkLib 'Name'\}} directive:
\begin{verbatim}
{$LinkLib 'gpm'}
\end{verbatim}
This will link to the \file{gpm} library. On \linux systems, you needn't
specify the extension or 'lib' prefix of the library. The compiler takes
care of that. On \dos or \windows systems, you need to specify the full
name.
\item You can also tell the compiler on the command-line to link in a
library: The \var{-k} option can be used for that. For example
\begin{verbatim}
ppc386 -k'-lgpm' myprog.pp
\end{verbatim}
Is equivalent to the above method, and tells the linker to link to the
\file{gpm} library.
\end{enumerate}

As an example; consider the following program :
\begin{verbatim}
program printlength;

{$linklib c} { Case sensitive }

{ Declaration for the standard C function strlen }
Function strlen (P : pchar) : longint; cdecl;external;

begin
  Writeln (strlen('Programming is easy !'));
end.
\end{verbatim}
This program can be compiled with :
\begin{verbatim}
pp  prlen.pp
\end{verbatim}
Supposing, of course, that the program source resides in \file{prlen.pp}.

You cannot use procedures or functions that have a variable number of
arguments in C. Pascal doesn't support this feature of C.

%%%%%%%%%%%%%%%%%%%%%%%%%%%%%%%%%%%%%%%%%%%%%%%%%%%%%%%%%%%%%%%%%%%%%
% Making a shared library
\section{Making a shared library}
\label{se:SharedLib}

\fpc supports making shared libraries in a straightforward and easy manner.
If you want to make libraries for other \fpc programmers, you just need to
provide a command line switch. If you want C programmers to be able to use
your code as well, you will need to adapt your code a little. This process
is described first.

% Adapting your code
\subsection{Adapting your code}

If you want to make your procedures and functions available to C
programmers, you can do this very easily. All you need to do is declare the
functions and procedures that you want to make available as \var{Export}, as
follows:
\begin{verbatim}
Procedure ExportedProcedure ; export;
\end{verbatim}
This tells the compiler that it shouldn't clear the stack upon exiting the
procedure (see \sees{Calling}), thus enabling a C program to call your
function. It also means that your Pascal program can't call this function,
since it will be using the C calling mechanism.

{\em Remark :} You can only declare a function as exported in the
\var{Implementation} section of a unit. This function may {\em not} appear
in the interface part of a unit. This is logical, since a Pascal routine
cannot call an exported function, anyway.

However, the generated object file will not contain the name of the function
as you declared it. The \fpc compiler ''mangles'' the name you give your
function. It makes the name all-uppercase, and adds the types of all
parameters to it. For \fpc units, this doesn't matter, since the \file{.ppu}
unit file contains all information to map your function declaration onto the
mangled name in the object file. For a C programmer, who has no access to
the \var{.ppu} file, this is not very convenient. That is why \fpc
has the \var{Alias} modifier. The \var{Alias} modifier allows you to specify
another name (a nickname) for your function or procedure.

The prototype for an aliased function or procedure is as follows :
\begin{verbatim}
Procedure AliasedProc; [ Alias : 'AliasName'];
\end{verbatim}
The procedure \var{AliasedProc} will also be known as \var{AliasName}. Take
care, the name you specify is case sensitive (as C is).

Of course, you want to combine these two features of \fpc, to export a
function under a reasonable name; If you want to do that, you must first
specify that the function is to be exported, and then only declare an alias:
\begin{verbatim}
Procedure ExportToCProc; Export; [Alias : 'procname'];
\end{verbatim}
After that, any C program will be able to use your procedure or function.

{\em Remark: }
If you use in your unit functions that are in other units, or
system functions, then the C program will need to link in the object files
from the units too.

% Compiling libraries
\subsection {Compiling libraries}

Once you have your (adapted) code, with exported and other functions,
you can compile your unit, and tell the compiler to make it into a library.
The compiler will simply compile your unit, and perform the necessary steps
to transform it into a \var{static} or \var{shared} (\var{dynamical}) library.

You can do this as follows, for a dynamical library:
\begin{verbatim}
ppc386 -Uld myunit
\end{verbatim}
On \linux this will leave you with a file \file{libmyunit.so}. On \windows
and \ostwo, this will leave you with \file{myunit.dll}.

If you want a static library, you can do
\begin{verbatim}
ppc386 -Uls myunit
\end{verbatim}
This will leave you with \file{libmyunit.a} and a file \file{myunit.ppl}.
The \file{myunit.ppl} is the unit file needed by the \fpc compiler.
The extension \file{.ppl} means that the file describes a unit that resides
in a library.

The resulting files are then libraries. To make static libraries, you need
the \file{ranlib} or \var{ar} program on your system. It is standard on any
\linux system, and is provided with the \file{GCC} compiler under \dos.

{\em BEWARE:} This command doesn't include anything but the current unit in
thelibrary. Other units are left out, so if you use code from other units,
you must dpley them together with your library.

% Moving units
\subsection{Moving units into a library}
You can put multiple units into a library with the \var{ppumove} command, as
follows:

\begin{verbatim}
ppumove unit1 unit2 unit3 name
\end{verbatim}
This will move 3 units in 1 library (called \file{libname.so} on linux,
\file{name.dll} on \windows) and it will create 3 files \file{unit1.ppl},
\file{unit2.ppl} and \file{file3.ppl}, which are unit files, but which tell
the compiler to look in library \var{name} when linking your executable.

The \var{ppumove} program has options to create statical or dynammical
libraries. It is provided with the compiler.

% unit searching
\subsection{Unit searching strategy}

When you compile a program or unit, the compiler will by
default always look for \file{.ppl} files. If it doesn't find one, it will
look for a \file{.ppu} file. You can disable this behaviour by
specifying the \var{-Cs} switch. This tells the compiler to make a static
binary, and refrains it from looking for units which reside in a library.

You can tell the compiler only to use dynamic libraries by specifying
the \var{-Cd} switch; the compiler will then only look for \var{.ppl} files,
and will give an error if it doesn't find one.

%%%%%%%%%%%%%%%%%%%%%%%%%%%%%%%%%%%%%%%%%%%%%%%%%%%%%%%%%%%%%%%%%%%%%
% Objects
%%%%%%%%%%%%%%%%%%%%%%%%%%%%%%%%%%%%%%%%%%%%%%%%%%%%%%%%%%%%%%%%%%%%%
\chapter{Objects}
\label{ch:Objects}
In this short chapter we give some technical things about objects. For
instructions on how to use and declare objects, see \refref.

%%%%%%%%%%%%%%%%%%%%%%%%%%%%%%%%%%%%%%%%%%%%%%%%%%%%%%%%%%%%%%%%%%%%%
% Constructor and Destructor calls.
\section{Constructor and Destructor calls}
\label{se:ConsDest}
When using objects that need virtual methods, the compiler uses two help
procedures that are in the run-time library. They are called
\var{Help\_Destructor} and \var{Help\_Constructor}, and they are written in
assembly language. They are used to allocate the necessary memory if needed,
and to insert the Virtual Method Table (VMT) pointer in the newly allocated
object.

When the compiler encounters a call to an object's constructor,
it sets up the stack frame for the call, and inserts a call to the
\var{Help\_Constructor}
procedure before issuing the call to the real constructor.
The helper procedure allocates the needed memory (if needed) and inserts the
VMT pointer in the object. After that, the real constructor is called.

A call to \var{Help\_Destructor} is inserted in every destructor declaration,
just before the destructor's exit sequence.

%%%%%%%%%%%%%%%%%%%%%%%%%%%%%%%%%%%%%%%%%%%%%%%%%%%%%%%%%%%%%%%%%%%%%
% memory storage of Objects
\section{Memory storage of objects}
\label{se:ObjMemory}
Objects are stored in memory just as ordinary records with an extra field :
a pointer to the Virtual Method Table (VMT). This field is stored first, and
all fields in the object are stored in the order they are declared.
This field is initialized by the call to the object's \var{Constructor} method.

If the object you defined has no virtual methods, then a \var{nil} is stored
in the VMT pointer. This ensures that the size of objects is equal, whether
they have virtual methods ore not.

The memory allocated looks as in \seet{ObjMem}.
\begin{FPCltable}{ll}{Object memory layout}{ObjMem} \hline
Offset & What \\ \hline
+0 & Pointer to VMT. \\
+4 & Data. All fields in the order the've been declared. \\
... & \\
\hline
\end{FPCltable}

%%%%%%%%%%%%%%%%%%%%%%%%%%%%%%%%%%%%%%%%%%%%%%%%%%%%%%%%%%%%%%%%%%%%%
% The virtual method table.
\section{The Virtual Method Table}
\label{se:VMT}
The Virtual Method Table (VMT) for each object type consists of 2 check
fields  (containing the size of the data), a pointer to the object's anchestor's
VMT (\var{Nil} if there is no anchestor), and then the pointers to all virtual
methods. The VMT layout is illustrated in \seet{VMTMem}.

The VMT is constructed by the compiler. Every instance of an object receives
a pointer to its VMT.

\begin{FPCltable}{ll}{Virtual Method Table memory layout}{VMTMem} \hline
Offset & What \\ \hline
+0 & Size of object type data \\
+4 & Minus the size of object type data. Enables determining of valid VMT
pointers. \\
+8 & Pointer to ancestor VMT, \var{Nil} if no ancestor available.\\
+12 & Pointers to the virtual methods. \\
... & \\
\hline
\end{FPCltable}

%%%%%%%%%%%%%%%%%%%%%%%%%%%%%%%%%%%%%%%%%%%%%%%%%%%%%%%%%%%%%%%%%%%%%
% Generated code
%%%%%%%%%%%%%%%%%%%%%%%%%%%%%%%%%%%%%%%%%%%%%%%%%%%%%%%%%%%%%%%%%%%%%
\chapter{Generated code}
\label{ch:GenCode}
The \fpc compiler relies on the assembler to make object files. It generates
just the assembly language file. In the following two sections, we discuss
what is generated when you compile a unit or a program.

%%%%%%%%%%%%%%%%%%%%%%%%%%%%%%%%%%%%%%%%%%%%%%%%%%%%%%%%%%%%%%%%%%%%%
%  Units
\section{Units}
\label{se:Units}
When you compile a unit, the \fpc compiler generates 2 files :
\begin{enumerate}
\item A unit description file (with extension \file{.ppu}).
\item An assembly language file (with extension \file{.s}).
\end{enumerate}
The assembly language file contains the actual source code for the
statements in your unit, and the necessary memory allocations for any
variables you use in your unit. This file is converted by the assembler to
an object file (with extension \file{.o}) which can then be linked to other
units and your program, to form an executable.

By default (compiler version 0.9.4 and up), the assembly file is removed
after it has been compiled. Only in the case of the \var{-s} command-line
option, the assembly file must be left on disk, so the assembler can be
called later.

The unit file contains all the information the compiler needs to use the
unit:
\begin{enumerate}
\item Other used units, both in interface and implementation.
\item Types and variables from the interface section of the unit.
\item Function declarations from the interface section of the unit.
\item Some debugging information, when compiled with debugging.
\item A date and time stamp.
\end{enumerate}
Macros, symbols and compiler directives are {\em not} saved to the unit
description file. Aliases for functions are also not written to this file,
which is logical, since they cannot appear in the interface section of a
unit.

The detailed contents and structure of this file are described in the first
appendix. You can examine a unit description file using the \file{dumpppu}
program, which shows the contents of the file.

If you want to distribute a unit without source code, you must provide both
the unit description file and the object file.

You can also provide a C header file to go with the object file. In that
case, your unit can be used by someone who wishes to write his programs in
C. However, you must make this header file yourself since the \fpc compiler
doesn't make one for you.

%%%%%%%%%%%%%%%%%%%%%%%%%%%%%%%%%%%%%%%%%%%%%%%%%%%%%%%%%%%%%%%%%%%%%
% Programs
\section{Programs}
\label{se:Programs}

When you compile a program, the compiler produces again 2 files :
\begin{enumerate}
\item An assembly language file containing the statements of your program,
and memory allocations for all used variables.
\item A linker response file. This file contains a list of object files the
linker must link together.
\end{enumerate}
The link response file is, by default, removed from the disk. Only when you
specify the \var{-s} command-line option or when linking fails, then the ile
is left on the disk. It is named \file{link.res}.

The assembly language file is converted to an object file by the assembler,
and then linked together with the rest of the units and a program header, to
form your final program.

The program header file is a small assembly program which provides the entry
point for the program. This is where the execution of your program starts,
so it depends on the operating system, because operating systems pass
parameters to executables in wildly different ways.

It's name is \file{prt0.o}, and the
source file resides in \file{prt0.s} or some variant of this name. It
usually resided where the system unit source for your system resides.
It's main function is to save the environment and command-line arguments,
set up the stack. Then it calls the main program.

%%%%%%%%%%%%%%%%%%%%%%%%%%%%%%%%%%%%%%%%%%%%%%%%%%%%%%%%%%%%%%%%%%%%%
% MMX Support
%%%%%%%%%%%%%%%%%%%%%%%%%%%%%%%%%%%%%%%%%%%%%%%%%%%%%%%%%%%%%%%%%%%%%
\chapter{Intel MMX support}
\label{ch:MMXSupport}

\section{What is it about ?}
\label{se:WhatisMMXabout}
\fpc supports the new MMX (Multi-Media extensions)
instructions of Intel processors. The idea of MMX is to
process multiple data with one instruction, for example the processor
can add simultaneously 4 words. To implement this efficiently, the
Pascal language needs to be extended. So Free Pascal allows
to add for example two \var{array[0..3] of word},
if MMX support is switched on. The operation is done
by the \var{MMX} unit and allows people without assembler knowledge to take
advantage of the MMX extensions.

Here is an example:
\begin{verbatim}
uses
   MMX;   { include some predefined data types }

const
   { tmmxword = array[0..3] of word;, declared by unit MMX }
   w1 : tmmxword = (111,123,432,4356);
   w2 : tmmxword = (4213,63456,756,4);

var
   w3 : tmmxword;
   l : longint;

begin
   if is_mmx_cpu then  { is_mmx_cpu is exported from unit mmx }
     begin
{$mmx+}   { turn mmx on }
        w3:=w1+w2;
{$mmx-}
     end
   else
     begin
        for i:=0 to 3 do
          w3[i]:=w1[i]+w2[i];
     end;
end.
\end{verbatim}

\section{Saturation support}
\label{se:SaturationSupport}

One important point of MMX is the support of saturated operations.
If a operation would cause an overflow, the value stays at the
highest or lowest possible value for the data type:
If you use byte values you get normally 250+12=6. This is very
annoying when doing color manipulations or changing audio samples,
when you have to do a word add and check if the value is greater than
255. The solution is saturation: 250+12 gives 255.
Saturated operations are supported by the \var{MMX} unit. If you
want to use them, you have simple turn the switch saturation on:
\var{\$saturation+}

Here is an example:
\begin{verbatim}
Program SaturationDemo;
{
  example for saturation, scales data (for example audio)
  with 1.5 with rounding to negative infinity
}

var
   audio1 : tmmxword;

const
   helpdata1 : tmmxword = ($c000,$c000,$c000,$c000);
   helpdata2 : tmmxword = ($8000,$8000,$8000,$8000);

begin
   { audio1 contains four 16 bit audio samples }
{$mmx+}
   { convert it to $8000 is defined as zero, multiply data with 0.75 }
   audio1:=tmmxfixed16(audio1+helpdata2)*tmmxfixed(helpdata1);
{$saturation+}
   { avoid overflows (all values>$7fff becomes $ffff) }
   audio1:=(audio1+helpdata2)-helpdata2;
{$saturation-}
   { now mupltily with 2 and change to integer }
   audio1:=(audio1 shl 1)-helpdata2;
{$mmx-}
end.
\end{verbatim}

\section{Restrictions of MMX support}
\label{se:MMXrestrictions}

In the beginning of 1997 the MMX instructions were introduced in the
Pentium processors, so multitasking systems wouldn't save the
newly introduced MMX registers. To work around that problem, Intel
mapped the MMX registers to the FPU register.

The consequence is that
you can't mix MMX and floating point operations. After using
MMX operations and before using floating point operations, you
have to call the routine \var{EMMS} of the \var{MMX} unit.
This routine restores the FPU registers.

{\em careful:} The compiler doesn't warn, if you mix floating point and
MMX operations, so be careful.

The MMX instructions are optimized for multi media (what else?).
So it isn't possible to perform each operation, some opertions
give a type mismatch, see section \ref {se:SupportedMMX} for the supported
MMX operations

An important restriction is that MMX operations aren't range or overflow
checked, even when you turn range and overflow checking on. This is due to
the nature of MMX operations.

The \var{MMX} unit must be always used when doing MMX operations
because the exit code of this unit clears the MMX unit. If it wouldn't do
that, other program will crash. A consequence of this is that you can't use
MMX operations in the exit code of your units or programs, since they would
interfere  with the exit code of the \var{MMX} unit. The compiler can't
check this, so you are responsible for this !

\section{Supported MMX operations}
\label{se:SupportedMMX}



\section{Optimizing MMX support}
\label{se:OptimizingMMX}
Here are some helpful hints to get optimal performance:
\begin{itemize}
\item The \var{EMMS} call takes a lot of time, so try to seperate floating
point and MMX operations.
\item Use MMX only in low level routines because the compiler
  saves all used MMX registers when calling a subroutine.
\item The NOT-operator isn't supported natively by MMX, so the
  compiler has to generate a workaround and this operation
  is inefficient.
\item Simple assignements of floating point numbers don't access
  floating point registers, so you need no call to the \var{EMMS}
  procedure. Only when doing arithmetic, you need to call the \var{EMMS}
procedure.
\end{itemize}


%%%%%%%%%%%%%%%%%%%%%%%%%%%%%%%%%%%%%%%%%%%%%%%%%%%%%%%%%%%%%%%%%%%%%
% Memory issues
%%%%%%%%%%%%%%%%%%%%%%%%%%%%%%%%%%%%%%%%%%%%%%%%%%%%%%%%%%%%%%%%%%%%%
\chapter{Memory issues}
\label{ch:Memory}

%%%%%%%%%%%%%%%%%%%%%%%%%%%%%%%%%%%%%%%%%%%%%%%%%%%%%%%%%%%%%%%%%%%%%
% The 32-bit model
\section{The 32-bit model.}
\label{se:ThirtytwoBit}
The \fpc Pascal compiler issues 32-bit code. This has several consequences:
\begin{itemize}
\item You need a 386 processor to run the generated code. The
compiler functions on a 286 when you compile it using Turbo Pascal,
but the generated programs cannot be assembled or executed.
\item You don't need to bother with segment selectors. Memory can be
addressed using a single 32-bit pointer.
The amount of memory is limited only by the available amount of (virtual)
memory on your machine.
\item The structures you define are unlimited in size. Arrays can be as long
as you want. You can request memory blocks from any size.
\end{itemize}

The fact that 32-bit code is used, means that some of the older Turbo Pascal
constructs and functions are obsolete. The following is a list of functions
which shouldn't be used anymore:
\begin{description}
\item [Seg()] : Returned the segment of a memory address. Since segments have
no more meaning, zero is returned in the \fpc run-time library implementation of
\var{Seg}.
\item [Ofs()] : Returned the offset of a memory address. Since segments have
no more meaning, the complete address is returned in the \fpc implementation
of this function. This has as a consequence that the return type is
\var{Longint} instead of \var{Word}.
\item [Cseg(), Dseg()] : Returned, respectively, the code and data segments
of your program.  This returns zero in the \fpc implementation of the
system unit, since both code and data are in the same memory space.
\item [Ptr] accepted a segment and offset from an address, and would return
a pointer to this address. This has been changed in the run-time library.
Standard it returns now simply the offset. If you want to retain the old
functionality, you can recompile the run-time library with the
\var{DoMapping} symbol defined. This will restore the Turbo Pascal
behaviour.
\item [memw and mem] these arrays gave access to the \dos memory. \fpc
supports them, they are mapped into \dos memory space. You need the
\var{GO32} unit for this.
\end{description}

You shouldn't use these functions, since they are very non-portable, they're
specific to \dos and the ix86 processor. The \fpc compiler is designed to be
portable to other platforms, so you should keep your code as portable as
possible, and not system specific. That is, unless you're writing some driver
units, of course.

%%%%%%%%%%%%%%%%%%%%%%%%%%%%%%%%%%%%%%%%%%%%%%%%%%%%%%%%%%%%%%%%%%%%%
% The stack
\section{The stack}
\label{se:Stack}
The stack is used to pass parameters to procedures or functions,
to store local variables, and, in some cases, to return function
results.

When a function or procedure is called, then the following is done by the
compiler :
\begin{enumerate}
\item If there are any parameters to be passed to the procedure, they are
pushed from right to left on the stack.
\item If a function is called that returns a variable of type \var{String},
\var{Set}, \var{Record}, \var{Object} or \var{Array}, then an address to
store the function result in, is pushed on the stack.
\item If the called procedure or function is an object method, then the
pointer to \var{self} is pushed on the stack.
\item If the procedure or function is nested in another function or
procedure, then the frame pointer of the parent procedure is pushed on the
stack.
\item The return address is pushed on the stack (This is done automatically
by the instruction which calls the subroutine).
\end{enumerate}

The resulting stack frame upon entering looks as in \seet{StackFrame}.
\begin{FPCltable}{llc}{Stack frame when calling a procedure}{StackFrame}
\hline
Offset & What is stored & Optional ? \\ \hline
+x & parameters & Yes \\
+12 & function result & Yes \\
+8 & self & Yes \\
+4 & Frame pointer of parent procedure & Yes \\
+0 & Return address & No\\ \hline
\end{FPCltable}

\subsection{ Intel x86 version }

The stack is cleared with the \var{ret} I386 instruction, meaning that the
size of all pushed parameters is limited to 64K.

\subsubsection{ DOS }

Under the DOS targets , the default stack is set to 256Kb. This value
cannot be modified for the GO32V1 target. But this can be modified
with the GO32V2 target using a special DJGPP utility \var{stubedit}.
It is to note that the stack size may be changed with some compiler
switches, this stack size, if \emph{greater} then the default stack
size will be used instead, otherwise the default stack size is used.

\subsubsection{ Linux }

Under Linux, stack size is only limited by the available memory by
the system.

\subsubsection{ OS/2 }

Under OS/2, stack size is determined by one of the runtime
environment variables set for EMX. Therefore, the stack size
is user defined.

\subsection{ Motorola 680x0 version }

All depending on the processor target, the stack can be cleared in two
manners, if the target processor is a MC68020 or higher, the stack will
be cleared with a simple \var{rtd} instruction, meaning that the size
of all pushed parameters is limited to 32K.

Otherwise on MC68000/68010 processors, the stack clearing mechanism
is sligthly more complicated, the exit code will look like this:

\begin{verbatim}
{
  move.l  (sp)+,a0
  add.l   paramsize,a0
  move.l  a0,-(sp)
  rts
}
\end{verbatim}

\subsubsection{ Amiga }

Under AmigaOS, stack size is determined by the user, which sets this
value using the stack program. Typical sizes range from 4K to 40K.

\subsubsection{ Atari }

Under Atari TOS, stack size is currently limited to 8K, and it cannot
be modified. This may change in a future release of the compiler.

%%%%%%%%%%%%%%%%%%%%%%%%%%%%%%%%%%%%%%%%%%%%%%%%%%%%%%%%%%%%%%%%%%%%%
% The heap
\section{The heap}
\label{se:Heap}
The heap is used to store all dynamic variables, and to store class
instances. The interface to the heap is the same as in Turbo Pascal,
although the effects are maybe not the same. On top of that, the \fpc
run-time library has some extra possibilities, not available in Turbo
Pascal. These extra possibilities are explained in the next subsections.


% The heap grows
\subsection{The heap grows}
\fpc supports the \var{HeapEerror} procedural variable. If this variable is
non-nil, then it is called in case you try to allocate memory, and the heap
is full. By default, \var{HeapError} points to the \var{GrowHeap} function,
which tries to increase the heap.

The growheap function issues a system call to try to increase the size of the
memory available to your program. It first tries to increase memory in a 1 Mb.
chunk. If this fails, it tries to increase the heap by the amount you
requested from the heap.

If the call to \var{GrowHeap} has failed, then a run-time error is generated,
or nil is returned, depending on the \var{GrowHeap} result.

If the call to \var{GrowHeap} was successful, then the needed memory will be
allocated.

% Using Blocks
\subsection{Using Blocks}
If you need to allocate a lot of small block for a small period, then you
may want to recompile the run-time library with the \var{USEBLOCKS} symbol
defined. If it is recompiled, then the heap management is done in a
different way.

The run-time library keeps a linked list of allocated blocks with size
up to 256 bytes\footnote{The size can be set using the \var{max\_size}
constant in the \file{heap.inc} source file.}. By default, it keeps 32 of
these lists\footnote{The actual size is \var{max\_size div 8}.}.

When a piece of memory in a block is deallocated, the heap manager doesn't
really deallocate the occupied memory. The block is simply put in the linked
list corresponding to its size.

When you then again request a block of memory, the manager checks in the
list if there is a non-allocated block which fits the size you need (rounded
to 8 bytes). If so, the block is used to allocate the memory you requested.

This method of allocating works faster if the heap is very fragmented, and
you allocate a lot of small memory chunks.

Since it is invisible to the program, this provides an easy way of improving
the performance of the heap manager.

% The splitheap
\subsection{Using the split heap}
{\em Remark : The split heap is still somewhat buggy. Use at your own risk
for the moment.}

The split heap can be used to quickly release a lot of blocks you alloated
previously.

Suppose that in a part of your program, you allocate a lot of memory chunks
on the heap. Suppose that you know that you'll release all this memory when
this particular part of you program is finished.

In Turbo Pascal, you could foresee this, and mark the position of the heap
(using the \var{Mark} function) when entering this particular part of your
program, and release the occupied memory in one call with the \var{Release}
call.

For most purposes, this works very good. But sometimes, you may need to
allocate something on the heap that you {\em don't} want deallocated when you
release the allocated memory. That is where the split heap comes in.

When you split the heap, the heap manager keeps 2 heaps: the base heap (the
normal heap), and the temporary heap. After the call to split the heap,
memory is allocated from the temporary heap. When you're finished using all
this memory, you unsplit the heap. This clears all the memory on the split
heap with one call. After that, memory will be allocated from the base heap
again.

So far, nothing special, nothing that can't be done with calls to \var{mark}
and \var{release}. Suppose now that you have split the heap, and that you've
come to a point where you need to allocate memory that is to stay allocated
after you unsplit the heap again. At this point, mark and release are of no
use. But when using the split heap, you can tell the heap manager to
--temporarily-- use the base heap again to allocate memory.
When you've allocated the needed memory, you can tell the heap manager that
it should start using the temporary heap again.
When you're finished using the temporary heap, you release it, and the
memory you allocated on the base heap will still be allocated.

To use the split-heap, you must recompile the run-time library with the \var{TempHeap}
symbol defined.
This means that the following functions are available :
\begin{verbatim}
  procedure Split_Heap;
  procedure Switch_To_Base_Heap;
  procedure Switch_To_Temp_Heap;
  procedure Switch_Heap;
  procedure ReleaseTempHeap;
  procedure GetempMem(var p : pointer;size : longint);
\end{verbatim}
\var{split\_heap} is used to split the heap. It cannot be called two times
in a row, without a call to \var{releasetempheap}. \var{Releasetempheap}
completely releases the memory used by the temporary heap.
Switching temporarily back to the base heap can be done using the
\var{switch\_to\_base\_heap} call, and returning to the temporary heap is done
using the \var{switch\_to\_temp\_heap} call. Switching from one to the other
without knowing on which one your are right now, can be done using the
\var{switch\_heap} call, which will split the heap first if needed.

A call to \var{GetTempMem} will allocate a memory block on the temporary
heap, whatever the current heap is. The current heap after this call will be
the temporary heap.

Typically, what will appear in your code is the following sequence :
\begin{verbatim}
Split_Heap
...
{ Memory allocation }
...
{ !! non-volatile memory needed !!}
Switch_To_Base_Heap;
getmem (P,size);
Switch_To_Temp_Heap;
...
{Memory allocation}
...
ReleaseTempHeap;
{All allocated memory is now freed, except for the memory pointed to by 'P' }
...
\end{verbatim}

%%%%%%%%%%%%%%%%%%%%%%%%%%%%%%%%%%%%%%%%%%%%%%%%%%%%%%%%%%%%%%%%%%%%%
% Accessing DOS memory under the GO32 extender
\section{using \dos memory under the Go32 extender}
\label{se:AccessingDosMemory}

Because \fpc is a 32 bit compiler, and uses a \dos extender, accessing DOS
memory isn't trivial. What follows is an attempt to an explanation of how to
access and use \dos or real mode memory\footnote{Thanks to an explanation of
Thomas schatzl (E-mail:\var{tom\_at\_work@geocities.com}).}.

In {\em Proteced Mode}, memory is accessed through {\em Selectors} and
{\em Offsets}. You can think of Selectors as the protected mode
equivalents of segments.

In \fpc, a pointer is an offset into the \var{DS} selector, which points to
the Data of your program.

To access the (real mode) \dos memory, somehow you need a selector that
points to the \dos memory.
The \file{GO32} unit provides you with such a selector: The
\var{DosMemSelector} variable, as it is conveniently called.

You can also allocate memory in \dos's memory space, using the
\var{global\_dos\_alloc} function of the \file{GO32} unit.
This function will allocate memory in a place where \dos sees it.

As an example, here is a function that returns memory in real mode \dos and
returns a selector:offset pair for it.
\begin{verbatim}
procedure dosalloc(var selector : word;
                   var segment : word;
                   size : longint);

var result : longint;

begin
     result := global_dos_alloc(size);
     selector := word(result);
     segment := word(result shr 16);
end;
\end{verbatim}
(you need to free this memory using the \var{global\_dos\_free} function.)

You can access any place in memory using a selector. You can get a selector
using the \var{allocate\_ldt\_descriptor} function, and then let this selector
point to the physical memory you want using the
\var{set\_segment\_base\_address} function, and set its length using
\var{set\_segment\_limit} function.
You can manipulate the memory pointed to by the selector using the functions
of the GO32 unit. For instance with the \var{seg\_fillchar} function.
After using the selector, you must free it again using the
\var{free\_ldt\_selector} function.

More information on all this can be found in the \unitsref, the chapter on
the \file{GO32} unit.

%%%%%%%%%%%%%%%%%%%%%%%%%%%%%%%%%%%%%%%%%%%%%%%%%%%%%%%%%%%%%%%%%%%%%
% Optimizations done in the compiler
%%%%%%%%%%%%%%%%%%%%%%%%%%%%%%%%%%%%%%%%%%%%%%%%%%%%%%%%%%%%%%%%%%%%%
\chapter{Optimizations}

\section{ Non processor specific }

The following sections describe the general optimizations
done by the compiler, they are non processor specific. Some
of these require some compiler switch override while others are done
automatically (those which require a switch will be noted as such).

\subsection{ Constant folding }

In \fpc, if the operand(s) of an operator are constants, they
will be evaluated at compile time.

Example

\begin{verbatim}
   x:=1+2+3+6+5;
will generate the same code as
   x:=17;
\end{verbatim}

Furthermore, if an array index is a constant, the offset will
be evaluated at compile time. This means that accessing MyData[5]
is as efficient as accessing a normal variable.

Finally, calling \var{Chr}, \var{Hi}, \var{Lo}, \var{Ord}, \var{Pred},
or \var{Succ} functions with constant parameters generates no
run-time library calls, instead, the values are evaluated at
compile time.

\subsection{ Constant merging }

Using the same constant string two or more times generates only
one copy of the string constant.

\subsection{ Short cut evaluation }

Evaluation of boolean expression stops as soon as the result is
known, which makes code execute faster then if all boolean operands
were evaluted.

\subsection{ Constant set inlining }

Using the \var{in} operator is always more efficient then using the
equivalent <>, =, <=, >=, < and > operators. This is because
range comparisons can be done more easily with \var{in} then with
normal comparison operators.

\subsection{ Small sets }

Sets which contain less then 33 elements can be directly encoded
using a 32-bit value, therefore no run-time library calls to
evaluate operands on these sets are required; they are directly encoded
by the code generator.

\subsection{ Range checking }

Assignments of constants to variables are range checked at compile
time, which removes the need the generation of runtime range checking
code.

\emph{Remark:} This feature was not implemented before version
0.99.5 of \fpc.

\subsection{ Shifts instead of multiply or divide }

When one of the operands in a multiplication is a power of
two, they are encoded using arithmetic shifts instructions,
which generates more efficient code.

Similarly, if the divisor in a \var{div} operation is a power
of two, it is encoded using arithmetic shifts instructions.

The same is true when accessing array indexes which are
powers of two, the address is calculated using arithmetic
shifts instead of the multiply instruction.

\subsection{ Automatic alignment }

By default all variables larger then a byte are guaranteed to be aligned
at least on a word boundary.

Furthermore all pointers allocated using the standard runtime
library (\var{New} and \var{GetMem} among others) are guaranteed
to return pointers aligned on a quadword boundary (64-bit alignment).

Alignment of variables on the stack depends on the target processor.

\emph{ Remark: } Quadword alignment of pointers is not guaranteed
on systems which don't use an internal heap, such as for the Win32
target.

\emph{ Remark: } Alignment is also done \emph{between} fields in
records, objects and classes, this is \emph{not} the same as
in Turbo Pascal and may cause problems when using disk I/O with these
types. To get no alignment between fields use the \var{packed} directive
or the \var{\{\$PackRecords n\}} switch. For further information, take a
look at the reference manual under the \var{record} heading.

\subsection{ Smart linking }

This feature removes all unreferenced code in the final executable
file, making the executable file much smaller.

\emph{ Remark: } Smart linking was implemented starting with
version 0.99.6 of \fpc.

\subsection{ Inline routines }

The following runtime library routines are coded directly into the
final executable : \var{Lo}, \var{Hi}, \var{High}, \var{Sizeof},
\var{TypeOf}, \var{Length}, \var{Pred}, \var{Succ}, \var{Inc},
\var{Dec} and \var{Assigned}.

\emph{ Remark: } Inline \var{Inc} and \var{Dec} were not completely
implemented until version 0.99.6 of \fpc.

\subsection{ Case optimization }

When using the \var{-Oa} switch, case statements in certain cases will
be decoded using a jump table, which in certain cases will make the
case statement execute faster.

\subsection{ Stack frame omission }

When using the \var{-Ox} switch, under certain specific conditions,
the stack frame (entry and exit code for the routine) will be omitted, and
the variable will directly be accessed via the stack pointer.

Conditions for omission of the stack frame :

\begin{itemize}
\item Routine does not call other routines
\item Routine does not contain assembler statements
\item Routine is not declared using the \var{Interrupt} directive
\item Routine is not a constructor or destructor
\end{itemize}

\subsection{ Register variables }

When using the \var{-Ox} switch, local variables or parameters
which are used very often will be moved to registers for faster
access.

\emph{ Remark: } Register variable allocation is currently
broken and should not be used.

\subsection{ Intel x86 specific }

Here follows a listing of the opimizing techniques used in the compiler:
\begin{enumerate}
\item When optimizing for a specific Processor (\var{-O3, -O4, -O5 -O6},
the following is done:
\begin{itemize}
\item In \var{case} statements, a check is done whether a jump table
or a sequence of conditional jumps should be used for optimal performance.
\item Determines a number of strategies when doing peephole optimization:
\var{movzbl (\%ebp), \%eax} on PentiumPro and PII systems will be changed
into \var{xorl \%eax,\%eax; movb (\%ebp),\%al } for lesser systems.
\end{itemize}
Cyrix \var{6x86} processor owners should optimize with \var{-O4} instead of
\var{-O5}, because \var{-O5} leads to larger code, and thus to smaller
speed, according to the Cyrix developers FAQ.
  \item When optimizing for speed (\var{-OG}) or size (\var{-Og}), a choice is
made between using shorter instructions (for size) such as \var{enter \$4},
or longer instructions \var{subl \$4,\%esp} for speed. When smaller size is
requested, things aren't aligned on 4-byte boundaries.  When speed is
requested, things are aligned on 4-byte boundaries as much as possible.
\item Simple optimization (\var{-Oa}) makes sure the peephole optimizer is
used, as well as the reloading optimizer.
\item Uncertain optimizations (\var{-Oz}): With this switch, the reloading
optimizer (enabled with \var{-Oa}) can be forced into making uncertain
optimizations.

You can enable uncertain optimizations only in certain cases,
otherwise you will produce a bug; the following technical description
tells you when to use them:
\begin{quote}
% Jonas's own words..
\em
If uncertain optimizations are enabled, the reloading optimizer assumes
that
\begin{itemize}
\item If something is written to a local/global register or a
procedure/function parameter, this value doesn't overwrite the value to
which a pointer points.
\item If something is written to memory pointed to by a pointer variable,
this value doesn't overwrite the value of a local/global variable or a
procedure/function parameter.
\end{itemize}
% end of quote
\end{quote}
The practical upshot of this is that you cannot use the uncertain
optimizations if you access any local or global variables through pointers. In
theory, this includes \var{Var} parameters, but it is all right
if you don't both read the variable once through its \var{Var} reference
and then read it using it's name.

The following example will produce bad code when you switch on
uncertain optimizations:
\begin{verbatim}
Var temp: Longint;

Procedure Foo(Var Bar: Longint);
Begin
  If (Bar = temp)
    Then
      Begin
        Inc(Bar);
        If (Bar <> temp) then Writeln('bug!')
      End
End;

Begin
  Foo(Temp);
End.
\end{verbatim}
The reason it produces bad code is because you access the global variable
\var{Temp} both through its name \var{Temp} and through a pointer, in this
case using the \var{Bar} variable parameter, which is nothing but a pointer
to \var{Temp} in the above code.

On the other hand, you can use the uncertain optimizations if
you access global/local variables or parameters through pointers,
and {\em only} access them through this pointer\footnote{
You can use multiple pointers to point to the same variable as well, that
doesn't matter.}.

For example:
\begin{verbatim}
Type TMyRec = Record
                a, b: Longint;
              End;
     PMyRec = ^TMyRec;


     TMyRecArray = Array [1..100000] of TMyRec;
     PMyRecArray = ^TMyRecArray;

Var MyRecArrayPtr: PMyRecArray;
    MyRecPtr: PMyRec;
    Counter: Longint;

Begin
  New(MyRecArrayPtr);
  For Counter := 1 to 100000 Do
    Begin
       MyRecPtr := @MyRecArrayPtr^[Counter];
       MyRecPtr^.a := Counter;
       MyRecPtr^.b := Counter div 2;
    End;
End.
\end{verbatim}
Will produce correct code, because the global variable \var{MyRecArrayPtr}
is not accessed directly, but through a pointer (\var{MyRecPtr} in this
case).

In conclusion, one could say that you can use uncertain optimizations {\em
only} when you know what you're doing.
\end{enumerate}

\subsection{ Motorola 680x0 specific }

Using the \var{-O2} switch does several optimizations in the
code produced, the most notable being:

\begin{itemize}
\item Sign extension from byte to long will use \var{EXTB}
\item Returning of functions will use \var{RTD}
\item Range checking will generate no run-time calls
\item Multiplication will use the long \var{MULS} instruction, no
runtime library call will be generated
\item Division will use the long \var{DIVS} instruction, no
runtime library call will be generated
\end{itemize}


\section{ Floating point }

This is where can be found processor specific information on Floating
point code generated by the compiler.

\subsection{ Intel x86 specific }

All normal floating point types map to their real type, including
\var{comp} and \var{extended}.

\subsection{ Motorola 680x0 specific }

Early generations of the Motorola 680x0 processors did not have integrated
floating point units, so to circumvent this fact, all floating point
operations are emulated (when the \var{\$E+} switch ,which is the default)
using the IEEE \var{Single} floating point type. In other words when
emulation is on, Real, Single, Double and Extended all map to the
\var{single} floating point type.

When the \var{\$E} switch is turned off, normal 68882/68881/68040
floating point opcodes are emitted. The Real type still maps to
\var{Single} but the other types map to their true floating point
types. Only basic FPU opcodes are used, which means that it can
work on 68040 processors correctly.

\emph{ Remark: } \var{Double} and \var{Extended} types in true floating
point mode have not been extensively tested as of version 0.99.5.

\emph{ Remark: } The \var{comp} data type is currently not supported.

%%%%%%%%%%%%%%%%%%%%%%%%%%%%%%%%%%%%%%%%%%%%%%%%%%%%%%%%%%%%%%%%%%%%%
% Appendices
%%%%%%%%%%%%%%%%%%%%%%%%%%%%%%%%%%%%%%%%%%%%%%%%%%%%%%%%%%%%%%%%%%%%%
\appendix

%%%%%%%%%%%%%%%%%%%%%%%%%%%%%%%%%%%%%%%%%%%%%%%%%%%%%%%%%%%%%%%%%%%%%
% Appendix A
%%%%%%%%%%%%%%%%%%%%%%%%%%%%%%%%%%%%%%%%%%%%%%%%%%%%%%%%%%%%%%%%%%%%%

\chapter{Anatomy of a unit file}
\label{ch:AppA}

%%%%%%%%%%%%%%%%%%%%%%%%%%%%%%%%%%%%%%%%%%%%%%%%%
% Basics
\section{Basics}

The best and most updated documentation about the ppu files can be found
in \file{ppu.pas} and \file{ppudump.pp} which can be found in
\file{rtl/utils/}.

To read or write the ppufile, you can use the ppu unit \file{ppu.pas}
which has an object called tppufile which holds all routines that deal
with ppufile handling. Describing the layout of a ppufile, the methods
which can be used for it are described.

A unit file consists of basically five or six parts:
\begin{enumerate}
\item A unit header.
\item A file interface part.
\item A definition part. Contains all type and procedure definitions.
\item A symbol part. Contains all symbol names and references to their
definitions.
\item A browser part. Contains all references from this unit to other
units and inside this unit. Only available when the \var{uf\_has\_browser} flag is
set in the unit flags
\item A file implementation part (currently unused).
implementation part.
\end{enumerate}

\section{reading ppufiles}

We will first create an object ppufile which will be used below. We are
opening unit \file{test.ppu} as an example.

\begin{verbatim}
var
  ppufile : pppufile;
begin
{ Initialize object }
  ppufile:=new(pppufile,init('test.ppu');
{ open the unit and read the header, returns false when it fails }
  if not ppufile.open then
    error('error opening unit test.ppu');

{ here we can read the unit }

{ close unit }
  ppufile.close;
{ release object }
  dispose(ppufile,done);
end;
\end{verbatim}

Note: When a function fails (for example not enough bytes left in an
entry) it sets the \var{ppufile.error} variable.

%%%%%%%%%%%%%%%%%%%%%%%%%%%%%%%%%%%%%%%%%%%%%%%%%
% The Header
\section{The Header}

The header consists of a record containing 24 bytes:

\begin{verbatim}
tppuheader=packed record                                                      
    id       : array[1..3] of char; { = 'PPU' }                                 
    ver      : array[1..3] of char;                                             
    compiler : word;                                                            
    cpu      : word;                                                            
    target   : word;                                                            
    flags    : longint;                                                         
    size     : longint; { size of the ppufile without header }                  
    checksum : longint; { checksum for this ppufile }                           
  end;                 
\end{verbatim}

The header is already read by the \var{ppufile.open} command. 
You can access all fields using \var{ppufile.header} which holds 
the current header record.

\begin{tabular}{lp{10cm}}
\raggedright
field & description \\ \hline
\var{id} & 
this is allways 'PPU', can be checked with
\mbox{\var{function ppufile.CheckPPUId:boolean;}} \\
\var{ver} & ppu version, currently '015', can be checked with
\mbox{\var{function ppufile.GetPPUVersion:longint;}} (returns 15) \\
\var{compiler}
 & compiler version used to create the unit. Doesn't contain the
	 patchlevel. Currently 0.99 where 0 is the high byte and 99 the
	 low byte \\
\var{cpu} & cpu for which this unit is created.
          0 = i386
          1 = m68k \\
\var{target} & target for which this unit is created, this depends also on the
	 cpu! 

	 For i386:
\begin{tabular}[t]{ll}
0 & Go32v1 \\
1 & Go32V2 \\
2 & Linux-i386 \\
3 & OS/2 \\
4 & Win32
\end{tabular}

For m68k:
\begin{tabular}[t]{ll}
0 & Amiga \\
1 & Mac68k \\
2 & Atari \\
3 & Linux-m68k
\end{tabular} \\
\var{flag} & 
the unit flags, contains a combination of the uf\_ constants which
are definied in \file{ppu.pas} \\
\var{size} & size of this unit without this header \\
\var{checksum} & 
  checksum of the interface parts of this unit, which determine if
         a unit is changed or not, so other units can see if they need to
	 be recompiled 
\\ \hline
\end{tabular}

% The sections
\section{The sections}

After this header follow the sections. All sections work the same! 
A section contains of entries and is ended with also an entry, but
containing the specific ibend constant (see \file{ppu.pas} for a list).

Each entry starts with an entryheader.
\begin{verbatim}
  tppuentry=packed record                                                       
    id   : byte;                                                                
    nr   : byte;                                                                
    size : longint;                                                             
  end;      
\end{verbatim}

\begin{tabular}{lp{10cm}}
field & Description \\ \hline
id & this is 1 or 2 and can be check if it the entry is correctly
found. 1 means its a main entry, which says that it is part of the
basic layout as explained before. 2 toggles that it it a sub entry
of a record or object \\
nr & contains the ib constant number which determines what kind of
entry it is \\
size & size of this entry without the header, can be used to skip entries
very easily. \\ \hline
\end{tabular}

To read an entry you can simply call \var{ppufile.readentry:byte}, 
it returns the
\var{tppuentry.nr} field, which holds the type of the entry. 
A common way how this works is (example is for the symbols):

\begin{verbatim}
  repeat
    b:=ppufile.readentry;
    case b of
   ib<etc> : begin
             end;
 ibendsyms : break;
    end;
  until false;
\end{verbatim}

Then you can parse each entry type yourself. \var{ppufile.readentry} will take
care of skipping unread bytes in the entry an read the next entry
correctly! A special function is \var{skipuntilentry(untilb:byte):boolean;}
which will read the ppufile until it finds entry \var{untilb} in the main
entries.

Parsing an entry can be done with \var{ppufile.getxxx} functions. The
available functions are:
\begin{verbatim}
procedure ppufile.getdata(var b;len:longint);
function  getbyte:byte;                                                     
function  getword:word;                                                     
function  getlongint:longint;                                               
function  getreal:ppureal;                                                  
function  getstring:string;      
\end{verbatim}

To check if you're at the end of an entry you can use the following
function:

\begin{verbatim}
function  EndOfEntry:boolean;                                               
\end{verbatim}
{\em notes:}
\begin{enumerate}
\item \var{ppureal} is the best real that exists for the cpu where the
unit is created for. Currently it is \var{extended} for i386 and 
\var{single} for m68k.
\item the \var{ibobjectdef} and \var{ibrecorddef} have stored a definition 
and symbol section for themselves. So you'll need a recursive call. See
\file{ppudump.pp} for a correct implementation.
\end{enumerate}

A complete list of entries and what their fields contain can be found
in \file{ppudump.pp}.

\section{Creating ppufiles}

Creating a new ppufile works almost the same as writing. First you need
to init the object and call create:
\begin{verbatim}
  ppufile:=new(pppufile,'output.ppu');
  ppufile.create;
\end{verbatim}

After that you can simply write all needed entries. You'll have to take
care that you write at least the basic entries for the sections:
\begin{verbatim}
  ibendinterface
  ibenddefs
  ibendsyms
  ibendbrowser (only when you've set uf_has_browser!)
  ibendimplementation
  ibend
\end{verbatim}

Writing an entry is a little different than reading it. You need to first
put everything in the entry with ppufile.putxxx:
\begin{verbatim}
procedure putdata(var b;len:longint);                                       
procedure putbyte(b:byte);                                                  
procedure putword(w:word);                                                  
procedure putlongint(l:longint);                                            
procedure putreal(d:ppureal);                                               
procedure putstring(s:string); 
\end{verbatim}

After putting all the things in the entry you need to call
\var{ppufile.writeentry(ibnr:byte)} where \var{ibnr} is the entry number 
you're writing.

At the end of the file you need to call \var{ppufile.writeheader} to write the
new header to the file. This takes automatically care of the new size of the
ppufile. When that is also done you can call \var{ppufile.close} and dispose the
object.

Extra functions/variables available for writing are:
\begin{verbatim}
ppufile.NewHeader;                                           
ppufile.NewEntry;   
\end{verbatim}
This will give you a clean header or entry. Normally called automatically
in \var{ppufile.writeentry}, so you can't forget it.
\begin{verbatim}
ppufile.flush;                                                            
\end{verbatim}

to flush the current buffers to the disk
\begin{verbatim}
ppufile.do_crc:boolean;
\end{verbatim}
set to false if you don't want that the crc is updated, this is necessary
if you write for example the browser data.
   
%%%%%%%%%%%%%%%%%%%%%%%%%%%%%%%%%%%%%%%%%%%%%%%%%%%%%%%%%%%%%%%%%%%%%
% Appendix B
%%%%%%%%%%%%%%%%%%%%%%%%%%%%%%%%%%%%%%%%%%%%%%%%%%%%%%%%%%%%%%%%%%%%%

\chapter{Compiler and RTL source tree structure}
\label{ch:AppB}

%%%%%%%%%%%%%%%%%%%%%%%%%%%%%%%%%%%%%%%%%%%%%%%%%
% The compiler source tree
\section{The compiler source tree}

All compiler source files are in one directory, normally in
\file{source/compiler}. For more informations
about the structure of the compiler have a look at the
Compiler Manual which contains also some informations about
compiler internals.

%%%%%%%%%%%%%%%%%%%%%%%%%%%%%%%%%%%%%%%%%%%%%%%%%%%%%%%%%%%%%%%%%%%%%
% Appendix C
%%%%%%%%%%%%%%%%%%%%%%%%%%%%%%%%%%%%%%%%%%%%%%%%%%%%%%%%%%%%%%%%%%%%%

\chapter{Compiler limits}
\label{ch:AppC}
Although many of the restrictions imposed by the MS-DOS system are removed
by use of an extender, or use of another operating system, there still are
some limitations to the compiler:
\begin{enumerate}
\item Procedure or Function definitions can be nested to a level of 32.
\item Maximally 255 units can be used in a program when using the real-mode
compiler. When using the 32-bit compiler, the limit is set to 1024. You can
change this by redefining the \var{maxunits} constant in the 
\file{files.pas} compiler source file.
\end{enumerate}

\end{document}
