%
%   $Id$
%   This file is part of the FPC documentation.
%   Copyright (C) 2000 by Florian Klaempfl
%
%   The FPC documentation is free text; you can redistribute it and/or
%   modify it under the terms of the GNU Library General Public License as
%   published by the Free Software Foundation; either version 2 of the
%   License, or (at your option) any later version.
%
%   The FPC Documentation is distributed in the hope that it will be useful,
%   but WITHOUT ANY WARRANTY; without even the implied warranty of
%   MERCHANTABILITY or FITNESS FOR A PARTICULAR PURPOSE.  See the GNU
%   Library General Public License for more details.
%
%   You should have received a copy of the GNU Library General Public
%   License along with the FPC documentation; see the file COPYING.LIB.  If not,
%   write to the Free Software Foundation, Inc., 59 Temple Place - Suite 330,
%   Boston, MA 02111-1307, USA.
%
\chapter{The MATH unit}
\FPCexampledir{mathex}

This chapter describes the \file{math} unit. The \var{math} unit
was initially written by Florian Klaempfl. It provides mathematical
functions which aren't covered by the system unit.

This chapter starts out with a definition of all types and constants
that are defined, followed by a complete explanation of each function.
{\em Remark} This unit is compiled in Object Pascal mode so all
\var{integers} are 32 bit.

\section{Constants and types}

The following types are defined in the \file{math} unit:
\begin{verbatim}
Type
   Float = Extended;
\end{verbatim}

\begin{verbatim}
Type
   TPaymentTime = (PTEndOfPeriod,PTStartOfPeriod);
\end{verbatim}

\begin{verbatim}
Type
   EInvalidArgument = Class(EMathError);
\end{verbatim}

\section{Functions and Procedures}

\begin{function}{arccos}
\Declaration
Function arccos(x : float) : float;
\Description
\var{Arccos} returns the inverse cosine of its argument \var{x}. The
argument \var{x} should lie between -1 and 1 (borders included). 
\Errors
If the argument \var{x} is not in the allowed range, an \var{EMathError}
exception is raised.
\SeeAlso
\seef{arcsin}, \seef{arcosh}, \seef{arsinh}, \seef{arctanh}
\end{function}

\FPCexample{ex1}

\begin{function}{arcosh}
\Declaration
Function arcosh(x : float) : float;
Function arccosh(x : float) : float;
\Description
\var{Arcosh} returns the inverse hyperbolic cosine of its argument \var{x}. 
The argument \var{x} should be larger than 1. 

The \var{arccosh} variant of this function is supplied for \delphi 
compatibility.
\Errors
If the argument \var{x} is not in the allowed range, an \var{EMathError}
exception is raised.
\SeeAlso
\seef{cosh}, \seef{sinh}, \seef{arcsin}, \seef{arcsinh}, \seef{arctanh},
\seef{tanh}
\end{function}

\FPCexample{ex3}

\begin{function}{arcsin}
\Declaration
Function arcsin(x : float) : float;
\Description
\var{Arcsin} returns the inverse sine of its argument \var{x}. The
argument \var{x} should lie between -1 and 1. 
\Errors
If the argument \var{x} is not in the allowed range, an \var{EMathError}
exception is raised.
\SeeAlso
\seef{arccos}, \seef{arccosh}, \seef{arcsinh}, \seef{arctanh}
\end{function}

\FPCexample{ex2}


\begin{function}{arcsinh}
\Declaration
Function arcsinh(x : float) : float;
\Description

\Errors
\SeeAlso
\end{function}

\FPCexample{}


\begin{function}{arctan2}
\Declaration
Function arctan2(x,y : float) : float;
\Description

\Errors
\SeeAlso
\end{function}

\FPCexample{}


\begin{function}{arctanh}
\Declaration
Function arctanh(x : float) : float;
\Description

\Errors
\SeeAlso
\end{function}

\FPCexample{}


\begin{function}{arsinh}
\Declaration
Function arsinh(x : float) : float;
\Description

\Errors
\SeeAlso
\end{function}

\FPCexample{}


\begin{function}{artanh}
\Declaration
Function artanh(x : float) : float;
\Description

\Errors
\SeeAlso
\end{function}

\FPCexample{}


\begin{function}{ceil}
\Declaration
Function ceil(x : float) : longint;
\Description

\Errors
\SeeAlso
\end{function}

\FPCexample{}


\begin{function}{cosh}
\Declaration
Function cosh(x : float) : float;
\Description

\Errors
\SeeAlso
\end{function}

\FPCexample{}


\begin{function}{cotan}
\Declaration
Function cotan(x : float) : float;
\Description

\Errors
\SeeAlso
\end{function}

\FPCexample{}


\begin{function}{cycletorad}
\Declaration
Function cycletorad(cycle : float) : float;
\Description

\Errors
\SeeAlso
\end{function}

\FPCexample{}


\begin{function}{degtograd}
\Declaration
Function degtograd(deg : float) : float;
\Description

\Errors
\SeeAlso
\end{function}

\FPCexample{}


\begin{function}{degtorad}
\Declaration
Function degtorad(deg : float) : float;
\Description

\Errors
\SeeAlso
\end{function}

\FPCexample{}


\begin{function}{floor}
\Declaration
Function floor(x : float) : longint;
\Description

\Errors
\SeeAlso
\end{function}

\FPCexample{}


\begin{procedure}{frexp}
\Declaration
Procedure frexp(x : float;var mantissa,exponent : float);
\Description

\Errors
\SeeAlso
\end{procedure}

\FPCexample{}


\begin{function}{gradtodeg}
\Declaration
Function gradtodeg(grad : float) : float;
\Description

\Errors
\SeeAlso
\end{function}

\FPCexample{}


\begin{function}{gradtorad}
\Declaration
Function gradtorad(grad : float) : float;
\Description

\Errors
\SeeAlso
\end{function}

\FPCexample{}


\begin{function}{hypot}
\Declaration
Function hypot(x,y : float) : float;
\Description

\Errors
\SeeAlso
\end{function}

\FPCexample{}


\begin{function}{intpower}
\Declaration
Function intpower(base : float;exponent : longint) : float;
\Description

\Errors
\SeeAlso
\end{function}

\FPCexample{}


\begin{function}{ldexp}
\Declaration
Function ldexp(x : float;p : longint) : float;
\Description

\Errors
\SeeAlso
\end{function}

\FPCexample{}


\begin{function}{lnxpi}
\Declaration
Function lnxpi(x : float) : float;
\Description

\Errors
\SeeAlso
\end{function}

\FPCexample{}


\begin{function}{log10}
\Declaration
Function log10(x : float) : float;
\Description

\Errors
\SeeAlso
\end{function}

\FPCexample{}


\begin{function}{log2}
\Declaration
Function log2(x : float) : float;
\Description

\Errors
\SeeAlso
\end{function}

\FPCexample{}


\begin{function}{logn}
\Declaration
Function logn(n,x : float) : float;
\Description

\Errors
\SeeAlso
\end{function}

\FPCexample{}


  var m1,m2,m3,m4,skew,kurtosis : float);
\begin{function}{max}
\Declaration
Function max(Int1,Int2:Cardinal):Cardinal;
\Description

\Errors
\SeeAlso
\end{function}

\FPCexample{}


\begin{function}{max}
\Declaration
Function max(Int1,Int2:Integer):Integer;
\Description

\Errors
\SeeAlso
\end{function}

\FPCexample{}


\begin{function}{maxIntValue}
\Declaration
Function maxIntValue(const Data: array of Integer): Integer;
\Description

\Errors
\SeeAlso
\end{function}

\FPCexample{}


\begin{function}{maxvalue}
\Declaration
Function maxvalue(const data : array of float) : float;
\Description

\Errors
\SeeAlso
\end{function}

\FPCexample{}


\begin{function}{mean}
\Declaration
Function mean(const data : array of float) : float;
\Description

\Errors
\SeeAlso
\end{function}

\FPCexample{}


  var mean,stddev : float);
\begin{procedure}{meanandstddev}
\Declaration
Procedure meanandstddev(const data : array of float;
\Description

\Errors
\SeeAlso
\end{procedure}

\FPCexample{}


\begin{function}{min}
\Declaration
Function min(Int1,Int2:Cardinal):Cardinal;
\Description

\Errors
\SeeAlso
\end{function}

\FPCexample{}


\begin{function}{min}
\Declaration
Function min(Int1,Int2:Integer):Integer;
\Description

\Errors
\SeeAlso
\end{function}

\FPCexample{}


\begin{function}{minIntValue}
\Declaration
Function minIntValue(const Data: array of Integer): Integer;
\Description

\Errors
\SeeAlso
\end{function}

\FPCexample{}


\begin{function}{minvalue}
\Declaration
Function minvalue(const data : array of float) : float;
\Description

\Errors
\SeeAlso
\end{function}

\FPCexample{}


\begin{procedure}{momentskewkurtosis}
\Declaration
Procedure momentskewkurtosis(const data : array of float;
\Description

\Errors
\SeeAlso
\end{procedure}

\FPCexample{}


\begin{function}{norm}
\Declaration
Function norm(const data : array of float) : float;
\Description

\Errors
\SeeAlso
\end{function}

\FPCexample{}


\begin{function}{popnstddev}
\Declaration
Function popnstddev(const data : array of float) : float;
\Description

\Errors
\SeeAlso
\end{function}

\FPCexample{}


\begin{function}{popnvariance}
\Declaration
Function popnvariance(const data : array of float) : float;
\Description

\Errors
\SeeAlso
\end{function}

\FPCexample{}


\begin{function}{power}
\Declaration
Function power(base,exponent : float) : float;
\Description

\Errors
\SeeAlso
\end{function}

\FPCexample{}


\begin{function}{radtocycle}
\Declaration
Function radtocycle(rad : float) : float;
\Description

\Errors
\SeeAlso
\end{function}

\FPCexample{}


\begin{function}{radtodeg}
\Declaration
Function radtodeg(rad : float) : float;
\Description

\Errors
\SeeAlso
\end{function}

\FPCexample{}


\begin{function}{radtograd}
\Declaration
Function radtograd(rad : float) : float;
\Description

\Errors
\SeeAlso
\end{function}

\FPCexample{}


\begin{function}{randg}
\Declaration
Function randg(mean,stddev : float) : float;
\Description

\Errors
\SeeAlso
\end{function}

\FPCexample{}


\begin{procedure}{sincos}
\Declaration
Procedure sincos(theta : float;var sinus,cosinus : float);
\Description

\Errors
\SeeAlso
\end{procedure}

\FPCexample{}


\begin{function}{sinh}
\Declaration
Function sinh(x : float) : float;
\Description

\Errors
\SeeAlso
\end{function}

\FPCexample{}


\begin{function}{stddev}
\Declaration
Function stddev(const data : array of float) : float;
\Description

\Errors
\SeeAlso
\end{function}

\FPCexample{}


\begin{function}{sum}
\Declaration
Function sum(const data : array of float) : float;
\Description

\Errors
\SeeAlso
\end{function}

\FPCexample{}


  var sum,sumofsquares : float);
\begin{function}{sumofsquares}
\Declaration
Function sumofsquares(const data : array of float) : float;
\Description

\Errors
\SeeAlso
\end{function}

\FPCexample{}


\begin{procedure}{sumsandsquares}
\Declaration
Procedure sumsandsquares(const data : array of float;
\Description

\Errors
\SeeAlso
\end{procedure}

\FPCexample{}


\begin{function}{tan}
\Declaration
Function tan(x : float) : float;
\Description

\Errors
\SeeAlso
\end{function}

\FPCexample{}


\begin{function}{tanh}
\Declaration
Function tanh(x : float) : float;
\Description

\Errors
\SeeAlso
\end{function}

\FPCexample{}


\begin{function}{totalvariance}
\Declaration
Function totalvariance(const data : array of float) : float;
\Description

\Errors
\SeeAlso
\end{function}

\FPCexample{}


\begin{function}{variance}
\Declaration
Function variance(const data : array of float) : float;
\Description

\Errors
\SeeAlso
\end{function}

\FPCexample{}


